% \iffalse^^A meta-comment
% ======================================================================
% scrlayer-fancyhdr.dtx
% Copyright (c) Markus Kohm, 2019
%
% This file is part of the LaTeX2e KOMA-Script bundle.
%
% This work may be distributed and/or modified under the conditions of
% the LaTeX Project Public License, version 1.3c of the license.
% The latest version of this license is in
%   http://www.latex-project.org/lppl.txt
% and version 1.3c or later is part of all distributions of LaTeX
% version 2005/12/01 and of this work.
%
% This work has the LPPL maintenance status "author-maintained".
%
% The Current Maintainer and author of this work is Markus Kohm.
%
% This work consists of all files listed in manifest.txt and
% `scrlayer-fancyhdr.dtx'.
% ----------------------------------------------------------------------
% scrlayer-fancyhdr.dtx
% Copyright (c) Markus Kohm, 2019
%
% Diese Datei ist Teil der LaTeX2e KOMA-Script-Sammlung.
%
% Dieses Werk darf nach den Bedingungen der LaTeX Project Public Lizenz,
% Version 1.3c.
% Die neuste Version dieser Lizenz ist
%   http://www.latex-project.org/lppl.txt
% und Version 1.3c ist Teil aller Verteilungen von LaTeX
% Version 2005/12/01 und dieses Werks.
%
% Dieses Werk hat den LPPL-Verwaltungs-Status "author-maintained"
% (allein durch den Autor verwaltet).
%
% Der Aktuelle Verwalter und Autor dieses Werkes ist Markus Kohm.
%
% Dieses Werk besteht aus den in manifest.txt aufgefuehrten Dateien
% sowie `scrlayer-fancyhdr.dtx'.
% ======================================================================
% \fi^^A meta-comment
%
% \CharacterTable
%  {Upper-case    \A\B\C\D\E\F\G\H\I\J\K\L\M\N\O\P\Q\R\S\T\U\V\W\X\Y\Z
%   Lower-case    \a\b\c\d\e\f\g\h\i\j\k\l\m\n\o\p\q\r\s\t\u\v\w\x\y\z
%   Digits        \0\1\2\3\4\5\6\7\8\9
%   Exclamation   \!     Double quote  \"     Hash (number) \#
%   Dollar        \$     Percent       \%     Ampersand     \&
%   Acute accent  \'     Left paren    \(     Right paren   \)
%   Asterisk      \*     Plus          \+     Comma         \,
%   Minus         \-     Point         \.     Solidus       \/
%   Colon         \:     Semicolon     \;     Less than     \<
%   Equals        \=     Greater than  \>     Question mark \?
%   Commercial at \@     Left bracket  \[     Backslash     \\
%   Right bracket \]     Circumflex    \^     Underscore    \_
%   Grave accent  \`     Left brace    \{     Vertical bar  \|
%   Right brace   \}     Tilde         \~}
%
% \iffalse^^A meta-comment
%%% From File: $Id$
%<identify>%%%            (run: identify)
%<init>%%%            (run: init)
%<options>%%%            (run: options)
%<body>%%%            (run: body)
%<*dtx>
\ifx\ProvidesFile\undefined\def\ProvidesFile#1[#2]{}\fi
\begingroup
  \def\filedate$#1: #2-#3-#4 #5${\gdef\filedate{#2/#3/#4}}
  \filedate$Date$
  \def\filerevision$#1: #2 ${\gdef\filerevision{v0.1.#2 ALPHA}}
  \filerevision$Revision$
  \edef\reserved@a{%
    \noexpand\endgroup
    \noexpand\ProvidesFile{scrlayer-fancyhdr.dtx}%
                          [\filedate\space\filerevision\space
                           KOMA-Script package source
  }%
\reserved@a
%</dtx>
%<*identify>
%<interface>\NeedsTeXFormat{LaTeX2e}[1995/12/01]
%<interface>\ProvidesPackage{scrlayer-fancyhdr}[%
% Sometimes following will be replaced by !KOMAScriptVersion:
%!SCRLAYERFANCYHDRVERSION
%<interface>  package
%</identify>
%<*dtx|identify>
  (end user interface for scrlayer)]
%</dtx|identify>
%<*dtx>
\ifx\documentclass\undefined
  \input scrdocstrip.tex
  \@@input scrkernel-version.dtx
  \@@input scrstrip.inc
  \KOMAdefVariable{COPYRIGHTFROM}{2018}
  \KOMAdefVariable{SCRLAYERFANCYHDRVERSION}{%
    \space\space\filedate\space\filerevision\space KOMA-Script
  }%
  \expandafter\let\csname ifbetawas\expandafter\endcsname
  \csname ifbeta\endcsname
  \expandafter\let\csname ifbeta\expandafter\endcsname
  \csname iftrue\endcsname
  \generate{\usepreamble\defaultpreamble
    \file{scrlayer-fancyhdr.sty}{%
      \from{scrlayer-fancyhdr.dtx}{interface,trace,fancyhdr,identify}%
      \from{scrlayer.dtx}{interface,trace,fancyhdr,init}%
      \from{scrlayer-fancyhdr.dtx}{interface,trace,fancyhdr,init}%
      \from{scrlayer.dtx}{interface,trace,fancyhdr,options}%
      \from{scrlayer-fancyhdr.dtx}{interface,trace,fancyhdr,options}%
      \from{scrlayer.dtx}{interface,trace,fancyhdr,body}%
      \from{scrlayer-fancyhdr.dtx}{interface,trace,fancyhdr,body}%
      \from{scrlogo.dtx}{trace,logo}%
    }%
  }
  \expandafter\let\csname ifbeta\expandafter\endcsname
  \csname ifbetawas\endcsname
  \@@input scrstrop.inc
\else
  \let\endbatchfile\relax
\fi
\endbatchfile
  \documentclass{scrdoc}
  \addtolength{\textwidth}{-1em}
  \addtolength{\marginparwidth}{2em}
  \addtolength{\oddsidemargin}{2em}
  \usepackage[ngerman,english]{babel}
  \usepackage{url,babelbib}\bibliographystyle{babalpha-fl}
  \usepackage{listings}
  \usepackage{scrhack}
  \usepackage{etoolbox}
  \pretocmd\DescribeMacro{\ifhmode\else\bigskip\noindent\fi}{}{}
  \pretocmd\DescribeEnv{\ifhmode\else\bigskip\noindent\fi}{}{}
  \pretocmd\DescribeOption{\ifhmode\else\bigskip\noindent\fi}{}{}

  \CodelineIndex
  \RecordChanges
  \GetFileInfo{scrlayer-fancyhdr.dtx}
  \title{The \texttt{scrlayer} interface \texttt{scrlayer-fancyhdr}%
    \footnote{This is version \fileversion\ of file \texttt{\filename}.}}
  \date{\filedate}
  \author{Markus Kohm}

  \newenvironment{Explain}{\par}{\par}
  \newcommand*{\length}{}
  \let\length\Length
  \let\endlength\endLength
  \let\Macro\cs
  \let\Length\Macro
  \let\Package\textsf
  \let\Class\Package
  \let\File\texttt
  \let\Option\texttt
  \newcommand*{\KOption}[1]{\Option{#1}\texttt{=}}
  \newcommand*{\OptionValue}[2]{\Option{#1}\texttt{=}\PValue{#2}}
  \let\Counter\texttt
  \let\Environment\texttt
  \let\ShowOutput\quote
  \let\endShowOutput\endquote
  \let\Pagestyle\texttt
  \newcommand*{\Parameter}[1]{\texttt{\marg{#1}}\linebreak[1]}
  \newcommand*{\OParameter}[1]{\texttt{\oarg{#1}}\linebreak[1]}
  \newcommand*{\MParameter}[2]{\texttt{(\meta{#1},\meta{#2})}\linebreak[1]}
  \providecommand\PParameter[1]{\mbox{\texttt{\{#1\}}}\linebreak[1]}
  \let\PName\meta
  \let\PValue\texttt
  \providecommand*{\autoref}[1]{\expandafter\AUTOREF#1:}
  \providecommand*{\AUTOREF}{}
  \makeatletter
  \def\AUTOREF#1:#2:{%
    \edef\@tempa{#1}%
    \edef\@tempb{tab}\ifx\@tempa\@tempb table~\fi
    \edef\@tempb{sec}\ifx\@tempa\@tempb section~\fi
    \ref{#1:#2}%
  }
  \providecommand*{\IndexCmd}[2][]{}
  \providecommand*{\textnote}[2][]{}
  \providecommand*\eTeX{\leavevmode\hbox{$\varepsilon$}-\TeX}
  \providecommand*\NTS{%
    \leavevmode\hbox{$\cal N\kern-0.35em\lower0.5ex\hbox{$\cal T$}%
      \kern-0.2emS$}}

  \lstnewenvironment{lstcode}{\lstset{language=[LaTeX]TeX}}{}
  \makeatother
  \sloppy% YOU SHOULD NOT DO THIS!!!

  \begin{document}
  \maketitle
  \tableofcontents
  \DocInput{\filename}
  \PrintChanges
  \PrintIndex
  \end{document}
% \fi^^A meta-comment
%
% \selectlanguage{english}
%
% \changes{v0.0}{2018/09/01}{start of interface}
% \let\restorechapter\chapter
% \let\chapter\section
% \let\section\subsection
% \let\subsection\subsubsection
%
% \changes{v0.0}{2018/09/05}{some user documentation}
% \chapter{The Purpose of this Package}
% This package has been made to give users a chance to combine the features of
% Piet van Oostrum's \Package{fancyhdr} \cite{pkg:fancyhdr} with the features
% of \Package{scrlayer} \cite{pkg:komascript}. In other words: It has been
% made to combine the page layers of \Package{scrlayer} with the page styles
% of \Package{fancyhdr}.
%
% In this combination compatibility with \Package{fancyhdr} is the first
% aim! Usability and the freedom provided by \Package{scrlayer} is only the
% second one. Compatibility with other parts of \KOMAScript{} is not a primary
% aim. Perhaps it will become an optional feature in future. Abolishing any
% real or virtual limitations of \Package{fancyhdr} other than make it
% possible to use layers is not an aim and will not be an aim in future.
%
% If you need a better combination of page styles and layers, you should
% either use the low level interface of \Package{scrlayer} to define your page
% styles or\,---\,and this is the recommendation of the author\,---\,use
% \Package{scrlayer-scrpage} instead of \Package{scrlayer-fancyhdr} or
% \Package{fancyhdr}. If you need more compatibility with other parts of
% \KOMAScript, i.\,e.\@ with the \KOMAScript{} classes, you should use
% \Package{scrlayer-scrpage} instead of \Package{scrlayer-fancyhdr} or
% \Package{fancyhdr}.
%
%
% \chapter{How it works}
% To combine \Package{fancyhdr} and \Package{scrlayer},
% \Package{scrlayer-scrpage} loads both packages. After loading
% \Package{fancyhdr} it redefines page style \Pagestyle{@fancy} to make it a
% layer page style of \Package{scrlayer} using the newly defined layers
% \texttt{fancy.head.even}, \texttt{fancy.head.odd}, \texttt{fancy.foot.even}
% and \texttt{fancy.foot.odd}. The \texttt{\dots head\dots} layers are
% background layers like the page head of \Package{fancyhdr}'s page styles (or
% other usual page styles). The \texttt{\dots foot\dots} layers are foreground
% layers like the page footer of \Package{fancyhdr}'s page styles (or other
% usual page styles). The \texttt{\dots even} layers are restricted to even
% pages, that means left side pages in two-sided documents. The \texttt{\dots
% odd} layers are restricted to odd pages, that means right side pages in
% two-sided documents or all pages in single-sided documents.
%
% \DescribeMacro\ps@@fancy
% \DescribeMacro\ps@fancyplain
% \DescribeMacro\ps@plain@fancy
% Note: Page style \Pagestyle{@fancy} is an internal basic page style of
% package \Package{fancyhdr}. It is used for the user page style
% \Pagestyle{fancy}. The user page style \Pagestyle{fancy} is also used for
% \Package{fancyhdr}'s undocumented page style \Pagestyle{fancyplain}, that
% also redefines page style \Pagestyle{plain} to be \Package{fancyhdr}'s
% internal page style \Pagestyle{plain@fancy}, that also uses
% \Pagestyle{fancy} but with \Macro{if@fancyplain} set to
% \Macro{iftrue}. There is also an undocumented command
% \Macro{fancyplain}\Parameter{plain code}\Parameter{fancy code}, that uses
% the \meta{plain code} if \Macro{if@fancyplain} is \Macro{iftrue} and
% \meta{fancy code} if \Macro{if@fancyplain} is \Macro{iffalse}.
%
% Note: Page styles defined using \Package{fancyhdr}'s command
% \Macro{fancypagestyle} also always use page style \Pagestyle{fancy} and so
% the internal basic page style \Pagestyle{@fancy}.
%
% As a result of the two notes above, every page style of \Package{fancyhdr}
% always uses page style \Pagestyle{fancy} and the same internal basic page
% style \Pagestyle{@fancy}. As a result of redefining page style
% \Pagestyle{@fancy} to be a layer page style, users can add layers to or
% remove layers from all \Package{fancyhdr} page styles by adding layers to or
% remove layers from page style \Pagestyle{@fancy}. You cannot add layers to
% or remove layers from \Package{fancyhdr}'s single page styles
% \Pagestyle{fancy}, \Pagestyle{fancyplain}, \Pagestyle{plain@fancy} or the
% page styles defined using \Macro{fancypagestyle} directly. So using layers
% is a all or nothing feature with \Package{scrlayer-scrpage}. However, you
% can use the second argument of \Macro{fancypagestyle} to add or remove
% layers whenever one of the \Package{fancyhdr} page styles is activated. So
% this is a move from the \Package{scrlayer} interface of adding or removing
% layers to single page styles to the \Package{fancyhdr} interface of defining
% modifications of page style \Pagestyle{fancy}.
%
% Another such movement from an \Package{scrlayer} user interface to a
% \Package{fancyhdr} user interface it the decision whether or not automatic
% running heads are used. \Package{scrlayer} provides the options
% \Option{automark} and \Option{manualmark} and commands \Macro{automark} and
% \Macro{manualmark} to do this decision and also to configure commands like
% \Macro{partmark}, \Macro{chaptermark}, \Macro{sectionmark} etc. With
% \Package{scrlayer-fancyhdr} also using page style \Pagestyle{@fancy} does
% switch to automatic running heads. The first activation of page style
% \Pagestyle{fancy} still redefines \Macro{chaptermark} and
% \Macro{sectionmark}, if a class with \Macro{chapter} is used, or
% \Macro{sectionmark} and \Macro{subsectionmark}, if a class without
% \Macro{chapter} is used. However, you still can use \Macro{manualmark} and
% \Macro{automark} after switching to a \Package{fancyhdr} page style to
% configure the running heads. So this movement is only partial.
%
% \DescribeMacro\ps@headings
% \DescribeMacro\ps@myheadings
% \DescribeMacro\ps@plain
% Note: Currently, neither \Package{scrlayer} nor \Package{fancyhdr} nor
% \Package{scrlayer-fancyhdr} do redefine page styles \Pagestyle{headings} or
% \Pagestyle{myheadings}. And neither \Package{scrlayer} nor
% \Package{fancyhdr} nor \Package{scrlayer-fancyhdr} do redefine page style
% \Pagestyle{plain} unless you are activating the undocumented
% \Package{fancyhdr} page style \Pagestyle{fancyplain}. So if you like to use
% layers on \Pagestyle{plain} pages, i.\,e. usually the first page of a
% chapter or part or the page with a title head, you have to either use
% \Package{fancyhdr}'s undocumented page style \Pagestyle{fancyplain} or
% redefine page style \Pagestyle{plain} either using \Macro{fancypagestyle} as
% documented in the \Package{fancyhdr} manual or using
% \Macro{DeclareNewPageStyleByLayers}, documented in the \KOMAScript{} manual.
%
% \DescribeMacro\ps@empty
% \DescribeMacro\ps@@empty
% Note: Pagestyle \Pagestyle{empty} is somehow special. \Package{scrlayer}
% redefines it to be a layer page style. And \Package{fancyhdr}'s internal
% page style \Pagestyle{@empty} is the same like \Pagestyle{empty}. So
% \Pagestyle{@empty} also uses the layers of \Pagestyle{empty} but you should
% not try to modify it directly using the interface of
% \Package{scrlayer}. Moreover, if \Package{fancyhdr} is loaded before
% \Package{scrlayer-scrpage}, \Package{fancyhdr}'s internal page style
% \Pagestyle{@empty} is not a copy of \Package{scrlayer}'s layer page style
% \Pagestyle{empty} but the original standard page style
% \Pagestyle{empty}. However, with \Package{scrlayer-fancyhdr} package
% \Package{fancyhdr} does not longer use the internal page style
% \Pagestyle{@empty}. So you would not need to know this.
%
% \chapter{How to use the Package}
%
%^^A TODO: Remove the following alpha note.
% First of all, this package is experimental and should not be used for
% anything else but testing whether or not it works as documented! This
% package even is not an official part of \KOMAScript!
%
% To use the package you have to load it, e.\,g., using:
%\begin{verbatim}
% \usepackage{scrlayer-fancyhdr}
%\end{verbatim}
% instead of loading \Package{scrlayer} and \Package{fancyhdr} or before or
% after loading one of these packages. However it is recommended to replace
% loading \Package{scrlayer} and \Package{fancyhdr} by loading
% \Package{scrlayer-fancyhdr} because this avoids option
% clashes. \Package{scrlayer-fancyhdr} provides all options of
% \Package{scrlayer} and passes them to \Package{scrlayer}. Nevertheless
% sometimes it may be useful to be able to additionally load
% \Package{scrlayer} or \Package{fancyhdr}.
%
% \DescribeMacro\ps@@fancy
% \DescribeMacro\ps@fancy
% \DescribeMacro\ps@fancyplain
% \DescribeMacro\ps@plain@fancy
% \DescribeMacro\fancypagestyle
% After this you should be able to use the page styles and commands of
% \Package{fancyhdr} and to add layers to or remove layers from the
% \Package{fancyhdr}'s internal basic page style \Package{@fancy}. Note, you
% are not able to add layers to or remove layers from \Package{fancyhdr}'s
% page styles \Pagestyle{fancy}, \Pagestyle{fancyplain},
% \Pagestyle{plain@fancy} or page styles defined by
% \Macro{fancypagestyle}. However adding layers to or removing layers from
% \Pagestyle{@fancy} will always change all these page styles!
%
% \DescribeMacro\ps@plain
% \DescribeMacro\ps@headings
% \DescribeMacro\ps@myheadings
% \DescribeMacro\ps@empty
% Note: Loading \Package{scrlayer-fancyhdr} will not make page style
% \Pagestyle{plain} nor \Pagestyle{headings} nor \Pagestyle{myheadings} or any
% other page style but \Pagestyle{empty} to automatically be a layer page
% style!
%
% \chapter{Known Issues}
%
% Please note, the follow issues are either notes to the package author or
% notes to the user to avoid them reporting the same issues again and
% again. Listing these issues does not say they are bugs or features.
%
% \begin{itemize}
% \item
%   \DescribeMacro\ps@fancy
%   \DescribeMacro\ps@fancyplain
%   \DescribeMacro\fancypagestyle
%   \DescribeMacro\ps@@fancy
%   You are not able to add layers to the user level page style
%   \Pagestyle{fancy} or \Pagestyle{fancyplain} or any page style defined by
%   \Macro{fancypagestyle} but only to the internal page style
%   \Pagestyle{@fancy}. This is intended as explained in this manual.
% \item
%   \DescribeMacro\ps@@empty
%   Using \Package{fancyhdr}'s internal page style \Pagestyle{@empty}
%   could have strange results. However, there is a simple solution for this:
%   Don't use the internal \Pagestyle{@empty} but always the user level page
%   style \Pagestyle{empty}!
% \item
%   \DescribeMacro\ps@@fancy
%   The vertical position of the page header does differ a little bit, if
%   \Package{scrlayer-fancyhdr} is used instead of \Package{fancyhdr}. This
%   could be fixed by a modification of the layers \texttt{fancy.head.odd} and
%   \texttt{fancy.head.even} using \Option{addvoffset}. More tests are needed.
% \end{itemize}
%
%\iffalse
%</dtx>
%\fi
%
% \StopEventually{}
%
% \chapter{Implementation of \Package{scrlayer-fancyhdr}}
% \label{sec:scrlayer-fancyhdr}
%
% This section if for developers only.
%
% \iffalse
% TODO: Remove this alpha warning (but not the markers)!
%<*interface>
%<*identify>
\PackageWarningNoLine{scrlayer-fancyhdr}{%
  THIS IS AN ALPHA VERSION OF AN\MessageBreak
  EXPERIMENTAL PACKAGE!\MessageBreak
  YOU SHOULD NOT USE THIS PACKAGE FOR\MessageBreak
  ANYTHING ELSE BUT TESTING%
}
%</identify>
%\fi
%
% Note: The main problem of this interface is, that it tries to implement the
% user interface of package \Package{fancyhdr} by Piet van Oostrum, that is
% completely different from \Package{scrlayer} and not really compatible with
% \Package{scrlayer}, using \Package{scrlayer}. This means, that
% \Package{scrlayer-fancyhdr} never can be a drop-in replacement of
% \Package{fancyhdr}. Nevertheless it can help to let \Package{scrlayer} and
% \Package{fancyhdr} coexist. To do so
% \begin{itemize}
% \item the lowest level of \Package{fancyhdr} should not be the page style
% but a layer
% \item the page styles of \Package{fancyhdr} should be layer page styles
% \item init code of the page styles of \Package{fancyhdr} should use the
% layer page init code
% \end{itemize}
% Currently it is unsure whether it would be best to do a new implementation
% or to load original \Package{fancyhdr} and to only modify some things. First
% I'll try the second method and therefore:
%    \begin{macrocode}
%<*body>
\RequirePackage{fancyhdr}
%</body>
%    \end{macrocode}
%
% We need at least one new layer for the new layer page style
% \Pagestyle{fancy}. However, it could be useful to have not only one but
% four layers (even side head, odd side head, even side foot, odd side foot).
%    \begin{macrocode}
%<*body>
\DeclareNewLayer[%
  background,oddpage,
  head,
  contents={\hb@xt@ \layerwidth{%
      \f@nch@head\f@nch@Oolh\f@nch@olh\f@nch@och\f@nch@orh\f@nch@Oorh}}
]{fancy.head.odd}
\DeclareNewLayer[%
  background,evenpage,
  head,
  contents={\hb@xt@ \layerwidth{%
      \f@nch@head\f@nch@Oelh\f@nch@elh\f@nch@ech\f@nch@erh\f@nch@Oerh}}
]{fancy.head.even}
\DeclareNewLayer[%
  foreground,oddpage,
  foot,
  contents={\hb@xt@ \layerwidth{%
      \f@nch@foot\f@nch@Oolf\f@nch@olf\f@nch@ocf\f@nch@orf\f@nch@Oorf}}
]{fancy.foot.odd}
\DeclareNewLayer[%
  foreground,evenpage,
  foot,
  contents={\hb@xt@ \layerwidth{%
      \f@nch@foot\f@nch@Oelf\f@nch@elf\f@nch@ecf\f@nch@erf\f@nch@Oerf}}
]{fancy.foot.even}
%    \end{macrocode}
% \begin{macro}{\ps@@fancy}
% \begin{macro}{\@mkboth}
% And have to create a layer page style from this new layers, but we do not
% redefine page style \Pagestyle{fancy} but the low level page style
% \Pagestyle{@fancy}. 
%    \begin{macrocode}
\DeclarePageStyleByLayers[
%    \end{macrocode}
% \Package{fancyhdr} does some initialization at the very first call
% of \cs{pagestyle{fancy}}. To do so \Package{fancyhdr} first uses a different
% page style definition, that does the initialization and redefines the page
% style afterwards. This is still active with \Package{scrlayer-fancyhdr}.
% Additionally, \Package{fancyhdr} redefines \cs{@mkboth} at every selection
% of the internal page style \Pagestyle{@fancy}. This can be adapted using the
% \Option{onselect} feature of the new layer page style \Pagestyle{@fancy}. In
% my opinion, the |\let\@mkboth\markboth| used by page style
% \Pagestyle{headings} of, e.\,g., the standard classes or the \KOMAScript{}
% classes would be best here. However, \Package{fancyhdr} uses the uncommon
% |\def\@mkboth{\protect\markboth}|, which would fail if a class or package
% tests \cs{@mkboth} using |\ifx\@mkboth\markboth|. However, \KOMAScript's
% \cs{IfActiveMkBoth} (see the \Package{scrbase} chapter in the \KOMAScript{}
% manual) does also recognize the \Package{fancyhdr} definition and copying
% this is more compatible with \Package{fancyhdr}.
%    \begin{macrocode}
  onselect={\def\@mkboth{\protect\markboth}},
]{@fancy}{%
  fancy.head.odd,fancy.head.even,fancy.foot.odd,fancy.foot.even
}
%</body>
%    \end{macrocode}
% Note: Redefining page style \Pagestyle{@fancy} instead of \Pagestyle{fancy}
% does also mean, that features like options \Option{automark} and
% \Option{manualmark} resp. \cs{automark} and \cs{manualmark} are not fully
% supported by \Package{scrlayer-fancyhdr}. Also currently the font features
% of the \KOMAScript{} classes are not supported by
% \Package{scrlayer-fancyhdr}. However you are now able to combine other
% features of \Package{scrlayer} with features of \Package{fancyhdr} and you
% can, e.g., use \Package{scrlayer-notecolumn} with
% \Package{scrlayer-fancyhdr}.
% \end{macro}%^^A \@mkboth
% \end{macro}%^^A \ps@@fancy
%
% A future release of \Package{scrlayer-fancyhdr} may even provide the font
% features of the \KOMAScript{} classes and a working
% \Option{markcase}. However, in this case I would have to redefine the
% initial page style \Pagestyle{fancy} and the layers above.
%
% \iffalse
%</interface>
% \fi
%
% \begin{thebibliography}{99}
% \raggedright
% \bibitem[1]{pkg:fancyhdr} Piet van Oostrum: \emph{Page layout in \LaTeX},
%   release 3.9a 2017-06-30,
%   \url{https://ctan.org/tex-archive/macros/latex/contrib/fancyhdr},
%   last visited: 2018-09-06.
% \bibitem[2]{pkg:komascript} Markus Kohm: \emph{\KOMAScript},
%   release 3.25 2018-03-30,
%   \url{https://ctan.org/tex-archive/macros/latex/contrib/koma-script},
%   last visited: 2018-09-06.
% \end{thebibliography}
%
% \Finale
% \let\subsection\section
% \let\section\chapter
% \let\chapter\restorechapter
%
\endinput
%
% End of file `scrlayer-fancyhdr.dtx'.
%

%%% Local Variables:
%%% mode: doctex
%%% mode: flyspell
%%% ispell-local-dictionary: "en_GB"
%%% TeX-master: t
%%% End:
