% ======================================================================
% scrpage2.tex
% Copyright (c) Markus Kohm, 2001-2018
%
% This file is part of the LaTeX2e KOMA-Script bundle.
%
% This work may be distributed and/or modified under the conditions of
% the LaTeX Project Public License, version 1.3c of the license.
% The latest version of this license is in
%   http://www.latex-project.org/lppl.txt
% and version 1.3c or later is part of all distributions of LaTeX 
% version 2005/12/01 or later and of this work.
%
% This work has the LPPL maintenance status "author-maintained".
%
% The Current Maintainer and author of this work is Markus Kohm.
%
% This work consists of all files listed in manifest.txt.
% ----------------------------------------------------------------------
% scrpage2.tex
% Copyright (c) Markus Kohm, 2001-2018
%
% Dieses Werk darf nach den Bedingungen der LaTeX Project Public Lizenz,
% Version 1.3c, verteilt und/oder veraendert werden.
% Die neuste Version dieser Lizenz ist
%   http://www.latex-project.org/lppl.txt
% und Version 1.3c ist Teil aller Verteilungen von LaTeX
% Version 2005/12/01 oder spaeter und dieses Werks.
%
% Dieses Werk hat den LPPL-Verwaltungs-Status "author-maintained"
% (allein durch den Autor verwaltet).
%
% Der Aktuelle Verwalter und Autor dieses Werkes ist Markus Kohm.
% 
% Dieses Werk besteht aus den in manifest.txt aufgefuehrten Dateien.
% ======================================================================
%
% Chapter about scrpage2
% Maintained by Markus Kohm
%
% ----------------------------------------------------------------------
%
% Kapitel über scrpage2
% Verwaltet von Markus Kohm
%
% ============================================================================

\begin{filecontents*}{\jobname-ngerman.tex}
% ======================================================================
% guide-ngerman.tex
% Copyright (c) Markus Kohm, 2002-2017
%
% This file is part of the LaTeX2e KOMA-Script bundle.
%
% This work may be distributed and/or modified under the conditions of
% the LaTeX Project Public License, version 1.3c of the license.
% The latest version of this license is in
%   http://www.latex-project.org/lppl.txt
% and version 1.3c or later is part of all distributions of LaTeX 
% version 2005/12/01 or later and of this work.
%
% This work has the LPPL maintenance status "author-maintained".
%
% The Current Maintainer and author of this work is Markus Kohm.
%
% This work consists of all files listed in manifest.txt.
% ----------------------------------------------------------------------
% guide-ngerman.tex
% Copyright (c) Markus Kohm, 2002-2017
%
% Dieses Werk darf nach den Bedingungen der LaTeX Project Public Lizenz,
% Version 1.3c, verteilt und/oder veraendert werden.
% Die neuste Version dieser Lizenz ist
%   http://www.latex-project.org/lppl.txt
% und Version 1.3c ist Teil aller Verteilungen von LaTeX
% Version 2005/12/01 oder spaeter und dieses Werks.
%
% Dieses Werk hat den LPPL-Verwaltungs-Status "author-maintained"
% (allein durch den Autor verwaltet).
%
% Der Aktuelle Verwalter und Autor dieses Werkes ist Markus Kohm.
% 
% Dieses Werk besteht aus den in manifest.txt aufgefuehrten Dateien.
% ======================================================================
%
% Language dependencies (ngerman) of the KOMA-Script guide
% Maintained by Markus Kohm
%
% ----------------------------------------------------------------------
%
% Sprachabhaengigkeiten (ngerman) der KOMA-Script-Anleitung
% Verwaltet von Markus Kohm
%
% ======================================================================
%
\KOMAProvidesFile{guide-ngerman.tex}
                 [$Date$
                  KOMA-Script guide language dependencies]
%
% \section{Title}
%
% \begin{macro}{\GuideSubject}
% \begin{macro}{\GuideTitle}
% \begin{macro}{\GuideSubTitle}
% \begin{macro}{\GuideAuthorHeadline}
% \begin{macro}{\GuideTranslatorHeadline}
% \begin{macro}{\GuideWarrantyHeadline}
% \begin{macro}{\GuideWarranty}
% \begin{macro}{\GuideCopyright}
% \begin{macro}{\GuideDedication}
%   Language dependencies of all the title pages.
%    \begin{macrocode}
\newcommand*{\GuideSubject}{Die Anleitung}
\newcommand*{\GuideTitle}{\KOMAScript}%
\newcommand*{\GuideSubTitle}{ein wandelbares \LaTeXe-Paket}%
\newcommand*{\GuideAuthorHeadline}{Autoren des \KOMAScript-Pakets}%
\newcommand*{\GuideTranslatorHeadline}{An der deutschen \"Ubersetzung dieser
  Anleitung haben mitgewirkt: }
\newcommand*{\GuideWarrantyHeadline}{Rechtliche Hinweise:}%
\newcommand*{\GuideWarranty}{Der Autor dieser Anleitung ist in dieser
  Eigenschaft nicht verantwortlich f\"ur die
  Funktion oder Fehler der in dieser Anleitung beschriebenen Software. Bei der
  Erstellung von Texten und Abbildungen wurde mit gro\ss er Sorgfalt
  vorgegangen. Trotzdem k\"onnen Fehler nicht vollst\"andig ausgeschlossen
  werden.\par\medskip%
  Der Autor kann jedoch f\"ur fehlerhafte Angaben und deren Folgen weder
  eine juristische Verantwortung noch irgendeine Haftung \"ubernehmen. F\"ur
  Verbesserungsvorschl\"age und Hinweise auf Fehler ist der Autor
  dankbar.\par\medskip%
  In dieser Anleitung werden Warennamen ohne der Gew\"ahrleistung der freien
  Verwendbarkeit und ohne besondere Kennzeichnung benutzt. Es ist jedoch davon
  auszugehen, dass viele der Warennamen gleichzeitig eingetragene
  Warenzeichnen oder als solche zu betrachten sind.}%
\newcommand*{\GuideCopyright}{Freie Bildschirm-Version ohne Optimierung des
  Umbruchs\par\medskip%
  Diese Anleitung ist als Bestandteil von \KOMAScript{} frei im Sinne der
  \LaTeX{} Project Public License Version 1.3c. Eine f\"ur \KOMAScript{} g\"ultige
  deutsche \"Ubersetzung liegt \KOMAScript{} in der Datei �\texttt{lppl-de.txt}�
  bei. Diese Anleitung --~auch in gedruckter Form~-- darf nur zusammen mit den
  \"ubrigen Bestandteilen von \KOMAScript{} weitergegeben und verteilt werden.
  Eine Verteilung der Anleitung unabh\"angig von den \"ubrigen Bestandteilen von
  \KOMAScript{} bedarf der ausdr\"ucklichen Genehmigung des
  Autors.\par\medskip%
  Eine umbruchoptimierte und erweiterte Ausgabe der \KOMAScript-Anleitung ist
  in der dante-Edition von Lehmanns Media erschienen (siehe
  \cite{book:komascript}).}%
\newcommand*{\GuideDedication}{Den Freunden der Typografie!}%
%    \end{macrocode}
% \end{macro}
% \end{macro}
% \end{macro}
% \end{macro}
% \end{macro}
% \end{macro}
% \end{macro}
% \end{macro}
% \end{macro}
%
%
% \section{CTAN Server to be Used}
%
% \begin{macro}{\GuideCTANserver}
% Unused since 2015-09-30.
%    \begin{macrocode}
%\newcommand*{\GuideCTANserver}{mirror.ctan.org}
%    \end{macrocode}
% \end{macro}
%
%
% \section{Terms}
%
% Some terms, e.g. used at index notes.
%
% \begin{macro}{\GuideClass}
% \begin{macro}{\GuideClassIndexCategory}
% \begin{macro}{\GuideClassIndexCategoryExpanded}
%   The term ``class''.
%    \begin{macrocode}
\newcommand*{\GuideClass}{Klasse}
\newcommand*{\GuideClassIndexCategory}{Klassen}
\let\GuideClassIndexCategoryExpanded\GuideClassIndexCategory
%    \end{macrocode}
% \end{macro}
% \end{macro}
% \end{macro}
%
% \begin{macro}{\GuideCounter}
% \begin{macro}{\GuideCounterIndexCategory}
% \begin{macro}{\GuideCounterIndexCategoryExpanded}
%   The term ``counter''.
%    \begin{macrocode}
\newcommand*{\GuideCounter}{Z\"ahler}
\newcommand*{\GuideCounterIndexCategory}{Z\"ahler}
\let\GuideCounterIndexCategoryExpanded\GuideCounterIndexCategory
%    \end{macrocode}
% \end{macro}
% \end{macro}
% \end{macro}
%
% \begin{macro}{\GuideEnvironment}
% \begin{macro}{\GuideEnvIndexCategory}
% \begin{macro}{\GuideEnvIndexCategoryExpanded}
%   The term ``environment''.
%    \begin{macrocode}
\newcommand*{\GuideEnvironment}{Umgebung}
\newcommand*{\GuideEnvIndexCategory}{Umgebungen}
\let\GuideEnvIndexCategoryExpanded\GuideEnvIndexCategory
%    \end{macrocode}
% \end{macro}
% \end{macro}
% \end{macro}
%
% \begin{macro}{\GuideExample}
%   The term ``Example'' used at a kind of heading, so it should be upper case.
%    \begin{macrocode}
\newcommand*{\GuideExample}{Beispiel}
%    \end{macrocode}
% \end{macro}
%
% \begin{macro}{\GuideFile}
% \begin{macro}{\GuideFileIndexCategory}
% \begin{macro}{\GuideFileIndexCategoryExpanded}
%   The term ``file''.
%    \begin{macrocode}
\newcommand*{\GuideFile}{Datei}
\newcommand*{\GuideFileIndexCategory}{Dateien}
\let\GuideFileIndexCategoryExpanded\GuideFileIndexCategory
%    \end{macrocode}
% \end{macro}
% \end{macro}
% \end{macro}
%
% \begin{macro}{\GuideFloatstyle}
% \begin{macro}{\GuideFloatstyleIndexCategory}
% \begin{macro}{\GuideFloatstyleIndexCategoryExpanded}
%   The term ``\Package{float} style''.
%    \begin{macrocode}
\newcommand*{\GuideFloatstyle}{\Package{float}-Stil}
\newcommand*{\GuideFloatstyleIndexCategory}{\Package{float}-Stile}
\newcommand*{\GuideFloatstyleIndexCategoryExpanded}{float-Stile}
%    \end{macrocode}
% \end{macro}
% \end{macro}
% \end{macro}
%
% \begin{macro}{\GuideFontElement}
% \begin{macro}{\GuideFontElementIndexCategory}
% \begin{macro}{\GuideFontElementIndexCategoryExpanded}
%   The term ``element''.
%    \begin{macrocode}
\newcommand*{\GuideFontElement}{Element}
\newcommand*{\GuideFontElementIndexCategory}{Elemente}
\let\GuideFontElementIndexCategoryExpanded\GuideFontElementIndexCategory
%    \end{macrocode}
% \end{macro}
% \end{macro}
% \end{macro}
%
% \begin{macro}{\GuideLength}
% \begin{macro}{\GuideLengthIndexCategory}
% \begin{macro}{\GuideLengthIndexCategoryExpanded}
%   The term ``length''.
%    \begin{macrocode}
\newcommand*{\GuideLength}{L\"ange}
\newcommand*{\GuideLengthIndexCategory}{L\"angen}
\let\GuideLengthIndexCategoryExpanded\GuideLengthIndexCategory
%    \end{macrocode}
% \end{macro}
% \end{macro}
% \end{macro}
%
% \begin{macro}{\GuideMacro}
% \begin{macro}{\GuideMacroIndexCategory}
% \begin{macro}{\GuideMacroIndexCategoryExpanded}
% \begin{macro}{\GuideCommand}
% \begin{macro}{\GuideCommandIndexCategory}
% \begin{macro}{\GuideCommandIndexCategoryExpanded}
%   The term ``command''.
%    \begin{macrocode}
\newcommand*{\GuideMacro}{Befehl}
\let\GuideCommand\GuideMacro
\newcommand*{\GuideMacroIndexCategory}{Befehle}
\let\GuideCommandIndexCategory\GuideMacroIndexCategory
\let\GuideMacroIndexCategoryExpanded\GuideMacroIndexCategory
\let\GuideCommandIndexCategoryExpanded\GuideCommandIndexCategory
%    \end{macrocode}
% \end{macro}
% \end{macro}
% \end{macro}
% \end{macro}
% \end{macro}
% \end{macro}
%
% \begin{macro}{\GuideOption}
% \begin{macro}{\GuideOptionIndexCategory}
% \begin{macro}{\GuideOptionIndexCategoryExpanded}
%   The term ``option''.
%    \begin{macrocode}
\newcommand*{\GuideOption}{Option}
\newcommand*{\GuideOptionIndexCategory}{Optionen}
\let\GuideOptionIndexCategoryExpanded\GuideOptionIndexCategory
%    \end{macrocode}
% \end{macro}
% \end{macro}
% \end{macro}
%
% \begin{macro}{\GuidePackage}
% \begin{macro}{\GuidePackageIndexCategory}
% \begin{macro}{\GuidePackageIndexCategoryExpanded}
%   The term ``package''.
%    \begin{macrocode}
\newcommand*{\GuidePackage}{Paket}
\newcommand*{\GuidePackageIndexCategory}{Pakete}
\let\GuidePackageIndexCategoryExpanded\GuidePackageIndexCategory
%    \end{macrocode}
% \end{macro}
% \end{macro}
% \end{macro}
%
% \begin{macro}{\GuidePagestyle}
% \begin{macro}{\GuidePagestyleIndexCategory}
% \begin{macro}{\GuidePagestyleIndexCategoryExpanded}
%   The term ``page style''.
%    \begin{macrocode}
\newcommand*{\GuidePagestyle}{Seitenstil}
\newcommand*{\GuidePagestyleIndexCategory}{Seitenstile}
\let\GuidePagestyleIndexCategoryExpanded\GuidePagestyleIndexCategory
%    \end{macrocode}
% \end{macro}
% \end{macro}
% \end{macro}
%
% \begin{macro}{\GuidePLength}
% \begin{macro}{\GuidePLengthIndexCategory}
% \begin{macro}{\GuidePLengthIndexCategoryExpanded}
% \begin{macro}{\GuidePseudoPrefix}
%   The prefix ``pseudo'' at the term ``pseudo length'' and the term itself.
%    \begin{macrocode}
\newcommand*{\GuidePseudoPrefix}{Pseudo-}
\newcommand*{\GuidePLength}{Pseudol\"ange}
\newcommand*{\GuidePLengthIndexCategory}{Pseudol\"angen}
\let\GuidePLengthIndexCategoryExpanded\GuidePLengthIndexCategory
%    \end{macrocode}
% \end{macro}
% \end{macro}
% \end{macro}
% \end{macro}
%
% \begin{macro}{\GuideVariable}
% \begin{macro}{\GuideVariableIndexCategory}
% \begin{macro}{\GuideVariableIndexCategoryExpanded}
%   The term ``variable''.
%    \begin{macrocode}
\newcommand*{\GuideVariable}{Variable}
\newcommand*{\GuideVariableIndexCategory}{Variablen}
\let\GuideVariableIndexCategoryExpanded\GuideVariableIndexCategory
%    \end{macrocode}
% \end{macro}
% \end{macro}
% \end{macro}
%
% \begin{macro}{\GuideAnd}
%   The ``and'' at a list of two.
%   \begin{macrocode}
\newcommand*{\GuideAnd}{ und }
%   \end{macrocode}
% \end{macro}
%
% \begin{macro}{\GuideListAnd}
%   The ``and'' at a list of more than two.
%   \begin{macrocode}
\newcommand*{\GuideListAnd}{ und }
%   \end{macrocode}
% \end{macro}
%
% \begin{macro}{\GuideLoadNonFree}
%   Note, that the non free manual contains more information.
%    \begin{macrocode}
\newcommand*{\GuideLoadNonFree}{%
  Im \KOMAScript-Buch \cite{book:komascript} finden sich an dieser Stelle
  weitere Informationen.
}
%    \end{macrocode}
% \end{macro}
%
%
% \section{Index}
%
% \begin{macro}{\GuidegenIndex}
% \begin{macro}{\GuidecmdIndex}
% \begin{macro}{\GuidecmdIndexShort}
% \begin{macro}{\GuidelenIndex}
% \begin{macro}{\GuidelenIndexShort}
% \begin{macro}{\GuideelmIndex}
% \begin{macro}{\GuideelmIndexShort}
% \begin{macro}{\GuidefilIndex}
% \begin{macro}{\GuidefilIndexShort}
% \begin{macro}{\GuideoptIndex}
% \begin{macro}{\GuideoptIndexShort}
%   The titles and short titles of each single index.
\newcommand*{\GuidegenIndex}{Allgemeiner Index}
\newcommand*{\GuidecmdIndex}{Befehle, Umgebungen und
  Variablen}
\newcommand*{\GuidecmdIndexShort}{Index der Befehle etc.}
\newcommand*{\GuidelenIndex}{L\"angen und Z\"ahler}
\newcommand*{\GuidelenIndexShort}{Index der L\"angen etc.}
\newcommand*{\GuideelmIndex}{Elemente mit
  der M\"oglichkeit zur Schriftumschaltung}
\newcommand*{\GuideelmIndexShort}{Index der Elemente}
\newcommand*{\GuidefilIndex}{Dateien, Klassen und Pakete}
\newcommand*{\GuidefilIndexShort}{Index der Dateien etc.}
\newcommand*{\GuideoptIndex}{Klassen- und
  Paketoptionen}
\newcommand*{\GuideoptIndexShort}{Index der Optionen}
% \end{macro}
% \end{macro}
% \end{macro}
% \end{macro}
% \end{macro}
% \end{macro}
% \end{macro}
% \end{macro}
% \end{macro}
% \end{macro}
% \end{macro}
%
% \begin{macro}{\GuideIndexPreamble}
%   The preamble text of the whole index.
\newcommand*{\GuideIndexPreamble}{%
  Fett hervorgehobene Zahlen geben die Seiten der Erkl\"arung zu einem
  Stichwort wieder. Normal gedruckte Zahlen verweisen hingegen auf
  Seiten mit zus\"atzlichen Informationen zum jeweiligen Stichwort.%
}
% \end{macro}
%
% \begin{macro}{\GuideIndexSees}
%   These are all see index entries.
\newcommand*{\GuideIndexSees}{%
  \Index{Seitenmittenmarke|see{Lochermarke}}%
  \Index{Falzmarke|see{Faltmarke}}%
  \Index{in-page-Titel=\emph{in-page}-Titel|see{Titelkopf}}%
  \Index{Index|see{Stichwortverzeichnis}}%
  \Index{Register|see{Stichwortverzeichnis}}%
}
% \end{macro}
%
% \begin{macro}{\GuideIndexSeeAlsos}
%   These are all see also index entries.
\newcommand*{\GuideIndexSeeAlsos}{%
  \Index{Ueberschriften=�berschriften|seealso{Abschnitt, Gliederung, Kapitel}}%
  \Index{Gliederung|seealso{Abschnitt, Kapitel, �berschriften}}%
  \Index{Bindeanteil|seealso{Bindekorrektur}}%
  \Index{Bindung|seealso{Bindekorrektur}}%
}
% \end{macro}
%
%
% \section{Bibliography}
%
% \begin{macro}{\GuideBibPreamble}
%    \begin{macrocode}
\newcommand*{\GuideBibPreamble}{%
  Sie finden im Folgenden eine ganze Reihe von Literaturangaben. Auf
  all diese wird im Text verwiesen. In vielen F\"allen handelt es sich
  um Dokumente oder ganze Verzeichnisse, die im Internet verf\"ugbar
  sind. In diesen F\"allen ist statt eines Verlages eine URL angegeben.
  Wird auf ein \LaTeX-Paket verwiesen, so findet der Verweis in der
  Regel in der Form \glqq \url{CTAN://}\emph{Verweis}\grqq{} statt.
  Der Pr\"afix \glqq \url{CTAN://}\grqq{} steht dabei f\"ur das
  \TeX-Archiv eines jeden CTAN-Servers oder -Spiegels. Sie k\"onnen
  den Pr\"afix beispielsweise durch 
  \url{http://mirror.ctan.org/} ersetzen.  Bei
  \LaTeX-Paketen ist au\ss erdem zu beachten, dass versucht wurde,
  die Version anzugeben, auf die im Text Bezug genommen wurde. Bei
  einigen Paketen war es mehr ein Ratespiel, eine einheitliche
  Versionsnummer und ein Erscheinungsdatum zu finden. Auch muss die
  angegebene Version nicht immer die neueste verf\"ugbare Version sein.
  Wenn Sie sich ein Paket neu besorgen und installieren, sollten Sie
  jedoch zun\"achst immer die aktuelle Version ausprobieren. Bevor Sie
  ein Dokument oder Paket von einem Server herunterladen, sollten Sie
  au\ss erdem \"uberpr\"ufen, ob es sich nicht bereits auf Ihrem
  Rechner befindet.
}
%    \end{macrocode}
% \end{macro}
%
%
% \section{Change Log}
%
% \begin{macro}{\GuideChangeLogPreamble}
%   The preamble of the change log index.
%    \begin{macrocode}
\newcommand*{\GuideChangeLogPreamble}{%
  Sie finden im folgenden eine Auf\/listung aller wesentlichen \"Anderungen
  der Benutzerschnittstelle im \KOMAScript-Paket der neueren Zeit. Die Liste
  ist sowohl nach Versionen als auch nach Paket- und Klassennamen sortiert. Zu
  jeder Version, jedem Paket und jeder Klasse ist jeweils angegeben, auf
  welchen Seiten dieser Dokumentation die \"Anderungen zu finden sind. Auf den
  entsprechenden Seiten finden Sie dazu passende Randmarkierungen.%
}
%    \end{macrocode}
% \end{macro}
%
%
% \section{Jens-Uwe's Example Text}
%
% TODO: These should be merged with other example text.
% \begin{macro}{\XmpText}
%   Example text command; the default definition (50) returns the whole text.
%    \begin{macrocode}
\newcommand*{\XmpText}[1][50]{%
  \ifnum #1<51
    Dieser Blindtext wird gerade von 130 Millionen Rezeptoren
    Ihrer Netzhaut erfasst.
    \ifnum #1>1
      Die Zellen werden dadurch in einen Erregungs\char\defaulthyphenchar
      \kern-1pt\linebreak
      \ifnum #1>2
        zustand versetzt, der sich vom Sehnerv in den 
        \ifnum #1>3
          hinteren Teil Ihres Gehirns ausbreitet.
          Von dort aus \"ubertr\"agt sich die Erregung in
          Sekundenbruchteilen auch in andere Bereiche Ihres
          Gro\ss hirns.
          Ihr Stirnlappen wird stimuliert.
          Von dort aus gehen jetzt Willens\char\defaulthyphenchar
          \kern-1pt\linebreak
    \fi\fi\fi
  \fi
  \ifnum #1>49
    impulse aus, die Ihr zentrales Nervensystem in konkrete Handlungen
    umsetzt. Kopf und Augen reagieren bereits.
    Sie folgen dem Text, nehmen die darin enthaltenen Informationen
    auf und leiten diese \"uber den Sehnerv weiter.
  \fi
}
%    \end{macrocode}
% \end{macro}
%
% \begin{macro}{\XmpTopText}
% \begin{macro}{\XmpBotText}
%   These two commands are language-dependent, since in some languages
%   additional commands may required.
%    \begin{macrocode}
\newcommand*{\XmpTopText}{\XmpText[3]}
\newcommand*{\XmpBotText}{\XmpText[2]}
%    \end{macrocode}
% \end{macro}
% \end{macro}
%
% \begin{macro}{\XmpMarginTextA}
% \begin{macro}{\XmpMarginTextB}
%   Margin notes.
%    \begin{macrocode}
\newcommand*{\XmpMarginTextA}{Netzhaut}
\newcommand*{\XmpMarginTextB}{(\textit{Retina})}
%    \end{macrocode}
% \end{macro}
% \end{macro}
%
%
% \section{Language Extensions}
%
% Some terms should be part of the captions of the language.
% \begin{macro}{\GuideLanguageExtensions}
%  This macro has to be defined, because the class will add it to the language
%  at |\begin{document}| and it will also call it.
%    \begin{macrocode}
\newcommand*{\GuideLanguageExtensions}{%
%    \end{macrocode}
% \begin{macro}{\pageautorefname}
%   The term ``page'' that will be put before the reference of a page set by
%   |\autopageref|.
%    \begin{macrocode}
  \let\pageautorefname\pagename
%    \end{macrocode}
% \end{macro}
% \begin{macro}{\subsectionautorefname}
% \begin{macro}{\subsubsectionautorefname}
% \begin{macro}{\paragraphautorefname}
% \begin{macro}{\subparagraphautorefname}
%   I'll call them all ``section''.
%    \begin{macrocode}
  \let\subsectionautorefname=\sectionautorefname
  \let\subsubsectionautorefname=\sectionautorefname
  \let\paragraphautorefname=\sectionrefname
  \let\subparagraphautorefname=\sectionrefname
%    \end{macrocode}
% \end{macro}
% \end{macro}
% \end{macro}
% \end{macro}
% \begin{macro}{\changelogname}
%   The name of the change log index.
%    \begin{macrocode}
  \def\changelogname{\"Anderungsliste}%
%    \end{macrocode}
% \end{macro}
% \begin{macro}{\descriptionname}
% \begin{macro}{\contentsname}
%    \begin{macrocode]
  \def\descriptionname{Bezeichnung}%
  \def\contentname{Inhalt}%
%    \end{macrocode}
% \end{macro}
% \end{macro}
% \begin{macro}{\lengthname}
%    \begin{macrocode}
  \def\lengthname{L\"ange}%
%    \end{macrocode}
% \end{macro}
%    \begin{macrocode}
}
%    \end{macrocode}
% \end{macro}
%
%
% \section{Hyphenation}
%
% This is not realy a good place to put them~-- but better late than never:
%    \begin{macrocode}
\hyphenation{%
  Ab-schnitts-ebe-ne
  Back-slash
  Brief-um-ge-bung Brief-um-ge-bun-gen
  Da-tei-na-me Da-tei-na-men Da-tei-na-mens
  Da-tei-na-men-er-wei-te-rung Da-tei-na-men-er-wei-te-rung-en
  Ein-trags-ebe-ne
  Gleich-zei-tig
  Hin-ter-grund-ebe-ne Hin-ter-grund-ebe-nen
  Ka-pi-tel-ebe-ne Ka-pi-tel-ebe-nen
  Pa-ket-au-to-ren
  Pa-pier-rand Pa-pier-ran-des
  Rand-ein-stel-lung Rand-ein-stel-lung-en
  Sei-ten-um-bruch
  Stan-dard-an-wei-sung Stan-dard-an-wei-sun-gen
  Stan-dard-ein-stel-lung Stan-dard-ein-stel-lun-gen
  Unix
  Zei-len-um-bruch Zei-len-um-bruchs Zei-len-um-br�-che Zei-len-um-br�-chen
 }
%    \end{macrocode}
%
%
%
\endinput
%%% Local Variables: 
%%% mode: doctex
%%% coding: iso-latin-1
%%% TeX-master: "../guide.tex"
%%% End: 

% \section{Jens-Uwe's Example Text}
%
% TODO: These should be merged with other example text.
% \begin{macro}{\XmpText}
%   Example text command; the default definition (50) returns the whole text.
%    \begin{macrocode}
\newcommand*{\XmpText}[1][50]{%
  \ifnum #1<51
    Dieser Blindtext wird gerade von 130 Millionen Rezeptoren
    Ihrer Netzhaut erfasst.
    \ifnum #1>1
      Die Zellen werden dadurch in einen Erregungs\char\defaulthyphenchar
      \kern-1pt\linebreak
      \ifnum #1>2
        zustand versetzt, der sich vom Sehnerv in den 
        \ifnum #1>3
          hinteren Teil Ihres Gehirns ausbreitet.
          Von dort aus \"ubertr\"agt sich die Erregung in
          Sekundenbruchteilen auch in andere Bereiche Ihres
          Gro\ss hirns.
          Ihr Stirnlappen wird stimuliert.
          Von dort aus gehen jetzt Willens\char\defaulthyphenchar
          \kern-1pt\linebreak
    \fi\fi\fi
  \fi
  \ifnum #1>49
    impulse aus, die Ihr zentrales Nervensystem in konkrete Handlungen
    umsetzt. Kopf und Augen reagieren bereits.
    Sie folgen dem Text, nehmen die darin enthaltenen Informationen
    auf und leiten diese \"uber den Sehnerv weiter.
  \fi
}
%    \end{macrocode}
% \end{macro}
%
% \begin{macro}{\XmpTopText}
% \begin{macro}{\XmpBotText}
%   These two commands are language-dependent, since in some languages
%   additional commands may required.
%    \begin{macrocode}
\newcommand*{\XmpTopText}{\XmpText[3]}
\newcommand*{\XmpBotText}{\XmpText[2]}
%    \end{macrocode}
% \end{macro}
% \end{macro}
%
% \begin{macro}{\XmpMarginTextA}
% \begin{macro}{\XmpMarginTextB}
%   Margin notes.
%    \begin{macrocode}
\newcommand*{\XmpMarginTextA}{Netzhaut}
\newcommand*{\XmpMarginTextB}{(\textit{Retina})}
%    \end{macrocode}
% \end{macro}
% \end{macro}
\end{filecontents*}

\begin{filecontents*}{\jobname-english.tex}
% ======================================================================
% guide-english.tex
% Copyright (c) Markus Kohm, 2002-2018
%
% This file is part of the LaTeX2e KOMA-Script bundle.
%
% This work may be distributed and/or modified under the conditions of
% the LaTeX Project Public License, version 1.3c of the license.
% The latest version of this license is in
%   http://www.latex-project.org/lppl.txt
% and version 1.3c or later is part of all distributions of LaTeX 
% version 2005/12/01 or later and of this work.
%
% This work has the LPPL maintenance status "author-maintained".
%
% The Current Maintainer and author of this work is Markus Kohm.
%
% This work consists of all files listed in manifest.txt.
% ----------------------------------------------------------------------
% guide-english.tex
% Copyright (c) Markus Kohm, 2002-2018
%
% Dieses Werk darf nach den Bedingungen der LaTeX Project Public Lizenz,
% Version 1.3c, verteilt und/oder veraendert werden.
% Die neuste Version dieser Lizenz ist
%   http://www.latex-project.org/lppl.txt
% und Version 1.3c ist Teil aller Verteilungen von LaTeX
% Version 2005/12/01 oder spaeter und dieses Werks.
%
% Dieses Werk hat den LPPL-Verwaltungs-Status "author-maintained"
% (allein durch den Autor verwaltet).
%
% Der Aktuelle Verwalter und Autor dieses Werkes ist Markus Kohm.
% 
% Dieses Werk besteht aus den in manifest.txt aufgefuehrten Dateien.
% ======================================================================
%
% Language dependencies (english) of the KOMA-Script guide
% Maintained by Markus Kohm
%
% ----------------------------------------------------------------------
%
% Sprachabhaengigkeiten (english) der KOMA-Script-Anleitung
% Verwaltet von Markus Kohm
%
% ======================================================================
%
\KOMAProvidesFile{guide-english.tex}
                 [$Date$
                  KOMA-Script guide language dependencies]
%
% \section{Extra Packages}
%
\RequirePackage[normal]{engord}
%
% \section{Title}
%
% \begin{macro}{\GuideSubject}
% \begin{macro}{\GuideTitle}
% \begin{macro}{\GuideSubTitle}
% \begin{macro}{\GuideAuthorHeadline}
% \begin{macro}{\GuideTranslatorHeadline}
% \begin{macro}{\GuideWarrantyHeadline}
% \begin{macro}{\GuideWarranty}
% \begin{macro}{\GuideCopyright}
% \begin{macro}{\GuideDedication}
%   Language dependencies of all the title pages.
%    \begin{macrocode}
\newcommand*{\GuideSubject}{The Guide}%
\newcommand*{\GuideTitle}{\KOMAScript}%
\newcommand*{\GuideSubTitle}{a versatile {\LaTeXe} bundle%
  % Ugly note
  \vfill
  \noindent Note: This document is a translation of the German \KOMAScript{}
  manual. Several authors have been involved to this translation. Some of them
  are native English speakers. Others, like me, are not. Improvements of the
  translation by native speakers or experts are welcome at all times!%
}%
\newcommand*{\GuideAuthorHeadline}{Authors of the {\KOMAScript} Bundle}%
\newcommand*{\GuideTranslatorHeadline}{English translation of this manual by:
} \newcommand*{\GuideWarrantyHeadline}{Legal Notes:}%
\newcommand*{\GuideWarranty}{There is no warranty for any part of the
  documented software. The authors have taken care in the preparation of this
  guide, but make no expressed or implied warranty of any kind and assume no
  responsibility for errors or omissions. No liability is assumed for
  incidental or consequential damages in connection with or arising out of the
  use of the information or programs contained here.\par\medskip%
  Many of the designations used by manufacturers and sellers to distinguish
  their products are claimed as trademarks. Where those designations appear in
  this book, and the authors were aware of a trademark claim, the designations
  have been printed with initial capital letters or in all capitals.}%
\newcommand*{\GuideCopyright}{Free screen version without any optimization of
  paragraph and page breaks\par\medskip%
  This guide is part of {\KOMAScript}, which is free under the terms and
  conditions of {\LaTeX} Project Public License Version 1.3c. A version of
  this license, which is valid for {\KOMAScript}, is part of {\KOMAScript} (see
  \File{lppl.txt}). Distribution of this manual\,---\,even if it is
  printed\,---\,is allowed provided that all parts of {\KOMAScript} are
  distributed with it. Distribution without the other parts of {\KOMAScript}
  requires an explicit, additional authorization by the authors.}%
\newcommand*{\GuideDedication}{To all my friends all over the world!}%
%    \end{macrocode}
% \end{macro}
% \end{macro}
% \end{macro}
% \end{macro}
% \end{macro}
% \end{macro}
% \end{macro}
% \end{macro}
% \end{macro}
%
%
% \section{CTAN Server to be Used}
%
% \begin{macro}{\GuideCTANserver}
% Unused since 2015-09-30.
%    \begin{macrocode}
%\newcommand*{\GuideCTANserver}{ftp.ctan.org}
%    \end{macrocode}
% \end{macro}
%
%
% \section{Terms}
%
% Some terms, e.g. used at index notes.
%
% \begin{macro}{\GuideClass}
% \begin{macro}{\GuideClassIndexCategory}
% \begin{macro}{\GuideClassIndexCategoryExpanded}
%   The term ``class''.
%    \begin{macrocode}
\newcommand*{\GuideClass}{class}
\newcommand*{\GuideClassIndexCategory}{classes}
\let\GuideClassIndexCategoryExpanded\GuideClassIndexCategory
%    \end{macrocode}
% \end{macro}
% \end{macro}
% \end{macro}
%
% \begin{macro}{\GuideCounter}
% \begin{macro}{\GuideCounterIndexCategory}
% \begin{macro}{\GuideCounterIndexCategoryExpanded}
%   The term ``counter''.
%    \begin{macrocode}
\newcommand*{\GuideCounter}{counter}
\newcommand*{\GuideCounterIndexCategory}{counters}
\let\GuideCounterIndexCategoryExpanded\GuideCounterIndexCategory
%    \end{macrocode}
% \end{macro}
% \end{macro}
% \end{macro}
%
% \begin{macro}{\GuideEnvironment}
% \begin{macro}{\GuideEnvIndexCategory}
% \begin{macro}{\GuideEnvIndexCategoryExpanded}
%   The term ``environment''.
%    \begin{macrocode}
\newcommand*{\GuideEnvironment}{environment}
\newcommand*{\GuideEnvIndexCategory}{environments}
\let\GuideEnvIndexCategoryExpanded\GuideEnvIndexCategory
%    \end{macrocode}
% \end{macro}
% \end{macro}
% \end{macro}
%
% \begin{macro}{\GuideExample}
%   The term ``Example'' used at a kind of heading, so it should be upper case.
%    \begin{macrocode}
\newcommand*{\GuideExample}{Example}
%    \end{macrocode}
% \end{macro}
%
% \begin{macro}{\GuideFile}
% \begin{macro}{\GuideFileIndexCategory}
% \begin{macro}{\GuideFileIndexCategoryExpanded}
%   The term ``file''.
%    \begin{macrocode}
\newcommand*{\GuideFile}{file}
\newcommand*{\GuideFileIndexCategory}{files}
\let\GuideFileIndexCategoryExpanded\GuideFileIndexCategory
%    \end{macrocode}
% \end{macro}
% \end{macro}
% \end{macro}
%
% \begin{macro}{\GuideFloatstyle}
% \begin{macro}{\GuideFloatstyleIndexCategory}
% \begin{macro}{\GuideFloatstyleIndexCategoryExpanded}
%   The term ``\Package{float} style''.
%    \begin{macrocode}
\newcommand*{\GuideFloatstyle}{\Package{float} style}
\newcommand*{\GuideFloatstyleIndexCategory}{\Package{float} styles}
\newcommand*{\GuideFloatstyleIndexCategoryExpanded}{float styles}
%    \end{macrocode}
% \end{macro}
% \end{macro}
% \end{macro}
%
% \begin{macro}{\GuideFontElement}
% \begin{macro}{\GuideFontElementIndexCategory}
% \begin{macro}{\GuideFontElementIndexCategoryExpanded}
%   The term ``element''.
%    \begin{macrocode}
\newcommand*{\GuideFontElement}{element}
\newcommand*{\GuideFontElementIndexCategory}{elements}
\let\GuideFontElementIndexCategoryExpanded\GuideFontElementIndexCategory
%    \end{macrocode}
% \end{macro}
% \end{macro}
% \end{macro}
%
% \begin{macro}{\GuideLength}
% \begin{macro}{\GuideLengthIndexCategory}
% \begin{macro}{\GuideLengthIndexCategoryExpanded}
%   The term ``length''.
%    \begin{macrocode}
\newcommand*{\GuideLength}{length}
\newcommand*{\GuideLengthIndexCategory}{lengths}
\let\GuideLengthIndexCategoryExpanded\GuideLengthIndexCategory
%    \end{macrocode}
% \end{macro}
% \end{macro}
% \end{macro}
%
% \begin{macro}{\GuideMacro}
% \begin{macro}{\GuideMacroIndexCategory}
% \begin{macro}{\GuideMacroIndexCategoryExpanded}
% \begin{macro}{\GuideCommand}
% \begin{macro}{\GuideCommandIndexCategory}
% \begin{macro}{\GuideCommandIndexCategoryExpanded}
%   The term ``command''.
%    \begin{macrocode}
\newcommand*{\GuideMacro}{command}
\let\GuideCommand\GuideMacro
\newcommand*{\GuideMacroIndexCategory}{commands}
\let\GuideCommandIndexCategory\GuideMacroIndexCategory
\let\GuideMacroIndexCategoryExpanded\GuideMacroIndexCategory
\let\GuideCommandIndexCategoryExpanded\GuideCommandIndexCategory
%    \end{macrocode}
% \end{macro}
% \end{macro}
% \end{macro}
% \end{macro}
% \end{macro}
% \end{macro}
%
% \begin{macro}{\GuideOption}
% \begin{macro}{\GuideOptionIndexCategory}
% \begin{macro}{\GuideOptionIndexCategoryExpanded}
%   The term ``option''.
%    \begin{macrocode}
\newcommand*{\GuideOption}{option}
\newcommand*{\GuideOptionIndexCategory}{options}
\let\GuideOptionIndexCategoryExpanded\GuideOptionIndexCategory
%    \end{macrocode}
% \end{macro}
% \end{macro}
% \end{macro}
%
% \begin{macro}{\GuidePackage}
% \begin{macro}{\GuidePackageIndexCategory}
% \begin{macro}{\GuidePackageIndexCategoryExpanded}
%   The term ``package''.
%    \begin{macrocode}
\newcommand*{\GuidePackage}{package}
\newcommand*{\GuidePackageIndexCategory}{packages}
\let\GuidePackageIndexCategoryExpanded\GuidePackageIndexCategory
%    \end{macrocode}
% \end{macro}
% \end{macro}
% \end{macro}
%
% \begin{macro}{\GuidePagestyle}
% \begin{macro}{\GuidePagestyleIndexCategory}
% \begin{macro}{\GuidePagestyleIndexCategoryExpanded}
%   The term ``page style''.
%    \begin{macrocode}
\newcommand*{\GuidePagestyle}{page style}
\newcommand*{\GuidePagestyleIndexCategory}{page styles}
\let\GuidePagestyleIndexCategoryExpanded\GuidePagestyleIndexCategory
%    \end{macrocode}
% \end{macro}
% \end{macro}
% \end{macro}
%
% \begin{macro}{\GuidePLength}
% \begin{macro}{\GuidePLengthIndexCategory}
% \begin{macro}{\GuidePLengthIndexCategoryExpanded}
% \begin{macro}{\GuidePseudoPrefix}
%   The prefix ``pseudo'' at the term ``pseudo length'' and the term itself.
%    \begin{macrocode}
\newcommand*{\GuidePseudoPrefix}{pseudo-}
\newcommand*{\GuidePLength}{\GuidePseudoPrefix\GuideLength}
\newcommand*{\GuidePLengthIndexCategory}{\GuidePseudoPrefix\GuideLengthIndexCategory}
\let\GuidePLengthIndexCategoryExpanded\GuidePLengthIndexCategory
%    \end{macrocode}
% \end{macro}
% \end{macro}
% \end{macro}
% \end{macro}
%
% \begin{macro}{\GuideVariable}
% \begin{macro}{\GuideVariableIndexCategory}
% \begin{macro}{\GuideVariableIndexCategoryExpanded}
%   The termo ``variable''.
%    \begin{macrocode}
\newcommand*{\GuideVariable}{variable}
\newcommand*{\GuideVariableIndexCategory}{variables}
\let\GuideVariableIndexCategoryExpanded\GuideVariableIndexCategory
%    \end{macrocode}
% \end{macro}
% \end{macro}
% \end{macro}
%
% \begin{macro}{\GuideAnd}
%   The ``and'' at a list of two.
%   \begin{macrocode}
\newcommand*{\GuideAnd}{ and }
%   \end{macrocode}
% \end{macro}
%
% \begin{macro}{\GuideListAnd}
%   The ``and'' at a list of more than two.
%   \begin{macrocode}
\newcommand*{\GuideListAnd}{, and }
%   \end{macrocode}
% \end{macro}
%
% \begin{macro}{\GuideLoadNonFree}
%   Note, that the non free manual contains more information.
%    \begin{macrocode}
\newcommand*{\GuideLoadNonFree}{%
  Currently, additional information on this topic can be found at the same
  point in the German \KOMAScript{} book \cite{book:komascript} only.
}
%    \end{macrocode}
% \end{macro}
%
%
% \section{Index}
%
% \begin{macro}{\GuidegenIndex}
% \begin{macro}{\GuidecmdIndex}
% \begin{macro}{\GuidecmdIndexShort}
% \begin{macro}{\GuidelenIndex}
% \begin{macro}{\GuidelenIndexShort}
% \begin{macro}{\GuideelmIndex}
% \begin{macro}{\GuideelmIndexShort}
% \begin{macro}{\GuidefilIndex}
% \begin{macro}{\GuidefilIndexShort}
% \begin{macro}{\GuideoptIndex}
% \begin{macro}{\GuideoptIndexShort}
%   The titles and short titles of each single index.
\newcommand*{\GuidegenIndex}{General Index}%
\newcommand*{\GuidecmdIndex}{Index of Commands, Environments, and Variables}%
\newcommand*{\GuidecmdIndexShort}{Index of Commands, etc.}%
\newcommand*{\GuidelenIndex}{Index of Lengths and Counters}%
\newcommand*{\GuidelenIndexShort}{Index of Lengths, etc.}%
\newcommand*{\GuideelmIndex}{Index of Elements Capable of
  Adjusting Fonts}%
\newcommand*{\GuideelmIndexShort}{Index of Elements}%
\newcommand*{\GuidefilIndex}{Index of Files, Classes, and Packages}%
\newcommand*{\GuidefilIndexShort}{Index of Files, etc.}%
\newcommand*{\GuideoptIndex}{Index of Class and Package Options}%
\newcommand*{\GuideoptIndexShort}{Index of Options}%
% \end{macro}
% \end{macro}
% \end{macro}
% \end{macro}
% \end{macro}
% \end{macro}
% \end{macro}
% \end{macro}
% \end{macro}
% \end{macro}
% \end{macro}
%
% \begin{macro}{\GuideIndexPreamble}
%   The preamble text of the whole index.
\newcommand*{\GuideIndexPreamble}{%
  There are two kinds of page numbers in this index. The numbers in bold
  show the pages where the topic is defined or explained. The numbers in
  ordinary type show the pages of where the topic is mentioned.%
}
% \end{macro}
%
%
% \begin{macro}{\GuideIndexSees}
%   These are all see index entries.
\newcommand*{\GuideIndexSees}{%
}
% \end{macro}
%
% \begin{macro}{\GuideIndexSeeAlsos}
%   These are all see also index entries.
\newcommand*{\GuideIndexSeeAlsos}{%
}
% \end{macro}
%
% \section{Bibliography}
%
% \begin{macro}{\GuideBibPreamble}
%    \begin{macrocode}
\newcommand*{\GuideBibPreamble}{%
  In the following, you will find many references. All of them are referenced
  in the main text. In many cases the reference points to documents or
  directories which can be accessed via the Internet. In these cases, the
  reference includes a URL instead of a publisher. If the reference points to
  a {\LaTeX} package then the URL is written in the form
  ``\url{CTAN://}\emph{destination}''.  The prefix ``\url{CTAN://}'' means the
  \TeX{} archive on a CTAN server or mirror. For example, you can replace the
  prefix with \url{http://mirror.ctan.org/}. For {\LaTeX} packages, it is also
  important to mention that we have tried to give a version number appropriate
  to the text that cites the reference. But for some packages is is very
  difficult to find a consistent version number and release date.
  Additionally, the given version is not always the current version.  If you
  want install new packages, be sure that the package is the most up-to-date
  version and check first if the package is already available on your system.%
}
%    \end{macrocode}
% \end{macro}
%
%
% \section{Change Log}
%
% \begin{macro}{\GuideChangeLogPreamble}
%   The preamble of the change log index.
%    \begin{macrocode}
\newcommand*{\GuideChangeLogPreamble}{%
  In this list of changes, you will find all significant changes to the user
  interface of the {\KOMAScript} bundle at the last few versions. The list was
  sorted by the names of the classes and packages and their version. The
  numbers after the version are the pages where the changes are described. In
  the margins of these pages, you will find corresponding version marks.%
}
%    \end{macrocode}
% \end{macro}
%
%
% \section{Language Extensions}
%
% Some terms should be part of the captions of the language.
% \begin{macro}{\GuideLanguageExtensions}
%  This macro has to be defined, because the class will add it to the language
%  at |\begin{document}| and it will also call it.
%    \begin{macrocode}
\newcommand*{\GuideLanguageExtensions}{%
%    \end{macrocode}
% \begin{macro}{\pageautorefname}
%   The term ``page'' that will be put before the reference of a page set by
%   |\autopageref|.
%    \begin{macrocode}
  \def\pageautorefname{page}%
%    \end{macrocode}
% \end{macro}
% \begin{macro}{\partautorefname}
% \begin{macro}{\figureautorefname}
% \begin{macro}{\tableautorefname}
% \begin{macro}{\subsectionautorefname}
% \begin{macro}{\subsubsectionautorefname}
% \begin{macro}{\paragraphautorefname}
% \begin{macro}{\subparagraphautorefname}
%   I'll call them all ``section''.
%    \begin{macrocode}
  \def\partautorefname{part}%
  \def\figureautorefname{figure}%
  \def\tableautorefname{table}%
  \def\appendixautorefname{appendix}%
  \let\subsectionautorefname=\sectionautorefname
  \let\subsubsectionautorefname=\sectionautorefname
  \let\paragraphautorefname=\sectionrefname
  \let\subparagraphautorefname=\sectionrefname
%    \end{macrocode}
% \end{macro}
% \end{macro}
% \end{macro}
% \end{macro}
% \end{macro}
% \end{macro}
% \end{macro}
% \begin{macro}{\changelogname}
%   The name of the change log index.
%    \begin{macrocode}
  \def\changelogname{Change Log}%
%    \end{macrocode}
% \end{macro}
% \begin{macro}{\descriptionname}
% \begin{macro}{\contentsname}
%    \begin{macrocode]
  \def\descriptionname{description}%
  \def\contentname{contents}%
%    \end{macrocode}
% \end{macro}
% \end{macro}
% \begin{macro}{\lengthname}
%    \begin{macrocode}
  \def\lengthname{length}%
%    \end{macrocode}
% \end{macro}
%    \begin{macrocode}
}
%    \end{macrocode}
% \end{macro}
%
%
% \section{Hyphenation}
%
% This is not realy a good place to put them~-- but better late than never:
%    \begin{macrocode}
\hyphenation{%
}
%    \end{macrocode}
%
%
%
\endinput
%%% Local Variables: 
%%% mode: doctex
%%% coding: us-ascii
%%% TeX-master: "guide"
%%% End: 

% \begin{macro}{\XmpText}
%   Example text command; the default definition (50) returns the whole text.
%    \begin{macrocode}
\newcommand*{\XmpText}[1][50]{%
  \ifnum #1<51
    This f\/ill text is currently seized by 130 million receptors
    in your retina.
    \ifnum #1>1
      Thereby the nerve cells are put in a state of stimulation,
      which spreads
      \ifnum #1>2
        into the rear part of your brain originating from
        \ifnum #1>3
          the optic nerve.
          From there the stimulation is transmitted in a split second
          also in other parts of your cerebrum.
          Your frontal lobe becomes stimulated.
          Intention-impulses spread from there, which your
          central nervous\setlength{\parfillskip}{0pt}%
    \fi\fi\fi
  \fi
  \ifnum #1>49
    system transforms in actual deeds.
    Head and eyes already react.
    They follow the text, taking the information present there
    and transmit them via the optic nerve.
  \fi
}
%    \end{macrocode}
% \end{macro}
%
% \begin{macro}{\XmpTopText}
% \begin{macro}{\XmpBotText}
%   These two commands are language-dependent, since in some languages
%   additional commands may required.
%    \begin{macrocode}
\newcommand*{\XmpTopText}{\XmpText[3]}
\newcommand*{\XmpBotText}{\XmpText[2]}
%    \end{macrocode}
% \end{macro}
% \end{macro}
%
% \begin{macro}{\XmpMarginTextA}
% \begin{macro}{\XmpMarginTextB}
%   Margin notes.
%    \begin{macrocode}
\newcommand*{\XmpMarginTextA}{Retina}
\newcommand*{\XmpMarginTextB}{}
%    \end{macrocode}
% \end{macro}
% \end{macro}
\end{filecontents*}

\documentclass[ngerman,english]{scrguide}

\makeatletter
\renewcommand*{\@pnumwidth}{2em}
\makeatother

\KOMAProvidesFile{scrpage2.tex}%
                 [$Date: 2013-07-22 $
                  a user guide to obsolete package scrpage2]

\begin{document}

\addparttocentry{}{English}
\def\theHchapter{E.\thechapter}
\setcounter{chapter}{0}
\renewcommand*{\thepage}{E.\arabic{page}}
\title{The Obsolete Package \Package{scrpage2}}
\subtitle{Extract from former versions of the \KOMAScript{} Manual}
\author{Markus Kohm\and Jens-Uwe-Morakswi}
\date{2018-02-07}
\maketitle

\tableofcontents

\translator{Jens-Uwe Morawski\and Karl-Heinz Zimmer\and Christoph Bier\and
  Gernot Hassenpflug\and Markus Kohm}

\chapter{Adapting Page Headers and Footers with \Package{scrpage2}}
\labelbase{scrpage-en}

%
\BeginIndexGroup
\BeginIndex{Package}{scrpage2}%
\BeginIndex{}{page>style}
From \KOMAScript{} 3.12 the completely newly implemented package
\Package{scrlayer-scrpage} replaces the old \Package{scrpage2}. The new
package is a consequently developed extension of the design of
\Package{scrpage2}. In difference to \Package{scrpage2} it provides the
extended option interface of the \KOMAScript{} classes. Because
\Package{scrlayer-scrpage} predominates \Package{scrpage2} it is recommended
to not longer use \Package{scrpage2} but \Package{scrlayer-scrpage}. Because
of this the current version of \Package{scrpage2} is the final one. All
development resources will go into \Package{scrlayer-scrpage}. For more
information about \Package{scrlayer-scrpage} see the \KOMAScript{} manual.

\begin{Explain}
  In place of \Package{scrpage2} or \Package{scrlayer-scrpage} you can of
  course make use of \Package{fancyhdr}. However, \Package{scrpage2} and
  especially \Package{scrlayer-scrpage} integrated markedly better with the
  {\KOMAScript} bundle. For this reason, and because at the time the
  forerunner to \Package{fancyhdr} was missing many features,
  \Package{scrpage2} was developed. Naturally, \Package{scrpage2} and
  \Package{scrlayer-scrpage} are not limited to use only with the
  {\KOMAScript} classes, but can just as easily be used with other document
  classes.
\end{Explain}


\section{Basic Functionality}\label{sec:scrpage-en.basics}

\begin{Explain}% Introduction to headings
  To understand the following description, an overview of {\LaTeX}'s fairly
  involved header and footer mechanism is needed.  The {\LaTeX} kernel defines
  the page styles \PageStyle{empty}, which produces a completely empty header
  and footer, and \PageStyle{plain}, which produces usually only a page number
  in the footer and an empty header.  Apart from these, many document classes
  provide the style \PageStyle{headings}, which allows more complex style
  settings and running
  headings\Index{headings>running}\Index{headings>automatic}.  The
  \PageStyle{headings} style often has a related variant,
  \PageStyle{myheadings}, which is similar except for switching off the
  running headings\Index{headings>manual} and reverting them to manual control
  by the user.  A more detailed description is given in the page style section
  of the \KOMAScript{} manual where it is also noted that some {\LaTeX}
  commands automatically switch to another page style\,---\,usually page style
  \PageStyle{plain}\,---\,for the current page.
\end{Explain}


Package \Package{scrpage2} does not distinguish between page styles with
automatic, running headings and page styles with manual headings. The way to
deal with automatic and manual headings is independent from the page style and
so the page style is independent from the choice of automatic or manual
headings. More information about this in \autoref{sec:scrpage-en.basics.mark}.

  
\subsection{Predefined Page Styles}\label{sec:scrpage-en.basics.buildIn}

One of the basic features of \Package{scrpage2} is a set of predefined,
configurable page styles.

% \lehead ...
% scrheadings
% \headmark \pagemark 

\begin{Declaration}
  \PageStyle{scrheadings}%
  \PageStyle{scrplain}
\end{Declaration}%
Package \Package{scrpage2} delivers its own page style, named
\PageStyle{scrheadings}, which can be activated with the
\Macro{pagestyle}\PParameter{scrheadings}. When this page style is in use, an
appropriate \PageStyle{scrplain} page style is used for the plain page style.
In this case \emph{appropriate} means that this new plain page style is also
configureable by the commands introduced in
\autoref{sec:scrpage-en.basics.format}, which, for example, configure the header
and footer width and complies within the basic layout. Neither the activation
of \PageStyle{scrheadings} nor the attendant change to the appropriate plain
page style, \PageStyle{scrplain}, influences the mode of manual or automatic
headings\Index{headings>automatic}\Index{headings>manual} (see
\autoref{sec:scrpage-en.basics.mark}). The \PageStyle{scrplain} page style can
also be activated directly with \Macro{pagestyle}.


\begin{Declaration}
  \Macro{lehead}%
    \OParameter{scrplain-left-even}\Parameter{scrheadings-left-even}%
  \Macro{cehead}%
    \OParameter{scrplain-center-even}\Parameter{scrheadings-center-even}%
  \Macro{rehead}%
    \OParameter{scrplain-right-even}\Parameter{scrheadings-right-even}%
  \Macro{lefoot}%
    \OParameter{scrplain-left-even}\Parameter{scrheadings-left-even}%
  \Macro{cefoot}%
    \OParameter{scrplain-center-even}\Parameter{scrheadings-center-even}%
  \Macro{refoot}%
    \OParameter{scrplain-right-even}\Parameter{scrheadings-right-even}%
  \Macro{lohead}%
    \OParameter{scrplain-left-odd}\Parameter{scrheadings-left-odd}%
  \Macro{cohead}%
    \OParameter{scrplain-center-odd}\Parameter{scrheadings-center-odd}%
  \Macro{rohead}%
    \OParameter{scrplain-right-odd}\Parameter{scrheadings-right-odd}%
  \Macro{lofoot}%
    \OParameter{scrplain-left-odd}\Parameter{scrheadings-left-odd}%
  \Macro{cofoot}%
    \OParameter{scrplain-center-odd}\Parameter{scrheadings-center-odd}%
  \Macro{rofoot}%
    \OParameter{scrplain-right-odd}\Parameter{scrheadings-right-odd}%
  \Macro{ihead}%
    \OParameter{scrplain-inside}\Parameter{scrheadings-inside}%
  \Macro{chead}%
    \OParameter{scrplain-centered}\Parameter{scrheadings-centered}%
  \Macro{ohead}%
    \OParameter{scrplain-outside}\Parameter{scrheadings-outside}%
  \Macro{ifoot}%
    \OParameter{scrplain-inside}\Parameter{scrheadings-inside}%
  \Macro{cfoot}%
    \OParameter{scrplain-centered}\Parameter{scrheadings-centered}%
  \Macro{ofoot}%
    \OParameter{scrplain-outside}\Parameter{scrheadings-outside}
\end{Declaration}%
The page style of \Package{scrpage2} are defined to have flexible configurable
header and footer. To achieve this, the page styles include three boxes in
both the header and the footer. The contents of these boxes may be modified
easily. The commands modifying the content of these boxes can be seen in
\autoref{fig:scrpage-en.leheadetall}.  Commands in the middle column modify the
box contents on both odd and even pages. All of the commands have an optional
and a mandatory argument. The option Argument influences the content of
corresponding box of the plain page style, \PageStyle{scrplain}. The mandatory
argument influences the content of the corresponding box of the page style
\PageStyle{scrheadings}.

\begin{figure}
%  \centering
  \setcapindent{0pt}%
  \begin{captionbeside}
    [Commands for modification of \Package{scrpage2} page styles elements]%
    {Commands for modification of page styles \PageStyle{scrheadings} and
      \PageStyle{scrplain} and their relationship to header and footer
      elements}
    [l]
  \setlength{\unitlength}{.95mm}\begin{picture}(100,65)(0,8)\small
    \put(0,12){\dashbox{2}(40,56){even page}}
    \put(60,12){\dashbox{2}(40,56){odd page}}
                                % items top left
    \put(1,63){\framebox(8,4){~}}
    \put(16,63){\framebox(8,4){~}}
    \put(31,63){\framebox(8,4){~}}
                                % items top right
    \put(61,63){\framebox(8,4){~}}
    \put(76,63){\framebox(8,4){~}}
    \put(91,63){\framebox(8,4){~}}
                                % items bottom left
    \put(1,13){\framebox(8,4){~}}
    \put(16,13){\framebox(8,4){~}}
    \put(31,13){\framebox(8,4){~}}
                                % items bottom right
    \put(61,13){\framebox(8,4){~}}
    \put(76,13){\framebox(8,4){~}}
    \put(91,13){\framebox(8,4){~}}
                                % commands and arrow
                                % for twoside
                                % 
    \put(50,65){\makebox(0,0){\DescRef{scrpage-en.cmd.ihead}}}
    \put(44,65){\vector(-1,0){4}}\put(56,65){\vector(1,0){4}}
    \put(50,58){\makebox(0,0){\DescRef{scrpage-en.cmd.chead}}}
    \put(44,58){\line(-1,0){24}}\put(56,58){\line(1,0){24}}
    \put(20,58){\vector(0,1){4}}\put(80,58){\vector(0,1){4}}
    \put(50,51){\makebox(0,0){\DescRef{scrpage-en.cmd.ohead}}}
    \put(44,51){\line(-1,0){40}}\put(56,51){\line(1,0){40}}
    \put(4,51){\vector(0,1){11}}\put(96,51){\vector(0,1){11}}
                                %
    \put(50,15){\makebox(0,0){\DescRef{scrpage-en.cmd.ifoot}}}
    \put(56,15){\vector(1,0){4}}\put(44,15){\vector(-1,0){4}}
    \put(50,22){\makebox(0,0){\DescRef{scrpage-en.cmd.cfoot}}}
    \put(44,22){\line(-1,0){24}}\put(56,22){\line(1,0){24}}
    \put(20,22){\vector(0,-1){4}}\put(80,22){\vector(0,-1){4}}
    \put(50,29){\makebox(0,0){\DescRef{scrpage-en.cmd.ofoot}}}
    \put(44,29){\line(-1,0){40}}\put(56,29){\line(1,0){40}}
    \put(4,29){\vector(0,-1){11}}\put(96,29){\vector(0,-1){11}}
                                % commands for oneside
    \put(5,69){\makebox(0,0)[b]{\DescRef{scrpage-en.cmd.lehead}}}
    \put(20,69){\makebox(0,0)[b]{\DescRef{scrpage-en.cmd.cehead}}}
    \put(35,69){\makebox(0,0)[b]{\DescRef{scrpage-en.cmd.rehead}}}
                                %
    \put(65,69){\makebox(0,0)[b]{\DescRef{scrpage-en.cmd.lohead}}}
    \put(80,69){\makebox(0,0)[b]{\DescRef{scrpage-en.cmd.cohead}}}
    \put(95,69){\makebox(0,0)[b]{\DescRef{scrpage-en.cmd.rohead}}}
                                %
    \put(5,11){\makebox(0,0)[t]{\DescRef{scrpage-en.cmd.lefoot}}}
    \put(20,11){\makebox(0,0)[t]{\DescRef{scrpage-en.cmd.cefoot}}}
    \put(35,11){\makebox(0,0)[t]{\DescRef{scrpage-en.cmd.refoot}}}
                                %
    \put(65,11){\makebox(0,0)[t]{\DescRef{scrpage-en.cmd.lofoot}}}
    \put(80,11){\makebox(0,0)[t]{\DescRef{scrpage-en.cmd.cofoot}}}
    \put(95,11){\makebox(0,0)[t]{\DescRef{scrpage-en.cmd.rofoot}}}
  \end{picture}
  \end{captionbeside}
%  \caption[Commands for modification of \Package{scrpage2} page styles
%  elements]%
%  {Commands for modification of page styles \PageStyle{scrheadings} and
%    \PageStyle{scrplain} and their relationship to header and footer
%    elements}
  \label{fig:scrpage-en.leheadetall}
\end{figure}

\begin{Example}
  If one wants the page number within \PageStyle{scrheadings} be placed in the
  middle of the footer, then following can be used:
\begin{lstcode}
  \cfoot{\pagemark}
\end{lstcode}
  The next example shows how to place both running heading and page
  number\Index{page>number}\Index{pagination} in the header; the running
  heading\Index{headings} inside and the page number outside:
\begin{lstcode}
  \ohead{\pagemark}
  \ihead{\headmark}
  \cfoot{}
\end{lstcode}
  The command \Macro{cfoot}\PParameter{} is only required in order to
  empty the item in the middle of the footer, which normally contains
  the page number.
\end{Example}

The commands which are associated with only one item can be used for
more advanced settings.

\begin{Example}
  Assuming one has the order to write an annual report of a company, one could
  use commands like this:
\begin{lstcode}
  \ohead{\pagemark}
  \rehead{Annual Report 2001}
  \lohead{\headmark}
  \cefoot{TheCompanyName Inc.}
  \cofoot{Department: Development}
\end{lstcode}
  In order to keep the data in the footer synchronized with the content
  of the document, the footer has to be updated using \Macro{cofoot}
  when a new department is discussed in the report.
\end{Example}

As mentioned above, there is a new plain page style which corresponds to
\PageStyle{scrheadings}. Since it should also be possible to customize this
style, the commands support an optional argument with which the contents of
the appropriate fields of this plain page style can be modified.

\begin{Example}
  The position of the page number for the page style \PageStyle{scrheadings}
  can be declared as follows:
\begin{lstcode}
  \cfoot[\pagemark]{}
  \ohead[]{\pagemark}
\end{lstcode}
  When the command \Macro{chapter}, after it has started a new page, now
  switches to the page style \PageStyle{plain}, then the page number is
  centered in the footer.
\end{Example}
%
\EndIndexGroup


\begin{Declaration}
  \Macro{clearscrheadings}%
  \Macro{clearscrplain}%
  \Macro{clearscrheadfoot}
\end{Declaration}%
If one wants to redefine both the page style \PageStyle{scrheadings} and
the corresponding plain page style, frequently one must empty
some already occupied page elements. Since one rarely fills all items
with new content, in most cases several instructions with empty
parameters are necessary.  With the help of these three instructions
the quick and thorough deletion is possible.  While
\Macro{clearscrheadings} only deletes all fields of the page style
\PageStyle{scrheadings}, and \Macro{clearscrplain} deletes all fields of
the corresponding plain page style, \Macro{clearscrheadfoot}
sets all fields of both page styles to empty.
\begin{Example}
  If one wants to reset the page style to the default {\KOMAScript}
  settings, independent of the actual configuration, only these three
  commands are sufficient:
\begin{lstcode}
  \clearscrheadfoot
  \ohead{\headmark}
  \ofoot[\pagemark]{\pagemark}
\end{lstcode}
  Without the commands \Macro{clearscrheadfoot},
  \Macro{clearscrheadings} and \Macro{clearscrplain}, six commands with
  additional nine empty arguments would be required:
\begin{lstcode}
  \ihead[]{}
  \chead[]{}
  \ohead[]{\headmark}
  \ifoot[]{}
  \cfoot[]{}
  \ofoot[\pagemark]{\pagemark}
\end{lstcode}
  Of course, for a specific configuration, some of them could be
  dropped.%
\end{Example}%
%
\EndIndexGroup
%
\EndIndexGroup


In the previous examples two commands were used which have not been
introduced yet. The description of these commands follows.

\begin{Declaration}
  \Macro{leftmark}%
  \Macro{rightmark}
\end{Declaration}%
These two instructions make it possible to access the running headings, which
are normally meant for the left or for the right page. These two instruction
are not made available by \Package{scrpage2}, but directly by the {\LaTeX}
kernel. When in this section running headings of the left page or the right
page are mentioned, this refers to the contents of \Macro{leftmark} or
\Macro{rightmark}, respectively.
%
\EndIndexGroup


\begin{Declaration}
  \Macro{headmark}
\end{Declaration}%
This command gives access to the content of running headings.  In
contrast to \Macro{leftmark} and \Macro{rightmark}, one need not
regard the proper assignment to left or right page.%
%
\EndIndexGroup


\begin{Declaration}
  \Macro{pagemark}
\end{Declaration}%
This command returns the formatted page number. The formatting can be
controlled by \DescRef{scrpage-en.cmd.pnumfont}, introduced in
\autoref{sec:scrpage-en.basics.format}, \DescPageRef{scrpage-en.cmd.pnumfont},
which \Macro{pagemark} heeds automatically.%
%
\EndIndexGroup


\begin{Declaration}
  \PageStyle{useheadings}
\end{Declaration}%
The package \Package{scrpage2} is meant primarily for use of the supplied
styles or for defining one's own styles. However, it may be necessary to shift
back also to a style provided by the document class.  It might appear that
this should be done with
\Macro{pagestyle}\PParameter{headings}\IndexCmd{pagestyle}, but this has the
disadvantage that commands \DescRef{scrpage-en.cmd.automark} and \DescRef{scrpage-en.cmd.manualmark}, to be
discussed shortly, do not function as expected.  For this reason one should
shift back to the original styles using
\Macro{pagestyle}\PParameter{useheadings}\IndexCmd{pagestyle}, which chooses
the correct page styles automatically for both manual and automatic running
headings.%
%
\EndIndexGroup


\subsection{Manual and Running Headings}
\label{sec:scrpage-en.basics.mark}
% \automark \manualmark
\BeginIndexGroup
\BeginIndex{}{headings>running}%
\BeginIndex{}{headings>automatic}%
\BeginIndex{}{headings>manual}%
Usually there is a \emph{my}-version of the \PageStyle{headings}
page style.  If such a page style is active, then the running headings
are no longer updated no longer automatically and become manual
headings.  With \Package{scrpage2} a different path is taken. Whether
the headings are running or manual is determined by the instructions
\DescRef{scrpage-en.cmd.automark} and \DescRef{scrpage-en.cmd.manualmark}, respectively.  The default
can be set already while loading of the package, with the options
\DescRef{scrpage-en.option.automark} and \DescRef{scrpage-en.option.manualmark} (see
\autoref{sec:scrpage-en.basics.options},
\DescPageRef{scrpage-en.option.automark}).


\begin{Declaration}
  \Macro{manualmark}
\end{Declaration}%
As the name suggests, \Macro{manualmark} switches off the updating of
the running headings and makes them manual. It is left to the user to
update and provide contents for the headings.  For that purpose the
instructions \Macro{markboth}\IndexCmd{markboth} and
\Macro{markright}\IndexCmd{markright} are available.%
%
\EndIndexGroup


\begin{Declaration}
  \Macro{automark}\OParameter{right page}\Parameter{left page}
\end{Declaration}%
The macro \Macro{automark} activates the automatic updating, that is, running
headings.  For the two parameters the designations of the document sectioning
level whose title is to appear in appropriate place are to be used.  Valid
values for the parameters are:
\PValue{part}\ChangedAt{v2.2}{\Package{scrpage2}}, \PValue{chapter},
\PValue{section}, \PValue{subsection}, \PValue{subsubsection},
\PValue{paragraph}, and \PValue{subparagraph}.  For most of the classes use of
\PValue{part} will not produce the expected result. So far only {\KOMAScript}
classes from version~2.9s up are known to support this value. The optional
argument \PName{right page} is understandably meant only for double-sided
documents. In the single-sided case one should normally not use it.  With the
help of the option
\DescRef{scrpage-en.option.autooneside}\IndexOption{autooneside}\important{\DescRef{scrpage-en.option.autooneside}}
one can also set that the optional argument in single-sided mode is ignored
automatically (see \autoref{sec:scrpage-en.basics.options},
\DescPageRef{scrpage-en.option.autooneside}).
%
\begin{Example}
  Assuming that the document uses a \emph{book} class, whose topmost
  section level is \emph{chapter}, then after a preceding
  \DescRef{scrpage-en.cmd.manualmark}
\begin{lstcode}
  \automark[section]{chapter}
\end{lstcode}
  restores the original behaviour.  If one prefers lower section levels
  in running headings, the following can be used:
\begin{lstcode}
  \automark[subsection]{section}
\end{lstcode}
%  How useful the last declaration is, everybody has to decide
%  for themselves.
\end{Example}

\begin{Explain}
  For the upper section level, the data of the headings is set by the command
  \Macro{markboth}\IndexCmd{markboth}, while that for the lower section level
  by \Macro{markright}\IndexCmd{markright} or
  \Macro{markleft}\IndexCmd[indexmain]{markleft}.  These commands are called
  indirectly by the sectioning commands.  The macro \Macro{markleft} is
  provided by the package \Package{scrpage2} and is defined similarly to
  \Macro{markright} in the {\LaTeX} kernel.  Although \Macro{markleft} is not
  defined as an internal command, the direct use is not recommended.
\end{Explain}
\EndIndexGroup
%
\EndIndexGroup


\subsection{Formatting of Header and Footer}
\label{sec:scrpage-en.basics.format}
% \headfont \pnumfont
% \setheadwidth \setfootwidth
% \set(head|foot)(top|sep|bot)line
The previous section concerned itself mainly with the contents of the
header and footer. This is of course not sufficient to satisfy
formative ambitions. Therefore we devote this section exclusively to
this topic.

\begin{Declaration}
  \Macro{headfont}%
  \Macro{footfont}%
  \Macro{pnumfont}
\end{Declaration}%
The command \Macro{headfont} contains the commands which determine the font of
header and footer lines.  Command \Macro{footfont} contains the difference of
the footer to that. The difference for the style of the page number is defined
by the command \Macro{pnumfont}.
\begin{Example}
  If, for example, one wants the header to be typeset in bold sans serif, the
  footer in non-bold sans serif, and the page number in a slanted serif style,
  then one can use the following definitions:
\begin{lstcode}
  \renewcommand{\headfont}{\normalfont\sffamily\bfseries}
  \renewcommand*{\footfont}{\normalfont\sffamily}
  \renewcommand{\pnumfont}{\normalfont\rmfamily\slshape}
\end{lstcode}
\end{Example}

\BeginIndex{FontElement}{pagehead}%
\BeginIndex{FontElement}{pagefoot}%
\BeginIndex{FontElement}{pagenumber}%
From version 2.8p of the {\KOMAScript} classes a new unified user interface
scheme is implemented for font attributes. If \Package{scrpage2} is used
together with one of these classes, then it is recommended to set up font
attributes in the manner described in the \KOMAScript{} manual.
\begin{Example}
  Instead of \Macro{renewcommand} the command \Macro{setkomafont}
  should be used to configure the font attributes. The previous
  definitions can then be written as:
\begin{lstcode}
  \setkomafont{pagehead}\normalfont\sffamily\bfseries}
  \setkomafont{pagefoot}{\normalfont\sffamily}
  \setkomafont{pagenumber}{\normalfont\rmfamily\slshape}
\end{lstcode}
\end{Example}
\EndIndexGroup


\begin{Declaration}
\Macro{setheadwidth}\OParameter{shift}\Parameter{width}%
\Macro{setfootwidth}\OParameter{shift}\Parameter{width}
\end{Declaration}%
Normally the widths of header and footer lines correspond to the width of the
text body.  The commands
\Macro{setheadwidth}\Index{header>width}\Index{head>width} and
\Macro{setfootwidth}\Index{footer>width}\Index{foot>width} enable the user to
adapt in a simple manner the widths to his needs.  The mandatory argument
\PName{width} takes the value of the desired width of the page header or
footer, while \PName{shift} is a length parameter by which amount the
appropriate item is shifted toward the outside page edge.

For the most common situations the mandatory argument \PName{width}
accepts the following symbolic values:
\begin{labeling}[\,--]{\PValue{textwithmarginpar}}
\item[\PValue{paper}] the width of the paper
\item[\PValue{page}] the width of the page
\item[\PValue{text}] the width of the text body
\item[\PValue{textwithmarginpar}] the width of the text body including margin
\item[\PValue{head}] the current header width
\item[\PValue{foot}] the current footer width
\end{labeling}
The difference between \PValue{paper} and \PValue{page} is that \PValue{page}
means the width of the paper less the binding correction if the package
\Package{typearea}\IndexPackage{typearea} is used (see the chapter about
\Package{typearea} in the \KOMAScript{} manual). Without \Package{typearea}
both values are identical.

\begin{Example}
  Assume that one wants a layout like that of \emph{The
    {\LaTeX}\,Companion}, where the header projects into the
  margin. This can be obtained with:
\begin{lstcode}
  \setheadwidth[0pt]{textwithmarginpar}
\end{lstcode}
%
  which appears like this on an odd page:
%
  \begin{XmpTopPage}
    \XmpHeading{10,25}{85}
    \thinlines\XmpRule{10,23}{85}
    \XmpSetText[\XmpTopText]{10,21}
    \XmpMarginNote{83,11.8}
  \end{XmpTopPage}
%
  If the footer line should have the same width and alignment, then two
  ways to set this up are possible. The first way simply repeats the
  settings for the case of the footer line:
\begin{lstcode}
  \setfootwidth[0pt]{textwithmarginpar}
\end{lstcode}
%
  In the second way the symbolic value \PValue{head} is used, since the
  header already has the desired settings.
\begin{lstcode}
  \setfootwidth[0pt]{head}
\end{lstcode}
\end{Example}

If no \PName{shift} is indicated, i.\,e., without the optional argument,
then the header or footer appears arranged symmetrically on the page.
In other words, a value for the \PName{shift} is determined
automatically to correspond to the current page shape.
%
\begin{Example}
  Continuing with the previous example, we remove the optional
  argument:
\begin{lstcode}
  \setheadwidth{textwithmarginpar}
\end{lstcode}
%
  which appears like this on an odd page:
%
  \begin{XmpTopPage}
    \XmpHeading{5,25}{85}
    \thinlines\XmpRule{5,23}{85}
    \XmpSetText[\XmpTopText]{10,21}
    \XmpMarginNote{83,11.8}
  \end{XmpTopPage}
\end{Example}

As can be seen, the header is now shifted inward, while the header
width has not changed. The shift is calculated in a way that the
configuration of the typearea become visible also here.%
%
\EndIndexGroup


\begin{Declaration}
  \Macro{setheadtopline}\OParameter{length}\Parameter{thickness}%
                        \OParameter{commands}%
  \Macro{setheadsepline}\OParameter{length}\Parameter{thickness}%
                        \OParameter{commands}%
  \Macro{setfootsepline}\OParameter{length}\Parameter{thickness}%
                        \OParameter{commands}%
  \Macro{setfootbotline}\OParameter{length}\Parameter{thickness}%
                        \OParameter{commands}
\end{Declaration}%
Corresponding to the size configuration parameters of header and
footer there are commands to modify the rules above and below the
header and footer. But first of all the rules should be activated. See options  
\DescRef{scrpage-en.option.headtopline}, \DescRef{scrpage-en.option.headsepline},
\DescRef{scrpage-en.option.footsepline}, and \DescRef{scrpage-en.option.footbotline} in
\autoref{sec:scrpage-en.basics.options},
\DescPageRef{scrpage-en.option.headsepline} for this.

\begin{labeling}[\,--]{\Macro{setfootsepline}}
\item[\Macro{setheadtopline}] configures the line above the header
\item[\Macro{setheadsepline}] configures the line below the header
\item[\Macro{setfootsepline}] configures the line above the footer
\item[\Macro{setfootbotline}] configures the line below the footer
\end{labeling}

The mandatory argument \PName{thickness} determines how strongly the
line is drawn. The optional argument \PName{length} accepts the same
symbolic values as \PName{width} for \DescRef{scrpage-en.cmd.setheadwidth}, as well as
also a normal length expression.  As long as the optional argument
\PName{length} is not assigned a value, the appropriate line length
adapts automatically the width of the header or the footer.

Use \PValue{auto} in the length argument to restore this automation
for the length of a line.

\BeginIndexGroup
\BeginIndex{FontElement}{headtopline}%
\BeginIndex{FontElement}{headsepline}%
\BeginIndex{FontElement}{footsepline}%
\BeginIndex{FontElement}{footbotline}%
\BeginIndex{FontElement}{footbottomline}%
The\ChangedAt{v2.2}{\Package{scrpage2}} optional argument \PName{commands} may
be used to specify additional commands to be executed before the respective
line is drawn.  For example, such commands could be used for changing the
color\Index{header>color}\Index{footer>color}%
\Index{head>color}\Index{foot>color}%
\Index{color>in header}\Index{color>in footer} of the line.  When using a
{\KOMAScript} class you could also use
\Macro{setkomafont}\IndexCmd{setkomafont} to specify commands for one of the
elements \FontElement{headtopline}\important[i]{\FontElement{headtopline}\\
  \FontElement{headsepline}\\
  \FontElement{footsepline}\\
  \FontElement{footbottomline}}, \FontElement{headsepline},
\FontElement{footsepline}, \FontElement{footbottomline}, or
\FontElement{footbotline}.  These can then be extended via
\Macro{addtokomafont}\IndexCmd{addtokomafont}.  See the \KOMAScript{} manual
for details on the \Macro{setkomafont} and \Macro{addtokomafont} commands.
\EndIndexGroup


\begin{Declaration}
  \labelsuffix[auto+current]
  \Macro{setheadtopline}\POParameter{auto}%
                        \PParameter{current}%
  \labelsuffix[auto]
  \Macro{setheadtopline}\POParameter{auto}%
                        \Parameter{}%
  \labelsuffix[auto+]
  \Macro{setheadtopline}\POParameter{auto}%
                        \Parameter{}\OParameter{}
\end{Declaration}%
The arguments shown here for the command \DescRef{scrpage-en.cmd.setheadtopline} are of
course valid for the other three configuration commands too.

If the mandatory parameter has the value \PValue{current} or has been
left empty, then the line thickness is not changed.  This may be used
to modify the length of the line without changing its thickness.

If the optional argument \PName{commands} is omitted, then all command
settings that might have been specified before will remain active,
while an empty \PName{commands} argument will revoke any previously
valid commands.

\begin{Example}
  If the header, for example, is to be contrasted by a strong line of
  2\,pt above and a normal line of 0.4\,pt between header and body,
  one can achieve this with:
\begin{lstcode}
  \setheadtopline{2pt}
  \setheadsepline{.4pt}
\end{lstcode}
  Additionally\textnote{Attention!} the options \DescRef{scrpage-en.option.headtopline} and
  \DescRef{scrpage-en.option.headsepline} have to be used preferably globally in the optional
  argument of \Macro{documentclass}. In this case the result may be the
  following.
%
\begin{XmpTopPage}
        \XmpHeading{10,25}{70}
        \thinlines\XmpRule{10,23}{70}
        \thicklines\XmpRule{10,28}{70}
        \XmpSetText[\XmpTopText]{10,21}
        \XmpMarginNote{83,11.8}
\end{XmpTopPage}

   To specify that this line is to be drawn also, e.\,g., in red color, you 
   would change the commands like this:
\begin{lstcode}
  \setheadtopline{2pt}[\color{red}]
  \setheadsepline{.4pt}[\color{red}]
\end{lstcode}
  In this example, as well as in the following one, line color is
  activated by applying the syntax of the
  \Package{color}\IndexPackage{color} package, so this package must of
  course be loaded. Since \Package{scrpage2} comes without built-in
  color handling, any package providing color support may be used.

  {\KOMAScript} classes also support the following way of color specification:
\begin{lstcode}
  \setheadtopline{2pt}
  \setheadsepline{.4pt}
  \setkomafont{headtopline}[\color{red}]
  \setkomafont{headsepline}[\color{red}]
\end{lstcode}

  The automatic adjustment to the header and footer width is illustrated
  in the following example:
\begin{lstcode}
  \setfootbotline{2pt}
  \setfootsepline[text]{.4pt}
  \setfootwidth[0pt]{textwithmarginpar}
\end{lstcode}

% \phantomsection for hyperref-\pageref-links jum-2001/11/24
  \phantomsection\label{page:scrpage-en.autoLineLength}
  \begin{XmpBotPage}
    \XmpHeading{10,18}{85}
    \thinlines\XmpRule{17,21}{70}
    \thicklines\XmpRule{10,16}{85}
    \XmpSetText[\XmpBotText]{10,33}
    \XmpMarginNote{83,24}
  \end{XmpBotPage}
\end{Example}
%
Now not everyone will like the alignment of the line above the footer;
instead, one would expect the line to be left-aligned. This can only
be achieved with a global package option, which will be described
together with other package options in the next
\autoref{sec:scrpage-en.basics.options}.%
%
\EndIndexGroup
\EndIndexGroup


\subsection{Package Options}
\label{sec:scrpage-en.basics.options}
% head(in|ex)clude foot(in|ex)clude --> typearea
% headtopline headsepline footbotline footsepline (plain...)
% komastyle standardstyle
% markuppercase markusecase
% automark manualmark

In opposite to the \KOMAScript{} classes, where the most options may be
changed using \Macro{KOMAoptions} or \Macro{KOMAoption} also after loading the
class, package \Package{scrpage2} does not provide this feature\iffree{
  currently}{}. All options to \Package{scrpage2} have to be global options,
i.\,e. be part of the optional argument of \Macro{documentclass}, or
package option, i.\,e. be part of the optional argument of \Macro{usepackage}.

\begin{Declaration}
  \Option{headinclude}%
  \Option{headexclude}%
  \Option{footinclude}%
  \Option{footexclude}
\end{Declaration}%
Since\textnote{Attention!} \KOMAScript~3\ChangedAt{v2.3}{\Package{scrpage2}}
this options should not be passed to \Package{scrpage2} any longer using
\Macro{PassOptionsToPackage} or the optional argument of
\Macro{usepackage}. Only for compatibility reason \Package{scrpage2} still
declares them and pass them as \Option{headinclude},
\OptionValue{headinclude}{false}, \Option{footinclude}, and
\OptionValue{footinclude}{false} to package \Package{typearea}.
\EndIndexGroup


\begin{Declaration}
  \Option{headtopline}%
  \Option{plainheadtopline}%
  \Option{headsepline}%
  \Option{plainheadsepline}%
  \Option{footsepline}%
  \Option{plainfootsepline}%
  \Option{footbotline}%
  \Option{plainfootbotline}
\end{Declaration}%
Basic adjustment of the lines under and over header and footer can be
made with these options.  These adjustments are then considered the
default for all page styles defined with \Package{scrpage2}.  If one
of these options is used, then a line thickness 0.4\,pt is set.
Since there is a corresponding plain page style to the
page style \PageStyle{scrheadings}, the corresponding line in the plain
style can also be configured with the \Option{plain\dots} options.
These \Option{plain} options do however work only if the corresponding
options without \Option{plain} are activated.  Thus,
\Option{plainheadtopline} shows no effect without the
\Option{headtopline} option set.

With these options, it is to be noted that the appropriate page part,
header or footer, is considered as a part of the text area for the
calculation of the type area in case a line has been activated.  This
means that, if the separation line between header and text is
activated with \Option{headsepline}, then the package
\Package{typearea} calculates the type area in such a way that the
page header is part of the text block automatically.

The\textnote{Attention!} conditions for the options of the preceding paragraph
apply also to this automation. That means that the package \Package{typearea}
must be loaded after \Package{scrpage2}, or that on use of a {\KOMAScript}
class, the options \DescRef{scrpage-en.option.headinclude} and \DescRef{scrpage-en.option.footinclude} must be set
explicitly with \Macro{documentclass} in order to transfer header or footer
line in the text area.%
%
\EndIndexGroup


\begin{Declaration}
\Option{ilines}%
\Option{clines}%
\Option{olines}
\end{Declaration}%
\Index{rule>alignment}\Index{line>alignment}%
With the definition of the line lengths the case can arise where the
lengths are set correctly, but the justification is not as desired
because the line will be centered in the header or footer area.  With
the package options presented here, this specification can be modified
for all page styles defined with \Package{scrpage2}.  The option
\Option{ilines} sets the justification in such a way that the lines
align to the inside edge.  The option \Option{clines} behaves like the
default justification, and \Option{olines} aligns at the outside edge.

\begin{Example}
  The next example illustrates the influence of the
  option \Option{ilines}. Please compare to the example for
  \DescRef{scrpage-en.cmd.setfootsepline} on \autopageref{page:scrpage-en.autoLineLength}.
\begin{lstcode}
  \usepackage[ilines]{scrpage2}
  \setfootbotline{2pt}  
  \setfootsepline[text]{.4pt}
  \setfootwidth[0pt]{textwithmarginpar}
\end{lstcode}
  The mere use of the option \Option{ilines} leads to the different
  result shown below:
\begin{XmpBotPage}
        \XmpHeading{10,18}{85}
        \thinlines\XmpRule{10,21}{70}
        \thicklines\XmpRule{10,16}{85}
        \XmpSetText[\XmpBotText]{10,33}
        \XmpMarginNote{83,24}
\end{XmpBotPage}
  In contrast to the default configuration, the separation line between
  text and footer is now left-aligned, not centered.%
\end{Example}%
\EndIndexGroup


\begin{Declaration}
  \Option{automark}%
  \Option{manualmark}
\end{Declaration}%
\BeginIndex{}{headings>automatic}%
\BeginIndex{}{headings>manual}%
These options set at the beginning of the document whether to use running
headings or manual ones.  The option \Option{automark} switches the automatic
updating on, \Option{manualmark} deactivates it.  Without the use of one of
the two options, the setting which was valid when the package was loaded is
preserved.
%
\begin{Example}
  You load the package \Package{scrpage2} directly after the document class
  \Class{scrreprt} without any package options:
\begin{lstcode}
  \documentclass{scrreprt}
  \usepackage{scrpage2}
\end{lstcode}
  Since the default page style of \Class{scrreprt} is \PageStyle{plain},
  this page style is also now still active.  Futhermore, \PageStyle{plain}
  means manual headings.  If one now activates the page style
  \PageStyle{scrheadings} with
\begin{lstcode}
  \pagestyle{scrheadings}
\end{lstcode}
  then the manual headings are nevertheless still active.

  If you instead use the document class \Class{scrbook}, then after
\begin{lstcode}
  \documentclass{scrbook}
  \usepackage{scrpage2}
\end{lstcode}
  the page style \PageStyle{headings} is active and the running headings are
  updated automatically.  Switching to the page style
  \PageStyle{scrheadings} keeps this setting active.  The marking commands
  of \Class{scrbook} continue to be used.

  However, the use of
\begin{lstcode}
  \usepackage[automark]{scrpage2}
\end{lstcode}
  activates running headings independently of the used document class.
  The option does not of course affect the used page style \PageStyle{plain}
  of the class \Class{scrreprt}. The headings are not visible until the
  page style is changed to
  \PageStyle{scrheadings}\IndexPagestyle{scrheadings}, \PageStyle{useheadings}
  or another user-defined page style with headings.%
\end{Example}%
\EndIndexGroup


\begin{Declaration}
  \Option{autooneside}
\end{Declaration}%
\BeginIndex{}{headings>manual}%
\BeginIndex{}{headings>automatic}%
This option ensures that the optional parameter of
\DescRef{scrpage-en.cmd.automark}\IndexCmd{automark} will be ignored automatically in
one-sided mode. See also the explanation of the command
\DescRef{scrpage-en.cmd.automark} in \autoref{sec:scrpage-en.basics.mark},
\DescPageRef{scrpage-en.cmd.automark}.%
%
\EndIndexGroup


\begin{Declaration}
  \Option{komastyle}%
  \Option{standardstyle}
\end{Declaration}%
These options determine the look of the predefined page styles
\PageStyle{scrheadings} and \PageStyle{scrplain}. The option
\Option{komastyle} configures a look like that of the {\KOMAScript} classes.
This is the default for {\KOMAScript} classes and can in this way also be set
for other classes.

The option \Option{standardstyle} configures a page style as it is
expected by the standard classes. Furthermore, the option
\DescRef{scrpage-en.option.markuppercase} will be activated automatically, but only if
option \DescRef{scrpage-en.option.markusedcase} is not given.%
\EndIndexGroup


\begin{Declaration}
  \Option{markuppercase}%
  \Option{markusedcase}
\end{Declaration}%
In order to achieve the functionality of \DescRef{scrpage-en.cmd.automark}, the package
\Package{scrpage2} modifies internal commands which are used by the
document structuring commands to set the running headings.  Since some
classes, in contrast to the {\KOMAScript} classes, write the headings
in uppercase letters, \Package{scrpage2} has to know how the used
document class sets the headings.

Option \Option{markuppercase} shows \Package{scrpage2} that the document class
uses uppercase letters.  If the document class does not set the headings in
uppercase letters, then the option \Option{markusedcase} should be given.
These\textnote{Attention!} options are not suitable to force a representation;
thus, unexpected effects may occur if the given option does not match the
actual behaviour of the document class.%
\EndIndexGroup


\begin{Declaration}
  \Option{nouppercase}
\end{Declaration}%
In the previous paragraph dealing with \DescRef{scrpage-en.option.markuppercase} and
\DescRef{scrpage-en.option.markusedcase}, it has been already stated that some document classes
set the running headings\Index{heading} in uppercase letters using the
commands
\Macro{MakeUppercase}\IndexCmd{MakeUppercase}\important{\Macro{MakeUppercase}}
or \Macro{uppercase}\IndexCmd{uppercase}\important{\Macro{uppercase}}.
Setting the option \Option{nouppercase} allows disabling both these commands
in the headers and footers.  However, this is valid only for page styles
defined by \Package{scrpage2}, including \PageStyle{scrheadings} and its
corresponding plain page style.

The applied method is very brutal and can cause that desired changes
of normal letters to uppercase letters \Index{uppercase letters} do
not occur.  Since these cases do not occur frequently, the option
\Option{nouppercase} usually affords a useful solution.
\begin{Example}
  Your document uses the standard class \Class{book}\IndexClass{book},
  but you do not want the uppercase headings but mixed case
  headings. Then the preamble of your document could start with:
\begin{lstcode}
  \documentclass{book}
  \usepackage[nouppercase]{scrpage2}
  \pagestyle{scrheadings}
\end{lstcode}
  The selection of the page style \PageStyle{scrheadings} is necessary,
  since otherwise the page style \PageStyle{headings} is active, which
  does not respect the settings made by option \Option{nouppercase}.
\end{Example}

In some cases not only classes but also packages set the running
headings in uppercase letters.
Also in these cases the option \Option{nouppercase} should be able
to switch back to the normal mixed case headings.%
%
\EndIndexGroup


\section{Defining Own Page Styles}\label{sec:scrpage-en.UI}
 
\subsection{The Interface for Beginners}\label{sec:scrpage-en.UI.user}
% \deftripstyle

Now one would not like to remain bound to only the provided page
styles, but may wish to define one's own page styles.  Sometimes there
will be a special need, since a specific \emph{Corporate Identity} may
require the declaration of its own page styles.  The easiest way to
deal with this is:
\begin{Declaration}
  \Macro{deftripstyle}\Parameter{name}%
  \OParameter{LO}\OParameter{LI}%
  \Parameter{HI}\Parameter{HC}\Parameter{HO}%
  \Parameter{FI}\Parameter{FC}\Parameter{FO}
\end{Declaration}%
The individual parameters have the following meanings:
\begin{labeling}[\,--]{\PName{Name}}
\item[\PName{name}] the name of the page style, in order to activate it
        using the command \Macro{pagestyle}\Parameter{name}
\item[\PName{LO}] the thickness of the outside lines,
        i.\,e., the line above the header
        and the line below the footer (optional)
\item[\PName{LI}] the thickness of the separation lines,
        i.\,e., the line below the header
        and the line above the foot (optional)
\item[\PName{HI}] contents of the inside box in the page header for two-sided
        layout or left for one-sided layout
\item[\PName{HC}] contents of the centered box in the page header
\item[\PName{HO}] contents of the outside box in the page header for two-sided
        layout or right for one-sided layout
\item[\PName{FI}] contents of the inside box in the page footer for two-sided
        layout or left for one-sided layout
\item[\PName{FC}] contents of the centered box in the page footer
\item[\PName{FO}] contents of the outside box in the page footer for two-sided
        layout or right for one-sided layout
\end{labeling}

The command \Macro{deftripstyle} definitely represents the simplest
possibility of defining page styles.  Unfortunately, there are also
restrictions connected with this, since in a page range using a page
style defined via deftripstyle, no modification of the lines above and
below header and footer can take place.

\begin{Example}
  Assume a two-sided layout, where the running headings are placed on
  the inside.  Furthermore, the document title, here ``Report'', shall
  be placed outside in the header, the page number shall be centered
  in the footer.
\begin{lstcode}
  \deftripstyle{TheReport}%
                {\headmark}{}{Report}%
                {}{\pagemark}{}
\end{lstcode}

  If moreover the lines above the header and below the footer shall be
  drawn with a thickness of 2\,pt, and the text body be separated from
  header and footer with 0.4\,pt lines, then the definition has to be
  extended:
\begin{lstcode}
  \deftripstyle{TheReport}[2pt][.4pt]%
                {\headmark}{}{Report}%
                {}{\pagemark}{}
\end{lstcode}
  See \autoref{fig:scrpage2.tomuchlines} for the result.
%
\begin{figure}
  \typeout{^^J--- Ignore underfull and overfull \string\hbox:}
  \setcapindent{0pt}%
%  \begin{center}
  \begin{captionbeside}
    [{%
      Example of a user defined, line dominated page style%
    }]{%
      example of a user defined, line dominated page style
      with a static and a running heading at the page header and a page number
      centered at the page footer%
      \label{fig:scrpage2.tomuchlines}%
    }
    [l]
    \iffree{\setlength{\unitlength}{1.15mm}}{\setlength{\unitlength}{1mm}}%
    \begin{picture}(85,51)\scriptsize
      \thinlines
      \put(0,0){\line(0,1){51}}
      \put(45,0){\line(0,1){51}}
      \put(0,51){\line(1,0){40}}
      \put(45,51){\line(1,0){40}}
      % 
      \thicklines
      \put(40,0){\line(0,1){51}}
      \put(85,0){\line(0,1){51}}
      \put(0,0,){\line(1,0){40}}
      \put(45,0){\line(1,0){40}}
      % 
      \XmpHeading[Report\hfill 2. The Eye]{6,47}{30}
      \XmpHeading[2.1 Retina\hfill Report]{49,47}{30}
      \XmpHeading[\hfill 14\hfill]{6,6.5}{30}
      \XmpHeading[\hfill 15\hfill]{49,6.5}{30}
      \put(6,44){\makebox(0,0)[tl]{\parbox{30\unitlength}{\tiny%
            \textbf{2.1 Retina}\\
            \XmpText[49]}}}
      \put(49,44){\makebox(0,0)[tl]{\parbox{30\unitlength}{\tiny%
            \XmpText[51]}}}
      % 
      \thinlines
      \XmpRule{6,45.5}{30}\XmpRule{49,45.5}{30}
      \XmpRule{6,8}{30}\XmpRule{49,8}{30}
      \linethickness{1pt}
      \XmpRule{6,49}{30}\XmpRule{49,49}{30}
      \XmpRule{6,5}{30}\XmpRule{49,5}{30}
    \end{picture}
%  \end{center}
  \end{captionbeside}
  \typeout{^^J--- Don't ignore underfull and overfull \string\hbox:^^J}
\end{figure}
\end{Example}
\EndIndexGroup


\subsection{The Interface for Experts}\label{sec:scrpage-en.UI.expert}
% \defpagestyle \newpagestyle \providepagestyle \renewpagestyle
Simple page styles, as they can be defined with \DescRef{scrpage-en.cmd.deftripstyle},
are fairly rare according to experience.  Either a professor requires
that the thesis looks like his or her own\,---\,and who seriously wants
to argue against such a wish?\,---\,or a company would like that half the
financial accounting emerges in the page footer.  No problem, the
solution is:
%
\begin{Declaration}
  \Macro{defpagestyle}\Parameter{name}\Parameter{header definition}\Parameter{footer definition}%
  \Macro{newpagestyle}\Parameter{name}\Parameter{header definition}\Parameter{footer definition}%
  \Macro{renewpagestyle}\Parameter{name}\Parameter{header definition}\Parameter{footer definition}%
  \Macro{providepagestyle}\Parameter{name}\Parameter{header definition}\Parameter{footer definition}
\end{Declaration}%
These four commands give full access to the capabilities of
\Package{scrpage2} to define page styles.  Their structure is
indentical, they differ only in the manner of working.
\begin{labeling}[\ --]{\Macro{providepagestyle}}
\item[\Macro{defpagestyle}] defines a new page style.
If a page style with this name already exists it will be overwritten.
\item[\Macro{newpagestyle}] defines a new page style.
If a page style with this name already exists a error message will be given.
\item[\Macro{renewpagestyle}] redefines a page style.
If a page style with this name does not exist a error message will be given.
\item[\Macro{providepagestyle}] defines a new page style only if there is no page style with that name already present.
\end{labeling}

Using \Macro{defpagestyle} as an example, the syntax of the four
commands is explained below.
\begin{labeling}[~--]{\PName{head definition}}
\item[\PName{name}] the name of the page style for
  \Macro{pagestyle}\Parameter{name}
\item[\PName{header definition}] the declaration of the header, consisting
  of five element; elements in round parenthesis are optional:\hfill\\
  \hspace*{1em}\AParameter{ALL,ALT}%
  \Parameter{EP}\Parameter{OP}\Parameter{OS}\AParameter{BLL,BLT}
\item[\PName{footer definition}]  the declaration of the footer, consisting
  of five element; elements in round parenthesis are optional:\hfill\\  
  \hspace*{1em}\AParameter{ALL,ALT}%
  \Parameter{EP}\Parameter{OP}\Parameter{OS}\AParameter{BLL,BLT}
\end{labeling}

As can be seen, header and footer declaration have identical
structure.  The individual parameters have the following meanings:
\begin{labeling}[\ --]{\PName{OLD}}
\item[\PName{ALL}] above line length: (header = outside, footer = separation
  line)
\item[\PName{ALT}] above line thickness
\item[\PName{EP}]  definition for \emph{even} pages
\item[\PName{OP}]  definition for \emph{odd} pages
\item[\PName{OS}]  definition for \emph{one-sided} layout
\item[\PName{BLL}] below line length: (header = separation line, footer =
  outside)
\item[\PName{BLT}] below line thickness
\end{labeling}

If the optional line-parameters are omitted, then the line behaviour remains
configurable by the commands introduced in
\autoref{sec:scrpage-en.basics.format},
\DescPageRef{scrpage-en.cmd.setheadtopline}.

The three elements \PName{EP}, \PName{OP} and \PName{OS} are boxes with the
width of page header or footer, as appropriate.  The corresponding definitions
are set left-justified in the boxes. To set something left- \emph{and}
right-justified into the boxes, the space between two text elements can be
stretched using \Macro{hfill}:
%
\begin{lstcode}[belowskip=\dp\strutbox]
  {\headmark\hfill\pagemark}
\end{lstcode}

If one would like a third text-element centered in the box, then an
extended definition must be used. The commands \Macro{rlap} and
\Macro{llap} simply write the given arguments, but for {\LaTeX} they
take up no horizontal space. Only in this way is the middle text
really centered.
%
\begin{lstcode}
  {\rlap{\headmark}\hfill centered text\hfill\llap{\pagemark}}
\end{lstcode}

\iffalse% Umbruchkorrekturtext
This and the use of the expert interface in connection with other
commands provided by \Package{scpage2} follows now in the final
example.
\fi

\begin{Example}
  This examples uses the document class \Class{scrbook}, which means
  that the default page layout is two-sided.  The package
  \Package{scrpage2} is loaded with options \DescRef{scrpage-en.option.automark} and
  \DescRef{scrpage-en.option.headsepline}.  The first switches on the automatic update of
  running headings, the second determines that a separation line
  between header and text body is drawn in the \PageStyle{scrheadings}
  page style.

\begin{lstcode}
  \documentclass{scrbook}
  \usepackage[automark,headsepline]{scrpage2}
\end{lstcode}

  The expert interface is used to define two page styles.  The page
  style \PValue{withoutLines} does not define any line parameters.  The
  second page style \PValue{withLines} defines a line thicknes of 1\,pt
  for the line above the header and 0\,pt for the separation line
  between header and text.

\begin{lstcode}
  \defpagestyle{withoutLines}{%
    {Example\hfill\headmark}{\headmark\hfill without lines}
    {\rlap{Example}\hfill\headmark\hfill%
     \llap{without lines}}
  }{%
    {\pagemark\hfill}
    {\hfill\pagemark}
    {\hfill\pagemark\hfill}
  }
\end{lstcode}

\begin{lstcode}
  \defpagestyle{withLines}{%
    (\textwidth,1pt)
    {with lines\hfill\headmark}{\headmark\hfill with lines}
    {\rlap{\KOMAScript}\hfill \headmark\hfill%
     \llap{with lines}}
    (0pt,0pt)
  }{%
    (\textwidth,.4pt)
    {\pagemark\hfill}
    {\hfill\pagemark}
    {\hfill\pagemark\hfill}
    (\textwidth,1pt)
  }
\end{lstcode}

  Right at the beginning of the document the page style
  \PageStyle{scrheadings} is activated.  The command \Macro{chapter} starts
  a new chapter and automatically sets the page rstyle for this page to
  \PageStyle{plain}. Even though not a prime example, the command
  \DescRef{scrpage-en.cmd.chead} shows how running headings can be created even on a
  plain page.  However, in principle running headings on
  chapter start-pages are to be avoided, since otherwise the special
  character of the \PageStyle{plain} page style is lost. It is more
  important to indicate that a new chapter starts here than that a
  section of this page has a special title.

\begin{lstcode}
  \begin{document}
  \pagestyle{scrheadings}
  \chapter{Thermodynamics}
  \chead[\leftmark]{}
  \section{Main Laws}
  Every system has an extensive state quantity called
  Energy. In a closed system the energy is constant.
\end{lstcode}

  \begin{XmpTopPage}
    \XmpHeading[\hfill\textsl{1. Thermodynamics}\hfill]{10,27}{70}
    \put(10,17){\normalsize\textbf{\sffamily 1.Thermodynamics}}
    \put(10,12){\textbf{\sffamily 1.1 Main Laws}}
    \XmpSetText[%
    Every System has an extensive state quantity]{10,9}
  \end{XmpTopPage}

  After starting a new page the page style \PageStyle{scrheadings} is
  active and thus the separation line below the header is visible.
\begin{lstcode}
  There is a state quatity of a system, called entropy, whose temporal
  change consists of entropy flow and entropy generation.
\end{lstcode}
  \begin{XmpTopPage}
    \XmpHeading[\textsl{1. Thermodynamics}\hfill]{20,27}{70}
    \thinlines\XmpRule{20,25}{70}
    \XmpSetText[%
    There is a condition unit of a system, called entropy,
    whose temporal change consists of entropy flow
    and entropy generation.]{20,20}
  \end{XmpTopPage}

  After switching to the next page, the automatic update of the running
  headings is disabled using \DescRef{scrpage-en.cmd.manualmark}, and the page style
  \PValue{withoutLines} becomes active.  Since no line parameters are
  given in the definition of this page style, the default configuration
  is used, which draws a separation line between header and text body
  because \Package{scrpage2} was called with \DescRef{scrpage-en.option.headsepline}.
\begin{lstcode}
  \manualmark
  \pagestyle{withoutLines}
  \section{Exergy and Anergy}\markright{Energy Conversion}
  During the transition of a system to an equilibrium state
  with its environment, the maximum work gainable is called
  exergy.
\end{lstcode}
  \begin{XmpTopPage}
    \XmpHeading[\slshape Energy Conversion\hfill without lines]{10,27}{70}
    \thinlines\XmpRule{10,25}{70}
    \XmpSetText[\setlength{\parfillskip}{0pt plus 1fil}%
    \textbf{\sffamily 1.2 Exergy and Anergy}\\
    During the transition of a system to an equilibrium state
    with its environment, the maximum work gainable is called
    exergy.]{10,21}
  \end{XmpTopPage}

  At the next page of the document, the page style \PValue{withLines} is
  activated.  The line settings of its definition are taken in account
  and the lines are drawn accordingly.
\begin{lstcode}
  \pagestyle{mitLinien}
  \renewcommand{\headfont}{\itshape\bfseries}
  The portion of an energy not convertible in exergy
  is named anergy \Var{B}.
  \[ B = U + T (S_1 - S_u) - p (V_1 - V_u)\] 
  \end{document}
\end{lstcode}
  \begin{XmpTopPage}
    \XmpHeading[\itshape\bfseries with lines\hfill 1. Thermodynamics]{20,27}{70}
    \thicklines\XmpRule{20,29}{70}
    \XmpSetText[\setlength{\parfillskip}{0pt plus 1fil}%
    The portion of an energy not convertible in exergy
    is named anergy $B$. 
    \[ B = U + T (S_1 - S_u) - p (V_1 - V_u)\] ]{20,20}
  \end{XmpTopPage}
\end{Example}
\EndIndexGroup


\subsection{Managing Page Styles}\label{sec:scrpage-en.UI.cfgFile}
% scrpage.cfg
\BeginIndex{File}{scrpage.cfg}
Before long the work with different page styles will establish a
common set of employed page styles, depending on taste and tasks.  In
order to make the management of page styles easier and avoid
time-consuming copy operations each time a new project is started,
\Package{scrpage2} reads the file \File{scrpage.cfg} after
initialisation.  This file can contain a set of user-defined page
styles which many projects can share.
\EndIndex{File}{scrpage.cfg}
\EndIndexGroup


% ============================================================================

\cleardoubleoddpage

\selectlanguage{ngerman}
\let\savednewcommand\newcommand
\let\newcommand\renewcommand
\input{\jobname-ngerman.tex}
\let\newcommand\savednewcommand
\def\theHchapter{D.\thechapter}
\setcounter{chapter}{0}
\renewcommand*{\thepage}{D.\arabic{page}}

\phantomsection
\addparttocentry{}{Deutsch}

\title{Das obsolete Paket \Package{scrpage2}}
\subtitle{Auszug aus früheren Versionen der \KOMAScript-Anleitung}
\author{Markus Kohm\and Jens-Uwe-Morawski}
\date{2014-06-25}
\maketitle


\chapter{Kopf- und Fußzeilen mit \Package{scrpage2}}
\labelbase{scrpage-de}
%
\BeginIndexGroup
\BeginIndex{Package}{scrpage2}%
\BeginIndex{}{Seitenstil}
Mit \KOMAScript~3.12 wurde das komplett neu implementierte Paket
\Package{scrlayer-scrpage} vorgestellt, das eine konsequente Weiterführung des
Designs von \Package{scrpage2} mit anderen Mitteln darstellt. Siehe dazu das
entsprechende Kapitel der \KOMAScript-Anleitung. Während
\Package{scrpage2} die erweiterte Optionen-Schnittstelle der
\KOMAScript-Klassen nicht unterstützt, kommt diese in
\Package{scrlayer-scrpage} selbstverständlich zum Einsatz. Da der Ansatz von
\Package{scrlayer-scrpage} gegenüber \Package{scrpage2} als überlegen
betrachtet wird, wird somit empfohlen, \Package{scrlayer-scrpage} an Stelle
von \Package{scrpage2} zu verwenden. Die aktuelle Version von
\Package{scrpage2} ist daher auch als final zu betrachten. Sämtliche
Entwicklungsresourcen im Bereich Kopf- und Fußzeilen werden zukünftig in
\Package{scrlayer-scrpage} einfließen.

\begin{Explain}
  An Stelle von \Package{scrpage2} oder \Package{scrlayer-scrpage} kann
  natürlich auch \Package{fancyhdr} verwendet
  werden. \Package{scrpage2} und insbesondere \Package{scrlayer-scrpage}
  harmonieren jedoch mit den \KOMAScript-Klassen deutlich besser. Genau
  deshalb und weil der Vorläufer von \Package{fancyhdr} damals viele
  Möglichkeiten vermissen lies, ist \Package{scrpage2} entstanden. Natürlich
  ist das Paket \Package{scrpage2} ebenso wie das Paket
  \Package{scrlayer-scrpage} nicht an eine \KOMAScript-Klasse gebunden,
  sondern kann auch sehr gut mit anderen Klassen verwendet werden.
\end{Explain}

\section{Grundlegende Funktionen}\label{sec:scrpage-de.basics}

\begin{Explain}
  Um die nachfolgende Beschreibung zu verstehen, muss noch einiges zu \LaTeX{}
  gesagt werden. Im \LaTeX-Kern sind die Standardseitenstile
  \PageStyle{empty}, welcher eine völlig undekorierte Seite erzeugt, und
  \PageStyle{plain}, welcher meist nur die Seitenzahl enthält, definiert. In
  vielen Klassen ist der Stil \PageStyle{headings} zu finden, welcher eine
  komplexe Seitendekoration mit automatischen
  Kolumnentitel\Index{Kolumnentitel>automatisch} erzeugt. Die Variante
  \PageStyle{myheadings} gleicht \PageStyle{headings}. Die Kolumnentitel
  müssen dabei aber manuell\Index{Kolumnentitel>manuell} gesetzt werden.
  Ausführlicher wird das im Abschnitt über Seitenstile der
  \KOMAScript-Anleitung beschrieben. Dort wird auch erläutert, dass auf
  einigen Seiten der Seitenstil automatisch -- in der Regel zu
  \PageStyle{plain} -- wechselt.
\end{Explain}
  
Das Paket \Package{scrpage2} unterscheidet nicht mehr zwischen Seitenstilen
mit automatischem und mit manuellem Kolumnentitel. Die Wahl des Seitenstils
erfolgt unabhängig davon, ob mit automatischem oder manuellem Kolumnentitel
gearbeitet wird. Näheres dazu finden Sie in \autoref{sec:scrpage-de.basics.mark}.


\subsection{Vordefinierte Seitenstile}\label{sec:scrpage-de.basics.buildIn}

Zu den grundlegenden Funktionen von \Package{scrpage2} gehören unter anderem
vordefinierte, konfigurierbare Seitenstile.

% \lehead ...
% scrheadings
% \headmark \pagemark 

\begin{Declaration}
  \PageStyle{scrheadings}%
  \PageStyle{scrplain}
\end{Declaration}%
Das Paket \Package{scrpage2} liefert für Seiten mit Kolumnentitel einen
eigenen Seitenstil namens \PageStyle{scrheadings}.  Dieser Seitenstil kann
mittels \Macro{pagestyle}\PParameter{scrheadings} aktiviert werden.  Wird
dieser Seitenstil benutzt, dann wird gleichzeitig der plain-Stil durch den
dazu passenden Stil \PageStyle{scrplain} ersetzt. Passend bedeutet, dass auch
der plain-Stil auf in \autoref{sec:scrpage-de.basics.format} vorgestellte
Befehle, die beispielsweise die Kopfbreite ändern, reagiert und im Grundlayout
übereinstimmt. Die Aktivierung des Seitenstils \PageStyle{scrheadings} oder
des zugehörigen plain-Stils, \PageStyle{scrplain}, hat keine Auswirkung
darauf, ob mit manuellen oder automatischen
Kolumnentiteln\Index{Kolumnentitel>automatisch}\Index{Kolumnentitel>manuell}
gearbeitet wird (siehe \autoref{sec:scrpage-de.basics.mark}). Der Seitenstil
\PageStyle{scrplain} kann auch direkt per \Macro{pagestyle} aktiviert werden.


\begin{Declaration}
  \Macro{lehead}%
    \OParameter{scrplain-links-gerade}\Parameter{scrheadings-links-gerade}%
  \Macro{cehead}%
    \OParameter{scrplain-mittig-gerade}\Parameter{scrheadings-mittig-gerade}%
  \Macro{rehead}%
    \OParameter{scrplain-rechts-gerade}\Parameter{scrheadings-rechts-gerade}%
  \Macro{lefoot}%
    \OParameter{scrplain-links-gerade}\Parameter{scrheadings-links-gerade}%
  \Macro{cefoot}%
    \OParameter{scrplain-mittig-gerade}\Parameter{scrheadings-mittig-gerade}%
  \Macro{refoot}%
    \OParameter{scrplain-rechts-gerade}\Parameter{scrheadings-rechts-gerade}%
  \Macro{lohead}%
    \OParameter{scrplain-links-ungerade}\Parameter{scrheadings-links-ungerade}%
  \Macro{cohead}%
    \OParameter{scrplain-mittig-ungerade}\Parameter{scrheadings-mittig-ungerade}%
  \Macro{rohead}%
    \OParameter{scrplain-rechts-ungerade}\Parameter{scrheadings-rechts-ungerade}%
  \Macro{lofoot}%
    \OParameter{scrplain-links-ungerade}\Parameter{scrheadings-links-ungerade}%
  \Macro{cofoot}%
    \OParameter{scrplain-mittig-ungerade}\Parameter{scrheadings-mittig-ungerade}%
  \Macro{rofoot}%
    \OParameter{scrplain-rechts-ungerade}\Parameter{scrheadings-rechts-ungerade}%
  \Macro{ihead}%
    \OParameter{scrplain-innen}\Parameter{scrheadings-innen}%
  \Macro{chead}%
    \OParameter{scrplain-zentriert}\Parameter{scrheadings-zentriert}%
  \Macro{ohead}%
    \OParameter{scrplain-außen}\Parameter{scrheadings-außen}%
  \Macro{ifoot}%
    \OParameter{scrplain-innen}\Parameter{scrheadings-innen}%
  \Macro{cfoot}%
    \OParameter{scrplain-zentriert}\Parameter{scrheadings-zentriert}%
  \Macro{ofoot}%
    \OParameter{scrplain-außen}\Parameter{scrheadings-außen}
\end{Declaration}%
Die Seitenstile von \Package{scrpage2} sind so definiert, dass ihr Kopf und
Fuß flexibel angepasst werden kann. Hierzu sind sowohl im Kopf als auch im Fuß
drei Felder vorhanden, deren Inhalt modifiziert werden kann.  Die Befehle
zur Modifikation sind in \autoref{fig:scrpage-de.leheadetall} verdeutlicht. Die
in der Mitte dargestellten Befehle modifizieren sowohl die Felder der linken
als auch der rechten Seite. Alle Befehle haben sowohl ein optionales als auch
ein obligatorisches Argument. Das optionale Argument bestimmt jeweils das
durch den Befehl festgelegte Feld im plain-Seitenstil,
\PageStyle{scrplain}. Das obligatorische Argument definiert das entsprechende
Feld im Seitenstil \PageStyle{scrheadings}.

\begin{figure}
%  \centering
  \setcapindent{0pt}%
  \begin{captionbeside}
    [Befehle zur Manipulation der \Package{scrpage2}-Seitenstile]%
    {\hspace{0pt plus 1ex}%
      Zuord\-nung der Befehle zur Manipulation der beiden Seiten\-stile
      \PageStyle{scrheadings} und \PageStyle{scrplain} zu den Feldern im
      Kopf und Fuß der Seiten}
    [l]
  \setlength{\unitlength}{.95mm}\begin{picture}(100,65)(0,8)\small
    \put(0,12){\dashbox{2}(40,56){linke Seite}}
    \put(60,12){\dashbox{2}(40,56){rechte Seite}}
                                % Felder: oben links
    \put(1,63){\framebox(8,4){~}}
    \put(16,63){\framebox(8,4){~}}
    \put(31,63){\framebox(8,4){~}}
                                % Felder: oben rechts
    \put(61,63){\framebox(8,4){~}}
    \put(76,63){\framebox(8,4){~}}
    \put(91,63){\framebox(8,4){~}}
                                % Felder: unten links
    \put(1,13){\framebox(8,4){~}}
    \put(16,13){\framebox(8,4){~}}
    \put(31,13){\framebox(8,4){~}}
                                % Felder: unten rechts
    \put(61,13){\framebox(8,4){~}}
    \put(76,13){\framebox(8,4){~}}
    \put(91,13){\framebox(8,4){~}}
                                % die Befehle und Pfeile
                                % für beidseitig
                                % 
    \put(50,65){\makebox(0,0){\DescRef{scrpage-de.cmd.ihead}}}
    \put(44,65){\vector(-1,0){4}}\put(56,65){\vector(1,0){4}}
    \put(50,58){\makebox(0,0){\DescRef{scrpage-de.cmd.chead}}}
    \put(44,58){\line(-1,0){24}}\put(56,58){\line(1,0){24}}
    \put(20,58){\vector(0,1){4}}\put(80,58){\vector(0,1){4}}
    \put(50,51){\makebox(0,0){\DescRef{scrpage-de.cmd.ohead}}}
    \put(44,51){\line(-1,0){40}}\put(56,51){\line(1,0){40}}
    \put(4,51){\vector(0,1){11}}\put(96,51){\vector(0,1){11}}
                                %
    \put(50,15){\makebox(0,0){\DescRef{scrpage-de.cmd.ifoot}}}
    \put(56,15){\vector(1,0){4}}\put(44,15){\vector(-1,0){4}}
    \put(50,22){\makebox(0,0){\DescRef{scrpage-de.cmd.cfoot}}}
    \put(44,22){\line(-1,0){24}}\put(56,22){\line(1,0){24}}
    \put(20,22){\vector(0,-1){4}}\put(80,22){\vector(0,-1){4}}
    \put(50,29){\makebox(0,0){\DescRef{scrpage-de.cmd.ofoot}}}
    \put(44,29){\line(-1,0){40}}\put(56,29){\line(1,0){40}}
    \put(4,29){\vector(0,-1){11}}\put(96,29){\vector(0,-1){11}}
                                % Befehle für einseitig
    \put(5,69){\makebox(0,0)[b]{\DescRef{scrpage-de.cmd.lehead}}}
    \put(20,69){\makebox(0,0)[b]{\DescRef{scrpage-de.cmd.cehead}}}
    \put(35,69){\makebox(0,0)[b]{\DescRef{scrpage-de.cmd.rehead}}}
                                %
    \put(65,69){\makebox(0,0)[b]{\DescRef{scrpage-de.cmd.lohead}}}
    \put(80,69){\makebox(0,0)[b]{\DescRef{scrpage-de.cmd.cohead}}}
    \put(95,69){\makebox(0,0)[b]{\DescRef{scrpage-de.cmd.rohead}}}
                                %
    \put(5,11){\makebox(0,0)[t]{\DescRef{scrpage-de.cmd.lefoot}}}
    \put(20,11){\makebox(0,0)[t]{\DescRef{scrpage-de.cmd.cefoot}}}
    \put(35,11){\makebox(0,0)[t]{\DescRef{scrpage-de.cmd.refoot}}}
                                %
    \put(65,11){\makebox(0,0)[t]{\DescRef{scrpage-de.cmd.lofoot}}}
    \put(80,11){\makebox(0,0)[t]{\DescRef{scrpage-de.cmd.cofoot}}}
    \put(95,11){\makebox(0,0)[t]{\DescRef{scrpage-de.cmd.rofoot}}}
  \end{picture}
  \end{captionbeside}
%  \caption[Befehle zur Manipulation der \Package{scrpage2}-Seitenstile]%
%  {Zuordnung der Befehle zur Manipulation der Seitenstile
%    \PageStyle{scrheadings} und \PageStyle{scrplain} zu den manipulierten
%    Seitenelementen}
  \label{fig:scrpage-de.leheadetall}
\end{figure}

\begin{Example}
  Angenommen, man möchte bei \PageStyle{scrheadings} zentriert im Seitenfuß
  die Seitenzahl dargestellt haben, dann benutzt man einfach:
\begin{lstcode}
  \cfoot{\pagemark}
\end{lstcode}
  Sollen die Seitenzahlen\Index{Paginierung} im Kopf außen und die
  Kolumnentitel\Index{Kolumnentitel} innen stehen, dann erfolgt dies mit:
\begin{lstcode}
  \ohead{\pagemark}
  \ihead{\headmark}
  \cfoot{}
\end{lstcode}
  Das \Macro{cfoot}\PParameter{} ist nur notwendig, um eine möglicherweise in
  der Mitte des Fußes vorhandene Seitenzahl zu entfernen.
\end{Example}

Die Befehle, die direkt nur einem Feld zugeordnet sind, können für
anspruchsvollere Vorhaben genutzt werden.

\begin{Example}
  Angenommen, man hat den Auf"|trag, einen Jahresbericht einer Firma zu
  erstellen, dann könnte das so angegangen werden:
\begin{lstcode}
  \ohead{\pagemark}
  \rehead{Jahresbericht 2001}
  \lohead{\headmark}
  \cefoot{Firma WasWeißIch}
  \cofoot{Abteilung Entwicklung}
\end{lstcode}
  Natürlich muss man hier dafür sorgen, dass mittels \Macro{cofoot} der Fuß
  der rechten Seite aktualisiert wird, wenn eine neue Abteilung im Bericht
  besprochen wird.
\end{Example}

Wie oben dargestellt, gibt es einen zu \PageStyle{scrheadings}
korrespondierenden plain-Seitenstil.  Da es auch möglich sein soll, diesen
Stil anzupassen, unterstüt"-zen die Befehle ein optionales Argument.  Damit
kann der Inhalt des entsprechenden Feldes im plain-Seitenstil modifiziert
werden.

\begin{Example}
  Um für die Nutzung von \PageStyle{scrheadings} die Position der Seitenzahlen
  festzulegen, kann man folgendes benutzen:
\begin{lstcode}
  \cfoot[\pagemark]{}
  \ohead[]{\pagemark}
\end{lstcode}
  Wird anschließend der Stil \PageStyle{plain} genutzt, beispielsweise weil
  \Macro{chapter} eine neue Seite beginnt und darauf umschaltet, dann steht
  die Seitenzahl zentriert im Seitenfuß.
\end{Example}
%
\EndIndexGroup


% Die folgende Deklaration und Erklaerung wurde eingefuegt von
% mjk 2001-08-18
\begin{Declaration}
  \Macro{clearscrheadings}%
  \Macro{clearscrplain}%
  \Macro{clearscrheadfoot}
\end{Declaration}%
Will man sowohl den Seitenstil \PageStyle{scrheadings} als auch den dazu
gehörenden plain-Seitenstil von Grund auf neu definieren,
muss man häufig zusätzlich einige der bereits belegten Seitenelemente
löschen.  Da man selten alle Elemente mit neuem Inhalt füllt, sind
dazu in den meisten Fällen mehrere Befehle mit leeren Parametern
notwendig.
Mit Hilfe dieser drei Befehle ist das Löschen schnell und
gründlich möglich. Während \Macro{clearscrheadings} lediglich alle
Felder des Seitenstils \PageStyle{scrheadings} und \Macro{clearscrplain}
alle Felder des zugehörigen plain-Seitenstils löscht, werden
von \Macro{clearscrheadfoot} alle Felder beider Seitenstile auf leere
Inhalte gesetzt.
\begin{Example}
  Sie wollen unabhängig davon, wie die Seitenstile derzeit aussehen,
  die Standardform der \KOMAScript-Klassen bei zweiseitigem Satz
  erreichen. Dies ist mit nur drei Befehlen möglich:
\begin{lstcode}
  \clearscrheadfoot
  \ohead{\headmark}
  \ofoot[\pagemark]{\pagemark}
\end{lstcode}
  Ohne die Befehle \Macro{clearscrheadings}, \Macro{clearscrplain} und
  \Macro{clearscrheadfoot} wären doppelt so viele Anweisungen und neun weitere
  leere Argumente notwendig:
\begin{lstcode}
  \ihead[]{}
  \chead[]{}
  \ohead[]{\headmark}
  \ifoot[]{}
  \cfoot[]{}
  \ofoot[\pagemark]{\pagemark}
\end{lstcode}
  Einige davon könnten natürlich entfallen, wenn man von einer
  konkreten Vorbelegung ausginge.
\end{Example}
%
\EndIndexGroup
%
\EndIndexGroup


% Satz leicht geaendert: mjk 2001-08-18
In den vorausgehenden Beispielen wurden schon zwei Befehle benutzt, die noch
gar nicht besprochen wurden.  Das soll jetzt nachgeholt werden.

% Die folgende Deklaration und Erklaerung wurde eingefuegt von
% mjk 2001-08-17
\begin{Declaration}
  \Macro{leftmark}%
  \Macro{rightmark}
\end{Declaration}%
Diese beiden Befehle erlauben es, auf die Kolumnentitel zuzugreifen, die
normalerweise für die linke bzw. die rechte Seite gedacht sind. Diese beiden
Befehle werden nicht von \Package{scrpage2}, sondern direkt vom \LaTeX-Kern
zur Verfügung gestellt. Wenn in diesem Kapitel vom Kolumnentitel der linken
Seite oder vom Kolumnentitel der rechten Seite die Rede ist, dann ist damit
eigentlich der Inhalt von \Macro{leftmark} und \Macro{rightmark} gemeint.
%
\EndIndexGroup


\begin{Declaration}
\Macro{headmark}
\end{Declaration}%
Dieser Befehl ermöglicht es, auf die Inhalte der Kolumnentitel zuzugreifen.
Im Gegensatz zu den originalen \LaTeX{}-Befehlen \Macro{leftmark} und
\Macro{rightmark} braucht man nicht auf die richtige Zuordnung zur linken oder
rechten Seite zu achten.%
%
\EndIndexGroup


\begin{Declaration}
  \Macro{pagemark}
\end{Declaration}%
Dieser Befehl ermöglicht den Zugriff auf die Seitenzahl. Im
\autoref{sec:scrpage-de.basics.format}, \DescPageRef{scrpage-de.cmd.pnumfont}
wird der Befehl \DescRef{scrpage-de.cmd.pnumfont} zur Formatierung der Seitenzahl vorgestellt,
den \Macro{pagemark} automatisch berücksichtigt.%
%
\EndIndexGroup


\begin{Declaration}
  \PageStyle{useheadings}
\end{Declaration}%
Das Paket \Package{scrpage2} ist in erster Linie dafür gedacht, dass die
bereitgestellten Stile benutzt oder eigene Stile definiert werden.  Jedoch
kann es notwendig sein, auch auf einen von der Dokumentklasse zur Verfügung
gestellten Stil zurückzuschalten. Es wäre nahe liegend, dieses mit
\Macro{pagestyle}\PParameter{headings} vorzunehmen. Das hätte aber den
Nachteil, dass die nachfolgend besprochenen Befehle \DescRef{scrpage-de.cmd.automark} und
\DescRef{scrpage-de.cmd.manualmark} nicht wie erwartet funktionierten. Daher sollte mit
\Macro{pagestyle}\PParameter{useheadings}\IndexCmd{pagestyle} auf die
originalen Stile umgeschaltet werden. Eine solche Umschaltung hat dann keine
Auswirkung darauf, ob mit manuellen oder automatischen Kolumnentiteln
gearbeitet wird.%
%
\EndIndexGroup


\subsection{Manuelle und automatische Kolumnentitel}
\label{sec:scrpage-de.basics.mark}
% \automark \manualmark 
%
\BeginIndexGroup
\BeginIndex{}{Kolumnentitel>automatisch}%
\BeginIndex{}{Kolumnentitel>manuell}%
Gewöhnlich\textnote{Achtung!} gibt es zu einem \PageStyle{headings}-Stil eine
\emph{my}-Variante. Ist ein solcher Stil aktiv, dann werden die Kolumnentitel
nicht mehr automatisch aktualisiert.  Bei \Package{scrpage2} wird ein anderer
Weg beschritten.  Ob die Kolumnentitel lebend sind oder nicht, bestimmen die
Befehle \DescRef{scrpage-de.cmd.automark} und \DescRef{scrpage-de.cmd.manualmark}. Die Voreinstellung kann auch
bereits beim Laden des Paketes über die Optionen \DescRef{scrpage-de.option.automark} und
\DescRef{scrpage-de.option.manualmark} beeinflusst werden (siehe
\autoref{sec:scrpage-de.basics.options},
\DescPageRef{scrpage-de.option.automark}).


\begin{Declaration}
\Macro{manualmark}
\end{Declaration}%
Wie der Name bereits verdeutlicht, schaltet \Macro{manualmark} die
Aktualisierung der Kolumnentitel aus.  Es bleibt somit dem Nutzer überlassen,
für eine Aktualisierung bzw. für den Inhalt der Kolumnentitel zu sorgen.  Dazu
stehen die Befehle \Macro{markboth}\IndexCmd{markboth} und
\Macro{markright}\IndexCmd{markright} aus dem \LaTeX-Kern bereit. Diese
Anweisungen sind im Abschnitt über Seitenstile in der \KOMAScript-Anleitung
erklärt.%
%
\EndIndexGroup


% Ergaenzt mjk 2001-07-17
\begin{Declaration}
  \Macro{automark}\OParameter{rechte Seite}\Parameter{linke Seite}
\end{Declaration}%
Die Anweisung \Macro{automark} aktiviert die automatische Aktualisierung des
Kolumnentitels.  Für die beiden Parameter sind die Bezeichnungen der
Gliederungsebenen\Index{Gliederungsebenen} einzusetzen, deren Titel an
entsprechender Stelle erscheinen soll. Gültige Werte für die Parameter sind:
\PValue{part}\ChangedAt{v2.2}{\Package{scrpage2}}, \PValue{chapter},
\PValue{section}, \PValue{subsection}, \PValue{subsubsection},
\PValue{paragraph} und \PValue{subparagraph}. Der Wert \PValue{part} führt bei
Verwendung der meisten Klassen nicht zu dem gewünschten Ergebnis. Bisher ist
nur von den \KOMAScript-Klassen ab Version~2.9s bekannt, dass dieser Wert
unterstützt wird.  Das optionale Argument \PName{rechte Seite} ist
ver\-ständlicherweise nur für zweiseitigen Satz gedacht. Im einseitigen Satz
sollten Sie normalerweise darauf verzichten. Mit Hilfe der Option
\DescRef{scrpage-de.option.autooneside}\IndexOption{autooneside}\important{\DescRef{scrpage-de.option.autooneside}}
können Sie auch einstellen, dass das optionale Argument im
einseitigen\Index{einseitig} Satz automatisch ignoriert wird (siehe
\autoref{sec:scrpage-de.basics.options},
\DescPageRef{scrpage-de.option.autooneside}).
\begin{Example}
  Wird beispielsweise mit einer \emph{book}-Klasse gearbeitet, deren höch\-ste
  Gliederungsebene \emph{chapter} ist, dann stellt nach einem vorhergehenden
  \DescRef{scrpage-de.cmd.manualmark} der Befehl
\begin{lstcode}
  \automark[section]{chapter}
\end{lstcode}
  den Originalzustand wieder her. Bevorzugt man stattdessen, die tieferen
  Gliederungsebenen angezeigt zu bekommen, dann erfolgt dies mit:
\begin{lstcode}
  \automark[subsection]{section}
\end{lstcode}
\end{Example}

\begin{Explain}
  Die Markierung der jeweils höheren Gliederungsebene wird mit Hilfe von
  \Macro{markboth}\IndexCmd{markboth} gesetzt. Die Markierung der
  tieferen Gliederungsebene wird mit \Macro{markright}\IndexCmd{markright}
  bzw.  \Macro{markleft}\IndexCmd{markleft} gesetzt.  Der entsprechende Aufruf
  erfolgt indirekt über die Gliederungsbefehle. Die Anweisung \Macro{markleft}
  wird von \Package{scrpage2} bereitgestellt und ist vergleichbar zu
  \Macro{markright} aus dem \LaTeX-Kern
  definiert. Obwohl sie nicht als internes Makro definiert ist, wird von einem
  direkten Gebrauch abgeraten.
\end{Explain}
\EndIndexGroup
\EndIndexGroup


\subsection{Formatierung der Kopf- und Fußzeilen}
\label{sec:scrpage-de.basics.format}
% \headfont \pnumfont
% \setheadwidth \setfootwidth
% \set(head|foot)(top|sep|bot)line
Im vorherigen Abschnitt ging es hauptsächlich um inhaltliche Dinge.  Das
genügt natürlich nicht, um die gestalterischen Ambitionen zu
befriedigen. Deshalb soll es sich in diesem Abschnitt ausschließlich darum
drehen.

\begin{Declaration}
  \Macro{headfont}%
  \Macro{footfont}%
  \Macro{pnumfont}
\end{Declaration}%
Die Schriftformatierung für den Seitenkopf und -fuß übernimmt der Befehl
\Macro{headfont}, \Macro{footfont} die Abweichung davon für den Fuß und
\Macro{pnumfont} wiederum die Abweichung davon für die Seitenzahl.
\begin{Example}
  Um beispielsweise den Kopf in fetter, serifenloser Schrift und den Fuß in
  nicht fetter, serifenloser Schrift zu setzen und die Seitenzahl geneigt
  mit Serifen erscheinen zu lassen, nutzt man folgende Definitionen:
\begin{lstcode}
  \renewcommand*{\headfont}{%
    \normalfont\sffamily\bfseries}
  \renewcommand*{\footfont}{%
    \normalfont\sffamily}
  \renewcommand*{\pnumfont}{%
    \normalfont\rmfamily\slshape}
\end{lstcode}
\end{Example}

\BeginIndexGroup
\BeginIndex{FontElement}{pagehead}%
\BeginIndex{FontElement}{pagefoot}%
\BeginIndex{FontElement}{pagenumber}%
Ab Version 2.8p der \KOMAScript{}-Klassen wurde die Schnittstelle für
Schriftattribute vereinheitlicht.  Wird \Package{scrpage2} in Verbindung mit
einer dieser Klassen verwendet, dann sollte die Zuweisung in der Art erfolgen,
wie sie in der \KOMAScript-Anleitung beschrieben wird.
\begin{Example}
  Statt \Macro{renewcommand} wird bei Verwendung einer \KOMAScript-Klasse
  vorzugsweise der Befehl \Macro{setkomafont} verwendet. Die vorhergehenden
  Definitionen lauten damit:
\begin{lstcode}
  \setkomafont{pagehead}{%
    \normalfont\sffamily\bfseries}
  \setkomafont{pagefoot}{%
    \normalfont\sffamily}
  \setkomafont{pagenumber}{%
    \normalfont\rmfamily\slshape}
\end{lstcode}
\end{Example}
\EndIndexGroup
\EndIndexGroup


\begin{Declaration}
\Macro{setheadwidth}\OParameter{Verschiebung}\Parameter{Breite}%
\Macro{setfootwidth}\OParameter{Verschiebung}\Parameter{Breite}
\end{Declaration}%
Normalerweise entsprechen die Breiten von Kopf- und Fußzeile der Breite des
Textbereichs.\Index{Kopf>Breite}\Index{Fuss=Fuß>Breite} Die beiden Befehle
\Macro{setheadwidth} und \Macro{setfootwidth} ermöglichen dem Anwender, auf
einfache Weise die Breiten seinen Bedürfnissen anzupassen.  Das obligatorische
Argument \PName{Breite} nimmt den Wert der Breite des Kopfes bzw. des Fußes
auf, \PName{Verschiebung} ist ein Längenmaß für die Verschiebung des
entsprechenden Elements in Richtung des äußeren Seitenrandes.

Für die möglichen Standardfälle akzeptiert das obligatorische Argument
\PName{Breite} auch folgende symbolische Werte:
\begin{labeling}[\ --]{\PValue{textwithmarginpar}}
\item[\PValue{paper}] die Breite des Papiers
\item[\PValue{page}] die Breite der Seite
\item[\PValue{text}] die Breite des Textbereichs
\item[\PValue{textwithmarginpar}] die Breite des Textbereichs inklusive
Seitenrand
\item[\PValue{head}] die aktuelle Breite des Seitenkopfes
\item[\PValue{foot}] die aktuelle Breite des Seitenfußes
\end{labeling}
Der Unterschied zwischen \PValue{paper} und \PValue{page} besteht darin, dass
\PValue{page} die Breite des Papiers abzüglich der Bindekorrektur ist, falls
das \Package{typearea}-Paket\IndexPackage{typearea} verwendet wird (siehe das
Kapitel über \Package{typearea} in der \KOMAScript-Anleitung). Ohne Verwendung
von \Package{typearea} sind \PValue{paper} und \PValue{page}
identisch.%mjk-2001-08-17

\begin{Example}
  Angenommen, man möchte ein Seitenlayout wie im
  \emph{\LaTeX-Begleiter}, bei dem die Kopfzeile in den Rand ragt,
  dann geschieht das ganz einfach mit:
\begin{lstcode}
  \setheadwidth[0pt]{textwithmarginpar}
\end{lstcode}
%
  und sieht dann auf einer rechten Seite folgendermaßen aus:
%
  \begin{XmpTopPage}
    \XmpHeading{10,29}{85}
    \thinlines\XmpRule{10,27}{85}
    \XmpSetText[\XmpTopText]{10,24}
    \XmpMarginNote{83,14}
  \end{XmpTopPage}
%
  Soll der Seitenfuß die gleiche Breite und Ausrichtung haben, dann hat man
  jetzt zwei Wege.  Der erste ist, man wiederholt das Gleiche für den
  Seitenfuß mit:
\begin{lstcode}
  \setfootwidth[0pt]{textwithmarginpar}
\end{lstcode}
%
  oder man greift auf den anderen symbolischen Wert \PValue{head} zurück, da
  der Kopf bereits die gewünschte Breite hat.
\begin{lstcode}
  \setfootwidth[0pt]{head}
\end{lstcode}
\end{Example}

Wird keine Verschiebung angegeben, das heißt auf das optionale Argument
verzichtet, dann erscheint der Kopf bzw. der Fuß symmetrisch auf der Seite
angeordnet. Es wird somit ein Wert für die Verschiebung automatisch ermittelt,
der der aktuellen Seitengestalt entspricht.

\begin{Example}
  Entsprechend dem vorherigen Beispiel wird hier auf das optionale Argument
  verzichtet:
\begin{lstcode}
  \setheadwidth{textwithmarginpar}
\end{lstcode}
%
  und sieht dann auf einer rechten Seite folgendermaßen aus:
%
  \begin{XmpTopPage}
        \XmpHeading{5,29}{85}
        \thinlines\XmpRule{5,27}{85}
        \XmpSetText[\XmpTopText]{10,24}
        \XmpMarginNote{83,14}
  \end{XmpTopPage}
\end{Example}

Wie zu sehen, ist der Kopf jetzt nach innen verschoben, wobei die Kopfbreite
sich nicht geändert hat. Die Verschiebung ist so berechnet, dass die
Seitenproportionen auch hier sichtbar werden.
\EndIndexGroup


\begin{Declaration}
  \Macro{setheadtopline}\OParameter{Länge}\Parameter{Dicke}%
                        \OParameter{Anweisungen}%
  \Macro{setheadsepline}\OParameter{Länge}\Parameter{Dicke}%
                        \OParameter{Anweisungen}%
  \Macro{setfootsepline}\OParameter{Länge}\Parameter{Dicke}%
                        \OParameter{Anweisungen}%
  \Macro{setfootbotline}\OParameter{Länge}\Parameter{Dicke}%
                        \OParameter{Anweisungen}
\end{Declaration}%
Entsprechend den Größenparametern für die Kopf- und Fußzeile gibt es auch
Befehle, die die Dimensionen der Linien im Kopf und Fuß modifizieren
können. Dazu sollten diese Linien aber zunächst erst einmal eingeschaltet
werden. Siehe hierzu die Optionen \DescRef{scrpage-de.option.headtopline}, \DescRef{scrpage-de.option.headsepline},
\DescRef{scrpage-de.option.footsepline}, \DescRef{scrpage-de.option.footbotline} in
\autoref{sec:scrpage-de.basics.options},
\DescPageRef{scrpage-de.option.headsepline}.

\begin{labeling}[\ --]{\Macro{setfootsepline}}
\item[\Macro{setheadtopline}] modifiziert die Parameter für die Linie
über dem Seitenkopf
\item[\Macro{setheadsepline}] modifiziert die Parameter für die Linie
zwischen Kopf und Textkörper
\item[\Macro{setfootsepline}] modifiziert die Parameter für die Linie
zwischen Text und Fuß
\item[\Macro{setfootbotline}] modifiziert die Parameter für die Linie
unter dem Seitenfuß
\end{labeling}

Das obligatorische Argument \PName{Dicke} bestimmt, wie stark die Linie
gezeichnet wird.  Das optionale Argument \PName{Länge} akzeptiert die gleichen
symbolischen Werte wie \PName{Breite} bei \DescRef{scrpage-de.cmd.setheadwidth}, also auch einen
normalen Längenausdruck.  Solange im Dokument dem optionalen Argument
\PName{Länge} kein Wert zugewiesen wurde, passt sich die entsprechende
Linienlänge automatisch der Breite des Kopfes bzw. des Fußes an.

Möchte man diesen Automatismus für die Länge einer Linie wieder restaurieren,
dann nutzt man im Längenargument den Wert \PValue{auto}.

Mit\ChangedAt{v2.2}{\Package{scrpage2}} dem optionalen Argument
\PName{Anweisungen} können zusätzliche Anweisungen definiert werden, die vor
dem Zeichnen der jeweiligen Linie auszuführen sind.  Das können beispielsweise
Anweisungen sein, um die
Farbe\Index{Kopf>Farbe}\Index{Fuss=Fuß>Farbe}\Index{Farbe>im
  Kopf}\Index{Farbe>im Fuss=im Fuß} der Linie zu ändern.  Bei Verwendung einer
\KOMAScript-Klasse können diese Anweisungen auch über
\Macro{setkomafont}\IndexCmd{setkomafont} für eines der Elemente
\FontElement{headtopline}\IndexFontElement[indexmain]{headtopline}%
\important[i]{\FontElement{headtopline}\\
  \FontElement{headsepline}\\
  \FontElement{footsepline}\\
  \FontElement{footbottomline}},
\FontElement{headsepline}\IndexFontElement[indexmain]{headsepline},
\FontElement{footsepline}\IndexFontElement[indexmain]{footsepline},
\FontElement{footbottomline}\IndexFontElement[indexmain]{footbottomline} oder
auch \FontElement{footbotline}\IndexFontElement[indexmain]{footbotline}
gesetzt und mit \Macro{addtokomafont}\IndexCmd{addtokomafont} erweitert
werden. Die beiden Anweisungen \Macro{setkomafont} und \Macro{addtokomafont}
sind in der \KOMAScript-Anleitung näher beschrieben.


\begin{Declaration}
  \labelsuffix[auto+current]%
  \Macro{setheadtopline}\POParameter{auto}
                        \PParameter{current}%
  \labelsuffix[auto]
  \Macro{setheadtopline}\POParameter{auto}\Parameter{}%
  \labelsuffix[auto+]
  \Macro{setheadtopline}\POParameter{auto}\Parameter{}\OParameter{}
\end{Declaration}%
Die hier am Befehl \DescRef{scrpage-de.cmd.setheadtopline} illustrierten Argumente
sind natürlich auch für die anderen drei Längenbefehle gültig.

Enthält das obligatorische Argument den Wert \PValue{current} oder wird
leer gelassen, dann wird die Dicke der Linie nicht verändert.  Das kann
genutzt werden, wenn die Länge der Linie, aber nicht die Dicke modifiziert
werden soll.

Wird das optionale Argument \PName{Anweisungen} weggelassen, so bleiben
eventuell zuvor gesetzte Anweisungen erhalten. Wird hingegen ein leeres
Argument \PName{Anweisungen} gesetzt, so werden eventuell zuvor gesetzte
Anweisungen wieder gelöscht.

\begin{Example}
  Soll beispielsweise der Kopf mit einer kräftigen Linie von 2\,pt darüber und
  einer normalen von 0,4\,pt zwischen Kopf und Text abgesetzt werden, dann
  erfolgt das mit:
\begin{lstcode}
  \setheadtopline{2pt}
  \setheadsepline{.4pt}
\end{lstcode}
Zusätzlich\textnote{Achtung!} sind unbedingt die Optionen
\DescRef{scrpage-de.option.headtopline} und \DescRef{scrpage-de.option.headsepline} vorzugsweise global im
optionalen Argument von \Macro{documentclass} zu setzen. Das Ergebnis könnte
dann wie folgt aussehen.
%
  \begin{XmpTopPage}
        \XmpHeading{10,29}{70}
        \thinlines\XmpRule{10,27}{70}
        \thicklines\XmpRule{10,32}{70}
        \XmpSetText[\XmpTopText]{10,24}
        \XmpMarginNote{83,14}
  \end{XmpTopPage}

  Sollen diese Linien zusätzlich in roter Farbe gesetzt werden, dann sind die
  Anweisungen beispielsweise wie folgt zu ändern:
\begin{lstcode}
  \setheadtopline{2pt}[\color{red}]
  \setheadsepline{.4pt}[\color{red}]
\end{lstcode}
  In diesem und auch dem folgenden Beispiel wurde für die Aktivierung der
  Farbe die Syntax des \Package{color}\IndexPackage{color}-Pakets verwendet,
  das dann natürlich auch geladen werden muss. \Package{scrpage2} selbst
  bietet keine direkte Farbunterstützung. Damit ist jedes beliebige
  Farbunterstützungspaket verwendbar.

  Mit einer \KOMAScript-Klasse kann alternativ
\begin{lstcode}
  \setheadtopline{2pt}
  \setheadsepline{.4pt}
  \setkomafont{headtopline}{\color{red}}
  \setkomafont{headsepline}{\color{red}}
\end{lstcode}
  verwendet werden.

  Die automatische Anpassung an die Kopf- und Fußbreiten illustriert folgendes
  Beispiel, für das die Optionen \DescRef{scrpage-de.option.footbotline} und \DescRef{scrpage-de.option.footsepline}
  gesetzt sein sollten:
\begin{lstcode}
  \setfootbotline{2pt}
  \setfootsepline[text]{.4pt}
  \setfootwidth[0pt]{textwithmarginpar}
\end{lstcode}

%\phantomsection für hyperref-\pageref-Link jum-2001/11/24
  \phantomsection\label{page:scrpage-de.autoLineLength}%
  \begin{XmpBotPage}
        \XmpHeading{10,18}{85}
        \thinlines\XmpRule{17,21}{70}
        \thicklines\XmpRule{10,16}{85}
        \XmpSetText[\XmpBotText]{10,36}
        \XmpMarginNote{83,26}
  \end{XmpBotPage}
\end{Example}
%
Nun mag nicht jedem die Ausrichtung der Linie über der Fußzeile gefallen,
sondern es wird in einem solchen Fall erwartet, dass sie wie der Kolumnentitel
linksbündig zum Text ist.  Diese Einstellung kann nur global in Form einer
Paketoption erfolgen und wird im folgenden
\autoref{sec:scrpage-de.basics.options} mit anderen Optionen beschrieben.%
%
\EndIndexGroup
\EndIndexGroup


\subsection{Optionen beim Laden des Paketes}
\label{sec:scrpage-de.basics.options}

Während bei den \KOMAScript-Klassen die Mehrzahl der Optionen auch noch nach
dem Laden der Klasse mit \Macro{KOMAoptions} und \Macro{KOMAoption} geändert
werden kann, trifft dies für das Paket \Package{scrpage2} \iffree{derzeit noch
}{}nicht zu. Alle Optionen für dieses Paket müssen als globale Optionen, also
im optionalen Argument von \Macro{documentclass}, oder als Paketoptionen, also
im optionalen Argument von \Macro{usepackage}, angegeben werden.

% head(in|ex)clude foot(in|ex)clude --> typearea
% headtopline headsepline footbotline footsepline (plain...)
% komastyle standardstyle
% markuppercase markusecase
% automark manualmark
\begin{Declaration}
  \Option{headinclude}%
  \Option{headexclude}%
  \Option{footinclude}%
  \Option{footexclude}
\end{Declaration}%
Diese\textnote{Achtung!} Optionen sollten bei Verwendung von
\KOMAScript~3\ChangedAt{v2.3}{\Package{scrpage2}} nicht mehr beispielsweise
per optionalem Argument von \Macro{usepackage} oder per
\Macro{PassOptionsToPackage} direkt an \Package{scrpage2} übergeben
werden. Lediglich aus Gründen der Kompatibilität sind sie noch in
\Package{scrpage2} deklariert und werden von diesem als \Option{headinclude},
\OptionValue{headinclude}{false}, \Option{footinclude} und
\OptionValue{footinclude}{false} an das Paket \Package{typearea}
weitergereicht.
\EndIndexGroup


\begin{Declaration}
  \Option{headtopline}%
  \Option{plainheadtopline}%
  \Option{headsepline}%
  \Option{plainheadsepline}%
  \Option{footsepline}%
  \Option{plainfootsepline}%
  \Option{footbotline}%
  \Option{plainfootbotline}
\end{Declaration}%
Eine Grundeinstellung für die Linien unter und über den Kopf- und Fußzeilen
kann mit diesen Optionen vorgenommen werden.  Diese Einstellungen gelten dann
als Standard für alle mit \Package{scrpage2} definierten Seitenstile.  Wird
eine von diesen Optionen verwendet, dann wird eine Linienstärke von 0,4\,pt
eingesetzt.  Da es zum Seitenstil \PageStyle{scrheadings} einen entsprechenden
plain-Stil gibt, kann mit den \Option{plain\dots}-Optionen auch die
entsprechende Linie des plain-Stils konfiguriert werden. Diese
\Option{plain}-Optionen wirken aber nur, wenn auch die korrespondierende
Option ohne \Option{plain} aktiviert wurde.  Somit zeigt die Option
\Option{plainheadtopline} ohne \Option{headtopline} keine Wirkung.

Bei diesen Optionen ist zu beachten, dass der entsprechende Seitenteil in den
Textbereich des Satzspiegels mit übernommen wird, wenn eine Linie aktiviert
wurde.  Wird also mittels \Option{headsepline} die Trennlinie zwischen Kopf
und Text aktiviert, dann wird automatisch mittels \Package{typearea} der
Satzspiegel so berechnet, dass der Seitenkopf Teil des Textblocks ist.

Die\textnote{Achtung!} Bedingungen für die Optionen des vorhergehenden
Abschnitts gelten auch für diesen Automatismus.  Das bedeutet, dass das Paket
\Package{typearea} nach \Package{scrpage2} geladen werden muss,
beziehungsweise, dass bei Verwendung einer \KOMAScript{}-Klasse die Optionen
\DescRef{scrpage-de.option.headinclude} und \DescRef{scrpage-de.option.footinclude} explizit bei
\Macro{documentclass} gesetzt werden müssen, um Kopf- bzw. Fußzeile in den
Textblock zu übernehmen.%
%
\EndIndexGroup


\begin{Declaration}
  \Option{ilines}%
  \Option{clines}%
  \Option{olines}
\end{Declaration}%
\Index{Linienausrichtung}%
Bei der Festlegung der Linienlängen kann es vorkommen, dass die Linie zwar die
gewünschte Länge, aber nicht die erwünschte Ausrichtung hat, da sie im Kopf-
bzw. Fußbereich zentriert wird. Mit den hier vorgestellten Paketoptionen kann
global für alle mit \Package{scrpage2} definierten Seitenstile diese Vorgabe
modifiziert werden.  Dabei setzt \Option{ilines} die Ausrichtung so, dass die
Linien an den inneren Rand verschoben werden. Die Option \Option{clines}
verhält sich wie die Standardeinstellung und \Option{olines} richtet am
äußeren Rand aus.

\begin{Example}
  Hier gilt es, das Beispiel zu \DescRef{scrpage-de.cmd.setfootsepline} auf
  \autopageref{page:scrpage-de.autoLineLength} mit dem folgenden zu
  vergleichen, um die Wirkung der Option \Option{ilines} zu sehen.
\begin{lstcode}
  \usepackage[ilines,footsepline,footbotline]
             {scrpage2}
  \setfootbotline{2pt}  
  \setfootsepline[text]{.4pt}
  \setfootwidth[0pt]{textwithmarginpar}
\end{lstcode}
% Der folgende Text wurde erg"anzt um den Seitenumbruch zu verbessern,
% mjk-2001-10-05:
%  \enlargethispage*{2\baselineskip}
  Allein die Verwendung der Option \Option{ilines} führt dabei zu der
  geänderten Ausgabe, die nachfolgend veranschaulicht wird:
%
\begin{XmpBotPage}
        \XmpHeading{10,18}{85}
        \thinlines\XmpRule{10,21}{70}
        \thicklines\XmpRule{10,16}{85}
        \XmpSetText[\XmpBotText]{10,36}
        \XmpMarginNote{83,26}
\end{XmpBotPage}
  Die Trennlinie zwischen Text und Fuß wird bündig innen im Fußteil
  gesetzt und nicht wie bei der Standardeinstellung zentriert.
\end{Example}
\EndIndexGroup


% \phantomsection für hyperref-\pageref-Link jum-2001/11/24
\begin{Declaration}
  \Option{automark}%
  \Option{manualmark}
\end{Declaration}%
\BeginIndex{}{Kolumnentitel>automatisch}%
\BeginIndex{}{Kolumnentitel>manuell}%
Diese Optionen bestimmen gleich zu Beginn des Dokuments, ob eine automatische
Aktualisierung der Kolumnentitel erfolgt.  Die Option \Option{automark}
schaltet die automatische Aktualisierung ein, \Option{manualmark} deaktiviert
sie.
% Das Folgende wurde ergaenzt, weil dazu die haeufigsten Fragen
% bzw. falschen Bug-Reports im Support kamen.
% mjk-2001-09-21:
Ohne Verwendung einer der beiden Optionen bleibt die Einstellung
erhalten, die beim Laden des Paketes gültig war.
\begin{Example}
  Sie laden das Paket \Package{scrpage2} unmittelbar nach der Klasse
  \Class{scrreprt} und ohne weitere Optionen. Dazu schreiben Sie:
\begin{lstcode}
  \documentclass{scrreprt}
  \usepackage{scrpage2}
\end{lstcode}
  Da bei \Class{scrreprt} der Seitenstil \PageStyle{plain} voreingestellt
  ist, ist dies auch jetzt noch der Fall. Außerdem entspricht die
  Voreinstellung \PageStyle{plain} manuellen Kolumnentiteln. Wenn Sie
  also anschließend mit
\begin{lstcode}
  \pagestyle{scrheadings}
\end{lstcode}
  auf den Seitenstil \PageStyle{scrheadings} umschalten, sind noch immer
  manuelle Kolumnentitel eingestellt.

  Verwenden Sie stattdessen die Dokumentklasse \Class{scrbook}, so
  ist nach
\begin{lstcode}
  \documentclass{scrbook}
  \usepackage{scrpage2}
\end{lstcode}
  der Seitenstil \PageStyle{headings} mit automatischen Kolumnentiteln
  aktiviert. Bei anschließender Umschaltung auf den Seitenstil
  \PageStyle{scrheadings} bleiben automatische Kolumnentitel
  eingeschaltet. Dabei werden dann weiterhin die Markierungsmakros von
  \Class{scrbook} verwendet.

  Verwenden Sie hingegen
\begin{lstcode}
  \usepackage[automark]{scrpage2}
\end{lstcode}
  so wird unabhängig von der verwendeten Klasse auf automatische
  Kolumnentitel umgeschaltet, wobei die Markierungsmakros von
  \Package{scrpage2} genutzt werden.
  Natürlich wirkt sich dies auf den Seitenstil
  \PageStyle{plain} von \Class{scrreprt} nicht aus. Die Kolumnentitel
  werden erst sichtbar, wenn auf den Seitenstil
  \PageStyle{scrheadings}\IndexPagestyle{scrheadings} oder
  \PageStyle{useheadings} oder einen selbst definierten Seitenstil mit
  Kolumnentiteln umgeschaltet wird.
\end{Example}
\EndIndexGroup


% Folgende Deklaration und Erklaerung wurde eingefügt von
% mjk 2001-08-17:
\begin{Declaration}
  \Option{autooneside}
\end{Declaration}%
\BeginIndex{}{Kolumnentitel>automatisch}%
\BeginIndex{}{Kolumnentitel>manuell}%
Mit dieser Option wird das optionale Argument von
\DescRef{scrpage-de.cmd.automark}\IndexCmd{automark}\Index{Kolumnentitel>automatisch} im
einseitigen Satz automatisch ignoriert. Siehe hierzu auch die Erläuterung
zum Befehl \DescRef{scrpage-de.cmd.automark} in \autoref{sec:scrpage-de.basics.mark},
\DescPageRef{scrpage-de.cmd.automark}.%
%
\EndIndexGroup


\begin{Declaration}
  \Option{komastyle}%
  \Option{standardstyle}
\end{Declaration}%
Diese Optionen bestimmen, wie die beiden vordefinierten Seitenstile
\PageStyle{scrheadings} und \PageStyle{scrplain} gestaltet sind.
Bei \Option{komastyle} wird eine Definition vorgenommen, wie
sie den \KOMAScript{}-Klassen entspricht.
Bei den \KOMAScript{}-Klassen ist dies die Voreinstellung und 
kann somit auch für andere Klassen gesetzt werden.

Die Option \Option{standardstyle} definiert die beiden Seitenstile
wie es von den Standardklassen erwartet wird.
Außerdem wird hier automatisch \DescRef{scrpage-de.option.markuppercase} aktiviert,
es sei denn, \DescRef{scrpage-de.option.markusedcase} wird ebenfalls als Option übergeben.%
\EndIndexGroup


\begin{Declaration}
  \Option{markuppercase}%
  \Option{markusedcase}
\end{Declaration}%
Für die Funktionalität von \DescRef{scrpage-de.cmd.automark} modifiziert
\Package{scrpage2} interne Befehle, die die Gliederungsbefehle
benutzen, um die lebenden Kolumnentitel zu setzen.
Da einige Klassen, im Gegensatz
zu den \KOMAScript{}-Klassen, die Kolumnentitel in Großbuchstaben
schreiben, muss \Package{scrpage2} wissen, wie die genutzte
Dokumentklasse die lebenden Kolumnentitel darstellt.

Die Option \Option{markuppercase} zeigt \Package{scrpage2}, dass die benutzte
Klasse die Großschreibweise benutzt.  Die Option \Option{markusedcase} sollte
angegeben werden, wenn die benutzte Dokumentklasse keine Großschreibweise
verwendet.  Die\textnote{Achtung!} Optionen sind nicht geeignet, eine
entsprechende Darstellung zu erzwingen. Es kann somit zu unerwünschten
Effekten kommen, wenn die Angabe nicht dem Verhalten der Dokumentklasse
entspricht.%
\EndIndexGroup


% Die folgende Deklaration und Erklaerung wurde eingefuegt von
% mjk 2001-08-18
\begin{Declaration}
  \Option{nouppercase}
\end{Declaration}%
Wie in obiger Erklärung zu \DescRef{scrpage-de.option.markuppercase} und \DescRef{scrpage-de.option.markusedcase}
bereits ausgeführt wurde, gibt es Klassen und auch Pakete, die beim Setzen der
lebenden Kolumnentitel\Index{Kolumnentitel>automatisch} mit Hilfe einer der
Anweisungen \Macro{uppercase}\IndexCmd{uppercase}\important{\Macro{uppercase}}
oder
\Macro{MakeUppercase}\IndexCmd{MakeUppercase}\important{\Macro{MakeUppercase}}
den gesamten Eintrag in Großbuchstaben wandeln. Mit der Option
\Option{nouppercase} können diese beiden Anweisungen im Kopf und im Fuß außer
Kraft gesetzt werden. Das gilt aber nur für Seitenstile, die mit Hilfe von
\Package{scrpage2} definiert werden. Dazu zählen auch \PageStyle{scrheadings}
und der zugehörige plain-Seitenstil.

Die verwendete Methode ist äußerst brutal und kann dazu führen, dass
auch erwünschte Änderungen von Klein- in
Großbuchstaben\Index{Grossbuchstaben=Großbuchstaben} unterbleiben. Da
diese Fälle nicht sehr häufig sind, stellt \Option{nouppercase} aber
meist eine brauchbare Lösung dar.
\begin{Example}
  Sie verwenden die Standardklasse \Class{book}\IndexClass{book},
  wollen aber, dass die lebenden Kolumnentitel nicht in
  Großbuchstaben, sondern in normaler gemischter Schreibweise gesetzt
  werden. Die Präambel Ihres Dokuments könnte dann wie folgt beginnen:
\begin{lstcode}
  \documentclass{book}
  \usepackage[nouppercase]{scrpage2}
  \pagestyle{scrheadings}
\end{lstcode}
  Die Umschaltung auf den Seitenstil \PageStyle{scrheadings} ist
  notwendig, weil sonst der Seitenstil \PageStyle{headings} verwendet
  wird, der von der Option \Option{nouppercase} nicht behandelt wird.
\end{Example}

In einigen Fällen setzen nicht nur Klassen, sondern auch Pakete
lebende Kolumnentitel in Großbuchstaben. Auch in diesen Fällen hilft
\Option{nouppercase} meist, um zu gemischter Schreibweise
zurückzuschalten.%
%
\EndIndexGroup


\section{Seitenstile selbst gestalten}\label{sec:scrpage-de.UI}
%
% 
\subsection{Die Anwenderschnittstelle}\label{sec:scrpage-de.UI.user}
% \deftripstyle

Nun möchte man ja nicht immer an die vorgegebenen Seitenstile gebunden sein,
sondern auch seiner Kreativität freien Lauf lassen.
Manchmal ist man auch dazu gezwungen, weil eine bestimmte
\emph{Corporate Identity} einer Firma es verlangt.
Der einfachste Weg damit umzugehen ist
\begin{Declaration}
  \Macro{deftripstyle}\Parameter{Name}%
  \OParameter{LA}\OParameter{LI}%
  \Parameter{KI}\Parameter{KM}\Parameter{KA}%
  \Parameter{FI}\Parameter{FM}\Parameter{FA}
\end{Declaration}%
Die einzelnen Felder haben folgende Bedeutung:
\begin{labeling}[~--]{\PName{Name}}
\item[\PName{Name}] die Bezeichnung des Seitenstils, die dann bei der
  Aktivierung mit \Macro{pagestyle}\Parameter{Name} oder
  \Macro{thispagestyle}\Parameter{Name} verwendet wird
\item[\PName{LA}] die Dicke der äußeren Linien, d.\,h. der Linien über der
  Kopfzeile und unter der Fußzeile (optional)
\item[\PName{LI}] die Dicke der inneren Linie, d.\,h. der Linien die Kopf und
  Fuß vom Textkörper trennen (optional)
\item[\PName{KI}] Inhalt des Feldes im Kopf innenseitig oder bei einseitigem
  Layout links
\item[\PName{KM}] Inhalt des Feldes im Kopf zentriert
\item[\PName{KA}] Inhalt des Feldes im Kopf außenseitig oder bei einseitigem
  Layout rechts
\item[\PName{FI}] Inhalt des Feldes im Fuß innenseitig oder bei einseitigem
  Layout links
\item[\PName{FM}] Inhalt des Feldes im Fuß zentriert
\item[\PName{FA}] Inhalt des Feldes im Fuß außenseitig oder bei einseitigem
  Layout rechts
\end{labeling}

Der Befehl \Macro{deftripstyle} stellt sicherlich die einfachste
Möglichkeit dar, Seitenstile zu definieren.
Leider sind damit auch Einschränkungen verbunden,
da in einem Seitenbereich mit einem durch \Macro{deftripstyle} deklarierten 
Seitenstil keine Änderung der Kopf- und Fußlinien erfolgen kann.

\begin{Example}
  Vorgegeben sei ein doppelseitiges Layout, bei dem die Kolumnentitel innen
  erscheinen sollen.  Weiterhin soll der Dokumenttitel, in diesem Fall kurz
  \glqq Bericht\grqq, an den Außenrand in den Kopf, die Seitenzahl soll
  zentriert in den Fuß.
\begin{lstcode}
  \deftripstyle{DerBericht}%
                {\headmark}{}{Bericht}%
                {}{\pagemark}{}
\end{lstcode}

  Sollen weiterhin die Linien über dem Kopf und unter dem Fuß
  mit 2\,pt erscheinen und der ganze Textkörper mit dünnen Linien
  von 0,4\,pt von Kopf und Fuß abgesetzt werden, dann erweitert
  man vorherige Definition.
\begin{lstcode}
  \deftripstyle{DerBericht}[2pt][.4pt]%
                {\headmark}{}{Bericht}%
                {}{\pagemark}{}
\end{lstcode}
  Das Ergebnis ist in \autoref{fig:scrpage-de2.tomuchlines} zu sehen.
%
\begin{figure}
  \typeout{^^J--- Ignore underfull and overfull \string\hbox:}%
  \setcapindent{0pt}%
  \begin{captionbeside}
    [{%
      Beispiel eines selbst definierten, von Linien dominierten Seitenstils%
    }]{%
      Beispiel für einen selbst definierten, von Linien dominierten Seitenstil
      mit einem statischen und einem lebenden Kolumnentitel im Kopf und der
      Seitenzahl in der Mitte des Fußes.%
      \label{fig:scrpage-de2.tomuchlines}%
    }
    [l]
    \iffree{\setlength{\unitlength}{1.15mm}}{\setlength{\unitlength}{1mm}}%
    \begin{picture}(85,51)\scriptsize
      \thinlines
      \put(0,0){\line(0,1){51}}
      \put(45,0){\line(0,1){51}}
      \put(0,51){\line(1,0){40}}
      \put(45,51){\line(1,0){40}}
      % 
      \thicklines
      \put(40,0){\line(0,1){51}}
      \put(85,0){\line(0,1){51}}
      \put(0,0){\line(1,0){40}}
      \put(45,0){\line(1,0){40}}
      % 
      \XmpHeading[Bericht\hfill 2 Das Auge]{6,47}{30}
      \XmpHeading[2.1 Netzhaut\hfill Bericht]{49,47}{30}
      \XmpHeading[\hfill 14\hfill]{6,6.5}{30}
      \XmpHeading[\hfill 15\hfill]{49,6.5}{30}
      \put(6,44){\makebox(0,0)[tl]{\parbox{30\unitlength}{\tiny%
            \textbf{2.1 Netzhaut}\\
            \XmpText[49]}}}
      \put(49,44){\makebox(0,0)[tl]{\parbox{30\unitlength}{\tiny%
            \XmpText[51]}}}
      % 
      \thinlines
      \XmpRule{6,45.5}{30}\XmpRule{49,45.5}{30}
      \XmpRule{6,8}{30}\XmpRule{49,8}{30}
      \linethickness{1pt}
      \XmpRule{6,49}{30}\XmpRule{49,49}{30}
      \XmpRule{6,5}{30}\XmpRule{49,5}{30}
    \end{picture}
  \end{captionbeside}
  \typeout{^^J--- Don't ignore underfull and overfull \string\hbox:^^J}%
\end{figure}
\end{Example}
\EndIndexGroup


\subsection{Die Expertenschnittstelle}\label{sec:scrpage-de.UI.expert}
% \defpagestyle \newpagestyle \providepagestyle \renewpagestyle
Einfache Seitenstile, wie sie mit \DescRef{scrpage-de.cmd.deftripstyle} deklariert
werden können, sind erfahrungsgemäß selten.
Entweder verlangt ein Professor, dass die Diplomarbeit so aussieht
wie seine eigene -- und wer will ihm da \emph{ernsthaft} widersprechen --
oder eine Firma möchte, dass die halbe Finanzbuchhaltung im
Seitenfuß auf"|taucht. Alles kein Problem, denn es gibt noch:
\begin{Declaration}
  \Macro{defpagestyle}\Parameter{Name}\Parameter{Kopfdefinition}\Parameter{Fußdefinition}%
  \Macro{newpagestyle}\Parameter{Name}\Parameter{Kopfdefinition}\Parameter{Fußdefinition}%
  \Macro{renewpagestyle}\Parameter{Name}\Parameter{Kopfdefinition}\Parameter{Fußdefinition}%
  \Macro{providepagestyle}\Parameter{Name}\Parameter{Kopfdefinition}\Parameter{Fußdefinition}
\end{Declaration}%
Dies sind die Befehle, die die volle Kontrolle über die Gestaltung
eines Seitenstils ermöglichen. Der Aufbau ist bei allen vier
Definitionen gleich, sie unterscheiden sich nur hinsichtlich
der Wirkungsweise.
\begin{labeling}[\ --]{\Macro{providepagestyle}}
\item[\Macro{defpagestyle}] definiert einen neuen Seitenstil.  Existiert
  bereits einer mit diesem Namen, wird dieser überschrieben.
\item[\Macro{newpagestyle}] definiert einen neuen Seitenstil. Wenn schon einer
  mit diesem Namen existiert, wird ein Fehler ausgegeben.
\item[\Macro{renewpagestyle}] definiert einen bestehenden Seitenstil um. Wenn
  noch keiner mit diesem Namen existiert, wird ein Fehler ausgegeben.
\item[\Macro{providepagestyle}] definiert einen neuen Seitenstil nur dann,
  wenn dieser vorher noch nicht existiert.
\end{labeling}

Am Beispiel von \Macro{defpagestyle} soll die Syntax der Definitionen
im Folgenden erläutert werden.
\begin{labeling}[~--]{\PName{Kopfdefinition}\,}
\item[\PName{Name}] die Bezeichnung des Seitenstils
\item[\PName{Kopfdefinition}] die Deklaration des Seitenkopfes bestehend aus
  fünf Teilen, wobei die in runden Klammern stehenden Angaben optional
  sind:\hfill\\
  \hspace*{1em}\AParameter{OLL,OLD}%
  \Parameter{GS}\Parameter{US}\Parameter{ES}\AParameter{ULL,ULD}
\item[\PName{Fußdefinition}] die Deklaration des Seitenfußes bestehend aus
  fünf Teilen, wobei die in runden Klammern stehenden Angaben optional
  sind:\hfill\\
  \hspace*{1em}\AParameter{OLL,OLD}%
  \Parameter{GS}\Parameter{US}\Parameter{ES}\AParameter{ULL,ULD}
\end{labeling}

Wie zu sehen ist, haben Kopf- und Fußdefinition identischen Aufbau. Die
einzelnen Parameter haben folgende Bedeutung:
\begin{labeling}[\ --]{\PName{OLD}}
\item[\PName{OLL}] obere Linienlänge: Kopf = außen, Fuß = Trennlinie
\item[\PName{OLD}] obere Liniendicke
\item[\PName{GS}]  Definition für die \emph{gerade} Seite
\item[\PName{US}]  Definition für die \emph{ungerade} Seite
\item[\PName{ES}]  Definition für \emph{einseitiges} Layout
\item[\PName{ULL}] untere Linienlänge Kopf = Trennlinie, Fuß = außen
\item[\PName{ULD}] untere Liniendicke
\end{labeling}

Werden die optionalen Linienargumente nicht gesetzt, dann bleibt das Verhalten
weiterhin durch die in \autoref{sec:scrpage-de.basics.format},
\DescPageRef{scrpage-de.cmd.setheadtopline} vorgestellten Linienbefehle
konfigurierbar.

Die drei Felder \PName{GS}, \PName{US} und \PName{ES} entsprechen Boxen, die
die Breite des Kopf- bzw. Fußteils haben.
Die entsprechenden Definitionen erscheinen in diesen Boxen linksbündig.
Um somit etwas links- \emph{und} rechtsseitig in den Boxen zu platzieren,
kann der Zwischenraum mit \Macro{hfill} gestreckt werden:
%
\begin{lstcode}[belowskip=\dp\strutbox]
  {\headmark\hfill\pagemark}
\end{lstcode}

Um zusätzlich etwas zentriert erscheinen zu lassen, ist eine erweiterte
Definition notwendig.
Die Befehle \Macro{rlap} und \Macro{llap} setzen die übergebenen Argumente.
Für \LaTeX{} erscheint es aber so, dass diese Texte eine Breite von Null
haben. Nur so erscheint der mittlere Text auch wirklich zentriert.
%
\begin{lstcode}
  {\rlap{\headmark}\hfill zentriert\hfill\llap{\pagemark}}
\end{lstcode}

\iffalse% Umbruchkorrekturtext
Dies und die Verwendung der Expertenschnittstelle in Zusammenhang mit
anderen Befehlen von \Package{scrpage2} nun als abschließendes Beispiel.
\fi

\begin{Example}
  Angenommen es wird die Dokumentklasse \Class{scrbook} genutzt.  Damit
  liegt ein zweiseitiges Layout vor.  Für das Paket \Package{scrpage2} wird
  festgelegt, dass mit automatisch aktualisierten Kolumnentiteln gearbeitet
  wird und dass im Seitenstil \PageStyle{scrheadings} eine Trennlinie zwischen
  Kopf und Text gezogen wird.

\begin{lstcode}
  \documentclass{scrbook}
  \usepackage[automark,headsepline]{scrpage2}
\end{lstcode}

  Mit Hilfe der Expertenschnittstelle werden zwei Seitenstile definiert. Der
  erste legt keine Linienargumente fest, im zweiten wird die Linie über dem
  Kopf mit einer Dicke von 1\,pt und die Linie unter dem Kopf mit 0\,pt
  festgelegt.

\begin{lstcode}
  \defpagestyle{ohneLinien}{%
    {Beispiel\hfill\headmark}
    {\headmark\hfill ohne Linien}
    {\rlap{Beispiel}\hfill\headmark\hfill%
     \llap{ohne Linien}}
  }{%
    {\pagemark\hfill}
    {\hfill\pagemark}
    {\hfill\pagemark\hfill}
  }
  \defpagestyle{mitLinien}{%
    (\textwidth,1pt)
    {mit Linien\hfill\headmark}
    {\headmark\hfill mit Linien}
    {\rlap{\KOMAScript}\hfill \headmark\hfill%
     \llap{mit Linien}}
    (0pt,0pt)
  }{%
    (\textwidth,.4pt)
    {\pagemark\hfill}
    {\hfill\pagemark}
    {\hfill\pagemark\hfill}
    (\textwidth,1pt)
  }
\end{lstcode}

  Gleich zu Beginn wird der Seitenstil \PageStyle{scrheadings} aktiviert.  Mit
  \Macro{chapter} wird ein neues Kapitel begonnen.  Weiterhin wird automatisch
  durch \Macro{chapter} der Seitenstil für diese Seite auf \PageStyle{plain}
  gesetzt.  Das folgende \DescRef{scrpage-de.cmd.chead} zeigt, wie durch Modifikation des
  plain-Stils ein Kolumnentitel erzeugt werden kann.  Grundsätzlich sollte
  jedoch davon Abstand genommen werden, da sonst der Markierungscharakter der
  plain-Seite verloren geht. Es ist wichtiger anzuzeigen, dass hier ein neues
  Kapitel beginnt, als dass ein Abschnitt dieser Seite einen bestimmten Titel
  trägt.

\begin{lstcode}
  \begin{document}
  \pagestyle{scrheadings}
  \chapter{Thermodynamik}
  \chead[\leftmark]{}
  \section{Hauptsätze}
  Jedes System besitzt eine extensive Zustandsgröße 
  Energie. Sie ist in einem abgeschlossenen System 
  konstant.
\end{lstcode}%
  \begin{XmpTopPage}
        \XmpHeading[\hfill\textsl{1 Thermodynamik}\hfill]{10,27}{70}
        \put(10,17){\normalsize\textbf{\sffamily 1 Thermodynamik}}
        \put(10,12){\textbf{\sffamily 1.1 Hauptsätze}}
        \XmpSetText[%
        Jedes System besitzt eine extensive Zustands\char\defaulthyphenchar
        \linebreak]{10,9}
  \end{XmpTopPage}

  Nach dem Seitenwechsel ist der Seitenstil \PageStyle{scrheadings}
  aktiv, und somit auch die Trennlinie aus den Paketoptionen sichtbar.
\begin{lstcode}
  Es existiert eine Zustandsgröße, genannt die 
  Entropie eines Systems, deren zeitliche Änderung
  sich aus Entropieströmung und Entropieerzeugung
  zusammensetzt.
\end{lstcode}
  \begin{XmpTopPage}
        \XmpHeading[\textsl{1 Thermodynamik}\hfill]{20,27}{70}
        \thinlines\XmpRule{20,25}{70}
        \XmpSetText[%
        Es existiert eine Zustandsgröße, genannt die En"-tropie
     eines Systems, deren zeitliche Änderung sich aus
     Entropieströmung und Entropie\char\defaulthyphenchar\kern-1pt
     \linebreak]{20,21}
  \end{XmpTopPage}

  Wiederum nach einem Seitenwechsel wird auf manuelle Kolumnentitel gewechselt
  und der Seitenstil \PValue{ohneLinien} aktiviert.  Da keine Linienargumente
  bei der Definition dieses Stils genutzt wurden, wird die
  Standard-Linien\-konfiguration verwendet. Diese zeichnet hier eine Linie
  zwischen Kopf und Text, da \DescRef{scrpage-de.option.headsepline} als Argument für
  \Package{scrpage2} angegeben wurde.%
\begin{lstcode}
  \manualmark
  \pagestyle{ohneLinien}
  \section{Exergie und Anergie}
  \markright{Energieumwandlung}
  Man bezeichnet die bei der Einstellung des 
  Gleichgewichts mit der Umgebung maximal gewinnbare
  Arbeit als Exergie.
\end{lstcode}%
  \begin{XmpTopPage}
        \XmpHeading[\slshape Energieumwandlung\hfill ohne Linien]{10,27}{70}
        \thinlines\XmpRule{10,25}{70}
        \XmpSetText[{%
        \textbf{\sffamily 1.2 Exergie und Anergie}\\[2pt]
        Man bezeichnet die bei der Einstellung des
        Gleichgewichts mit der Umgebung maximal\linebreak}]{10,21}
  \end{XmpTopPage}

  Nach dem Wechsel auf die folgende linke Seite wird der Seitenstil
  \PValue{mitLinien} aktiviert.  Die Linieneinstellungen werden hier nun
  angewendet und entsprechend der Definition dargestellt.
\begin{lstcode}
  \pagestyle{mitLinien}
  \renewcommand{\headfont}{\itshape\bfseries}
  Den nicht in Exergie umwandelbaren Anteil einer
  Energie nennt man Anergie \Var{B}.
  \[ B = U + T (S_1 - S_u) - p (V_1 - V_u)\] 
  \end{document}
\end{lstcode}
  \begin{XmpTopPage}
        \XmpHeading[\itshape\bfseries mit Linien\hfill 1 Thermodynamik]{20,27}{70}
        \thicklines\XmpRule{20,29}{70}
        \XmpSetText[%
        Den nicht in Exergie umwandelbaren Anteil einer
        Energie nennt man Anergie $B$. 
        \vspace{-.5ex}\[ B = U + T (S_1 - S_u) - p (V_1 - V_u)\] ]{20,21}
  \end{XmpTopPage}
\end{Example}
\EndIndexGroup


\subsection{Seitenstile verwalten}\label{sec:scrpage-de.UI.cfgFile}
% scrpage.cfg
\BeginIndex{File}{scrpage.cfg}
Bei längerer Arbeit mit verschiedenen Seitenstilen wird sich, je nach
Geschmack und Aufgabenstellung, ein fester Satz an benutzten Stilen
etablieren.
Um nicht bei jedem neuen Projekt eine große Kopieraktion von den Daten
eines Projekts zum neuen Projekt starten zu müssen, liest \Package{scrpage2}
am Ende seiner Initialisierungsphase die Datei \File{scrpage.cfg} ein.
In dieser Datei können dann Seitenstile definiert sein, die viele
Projekte gemeinsam nutzen können.
\EndIndex{File}{scrpage.cfg}
%
\EndIndexGroup

\end{document}

%%% Local Variables:
%%% mode: latex
%%% coding: utf-8
%%% TeX-master: t
%%% End: 
