% ======================================================================
% common-interleafpage.tex
% Copyright (c) Markus Kohm, 2001-2019
%
% This file is part of the LaTeX2e KOMA-Script bundle.
%
% This work may be distributed and/or modified under the conditions of
% the LaTeX Project Public License, version 1.3c of the license.
% The latest version of this license is in
%   http://www.latex-project.org/lppl.txt
% and version 1.3c or later is part of all distributions of LaTeX 
% version 2005/12/01 or later and of this work.
%
% This work has the LPPL maintenance status "author-maintained".
%
% The Current Maintainer and author of this work is Markus Kohm.
%
% This work consists of all files listed in manifest.txt.
% ----------------------------------------------------------------------
% common-interleafpage.tex
% Copyright (c) Markus Kohm, 2001-2019
%
% Dieses Werk darf nach den Bedingungen der LaTeX Project Public Lizenz,
% Version 1.3c, verteilt und/oder veraendert werden.
% Die neuste Version dieser Lizenz ist
%   http://www.latex-project.org/lppl.txt
% und Version 1.3c ist Teil aller Verteilungen von LaTeX
% Version 2005/12/01 oder spaeter und dieses Werks.
%
% Dieses Werk hat den LPPL-Verwaltungs-Status "author-maintained"
% (allein durch den Autor verwaltet).
%
% Der Aktuelle Verwalter und Autor dieses Werkes ist Markus Kohm.
% 
% Dieses Werk besteht aus den in manifest.txt aufgefuehrten Dateien.
% ======================================================================
%
% Paragraphs that are common for several chapters of the KOMA-Script guide
% Maintained by Markus Kohm
%
% ----------------------------------------------------------------------
%
% Absaetze, die mehreren Kapiteln der KOMA-Script-Anleitung gemeinsam sind
% Verwaltet von Markus Kohm
%
% ======================================================================

\KOMAProvidesFile{common-interleafpage.tex}%
                 [$Date$
                  KOMA-Script guide (common paragraphs: Interleaf Pages)]
\translator{Markus Kohm\and Gernot Hassenpflug\and Krickette Murabayashi\and
	Karl Hagen}

% Date of the translated German file: 2019-10-21

\section{Interleaf Pages}
\seclabel{emptypage}%
\BeginIndexGroup
\BeginIndex{}{interleaf page}%%
\BeginIndex{}{page>style}%

\IfThisCommonFirstRun{}{%
  The information in \autoref{sec:\ThisCommonFirstLabelBase.emptypage} applies
  equally to this chapter. So if you have already read and understood
  \autoref{sec:\ThisCommonFirstLabelBase.emptypage}, you can skip ahead to
  \autoref{sec:\ThisCommonLabelBase.emptypage.next},
  \autopageref{sec:\ThisCommonLabelBase.emptypage.next}.%
}

Interleaf pages are pages that are inserted between parts of a document. 
Traditionally, these pages are completely blank. \LaTeX{}, however, 
sets them by default with the current page style. \KOMAScript{} provides 
several extensions to this functionality.

Interleaf pages are mostly found in books. Because book chapters commonly
start on the right (recto) page of a two-page spread, an empty left (verso)
page must be inserted if the previous chapter ends on a recto page. For this
reason, interleaf pages really only exist for two-sided printing. 
%
\iffalse % Umbruchkorrektur
  The blank versos in one-sided printing are not true interleaf pages,
  although they may appear as such in counting the printed sheets.
\fi%

\IfThisCommonLabelBase{scrlttr2}{%
  Interleaf pages are unusual in letters. This is not least because two-sided
  letters are rare, as letters are usually not bound. Nevertheless,
  \KOMAScript{} also supports interleaf pages for two-sided letters. However,
  since the commands described here are seldom used in letters, you will not
  find any examples here. If necessary, please refer to the examples in
  \autoref{sec:maincls.emptypage}, starting on
  \autopageref{sec:maincls.emptypage}.%
}{}%

\begin{Declaration}
  \OptionVName{cleardoublepage}{page style}
  \OptionValue{cleardoublepage}{current}
\end{Declaration}%
With this option,\IfThisCommonLabelBase{maincls}{%
  \ChangedAt{v3.00}{\Class{scrbook}\and \Class{scrreprt}\and
    \Class{scrartcl}}%
}{%
  \IfThisCommonLabelBase{scrlttr2}{%
    \ChangedAt{v3.00}{\Class{scrlttr2}}%
  }{}%
} you can define the page style of the interleaf pages created by the commands
\DescRef{\LabelBase.cmd.cleardoublepage},
\DescRef{\LabelBase.cmd.cleardoubleoddpage}, or
\DescRef{\LabelBase.cmd.cleardoubleevenpage} to advance to the desired page.
You can use any previously defined \PName{page style} (see
\autoref{sec:\ThisCommonLabelBase.pagestyle} from
\autopageref{sec:\ThisCommonLabelBase.pagestyle} and
\autoref{cha:scrlayer-scrpage} from \autopageref{cha:scrlayer-scrpage}).
In addition, \OptionValue{cleardoublepage}{current} is also possible.
This case corresponds to the default prior to \KOMAScript~2.98c and creates an
interleaf page without changing the page style. Starting with
\KOMAScript~3.00\IfThisCommonLabelBase{maincls}{%
  \ChangedAt{v3.00}{\Class{scrbook}\and \Class{scrreprt}\and
    \Class{scrartcl}}%
}{%
  \IfThisCommonLabelBase{scrlttr2}{%
    \ChangedAt{v3.00}{\Class{scrlttr2}}%
  }{}%
}, the default\textnote{default} follows the recommendation of most
typographers and creates interleaf pages with the
\IfThisCommonLabelBase{scrextend}{%
  \DescRef{maincls.pagestyle.empty}}{%
  \DescRef{\ThisCommonLabelBase.pagestyle.empty}}\IndexPagestyle{empty}
page style unless you switch compatibility to earlier \KOMAScript{} versions
(see option \DescRef{\ThisCommonLabelBase.option.version}%
\important{\OptionValueRef{\LabelBase}{version}{2.98c}},
\autoref{sec:\ThisCommonLabelBase.compatibilityOptions},
\DescPageRef{\ThisCommonLabelBase.option.version}).
\IfThisCommonLabelBase{maincls}{\iftrue}{\csname iffalse\endcsname}
  \begin{Example}
    \phantomsection\xmpllabel{option.cleardoublepage}%
    Suppose you want interleaf pages that are empty except for the pagination% 
    \iffree{, so they are created with \IfThisCommonLabelBase{scrextend}{%
        \DescRef{maincls.pagestyle.plain}}{%
        \DescRef{\LabelBase.pagestyle.plain}}}{}. You can achieve this,
    for example, with:
\begin{lstcode}
  \KOMAoptions{cleardoublepage=plain}
\end{lstcode}
    You can find more information about the
    \IfThisCommonLabelBase{scrextend}{%
      \DescRef{maincls.pagestyle.plain}}{\DescRef{\LabelBase.pagestyle.plain}}
    page style in \IfThisCommonLabelBase{scrextend}{%
      \autoref{sec:maincls.pagestyle}}{%
      \autoref{sec:\LabelBase.pagestyle}},
    \IfThisCommonLabelBase{scrextend}{%
      \DescPageRef{maincls.pagestyle.plain}}{%
      \DescPageRef{\LabelBase.pagestyle.plain}}.
  \end{Example}
\else
  \IfThisCommonLabelBase{scrextend}{%
    You can find an example for setting the page style of interleaf pages in
    \autoref{sec:\ThisCommonFirstLabelBase.emptypage},
    \PageRefxmpl{\ThisCommonFirstLabelBase.option.cleardoublepage}.%
    \iffalse% Umbruchvariante ohne Beispiel
  }{\csname iffalse\endcsname}
    \begin{Example}
      \phantomsection\xmpllabel{option.cleardoublepage}%
      Suppose you want interleaf pages that are empty except for the pagination,
      so they are created with the \IfThisCommonLabelBase{scrextend}{%
        \DescRef{maincls.pagestyle.plain}}{\DescRef{\LabelBase.pagestyle.plain}}
      page style. You can achieve this with
\begin{lstcode}
  \KOMAoptions{cleardoublepage=plain}
\end{lstcode}
      You can find more information about the
      \DescRef{maincls.pagestyle.plain} page style in
      \autoref{sec:maincls.pagestyle}, \DescPageRef{maincls.pagestyle.plain}.
    \end{Example}%
  \fi%
\fi%
\EndIndexGroup


\begin{Declaration}
  \Macro{clearpage}%
  \Macro{cleardoublepage}%
  \Macro{cleardoublepageusingstyle}\Parameter{page style}%
  \Macro{cleardoubleemptypage}%
  \Macro{cleardoubleplainpage}%
  \Macro{cleardoublestandardpage}%
  \Macro{cleardoubleoddpage}%
  \Macro{cleardoubleoddpageusingstyle}\Parameter{page style}%
  \Macro{cleardoubleoddemptypage}%
  \Macro{cleardoubleoddplainpage}%
  \Macro{cleardoubleoddstandardpage}%
  \Macro{cleardoubleevenpage}%
  \Macro{cleardoubleevenpageusingstyle}\Parameter{page style}%
  \Macro{cleardoubleevenemptypage}%
  \Macro{cleardoubleevenplainpage}%
  \Macro{cleardoubleevenstandardpage}
\end{Declaration}%
The\textnote{standard classes} {\LaTeX} kernel provides the \Macro{clearpage}
command, which ensures that all pending floats are output and then starts a
new page. There is also the \Macro{cleardoublepage} command, which works like
\Macro{clearpage} but which starts a new right-hand page in two-sided printing
(see the \Option{twoside} layout option in \autoref{sec:typearea.options},
\DescPageRef{typearea.option.twoside}). An empty left-hand page in the current
page style is output if necessary.

With\IfThisCommonLabelBase{maincls}{%
  \ChangedAt{v3.00}{\Class{scrbook}\and \Class{scrreprt}\and
    \Class{scrartcl}}%
}{%
  \IfThisCommonLabelBase{scrlttr2}{%
    \ChangedAt{v3.00}{\Class{scrlttr2}}%
  }{}%
} \Macro{cleardoubleoddstandardpage}, {\KOMAScript}\textnote{\KOMAScript}
works as exactly in the way just described for the standard classess.  The
\Macro{cleardoubleoddplainpage}%
\important{\IfThisCommonLabelBase{scrextend}{%
    \DescRef{maincls.pagestyle.plain}}{\DescRef{\LabelBase.pagestyle.plain}}}
command, on the other hand, additionally changes the page style of the empty
left page to \IfThisCommonLabelBase{scrextend}{%
  \DescRef{maincls.pagestyle.plain}}{\DescRef{\LabelBase.pagestyle.plain}}%
\IndexPagestyle{plain} in order to suppress the
\IfThisCommonLabelBase{scrlttr2}{page header}{running title}.  Likewise, the
\Macro{cleardoubleoddemptypage}\important{%
  \IfThisCommonLabelBase{scrextend}{\DescRef{maincls.pagestyle.empty}}{%
    \DescRef{\LabelBase.pagestyle.empty}}} command uses the
\IfThisCommonLabelBase{scrextend}{\DescRef{maincls.pagestyle.empty}}{%
  \DescRef{\LabelBase.pagestyle.empty}}\IndexPagestyle{empty} page style to
suppress both \IfThisCommonLabelBase{scrlttr2}{page header and page footer}%
{running title and page number} on the empty left-hand side. The page is thus
completely empty. If you want to specify your own \PName{page style} for the
interleaf page, this should be given as an argument of
\Macro{cleardoubleoddusingpagestyle}. You can use any previously defined
\PName{page style} (see \autoref{cha:scrlayer-scrpage}).

\IfThisCommonLabelBase{scrlttr2}{}{%
  Sometimes\textnote{odd-side interleaf pages} you want chapters to start not
  on the right-hand but on the left-hand page. Although this layout contradicts
  classic typography, it can be appropriate if the double-page spread at the
  beginning of the chapter very specific contents. For this reason,
  \KOMAScript{} provides the \Macro{cleardoubleevenstandardpage} command,
  which is equivalent to the \Macro{cleardoubleoddstandardpage} command
  except that the next page is a left page. The same applies to the
  \Macro{cleardoubleevenplainpage}, \Macro{cleardoubleevenemptypage}, and
  \Macro{cleardoubleevenpageusingstyle} commands%
  \IfThisCommonLabelBase{maincls}{% Umbruchoptimierungsalternative
  , the last of which expects an argument}{}.%
}

The \Macro{cleardoublestandardpage}, \Macro{cleardoubleemptypage}, and
\Macro{cleardoubleplainpage} commands, and the single-argument
\Macro{cleardoublepageusingstyle} command, as well as the standard
\Macro{cleardoublepage} command, %
\IfThisCommonLabelBase{maincls}{%
  depend on the \DescRef{maincls.option.open}\IndexOption{open}%
  \important{\DescRef{maincls.option.open}} option explained in
  \autoref{sec:maincls.structure}, \DescPageRef{maincls.option.open} and,
  depending on that setting, correspond to one of the commands explained in
  the preceding paragraphs.  }{%
  correspond to the commands previously explained for the
  \IfThisCommonLabelBase{scrlttr2}{\Class{scrlttr2} class}{%
    \IfThisCommonLabelBase{scrextend}{\Package{scrextend} package}{%
      \InternalCommonFileUsageError}%
  }%
  \IfThisCommonLabelBase{scrlttr2}{. %
    The remaining commands are defined in \Class{scrlttr2} for completeness
    only. You can find more information on them in
    \autoref{sec:maincls.emptypage}, \DescPageRef{maincls.cmd.cleardoublepage}
    if necessary%
  }{%
    \ to transition to the next odd page%
  }.%
}%
\IfThisCommonLabelBaseOneOf{scrlttr2,scrextend}{\iffalse}{\csname
  iftrue\endcsname}
  \begin{Example}
    \phantomsection\xmpllabel{cmd.cleardoublepage}%
    Suppose you want to insert a double-page spread into your document with a
    picture on the left-hand page and a new chapter starting on the right-hand
    page. If the previous chapter ends with a left-hand page, an interleaf 
    page has to be added, which should be completely empty. The picture should
    be the same size as the text area without any header or footer. 
\iffalse% Umbruchkorrekturtext
      First of all,
\begin{lstcode}
  \KOMAoptions{cleardoublepage=empty}
\end{lstcode}
      ensures that interleaf pages use the
      \IfThisCommonLabelBase{scrextend}{\DescRef{maincls.pagestyle.empty}}{%
    	\DescRef{\LabelBase.pagestyle.empty}} page style. You can put this
      setting in the document preamble, or you can set it as an optional 
      argument of \DescRef{\ThisCommonLabelBase.cmd.documentclass}.
\fi

    At the relevant place in your document, write:
\begin{lstcode}
  \cleardoubleevenemptypage
  \thispagestyle{empty}
  \includegraphics[width=\textwidth,%
                   height=\textheight,%
                   keepaspectratio]%
                  {picture}
  \chapter{Chapter Heading}
\end{lstcode}
    The first of these lines switches to the next left-hand page. If needed it
    also adds a completely blank right-hand page. The second line makes sure
    that the following left-hand page will also be set using the 
    \IfThisCommonLabelBase{scrextend}{\DescRef{maincls.pagestyle.empty}}{%
    	\DescRef{\LabelBase.pagestyle.empty}} page style. The third through
    sixth lines load an image file named \File{picture} and scale it to the
    desired size without distorting it. This command requires the 
    \Package{graphicx}\IndexPackage{graphicx} package (see
    \cite{package:graphics}). The last line starts
    a new chapter on the next page, which will be a right-hand one.
  \end{Example}%
\fi%

In two-sided printing, \Macro{cleardoubleoddpage} always moves to the next
left-hand page and \Macro{cleardoubleevenpage} to the next right-hand
page. The style of the interleaf page to be inserted if necessary is defined
with the \DescRef{\LabelBase.option.cleardoublepage} option.%
\IfThisCommonLabelBase{scrextend}{\par%
  For an example that uses \Macro{cleardoubleevenemptypage}, see
  \autoref{sec:maincls.emptypage},
  \PageRefxmpl{\ThisCommonFirstLabelBase.cmd.cleardoublepage}.%
}{}%
\EndIndexGroup
%
\EndIndexGroup

%%% Local Variables:
%%% mode: latex
%%% mode: flyspell
%%% coding: us-ascii
%%% ispell-local-dictionary: "en_GB"
%%% TeX-master: "../guide"
%%% End:

%  LocalWords:  mutatis mutandis Interleaf interleaf
