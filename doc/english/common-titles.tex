% ======================================================================
% common-titles.tex
% Copyright (c) Markus Kohm, 2001-2016
%
% This file is part of the LaTeX2e KOMA-Script bundle.
%
% This work may be distributed and/or modified under the conditions of
% the LaTeX Project Public License, version 1.3c of the license.
% The latest version of this license is in
%   http://www.latex-project.org/lppl.txt
% and version 1.3c or later is part of all distributions of LaTeX 
% version 2005/12/01 or later and of this work.
%
% This work has the LPPL maintenance status "author-maintained".
%
% The Current Maintainer and author of this work is Markus Kohm.
%
% This work consists of all files listed in manifest.txt.
% ----------------------------------------------------------------------
% common-titles.tex
% Copyright (c) Markus Kohm, 2001-2016
%
% Dieses Werk darf nach den Bedingungen der LaTeX Project Public Lizenz,
% Version 1.3c, verteilt und/oder veraendert werden.
% Die neuste Version dieser Lizenz ist
%   http://www.latex-project.org/lppl.txt
% und Version 1.3c ist Teil aller Verteilungen von LaTeX
% Version 2005/12/01 oder spaeter und dieses Werks.
%
% Dieses Werk hat den LPPL-Verwaltungs-Status "author-maintained"
% (allein durch den Autor verwaltet).
%
% Der Aktuelle Verwalter und Autor dieses Werkes ist Markus Kohm.
% 
% Dieses Werk besteht aus den in manifest.txt aufgefuehrten Dateien.
% ======================================================================
%
% Paragraphs that are common for several chapters of the KOMA-Script guide
% Maintained by Markus Kohm
%
% ----------------------------------------------------------------------
%
% Absaetze, die mehreren Kapiteln der KOMA-Script-Anleitung gemeinsam sind
% Verwaltet von Markus Kohm
%
% ======================================================================

\KOMAProvidesFile{common-titles.tex}
                 [$Date$
                  KOMA-Script guide (common paragraphs)]

\translator{Gernot Hassenpflug\and Markus Kohm\and Krickette Murabayashi}

% Date of the translated German file: 2016-11-14

\section{Document Titles}
\seclabel{titlepage}%
\BeginIndexGroup
\BeginIndex{}{title}%
\BeginIndex{}{document>title}%

\IfThisCommonFirstRun{}{%
  What is written in \autoref{sec:\ThisCommonFirstLabelBase.titlepage}
  applies, mutatis mutandis.  So if you have alread read and understood
  \autoref{sec:\ThisCommonFirstLabelBase.titlepage} you can jump to
  \autoref{sec:\ThisCommonLabelBase.titlepage.next} on
  \autopageref{sec:\ThisCommonLabelBase.titlepage.next}.%
}%
\IfThisCommonLabelBase{scrextend}{\iftrue}{\csname iffalse\endcsname}%
  \ But there's a difference: The\textnote{Attention!} document title
  capabilities of \Package{scrextend} are part of the optional, advanced
  features. Therfore they are only available, if
  \OptionValueRef{\ThisCommonLabelBase}{extendedfeature}{title} has been
  selected while loading the package (see
  \autoref{sec:\ThisCommonLabelBase.optionalFeatures},
  \DescRef{\ThisCommonLabelBase.option.extendedfeature}).

  Beyond that \Package{scrextend} cannot be used with a \KOMAScript{}
  class together. Because of this
\begin{lstcode}
  \documentclass{scrbook}
\end{lstcode}
  must be replaced by
\begin{lstcode}
  \documentclass{book}
  \usepackage[extendedfeature=title]{scrextend}
\end{lstcode}
  at all examples from \autoref{sec:maincls.titlepage}, if \Package{scrextend}
  should be used.
\fi

In general we distinguish two kinds of document titles. First known are title
pages. In this case the document title will be placed together with additional
document information, like the author, on a page of its own. Besides the main
title page, several further title pages may exist, like the half-title or
bastard title, publisher data, dedication, or similar. The second known kind
of document title is the in-page title. In this case, the document title is
placed at the top of a page and specially emphasized, and may be accompanied
by some additional information too, but it will be followed by more material
in the same page, for instance by an abstract, or the table of contents, or
even a section.


\begin{Declaration}
  \OptionVName{titlepage}{simple switch}%
  \OptionValue{titlepage}{firstiscover}
\end{Declaration}%
Using \DescRef{\ThisCommonLabelBase.cmd.maketitle} (see
\DescPageRef{\ThisCommonLabelBase.cmd.maketitle}), this option%
\IfThisCommonLabelBase{maincls}{%
  \ChangedAt{v3.00}{\Class{scrbook}\and \Class{scrreprt}\and
    \Class{scrartcl}}%
}{} switches between document title pages\Index{title>pages} and in-page
title\Index{title>in-page}. For \PName{simple switch}, any value from
\autoref{tab:truefalseswitch}, \autopageref{tab:truefalseswitch} may be used.

The option
\OptionValue{titlepage}{true}\important{\OptionValue{titlepage}{true}}
\IfThisCommonLabelBase{scrextend}{}{or \Option{titlepage} }makes {\LaTeX} use
separate pages for the titles.  Command
\DescRef{\ThisCommonLabelBase.cmd.maketitle} sets these pages inside a
\DescRef{\ThisCommonLabelBase.env.titlepage} environment and the pages
normally have neither header nor footer. In comparison with standard {\LaTeX},
{\KOMAScript} expands the handling of the titles significantly.

The option
\OptionValue{titlepage}{false}\important{\OptionValue{titlepage}{false}}
specifies that an \emph{in-page} title is used. This means that the title is
specially emphasized, but it may be followed by more material on the same
page, for instance by an abstract or a section.%

The third choice,%
\IfThisCommonLabelBase{maincls}{%
  \ChangedAt{v3.12}{\Class{scrbook}\and \Class{scrreprt}\and
    \Class{scrartcl}}%
}{%
  \IfThisCommonLabelBase{scrextend}{%
    \ChangedAt{v3.12}{\Package{scrextend}}%
  }{\InternalCommonFileUseError}%
} \OptionValue{titlepage}{firstiscover}%
\important{\OptionValue{titlepage}{firstiscover}} does not only select title
pages. It additionally prints the first title page of
\DescRef{\ThisCommonLabelBase.cmd.maketitle}\IndexCmd{maketitle}, this is
either the extra title or the main title, as a cover\Index{cover} page. Every
other setting of option \Option{titlepage} would cancel this setting. The
margins\important{%
  \Macro{coverpagetopmargin}\\
  \Macro{coverpagebottommargin}\\
  \Macro{coverpageleftmargin}\\
  \Macro{coverpagerightmargin}} of the cover page are given by
\Macro{coverpagetopmargin}\IndexCmd[indexmain]{coverpagetopmargin},
\Macro{coverpagebottommargin}\IndexCmd[indexmain]{coverpagebottommargin},
\Macro{coverpageleftmargin}\IndexCmd[indexmain]{coverpageleftmargin} und
\Macro{coverpagerightmargin}\IndexCmd[indexmain]{coverpagerightmargin}. The
defaults of these depend on the lengths
\Length{topmargin}\IndexLength{topmargin} and
\Length{evensidemargin}\IndexLength{evensidemargin} and can be changed using
\Macro{renewcommand}.

\IfThisCommonLabelBase{maincls}{%
  The default of classes \Class{scrbook} and \Class{scrreprt} is usage of
  title pages. Class \Class{scrartcl}, on the other hand, uses in-page titles
  as default.%
}{%
  \IfThisCommonLabelBase{scrextend}{%
    The default depends on the used class and \Package{scrextend} handles it
    compatible to the standard class. If a class does not set up a comparable
    default, in-page title will be used.%
  }{\InternalCommonFileUsageError}%
}%
%
\EndIndexGroup


\begin{Declaration}
  \begin{Environment}{titlepage}\end{Environment}%
\end{Declaration}%
With the standard classes and with {\KOMAScript}, all title pages are defined
in a special environment, the \Environment{titlepage} environment.  This
environment always starts a new page\,---\,in the two-sided layout a new right
page\,---\,and in single column mode. For this page, the style is changed by
\DescRef{maincls.cmd.thispagestyle}\PParameter{empty}, so that
neither page number nor running heading are output. At the end of the
environment the page is automatically shipped out. Should you not be able to
use the automatic layout of the title pages provided by
\DescRef{\ThisCommonLabelBase.cmd.maketitle}, that will be described next; it
is advisable to design a new one with the help of this environment.

\IfThisCommonFirstRun{\iftrue}{%
  A simple example for a title page with \Environment{titlepage} is shown in
  \autoref{sec:\ThisCommonFirstLabelBase.titlepage} on
  \PageRefxmpl{\ThisCommonFirstLabelBase.env.titlepage}%
  \csname iffalse\endcsname%
}%
  \begin{Example}
    \phantomsection\xmpllabel{env.titlepage}
    Assume you want a title page on which only the word ``Me'' stands at
    the top on the left, as large as possible and in bold\,---\,no
    author, no date, nothing else. The following document creates just
    that:
\begin{lstcode}
  \documentclass{scrbook}
  \begin{document}
  \begin{titlepage}
    \textbf{\Huge Me}
  \end{titlepage}
  \end{document}
\end{lstcode}
    It's simple, isn't it?
  \end{Example}
\fi%
\EndIndexGroup


\begin{Declaration}
  \Macro{maketitle}\OParameter{page number}
\end{Declaration}%
While the the standard classes produce at least one title page that may have
the three items title, author, and date, with {\KOMAScript} the
\Macro{maketitle} command can produce up to six pages. In contrast to the
standard classes, the \Macro{maketitle} macro in {\KOMAScript} accepts an
optional numeric argument. If it is used, this number is made the page number
of the first title page.  However, this page number is not output, but affects
only the numbering. You should choose an odd number, because otherwise the
whole count gets mixed up. In my opinion there are only two meaningful
applications for the optional argument. On the one hand, one could give to the
half-title\Index[indexmain]{half-title} the logical page number \(-\)1 in
order to give the full title page the number 1. On the other hand, it could be
used to start at a higher page number, for instance, 3, 5, or 7, to
accommodate other title pages added by the publishing house.  The optional
argument is ignored for \emph{in-page} titles. However, the page style of such
a title page can be changed by redefining the
\DescRef{\ThisCommonLabelBase.cmd.titlepagestyle} macro.  For that see
\autoref{sec:maincls.pagestyle}, \DescPageRef{maincls.cmd.titlepagestyle}.

The following commands do not lead immediately to the ship-out of the
titles. The typesetting and ship-out of the title pages are always done by
\Macro{maketitle}. By the way, you should note that \Macro{maketitle} should
not be used inside a \DescRef{\ThisCommonLabelBase.env.titlepage}
environment. Like\textnote{Attention!} shown in the examples, one should use
either \Macro{maketitle} or \DescRef{\ThisCommonLabelBase.env.titlepage} only,
but not both.

The commands explained below only define the contents of the title
pages. Because of this, they have to be used before \Macro{maketitle}. It is,
however, not necessary and, when using, e.\,g., the \Package{babel}
package\IndexPackage{babel}, not recommended to use these in the preamble
before \Macro{begin}\PParameter{document} (see \cite{package:babel}). Examples
can be found at the end of this section.


\begin{Declaration}
  \Macro{extratitle}\Parameter{half-title}
\end{Declaration}%
\begin{Explain}%
  In earlier times the inner book was often not protected from dirt by a cover.
  This task was then taken over by the first page of the book which
  carried mostly a shortened title called the \emph{half-title}.
  Nowadays the extra page is often applied before the real full title
  and contains information about the publisher, series number and similar
  information.
\end{Explain}
With {\KOMAScript} it is possible to include a page before the real title
page.  The \PName{half-title} can be arbitrary text\,---\,even several
paragraphs. The contents of the \PName {half-title} are output by
{\KOMAScript} without additional formatting. Their organisation is completely
left to the user. The back of the half-title remains empty.  The half-title
has its own title page even when \emph{in-page} titles are used. The output of
the half-title defined with \Macro{extratitle} takes place as part of the
titles produced by \DescRef{\ThisCommonLabelBase.cmd.maketitle}.

\IfThisCommonFirstRun{\iftrue}{%
  A simple example for a title page with extra title and main title is shown
  in \autoref{sec:\ThisCommonFirstLabelBase.titlepage} on
  \PageRefxmpl{\ThisCommonFirstLabelBase.cmd.extratitle}%
  \csname iffalse\endcsname%
}%
  \begin{Example}
    \phantomsection\xmpllabel{cmd.extratitle}
    Let's go back to the previous example and assume
    that the spartan ``Me'' is the half-title. The full title should
    still follow the half-title. One can proceed as follows:
\begin{lstcode}
  \documentclass{scrbook}
  \begin{document}
    \extratitle{\textbf{\Huge Me}}
    \title{It's me}
    \maketitle
  \end{document}
\end{lstcode}
    You can center the half-title horizontally and put it a little lower down
    the page:
\begin{lstcode}
  \documentclass{scrbook}
  \begin{document}
    \extratitle{\vspace*{4\baselineskip}
      \begin{center}\textbf{\Huge Me}\end{center}}
    \title{It's me}
    \maketitle
  \end{document}
\end{lstcode}
    The command\textnote{Attention!} \DescRef{\ThisCommonLabelBase.cmd.title}
    is necessary in order to make the examples above work correctly. It is
    explained next.
  \end{Example}
\fi%
\EndIndexGroup


\begin{Declaration}
  \Macro{titlehead}\Parameter{title head}%
  \Macro{subject}\Parameter{subject}%
  \Macro{title}\Parameter{title}%
  \Macro{subtitle}\Parameter{subtitle}%
  \Macro{author}\Parameter{author}%
  \Macro{date}\Parameter{date}%
  \Macro{publishers}\Parameter{publisher}%
  \Macro{and}%
  \Macro{thanks}\Parameter{footnote}
\end{Declaration}%
The contents of the full title page are defined by seven elements. The output
of the full title page occurs as part of the title pages of
\DescRef{\ThisCommonLabelBase.cmd.maketitle}, whereas the now listed elements
only define the corresponding elements.

\BeginIndexGroup
\BeginIndex{FontElement}{titlehead}\LabelFontElement{titlehead}%
The\important{\Macro{titlehead}} \PName{title
  head}\Index[indexmain]{title>head} is defined with the command
\Macro{titlehead}. It is typeset with the font of the homonymous element in
regular justification and full width at the top of the page. It can be freely
designed by the user.%
\EndIndexGroup

\BeginIndexGroup
\BeginIndex{FontElement}{subject}\LabelFontElement{subject}%
The\important{\Macro{subject}} \PName{subject}\Index[indexmain]{subject} is
output with the font of the homonymous element immediately above the
\PName{title}.%
\EndIndexGroup

\BeginIndexGroup
\BeginIndex{FontElement}{title}\LabelFontElement{title}%
The\important{\Macro{title}} \PName{title} is output with a very large font
size.  Beside\IfThisCommonLabelBase{maincls}{%
  \ChangedAt{v2.8p}{\Class{scrbook}\and \Class{scrreprt}\and
    \Class{scrartcl}}}{} all other element the font size\textnote{Attention!}
is, however, not affected by the font switching element \FontElement{title}
(see \autoref{tab:\ThisCommonFirstLabelBase.mainTitle},
\autopageref{tab:\ThisCommonFirstLabelBase.mainTitle}).%
\EndIndexGroup

\BeginIndexGroup
\BeginIndex{FontElement}{subtitle}\LabelFontElement{subtitle}%
The\important{\Macro{subtitle}}
\PName{subtitle}\IfThisCommonLabelBase{maincls}{%
  \ChangedAt{v2.97c}{\Class{scrbook}\and \Class{scrreprt}\and
    \Class{scrartcl}}}{} is output with the font of the homonymous element
just below the title.%
\EndIndexGroup

\BeginIndexGroup
\BeginIndex{FontElement}{author}\LabelFontElement{author}%
Below\important{\Macro{author}} the \PName{subtitle} appears the
\PName{author}\Index[indexmain]{author}.  Several authors can be specified in
the argument of \Macro{author}. They should be separated by \Macro{and}. The
font of element \FontElement{author} is used for the output of the authors.%
\EndIndexGroup

\BeginIndexGroup
\BeginIndex{FontElement}{date}\LabelFontElement{date}%
Below\important{\Macro{date}} the author or authors appears the
date\Index{date} in the font of the homonymous element. The default value is
the present date, as produced by \Macro{today}\IndexCmd{today}. The
\Macro{date} command accepts arbitrary information\,---\,even an empty
argument.%
\EndIndexGroup

\BeginIndexGroup
\BeginIndex{FontElement}{publishers}\LabelFontElement{publishers}%
Finally\important{\Macro{publishers}} comes the
\PName{publisher}\Index[indexmain]{publisher}. Of course this command can also
be used for any other information of little importance. If necessary, the
\Macro{parbox} command can be used to typeset this information over the full
page width like a regular paragraph instead of centering it.  Then it is to be
considered equivalent to the title head. However, note that this field is put
above any existing footnotes. The font of element \FontElement{publishers} is
used for the output.%
\EndIndexGroup

Footnotes\important{\Macro{thanks}}\Index{footnotes} on the title page are
produced not with \Macro{footnote}, but with \Macro{thanks}. They serve
typically for notes associated with the authors. Symbols are used as footnote
markers instead of numbers. Note\textnote{Attention!}, that \Macro{thanks} has
to be used inside the argument of another command, e.\,g., at the argument
\PName{author} of the command \Macro{author}.

While%
\IfThisCommonLabelBase{maincls}{%
  \ChangedAt{v3.12}{\Class{scrbook}\and \Class{scrreprt}\and
    \Class{scrartcl}}%
}{%
  \IfThisCommonLabelBase{scrextend}{%
    \ChangedAt{v3.12}{\Package{scrextend}}%
  }{\InternalCommonFileUsageError}%
} printing the title elements the equal named font switching elements will be
used for all them. The defaults, that may be found in
\autoref{tab:\ThisCommonFirstLabelBase.titlefonts}, %
\IfThisCommonFirstRun{}{%
  \autopageref{tab:\ThisCommonFirstLabelBase.titlefonts}, %
}%
may be changed using the commands
\DescRef{\ThisCommonLabelBase.cmd.setkomafont} and
\DescRef{\ThisCommonLabelBase.cmd.addtokomafont} (see
\autoref{sec:\ThisCommonLabelBase.textmarkup},
\DescPageRef{\ThisCommonLabelBase.cmd.setkomafont}).%
\IfThisCommonFirstRun{%
  \begin{table}
%  \centering
    \KOMAoptions{captions=topbeside}%
    \setcapindent{0pt}%
%  \caption
    \begin{captionbeside}
      [{Font defaults for the elements of the title}]
      {\label{tab:\ThisCommonLabelBase.titlefonts}%
        Font defaults for the elements of the title}
      [l]
      \begin{tabular}[t]{ll}
        \toprule
        Element name & Default \\
        \midrule
        \DescRef{\ThisCommonLabelBase.fontelement.author} & \Macro{Large} \\
        \DescRef{\ThisCommonLabelBase.fontelement.date} & \Macro{Large} \\
        \DescRef{\ThisCommonLabelBase.fontelement.dedication} & \Macro{Large} \\
        \DescRef{\ThisCommonLabelBase.fontelement.publishers} & \Macro{Large} \\
        \DescRef{\ThisCommonLabelBase.fontelement.subject} &
          \Macro{normalfont}\Macro{normalcolor}\Macro{bfseries}\Macro{Large} \\
        \DescRef{\ThisCommonLabelBase.fontelement.subtitle} &
          \Macro{usekomafont}\PParameter{title}\Macro{large} \\
        \DescRef{\ThisCommonLabelBase.fontelement.title} & \Macro{usekomafont}\PParameter{disposition} \\
        \DescRef{\ThisCommonLabelBase.fontelement.titlehead} & \\
        \bottomrule
      \end{tabular}
    \end{captionbeside}
  \end{table}%
}{} 

With the exception of \PName{titlehead} and possible footnotes, all
the items are centered horizontally.  The information is summarised in
\autoref{tab:\ThisCommonFirstLabelBase.mainTitle}%
\IfThisCommonFirstRun{}{%
  , \autopageref{tab:\ThisCommonFirstLabelBase.mainTitle}}.
\IfThisCommonFirstRun{%
  \begin{table}
%  \centering
    \KOMAoptions{captions=topbeside}%
    \setcapindent{0pt}%
    % \caption
    \begin{captionbeside}[Main title]{%
        Font and horizontal positioning of the
        elements in the main title page in the order of their vertical
        position from top to bottom when typeset with
        \DescRef{\ThisCommonLabelBase.cmd.maketitle}}
      [l]
      \setlength{\tabcolsep}{.85\tabcolsep}% Umbruchoptimierung
      \begin{tabular}[t]{llll}
        \toprule
        Element    & Command            & Font               & Orientation     \\
        \midrule
        Title head & \Macro{titlehead}  & \DescRef{\ThisCommonLabelBase.cmd.usekomafont}\PParameter{titlehead} & justified \\
        Subject    & \Macro{subject}    & \DescRef{\ThisCommonLabelBase.cmd.usekomafont}\PParameter{subject} & centered          \\
        Title      & \Macro{title}      & \DescRef{\ThisCommonLabelBase.cmd.usekomafont}\PParameter{title}\Macro{huge}       & centered          \\
        Subtitle   & \Macro{subtitle}   & \DescRef{\ThisCommonLabelBase.cmd.usekomafont}\PParameter{subtitle}  & centered          \\
        Authors    & \Macro{author}     & \DescRef{\ThisCommonLabelBase.cmd.usekomafont}\PParameter{author}  & centered          \\
        Date       & \Macro{date}       & \DescRef{\ThisCommonLabelBase.cmd.usekomafont}\PParameter{date}  & centered          \\
        Publishers & \Macro{publishers} & \DescRef{\ThisCommonLabelBase.cmd.usekomafont}\PParameter{publishers} & centered          \\
        \bottomrule
      \end{tabular}
    \end{captionbeside}
    \label{tab:maincls.mainTitle}
  \end{table}
}{} %
Please note\textnote{Attention!}, that for the main title \Macro{huge} will be
used after the font switching element
\DescRef{\ThisCommonLabelBase.fontelement.title}\IndexFontElement{title}. So
you cannot change the size of the main title using
\DescRef{\ThisCommonLabelBase.cmd.setkomafont} or
\DescRef{\ThisCommonLabelBase.cmd.addtokomafont}.%

\IfThisCommonFirstRun{\iftrue}{%
  An example for a title page with all elements provided by \KOMAScript{} for
  the main title page is shown in
  \autoref{sec:\ThisCommonFirstLabelBase.titlepage} on
  \PageRefxmpl{\ThisCommonFirstLabelBase.maintitle}.%
  \csname iffalse\endcsname%
}%
  \begin{Example}
    \phantomsection\xmpllabel{maintitle}%
    Assume you are writing a dissertation. The title page should have the
    university's name and address at the top, flush left, and the semester,
    flush right. As usual, a title is to be used, including author and
    delivery date.  The adviser must also be indicated, together with the fact
    that the document is a dissertation. This can be obtained as follows:
\begin{lstcode}
  \documentclass{scrbook}
  \usepackage[english]{babel}
  \begin{document}
  \titlehead{{\Large Unseen University
      \hfill SS~2002\\}
    Higher Analytical Institute\\
    Mythological Rd\\
    34567 Etherworld}
  \subject{Dissertation}
  \title{Digital space simulation with the DSP\,56004}
  \subtitle{Short but sweet?}
  \author{Fuzzy George}
  \date{30. February 2002}
  \publishers{Adviser Prof. John Eccentric Doe}
  \maketitle
  \end{document}
\end{lstcode}
  \end{Example}%
\fi

\begin{Explain}
  A frequent misunderstanding concerns the role of the full title page.  It is
  often erroneously assumed that the cover\Index{cover} or dust cover is
  meant.  Therefore, it is frequently expected that the title page does not
  follow the normal page layout, but has equally large left and right margins.

  However, if one takes a book and opens it, one notices very quickly at least
  one title page under the cover within the so-called inner book.  Precisely
  these title pages are produced by
  \DescRef{\ThisCommonLabelBase.cmd.maketitle}.

  As is the case with the half-title, the full title page belongs to the inner
  book, and therefore should have the same page layout as the rest of the
  document.  A cover is actually something that should be created in a
  separate document. The cover often has a very individual format. It can also
  be designed with the help of a graphics or DTP program. A separate document
  should also be used because the cover will be printed on a different medium,
  possibly cardboard, and possibly with another printer.

  Nevertheless, since \KOMAScript~3.12 the first title page of
  \DescRef{\ThisCommonLabelBase.cmd.maketitle} can be printed as a cover page
  with different margins. For more information about this see the description
  of option
  \OptionValueRef{\ThisCommonLabelBase}{titlepage}{firstiscover}%
  \IndexOption{titlepage~=\PValue{firstiscover}} on
  \DescPageRef{\ThisCommonLabelBase.option.titlepage}.
\end{Explain}
%
\EndIndexGroup


\begin{Declaration}
  \Macro{uppertitleback}\Parameter{titlebackhead}%
  \Macro{lowertitleback}\Parameter{titlebackfoot}
\end{Declaration}%
With the standard classes, the back of the title page of a double-side print
is left empty.  However, with {\KOMAScript} the back of the full title page
can be used for other information. Exactly two elements which the user can
freely format are recognized: \PName
{titlebackhead}\Index{title>back}\Index{title>rear side}\Index{title>flipside}
and \PName {titlebackfoot}. The head can reach up to the foot and vice
versa. \iffree{If one takes this manual as an example, the exclusion of
  liability was set with the help of the \Macro{uppertitleback} command.}{The
  publishers information of this book. e.\,g., could have been set either with
  \Macro{uppertitleback} or \Macro{lowertitleback}.}%
%
\EndIndexGroup


\begin{Declaration}
  \Macro{dedication}\Parameter{dedication}
\end{Declaration}%
\BeginIndexGroup
\BeginIndex{FontElement}{dedication}\LabelFontElement{dedication}%
{\KOMAScript} provides a page for dedications. The
dedication\Index{dedication} is centered and uses a slightly larger type size
given by the font of the homonymous element. The font can be changed using
command \DescRef{\ThisCommonLabelBase.cmd.setkomafont} or
\DescRef{\ThisCommonLabelBase.cmd.addtokomafont} (see
\autoref{sec:\ThisCommonLabelBase.textmarkup},
\autopageref{sec:\ThisCommonLabelBase.textmarkup}). The back is empty like
the back page of the half-title.  The dedication page is produced by
\DescRef{\ThisCommonLabelBase.cmd.maketitle} and must therefore be defined
before this command is issued.%
\EndIndexGroup

\IfThisCommonFirstRun{\iftrue}{%
  An example with all title pages provided by \KOMAScript{} is shown in
  \autoref{sec:\ThisCommonFirstLabelBase.titlepage} on
  \PageRefxmpl{\ThisCommonFirstLabelBase.fulltitle}.%
  \csname iffalse\endcsname%
}%
  \begin{Example}
    \phantomsection\xmpllabel{fulltitle}%
    This time assume that you have written a poetry book and you want to
    dedicate it to your wife. A solution would look like this:
\begin{lstcode}
  \documentclass{scrbook}
  \usepackage[english]{babel}
  \begin{document}
  \extratitle{\textbf{\Huge In Love}}
  \title{In Love}
  \author{Prince Ironheart}
  \date{1412}
  \lowertitleback{This poem book was set with%
       the help of {\KOMAScript} and {\LaTeX}}
  \uppertitleback{Selfmockery Publishers}
  \dedication{To my treasure hazel-hen\\
    in eternal love\\
    from your dormouse.}
  \maketitle
  \end{document}
\end{lstcode}
    Please use your own favorite pet names.
  \end{Example}%
\fi%
\EndIndexGroup
%
\EndIndexGroup
%
\EndIndexGroup

%%% Local Variables:
%%% mode: latex
%%% coding: us-ascii
%%% TeX-master: "../guide"
%%% End:
