% ======================================================================
% common-lists.tex
% Copyright (c) Markus Kohm, 2001-2021
%
% This file is part of the LaTeX2e KOMA-Script bundle.
%
% This work may be distributed and/or modified under the conditions of
% the LaTeX Project Public License, version 1.3c of the license.
% The latest version of this license is in
%   http://www.latex-project.org/lppl.txt
% and version 1.3c or later is part of all distributions of LaTeX 
% version 2005/12/01 or later and of this work.
%
% This work has the LPPL maintenance status "author-maintained".
%
% The Current Maintainer and author of this work is Markus Kohm.
%
% This work consists of all files listed in manifest.txt.
% ----------------------------------------------------------------------
% common-lists.tex
% Copyright (c) Markus Kohm, 2001-2021
%
% Dieses Werk darf nach den Bedingungen der LaTeX Project Public Lizenz,
% Version 1.3c, verteilt und/oder veraendert werden.
% Die neuste Version dieser Lizenz ist
%   http://www.latex-project.org/lppl.txt
% und Version 1.3c ist Teil aller Verteilungen von LaTeX
% Version 2005/12/01 oder spaeter und dieses Werks.
%
% Dieses Werk hat den LPPL-Verwaltungs-Status "author-maintained"
% (allein durch den Autor verwaltet).
%
% Der Aktuelle Verwalter und Autor dieses Werkes ist Markus Kohm.
% 
% Dieses Werk besteht aus den in manifest.txt aufgefuehrten Dateien.
% ======================================================================
%
% Paragraphs that are common for several chapters of the KOMA-Script guide
% Maintained by Markus Kohm
%
% ----------------------------------------------------------------------
%
% Absaetze, die mehreren Kapiteln der KOMA-Script-Anleitung gemeinsam sind
% Verwaltet von Markus Kohm
%
% ======================================================================

\KOMAProvidesFile{common-lists.tex}
                 [$Date$
                  KOMA-Script guide (common paragraphs)]
\translator{Gernot Hassenpflug\and Markus Kohm\and Krickette Murabayashi\and
  Karl Hagen}

% Date of the translated German file: 2021-02-15

\section{Lists}
\seclabel{lists}%
\BeginIndexGroup
\BeginIndex{}{lists}%

\IfThisCommonFirstRun{}{%
  The information in \autoref{sec:\ThisCommonFirstLabelBase.lists} applies
  equally to this chapter. So if you have already read and understood
  \autoref{sec:\ThisCommonFirstLabelBase.lists}, you can skip ahead to
  \autoref{sec:\ThisCommonLabelBase.lists.next},
  \autopageref{sec:\ThisCommonLabelBase.lists.next}.%
  \IfThisCommonLabelBaseOneOf{scrextend,scrlttr2}{ %
    \IfThisCommonLabelBase{scrlttr2}{%
      The \Package{scrletter}\OnlyAt{\Package{scrletter}} package does not
      define any list environments itself but leaves them to the class
      used. If this is not a \KOMAScript{} class, it will load
      \hyperref[cha:scrextend]{\Package{scrextend}}\IndexPackage{scrextend}%
      \important{\hyperref[cha:scrextend]{\Package{scrextend}}}. %
    }{}%
    However, the \Package{scrextend} package only defines the
    \DescRef{\ThisCommonLabelBase.env.labeling},
    \DescRef{\ThisCommonLabelBase.env.addmargin}, and
    \DescRef{\ThisCommonLabelBase.env.addmargin*} environments. All other list
    environments are left to the responsibility of the class used.%
  }{}%
}

\IfThisCommonLabelBase{scrextend}{}{%
  Both {\LaTeX} and the standard classes\textnote{\KOMAScript{} vs. standard
    classes} offer different environments for lists. Naturally, {\KOMAScript}
  also offers all these environments, though slightly modified or extended in
  some cases. In general, all lists\,---\,even those of different
  kinds\,---\,can be nested up to four levels deep. From a typographical view,
  anything more would make no sense, as lists of more than three levels cannot
  easily be apprehended. In such cases, I recommend\textnote{Hint!} that you
  split a large list into several smaller ones.%
}

\IfThisCommonFirstRun{}{%
  Because lists are standard elements of \LaTeX{}, examples have been omitted
  in this section. Nevertheless, you can find examples either in
  \autoref{sec:\ThisCommonFirstLabelBase.lists},
  \autopageref{sec:\ThisCommonFirstLabelBase.lists} or in any \LaTeX{}
  tutorial.%
}

\IfThisCommonLabelBase{scrextend}{\iffalse}{\csname iftrue\endcsname}
  \begin{Declaration}
    \begin{Environment}{itemize}
      \begin{Body}
        \Macro{item} \dots
        \BodyDots
      \end{Body}
    \end{Environment}
    \Macro{labelitemi}
    \Macro{labelitemii}
    \Macro{labelitemiii}
    \Macro{labelitemiv}
  \end{Declaration}%
  \IfThisCommonLabelBase{scrlttr2}{\OnlyAt{\Class{scrlttr2}}}{}%
  The simplest form of a list is the itemized list\textnote{itemized list},
  \Environment{itemize}. %
  \iffalse % Umbruckoptimierungstext
  The users of a certain disliked word processing package often refer to this
  form of a list as \emph{bullet points}. Presumably, they cannot imagine
  that, depending on the level, a symbol other than a large dot could be used
  to introduce each point. %
  \fi%
  Depending on the level, \KOMAScript{} classes use the following marks:
  ``{\labelitemi}'', ``{\labelitemii}'', ``{\labelitemiii}'', and
  ``{\labelitemiv}''. The definition of these symbols is specified in the
  macros \Macro{labelitemi}, \Macro{labelitemii}, \Macro{labelitemiii}, and
  \Macro{labelitemiv}, all of which you can redefine using
  \Macro{renewcommand}.
  \BeginIndex{FontElement}{itemizelabel}\LabelFontElement{itemizelabel}%
  \BeginIndex{FontElement}{labelitemi}\LabelFontElement{labelitemi}%
  \BeginIndex{FontElement}{labelitemii}\LabelFontElement{labelitemii}%
  \BeginIndex{FontElement}{labelitemiii}\LabelFontElement{labelitemiii}%
  \BeginIndex{FontElement}{labelitemiv}\LabelFontElement{labelitemiv}%
  With the \KOMAScript{} classes the
  fonts\Index{font>style}\ChangedAt{v3.33}{\Class{scrbook}\and
    \Class{scrreprt}\and \Class{scrartcl}} used to format the symbols for the
  different levels can be changed using
  \DescRef{\ThisCommonLabelBase.cmd.setkomafont} and
  \DescRef{\ThisCommonLabelBase.cmd.addtokomafont} (see
  \autoref{sec:\ThisCommonLabelBase.textmarkup},
  \DescPageRef{\ThisCommonLabelBase.cmd.setkomafont}) for the elements
  \FontElement{labelitemi}\important{\FontElement{labelitemi}},
  \FontElement{labelitemii}\important{\FontElement{labelitemii}},
  \FontElement{labelitemiii}\important{\FontElement{labelitemiii}} and
  \FontElement{labelitemiv}\important{\FontElement{labelitemiv}}. By default
  these all use the font setting for element
  \FontElement{itemizelabel}\important{\FontElement{itemizelabel}}. Only
  element \FontElement{labelitemii} additionally uses \Macro{bfseries}. The
  default of \FontElement{itemizelabel} itself is \Macro{normalfont}.  Every
  item is introduced with \Macro{item}.%
  \IfThisCommonFirstRun{\iftrue}{\csname iffalse\endcsname}
    \begin{Example}
      \phantomsection\xmpllabel{env.itemize}%
      You have a simple list which is nested in several levels. You write,
      for example:
\begin{lstcode}
  \minisec{Vehicles in the game}
  \begin{itemize}
    \item aeroplanes
    \begin{itemize}
      \item biplane
      \item transport planes
      \begin{itemize}
        \item single-engine
        \begin{itemize}
          \item jet propelled
          \item propeller driven
        \end{itemize}
        \item twin-engine
        \begin{itemize}
	      \item jet propelled
		  \item propeller driven
		\end{itemize}
      \end{itemize}
      \item helicopters
    \end{itemize}
    \item motorcycles
    \item automobiles
    \begin{itemize}
      \item racing cars
      \item passenger cars
      \item lorries
    \end{itemize}
    \item bicycles
  \end{itemize}
\end{lstcode}
      As output you get:
      \begin{ShowOutput}[\baselineskip]
        \minisec{Vehicles in the game}
        \begin{itemize}
        \item aeroplanes
          \begin{itemize}
          \item biplanes
          \item transport planes
            \begin{itemize}
            \item single-engine
              \begin{itemize}
              \item jet-propelled
              \item propeller-driven
              \end{itemize}
            \item twin-engine
              \begin{itemize}
              \item jet propelled
              \item propeller driven
              \end{itemize}
            \end{itemize}
          \item helicopters
          \end{itemize}
        \item motorcycles
          % \begin{itemize}
          % \item historically accurate
          % \item futuristic, not real
          % \end{itemize}
        \item automobiles
          \begin{itemize}
          \item racing cars
          \item passenger cars
          \item lorries
          \end{itemize}
        \item bicycles
        \end{itemize}
      \end{ShowOutput}
    \end{Example}
  \fi
  % 
  \EndIndexGroup


  \begin{Declaration}
    \begin{Environment}{enumerate}\labelsuffix[enumerate]
      \begin{Body}
        \Macro{item} \dots
        \BodyDots
      \end{Body}
    \end{Environment}
    \Macro{theenumi}%
    \Macro{theenumii}%
    \Macro{theenumiii}%
    \Macro{theenumiv}%
    \Macro{labelenumi}%
    \Macro{labelenumii}%
    \Macro{labelenumiii}%
    \Macro{labelenumiv}
  \end{Declaration}%
  \IfThisCommonLabelBase{scrlttr2}{\OnlyAt{\Class{scrlttr2}}}{}The numbered
  list\textnote{numbered list} is also very common and already provided by the
  {\LaTeX} kernel. The numbering\Index{numbering} differs according to the
  level, with Arabic numbers, small letters, small Roman numerals, and capital
  letters, respectively. The style of numbering is defined with the macros
  \Macro{theenumi} down to \Macro{theenumiv}. The output format is determined
  by the macros \Macro{labelenumi} to \Macro{labelenumiv}. While the small
  letter of the second level is followed by a right parenthesis, the values of
  all other levels are followed by a dot. Every item is introduced with
  \Macro{item}.%
  \IfThisCommonFirstRun{\iftrue}{\csname iffalse\endcsname}
    \begin{Example}
      \phantomsection\xmpllabel{env.enumerate}%
      Let's shorten the previous example, using an 
      \DescRef{\ThisCommonLabelBase.env.itemize} environment instead of the
      \Environment{enumerate} environment:
      \begin{ShowOutput}[\baselineskip]
        \minisec{Vehicles in the game}
        \begin{enumerate}
        \item aeroplanes
          \begin{enumerate}
          \item biplanes
          \item transport planes
            \begin{enumerate}
            \item single-engine
              \begin{enumerate}
              \item jet-propelled\label{xmp:maincls.jets}
              \item propeller-driven
              \end{enumerate}
            \item twin-engine
            \end{enumerate}
          % \item helicopters
          \end{enumerate}
          \item motorcycles
          \begin{enumerate}
             \item historically accurate
             \item futuristic, not real
          \end{enumerate}
        %\item automobiles
        %  \begin{enumerate}
        %  \item racing cars
        %  \item private cars
        %  \item lorries
        %  \end{enumerate}
        %\item bicycles
        \end{enumerate}
      \end{ShowOutput}
      Within the list, you can set labels in the normal way with 
      \Macro{label} and then reference then with \Macro{ref}.
      In the example above, a label was set after the jet-propelled,
      single-engine transport planes with
      ``\Macro{label}\PParameter{xmp:jets}''. The \Macro{ref} value is then
      ``\ref{xmp:maincls.jets}''.
    \end{Example}
  \fi%
  % 
  \EndIndexGroup


  \begin{Declaration}
    \begin{Environment}{description}\labelsuffix[description]
      \begin{Body}
        \Macro{item}\OParameter{keyword} \dots
        \BodyDots
      \end{Body}
    \end{Environment}
  \end{Declaration}%
  \IfThisCommonLabelBase{scrlttr2}{\OnlyAt{\Class{scrlttr2}}}{}Another list
  form is the description list\textnote{description list}. It primarily serves
  to describe individual items or keywords. The item itself is specified as an
  optional parameter in \Macro{item}. %
  \BeginIndex{FontElement}{descriptionlabel}%
  \LabelFontElement{descriptionlabel}%
  The font\Index{font>style}\ChangedAt{v2.8p}{%
    \Class{scrbook}\and \Class{scrreprt}\and \Class{scrartcl}}%
  \ used to format the keyword can be changed with the
  \DescRef{\ThisCommonLabelBase.cmd.setkomafont} and
  \DescRef{\ThisCommonLabelBase.cmd.addtokomafont} commands (see
  \autoref{sec:\ThisCommonLabelBase.textmarkup},
  \DescPageRef{\ThisCommonLabelBase.cmd.setkomafont}) for the 
  \FontElement{descriptionlabel}\important{\FontElement{descriptionlabel}}
  element (see \autoref{tab:\ThisCommonLabelBase.fontelements},
  \autopageref{tab:\ThisCommonLabelBase.fontelements}). The default is
  \Macro{sffamily}\linebreak[1]\Macro{bfseries}.%
  \IfThisCommonFirstRun{\iftrue}{\csname iffalse\endcsname}
    \begin{Example}
      \phantomsection\xmpllabel{env.description}%
      You want the keywords to be printed bold and in the normal font instead 
      of bold and sans serif. Using
\begin{lstcode}
  \setkomafont{descriptionlabel}{\normalfont\bfseries}
\end{lstcode}
      you redefine the font accordingly.

      An example for a description list is the output of the page styles
      listed in \autoref{sec:maincls.pagestyle}. The (abbreviated) source is:
\begin{lstcode}
  \begin{description}
    \item[empty] is the page style without any header or footer.
    \item[plain] is the page style without headings.
    \item[headings] is the page style with running headings.
    \item[myheadings] is the page style for manual headings.
  \end{description}
\end{lstcode}
      This gives:
      \begin{ShowOutput}
        \begin{description}
        \item[empty] is the page style without any header or footer.
        \item[plain] is the page style without headings.
        \item[headings] is the page style with running headings.
        \item[myheadings] is the page style for manual headings.
        \end{description}
      \end{ShowOutput}
    \end{Example}
  \fi%
  % 
  \EndIndexGroup%
\fi

\begin{Declaration}
  \begin{Environment}{labeling}\OParameter{delimiter}
    \Parameter{widest pattern}
    \labelsuffix[labeling]
    \begin{Body}
      \Macro{item}\OParameter{keyword}\dots
      \BodyDots
    \end{Body}
  \end{Environment}
\end{Declaration}%
Another form of description list\textnote{alternative description list} is
only available in the {\KOMAScript} classes%
\IfThisCommonLabelBase{scrextend}{ and \Package{scrextend} }{%
  \IfThisCommonLabelBase{scrlttr2}{ and
    \hyperref[cha:scrextend]{\Package{scrextend}}}{}%
}%
: the \Environment{labeling} environment. Unlike
\IfThisCommonLabelBase{scrextend}{%
  \DescRef{\ThisCommonFirstLabelBase.env.description}%
}{%
  the \DescRef{\ThisCommonLabelBase.env.description} described above%
}, you can specify a pattern for \Environment{labeling} whose length
determines the indentation of all items. Furthermore, you can put an optional
\PName{delimiter} between the item and its description. %
\BeginIndexGroup
\BeginIndex{FontElement}{labelinglabel}\LabelFontElement{labelinglabel}%
\BeginIndex{FontElement}{labelingseparator}%
\LabelFontElement{labelingseparator}%
The font\Index{font>style}%
\IfThisCommonLabelBase{maincls}{%
  \ChangedAt{v3.02}{\Class{scrbook}\and \Class{scrreprt}\and
    \Class{scrartcl}}%
}{%
  \IfThisCommonLabelBase{scrlttr2}{%
    \ChangedAt{v3.02}{\Class{scrlttr2}}%
  }{%
    \IfThisCommonLabelBase{scrextend}{%
      \ChangedAt{v3.02}{\Package{scrextend}}%
    }{\InternalCommonFileUsageError}%
  }%
} used to format the item and the separator can be changed with the 
\DescRef{\ThisCommonLabelBase.cmd.setkomafont} and
\DescRef{\ThisCommonLabelBase.cmd.addtokomafont} commands (see
\autoref{sec:\ThisCommonLabelBase.textmarkup},
\DescPageRef{\ThisCommonLabelBase.cmd.setkomafont}) for the element
\FontElement{labelinglabel} and \FontElement{labelingseparator} (see
\autoref{tab:\ThisCommonLabelBase.fontelements},
\autopageref{tab:\ThisCommonLabelBase.fontelements}).
\IfThisCommonFirstRun{\iftrue}{\par\csname iffalse\endcsname}
  \begin{Example}
    \phantomsection\xmpllabel{env.labeling}%
    \IfThisCommonLabelBase{scrextend}{%
      A small example of a list like this can be written as follows:%
    }{%
      Slightly changing the example from the
      \DescRef{\ThisCommonLabelBase.env.description} environment, we could
      write the following:%
    }%
\begin{lstcode}
  \setkomafont{labelinglabel}{\ttfamily}
  \setkomafont{labelingseparator}{\normalfont}
  \begin{labeling}[~--]{myheadings}
    \item[empty]
      Page style without header or footer
    \item[plain]
      Page style for chapter beginnings without headings
    \item[headings]
      Page style for running headings
    \item[myheadings]
      Page style for manual headings
  \end{labeling}
\end{lstcode}
    The result is this:
    \begin{ShowOutput}
      \setkomafont{labelinglabel}{\ttfamily}
      \setkomafont{labelingseparator}{\normalfont}
      \begin{labeling}[~--]{myheadings}
      \item[empty]
        Page style without header or footer
      \item[plain]
        Page style for chapter beginnings without headings
      \item[headings]
        Page style for running headings
      \item[myheadings]
        Page style for manual headings
      \end{labeling}
    \end{ShowOutput}
    As this example shows, you can set a font-changing command in the usual
    way. But if you do not want the font of the separator to be changed in the
    same way as the font of the label, you have to set the font of the
    separator as well.
  \end{Example}
\fi%
\EndIndexGroup
Originally, this environment was implemented for things like ``Premise,
Evidence, Proof'', or ``Given, Find, Solution'' that are often used in
lecture handouts.  These days, however, the environment has very different
applications. For example, the environment for examples in this guide was
defined with the \Environment{labeling} environment.%
%
\EndIndexGroup


\IfThisCommonLabelBase{scrextend}{\iffalse}{\csname iftrue\endcsname}
  \begin{Declaration}
    \begin{Environment}{verse}\end{Environment}
  \end{Declaration}%
  \IfThisCommonLabelBase{scrlttr2}{\OnlyAt{\Class{scrlttr2}}}{} The
  \Environment{verse} environment\textnote{verse} is not normally perceived
  as a list environment because you do not work with \Macro{item} commands.
  Instead, fixed line breaks are used within the \Environment{flushleft}
  environment. Internally, however, both the standard classes as well as
  {\KOMAScript} implement it as a list environment.

  In general, the \Environment{verse} environment is used for
  poetry\Index{poetry}.  Lines are indented both left and right. Individual
  lines of verse are ended by a fixed line break: \verb|\\|. Verses are set as
  paragraphs, separated by an empty line. Often also
  \Macro{medskip}\IndexCmd{medskip} or \Macro{bigskip}\IndexCmd{bigskip} is
  used instead. To avoid a page break at the end of a line of verse you can,
  as usual, insert \verb|\\*| instead of \verb|\\|.
  \IfThisCommonFirstRun{\iftrue}{\csname iffalse\endcsname}
    \begin{Example}
      \phantomsection\xmpllabel{env.verse}%
      \iffalse
        As an example, the first lines of ``Little Red Riding Hood and the
        Wolf'' by Roald Dahl:
\begin{lstcode}
  \begin{verse}
    As soon as Wolf began to feel\\*
    that he would like a decent meal,\\*
    He went and knocked on Grandma's door.\\*
    When Grandma opened it, she saw\\*
    The sharp white teeth, the horrid grin,\\*
    And Wolfie said, `May I come in?'
  \end{verse}
\end{lstcode}
        The result is as follows:
        \begin{ShowOutput}
          \begin{verse}
            As soon as Wolf began to feel\\*
            That he would like a decent meal,\\*
            He went and knocked on Grandma's door.\\*
            When Grandma opened it, she saw\\*
            The sharp white teeth, the horrid grin,\\*
            And Wolfie said, `May I come in?'
          \end{verse}
        \end{ShowOutput}
      \else
        As an example, Emma Lazarus's sonnet from the pedestal of Liberty
        Enlightening the World\footnote{The lines from Roald Dahl's poem
          ``Little Red Riding Hood and the Wolf'', which was used in former
          releases, has been replaced, because in these times certain
          politicians around the world really seem to need ``The New
          Colossus'' as urgent reminder.}:
\begin{lstcode}
  \begin{verse}
    Not like the brazen giant of Greek fame\\*
    With conquering limbs astride from land to land\\*
    Here at our sea-washed, sunset gates shall stand\\*
    A mighty woman with a torch, whose flame\\*
    Is the imprisoned lightning, and her name\\*
    Mother of Exiles. From her beacon-hand\\*
    Glows world-wide welcome; her mild eyes command\\*
    The air-bridged harbor that twin cities frame.\\*
    ``Keep, ancient lands, your storied pomp!'' cries she\\*
    With silent lips. ``Give me your tired, your poor,\\*
    Your huddled masses yearning to breathe free,\\*
    The wretched refuse of your teeming shore.\\*
    Send these, the homeless, tempest-tossed to me:\\*
    I lift my lamp beside the golden door.''
  \end{verse}
\end{lstcode}
        The result is as follows:
        \begin{ShowOutput}
          \begin{verse}
            Not like the brazen giant of Greek fame\\*
            With conquering limbs astride from land to land\\*
            Here at our sea-washed, sunset gates shall stand\\*
            A mighty woman with a torch, whose flame\\*
            Is the imprisoned lightning, and her name\\*
            Mother of Exiles. From her beacon-hand\\*
            Glows world-wide welcome; her mild eyes command\\*
            The air-bridged harbor that twin cities frame.\\*
            ``Keep, ancient lands, your storied pomp!'' cries she\\*
            With silent lips. ``Give me your tired, your poor,\\*
            Your huddled masses yearning to breathe free,\\*
            The wretched refuse of your teeming shore.\\*
            Send these, the homeless, tempest-tossed to me:\\*
            I lift my lamp beside the golden door.''
          \end{verse}
        \end{ShowOutput}
      \fi
      However, if you have very long lines of verse where a line 
      break occurs within a line of verse:
\begin{lstcode}
  \begin{verse}
    Both the philosopher and the house-owner
    always have something to repair.\\*
    \bigskip
    Don't trust a man, my son, who tells you
    that he has never lied.
  \end{verse}
\end{lstcode}
      \begin{ShowOutput}
        \begin{verse}
    	  Both the philosopher and the house-owner always have something to
          repair.\\
          \bigskip Don't trust a man, my son, who tells you that he has never
          lied.
        \end{verse}
      \end{ShowOutput}
      in this case \verb|\\*| can not prevent a page break occurring within a
      verse at such a line break. To prevent such a page break, a change of
      \Macro{interlinepenalty}\IndexCmd{interlinepenalty} would have to be
      inserted at the beginning of the environment:
\begin{lstcode}
  \begin{verse}\interlinepenalty 10000
    Both the philosopher and the house-owner
    always have something to repair.\\
    \bigskip
    Don't trust a man, my son, who tells you
    that he has never lied.
  \end{verse}
\end{lstcode}
      \iftrue% Umbruchkorrekturtext
        Here are two sayings that should always be considered when confronted
        with seemingly strange questions about {\LaTeX} or its accompanying
        explanations:
\begin{lstcode}
  \begin{verse}
    A little learning is a dangerous thing.\\*
    Drink deep, or taste not the Pierian Spring;\\
    \bigskip
    Our judgments, like our watches, none\\*
    go just alike, yet each believes his own.
  \end{verse}
\end{lstcode}
        \begin{ShowOutput}
          \iffree{}{\vskip-.8\baselineskip}% Umbruchkorrektur
          \begin{verse}
            A little learning is a dangerous thing.\\*
            Drink deep, or taste not the Pierian Spring;\\
            \bigskip
            Our judgments, like our watches, none\\*
            go just alike, yet each believes his own.
          \end{verse}
        \end{ShowOutput}
      \fi
      Incidentally, \Macro{bigskip} was used in these examples to separate two
      sayings.
    \end{Example}
  \fi
  % 
  \EndIndexGroup

  \iffalse% Umbruchkorrekturvarianten
    \begin{Declaration}
  	  \begin{Environment}{quote}\end{Environment}
  	\end{Declaration}%
  	\IfThisCommonLabelBase{scrlttr2}{\OnlyAt{\Class{scrlttr2}}}{}This is
  	internally also a list environment\textnote{block quote with paragraph
  	spacing} and can be found both in the standard classes and in
  	{\KOMAScript}. The content of the environment is set fully justified.
    The environment is often used to format longer quotes\Index{quotes}.
    Paragraphs within the environment are distinguished with vertical space.%
    \EndIndexGroup
  
    \begin{Declaration}
      \begin{Environment}{quotation}\end{Environment}
    \end{Declaration}%
    \IfThisCommonLabelBase{scrlttr2}{\OnlyAt{\Class{scrlttr2}}}{}This
    environment\textnote{block quote with paragraph indent} is comparable to
    \DescRef{\ThisCommonLabelBase.env.quote}. While
    \DescRef{\ThisCommonLabelBase.env.quote} paragraphs are indicated by
    vertical spacing, \Environment{quotation} indents the first line of each
    paragraph horizontally. This also applies to the first paragraph of a
    \Environment{quotation} environment. If you want to prevent the
    indentation there, you must precede it with the
    \Macro{noindent}\IndexCmd{noindent} command.%
  \else
    \begin{Declaration}
      \begin{Environment}{quote}\end{Environment}
      \begin{Environment}{quotation}\end{Environment}
    \end{Declaration}%
    \IfThisCommonLabelBase{scrlttr2}{\OnlyAt{\Class{scrlttr2}}}{} These two
    environments\textnote{block quotes} are also set internally as list
    environments and can be found in both the standard and the {\KOMAScript}
    classes. Both environments use justified text which is indented on both
    the left and the right side. Often they are used to separate longer
    quotations\Index{quotations} from the main text. The difference between
    the two lies in in the manner in which paragraphs are typeset. While
    \Environment{quote} paragraphs are distinguished by vertical space, in
    \Environment{quotation} paragraphs, the first line is indented. This also
    applies to the first line of a \Environment{quotation}
    environment% Umbruchkorrektur
    \IfThisCommonLabelBase{maincls}{%
      , unless it is preceded by \Macro{noindent}\IndexCmd{noindent}.%
    }{%
      \IfThisCommonLabelBase{scrlttr2}{%
        . If you want to prevent the indentation there, you must precede it
        with the \Macro{noindent} command\IndexCmd{noindent}.%
      }{\InternalCommonFileUsageError}%
    }%
  \fi % Umbruchkorrekturvarianten
  \IfThisCommonFirstRun{\iftrue}{\csname iffalse\endcsname}
    \begin{Example}
      \phantomsection\xmpllabel{env.quote}%
      You want to highlight a short anecdote. You write the following
      \Environment{quotation} environment for this:% 
\begin{lstcode}
  A small example for a short anecdote:
  \begin{quotation}
    The old year was turning brown; the West Wind was
    calling;
        
    Tom caught the beechen leaf in the forest falling.
    ``I've caught the happy day blown me by the breezes!
    Why wait till morrow-year? I'll take it when me pleases.
    This I'll mend my boat and journey as it chances
    west down the withy-stream, following my fancies!''
    
    Little Bird sat on twig. ``Whillo, Tom! I heed you.
    I've a guess, I've a guess where your fancies lead you.
    Shall I go, shall I go, bring him word to meet you?''
  \end{quotation}
\end{lstcode}
      The result is:
      \begin{ShowOutput}
        A small example for a short anecdote:
        \begin{quotation}
          The old year was turning brown; the West Wind was
          calling;

          Tom caught the beechen leaf in the forest falling.
          ``I've caught the happy day blown me by the breezes!
          Why wait till morrow-year? I'll take it when me pleases.
          This I'll mend my boat and journey as it chances
          west down the withy-stream, following my fancies!''

          Little Bird sat on twig. ``Whillo, Tom! I heed you.
          I've a guess, I've a guess where your fancies lead you.
          Shall I go, shall I go, bring him word to meet you?''
        \end{quotation}
      \end{ShowOutput}
      % 
      Using a \Environment{quote} environment instead you get:
      % 
      \begin{ShowOutput}
        A small example for a short anecdote:
        \begin{quote}\setlength{\parskip}{4pt plus 2pt minus 2pt}
          The old year was turning brown; the West Wind was
          calling;

          Tom caught the beechen leaf in the forest falling.
          ``I've caught the happy day blown me by the breezes!
          Why wait till morrow-year? I'll take it when me pleases.
          This I'll mend my boat and journey as it chances
          west down the withy-stream, following my fancies!''

          Little Bird sat on twig. ``Whillo, Tom! I heed you.
          I've a guess, I've a guess where your fancies lead you.
          Shall I go, shall I go, bring him word to meet you?''
        \end{quote}
      \end{ShowOutput}
      % 
    \end{Example}
  \fi
  % 
  \EndIndexGroup
\fi

\begin{Declaration}
  \begin{Environment}{addmargin}
                     \OParameter{left indentation}\Parameter{indentation}
  \end{Environment}
  \begin{Environment}{addmargin*}
                     \OParameter{inner indentation}\Parameter{indentation}
  \end{Environment}
\end{Declaration}
Like \IfThisCommonLabelBase{scrextend}{%
  \DescRef{\ThisCommonFirstLabelBase.env.quote} and
  \DescRef{\ThisCommonFirstLabelBase.env.quotation}, which are available in
  the standard and the \KOMAScript{} classes}{%
  \DescRef{\ThisCommonLabelBase.env.quote} and
  \DescRef{\ThisCommonLabelBase.env.quotation}%
}, the \Environment{addmargin} environment changes the margin\Index{margin}.
However, unlike the first two environments, \Environment{addmargin} lets the
user change the width of the indentation. Apart from this change, this
environment does not change the indentation of the first line nor the vertical
spacing between paragraphs.

If only the obligatory argument \PName{indentation} is given, both the left
and right margin are expanded by this value. If the optional argument
\PName{left indentation} is given as well, then the value \PName{left
  indentation} is used for the left margin instead of \PName{indentation}.

The starred variant \Environment{addmargin*}%
\important{\Environment{addmargin*}} differs from the normal version only in
the two-sided mode. Furthermore, the difference only occurs if the optional
argument \PName{inner indentation} is used. In this case, the value of
\PName{inner indentation} is added to the normal inner indentation. For
right-hand pages this is the left margin; for left-hand pages, the right
margin. Then the value of \PName{indentation} determines the width of the
opposite margin.

Both versions of this environment allow negative values for all parameters.
\IfThisCommonLabelBase{scrextend}{%
  The environment then protrudes into the margin accordingly.%
}{%
  This can be done so that the environment protrudes into the margin.%
}%
\IfThisCommonFirstRun{\iftrue}{\csname iffalse\endcsname}
  \begin{Example}
    \phantomsection\xmpllabel{env.addmargin}%
\begin{lstcode}
  \newenvironment{SourceCodeFrame}{%
    \begin{addmargin*}[1em]{-1em}%
      \begin{minipage}{\linewidth}%
        \rule{\linewidth}{2pt}%
  }{%
      \rule[.25\baselineskip]{\linewidth}{2pt}%
      \end{minipage}%
    \end{addmargin*}%
  }
\end{lstcode}
    If you now put your source code in such an environment, it will show
    up as:
    \begin{ShowOutput}
      \newenvironment{SourceCodeFrame}{%
        \begin{addmargin*}[1em]{-1em}%
          \begin{minipage}{\linewidth}%
            \rule{\linewidth}{2pt}%
          }{%
            \rule[.25\baselineskip]{\linewidth}{2pt}%
          \end{minipage}%
        \end{addmargin*}%
      }
      You define the following environment:
      \begin{SourceCodeFrame}
\begin{lstcode}
\newenvironment{\SourceCodeFrame}{%
  \begin{addmargin*}[1em]{-1em}%
    \begin{minipage}{\linewidth}%
      \rule{\linewidth}{2pt}%
}{%
    \rule[.25\baselineskip]{\linewidth}{2pt}%
    \end{minipage}%
  \end{addmargin*}%
}
\end{lstcode}
      \end{SourceCodeFrame}
      This may be feasible or not. In any case, it shows the usage of this
      environment.
    \end{ShowOutput}
    The optional argument of the \Environment{addmargin*} environment
    makes sure that the inner margin is extended by 1\Unit{em}. In turn
    the outer margin is decreased by 1\Unit{em}. The result is a shift
    by 1\Unit{em} to the outside.  Instead of \PValue{1em}, you can of
    course use a length, for example, \PValue{2\Macro{parindent}}.
  \end{Example}
\fi%

Whether a page is going to be on the left or right side of the book cannot be
determined reliably on the first {\LaTeX} run. For details please refer to
the explanation of the commands
\DescRef{\ThisCommonLabelBase.cmd.Ifthispageodd}
(\autoref{sec:\ThisCommonLabelBase.oddOrEven},
\DescPageRef{\ThisCommonLabelBase.cmd.Ifthispageodd}) and
\iffree{\Macro{ifthispagewasodd}}{%
  \DescRef{maincls-experts.cmd.ifthispagewasodd}}
(\autoref{sec:maincls-experts.addInfos}\iffree{}{,
\DescPageRef{maincls-experts.cmd.Ifthispageodd}}).
\IfThisCommonLabelBase{scrlttr2}{}{%
\begin{Explain}
  The interplay of environments such as lists and paragraphs gives rise to
  frequent questions. Therefore, you can find further explanation in the
  description of the \Option{parskip} option in
  \autoref{sec:maincls-experts.addInfos}\iffree{}{,
  \DescPageRef{maincls-experts.option.parskip}. Also in the expert section, in
  \autoref{sec:maincls-experts.addInfos},
  \DescPageRef{maincls-experts.env.addmargin*}, you can find additional
  information about page breaks inside \Environment{addmargin*}}.%
\end{Explain}}%
%
\EndIndexGroup
%
\EndIndexGroup

%%% Local Variables:
%%% mode: latex
%%% coding: us-ascii
%%% TeX-master: "../guide"
%%% End:
