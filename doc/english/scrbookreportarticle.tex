% ======================================================================
% scrbookreportarticle.tex
% Copyright (c) Markus Kohm, 2001-2019
%
% This file is part of the LaTeX2e KOMA-Script bundle.
%
% This work may be distributed and/or modified under the conditions of
% the LaTeX Project Public License, version 1.3c of the license.
% The latest version of this license is in
%   http://www.latex-project.org/lppl.txt
% and version 1.3c or later is part of all distributions of LaTeX 
% version 2005/12/01 or later and of this work.
%
% This work has the LPPL maintenance status "author-maintained".
%
% The Current Maintainer and author of this work is Markus Kohm.
%
% This work consists of all files listed in manifest.txt.
% ----------------------------------------------------------------------
% scrbookreportarticle.tex
% Copyright (c) Markus Kohm, 2001-2019
%
% Dieses Werk darf nach den Bedingungen der LaTeX Project Public Lizenz,
% Version 1.3c, verteilt und/oder veraendert werden.
% Die neuste Version dieser Lizenz ist
%   http://www.latex-project.org/lppl.txt
% und Version 1.3c ist Teil aller Verteilungen von LaTeX
% Version 2005/12/01 oder spaeter und dieses Werks.
%
% Dieses Werk hat den LPPL-Verwaltungs-Status "author-maintained"
% (allein durch den Autor verwaltet).
%
% Der Aktuelle Verwalter und Autor dieses Werkes ist Markus Kohm.
% 
% Dieses Werk besteht aus den in manifest.txt aufgefuehrten Dateien.
% ======================================================================
%
% Chapter about scrbook, scrreprt, and scrartcl of the KOMA-Script guide
% Maintained by Markus Kohm
%
% ----------------------------------------------------------------------
%
% Kapitel ueber scrbook, scrreprt und scrartcl in der KOMA-Script-Anleitung
% Verwaltet von Markus Kohm
%
% ============================================================================

\KOMAProvidesFile{scrbookreportarticle.tex}
                 [$Date$
                  KOMA-Script guide (chapter: scrbook, scrreprt, scrartcl)]

\translator{Jens-Uwe Morawski\and Gernot Hassenpflug\and Markus Kohm\and
	Krickette Murabayashi\and Jana Schubert\and Jens H\"uhne\and Karl Hagen}

% Date of the translated German file: 2019-11-19

\chapter{The Main Classes: \Class{scrbook}, \Class{scrreprt}, and
	\Class{scrartcl}}
\labelbase{maincls}%
\BeginIndexGroup
\BeginIndex{Class}{scrbook}%
\BeginIndex{Class}{scrreprt}%
\BeginIndex{Class}{scrartcl}%

\AddSeeIndex{instruction}{gen}{\GuidecmdIndexShort}{cmd}%
\AddSeeIndex{macro}{gen}{\GuidecmdIndexShort}{cmd}%
The main classes of the {\KOMAScript} bundle are designed as counterparts to
the standard {\LaTeX} classes. This means that the {\KOMAScript} bundle
contains replacements for the three standard classes,
\Class{book}\IndexClass{book}, \Class{report}\IndexClass{report}, and
\Class{article}\IndexClass{article}. There is also a replacement for the
standard \Class{letter}\IndexClass{letter} class. The document class for
letters is described in a separate chapter because it is fundamentally
different from the three main classes (see \autoref{cha:scrlttr2}).

The simplest way to use a {\KOMAScript} class instead of a standard one is to
substitute the class name in the \verb|\documentclass| command in accordance
with \autoref{tab:\LabelBase.overview}. For example, you can replace
\Macro{documentclass}\PParameter{book} with
\Macro{documentclass}\PParameter{scrbook}. Normally, {\LaTeX} should process
the document without errors, just as before the substitution. The layout,
however, should be different. Additionally, the {\KOMAScript} classes provide
new possibilities and options that are described in the following sections.

\begin{table}
%  \centering
  \KOMAoptions{captions=topbeside}%
  \setcapindent{0pt}%
%  \caption
  \begin{captionbeside}
  [Class correspondence]{\label{tab:\LabelBase.overview}Correspondence between
  	standard classes and {\KOMAScript} classes}
  [l]
  \begin{tabular}[t]{ll}
    \toprule
    standard class  & \KOMAScript{} class \\%& \Script-Stil (\LaTeX2.09)\\
    \midrule
    \Class{article} & \Class{scrartcl}  \\%& \File{script\textunderscore s} \\
    \Class{report}  & \Class{scrreprt}  \\%& \File{script}   \\
    \Class{book}    & \Class{scrbook}   \\%& \File{script}   \\
    \Class{letter}  & \Class{scrlttr2}  \\%& \File{script\textunderscore l} \\
    \bottomrule
  \end{tabular}
  \end{captionbeside}
\end{table}

% TODO: new translation missing
\iffalse% don't show the not yet translated text
Es sei an dieser Stelle jedoch nicht verschwiegen, dass einige Paketautoren
ihre Pakete auf Basis der Implementierung und sogar von internem Code der
Standardklassen entwickeln und dabei keine Ruecksicht auf komplett unabhaengige
Entwicklungen wie die \KOMAScript-Klassen nehmen. In solchen Faellen kann es
beim ersten \LaTeX-Lauf nach der Umstellung durchaus zu Fehlermeldungen oder
zusaezlichen Warnungen kommen. Meist lassen sich diese auf einfache Weise
beheben. Oftmals koennen dazu die erweiterten Moeglichkeiten von \KOMAScript{}
genutzt werden, wodurch das problematische Paket dann vollstaendig
entfaellt. Manchmal kann auch das in \autoref{cha:scrhack} ab Seite
\autopageref{cha:scrhack} dokumentierte \hyperref[cha:scrhack]{Paket
  \Package{scrhack}}\IndexPackage{scrhack} Abhilfe schaffen. Auch der Ersatz
von veralteten Paketen durch aktuelle Nachfolger kann zur Beseitigung
derartiger Probleme beitragen. Teilweise geben sogar die \KOMAScript-Klassen
durch entsprechende Warnungen Hilfestellung bei der Loesung von
Inkompatibilitaeten.
\fi

Let me say something before describing the classes. When beginning to write a
document, you are often unsure which specific options to choose. Some
settings, for instance the choice of paper size, may be fixed in advance. But
even the question of the appropriate page layout could be difficult to answer
initially. On the other hand, these settings should be nearly irrelevant, in
the beginning, to the main business of an author: planning the document
structure, writing the text, preparing figures, tables, lists, index, and
other data. As an author, you should concentrate initially on the content.
When that is done, you can take on the fine points of presentation. In
addition to the choice of options, this includes correcting hyphenation,
optimizing page breaks, and placing tables and figures.


% %%%%%%%%%%%%%%%%%%%%%%%%%%%%%%%%%%%%%%%%%%%%%%%%%%%%%%%%%%%%%%%%%%%%%%

\LoadCommonFile{options}% \section{Early or Late Selection of Options}

\LoadCommonFile{compatibility}% \section{Compatibility with Earlier Versions of KOMA-Script}

\LoadCommonFile{draftmode}% \section{Draft Mode}

\LoadCommonFile{typearea}% \section{Page Layout}

\begin{Declaration}
  \Macro{flushbottom}
  \Macro{raggedbottom}
\end{Declaration}
\begin{Explain}
  In two-sided documents especially, it is preferable to have the same visual
  baseline not only for the first lines of each text area in a two-page spread
  but also for the last lines. If a page consists only of text without
  paragraphs or headings, this is generally the result. But a paragraph
  spacing of half a line would be enough to prevent you from achieving this
  goal if the number of paragraphs on each page of the two-page spread differs
  by an odd number. In this case, at least some of the paragraph distances
  need to be stretched or shrunk to reach the target again. \TeX{} defines
  stretchable and shrinkable distances for this purpose, and \LaTeX{} lets you
  perform this kind of \emph{vertical adjustment}\Index{adjustment}
  automatically.
\end{Explain}

Using two-sided printing with the
\Option{twoside}\IndexOption{twoside}\important{\Option{twoside}} option (see
\autoref{sec:typearea.options}, \DescPageRef{typearea.option.twoside}) or
two-column formatting with the
\Option{twocolumn}\IndexOption{twocolumn}\important{\Option{twocolumn}} option
(see \DescPageRef{typearea.option.twocolumn}) also activates this vertical
adjustment. But\ChangedAt{v3.17}{\Class{scrbook}\and \Class{scrreprt}\and
  \Class{scrartcl}} this does not apply with a compatibility setting for a
\KOMAScript{} version prior to 3.17 (see
\autoref{sec:\LabelBase.compatibilityOptions},
\DescPageRef{\LabelBase.option.version}, option
\DescRef{\LabelBase.option.version}\IndexOption{version}%
\important{\OptionValueRef{\LabelBase}{version}{3.17}}) if you use
\DescRef{\LabelBase.cmd.KOMAoption} or \DescRef{\LabelBase.cmd.KOMAoptions} to
change the setting of these options.

You can also explicitly request vertical adjustment at any time starting with
the current page by using \Macro{flushbottom}. \Macro{raggedbottom} has the
opposite effect, switching off vertical adjustment starting with the current
page. This corresponds to the default for one-sided printing.

By the way, \KOMAScript{} uses a slightly modified method for adjusting the
vertical skip. This has been done to move footnotes to the bottom of the
text area instead of having them close to the last text line used.
\iffree{}{
  You can find more information about this at
  \autoref{sec:maincls-experts.addInfos},
  \DescPageRef{maincls-experts.cmd.footnoterule}.}%
%
\EndIndexGroup
%
\EndIndexGroup


\LoadCommonFile{fontsize}% \section{Choosing the Document Font Size}

\LoadCommonFile{textmarkup}% \section{Text Markup}

\LoadCommonFile{titles}% \section{Document Titles}

\section{Abstract}
\seclabel{abstract}
\BeginIndexGroup
\BeginIndex{}{abstract}%

Particularly with articles, more rarely with reports, there is an abstract, or
summary, directly beneath the title and before the table of contents. When
using an in-page title, this abstract is normally a kind of left- and
right-indented block. In comparison, the abstract appears as a chapter or
section when using title pages.

\begin{Declaration}
  \OptionVName{abstract}{simple switch}
\end{Declaration}%
In\OnlyAt{\Class{scrreprt}\and\Class{scrartcl}} the standard
classes\textnote{\KOMAScript{} vs. standard classes}, the
\DescRef{\LabelBase.env.abstract} environment sets the text ``\abstractname''
centred before the abstract text. This used to be the normal practice. Since
then, reading newspapers has trained us to recognize a suitably highlighted
text at the beginning of an article or report as the abstract. This is even
more true when the text comes before the table of contents. It is also
confusing if, of all things, this title appears small and centred.
{\KOMAScript}\ChangedAt{v3.00}{\Class{scrreprt}\and \Class{scrartcl}} offers
the option to include or exclude the abstract's title with the
\Option{abstract} option. For \PName{simple switch}, you can use any value
from \autoref{tab:truefalseswitch}, \autopageref{tab:truefalseswitch}. The
default for \KOMAScript{} is \PValue{false}.

Books typically use a different kind of summary. There, you usually place an
appropriate chapter at the beginning or the end of the work. This chapter is
often combined with either the introduction or a description of a larger
prospectus. Therefore, the \Class{scrbook} class has no
\DescRef{\LabelBase.env.abstract} environment. A\textnote{Hint!} summary
chapter is also recommended for reports in a wider sense, such as a Master's
thesis or Ph.D. dissertation. See the commands
\DescRef{\LabelBase.cmd.chapter*}\IndexCmd{chapter*},
\DescRef{\LabelBase.cmd.addchap}\IndexCmd{addchap}, and
\DescRef{\LabelBase.cmd.addchap*}\IndexCmd{addchap*} documented in
\autoref{sec:maincls.structure}, from \DescPageRef{\LabelBase.cmd.chapter*}.%
\EndIndexGroup


\begin{Declaration}
  \begin{Environment}{abstract}\end{Environment}
\end{Declaration}%
\OnlyAt{\Class{scrartcl}\and \Class{scrreprt}}%
Some {\LaTeX} classes provide a special environment for this summary: the
\Environment{abstract} environment. This is output directly, so it is not part
of the title created with \DescRef{\LabelBase.cmd.maketitle}.
Please\textnote{Attention!} note that \Environment{abstract} is an
environment, not a command. Whether the abstract has a heading or not is
determined by the \DescRef{\LabelBase.option.abstract} option (see above).

For books, the abstract is usually part of the introduction or a separate
chapter at the end of the document. Therefore \Class{scrbook} does not provide
an \Environment{abstract} environment. When using the \Class{scrreprt} class,
it is definitely worth considering whether to proceed in the same way. See the
commands \DescRef{\LabelBase.cmd.chapter*}\IndexCmd{chapter*} and
\DescRef{\LabelBase.cmd.addchap}\IndexCmd{addchap}, or
\DescRef{\LabelBase.cmd.addchap*} in \autoref{sec:\LabelBase.structure}
from \DescPageRef{\LabelBase.cmd.chapter*} for more on this.

When using an in-page title\Index{title>in-page} (see option
\DescRef{\LabelBase.option.titlepage}, \autoref{sec:\LabelBase.titlepage},
\DescPageRef{\LabelBase.option.titlepage}), the abstract is set internally
using the \DescRef{\LabelBase.env.quotation}\IndexEnv{quotation} environment
(see \autoref{sec:\LabelBase.lists}, \DescPageRef{\LabelBase.env.quotation}).
This way paragraphs will be set with the first line indented. If the first
paragraph of the abstract should not be indented, you can suppress this indent
by using \Macro{noindent}\IndexCmd{noindent}%
\important{\Macro{noindent}} \iffree{just after
  \Macro{begin}\PParameter{abstract}}{at the beginning of the environment}.%
%
\EndIndexGroup
%
\EndIndexGroup


\section{Table of Contents}
\seclabel{toc}
\BeginIndexGroup
\BeginIndex{}{table of contents}

The title and optional abstract are normally followed by a table of contents.
Often you also find additional lists of the floating environments, such as
tables and figures, after the table of contents (see
\autoref{sec:\LabelBase.floats}).

\iffalse%
  Note that in addition to the options documented in this section, the
  commands \DescRef{maincls-experts.cmd.DeclareSectionCommand},
  \DescRef{maincls-experts.cmd.DeclareNewSectionCommand},
  \DescRef{maincls-experts.cmd.RedeclareSectionCommand}, and
  \DescRef{maincls-experts.cmd.ProvideSectionCommand} can also affect the
  table of contents. See \autoref{sec:maincls-experts.sections},
  \DescPageRef{maincls-experts.cmd.DeclareSectionCommand}.%
\else%
  In addition to the options documented in this section, the
  \hyperref[cha:tocbasic]{\Package{tocbasic}}%
  \important{\hyperref[cha:tocbasic]{\Package{tocbasic}}}%
  \IndexPackage{tocbasic} package style selected and configured with
  \DescRef{tocbasic.cmd.DeclareTOCStyleEntry}\IndexCmd{DeclareTOCStyleEntry}%
  \important[O]{\DescRef{tocbasic.cmd.DeclareTOCStyleEntry}} (see
  \DescPageRef{tocbasic.cmd.DeclareTOCStyleEntry}) also has a significant
  impact on the appearance of the table of contents. Similarly, the commands
  \DescRef{maincls-experts.cmd.DeclareSectionCommand}%
  \important[O]{\DescRef{maincls-experts.cmd.DeclareSectionCommand}}%
  \IndexCmd{DeclareSectionCommand},
  \DescRef{maincls-experts.cmd.ProvideSectionCommand}%
  \IndexCmd{ProvideSectionCommand},
  \DescRef{maincls-experts.cmd.DeclareNewSectionCommand}%
  \IndexCmd{DeclareNewSectionCommand} and
  \DescRef{maincls-experts.cmd.RedeclareSectionCommand}%
  \IndexCmd{RedeclareSectionCommand} documented in
  \autoref{sec:maincls-experts.sections},
  \DescPageRef{maincls-experts.cmd.DeclareSectionCommand} can also affect the
  table of contents.%
\fi


\begin{Declaration}
  \OptionVName{toc}{setting}
\end{Declaration}
It is becoming increasingly common to to include lists of tables and figures,
the bibliography, and sometimes even the index in the table of contents. This
is surely related to the recent trend of putting lists of figures and tables
at the end of the document. Both lists are similar to the table of contents in
structure and intention. I'm therefore sceptical of this evolution.
Since\important{\OptionValue{toc}{listof}} it makes no sense to include only
the list of tables or that of figures in the table of contents without the
other, there is only one \PName{setting}\ChangedAt{v3.00}{\Class{scrbook}\and
  \Class{scrreprt}\and \Class{scrartcl}}
\PValue{listof}\IndexOption{toc~=\textKValue{listof}}, which causes entries
for both types of lists to be included. This also includes any lists produced
with version~1.2e or later of the \Package{float} package\IndexPackage{float}
from Version~1.2e (see \cite{package:float}) or \Package{floatrow} (see
\cite{package:floatrow}).  None\important{\OptionValue{toc}{listofnumbered}}
of these lists are generally given a chapter number. If you want to ignore
this principle, use the \PName{setting}
\PValue{listofnumbered}\IndexOption{toc~=\textKValue{listofnumbered}}.

\leavevmode\LabelOptionValue{toc}{index}\nobreak
The\important{\OptionValue{toc}{index}}
\OptionValue{toc}{index}\IndexOption{toc~=\textKValue{index}} option causes an
entry for the index to be included in the table of contents. The index is
unnumbered since it too only includes references to the contents of the other
sectioning levels. Despite\ChangedAt{v3.18}{\Class{scrbook}\and
  \Class{scrreprt}\and \Class{scrartcl}}
\important{\OptionValue{toc}{indexnumbered}} the author's concerns,
\KOMAScript{} does support deviating from this principle with
\OptionValue{toc}{indexnumbered}\IndexOption{toc~=\textKValue{indexnumbered}}.

\leavevmode\LabelOptionValue{toc}{bibliography}\nobreak The bibliography is a
slightly different kind of listing. It does not list the contents of the
present document but refers instead to external sources.
For\important{\OptionValue{toc}{bibliographynumbered}} that reason, it could
be argued that it qualifies as a chapter (or section) and, as such, should be
numbered. The \OptionValue{toc}{bibliographynumbered}%
\IndexOption{toc~=\textKValue{bibliographynumbered}} option has this effect,
and puts the appropriate entry in the table of contents. However, I think that
this reasoning would lead us to consider even a classic, annotated source list
to be a separate chapter. Moreover, the bibliography is ultimately not
something that you wrote yourself.
Therefore\important{\OptionValue{toc}{bibliography}} the bibliography merits,
at best, an unnumbered entry in the table of contents, and you can achieve
this achieved with
\OptionValue{toc}{bibliography}\IndexOption{toc~=\textKValue{bibliography}}.

\leavevmode\LabelOptionValue{toc}{graduated}\nobreak
The table of contents is normally\ChangedAt{v2.8q}{%
  \Class{scrbook}\and \Class{scrreprt}\and \Class{scrartcl}}%
\important{\OptionValue{toc}{graduated}} formatted so that different levels of
sectioning commands have different indentations. The number for each level is
set left-justified in a fixed-width field. This default set-up is selected
with the\ChangedAt{v3.00}{\Class{scrbook}\and \Class{scrreprt}\and
  \Class{scrartcl}}
\OptionValue{toc}{graduated}\IndexOption{toc~=\textKValue{graduated}} option.

\leavevmode\LabelOptionValue{toc}{flat}\nobreak
If the sectioning level which appears in the table of contents is too deep, the
number for that level can be so wide that the space reserved for the number is
insufficient. The German FAQ \cite{DANTE:FAQ} suggests redefining the table of
contents in such a case. \KOMAScript{}\important{\OptionValue{toc}{flat}}
offers an alternative format that avoids the problem completely. If you use
the \OptionValue{toc}{flat}\IndexOption{toc~=\textKValue{flat}} option, no
graduated indentation is applied to the headings of the sectioning levels.
Instead, a table-like organisation is used, where all sectioning numbers and
headings are set in a left-justified column. The space necessary for the
section numbers is thus determined automatically.

You can find an overview of all available values for the \PName{setting} of
\Option{toc}.  in \autoref{tab:\LabelBase.toc}.

\begin{desclist}
  \desccaption[{Available values for the \Option{toc} option}]{%
    Available values for the \Option{toc} option to set the format and contents
    of the table of contents\label{tab:\LabelBase.toc}%
  }{%
    Available values for the \Option{toc} option (\emph{continued})%
  }%
  \entry{\PValue{bibliography}, \PValue{bib}}{%
    The bibliography has an unnumbered entry in the table of contents.%
    \IndexOption{toc~=\textKValue{bibliography}}}%
  \entry{\PValue{bibliographynumbered}, \PValue{bibnumbered},
    \PValue{numberedbibliography}, \PValue{numberedbib}}{%
    The bibliography has a numbered entry in the table of contents.%
    \IndexOption{toc~=\textKValue{bibliographynumbered}}}%
  \entry{\PValue{chapterentrywithdots}, \PValue{chapterentrydotfill}}{%
    \ChangedAt[2014/12]{v3.15}{\Class{scrbook}\and \Class{scrreprt}}%
    The chapter entries for the \Class{scrbook} and \Class{scrreprt} classes
    also use dots to separate the heading text from the page numbers.%
    \IndexOption{toc~=\textKValue{chapterentrywithdots}}}%
  \entry{\PValue{chapterentrywithoutdots}, \PValue{chapterentryfill}}{%
    \ChangedAt{v3.15}{\Class{scrbook}\and \Class{scrreprt}}%
    The chapter entries of the \Class{scrbook} and \Class{scrreprt} classes
    use white space to separate the heading text from the page numbers.  This
    corresponds to the default setting.%
    \IndexOption{toc~=\textKValue{chapterentrywithoutdots}}}%
  \entry{\PValue{flat}, \PValue{left}}{%
    The table of contents is set in table form. The numbers of the headings
    are in the first column, the heading text in the second column, and the
    page number in the third column. The amount of space needed for the
    numbers of the headings is determined by the required amount of space
    detected during the previous \LaTeX{} run.%
    \IndexOption{toc~=\textKValue{flat}}}%
  \entry{\PValue{graduated}, \PValue{indent}, \PValue{indented}}{%
    The table of contents is set in hierarchical form. The amount of space for
    the heading numbers is limited. This corresponds to the default setting.%
    \IndexOption{toc~=\textKValue{graduated}}}%
  \entry{\PValue{indenttextentries}, \PValue{indentunnumbered},
    \PValue{numberline}}{%
    \ChangedAt{v3.12}{\Class{scrbook}\and \Class{scrreprt}\and
      \Class{scrartcl}}%
    The \PValue{numberline} property (see \autoref{sec:tocbasic.toc},
    \DescPageRef{tocbasic.cmd.setuptoc}) is set for the table of contents.  As
    a result, unnumbered entries are left aligned with the text of numbered
    entries of the same level.%
    \IndexOption{toc~=\textKValue{numberline}}}%
  \entry{\PValue{index}, \PValue{idx}}{%
    The index has an unnumbered entry in the table of contents.%
    \IndexOption{toc~=\textKValue{index}}}%
  \entry{\PValue{indexnumbered}, \PValue{idxnumbered}, \PValue{numberedindex},
    \PValue{numberedidx}}{%
    \ChangedAt{v3.18}{\Class{scrbook}\and \Class{scrreprt}\and
      \Class{scrartcl}}%
    The index has a numbered entry in the table of contents.%
    \IndexOption{toc~=\textKValue{index}}}%
  \entry{\PValue{leftaligntextentries}, \PValue{leftalignunnumbered},
    \PValue{nonumberline}}{%
    \ChangedAt{v3.12}{\Class{scrbook}\and \Class{scrreprt}\and
      \Class{scrartcl}}%
    The \PValue{numberline} property (see \autoref{sec:tocbasic.toc},
    \DescPageRef{tocbasic.cmd.setuptoc}) is deleted for the table of contents.
    This places unnumbered entries left-aligned with the number of
    numbered entries of the same level. This corresponds to the default
    setting.%
    \IndexOption{toc~=\textKValue{numberline}}}%
  \pventry{listof}{%
    The lists of floating environments, e.\,g. figures and tables, have
    unnumbered entries in the table of contents.%
    \IndexOption{toc~=\textKValue{listof}}}%
  \entry{\PValue{listofnumbered}, \PValue{numberedlistof}}{%
    The lists of floating environments, e.\,g. figures and tables, have
    numbered entries in the table of contents.%
    \IndexOption{toc~=\textKValue{listofnumbered}}}%
  \entry{\PValue{nobibliography}, \PValue{nobib}}{%
    The bibliography does not have an entry in the table of contents. This
    corresponds to the default setting.%
    \IndexOption{toc~=\textKValue{nobibliography}}}%
  \entry{\PValue{noindex}, \PValue{noidx}}{%
    The index does not have an entry in the table of contents. This
    corresponds to the default setting.%
    \IndexOption{toc~=\textKValue{noindex}}}%
  \pventry{nolistof}{%
    The lists of floating environments, e.\,g. figures and tables, do not
    have entries in the table of contents. This corresponds to the default
    setting.%
    \IndexOption{toc~=\textKValue{nolistof}}}%
  \entry{\PValue{sectionentrywithdots}, \PValue{sectionentrydotfill}}{%
    \ChangedAt[2014/12]{v3.15}{\Class{scrartcl}}%
    The section entries of the \Class{scrartcl} class also use dots to
    separate the heading text from the page numbers.%
    \IndexOption{toc~=\textKValue{sectionentrywithdots}}}%
  \entry{\PValue{sectionentrywithoutdots}, \PValue{sectionentryfill}}{%
    \ChangedAt{v3.15}{\Class{scrartcl}}%
    The section entries of the \Class{scrartcl} class use white space to
    separate the heading text from the page number. This corresponds to the
    default setting.%
    \IndexOption{toc~=\textKValue{sectionentrywithoutdots}}}%
\end{desclist}
%
\EndIndexGroup


\begin{Declaration}
  \OptionVName{chapterentrydots}{simple switch}\\
  \OptionVName{sectionentrydots}{simple switch}
\end{Declaration}
These\ChangedAt[2014/12]{v3.15}{\Class{scrbook}\and \Class{scrreprt}} options
configure a dotted connecting line between the text and page number of the
chapter entries for the \Class{scrbook} and
\Class{scrreprt}\OnlyAt{\Class{scrbook}\and \Class{scrreprt}} classes, or for
the section entries of the \Class{scrartcl}\OnlyAt{\Class{scrartcl}} class, in
the table of contents. For the \PName{simple switch}, you can use any value from
\autoref{tab:truefalseswitch}, \autopageref{tab:truefalseswitch}.
The default is \PValue{false}. It selects an empty gap instead of dots.

\BeginIndex{FontElement}{chapterentrydots}\LabelFontElement{chapterentrydots}%
\BeginIndex{FontElement}{sectionentrydots}\LabelFontElement{sectionentrydots}%
If a dotted line is used, you can change its font using the element
\FontElement{chapterentrydots}%
\important[i]{\FontElement{chapterentrydots}\\
  \FontElement{sectionentrydots}} or \FontElement{sectionentrydots} (see also
\DescRef{\LabelBase.cmd.setkomafont} and
\DescRef{\LabelBase.cmd.addtokomafont}, \autoref{sec:\LabelBase.textmarkup},
\DescPageRef{\LabelBase.cmd.setkomafont}, as well as
\autoref{tab:maincls.fontelements},
\autopageref{tab:maincls.fontelements}). The defaults of the elements are
shown in \autoref{tab:maincls.tocelements}, from
\autopageref{tab:maincls.tocelements}. Note\textnote{Attention!} that the dots
of all entries will be equally spaced only if all dots use the same
font. Because of this the base font is always
\Macro{normalfont}\Macro{normalsize} and only the colour of
\DescRef{\LabelBase.fontelement.chapterentry} or
\DescRef{\LabelBase.fontelement.sectionentry} is also used for the dots.%
\EndIndexGroup


\begin{Declaration}
  \Macro{tableofcontents}
\end{Declaration}%
The table of contents is output by the \Macro{tableofcontents} command. To get
correct values in the table of contents requires at least two {\LaTeX} runs
after every change. The \DescRef{\LabelBase.option.toc} option described above
can also affect the extent and format of the table of contents. After changing
the settings of this option, at least two \LaTeX{} runs are needed again.

Entries for \DescRef{\LabelBase.cmd.chapter}\IndexCmd{chapter} with
\Class{scrbook}\IndexClass{scrbook} and \Class{scrreprt}\IndexClass{scrreprt},
or \DescRef{\LabelBase.cmd.section}\IndexCmd{section} with
\Class{scrartcl}\IndexClass{scrartcl}, and the sectioning level
\DescRef{\LabelBase.cmd.part}\IndexCmd{part} are not indented. Additionally,
there are no dots between the text of this heading and the page number. The
typographical logic for this behaviour is that the font is usually distinct
and appropriate emphasis is desirable.
However\ChangedAt{v3.15}{\Class{scrbook}\and \Class{scrreprt}\and
  \Class{scrartcl}}, you can change this behaviour with the previously
documented options. The table of contents of this \iffree{guide}{book} is
created with the default settings and serves as an example.

\BeginIndex{FontElement}{partentry}\LabelFontElement{partentry}%
\BeginIndex{FontElement}{chapterentry}\LabelFontElement{chapterentry}%
\BeginIndex{FontElement}{sectionentry}\LabelFontElement{sectionentry}%
The\ChangedAt{v2.97c}{\Class{scrbook}\and \Class{scrreprt}\and
  \Class{scrartcl}}\important{\FontElement{partentry}\\
  \FontElement{chapterentry}\\
  \FontElement{sectionentry}} font style of the top two levels in the table of
contents is also affected by the settings for the \FontElement{partentry}
element, as well as by the \FontElement{chapterentry} element for the
\Class{scrbook} and \Class{scrreprt} classes, and by the
\FontElement{sectionentry} element for the \Class{scrartcl} class. 
\BeginIndex{FontElement}{partentrypagenumber}%
\LabelFontElement{partentrypagenumber}%
\LabelFontElement{pagination}%
\BeginIndex{FontElement}{chapterentrypagenumber}%
\LabelFontElement{chapterentrypagenumber}%
\BeginIndex{FontElement}{sectionentrypagenumber}%
\LabelFontElement{sectionentrypagenumber}%
You can set the font style of the page numbers separately from these elements
using \FontElement{partentrypagenumber}%
\important{\FontElement{partentrypagenumber}} and
\FontElement{chapterentrypagenumber}%
\important{\FontElement{chapterentrypagenumber}\\
  \FontElement{sectionentrypagenumber}}\,---\,for \Class{scrbook} and
\Class{scrreprt}\,---\,or \FontElement{sectionentrypagenumber}\,---\,for
\Class{scrartcl}\,---\, (see \DescRef{\LabelBase.cmd.setkomafont} and
\DescRef{\LabelBase.cmd.addtokomafont} in \autoref{sec:\LabelBase.textmarkup},
\DescPageRef{\LabelBase.cmd.setkomafont}, or
\autoref{tab:maincls.fontelements}, \autopageref{tab:maincls.fontelements}).
If\ChangedAt{v3.15}{\Class{scrbook}\and \Class{scrreprt}\and
  \Class{scrartcl}} you use dotted lines connecting the heading entries
(chapter or section depending on the class) to the page numbers using the
\DescRef{\LabelBase.option.toc}%
\IndexOption{toc~=\textKValue{chapterentrywithdots}}%
\IndexOption{toc~=\textKValue{sectionentrywithdots}}
\DescRef{\LabelBase.option.chapterentrydots}%
\IndexOption{chapterentrydots~=\PName{simple switch}} or
\DescRef{\LabelBase.option.sectionentrydots}%
\IndexOption{sectionentrydots~=\PName{simple switch}} option, you can change
their font style using the \DescRef{\LabelBase.fontelement.chapterentrydots}%
\IndexFontElement{chapterentrydots}%
\important{\DescRef{\LabelBase.fontelement.chapterentrydots}\\
  \DescRef{\LabelBase.fontelement.sectionentrydots}} and
\DescRef{\LabelBase.fontelement.sectionentrydots}%
\IndexFontElement{sectionentrydots} elements. The defaults for these
elements are found in \autoref{tab:maincls.tocelements}.
\begin{table}
%  \centering
%  \caption
  \KOMAoptions{captions=topbeside}%
  \setcapindent{0pt}%
  \begin{captionbeside}
    [Default font styles for the elements of the table of contents]
    {\label{tab:maincls.tocelements}%
      Default font styles for the elements of the table of contents}
    [l]
    \setlength{\tabcolsep}{.9\tabcolsep}% Umbruchoptimierung!
  \begin{tabular}[t]{ll}
    \toprule
    Element & Default font style \\
    \midrule
    \FontElement{partentry} &
    \DescRef{\LabelBase.cmd.usekomafont}\PParameter{disposition}\Macro{large} \\
    \FontElement{partentrypagenumber} & \\
    \FontElement{chapterentry} & \DescRef{\LabelBase.cmd.usekomafont}\PParameter{disposition}\\
    \FontElement{chapterentrydots} & \Macro{normalfont} \\
    \FontElement{chapterentrypagenumber} & \\
    \FontElement{sectionentry} & \DescRef{\LabelBase.cmd.usekomafont}\PParameter{disposition} \\
    \FontElement{sectionentrydots} & \Macro{normalfont} \\
    \FontElement{sectionentrypagenumber} & \\
    \bottomrule
  \end{tabular}
  \end{captionbeside}
\end{table}
%
\EndIndexGroup


\begin{Declaration}
  \Counter{tocdepth}
  \Macro{parttocdepth}
  \Macro{sectiontocdepth}
  \Macro{subsectiontocdepth}
  \Macro{subsubsectiontocdepth}
  \Macro{paragraphtocdepth}
  \Macro{subparagraphtocdepth}
\end{Declaration}%
Normally, the sectioning divisions included in the table of contents are all
those from \DescRef{\LabelBase.cmd.part} to
\DescRef{\LabelBase.cmd.subsection} for the \Class{scrbook} and
\Class{scrreprt} classes, or from \DescRef{\LabelBase.cmd.part} to
\DescRef{\LabelBase.cmd.subsubsection} for the \Class{scrartcl} class. Whether
or not to include a sectioning level in the table of contents is controlled by
the \Counter{tocdepth} counter. This has the value -1 for
\DescRef{\LabelBase.cmd.part}, 0 for \DescRef{\LabelBase.cmd.chapter}, and so
on. By incrementing or decrementing the counter, you can choose the lowest
sectioning level to include in the table of contents. Incidentally, the
standard classes work the same way.
Unlike\ChangedAt{v3.15}{\Class{scrbook}\and \Class{scrreprt}\and
	\Class{scrartcl}}\textnote{\KOMAScript{} vs. standard classes} with the
standard classes, with \KOMAScript{} you do not need to remember these values.
\KOMAScript{} defines a \Macro{\PName{level}tocdepth} command for each
sectioning level with the appropriate value which you can use to set
\Counter{tocdepth}.

Please note\textnote{Attention!} that in
\Class{scrartcl}\OnlyAt{\Class{scrartcl}}, the values of \Counter{tocdepth}
and \DescRef{\LabelBase.counter.secnumdepth}\IndexCounter{secnumdepth} (see
\autoref{sec:\LabelBase.structure},
\DescPageRef{\LabelBase.counter.secnumdepth})  for
\DescRef{\LabelBase.cmd.part} are not the same. This behaviour was copied from
the standard \Class{article} class for compatibility. Thus, for example, you
should not use \DescRef{\LabelBase.cmd.partnumdepth}\IndexCmd{partnumdepth} to
set the value of \Counter{tocdepth}.%
\begin{Example}
  Suppose you are preparing an article that uses the sectioning level 
  \DescRef{\LabelBase.cmd.subsubsection}. However, you do not want this 
  sectioning level to appear in the table of contents. The preamble of your
  document might contain the following:
\begin{lstcode}
  \documentclass{scrartcl}
  \setcounter{tocdepth}{\subsectiontocdepth}
\end{lstcode}
  Thus you set the \Counter{tocdepth} counter to the value of the
  \Macro{subsectiontocdepth} command. That value is normally 2, but this way,
  you do not have to remember it.
  
  If instead you simply want to include one less level in the table of
  contents than you normally would, you can simply subtract one from the
  default value of \Counter{tocdepth}:
\begin{lstcode}
  \documentclass{scrartcl}
  \addtocounter{tocdepth}{-1}
\end{lstcode}
  The\textnote{Hint!} value that you need to add to or subtract from
  \Counter{tocdepth} is listed in the table of contents after at least two
  \LaTeX{} runs.
\end{Example}%
\EndIndexGroup
%
\EndIndexGroup


\LoadCommonFile{parmarkup}% \section{Paragraph Markup}

\LoadCommonFile{oddorevenpage}% \section{Detecting Odd or Even Pages}


\section{Headers and Footers Using Predefined Page Styles}
\seclabel{pagestyle}

\BeginIndexGroup
\BeginIndex{}{page>style}%
One of the general characteristics of a document is the page style. In
\LaTeX{} this primarily consists of the contents of headers and footers.


\begin{Declaration}
  \OptionVName{headsepline}{simple switch}
  \OptionVName{footsepline}{simple switch}
\end{Declaration}%
You\ChangedAt{v3.00}{\Class{scrbook}\and \Class{scrreprt}\and
  \Class{scrartcl}} can use these options to specify whether a horizontal rule
appears beneath the header or above the footer. You can use any of the values
for simple switches shown in \autoref{tab:truefalseswitch},
\autopageref{tab:truefalseswitch}. Setting the \Option{headsepline} option to
true or invoking it with no value results in a line beneath the header.
Similarly, activating the \Option{footsepline} option results in a rule above
the footer. Deactivating either option switches off the respective rule.

The\textnote{Attention!} \Option{headsepline} option naturally has no effect
with the \DescRef{\LabelBase.pagestyle.empty} and
\DescRef{\LabelBase.pagestyle.plain} page styles, which are described below,
because these styles explicitly dispense with a header\Index{page>header}.
Typographically, such a line has the effect of making the header appear to be
closer to the text. This does not mean that the header then needs to be moved
farther away from the body of the text\Index{text>body}. Instead, the header
should be considered as belonging to the text body for the purpose of
calculating the type area. \KOMAScript{} takes this into account by passing
the \Option{headsepline} option to the
\hyperref[cha:typearea]{\Package{typearea}}%
\important{\hyperref[cha:typearea]{\Package{typearea}}}\IndexPackage{typearea}
package, which then automatically executes the package option
\DescRef{typearea.option.headinclude}%
\IndexOption{headinclude}\important{\DescRef{typearea.option.headinclude}}
with the same value. The same applies to the footer separation line. Unlike
\Option{headsepline}, the \Option{footsepline} option also affects the
\PValue{plain} page style because \PValue{plain} prints a page number in the
footer.

The options themselves do not automatically recalculate the type area. To
recalculate it, use the \DescRef{typearea.option.DIV} option with the values
\hyperref[desc:typearea.option.DIV.last]{\PValue{last}} or
\hyperref[desc:typearea.option.DIV.current]{\PValue{current}} (see
\DescPageRef{typearea.option.DIV.last}) or the
\DescRef{typearea.cmd.recalctypearea} command (see
\DescPageRef{typearea.cmd.recalctypearea}) in \autoref{cha:typearea}.

The \hyperref[cha:scrlayer-scrpage]{\Package{scrlayer-scrpage}}%
\IndexPackage{scrlayer-scrpage}%
\important{\hyperref[cha:scrlayer-scrpage]{\Package{scrlayer-scrpage}}}
package (see \autoref{cha:scrlayer-scrpage}) offers further possibilities for
adjusting lines in headers and footers.%
%
\EndIndexGroup


\begin{Declaration}
  \Macro{pagestyle}\Parameter{page style}
  \Macro{thispagestyle}\Parameter{local page style}
\end{Declaration}%
There are usually four different page styles:
\begin{description}
\item[{\PageStyle{empty}%
    \BeginIndex[indexmain]{Pagestyle}{empty}\LabelPageStyle{empty}}] is the
  page style with completely empty headers and footers. In {\KOMAScript}
  this is identical to the standard classes.%
\item[{\PageStyle{headings}%
    \BeginIndex[indexmain]{Pagestyle}{headings}\LabelPageStyle{headings}}] is
  the page style with running heads in the header. In this style, headings are
  automatically\Index[indexmain]{running heads>automatic} inserted into the
  header. \OnlyAt{\Class{scrbook}\and\Class{scrreprt}}%
  With the classes \Class{scrbook}\IndexClass{scrbook} and
  \Class{scrreprt}\IndexClass{scrreprt}, the headings of chapters and sections
  are repeated in the header for two-sided printing\,---\,on the outer side
  with {\KOMAScript}\textnote{\KOMAScript{} vs. standard classes}, on the
  inner side with the standard classes. {\KOMAScript} puts the page number on
  the outer side of the footer; the standard classes put it on the inner side
  of the header. In one-sided printing, {\KOMAScript} uses only the chapter
  headings, which are centred in the header, and puts the page numbers centred
  in the footer.
  \OnlyAt{\Class{scrartcl}}\Class{scrartcl}\IndexClass{scrartcl} behaves
  similarly but starts one a level deeper in the sectioning hierarchy, with
  sections and subsections, because the chapter level does not exist in this
  case.

  While the standard classes\textnote{\KOMAScript{} vs. standard classes}
  automatically convert the running heads to upper-case letters,
  {\KOMAScript} uses the capitalisation found in the headings. There are
  several typographical reasons for this. Upper-case letters are actually far
  too massive as a text decoration. If you use them anyway, they should be set
  one point smaller and with slightly tighter spacing. The standard classes do
  not take these points into consideration.

  In addition, the {\KOMAScript} classes support rules below the header and
  above the footer with the \DescRef{\LabelBase.option.headsepline} and
  \DescRef{\LabelBase.option.footsepline} options (see
  \DescPageRef{\LabelBase.option.headsepline}).%
\item[{\PageStyle{myheadings}%
    \BeginIndex[indexmain]{Pagestyle}{myheadings}\LabelPageStyle{myheadings}}]
  mostly corresponds to the \PValue{headings} page style, but the running
  heads are not generated automatically%
  \Index[indexmain]{running heads>manual}\,---\,they have to be defined by the
  user. You can use the \DescRef{\LabelBase.cmd.markboth}\IndexCmd{markboth}
  and \DescRef{\LabelBase.cmd.markright}\IndexCmd{markright}\important{%
    \DescRef{\LabelBase.cmd.markboth}\\
    \DescRef{\LabelBase.cmd.markright}} commands for that purpose (see
  \DescPageRef{\LabelBase.cmd.markright}).
\item[{\PageStyle{plain}%
    \BeginIndex[indexmain]{Pagestyle}{plain}\LabelPageStyle{plain}}] is the
  page style with no running head\Index{running head} and only a page number%
  \Index{page>number} in the footer. The\textnote{\KOMAScript{} vs. standard
  	classes} standard classes always centre this page number in the footer.
  {\KOMAScript} puts the page number on the outer side of the footer in
  two-sided\Index{two-sided} mode. {\KOMAScript} behaves like the
  standard classes in one-sided printing.%
\end{description}

You\important{\Macro{pagestyle}} can set the page style at any time with the
help of the \Macro{pagestyle} command, and this setting takes effect with the
next page that is output. If\textnote{Hint!} you use \Macro{pagestyle} just
before a command that results in an implicit page break and if the new page
style should be used on the resulting new page, a
\DescRef{\LabelBase.cmd.cleardoublepage} just before \Macro{pagestyle} will be
useful. But usually you set the page style only once, at the beginning of the
document or in the preamble.

To\important{\Macro{thispagestyle}} change the page style of the current page
only, use the \Macro{thispagestyle} command. This occurs automatically at some
points in the document. For example, the
\Macro{thispagestyle}\PParameter{\DescRef{\LabelBase.cmd.chapterpagestyle}}
command is issued implicitly on the first page of a chapter.

Note\textnote{Attention!} that when you use the
\hyperref[cha:scrlayer-scrpage]{\Package{scrlayer-scrpage}}%
\important{\hyperref[cha:scrlayer-scrpage]{\Package{scrlayer-scrpage}}}
package, switching between automatic and manual running heads is no longer
accomplished by changing the page styles but with special instructions. You
should not use the \PValue{headings} and \PValue{myheadings} page styles with
this package.

\BeginIndexGroup \BeginIndex[indexother]{}{font>style}%
\BeginIndex{FontElement}{pageheadfoot}\LabelFontElement{pageheadfoot}%
\LabelFontElement{pagehead}%
\BeginIndex{FontElement}{pagefoot}\LabelFontElement{pagefoot}%
\BeginIndex{FontElement}{pagenumber}\LabelFontElement{pagenumber}%
To change the font style used for the header, the footer, or the page
number\ChangedAt{v2.8p}{\Class{scrbook}\and \Class{scrreprt}\and
	\Class{scrartcl}}, use the \DescRef{\LabelBase.cmd.setkomafont} and
\DescRef{\LabelBase.cmd.addtokomafont} commands (see
\autoref{sec:\LabelBase.textmarkup},
\DescPageRef{\LabelBase.cmd.setkomafont}). The same element,
\FontElement{pageheadfoot}\IndexFontElement{pageheadfoot}%
\important{\FontElement{pageheadfoot}}, is used for the header and the footer.
The element for the page number within the header or footer is called
\FontElement{pagenumber}\IndexFontElement{pagenumber}%
\important{\FontElement{pagenumber}}. The
\FontElement{pagefoot}\IndexFontElement{pagefoot}%
\important{\FontElement{pagefoot}} element, which is also provided by the
\KOMAScript{} classes, is used only if you define a  page style with the
\hyperref[cha:scrlayer-scrpage]{\Package{scrlayer-scrpage}}%
\IndexPackage{scrlayer-scrpage} package in which the footer contains text (see
\autoref{cha:scrlayer-scrpage},
\DescPageRef{scrlayer-scrpage.fontelement.pagefoot}).

You can find the default settings in
\autoref{tab:\LabelBase.defaultFontsHeadFoot}.
%
\begin{table}
%  \centering%
%  \caption
  \KOMAoptions{captions=topbeside}%
  \setcapindent{0pt}%
%  \addtokomafont{caption}{\raggedright}%
  \begin{captionbeside}
    [{Default values for page style elements}]
    {\label{tab:maincls.defaultFontsHeadFoot}%
      \hspace{0pt plus 1ex}%
      Default values for page style elements}
    [l]
  \begin{tabular}[t]{ll}
    \toprule
    Element & Default \\
    \midrule
    \FontElement{pagefoot}\IndexFontElement{pagefoot} &
    \\
    \FontElement{pageheadfoot}\IndexFontElement{pageheadfoot} &
    \Macro{normalfont}\Macro{normalcolor}\Macro{slshape} \\
    \FontElement{pagenumber}\IndexFontElement{pagenumber} &
    \Macro{normalfont}\Macro{normalcolor}\\
    \bottomrule
  \end{tabular}
  \end{captionbeside}
\end{table}
%
\begin{Example}
  \leavevmode\phantomsection\xmpllabel{cmd.pagestyle}%
  Suppose you want to set header and footer in a smaller type size
  and in italics. However, the page number should not be set in
  italics but in bold. Apart from the fact that the result will look
  horrible, you can do this as follows:
\begin{lstcode}
  \setkomafont{pageheadfoot}{%
    \normalfont\normalcolor\itshape\small}
  \setkomafont{pagenumber}{\normalfont\bfseries}
\end{lstcode}
  On the other hand, if you only want a smaller font to be used along with the
  default slanted text, you can use the following:
\begin{lstcode}
  \addtokomafont{pagehead}{\small}
\end{lstcode}
  As you can see, the previous example uses the
  \FontElement{pagehead}\important{\FontElement{pagehead}} element. You can
  achieve the same result using \PValue{pageheadfoot} instead (see
  \autoref{tab:\LabelBase.fontelements} on
  \autopageref{tab:\LabelBase.fontelements}).
\end{Example}
It is not possible to use these methods to force upper-case letters to be used
automatically for the running heads. Although you can redefine
\DescRef{tocbasic.cmd.MakeMarkcase}\IndexCmd{MakeMarkcase}, in such cases you
should instead use the
\hyperref[cha:scrlayer-scrpage]{\Package{scrlayer-scrpage}} package (see
\autoref{cha:scrlayer-scrpage},
\DescPageRef{scrlayer-scrpage.option.markcase}).

If\textnote{Hint!} you define your own page styles, the commands
\DescRef{\LabelBase.cmd.usekomafont}\PParameter{pageheadfoot},
\DescRef{\LabelBase.cmd.usekomafont}\PParameter{pagenumber}, and
\DescRef{\LabelBase.cmd.usekomafont}\PParameter{pagefoot} can be useful. In
particular, if you do not use the {\KOMAScript} package
\hyperref[cha:scrlayer-scrpage]{\Package{scrlayer-scrpage}} (see
\autoref{cha:scrlayer-scrpage}) but use, for example, the
\Package{fancyhdr}\IndexPackage{fancyhdr}\important{\Package{fancyhdr}}
package (see \cite{package:fancyhdr}), you can use these commands in your
definitions. In this way you can maintain compatibility with {\KOMAScript} as
much as possible. If you do not use these commands in your own definitions,
changes such as those shown in the previous examples have no effect. The
\hyperref[cha:scrlayer-scrpage]{\Package{scrlayer-scrpage}}%
\IndexPackage{scrlayer-scrpage} package tries to maintain maximum
compatibility as long as, for example,
\iffree{\Macro{thepage}}{\DescRef{\LabelBase-experts.cmd.thepage}}%
\IndexCmd{thepage} is not used directly for the page number rather than the
\iffree{\Macro{pagemark}}{\DescRef{\LabelBase-experts.cmd.pagemark}}%
\IndexCmd{pagemark} which is provided for it.%
\EndIndexGroup
%
\EndIndexGroup


\begin{Declaration}
  \Macro{markboth}\Parameter{left mark}\Parameter{right mark}
  \Macro{markright}\Parameter{right mark}
\end{Declaration}
The \DescRef{\LabelBase.pagestyle.myheadings}%
\important{\DescRef{\LabelBase.pagestyle.myheadings}}%
\IndexPagestyle{myheadings} page style does not set the running head. Instead,
you set it with the help of the \Macro{markboth} and \Macro{markright}
commands. This way the \PName{left mark} will normally be used in the header
of even pages and \PName{right mark} in the header of odd pages. With
one-sided printing, only the \PName{right mark} exists. With the
\hyperref[cha:scrlayer-scrpage]{\Package{scrlayer-scrpage}}%
\IndexPackage{scrlayer-scrpage}%
\important{\hyperref[cha:scrlayer-scrpage]{\Package{scrlayer-scrpage}}}
package, the \DescRef{scrlayer-scrpage.cmd.markleft}%
\IndexCmd{markleft}\important{\DescRef{scrlayer-scrpage.cmd.markleft}} command
is also available.

You can use these commands with other page styles too. However, when combined
with automatic running heads, for example with the
\DescRef{\LabelBase.pagestyle.headings}\IndexPagestyle{headings} page style,
the effect of the commands lasts only until the next time the respective marks
are set automatically.%
%
\EndIndexGroup

\begin{Declaration}
  \Macro{titlepagestyle}
  \Macro{partpagestyle}
  \Macro{chapterpagestyle}
  \Macro{indexpagestyle}
\end{Declaration}%
\Index{title>page style}%
\Index{part>page style}%
\Index{chapter>page style}%
\Index{index>page style}%
On some pages, a different page style is chosen automatically with the help of
the \DescRef{\LabelBase.cmd.thispagestyle} command. Which page style this
actually is, is defined by these four macros, of which \Macro{partpagestyle}
and \Macro{chapterpagestyle}\OnlyAt{\Class{scrbook}\and\Class{scrreprt}} are
found only with classes \Class{scrbook} and \Class{scrreprt}, and not in
\Class{scrartcl}. The default value for all four cases is
\DescRef{\LabelBase.pagestyle.plain}\IndexPagestyle{plain}. You can find the
meaning of these macros in \autoref{tab:specialpagestyles}.
%
\begin{table}
  \centering
  \caption{Macros to set up the page style of special pages}
  \label{tab:specialpagestyles}
  \begin{desctabular}
    \mentry{titlepagestyle}{Page style for a title page when using
      \emph{in-page} titles.}%
    \mentry{partpagestyle}{Page style for pages with
      \DescRef{\LabelBase.cmd.part} titles.}%
    \mentry{chapterpagestyle}{Page style for the first page of a chapter.}%
    \mentry{indexpagestyle}{Page style for the first page of the index.}%
  \end{desctabular}
\end{table}
%
You can redefine the page styles with the \Macro{renewcommand} macro.
\begin{Example}
  Suppose you do not want the pages with a \DescRef{\LabelBase.cmd.part}
  heading to be numbered. You can use the following command in the preamble of
  your document:
\begin{lstcode}
  \renewcommand*{\partpagestyle}{empty}
\end{lstcode}
  As mentioned previously on \DescPageRef{\LabelBase.pagestyle.empty},
  the \PValue{empty} page style is exactly what is required in this
  example. Of course, you can also use a user-defined page style.
	
  Suppose you have defined your own page style for initial chapter pages with
  the \hyperref[cha:scrlayer]{\Package{scrlayer}} (see
  \autoref{sec:scrlayer.pagestyles}) or the
  \hyperref[cha:scrlayer-scrpage]{\Package{scrlayer-scrpage}} package (see
  \autoref{sec:scrlayer-scrpage-experts.pagestyle.pairs}). You have given this
  page style the fitting name of \PValue{chapter}. To actually use this style,
  you must redefine \Macro{chapterpagestyle} in this way:
\begin{lstcode}
  \renewcommand*{\chapterpagestyle}{chapter}
\end{lstcode}
	
  Suppose you do not want the table of contents\Index{table of contents}%
  \textnote{table of contents} of a book to have page numbers. Everything
  after the table of contents, however, should use the \PValue{headings}
  page style, including the \PValue{plain} page style for the first
  page of every chapter. You can use the following:
\begin{lstcode}
  \clearpage
  \pagestyle{empty}
  \renewcommand*{\chapterpagestyle}{empty}
  \tableofcontents
  \clearpage
  \pagestyle{headings}
  \renewcommand*{\chapterpagestyle}{plain}
\end{lstcode}
  You can also keep the redefinition local by using a group. This method has 
  the advantage that you do not need to make any assumptions about the what 
  the previous page style was in order to restore it after your local change:
\begin{lstcode}
  \clearpage
  \begingroup
  \pagestyle{empty}
  \renewcommand*{\chapterpagestyle}{empty}
  \tableofcontents
  \clearpage
  \endgroup
\end{lstcode}
  Note,\important{Attention!} however, that you never should put a numbered 
  sectioning command into a group. Otherwise you may get unpredictable results 
  with commands like \Macro{label}.

  On \DescPageRef{tocbasic.cmd.AfterTOCHead} in \autoref{sec:tocbasic.toc},
  you will discover the \DescRef{tocbasic.cmd.AfterTOCHead} command, which
  makes a solution even easier:
\begin{lstcode}
  \AfterTOCHead[toc]{%
    \thispagestyle{empty}%
    \pagestyle{empty}%
  }
\end{lstcode}%
  This takes advantage of the fact that if there are several
  \DescRef{\LabelBase.cmd.thispagestyle} commands on the same page, the last
  one always wins.
\end{Example}

\begin{Explain}
  You might think that you can put running heads on the first page of a
  chapter simply by using the
\begin{lstcode}
  \renewcommand*{\chapterpagestyle}{headings}
\end{lstcode}
  command. Before you try this, you should read \iffree{}{more about the
    background to \DescRef{maincls-experts.cmd.rightmark} in
    \autoref{sec:maincls-experts.addInfos} on
    \DescPageRef{maincls-experts.cmd.rightmark}. Also, }the remarks on
  \DescRef{scrlayer-scrpage-experts.cmd.rightfirstmark} starting on page
  \DescPageRef{scrlayer-scrpage-experts.cmd.rightfirstmark} in
  \autoref{cha:scrlayer-scrpage-experts}, \autoref{part:forExperts}
  \iffree{}{provide important information}.%
\end{Explain}
%
\EndIndexGroup


\begin{Declaration}
  \Macro{pagenumbering}\Parameter{numbering style}
\end{Declaration}
This command works the same way in {\KOMAScript} as in the standard classes.
Strictly speaking, it is a feature of neither the standard classes nor the
\KOMAScript{} classes but of the {\LaTeX} kernel.  This command is used to
change the \PName{numbering style} of page numbers.

The changes take effect immediately, i.\,e., starting from the page that
contains the command. If necessary, you should first close the current page
with \DescRef{\LabelBase.cmd.clearpage} or better
\DescRef{\LabelBase.cmd.cleardoubleoddpage}%
\important{\DescRef{\LabelBase.cmd.cleardoubleoddpage}}%
\IndexCmd{cleardoubleoddpage}. You can find the available settings for
\PName{numbering style} in \autoref{tab:numberKind}.
%
\begin{table}
  %  \centering
  \KOMAoptions{captions=topbeside}%
  \setcapindent{0pt}%
  %  \caption
  \begin{captionbeside}
    {\label{tab:numberKind}%
      Available numbering styles of page numbers}
  \begin{tabular}[t]{lll}
    \toprule
    numbering style & example & description \\
    \midrule
    \PValue{arabic} & 8 & Arabic numbers \\
    \PValue{roman}  & viii & lower-case Roman numbers \\
    \PValue{Roman}  & VIII & upper-case Roman numbers \\
    \PValue{alph}   & h    & letters \\
    \PValue{Alph}   & H    & capital letters \\
    \bottomrule
  \end{tabular}
  \end{captionbeside}
\end{table}

Calling\textnote{Attention!} \Macro{pagenumbering} always resets the page
number\Index{page>number}. The current page becomes number 1 in the selected
\PName{numbering style}. In order that two-sided documents produce the correct
results on an even page, so that the left-hand page is not missing, you should
always add \DescRef{\LabelBase.cmd.cleardoubleoddpage}%
\important{\DescRef{\LabelBase.cmd.cleardoubleoddpage}}%
\IndexCmd{cleardoubleoddpage} before \Macro{pagenumbering}. The next section
provides more information about potentially inserted blank pages.

\begin{Explain}
  Let me say a word about a common mistake found in various templates
  circulating on the Internet. If you encounter lines like the
  following\,---\,without the initial comment naturally\,---\,this is an
  unmistakable sign that the creator did not read or understand the
  remark above:
\begin{lstcode}
  % Attention! This example contains errors!
  % Please note the explanation in the text!
  \tableofcontents
  \pagenumbering{arabic}
  \setcounter{page}{1}
\end{lstcode}
  Since \DescRef{\LabelBase.cmd.tableofcontents} outputs the table of contents
  but does not automatically issue a page break at the end, the page numbering
  is already changed on the last page of the table of contents. Because it
  lacks a \DescRef{\LabelBase.cmd.cleardoubleoddpage} command before
  \Macro{pagenumbering}, it receives a pagination of the Arabic number 1.
  Additionally, the final line which sets the page numbering to 1 is
  superfluous, since this is already done by \Macro{pagenumbering}.

  Sometimes\,---\,without the initial comment, naturally\,---\, you find:
\begin{lstcode}
  % Attention! This example contains errors!
  % Please note the explanation in the text!
  \tableofcontents
  \pagebreak
  \pagenumbering{arabic}
  \setcounter{page}{1}
\end{lstcode}
  Here the creator tried to solve the problem with the final page of the table
  of contents with the help of \Macro{pagebreak}\IndexCmd{pagebreak}.

  Unfortunately, this solution is not much better. Here there is a page break
  after the last page of the table of contents. This may cause entries on the
  last page of a two-sided document to have excess vertical spacing (see
  \DescRef{\LabelBase.cmd.flushbottom},
  \DescPageRef{\LabelBase.cmd.flushbottom}). \Macro{pagebreak} is clearly
  the wrong command here.

  Furthermore, \Macro{newpage}\IndexCmd{newpage} or
  \DescRef{\LabelBase.cmd.clearpage} would not be sufficient for a two-sided
  document. For example, if the last page of the table of contents had the
  Roman numeral vii, the Arabic numbered right-side page 1 would immediately
  follow the Roman numeral right-side page. A left-side page between the two
  would be missing, which could cause serious problems with later printing. 

  My advice: Avoid using templates that contain errors with respect to such
  simple things. Incidentally, the correct way would be:
\begin{lstcode}
  \tableofcontents
  \cleardoubleoddpage
  \pagenumbering{arabic}
\end{lstcode}
  This also applies if \Class{scrartcl}\OnlyAt{\Class{scrartcl}} uses a class
  that usually does not start a new page after the table of contents. If you
  switch the page numbering, a new right-hand page must be started. If you do
  not want such a change, you should keep the numbering style of pages
  consistent throughout the document without changing it in between.
\iffalse% Umbruchkorrekturtext
  For articles or other short documents, this is generally recommended.
\fi

  It is easier to change the numbering style when using
  \Class{scrbook}\OnlyAt{\Class{scrbook}}. There you have the support of two
  commands, \DescRef{\LabelBase.cmd.frontmatter}%
  \important{\DescRef{\LabelBase.cmd.frontmatter}\\
    \DescRef{\LabelBase.cmd.mainmatter}}%
  \IndexCmd{fontmatter} and \DescRef{\LabelBase.cmd.mainmatter}%
  \IndexCmd{mainmatter}, for the most commonly used switching. For more
  information, please see
  \autoref{sec:maincls.separation},
  \DescPageRef{\LabelBase.cmd.mainmatter}.%
\end{Explain}%
\EndIndexGroup
\EndIndexGroup


\LoadCommonFile{interleafpage}% \section{Interleaf Pages}

\LoadCommonFile{footnotes}% \section{Footnotes}

\section[Book Structure]{Book Structure\protect\OnlyAt{\Class{scrbook}}}
\seclabel{separation}

Sometimes books are loosely divided into \emph{front matter}%
\Index{front matter}, \emph{main matter}\Index{main matter}, and
\emph{back matter}\Index{back matter}. \KOMAScript{} also provides this
capability for \Class{scrbook}.


\begin{Declaration}
  \Macro{frontmatter}
  \Macro{mainmatter}
  \Macro{backmatter}
\end{Declaration}%
The front matter includes all the material which appears before the main
text begins, including title pages, preface, and table of contents. It is
initiated with
\Macro{frontmatter}\important{\Macro{frontmatter}}. In the front matter,
Roman numerals are used for the page numbers, and chapter headings in 
the header are not numbered. However, section headings are numbered
consecutively, starting from chapter 0. This typically does not matter, as the 
front matter is used only for the title pages, table of contents%
\Index{table of contents}, lists of figures\Index{list of>figures}%
\Index{figures>list of} and tables\Index{tables>list of}, and a 
preface\Index{preface} or foreword\Index{foreword}. The preface can thus be 
created as a normal chapter. A preface should be as short as possible and
never divided into sections. The preface thus does not require a deeper level
of structure than the chapter.

If you see things differently and want to use numbered
sections\Index{section>number} in the chapters of the front matter, as of
version~2.97e,\ChangedAt{v2.97e}{\Class{scrbook}}%
\important{\OptionValueRef{\LabelBase}{version}{2.97e}} the section numbering
no longer contains the chapter number. This change only takes effect when the
compatibility option is set to at least version~2.97e (see option
\DescRef{\LabelBase.option.version},
\autoref{sec:\LabelBase.compatibilityOptions},
\DescPageRef{\LabelBase.option.version}). It is explicitly noted that this
creates confusion with chapter numbers! The use of
\DescRef{\LabelBase.cmd.addsec} and \DescRef{\LabelBase.cmd.section*} (see
\autoref{sec:\LabelBase.structure}, \DescPageRef{\LabelBase.cmd.section*} and
\DescPageRef{\LabelBase.cmd.addsec}) are thus, in the author's opinion,
greatly preferable.

As of version~2.97e\ChangedAt{v2.97e}{\Class{scrbook}} the numbering of
floating environments, such as tables\Index{table>number} and
figures\Index{figure>number}, and equation numbers\Index{equation>number} in
the front matter also contains no chapter-number part. To take effect, this
too requires the corresponding compatibility setting (see the
\DescRef{\LabelBase.option.version} option,
\autoref{sec:\LabelBase.compatibilityOptions},
\DescPageRef{\LabelBase.option.version}).

The part of the book with the main text is introduced
with \Macro{mainmatter}\important{\Macro{mainmatter}}. If there is no front
matter, you can omit this command. The default page numbering in the main
matter uses Arabic numerals and (re)starts the page count at 1 at the start of
the main matter.

The back matter is introduced with
\Macro{backmatter}\important{\Macro{backmatter}}. Opinions differ as to what
belongs in the back matter. So in some cases you will find only the
bibliography\Index{bibliography}, in some cases only the index\Index{index},
and in other cases both of these as well as the appendices. The chapters in
the back matter are similar to the chapters in the front matter, but page
numbering is not reset. If you do require separate page numbering, you can use
the \DescRef{\LabelBase.cmd.pagenumbering}%
\important{\DescRef{\LabelBase.cmd.pagenumbering}} command in
\autoref{sec:\LabelBase.pagestyle},
\DescPageRef{\LabelBase.cmd.pagenumbering}.%
%
\EndIndexGroup


\section{Document Structure}
\seclabel{structure}%
\BeginIndexGroup
\BeginIndex{}{document structure}

The structure refers to dividing a document into parts, chapters, sections,
and additional levels of structure.


\begin{Declaration}
	\OptionVName{open}{method}
\end{Declaration}%
The\OnlyAt{\Class{scrbook}\and\Class{scrreprt}} \KOMAScript{} classes
\Class{scrbook} and \Class{scrreprt} give you the choice of where to start a
new chapter with two-sided printing. By default \Class{scrreprt} starts a new
chapter\Index{chapter>start} on the next page. This is equivalent to
\PName{method} \PValue{any}. However, \Class{scrbook} starts new chapters at
the next right-hand page. This is equivalent to \PName{method} \PValue{right}
and is usually used in books. But sometimes chapters should start on the
left-hand page of a two-page spread. You can accomplish this with the
\PName{method} \PValue{left}\ChangedAt{v3.00}{\Class{scrbook}\and
	\Class{scrreprt}}. You can find a summary of the available values in
\autoref{tab:\LabelBase.open}. The table also describes the effects of
\DescRef{\LabelBase.cmd.cleardoublepage},
\DescRef{\LabelBase.cmd.cleardoublepageusingstyle},
\DescRef{\LabelBase.cmd.cleardoublestandardpage},
\DescRef{\LabelBase.cmd.cleardoubleplainpage}, and
\DescRef{\LabelBase.cmd.cleardoubleemptypage} (see
\autoref{sec:\LabelBase.emptypage},
\DescPageRef{\LabelBase.cmd.cleardoublepage}).

\begin{table}
  \caption[{Available values for the \Option{open} option}]{Available values
    for the \Option{open} option to select page breaks with interleaf pages
    using \Class{scrbook} or \Class{scrreprt}}
  \begin{desctabular}
    \pventry{any}{%
      Parts, chapter, index, and back matter use
      \DescRef{\LabelBase.cmd.clearpage}, not
      \DescRef{\LabelBase.cmd.cleardoublepage}; 
      \DescRef{\LabelBase.cmd.cleardoublepageusingstyle},
      \DescRef{\LabelBase.cmd.cleardoublestandardpage},
      \DescRef{\LabelBase.cmd.cleardoubleplainpage},
      \DescRef{\LabelBase.cmd.cleardoubleemptypage}, and
      \DescRef{\LabelBase.cmd.cleardoublepage} behave the same as using
      \OptionValue{open}{right}.%
      \IndexOption{open~=\textKValue{any}}}%
    \pventry{left}{%
      Part, chapter, index, and back matter use
      \DescRef{\LabelBase.cmd.cleardoublepage}; the commands
      \DescRef{\LabelBase.cmd.cleardoublepageusingstyle},
      \DescRef{\LabelBase.cmd.cleardoublestandardpage},
      \DescRef{\LabelBase.cmd.cleardoubleplainpage},
      \DescRef{\LabelBase.cmd.cleardoubleemptypage}, and
      \DescRef{\LabelBase.cmd.cleardoublepage} result in a page break and add
      an interleaf page\Index{page>interleaf} if needed to reach the next
      left-hand page.%
      \IndexOption{open~=\textKValue{left}}}%
    \pventry{right}{%
      Part, chapter, index, and back matter use
      \DescRef{\LabelBase.cmd.cleardoublepage}; the commands
      \DescRef{\LabelBase.cmd.cleardoublepageusingstyle},
      \DescRef{\LabelBase.cmd.cleardoublestandardpage},
      \DescRef{\LabelBase.cmd.cleardoubleplainpage},
      \DescRef{\LabelBase.cmd.cleardoubleemptypage}, and
      \DescRef{\LabelBase.cmd.cleardoublepage} result in a page break and add
      an interleaf page\Index{page>interleaf} if needed to reach the next
      right-hand page.%
      \IndexOption{open~=\textKValue{right}}}%
  \end{desctabular}
  \label{tab:\LabelBase.open}
\end{table}

Since \LaTeX{} does not differentiate between left-hand and right-hand pages
in one-sided printing, the option has no effect in that case.

In the \Class{scrartcl} class, the section is the first structural element
below the part. For this reason, \Class{scrartcl} does not support this
option.
%
\EndIndexGroup


\begin{Declaration}
  \OptionVName{chapterprefix}{simple switch}
  \OptionVName{appendixprefix}{simple switch}
  \Macro{IfChapterUsesPrefixLine}\Parameter{then code}\Parameter{else code}
\end{Declaration}%
With\OnlyAt{\Class{scrbook}\and\Class{scrreprt}} the standard classes
\Class{book} and \Class{report}, a chapter
heading\Index[indexmain]{chapter>heading} consists of a line with the word
``Chapter''\footnote{When using another language the word ``Chapter'' is of
course translated to the appropriate language.} followed by the chapter
number. The heading itself is set left-justified on the following line. The
same effect is obtained in {\KOMAScript} with the \Option{chapterprefix}
option. You can use any value from \autoref{tab:truefalseswitch},
\autopageref{tab:truefalseswitch} as the \PName{simple switch}. The default,
however, is \OptionValue{chapterprefix}{false}, the opposite
behaviour\textnote{\KOMAScript{} vs. standard classes} of that of the standard
classes, which corresponds to \OptionValue{chapterprefix}{true}. These options
also affect the automatic running heads in the headers (see
\autoref{sec:\LabelBase.pagestyle},
\DescPageRef{\LabelBase.pagestyle.headings}).

Sometimes you may want to use the simplified chapter headings produced by 
\OptionValue{chapterprefix}{false} but at the same time to have the heading of
an appendix\Index{appendix} preceded by a line with ``Appendix'' followed by
the appendix letter. This is achieved by using the \Option{appendixprefix}
option (see \autoref{tab:truefalseswitch}, \autopageref{tab:truefalseswitch}).
Since this results in an inconsistent document layout, I advise against using
this option. Ultimately, this option changes the \Option{chapterprefix} option
automatically at the beginning of the \DescRef{\LabelBase.cmd.appendix}.

You\ChangedAt{v3.18}{\Class{scrbook}\and \Class{scrreprt}} can execute code
depending on the current setting for the chapter preceding line using
\Macro{IfChapterUsesPrefixLine}. If \Option{chapterprefix} is currently
active, the \PName{then code} is executed; otherwise, the \PName{else code} is
executed.

\BeginIndex{FontElement}{chapterprefix}%
You can change the font style of the chapter number line that uses
\OptionValue{chapterprefix}{true} or \OptionValue{appendixprefix}{true}
by using the
\FontElement{chapterprefix}\ChangedAt{v2.96a}{\Class{scrbook}\and
  \Class{scrreprt}} element with the \DescRef{\LabelBase.cmd.setkomafont} and
\DescRef{\LabelBase.cmd.addtokomafont} commands (see
\autoref{sec:\LabelBase.textmarkup},
\DescPageRef{\LabelBase.cmd.setkomafont}). The default is to use the
\DescRef{\LabelBase.fontelement.chapter}\IndexFontElement{chapter} element
(see \DescPageRef{\LabelBase.cmd.chapter}, as well as
\autoref{tab:\LabelBase.structureElementsFont},
\autopageref{tab:\LabelBase.structureElementsFont}).

You can find additional settings for chapter headings in the explanation for 
\DescRef{maincls-experts.cmd.RedeclareSectionCommand} and the commands
\DescRef{maincls-experts.cmd.chapterlineswithprefixformat} and
\DescRef{maincls-experts.cmd.chapterlinesformat} in
\autoref{sec:maincls-experts.sections}, \autoref{part:forExperts}.%
\EndIndexGroup


\begin{Declaration}
  \OptionVName{headings}{setting}
\end{Declaration}
Headings\index{heading}\index{document structure} of sectioning levels
normally use a relatively large font size in both the standard classes and
{\KOMAScript}.  This choice does not appeal to everyone and is especially
problematic for small paper sizes. Consequently, {\KOMAScript} provides, in
addition to the large headings defined by the
\OptionValue{headings}{big}\IndexOption{headings~=\textKValue{big}}%
\ChangedAt{v3.00}{\Class{scrbook}\and \Class{scrreprt}\and \Class{scrartcl}}%
\important{\OptionValue{headings}{big}\\
  \OptionValue{headings}{normal}\\
  \OptionValue{headings}{small}} option, the options
\OptionValue{headings}{normal}\IndexOption{headings~=\textKValue{normal}} and
\OptionValue{headings}{small}\IndexOption{headings~=\textKValue{small}}, which
allow for smaller headings. The font sizes resulting from these options for
\Class{scrbook} and \Class{scrreprt} are shown in
\autoref{tab:\LabelBase.structureElementsFont},
\autopageref{tab:\LabelBase.structureElementsFont}. Specifically, all three
settings reset the elements \DescRef{\LabelBase.fontelement.chapter}%
\IndexFontElement{chapter},
\DescRef{\LabelBase.fontelement.section}\IndexFontElement{section},
\DescRef{\LabelBase.fontelement.subsection}\IndexFontElement{subsection},
\DescRef{\LabelBase.fontelement.subsubsection}%
\IndexFontElement{subsubsection},
\DescRef{\LabelBase.fontelement.paragraph}\IndexFontElement{paragraph}, and
\DescRef{\LabelBase.fontelement.subparagraph}\IndexFontElement{subparagraph}
to the corresponding defaults. \Class{scrartcl}\OnlyAt{\Class{scrbook}\and
  \Class{scrreprt}} generally uses slightly smaller headings. The spacing
before and after chapter headings is also reset by these options.

\iffalse%
  Chapter headings also have the \Option{headings} option with the
  \PValue{twolinechapter} or \PValue{onelinechapter} settings, which are
  equivalent to the
  \OptionValueRef{\LabelBase}{chapterprefix}{true} and
  \OptionValueRef{\LabelBase}{chapterprefix}{false} options. In the appendix
  there are additional effects of the \PValue{onelineappendix} and
  \PValue{twolineappendix} options upon the \Option{headings} option, which
  corresponds to
  \OptionValueRef{\LabelBase}{appendixprefix}{false} and
  \OptionValueRef{\LabelBase}{appendixprefix}{true} (see also
  above). %
\else%
  Chapter headings\OnlyAt{\Class{scrbook}\and\Class{scrreprt}} also
  have\important{%
    \OptionValue{headings}{twolinechapter}\\
    \OptionValue{headings}{onelinechapter}\\
    \OptionValue{headings}{twolineappendix}\\
    \OptionValue{headings}{onelineappendix}%
  } the two options \OptionValue{headings}{twolinechapter} and
  \OptionValue{headings}{onelinechapter}, which correspond to
  \OptionValueRef{\LabelBase}{chapterprefix}{true} and
  \OptionValueRef{\LabelBase}{chapterprefix}{false} explained above. For the
  appendix,
  \OptionValueRef{\LabelBase}{appendixprefix}{true} and
  \OptionValueRef{\LabelBase}{appendixprefix}{false} serve as alternatives for
  the \OptionValue{headings}{twolineappendix} and
  \OptionValue{headings}{onelineappendix} options. %
  \iffalse%
    For the appendix, \OptionValue{headings}{twolineappendix} and
    \OptionValue{headings}{onelineappendix} are available, which correspond to
    the \OptionValueRef{\LabelBase}{appendixprefix}{true} and
    \OptionValueRef{\LabelBase}{appendixprefix}{false} options (also see above).
  \fi%
  Of course, these options do not exist with \Class{scrartcl}%
  \iffree{}{% Umbruchkorrekturtext
    , since this class does not have chapters}%
  .%
\fi

The\ChangedAt{v3.12}{\Class{scrbook}\and \Class{scrreprt}\and
  \Class{scrartcl}} \OptionValue{headings}{standardclasses}%
\important{\OptionValue{headings}{standardclasses}} option adjusts the font
sizes of the headings to those of the standard classes. In addition, the font
for the \DescRef{\LabelBase.fontelement.disposition}%
\IndexFontElement{disposition} element is set to \Macro{bfseries}. It
therefore no longer uses a sans-serif font for the headings. If you use
\Class{scrbook}\OnlyAt{\Class{scrbook}\and\Class{scrreprt}} or
\Class{scrreprt}, \OptionValue{headings}{twolinechapter} is also set and the
spacing between the chapter headings is adjusted to that of the standard
classes.

You\OnlyAt{\Class{scrbook}\and \Class{scrreprt}} can set the method to specify
the page on which new chapters begin with
\OptionValue{headings}{openany}\important{%
  \OptionValue{headings}{openany}\\
  \OptionValue{headings}{openright}\\
  \OptionValue{headings}{openleft}}, \OptionValue{headings}{openright}, and
\OptionValue{headings}{openleft}, or alternatively with the
\DescRef{\LabelBase.option.open} option, which takes the values \PValue{any},
\PValue{right}, and \PValue{left} (see above).

Another\ChangedAt{v3.10}{\Class{scrbook}\and \Class{scrreprt}\and
  \Class{scrartcl}} special feature of \KOMAScript{} is the handling of the
optional argument of the sectioning commands \DescRef{\LabelBase.cmd.part},
\DescRef{\LabelBase.cmd.chapter}, etc. down to
\DescRef{\LabelBase.cmd.subparagraph}. You can change its function and
meaning\important{\OptionValue{headings}{optiontohead}\\
  \OptionValue{headings}{optiontotoc}\\
  \OptionValue{headings}{optiontoheadandtoc}} with the options
\OptionValue{headings}{optiontohead}%
\IndexOption{headings~=\textKValue{optiontohead}},
\OptionValue{headings}{optiontotoc}%
\IndexOption{headings~=\textKValue{optiontotoc}} and
\OptionValue{headings}{optiontoheadandtoc}%
\IndexOption{headings~=\textKValue{optiontoheadandtoc}}.

See \autoref{tab:\LabelBase.headings} for a summary of all available settings
for the \Option{headings} option. The explanations of the sectioning commands
below contain examples using some of these settings.
\begin{desclist}
  \desccaption[{Available values for the \Option{headings} option}]{%
    Available values for the \Option{headings} option to format
    section headings%
    \label{tab:\LabelBase.headings}%
  }{%
    Available values for the \Option{headings} option (\emph{continued})%
  }%
  \pventry{big}{%
    Use large fonts in the headings for each of the default sectioning levels
    with wide spacing above and below the titles.
    \IndexOption{headings~=\textKValue{big}}}%
  \pventry{normal}{%
    Use medium-sized fonts in the headings with medium spacing above and below
    the titles.  \IndexOption{headings~=\textKValue{normal}}}%
  \entry{\PValue{onelineappendix}, \PValue{noappendixprefix},
    \PValue{appendixwithoutprefix}, \PValue{appendixwithoutprefixline}%
    \IndexOption{headings~=\textKValue{onelineappendix}}}{%
    Chapter headings in the appendix are set like other headings.%
  }%
  \entry{\PValue{onelinechapter}, \PValue{nochapterprefix},
    \PValue{chapterwithoutprefix}, \PValue{chapterwithoutprefixline}%
    \IndexOption{headings~=\textKValue{onelinechapter}}}{%
    Chapter titles are set like other headings.%
  }%
  \pventry{openany}{%
    The commands \DescRef{\LabelBase.cmd.cleardoublepageusingstyle},
    \DescRef{\LabelBase.cmd.cleardoublestandardpage},
    \DescRef{\LabelBase.cmd.cleardoubleplainpage},
    \DescRef{\LabelBase.cmd.cleardoubleemptypage}, and
    \DescRef{\LabelBase.cmd.cleardoublepage} generate a page break and insert
    an interleaf page\Index{interleaf page} if needed to reach the next
    right-hand page in two-sided printing, the same as in
    \OptionValue{headings}{openright}. Parts, chapter, the index, and back
    matter use \DescRef{\LabelBase.cmd.clearpage} instead of
    \DescRef{\LabelBase.cmd.cleardoublepage}.%
    \IndexOption{headings~=\textKValue{openany}}}%
  \pventry{openleft}{%
    The commands \DescRef{\LabelBase.cmd.cleardoublepageusingstyle},
    \DescRef{\LabelBase.cmd.cleardoublestandardpage},
    \DescRef{\LabelBase.cmd.cleardoubleplainpage},
    \DescRef{\LabelBase.cmd.cleardoubleemptypage}, and
    \DescRef{\LabelBase.cmd.cleardoublepage} generate a page break and insert
    an interleaf page if needed to reach the next left-hand page in two-sided
    printing. Part, chapter, index and back matter use
    \DescRef{\LabelBase.cmd.cleardoublepage}.%
    \IndexOption{headings~=\textKValue{openleft}}}%
  \pventry{openright}{%
    The commands \DescRef{\LabelBase.cmd.cleardoublepageusingstyle},
    \DescRef{\LabelBase.cmd.cleardoublestandardpage},
    \DescRef{\LabelBase.cmd.cleardoubleplainpage},
    \DescRef{\LabelBase.cmd.cleardoubleemptypage}, and
    \DescRef{\LabelBase.cmd.cleardoublepage} generate a page break and insert
    an interleaf page if needed to reach the next right-hand page in two-sided
    printing. Part, chapter, index and back matter use
    \DescRef{\LabelBase.cmd.cleardoublepage}.%
    \IndexOption{headings~=\textKValue{openright}}}%
  \pventry{optiontohead}{%
    The\ChangedAt{v3.10}{\Class{scrbook}\and \Class{scrreprt}\and
      \Class{scrartcl}} advanced functionality of the optional argument of the
    sectioning commands \DescRef{\LabelBase.cmd.part} down to
    \DescRef{\LabelBase.cmd.subparagraph} is activated. By default, the
    optional argument is used only for the running head.%
    \IndexOption{headings~=\textKValue{optiontohead}}%
  }%
  \entry{\PValue{optiontoheadandtoc}, \PValue{optiontotocandhead}%
    \IndexOption{headings~=\textKValue{optiontoheadandtoc}}}{%
    The\ChangedAt{v3.10}{\Class{scrbook}\and \Class{scrreprt}\and
      \Class{scrartcl}} advanced functionality of the optional argument of the
    sectioning commands \DescRef{\LabelBase.cmd.part} down to
    \DescRef{\LabelBase.cmd.subparagraph} is activated. By default, the
    optional argument is used for the running head and the table of contents.%
  }%
  \pventry{optiontotoc}{%
    The\ChangedAt{v3.10}{\Class{scrbook}\and \Class{scrreprt}\and
      \Class{scrartcl}} advanced functionality of the optional argument of the
    sectioning commands \DescRef{\LabelBase.cmd.part} down to
    \DescRef{\LabelBase.cmd.subparagraph} is activated. By default, the
    optional argument is used only for the table of contents.%
    \IndexOption{headings~=\textKValue{optiontohead}}%
  }%
  \pventry{small}{%
    Use small fonts in the headings with small spacing above and below the
    titles.%
    \IndexOption{headings~=\textKValue{small}}}%
  \pventry{standardclasses}{%
    \ChangedAt{v3.12}{\Class{scrbook}\and\Class{scrreprt}\and\Class{scrartcl}}%
    Reset the font settings for each of the standard sectioning levels and use
    headings with the sizes of the standard classes. For chapter headings,
    \Class{scrbook} und \Class{scrreprt} set
    \OptionValue{headings}{twolinechapter}.%
  }%
  \entry{\PValue{twolineappendix}, \PValue{appendixprefix},
    \PValue{appendixwithprefix}, \PValue{appendixwithprefixline}%
    \IndexOption{headings~=\textKValue{twolineappendix}}}{%
    Chapter titles in the appendix are set with a number line whose format is
    determined by
    \DescRef{\LabelBase.cmd.chapterformat}\IndexCmd{chapterformat}.%
  }%
  \entry{\PValue{twolinechapter}, \PValue{chapterprefix},
    \PValue{chapterwithprefix}, \PValue{chapterwithprefixline}%
    \IndexOption{headings~=\textKValue{twolinechapter}}}{%
    Chapter titles are set with a number line whose format is determined by
    \DescRef{\LabelBase.cmd.chapterformat}\IndexCmd{chapterformat}.%
  }%
\end{desclist}
%
\EndIndexGroup


\begin{Declaration}
  \OptionVName{numbers}{setting}
\end{Declaration}%
According to {\small DUDEN}, if only Arabic numerals are used to number
section headings\Index{headings}, the German practice is to have no point at
the end (see \cite[R\,3]{DUDEN}). On the other hand, if Roman numerals or
letters appear in the numbering, then a point should appear at the end of the
numbering (see \cite[R\,4]{DUDEN}). {\KOMAScript} has a mechanism that tries
to automate this somewhat complex rule. The result is that normally after
the sectioning commands \DescRef{\LabelBase.cmd.part} and
\DescRef{\LabelBase.cmd.appendix} the numbering switches to using a final
point. This information is saved in the \File{aux} file and takes effect on
the next {\LaTeX} run.

Sometimes the mechanism for placing or omitting the final point may fail.
Sometimes other languages have different rules. Therefore you can force the
use of the final point with the option
\OptionValue{numbers}{endperiod}\IndexOption{numbers~=\textKValue{endperiod}}%
\important{\OptionValue{numbers}{endperiod}\\
\OptionValue{numbers}{noendperiod}} or to prohibit it with
\OptionValue{numbers}{noendperiod}.

Note\textnote{Attention!} that this mechanism only takes effect on the
next {\LaTeX} run.  Therefore, before you try to use these options to force
the correct numbering format, you should always perform another \LaTeX{} run 
without modifying the document.

You can find a summary of the available values for the \PName{setting} of
\PName{numbers} in \autoref{tab:\LabelBase.numbers}. Unlike most other
settings, this option can only be set in the document preamble, i.\,e. before
\Macro{begin}\PParameter{document}.

\begin{table}
  \caption[{Available values for the \Option{numbers} option}]{Available values
    of the \Option{numbers} option to configure the final points of the
    numbers for section headings}
  \label{tab:\LabelBase.numbers}
  \begin{desctabular}
    \entry{\PValue{autoendperiod}, \PValue{autoenddot}, \PValue{auto}}{%
      \KOMAScript{} decides whether to set the point at the end of sectioning
      command numbers. If there are only Arabic numerals, the point will
      be omitted. If a letter or Roman numeral is found, the
      point is set. However, references to these numbers are set without
      a final point.%
      \IndexOption{numbers~=\textKValue{autoendperiod}}}%
    \entry{\PValue{endperiod}, \PValue{withendperiod}, \PValue{periodatend},
      \PValue{enddot}, \PValue{withenddot}, \PValue{dotatend}}{%
      All numbers of sectioning commands and their subordinate numbers are set
      with a final point. Only references will be set without the final
      point.% 
      \IndexOption{numbers~=\textKValue{endperiod}}}%
    \entry{\PValue{noendperiod}, \PValue{noperiodatend},
      \PValue{noenddot}, \PValue{nodotatend}}{%
      All numbers of sectioning commands are set without a final point.%
      \IndexOption{numbers~=\textKValue{noendperiod}}}%
  \end{desctabular}
\end{table}
%
\EndIndexGroup


\begin{Declaration}
  \Option{chapteratlists}%
  \OptionVName{chapteratlists}{value}
\end{Declaration}%
As mentioned in \autoref{sec:\LabelBase.floats},
\DescPageRef{\LabelBase.option.listof}\OnlyAt{\Class{scrbook}\and
  \Class{scrreprt}}, every chapter that is created with
\DescRef{\LabelBase.cmd.chapter} normally inserts a vertical space in the
lists of floating environments (e.\,g., tables and figures).
Since version~2.96a,\important{\OptionValueRef{\LabelBase}{version}{2.96a}}%
\ChangedAt{v2.96a}{\Class{scrbook}\and \Class{scrreprt}} this also applies to
the \DescRef{\LabelBase.cmd.addchap} command unless you choose a compatibility
setting for an earlier version (see the \DescRef{\LabelBase.option.version}
option in \autoref{sec:\LabelBase.compatibilityOptions},
\DescPageRef{\LabelBase.option.version}).

In addition, you can use the \Option{chapteratlists} option to change the
vertical spacing by specifying the desired distance as the \PName{value}.
The default with \OptionValueRef{\LabelBase}{listof}{chaptergapsmall}%
\IndexOption{listof~=\textKValue{chaptergapsmall}} %
is 10\Unit{pt} (see the\DescRef{\LabelBase.option.version} option in
\autoref{sec:\LabelBase.compatibilityOptions},
\DescPageRef{\LabelBase.option.version}).

If you use \OptionValue{chapteratlists}{entry}%
\IndexOption{chapteratlists~=\textKValue{entry}}%
\important{\OptionValue{chapteratlists}{entry}} or \Option{chapteratlists}
without specifying a value, instead of a vertical space, the chapter entry
itself will be entered into the lists. Note that such an entry occurs even if
the chapter does not contain a floating environment. A method where only
chapters with floating environments are displayed in the respective list can
be found in the German-language \KOMAScript{} forum at
\cite{https://komascript.de/comment/5070}.

Please\textnote{Attention!} note that changes to this option only takes
effect after two additional {\LaTeX} runs.%
%
\EndIndexGroup


\begin{Declaration}
  \Macro{part}\OParameter{short version}\Parameter{heading}%
  \Macro{chapter}\OParameter{short version}\Parameter{heading}%
  \Macro{section}\OParameter{short version}\Parameter{heading}%
  \Macro{subsection}\OParameter{short version}\Parameter{heading}%
  \Macro{subsubsection}\OParameter{short version}\Parameter{heading}%
  \Macro{paragraph}\OParameter{short version}\Parameter{heading}%
  \Macro{subparagraph}\OParameter{short version}\Parameter{heading}
\end{Declaration}%
\Index[indexmain]{part}%
\Index[indexmain]{chapter}%
\Index[indexmain]{section}%
The standard sectioning commands in {\KOMAScript} work the same way as those
in the standard classes. Thus, you can specify an alternative text for the
table of contents and running heads as an optional argument to the sectioning
commands.

However, with\ChangedAt{v3.10}{\Class{scrbook}\and \Class{scrreprt}\and
  \Class{scrartcl}}\textnote{\KOMAScript{} vs. standard classes} the
\important{\DescRef{\LabelBase.option.headings}}
\OptionValueRef{\LabelBase}{headings}{optiontohead}%
\IndexOption[indexmain]{headings~=\textKValue{optiontohead}} option,
\KOMAScript{} only uses the optional argument \PName{short version} in the
running head, not the table of contents. Of course, this text will only appear
if you use a page style\Index{page>style} that puts the corresponding
sectioning level in the running head. See \autoref{sec:\LabelBase.pagestyle}
and \autoref{cha:scrlayer-scrpage}. With the
\OptionValueRef{\LabelBase}{headings}{optiontotoc}%
\IndexOption[indexmain]{headings~=\textKValue{optiontotoc}} option,
\KOMAScript{} uses the optional argument \PName{short version} exclusively for
the table of contents and not the running head. However, the entry will be
shown only if the \DescRef{\LabelBase.counter.tocdepth} counter is great
enough (see \autoref{sec:\LabelBase.toc},
\DescPageRef{\LabelBase.counter.tocdepth}). With the
\OptionValueRef{\LabelBase}{headings}{optiontoheadandtoc}%
\IndexOption[indexmain]{headings~=\textKValue{optiontoheadandtoc}} option,
\KOMAScript{} uses the optional argument \PName{short version} in both the
table of contents and the running head. These three options all activate the
extended interpretation of the optional argument \PName{short version}, which
is not active by default.

The \ChangedAt{v3.10}{\Class{scrbook}\and \Class{scrreprt}\and
  \Class{scrartcl}} extended interpretation of the optional argument checks to
see if there is an equals sign in \PName{short version}. If so, the optional
argument will be interpreted as an \PName{option list}.
Three options\,---\,\OptionVName{head}{running head}\important{\Option{head}\\
  \Option{tocentry}\\
  \Option{reference}\\
  \Option{nonumber}}, \OptionVName{tocentry}{table of contents entry},
\OptionVName{reference}{reference title}%
\ChangedAt{v3.22}{\Class{scrbook}\and \Class{scrreprt}\and \Class{scrartcl}},
and \OptionVName{nonumber}{simple switch}\ChangedAt{v3.27}{\Class{scrbook}\and
  \Class{scrreprt}\and \Class{scrartcl}}\,---\,are supported with this
format. To use commas or equals signs within the values of those options, you
must enclose them in braces.

Please\textnote{Attention!} note that this mechanism only works as long as
\KOMAScript{} controls the sectioning commands. If you use package that
redefines the \KOMAScript{}'s or the internal \LaTeX{} kernel's sectioning
commands, \KOMAScript{} can no longer provide this extended mechanism. This
also applies to a \KOMAScript{} extension that is always active: sectioning
commands with no heading text\textnote{empty headings} do not create entries
in the table of contents. If you really want an entry with empty heading text,
you can use an invisible entry like \lstinline|\mbox{}|.

\begin{Example}
  Suppose you have a document with very long chapter headings. These headings
  should appear in the table of contents, but you want to limit the running
  head to short, single-line headings. You can do this with the optional
  argument of \Macro{chapter}.
\begin{lstcode}
  \chapter[short version of chapter heading]
          {The Sectioning Command for Chapters 
            Supports not only the Heading Text 
            Itself but also a Short Version Whose
            Use can be Controlled}
\end{lstcode}

  A little later you realize that the line breaks for this long heading are
  very inappropriate. You therefore want to choose the breaks
  yourself. Nevertheless, you still want automatic line breaking in the table
  of contents. With
\begin{lstcode}
  \chapter[head={short version of chapter heading},
           tocentry={The Sectioning
             Command for Chapters Supports not
             only the Heading Text Itself but
             also a Short Version Whose Use 
             can be Controlled}]
          {The Sectioning\\
          	Command for Chapters\\
            Supports not only\\
            the Heading Text Itself\\
            but also\\
            a Short Version Whose\\
            Use can be Controlled}
\end{lstcode}
  you create separate entries for the table of contents, running head, and
  chapter heading itself. The arguments of the options \Option{head} and
  \Option{tocentry} have been enclosed in braces so their contents
  can be arbitrary.

  The need for braces in the example above is best illustrated by another
  example. Suppose you chose the 
  \OptionValueRef{\LabelBase}{headings}{optiontotoc} option and set the title
  this way:
\begin{lstcode}
  \section[head=\emph{value}]
          {Option head=\emph{value}}
\end{lstcode}
  This results in the entry ``Option head=\emph{value}'' in the table of
  contents but ``\emph{value}'' in the running head. But surely you wanted the
  entry ``head=\emph{value}'' in the table of contents and the complete
  heading text in the running head. You can do this using braces:
\begin{lstcode}
  \section[head{=}\emph{value}]
          {Option head=\emph{value}}
\end{lstcode}

  A similar case concerns the comma. Using the same
  \DescRef{\LabelBase.option.headings} option as before
\begin{lstcode}
  \section[head=0, 1, 2, 3, \dots]
          {Natural Numbers Including the Zero}
\end{lstcode}
  results in an error because the comma is interpreted as the
  separator between the individual options of the option list %
  ``\lstinline|0, 1, 2, 3, \dots|''. But writing
\begin{lstcode}
  \section[head={0, 1, 2, 3, \dots}]
          {Natural Numbers Including the Zero}
\end{lstcode}
  makes ``\lstinline|0, 1, 2, 3, \dots|'' the argument of
  the \Option{head} option.
\end{Example}

Like\ChangedAt{v3.22}{\Class{scrbook}\and \Class{scrreprt}\and
  \Class{scrartcl}} setting the title of a running head with the \Option{head}
option, or setting the title of a table-of-contents entry with the
\Option{tocentry} option, you can set the title of a reference with the
\Package{nameref}\IndexPackage{nameref}\important{\Package{nameref}\\
  \Package{titleref}\\
  \Package{zref-titleref}} or
\Package{titleref}\IndexPackage{titleref} packages, or with the
\Package{titleref}\IndexPackage{zref-titleref} module of the
\Package{zref}\IndexPackage{zref} package, using the \Option{reference}%
\important{\Option{reference}} option. Note that the support for the
\Package{titleref} package is rather rudimentary, since that package's
performance is poor compared to the other two, and it is not fully compatible
with \Package{hyperref}\IndexPackage{hyperref}\important{\Package{hyperref}}.

You can deactivate the numbering using
\OptionValue{nonumber}{true}\important{\Option{nonumber}}%
\ChangedAt{v3.27}{\Class{scrbook}\and \Class{scrreprt}\and \Class{scrartcl}}
in the extended optional argument. In contrast to the
\hyperref[desc:maincls.cmd.part*]{starred versions of the sectioning commands}
explained below, the titles will still be added to the table of contents and,
if applicable, used for the running head. For \Macro{part}, \Macro{chapter},
and \Macro{section}, using \OptionValue{nonumber}{true} this essentially
corresponds to the \DescRef{\LabelBase.cmd.addpart},
\DescRef{\LabelBase.cmd.addchap}, and \DescRef{\LabelBase.cmd.addsec}
commands, which are explained on \DescPageRef{\LabelBase.cmd.addpart}.

The part-level title\important{\Macro{part}} (\Macro{part}) differs from other
sectioning levels by being numbered independently. This means that the chapter
level (in \Class{scrbook} or \Class{scrreprt}), or the section level (in
\Class{scrartcl}) is numbered consecutively over all parts. Furthermore, for
the \Class{scrbook}\OnlyAt{\Class{scrbook}\and \Class{scrreprt}} and
\Class{scrreprt} classes, the title of the part level together with the
corresponding preamble (see \DescRef{\LabelBase.cmd.setpartpreamble},
\DescPageRef{\LabelBase.cmd.setpartpreamble}) is set on a separate page.

The \Macro{chapter}\important{\Macro{chapter}}\OnlyAt{\Class{scrbook}\and
  \Class{scrreprt}} command only exists in the book and report classes, that
is, in \Class{book}, \Class{scrbook}, \Class{report} and \Class{scrreport},
but not in the article classes \Class{article} and \Class{scrartcl}.
Furthermore, the \Macro{chapter}\textnote{\KOMAScript{} vs. standard classes}
command in {\KOMAScript} differs substantially from the version in the
standard classes. In the standard classes, the chapter number is used together
with the prefix ``Chapter'', or the corresponding word in the appropriate
language, on a separate line above the actual chapter title text.
\KOMAScript{} replaces this
overpowering\important{\DescRef{\LabelBase.option.chapterprefix}\\
  \DescRef{\LabelBase.option.appendixprefix}} style with a simple chapter
number before the chapter title, but you can restore the original behaviour
with the \DescRef{\LabelBase.option.chapterprefix} option (see
\DescPageRef{\LabelBase.option.chapterprefix}).

Please\textnote{Attention!} note that \Macro{part} and \Macro{chapter} in the
\Class{scrbook} and \Class{scrreprt}
\OnlyAt{\Class{scrbook}\and\Class{scrreprt}} classes change the page style for
one page. The page style applied in {\KOMAScript} is defined in the macros
\DescRef{\LabelBase.cmd.partpagestyle} and
\DescRef{\LabelBase.cmd.chapterpagestyle} (see
\autoref{sec:\LabelBase.pagestyle},
\DescPageRef{\LabelBase.cmd.titlepagestyle}).

\BeginIndexGroup
\BeginIndex[indexother]{}{font}%
\BeginIndex[indexother]{}{font>style}%
\BeginIndex[indexother]{}{font>size}%
\BeginIndex{FontElement}{part}\LabelFontElement{part}%
\BeginIndex{FontElement}{partnumber}\LabelFontElement{partnumber}%
\BeginIndex{FontElement}{chapter}\LabelFontElement{chapter}%
\BeginIndex{FontElement}{chapterprefix}\LabelFontElement{chapterprefix}%
\BeginIndex{FontElement}{section}\LabelFontElement{section}%
\BeginIndex{FontElement}{subsection}\LabelFontElement{subsection}%
\BeginIndex{FontElement}{subsubsection}\LabelFontElement{subsubsection}%
\BeginIndex{FontElement}{paragraph}\LabelFontElement{paragraph}%
\BeginIndex{FontElement}{subparagraph}\LabelFontElement{subparagraph}%
\BeginIndex{FontElement}{disposition}\LabelFontElement{disposition}%
You\ChangedAt{v2.8p}{\Class{scrbook}\and \Class{scrreprt}\and
  \Class{scrartcl}} can change the font style for all headings with the
\DescRef{\LabelBase.cmd.setkomafont}\IndexCmd{setkomafont} and
\DescRef{\LabelBase.cmd.addtokomafont}\IndexCmd{addtokomafont} commands (see
\autoref{sec:\LabelBase.textmarkup},
\DescPageRef{\LabelBase.cmd.setkomafont}). In doing so, the element
\FontElement{disposition}\important{\FontElement{disposition}} is applied
first, followed by the specific element\important{%
  \FontElement{part}\\
  \FontElement{chapter}\\
  \FontElement{section}\\
  \FontElement{subsection}\\
  \FontElement{paragraph}\\
  \FontElement{subparagraph}} for each sectioning level (see
\autoref{tab:\LabelBase.fontelements},
\autopageref{tab:\LabelBase.fontelements}). There is a separate elements,
\FontElement{partnumber}, for the number of the part heading, and
\FontElement{chapterprefix}, for the optional prefix line of chapter headings.
The initial definition for the \FontElement{disposition} element is
\Macro{normalcolor}\linebreak[2]\Macro{sffamily}\linebreak[2]\Macro{bfseries}.
The default font sizes for the specific elements depends on the options
\OptionValueRef{\LabelBase}{headings}{big},
\OptionValueRef{\LabelBase}{headings}{normal}, and
\OptionValueRef{\LabelBase}{headings}{small} (see
\DescPageRef{\LabelBase.option.headings}). They are listed in
\autoref{tab:\LabelBase.structureElementsFont}.
% Umbruchkorrektur: Tabelle verschoben
\begin{table}
%  \centering%
%  \caption
  \KOMAoptions{captions=topbeside}%
  \setcapindent{0pt}%
  \begin{captionbeside}[{Default font sizes for different levels of document
    sectioning}]{%
    \label{tab:\LabelBase.structureElementsFont}%
    Default font sizes for different levels of document sectioning in
    \Class{scrbook} and \Class{scrreprt}}[l]
  \begin{tabular}[t]{lll}
    \toprule
    Class Option & Element & Default\\
    \midrule
    \OptionValueRef{\LabelBase}{headings}{big}
      & \DescRef{\LabelBase.fontelement.part}\IndexFontElement{part}
      & \Macro{Huge} \\
      & \DescRef{\LabelBase.fontelement.partnumber}%
        \IndexFontElement{partnumber}
      & \Macro{huge} \\
      & \DescRef{\LabelBase.fontelement.chapter}\IndexFontElement{chapter}
      & \Macro{huge} \\
      & \DescRef{\LabelBase.fontelement.chapterprefix}%
        \IndexFontElement{chapterprefix}
      & \DescRef{\LabelBase.cmd.usekomafont}\PParameter{chapter} \\
      & \DescRef{\LabelBase.fontelement.section}\IndexFontElement{section}
      & \Macro{Large} \\
      & \DescRef{\LabelBase.fontelement.subsection}%
        \IndexFontElement{subsection}
      & \Macro{large} \\
      & \DescRef{\LabelBase.fontelement.subsubsection}%
        \IndexFontElement{subsubsection}
      & \Macro{normalsize} \\
      & \DescRef{\LabelBase.fontelement.paragraph}\IndexFontElement{paragraph}
      & \Macro{normalsize} \\
      & \DescRef{\LabelBase.fontelement.subparagraph}%
        \IndexFontElement{subparagraph}
      & \Macro{normalsize} \\[1ex]
    \OptionValueRef{\LabelBase}{headings}{normal}
      & \DescRef{\LabelBase.fontelement.part}          & \Macro{huge} \\
      & \DescRef{\LabelBase.fontelement.partnumber}    & \Macro{huge} \\
      & \DescRef{\LabelBase.fontelement.chapter}       & \Macro{LARGE} \\
      & \FontElement{chapterprefix}\IndexFontElement{chapterprefix}
      & \DescRef{\LabelBase.cmd.usekomafont}\PParameter{chapter} \\
      & \DescRef{\LabelBase.fontelement.section}       & \Macro{Large} \\
      & \DescRef{\LabelBase.fontelement.subsection}    & \Macro{large} \\
      & \DescRef{\LabelBase.fontelement.subsubsection} & \Macro{normalsize} \\
      & \DescRef{\LabelBase.fontelement.paragraph}     & \Macro{normalsize} \\
      & \DescRef{\LabelBase.fontelement.subparagraph}  & \Macro{normalsize} \\[1ex]
    \OptionValueRef{\LabelBase}{headings}{small}
      & \DescRef{\LabelBase.fontelement.part}          & \Macro{LARGE} \\
      & \DescRef{\LabelBase.fontelement.partnumber}    & \Macro{LARGE} \\
      & \DescRef{\LabelBase.fontelement.chapter}       & \Macro{Large} \\
      & \FontElement{chapterprefix}\IndexFontElement{chapterprefix}
      & \DescRef{\LabelBase.cmd.usekomafont}\PParameter{chapter} \\
      & \DescRef{\LabelBase.fontelement.section}       & \Macro{large} \\
      & \DescRef{\LabelBase.fontelement.subsection}    & \Macro{normalsize} \\
      & \DescRef{\LabelBase.fontelement.subsubsection} & \Macro{normalsize} \\
      & \DescRef{\LabelBase.fontelement.paragraph}     & \Macro{normalsize} \\
      & \DescRef{\LabelBase.fontelement.subparagraph}  & \Macro{normalsize}\\
    \bottomrule
  \end{tabular}
  \end{captionbeside}
\end{table}
%
\begin{Example}
  Suppose you use the \OptionValueRef{\LabelBase}{headings}{big} class option
  and notice that the very large headings of the document parts are too bold. 
  You could change this as follows:
\begin{lstcode}
  \setkomafont{disposition}{\normalcolor\sffamily}
  \part{Appendices}
  \addtokomafont{disposition}{\bfseries}
\end{lstcode}
Using the command above you only switch off the font attribute
\textbf{bold} for a heading ``Appendices''. A much more convenient
and elegant solution is to change all \Macro{part} headings at once.
This is done either by:
\begin{lstcode}
  \addtokomafont{part}{\normalfont\sffamily}
  \addtokomafont{partnumber}{\normalfont\sffamily}
\end{lstcode}
  or simply:
\begin{lstcode}
  \addtokomafont{part}{\mdseries}
  \addtokomafont{partnumber}{\mdseries}
\end{lstcode}
The second version is preferred because it gives you the correct result
even if you change the \DescRef{\LabelBase.fontelement.disposition}
element\IndexFontElement{disposition}, for instance:
\begin{lstcode}
  \setkomafont{disposition}{\normalcolor\bfseries}
\end{lstcode}
  With this change, it is possible to set all section levels at once to no
  longer use sans serif fonts.
\end{Example}

I strongly advise against using the ability to switch fonts in order to mix
fonts, font sizes, and font attributes wildly. Picking the right font for the
job is a matter for experts and has almost nothing to do with the personal
tastes of non-experts. See the citation at the end of
\autoref{sec:typearea.tips}, \autopageref{sec:typearea.tips.cite} and to the
following explanation.

\begin{Explain}
  It is possible to use different font types for different sectioning
  levels in {\KOMAScript}. Non-experts in typography should absolutely avoid
  doing so for excellent typographical reasons.

  A rule of typography states that you should mix as few fonts as possible.
  Using sans serif for headings already seems to violate this rule. However,
  you should realize that large, bold, serif letters are much too heavy for
  headings. Strictly speaking, you should then use a normal instead of a bold
  or semi-bold font. However, in deeper levels of the sectioning, a normal
  font may then appear too light. On the other hand, sans serif fonts have a
  very pleasant appearance in headings, and almost solely in headings. There
  is, therefore, good reason why sans serif is the default in {\KOMAScript}.

  Greater variety should, however, be avoided. Font mixing is something for
  professionals. For this reason, if you want to use fonts other than the
  standard {\TeX} fonts\,---\,regardless of whether these are CM\Index{CM
    fonts}, EC\Index{EC fonts}, or LM fonts\Index{LM fonts}\,---\,you should
  consult an expert about the compatibility of the sans serif and serif fonts,
  or redefine the element
  \DescRef{\LabelBase.fontelement.disposition}\IndexFontElement{disposition}
  as a precautionary measure. The author considers the commonly encountered
  combinations of Times and Helvetica or Palatino with Helvetica to be
  awkward.
\end{Explain}
\EndIndexGroup
%
\EndIndexGroup


\begin{Declaration}
  \Macro{part*}\Parameter{heading}%
  \Macro{chapter*}\Parameter{heading}%
  \Macro{section*}\Parameter{heading}%
  \Macro{subsection*}\Parameter{heading}%
  \Macro{subsubsection*}\Parameter{heading}%
  \Macro{paragraph*}\Parameter{heading}%
  \Macro{subparagraph*}\Parameter{heading}
\end{Declaration}%
The starred variants of all sectioning commands produce
unnumbered\Index{numbering} headings which do not appear in the table of
contents\Index{table>of contents}\Index{contents>table of} or in the running
head\Index{running head}. The absence of a running head often has an unwanted
side effect. If, for example, a chapter set using \Macro{chapter*} spans
several pages, then the running head of the previous chapter suddenly
reappears.  {\KOMAScript} offers a solution for this problem, described below.
\Macro{chapter*}\OnlyAt{\Class{scrbook}\and\Class{scrreprt}} only exists in
book and report classes, that is, \Class{book}, \Class{scrbook},
\Class{report} and \Class{scrreport}, not in the article classes
\Class{article} and \Class{scrartcl}.

Please\textnote{Attention!} note that \DescRef{\LabelBase.cmd.part} and
\DescRef{\LabelBase.cmd.chapter} change the page style for one page. While the
standard classes use the \PageStyle{plain} page style, {\KOMAScript} applies
the style defined in the \DescRef{\LabelBase.cmd.partpagestyle} and
\DescRef{\LabelBase.cmd.chapterpagestyle} macros (see
\autoref{sec:\LabelBase.pagestyle},
\DescPageRef{\LabelBase.cmd.titlepagestyle}).

The possibilities for switching fonts\ChangedAt{v2.8p}{%
  \Class{scrbook}\and\Class{scrreprt}\and\Class{scrartcl}} %
described above for the unstarred variants apply without change. The elements
use the same names since they do not indicate variants but structuring
levels.%
%
\EndIndexGroup


\iffalse% Umbruchkorrekturtext
  In the standard classes\textnote{\KOMAScript{} vs. standard classes} there
  are no further structuring commands. In particular, there are no commands
  which can produce unnumbered chapters or sections which show up in the table
  of contents and in the running heading.%
\fi


\begin{Declaration}
  \Macro{addpart}\OParameter{short version}\Parameter{heading}%
  \Macro{addpart*}\Parameter{heading}%
  \Macro{addchap}\OParameter{short version}\Parameter{heading}%
  \Macro{addchap*}\Parameter{heading}%
  \Macro{addsec}\OParameter{short version}\Parameter{heading}%
  \Macro{addsec*}\Parameter{heading}
  \textnote[n]{\KOMAScript{} vs. standard classes}
\end{Declaration}%
In addition to the commands of the standard classes, {\KOMAScript} offers the
new commands \Macro{addpart}, \Macro{addsec} and \Macro{addchap}. They are
similar to the standard commands \DescRef{\LabelBase.cmd.part}\IndexCmd{part}%
\important{\DescRef{\LabelBase.cmd.part}, \DescRef{\LabelBase.cmd.chapter},
  \DescRef{\LabelBase.cmd.section}},
\DescRef{\LabelBase.cmd.chapter}\IndexCmd{chapter} and
\DescRef{\LabelBase.cmd.section}\IndexCmd{section} except that they are
unnumbered. They thus produce both a running head and an entry in the table of
contents which take into account the \DescRef{\LabelBase.option.headings}%
\important{\DescRef{\LabelBase.option.headings}} option settings, especially
the handling of the optional argument. However, enabling or disabling the
\Option{nonumber} switch will have no effect. 

The starred variants \Macro{addchap*} and \Macro{addsec*} are similar to the
standard commands \DescRef{\LabelBase.cmd.chapter*} and
\DescRef{\LabelBase.cmd.section*} except for a small but important difference:
the running heads are deleted. This eliminates the side effect of obsolete
headers mentioned above. Instead, the running heads on subsequent pages remain
empty. \OnlyAt{\Class{book}\and \Class{scrreprt}}\Macro{addchap} and
\Macro{addchap*} only exist, of course, in book and report classes, namely
\Class{book}, \Class{scrbook}, \Class{report} and \Class{scrreport}, not in
the article classes \Class{article} and \Class{scrartcl}.

The \Macro{addpart} command  produces an unnumbered document part with an
entry in the table of contents. Since the running heads are already deleted
by \DescRef{\LabelBase.cmd.part} and \DescRef{\LabelBase.cmd.part*} the
previously mentioned problem with obsolete headers does not exist. The starred
version \Macro{addpart*} is thus identical to \DescRef{\LabelBase.cmd.part*}
and is only defined for reasons of consistency.

Please note\textnote{Attention!} that \Macro{addpart} and \Macro{addchap} and
their starred variants change the page style for one page. The particular page
style is defined in the macros \DescRef{\LabelBase.cmd.partpagestyle} and
\DescRef{\LabelBase.cmd.chapterpagestyle} (see
\autoref{sec:\LabelBase.pagestyle},
\DescPageRef{\LabelBase.cmd.titlepagestyle}).
\iffree{\iffalse}{\csname iftrue\endcsname}% Umbruchkorrekturbeispiel
  \begin{Example}
    You write a book with a long afterword that spans several pages. This
    should receive neither a running head nor an entry in the table of
    contents. Therefore, you first use \DescRef{\LabelBase.cmd.chapter*}:
\begin{lstcode}
  \documentclass{scrbook}
  \usepackage{blindtext}
  \begin{document}
  \tableofcontents
  \blinddocument
  \chapter*{Afterword}
  \Blindtext[10]
  \end{document}
\end{lstcode}
    However, you discover that the running head of the previous chapter
    reappears on the second and third pages of the afterword. Naturally, you do
    not want that. Therefore, you use \Macro{addchap*}:
\begin{lstcode}
  \documentclass{scrbook}
  \usepackage{blindtext}
  \begin{document}
  \tableofcontents
  \blinddocument
  \addchap*{Afterword}
  \Blindtext[10]
  \end{document}
\end{lstcode}
    The complete postscript is now without a running head, if that is what you
    want. But even that does not quite meet your expectations. A running head
    would be nice. With
\begin{lstcode}
  \documentclass{scrbook}
  \usepackage{blindtext}
  \begin{document}
  \tableofcontents
  \blinddocument
  \addchap{Afterword}
  \Blindtext[10]
  \end{document}
\end{lstcode}
    you now get both ``Afterword'' as the running head on the even pages and an
    entry in the table of contents. But this entry bothers your publisher. He
    insists that the afterword have no entry in the table of contents. Even if
    that seems strange to you, you want to satisfy the publisher, so with
    \OptionValueRef{\LabelBase}{headings}{optiontotocandhead} you first
    activate the extended function for the optional argument and then
    explicitly set an empty argument for the table of contents:
\begin{lstcode}
  \documentclass[headings=optiontotocandhead]
                {scrbook}
  \usepackage{blindtext}
  \begin{document}
  \tableofcontents
  \blinddocument
  \addchap[tocentry={}]{Afterword}
  \Blindtext[10]
  \end{document}
\end{lstcode}
    The running head is not changed by \OptionValue{tocentry}{\{\}}. Since
    \Class{scrbook} automatically omits empty entries in the table of contents,
    the publisher is now satisfied.
  \end{Example}%
\fi

The possibilities for switching fonts\ChangedAt{v2.8p}{%
  \Class{scrbook}\and\Class{scrreprt}\and\Class{scrartcl}} %
described above for the unstarred variant of the sectioning commands apply
without change. The elements have the same names since they do not designate
variants but sectioning levels.%
%
\EndIndexGroup


\begin{Declaration}
  \Macro{minisec}\Parameter{heading}
\end{Declaration}%
Sometimes you want a heading\Index{heading} that is highlighted but also
closely linked to the following text. Such a heading should not be separated
by a large vertical skip.

The \Macro{minisec} command is designed for this situation. This heading is
not associated with any sectioning level\textnote{no sectioning level}. Such a
\emph{mini-section} does not produce an entry in the table of contents, nor
does it receive any numbering.

\BeginIndexGroup
\BeginIndex{FontElement}{minisec}\LabelFontElement{minisec}%
You can change the font of the \Macro{minisec} command using the
\DescRef{\LabelBase.fontelement.disposition}%
\important{\DescRef{\LabelBase.fontelement.disposition}\\
\FontElement{minisec}} and \FontElement{minisec}\ChangedAt{v2.96a}{%
  \Class{scrbook}\and \Class{scrreprt}\and \Class{scrartcl}} element (see
\autoref{tab:maincls.fontelements}, \autopageref{tab:maincls.fontelements}).
The default of the \FontElement{minisec} element is empty, so by default only
the \DescRef{\LabelBase.fontelement.disposition} element is used.%
\EndIndexGroup

\begin{Example}
  You have developed a kit for building a mouse trap and want the
  documentation separated into a list of necessary items and an
  assembly description. You could write the following:
\begin{lstcode}
  \documentclass{scrartcl}

  \begin{document}
	
	\title{DIY Projects}
	\author{Two Left Thumbs}
	\date{\today}
	\maketitle
	
	\section{Mousetrap}
	
	The first project is suitable for beginners and only requires
	a few components that should be found in every household.
	
    \minisec{Material Required}

    \begin{flushleft}
      1 board ($100\times 50 \times 12$)\\
      1 swing-top cap of a beer-bottle\\
      1 ballpoint pen\\
      1 push pin\\
      2 screws\\
      1 hammer\\
      1 knife
    \end{flushleft}

    \minisec{Assembly}
    First, find the mouse hole and put the push pin directly behind
    the hole so that the mouse cannot escape during the following
    actions.

    Next tap the swing-top cap into the mouse hole with the hammer.
    If the cap is not big enough to block the hole completely and
    permanently, take the board instead and screw it to the front
    of the mouse hole using the two screws and the knife. Of
    course, you can use a screwdriver instead of a knife. The
     ballpoint pen has fallen victim to animal welfare.
  \end{document}
\end{lstcode}
  The main part, starting with the heading ``Material Required'' will look
  like this:
  \begin{ShowOutput}[\baselineskip]\setlength{\parindent}{1em}
    \minisec{Material Required}

    \begin{flushleft}
      1 board ($100\times 50 \times 12$)\\
	  1 swing-top cap of a beer-bottle\\
	  1 ballpoint pen\\
	  1 push pin\\
	  2 screws\\
	  1 hammer\\
	  1 knife
    \end{flushleft}

    \minisec{Assembly}
    First, find the mouse hole and put the push pin directly behind the hole 
    so that the mouse cannot escape during the following actions.

    Next tap the swing-top cap into the mouse hole with the hammer. If the cap
    is not big enough to block the hole completely and permanently, take the 
    board instead and screw it to the front of the mouse hole using the two
    screws and the knife. Of course, you can use a screwdriver instead of a 
    knife. The ballpoint pen has fallen victim to animal welfare.
  \end{ShowOutput}
\end{Example}
%
\EndIndexGroup


\begin{Declaration}
  \Macro{raggedsection}%
  \Macro{raggedchapter}%
  \Macro{raggedpart}
\end{Declaration}%
In the standard classes\textnote{\KOMAScript{} vs. standard classes}, headings
are set as justified text. That means that hyphenated words can occur and
multi-line headings are stretched up to the text width. This approach is
rather uncommon in typography. {\KOMAScript} therefore sets the
headings left aligned with hanging indentation using \Macro{raggedsection}
with the default:
\begin{lstcode}
  \newcommand*{\raggedsection}{\raggedright}
\end{lstcode}
You can redefine this command with \Macro{renewcommand}.
\begin{Example}
  You prefer justified headings, so you write in the preamble of your document:
\begin{lstcode}
  \renewcommand*{\raggedsection}{}
\end{lstcode}
  or more compactly:
\begin{lstcode}
  \let\raggedsection\relax
\end{lstcode}
  You will get heading formatting which is very close to that of the standard
  classes. It will become even closer when you combine this change with the
  change to the 
  \DescRef{\LabelBase.fontelement.disposition}\IndexFontElement{disposition}
  element mentioned above.
\end{Example}

Because\ChangedAt{v3.15}{\Class{scrbook}\and \Class{scrreprt}} some users want
a different alignment for the \DescRef{\LabelBase.cmd.chapter} level than for
the other sectioning levels, you can change the
\DescRef{\LabelBase.cmd.chapter} justification separately by redefining
\Macro{raggedchapter}. By default, however, this command simply uses
\Macro{raggedsection}, so changing \Macro{raggedsection} indirectly affects
\Macro{raggedchapter}.

By default, part headings (\DescRef{\LabelBase.cmd.part}) are set horizontally
centred rather than ragged right. This formatting is performed by the
\Macro{raggedpart} statement, which has the default definition
\begin{lstcode}
  \let\raggedpart\centering
\end{lstcode}
You can also redefine this command using \Macro{renewcommand}.
\begin{Example}
  You want the headings for \DescRef{\LabelBase.cmd.part} to be formatted the 
  same as any other sectioning command. To do so, put
\begin{lstcode}
  \renewcommand*{\raggedpart}{\raggedsection}
\end{lstcode}
  in the preamble of your document. In\textnote{Hint!} this case, and unlike
  in the example above, we did not use \Macro{let} because \Macro{let} would
  set \Macro{raggedpart} to the underlying value of \Macro{raggedsection}.
  Subsequent changes to \Macro{raggedsection} would then not change the
  behaviour of \Macro{raggedpart}. By redefining with \Macro{renewcommand},
  \Macro{raggedpart} will use the current meaning of \Macro{raggedsection} at
  the time it is used rather than when it was redefined.
\end{Example}%
%
\EndIndexGroup


\begin{Declaration}
  \Macro{partformat}
  \Macro{chapterformat}
  \Macro{sectionformat}
  \Macro{subsectionformat}
  \Macro{subsubsectionformat}
  \Macro{paragraphformat}
  \Macro{subparagraphformat}
  \Macro{othersectionlevelsformat}\Parameter{sectioning name}\Parameter{}
                                  \Parameter{counter output}
  \Macro{IfUsePrefixLine}\Parameter{then code}\Parameter{else code}
  \Macro{autodot}
\end{Declaration}%
{\KOMAScript} adds another logical layer above \Macro{the\PName{sectioning
    name}} to format the sectioning numbers. The counters for each heading are
not merely output. They are formatted using the commands \Macro{partformat},
\Macro{chapterformat}, down to
\Macro{subparagraphformat}\ChangedAt{v3.17}{\Class{scrbook}\and
  \Class{scrreprt}\and \Class{scrartcl}}. Of course the \Macro{chapterformat}
command, like \Macro{thechapter}, does not exist in the \Class{scrartcl}
class, but only in the \Class{scrbook} and \Class{scrreprt}
classes.\OnlyAt{\Class{scrbook}\and\Class{scrreprt}}

As already explained for the \DescRef{\LabelBase.option.numbers}%
\important{\DescRef{\LabelBase.option.numbers}} option at the beginning of
this section (see \DescPageRef{\LabelBase.option.numbers}), {\KOMAScript}'s
handling of points in section numbers implements the rules given in
\cite{DUDEN}, which are standard in German-language typography, in the
\Macro{autodot} command. In all levels except for
\DescRef{\LabelBase.cmd.part}, a point is followed by a further
\Macro{enskip}.  This corresponds to a horizontal skip of 0.5\Unit{em}.

Since \KOMAScript~3.17\ChangedAt{v3.17}{\Class{scrbook}\and
  \Class{scrreprt}\and \Class{scrartcl}}, the command
\Macro{othersectionlevelsformat} is used only in rare circumstances, i.\,e.,
if the corresponding format command to a section command does not exist or is
\Macro{relax}. This should not happen for any section commands defined by
\KOMAScript{} itself. Therefore the command is no longer officially
documented. Nevertheless, if you select a compatibility level prior to
3.17\important{\OptionValueRef{\LabelBase}{version}{3.17}} (see option
\DescRef{\LabelBase.option.version},
\autoref{sec:\LabelBase.compatibilityOptions},
\DescPageRef{\LabelBase.option.version}), commands \Macro{sectionformat} down
to \Macro{subparagraphformat} are ignored by \KOMAScript{}. Instead, these
layers will continue to use \Macro{othersectionlevelsformat}.

You can redefine the formatting commands using \Macro{renewcommand} to fit
them to your personal needs. The following default definitions are used by the
{\KOMAScript} classes:
\begin{lstcode}
  \newcommand*{\partformat}{\partname~\thepart\autodot}
  \newcommand*{\chapterformat}{%
    \mbox{\chapappifchapterprefix{\nobreakspace}\thechapter
      \autodot\IfUsePrefixLine{}{\enskip}}}
  \newcommand*{\sectionformat}{\thesection\autodot\enskip}
  \newcommand*{\subsectionformat}{%
    \thesubsection\autodot\enskip}
  \newcommand*{\subsubsectionformat}{%
    \thesubsubsection\autodot\enskip}
  \newcommand*{\paragraphformat}{\theparagraph\autodot\enskip}
  \newcommand*{\subparagraphformat}{%
    \thesubparagraph\autodot\enskip}
  \newcommand*{\othersectionlevelsformat}[3]{%
    #3\autodot\enskip}
\end{lstcode}

Because\ChangedAt{v3.17}{\Class{scrbook}\and \Class{scrreprt}} it uses
\Macro{IfUsePrefixLine}, \Macro{chapterformat} should not be used outside of
\DescRef{\LabelBase.cmd.chapter}. \Macro{IfUsePrefixLine} is only valid inside
\KOMAScript{} sectioning commands. Within those commands, it executes the
\PName{then code} if a prefix line for the number is used and the \PName{else
code} otherwise.

Please also remember\textnote{Attention!} to replace \Macro{newcommand} with
\Macro{renewcommand} if you want to redefine one of the commands above.

\begin{Example}
  Suppose you do not want the word ``Part'' written in front of the part 
  number when you use \DescRef{\LabelBase.cmd.part}. You can put the
  following command in the preamble of your document:
\begin{lstcode}
  \renewcommand*{\partformat}{\thepart\autodot}
\end{lstcode}
  In fact, you could do without \Macro{autodot} here and insert a fixed point
  instead. Since \DescRef{\LabelBase.cmd.part} is numbered with Roman numerals,
  it must be followed by a point according to \cite{DUDEN}. However, by using
  \Macro{autodot} you retain the ability to use the
  \DescRef{\LabelBase.option.numbers}%
  \important{\DescRef{\LabelBase.option.numbers}} option
  \OptionValueRef{\LabelBase}{numbers}{endperiod} and thus deviate from
  the rule. You can find more details concerning class options on
  \DescPageRef{\LabelBase.option.numbers}.
  
  Another possibility is to place the section numbers in the left margin in
  such a way that the heading text is left aligned with the surrounding text.
  You can accomplish this with:
\begin{lstcode}
  \renewcommand*{\sectionformat}{%
    \makebox[0pt][r]{\thesection\autodot\enskip}}
  \renewcommand*{\subsectionformat}{%
    \makebox[0pt][r]{\thesubsection\autodot\enskip}}
  \renewcommand*{\subsubsectionformat}{%
    \makebox[0pt][r]{%
      \thesubsubsection\autodot\enskip}}
  \renewcommand*{\paragraphformat}{%
    \makebox[0pt][r]{\theparagraph\autodot\enskip}}
  \renewcommand*{\paragraphformat}{%
    \makebox[0pt][r]{%
      \thesubparagraph\autodot\enskip}}
\end{lstcode}
  The optional arguments of the \Macro{makebox} command require \LaTeX{} to
  create a zero-width box with right-justified content. As a result, the
  contents of the box are output to the left of the current position.%
  \iffree{}{ %Umbruchkorrekturtext
    This popular trick avoids the need first to measure the width of the
    number and the spacing, and then to change the current position
    appropriately with \Macro{hspace}.}
  You can find more about the optional arguments of \Macro{makebox} in
  \cite{latex:usrguide}.
\end{Example}

For formatting changes in the headings that go beyond merely formatting the
unit number, please refer to
\DescRef{maincls-experts.cmd.partlineswithprefixformat}%
\IndexCmd{partlineswithprefixformat},
\DescRef{maincls-experts.cmd.chapterlineswithprefixformat}%
\IndexCmd{chapterlineswithprefixformat} and
\DescRef{maincls-experts.cmd.chapterlinesformat}\IndexCmd{chapterlinesformat},
as well as
\DescRef{maincls-experts.cmd.sectionlinesformat}\IndexCmd{sectionlinesformat}
and its \DescRef{maincls-experts.cmd.sectioncatchphraseformat}%
\IndexCmd{sectioncatchphraseformat} format in
\autoref{sec:maincls-experts.sections}, starting from
\DescPageRef{maincls-experts.cmd.partlineswithprefixformat}.%
\EndIndexGroup


\begin{Declaration}
  \Macro{chapappifchapterprefix}\Parameter{additional text}%
  \Macro{chapapp}
\end{Declaration}%
These\ChangedAt{v2.8o}{\Class{scrbook}\and\Class{scrreprt}} two commands are
used internally by {\KOMAScript} and also made available to the user. Later,
you will see how to use them, for example to redefine other commands.%
\OnlyAt{\Class{scrbook}\and\Class{scrreprt}}

If you use the layout option \OptionValueRef{\LabelBase}{chapterprefix}{true}%
\important{\DescRef{\LabelBase.option.chapterprefix}} (see
\DescPageRef{\LabelBase.option.chapterprefix}), \Macro{chapappifchapterprefix}
outputs the word ``Chapter''\Index{chapter>heading} in the body of the text in
the current language, followed by \PName{additional text}. In the
appendix\Index{appendix}, the word ``Appendix'' in the current language is
output instead, followed by \PName{additional text}. If the option
\OptionValue{\LabelBase}{chapterprefix}{false} is set, then nothing is output.

The \Macro{chapapp} command always outputs the word ``Chapter'' or
``Appendix'',  regardless of the setting of the
\DescRef{\LabelBase.option.chapterprefix} option.

Since chapters only exist in the classes \Class{scrbook} and \Class{scrreprt},
these commands only exist in these classes.%
%
\EndIndexGroup


\begin{Declaration}
  \Macro{chaptermark}\Parameter{running head}%
  \Macro{addchapmark}\Parameter{running head}%
  \Macro{sectionmark}\Parameter{running head}%
  \Macro{addsecmark}\Parameter{running head}%
  \Macro{subsectionmark}\Parameter{running head}%
  \Macro{chaptermarkformat}%
  \Macro{sectionmarkformat}%
  \Macro{subsectionmarkformat}
\end{Declaration}%
\begin{Explain}%
  As mentioned in \autoref{sec:\LabelBase.pagestyle}, the 
  \DescRef{\LabelBase.pagestyle.headings}\IndexPagestyle{headings}%
  \important{\DescRef{\LabelBase.pagestyle.headings}} page style works with
  automatic running  heads\Index{running heads>automatic}. For this, the
  commands \Macro{chaptermark} and \Macro{sectionmark}, or \Macro{sectionmark}
  and \Macro{subsectionmark}, are defined accordingly. Every sectioning
  command (\DescRef{\LabelBase.cmd.chapter}, \DescRef{\LabelBase.cmd.section},
  etc.) automatically executes the corresponding \Macro{\dots mark} command.
  The parameter passed contains the text of the section heading%
  \Index{heading}. The corresponding section number is added automatically in
  the \Macro{\dots mark} command. The formatting is done according to the 
  sectioning level with one of the three commands \Macro{chaptermarkformat}, 
  \Macro{sectionmarkformat}, or \Macro{subsectionmarkformat}.

  Note\textnote{Attention!} that the running heads of
  \DescRef{\LabelBase.cmd.addchap}\IndexCmd{addchap} and
  \DescRef{\LabelBase.cmd.addsec}\IndexCmd{addsec} are also based on
  \Macro{chaptermark} and \Macro{sectionmark}. However, the
  \DescRef{\LabelBase.counter.secnumdepth} counter is set locally to a value
  that switches off the numbering of chapters or sections. You should consider
  this, for example, if you redefine \Macro{chaptermark} or
  \Macro{sectionmark} (see \Macro{ifnumbered} on
  \DescPageRef{\LabelBase.cmd.ifnumbered}). The starred variants
  \DescRef{\LabelBase.cmd.addchap*} and \DescRef{\LabelBase.cmd.addsec*} use
  additional commands \Macro{addchapmark} and \Macro{addsecmark} that are also
  defined based on \Macro{chaptermark} and \Macro{sectionmark} with local
  changes of \DescRef{\LabelBase.counter.secnumdepth}.

  Although\OnlyAt{\Class{scrartcl}} there is no \Macro{chaptermark} or
  \Macro{chaptermarkformat} command in \Class{scrartcl}, there are two
  commands, \Macro{subsectionmark} and \Macro{subsectionmarkformat}, which
  exist only in \Class{scrartcl}. However using the
  \hyperref[cha:scrlayer-scrpage]{\Package{scrlayer-scrpage}}%
  \IndexPackage{scrlayer-scrpage} package changes this (see
  \autoref{cha:scrlayer-scrpage}).
\end{Explain}
Just as numbers in the sectioning-command headers are formatted with
\DescRef{\LabelBase.cmd.partformat} down to
\DescRef{\LabelBase.cmd.subparagraphformat}, \Macro{chaptermarkformat},
\Macro{sectionmarkformat}, and \Macro{subsectionmarkformat} define the
formatting of the sectioning numbers in the automatic running heads. They can
be adapted to your personal needs with \Macro{renewcommand}. The original
definitions for the {\KOMAScript} classes are:
\begin{lstcode}
  \newcommand*{\chaptermarkformat}{%
    \chapappifchapterprefix{\ }\thechapter\autodot\enskip}
  \newcommand*{\sectionmarkformat}{%
    \thesection\autodot\enskip}
  \newcommand*{\subsectionmarkformat}{%
    \thesubsection\autodot\enskip}
\end{lstcode}
\begin{Example}
  Suppose you want the word ``Chapter'' to precede the chapter number in the 
  running head. You could put the following definition in the preamble of your 
  document:
\begin{lstcode}
  \renewcommand*{\chaptermarkformat}{%
    \chapapp~\thechapter\autodot\enskip}
\end{lstcode}
\end{Example}
As you can see, both \DescRef{\LabelBase.cmd.chapapp} and
\DescRef{\LabelBase.cmd.chapappifchapterprefix}, explained above, are used
here.%
\EndIndexGroup


\begin{Declaration}
  \Counter{secnumdepth}
  \Macro{partnumdepth}
  \Macro{chapternumdepth}
  \Macro{sectionnumdepth}
  \Macro{subsectionnumdepth}
  \Macro{subsubsectionnumdepth}
  \Macro{paragraphnumdepth}
  \Macro{subparagraphnumdepth}
\end{Declaration}%
\BeginIndex{}{Nummerierung}%
Normally, the \Class{scrbook}\IndexClass{scrbook} and
\Class{scrreport}\IndexClass{scrreprt} classes number the section headings
from \DescRef{\LabelBase.cmd.part}\IndexCmd{part}\IndexCmd{chapter}%
\IndexCmd{section} down to
\DescRef{\LabelBase.cmd.subsection}\IndexCmd{subsection} and the
\Class{scrartcl}\IndexClass{scrartcl} class numbers them from
\DescRef{\LabelBase.cmd.part} down to
\DescRef{\LabelBase.cmd.subsubsection}\IndexCmd{subsubsection}. The actual
depth to which headings are numbered is controlled by the \LaTeX{} counter
\Counter{secnumdepth}.

To\ChangedAt{v3.12}{\Class{scrbook}\and \Class{scrreprt}\and \Class{scrartcl}}
save users the trouble of having to remember abstract numbers, the commands 
\Macro{partnumdepth} to \Macro{subparagraphnumdepth} return the appropriate 
number for the section level in their name.

\begin{Example}
  For a book project, you want the section levels from part down to section
  to be numbered. To achieve this, you have to set counter
  \Counter{secnumdepth} to the value represented by \Macro{sectionnumdepth} in
  the preamble of your document:
\begin{lstcode}
  \setcounter{secnumdepth}{\sectionnumdepth}
\end{lstcode}
\end{Example}
No provision is made for redefining these commands. Doing so could lead to 
unexpected results, not only with \KOMAScript{} but also with third party 
packages. Thus, you should never redefine them.

Do\textnote{Attention!} not confuse the \Counter{secnumdepth} and
\DescRef{\LabelBase.counter.tocdepth} counters (see
\autoref{sec:\LabelBase.toc}, \DescPageRef{\LabelBase.counter.tocdepth}).
Depending on the class you are using, the meaning of the values of the
\Counter{secnumdepth} and \DescRef{\LabelBase.counter.tocdepth} counters may
differ for the same section level.%
%
\EndIndexGroup

\begin{Declaration}
  \Macro{Ifnumbered}\Parameter{section level}%
  \Parameter{then code}\Parameter{else code}%
  \Macro{Ifunnumbered}\Parameter{section level}%
  \Parameter{then code}\Parameter{else code}
\end{Declaration}
The\ChangedAt{v3.12}{\Class{scrbook}\and \Class{scrreprt}\and
  \Class{scrartcl}}\ChangedAt{v3.28}{\Class{scrbook}\and \Class{scrreprt}\and
  \Class{scrartcl}} commands \Macro{Ifnumbered} and \Macro{Ifunnumbered}
determine which section-level headings are numbered, using the technique
described above, and execute code depending on whether a \PName{section level}
is numbered or not. If a \PName{section level} is numbered with the current
settings, \Macro{Ifnumbered} executes the \PName{then code}. If the section
level is unnumbered, the \PName{else code} is executed. The
\Macro{Ifunnumbered} command behaves in exactly the opposite manner, executing
the \PName{then code} if the current level is unnumbered and the \PName{else
  code} if it is. The \PName{section level} parameter is simply the \LaTeX{}
name of a section like \PValue{part}, \PValue{chapter}, \PValue{section},
\PValue{subsection}, \PValue{subsubsection}, \PValue{paragraph} oder
\PValue{subparagraph}.\PValue{part}, \PValue{chapter}, \PValue{section},
\PValue{subsection}, \PValue{subsubsection}, \PValue{paragraph}, or
\PValue{subparagraph}.

\KOMAScript{} itself uses these tests, for example, in the definition of
\DescRef{\LabelBase.cmd.chaptermark}\IndexCmd{chaptermark} in the
\DescRef{\LabelBase.pagestyle.headings}\IndexPagestyle{headings} page style.
This indirectly ensures that headings inserted by
\DescRef{\LabelBase.cmd.addchap} do not set a number inside the running head
(see also \DescRef{\LabelBase.cmd.addchapmark},
\DescPageRef{\LabelBase.cmd.addchapmark}).%
\EndIndexGroup


\begin{Declaration}
  \Macro{setpartpreamble}%
  \OParameter{position}\OParameter{width}\Parameter{preamble}
  \Macro{setchapterpreamble}%
  \OParameter{position}\OParameter{width}\Parameter{preamble}
\end{Declaration}%
Parts\OnlyAt{\Class{scrbook}\and
  \Class{scrreprt}}\Index{part>preamble}\Index{chapter>preamble} and chapters
in {\KOMAScript} can be given a \PName{preamble}. This is particularly useful
when you are using a two-column format with the class option
\Option{twocolumn}\IndexOption{twocolumn}, since the heading and the
\PName{preamble} are always set in a one-column layout. The \PName{preamble}
can contain more than one paragraph. The command to set the \PName{preamble}
must come before the respective \DescRef{\LabelBase.cmd.part},
\DescRef{\LabelBase.cmd.addpart}, \DescRef{\LabelBase.cmd.chapter}, or
\DescRef{\LabelBase.cmd.addchap} command.
\begin{Example}
  You are writing a report about the condition of a company. You organize the 
  report in such a way that every department gets its own partial report.  
  Each of these parts should be introduced by an abstract on the corresponding 
  title page. You could write the following:
\begin{lstcode}
  \setpartpreamble{%
    \begin{abstract}
      This is a filler text. It serves merely to demonstrate the
      capabilities of {\KOMAScript}. If you read this text, you will
      get no information.
    \end{abstract}
  }
  \part{Department for Word Processing}
\end{lstcode}
  Depending on the settings for the heading font size\Index{heading}
  (see \DescPageRef{\LabelBase.option.headings}) and the options for
  the \DescRef{\LabelBase.env.abstract} environment\IndexEnv{abstract} (see
  \autoref{sec:\LabelBase.abstract},
  \DescPageRef{\LabelBase.option.abstract}), the result will look something 
  like this:
  \begin{ShowOutput}\centering
    {\LARGE\usekomafont{disposition} Part III.\par\vspace{14pt}}
    {\LARGE\usekomafont{disposition} Department for Word Processing\strut\par}
    \begin{quote}\small
      \vspace{4ex}
      \begin{center}
        \usekomafont{disposition}\abstractname
      \end{center}
      \vspace{2ex}       
      This is a filler text. It serves merely to demonstrate the
      capabilities of {\KOMAScript}. If you read this text, you will
      get no information.
    \end{quote}
  \end{ShowOutput}
\end{Example}
Please\textnote{Attention!} note that \emph{you} are responsible for the
spacing between the heading, preamble, and the following text. Note also that
there is no \DescRef{\LabelBase.env.abstract} environment in the
\Class{scrbook} class (see \autoref{sec:\LabelBase.abstract},
\DescPageRef{\LabelBase.env.abstract}).

The\ChangedAt{v2.8p}{\Class{scrbook}\and\Class{scrreprt}} first optional 
argument, \PName{position}, determines the position at which the preamble 
is placed with the help of one or two letters. For vertical placement 
there are two possibilities at present:
\begin{labeling}[~--]{\quad\PValue{o}}\itemsep=0pt
\item [\quad\texttt{o}] above the heading
\item [\quad\texttt{u}] below the heading
\end{labeling}
You can therefore put one preamble above and another below a heading. For
horizontal placement you have three options:
\begin{labeling}[~--]{\quad\PValue{o}}\itemsep=0pt
\item [\quad\texttt{l}] left-aligned
\item [\quad\texttt{r}] right-aligned
\item [\quad\texttt{c}] centred
\end{labeling}
This setting does not affect the alignment of the text in the
\PName{preamble}. Instead, it aligns the box that contains the preamble. The
width of this box is determined by the second optional argument,
\PName{width}. If you omit this second argument, the box uses the full text
width. In that case, the option for horizontal positioning has no effect. You
can combine exactly one letter from the vertical with one letter from the
horizontal positioning.

A typical use for \Macro{setchapterpreamble} would be something like an
epigraph, a wise saying, or a dictum. The
\DescRef{\LabelBase.cmd.dictum}\IndexCmd{dictum} command, which you can use
for this purpose, is described in the next section. You will also find an
example there.

Please note\textnote{Attention!} that a \PName{preamble} placed above the
heading is set within the existing vertical space above the heading. The
heading will not be moved down. You are therefore responsible for ensuring
that the \PName{preamble} is not too large and that the space above the
heading is sufficient. See also the \Option{beforeskip} setting for
\DescRef{maincls-experts.cmd.RedeclareSectionCommand} in
\autoref{sec:maincls-experts.sections},
\autoref{tab:maincls-experts.declarechapterstyle.keys},
\autopageref{tab:maincls-experts.declarechapterstyle.keys}.
%
%
\EndIndexGroup
%
\EndIndexGroup


\LoadCommonFile{dictum}% \section{Dicta}

\LoadCommonFile{lists}% \section{Lists}

\section{Mathematics}
\seclabel{math}%
\BeginIndexGroup
\BeginIndex{}{equations}%
\BeginIndex{}{formulas}%
\BeginIndex{}{mathematics}%

\KOMAScript{} classes do not provide their own environments for formulas,
systems of equations, or similar mathematical elements. Instead, \KOMAScript{}
relies fully on the maths features of the \LaTeX{} kernel. This also applies to
the the options \DescRef{\LabelBase.option.leqno} and
\DescRef{\LabelBase.option.fleqn}.

You will not find a description of the maths environments of the \LaTeX{}
kernel here. If you want to use
\Environment{displaymath}\IndexEnv{displaymath},
\Environment{equation}\IndexEnv{equation}, or
\Environment{eqnarray}\IndexEnv{eqnarray} you should read an introduction to
\LaTeX{} like \cite{lshort}. But\textnote{Hint!} if you want more than very
simple mathematics, you should use the \Package{amsmath} package (see
\cite{package:amsmath}).

\begin{Declaration}
  \Option{leqno}
\end{Declaration}%
Equations are normally numbered\Index{equation>number} on the right. The
\Option{leqno} option loads the standard option file \File{leqno.clo}. The
equations are then numbered on the left. You\textnote{Attention!} must use
this option as an optional argument of \DescRef{\LabelBase.cmd.documentclass}.
Using it as an argument of \DescRef{\LabelBase.cmd.KOMAoptions} or
\DescRef{\LabelBase.cmd.KOMAoption} is not supported. The latter
would not make sense because the recommended maths package \Package{amsmath}
can only respond to this option at load time and does not react to run-time
changes of the option.
%
\EndIndexGroup


\begin{Declaration}
  \Option{fleqn}
\end{Declaration}%
Displayed equations\Index{equation>alignment} are normally centred. The
standard option \Option{fleqn} loads the standard option file
\File{fleqn.clo}. Displayed equations are then left-justified.
You\textnote{Attention!} must use this option as an optional argument of
\DescRef{\LabelBase.cmd.documentclass}. Using it as an argument of
\DescRef{\LabelBase.cmd.KOMAoptions} or \DescRef{\LabelBase.cmd.KOMAoption} is
not supported. The latter would not make sense because the recommended maths
package \Package{amsmath} can only respond to this option at load time and
does not react to run-time changes of the option.
%
\EndIndexGroup
%
\EndIndexGroup


\section{Floating Environments for Tables and Figures}
\seclabel{floats}
\BeginIndexGroup
\BeginIndex{}{floats}%
\BeginIndex{}{tables}%
\BeginIndex{}{figures}%

With the floating environments, {\LaTeX} offers a powerful and convenient
mechanism to arrange figures\Index{figure} and tables\Index{table} 
automatically.
% Im Deutschen ist hier ein Absatz, den wir nicht uebernehmen!
Frequently,\textnote{Attention!} beginners do not properly understand these
floating environments\Index[indexmain]{floating environments}. They often ask
to specify the exact position of a table or figure within the text. However,
this is usually unnecessary, since the text will contain references to these
floating environments. It is also not sensible because such an object can only
be set on the page if there is enough space left for it. If this is not the
case, the object would have to be shifted onto the next page, possibly leaving
a huge empty space on the previous page.

Often\textnote{Attention!} a document will use the same optional argument to
position every floating object. This also makes no sense. In such cases, you
should instead change the default value globally. For more details, see
\cite{DANTE:FAQ}.

One\textnote{Attention!} final, important note before starting this section:
most of mechanisms described here, which extend the capabilities of the
standard classes, no longer work correctly when used with packages that modify
the appearance of figure and table captions. This should be self-evident, but
it is often overlooked.


\begin{Declaration}
  \OptionVName{captions}{setting}
\end{Declaration}
In the standard classes, the titles of floating environments, which are
formatted with the \DescRef{\LabelBase.cmd.caption}\IndexCmd{caption} command
(see below), are set off from the float with vertical spacing appropriate for
putting the caption beneath the float, like a signature. They also distinguish
between one-line and multi-line captions. One-line captions are centred while
multi-line captions are left-justified.

\leavevmode\LabelOptionValue{captions}{tableheading}\nobreak%
For tables, however, you want the caption%
\important{\Option{captions}{tableheading}} to appear as a heading instead of
a signature. That's because tables can span multiple pages. With such tables,
the reader wants to know the purpose of the table on the first page.
Furthermore, tables are usually read row by row, from top to bottom. So there
are normally at least two good reasons to provide all tables with headings.
\KOMAScript{} therefore offers the \OptionValue{captions}{tableheading}%
\IndexOption{captions~=\textKValue{tableheading}} option, which changes the
formatting of table captions for use above the table.

Note\textnote{Attention!} that multi-page tabulars cannot use a floating
environment. To have an automatic page break in a tabular you need an
additional package like \Package{longtable}\IndexPackage{longtable} (see
\cite{package:longtable}) or \Package{supertabular}\IndexPackage{supertabular}
(see \cite{package:supertabular}).

You\important{\OptionValue{captions}{tablesignature}} can switch back to the
default caption formatting using \OptionValue{captions}{tablesignature}%
\IndexOption{captions~=\textKValue{tablesignature}}. Note\textnote{Attention!}
that these options change only the formatting, not the actual position of the
caption. Whether the caption is placed above or below a float depends solely
upon where you use the \DescRef{\LabelBase.cmd.caption} command inside float
environment. However, this can change when using the \Package{float}%
\IndexPackage{float} package with the \Macro{restylefloats} command (see
\cite{package:float}).

Of course\important{\OptionValue{captions}{figureheading}\\
  \OptionValue{captions}{figuresignature}}, corresponding functions
exist\ChangedAt{v3.09}{\Class{scrbook}\and \Class{scrreprt}\and
  \Class{scrartcl}} for figures using the options
\OptionValue{captions}{figureheading}%
\IndexOption{captions~=\textKValue{figureheading}} and
\OptionValue{captions}{figuresignature}%
\IndexOption{captions~=\textKValue{figuresignature}}. However, figures such as
photos tend to be viewed as a whole, and a diagram or graph will mostly be
examined starting from the lower left. Therefore, it only rarely makes sense
to change the caption format for figures alone from signatures to headings.

Sometimes, however, all floating environments should use headings. Therefore
\KOMAScript{} provides\ChangedAt{v3.09}{\Class{scrbook}\and
  \Class{scrreprt}\and \Class{scrartcl}}
options\important{\OptionValue{captions}{heading}\\
	\OptionValue{captions}{signature}}
\OptionValue{captions}{heading}\IndexOption{captions~=\textKValue{heading}}
and \OptionValue{captions}{signature}%
\IndexOption{captions~=\textKValue{signature}} to switch the format of every
floating environment. These options can also be used inside a floating
environment.

Please note\textnote{Achtung!}\OnlyAt{\Package{float}}\IndexPackage{float}
when using the \Package{float} package that the settings for signatures or
headings will no longer work once \Macro{restylefloat} is applied to tables or
figures. For details about the \Package{float} package and
\Macro{restylefloat}, please refer to \cite{package:float}. This also applies
to \DescRef{\LabelBase.cmd.caption} within new floating environments defined
with \Package{float}. You can find additional support which \KOMAScript{}
provides when using the \Package{float} package in the explanation of
\PValue{komaabove} (see \DescPageRef{\LabelBase.floatstyle.komaabove}). As an
alternative to \Package{float}, you can also consult
\DescRef{\LabelBase.cmd.captionof} (see
\DescPageRef{\LabelBase.cmd.captionof}) and
\DescRef{tocbasic.cmd.DeclareNewTOC} (see
\DescPageRef{tocbasic.cmd.DeclareNewTOC}). Additionally, when using
\Package{float}, the
\hyperref[cha:scrhack]{\Package{scrhack}}\IndexPackage{scrhack}%
\important{\hyperref[cha:scrhack]{\Package{scrhack}}} package is
expressly recommended (see \autoref{cha:scrhack} from
\autopageref{cha:scrhack} in \autoref{part:forExperts}).

\leavevmode
\LabelOptionValue{captions}{nooneline}\nobreak
\LabelOptionValue{captions}{oneline}\nobreak
Furthermore\important{\OptionValue{captions}{nooneline}}, \KOMAScript{} offers
the option to disable the distinction between one-line and multi-line captions
using the \OptionValue{captions}{nooneline}%
\IndexOption{captions~=\textKValue{nooneline}}%
\important{\OptionValue{captions}{nooneline}} option. This can be useful, for 
example, if you do not want one-line captions to be centred. The default, 
where one-line captions are centred, corresponds to 
\OptionValue{captions}{oneline}.

Another special feature of \KOMAScript{} is the ability to put the caption
beside the floating object rather than above or below it. For this, you need
the environment\important{\DescRef{\LabelBase.env.captionbeside}}
\DescRef{\LabelBase.env.captionbeside}\IndexEnv{captionbeside}, which is
explained starting on \DescPageRef{\LabelBase.env.captionbeside}. The settings
for this environment can also be changed with the \Option{caption} option. You
can find the available values for the corresponding \PName{settings} in
\autoref{tab:\LabelBase.captions}.
%
\begin{desclist}
  \desccaption[{Available values for the \Option{captions}} option]{%
    Available values for the \Option{captions} option for setting formatting of
    captions as headings or signatures in floating environments%
    \label{tab:\LabelBase.captions}%
  }{%
    Available values for the \Option{captions} option (\emph{continued})%
  }%
  \entry{\PValue{bottombeside}, \PValue{besidebottom}}{%
    Captions for the \DescRef{\LabelBase.env.captionbeside} environment (see 
    \autoref{sec:\LabelBase.floats}, 
    \DescPageRef{\LabelBase.env.captionbeside}) are vertically aligned with 
    the bottommost baseline of the contents of the floating environment.%
    \IndexOption{captions~=\textKValue{bottombeside}}}%
  \entry{\PValue{centeredbeside}, \PValue{besidecentered},
    \PValue{middlebeside}, \PValue{besidemiddle}}{%
    Captions for the \DescRef{\LabelBase.env.captionbeside} environment (see 
    \autoref{sec:\LabelBase.floats},
    \DescPageRef{\LabelBase.env.captionbeside}) are vertically aligned with
    the center of the contents of the floating environment.%
    \IndexOption{captions~=\textKValue{centeredbeside}}}%
  \entry{\PValue{figureheading}, \PValue{figureabove}, \PValue{abovefigure},
    \PValue{topatfigure}%
    \IndexOption{captions~=\textKValue{figureheading}}}{%
    \ChangedAt{v3.09}{\Class{scrbook}\and \Class{scrreprt}\and
      \Class{scrartcl}}%
    Captions for figures use heading format\,---\,possibly deviating from
    \OptionValue{captions}{signature}.%
  }%
  \entry{\PValue{figuresignature}, \PValue{belowfigure},
    \PValue{bottomatfiggure}%
    \IndexOption{captions~=\textKValue{figuresignature}}}{%
    \ChangedAt{v3.09}{\Class{scrbook}\and \Class{scrreprt}\and
      \Class{scrartcl}}%
    Captions for figures use signature format\,---\,possibly deviating 
    from \OptionValue{captions}{headings}.%
  }%
  \entry{\PValue{heading}, \PValue{above}, \PValue{top}%
    \IndexOption{captions~=\textKValue{heading}}}{%
    \ChangedAt{v3.09}{\Class{scrbook}\and \Class{scrreprt}\and
      \Class{scrartcl}} Captions for floating environments use formatting
    suitable for use in a heading. This setting does not control whether they 
    are placed at the top or the bottom of the object. This option also 
    implies \OptionValue{captions}{tableheading} and
    \OptionValue{captions}{figureheading}.%
  }%
  \entry{\PValue{innerbeside},
    \PValue{besideinner}\IndexOption{captions~=\textKValue{innerbeside}}}{%
    Captions for the \DescRef{\LabelBase.env.captionbeside} environment (see
    \autoref{sec:\LabelBase.floats},
    \DescPageRef{\LabelBase.env.captionbeside}) are placed inside of and next 
    to the contents of the environment in two-sided printing. In one-sided 
    printing, \OptionValue{captions}{leftbeside} is used.%
  }%
  \entry{\PValue{leftbeside},
    \PValue{besideleft}\IndexOption{captions~=\textKValue{leftbeside}}}{%
    Captions for the \DescRef{\LabelBase.env.captionbeside} environment (see
    \autoref{sec:\LabelBase.floats},
    \DescPageRef{\LabelBase.env.captionbeside}) are placed to the left of
    the contents of the floating environment.%
  }%
  \pventry{nooneline}{%
    Single-line captions are handled the same as multi-line captions.%
    \IndexOption{captions~=\textKValue{nooneline}}}%
  \pventry{oneline}{%
    Single-line captions are centred horizontally.%
    \IndexOption{captions~=\textKValue{oneline}}}%
  \entry{\PValue{outerbeside}, \PValue{besideouter}%
    \IndexOption{captions~=\textKValue{outerbeside}}}{%
    Captions for the \DescRef{\LabelBase.env.captionbeside} environment (see
    \autoref{sec:\LabelBase.floats},
    \DescPageRef{\LabelBase.env.captionbeside}) are placed outside of and
    next to the contents of the environment in two-sided printing. In
    one-sided printing, \OptionValue{captions}{rightbeside} is used.%
  }%
  \entry{\PValue{rightbeside}, \PValue{besideright}%
    \IndexOption{captions~=\textKValue{rightbeside}}}{%
    Captions for the \DescRef{\LabelBase.env.captionbeside} environment (see
    \autoref{sec:\LabelBase.floats},
    \DescPageRef{\LabelBase.env.captionbeside}) are placed to the right of the
    contents of the floating environment.%
  }%
  \entry{\PValue{signature}, \PValue{below}, \PValue{bot}, \PValue{bottom}%
    \IndexOption{captions~=\textKValue{signature}}}{%
    \ChangedAt{v3.09}{\Class{scrbook}\and \Class{scrreprt}\and
      \Class{scrartcl}}%
    Captions for floating environments use signature format. This setting 
    does not control whether they are placed at the top or the bottom of the 
    object. This options also implies \OptionValue{captions}{tablesignature} 
    and \OptionValue{captions}{figuresignature}.%
  }%
  \entry{\PValue{tableheading}, \PValue{tableabove}, \PValue{abovetable},
    \PValue{abovetabular}, \PValue{topattable}%
    \IndexOption{captions~=\textKValue{tableheading}}}{%
    Captions for tables use heading format\,---\,possibly deviating from
    \OptionValue{captions}{signature}.%
  }%
  \entry{\PValue{tablesignature}, \PValue{belowtable}, \PValue{belowtabular},
    \PValue{bottomattable}%
    \IndexOption{captions~=\textKValue{tablesignature}}}{%
    Captions for tables use signature format\,---\,possibly deviating
    from \OptionValue{captions}{heading}.%
  }%
  \entry{\PValue{topbeside}, \PValue{besidetop}}{%
    Captions for the \DescRef{\LabelBase.env.captionbeside} environment (see 
    \autoref{sec:\LabelBase.floats},
    \DescPageRef{\LabelBase.env.captionbeside}) are vertically aligned to the
    baseline at the top of the floating environment.%
    \IndexOption{captions~=\textKValue{topbeside}}}%
\end{desclist}%
%
\EndIndexGroup


\begin{Declaration}
  \Macro{caption}\OParameter{entry}\Parameter{title}
  \Macro{captionbelow}\OParameter{entry}\Parameter{title}
  \Macro{captionabove}\OParameter{entry}\Parameter{title}
\end{Declaration}%
In the standard classes, tables and figures are given captions with the 
\Macro{caption} command placed below the table or figure. For figures, this is 
generally correct. For tables, opinions differ as to whether captions should 
be placed above the table or, consistent with captions of 
figures\Index{caption>of figure}, below it\Index{caption>of table}. Therefore
{\KOMAScript}\textnote{\KOMAScript{} vs. standard classes},
unlike the standard classes, offers \Macro{captionbelow} for captions
below and \Macro{captionabove} for captions above tables or
figures.

For tables and figures, or in general for all floating environments, you can
control the behaviour of \Macro{caption} with the
\DescRef{\LabelBase.option.captions}\IndexOption{captions} option described at
the beginning of this section. For compatibility reasons, the default
behaviour of \Macro{caption} for all floating environments is like
\Macro{captionbelow}. However, you\textnote{Hint!} should use the
\OptionValueRef{\LabelBase}{captions}{tableheading} option, which switches the
behaviour of \Macro{caption} inside table environments to
\Macro{captionabove}. Alternatively, you can use \Macro{captionabove} instead
of \Macro{caption} inside every \Environment{table} environment.
%
\begin{Example}
  Instead of using captions below a table, you want to place your
  captions above it\Index{table>caption}, because you have tables
  which span more then one page. In the standard classes you could
  only write:
\begin{lstcode}
  \begin{table}
    \caption{This is an example table}
    \begin{tabular}{llll}
      This & is & an & example.\\\hline
      This & is & an & example.\\
      This & is & an & example.
    \end{tabular}
  \end{table}
\end{lstcode}
  Then you would get this unsatisfying result:
  \begin{ShowOutput}\centering
    {\usekomafont{caption}{\usekomafont{captionlabel}\tablename~30.2:}
      This is an example table.}\\
        \begin{tabular}{llll}
      This & is & an & example.\\\hline
      This & is & an & example.\\
      This & is & an & example.
        \end{tabular}
\end{ShowOutput}
  Using {\KOMAScript} you write instead:
\begin{lstcode}
  \begin{table}
    \captionabove{This is just an example table}
    \begin{tabular}{llll}
      This & is & an & example.\\\hline
      This & is & an & example.\\
      This & is & an & example.
    \end{tabular}
  \end{table}
\end{lstcode}
  Then you get:
  \begin{ShowOutput}\centering
    {\usekomafont{caption}{\usekomafont{captionlabel}\tablename~30.2:}
      This is just an example table}\\\vskip\abovecaptionskip
    \begin{tabular}{llll}
      This & is & an & example.\\\hline
      This & is & an & example.\\
      This & is & an & example.
    \end{tabular}
  \end{ShowOutput}
  Since you want all your tables typeset with captions above, you could
  of course use the \OptionValueRef{\LabelBase}{captions}{tableheading} option
  instead (see \DescPageRef{\LabelBase.option.captions.tableheading}). Then
  you can use \Macro{caption} as you would in the standard classes. You will
  get the same result as with \Macro{captionabove}.
\end{Example}

\BeginIndex[indexother]{}{font>style}%
\BeginIndex{FontElement}{caption}\LabelFontElement{caption}%
\BeginIndex{FontElement}{captionlabel}\LabelFontElement{captionlabel}%
The font style\ChangedAt{v2.8p}{%
  \Class{scrbook}\and\Class{scrreprt}\and\Class{scrartcl}} for the description
and the label\,---\,``Figure'' or ``Table'', followed by the number and the
delimiter\,---\,can be changed with the commands
\DescRef{\LabelBase.cmd.setkomafont} and
\DescRef{\LabelBase.cmd.addtokomafont} (see
\autoref{sec:\LabelBase.textmarkup},
\DescPageRef{\LabelBase.cmd.setkomafont}). The respective elements for this
are \FontElement{caption}\important[i]{\FontElement{caption}\\
  \FontElement{captionlabel}} and \FontElement{captionlabel} (see
\autoref{tab:\LabelBase.fontelements},
\autopageref{tab:\LabelBase.fontelements}). The font style for the element
\FontElement{caption} is applied to the element \FontElement{captionlabel}
before the font style of \FontElement{captionlabel} itself is applied. The
default settings are listed in \autoref{tab:\LabelBase.captionFont}.
%
\begin{table}
%  \centering%
%  \caption
  \begin{captionbeside}
  {Font defaults for the elements of figure or table captions}
  [l]
  \begin{tabular}[t]{ll}
    \toprule
    element & default \\
    \midrule
    \DescRef{\LabelBase.fontelement.caption} & \Macro{normalfont} \\
    \DescRef{\LabelBase.fontelement.captionlabel} & \Macro{normalfont}\\
    \bottomrule
  \end{tabular}
  \end{captionbeside}
  \label{tab:\LabelBase.captionFont}
\end{table}
%
\begin{Example}
  You want the table and figure descriptions typeset in a smaller font
  size. Thus you could write the following in the preamble of your
  document:
\begin{lstcode}
  \addtokomafont{caption}{\small}
\end{lstcode}
  Furthermore, you would like the labels to be printed in sans serif and
  bold. You add:
\begin{lstcode}
  \setkomafont{captionlabel}{\sffamily\bfseries}
\end{lstcode}
  As you can see, simple extensions of the default definitions are
  possible.
\end{Example}%
\EndIndexGroup


\begin{Declaration}
  \Macro{captionof}\Parameter{float type}\OParameter{entry}%
  \Parameter{title}%
  \Macro{captionbelowof}\Parameter{float type}\OParameter{entry}%
  \Parameter{title}%
  \Macro{captionaboveof}\Parameter{float type}\OParameter{entry}%
  \Parameter{title}
\end{Declaration}%
Like\ChangedAt{v3.05}{\Class{scrbook}\and \Class{scrreprt}\and
	\Class{scrartcl}} the \Package{caption}\IndexPackage{caption} and
\Package{capt-of}\IndexPackage{capt-of} packages, \KOMAScript{} offers the
\Macro{captionof} command, with which you can put a caption for a floating
environment, together with an entry in the corresponding environment list,
outside of the floating environment or even in a different floating
environment. Unlike \DescRef{\LabelBase.cmd.caption}, the type of floating
environment must be specified as the first parameter.

In addition, \KOMAScript{} also provides the commands
\Macro{captionaboveof}\ChangedAt{v3.09}{\Class{scrbook}\and
  \Class{scrreprt}\and \Class{scrartcl}} and \Macro{captionbelowof}. These are
like \DescRef{\LabelBase.cmd.captionabove} and
\DescRef{\LabelBase.cmd.captionbelow} but with the additional features and
parameter of \Macro{captionof}.

Of course\ChangedAt{v3.09a}{\Class{scrbook}\and \Class{scrreprt}\and 
  \Class{scrartcl}} \Macro{captionof} takes into account the \PValue{heading}
and \PValue{signature} settings of the \DescRef{\LabelBase.option.captions}
option. But this feature may be lost if you load the
\Package{capt-of}\IndexPackage{capt-of} or
\Package{caption}\IndexPackage{caption} packages. When using
\Package{caption}, you must follow the instructions for that package (see
\cite{package:caption})!
\begin{Example}
  Suppose you want to create a floating object with a table and a figure next 
  to each other. Since there are no mixed floating environments, you primarily
  use a \Environment{figure} environment:
\begin{lstcode}
  \begin{figure}
    \begin{minipage}{.5\linewidth}
      \centering
      \rule{4cm}{5cm}
      \caption{A rectangle}\label{fig:rechteck}
    \end{minipage}%
    \begin{minipage}{.5\linewidth}
      \centering
      \captionaboveof{table}
      [Measure of the rectangle in
        figure~\ref{fig:rechteck}]%
      {Measure of the rectangle}
      \label{tab:rechteck}
      \begin{tabular}{ll}
        Width:  & 4\,cm\\
        Height: & 5\,cm
      \end{tabular}
    \end{minipage}
  \end{figure}
\end{lstcode}
  Two \Environment{minipage} environments were used to place the figure 
  and the table side by side. The\textnote{Attention!} percent sign after the 
  end of the first \Environment{minipage} is important. Without it, an 
  additional space would appear between the \Environment{minipage}
  environments.

  The figure caption was created with \DescRef{\LabelBase.cmd.caption}. The 
  table caption was created with \Macro{captionaboveof} with \PValue{table}
  as the first argument. Because of this, \KOMAScript{} knows that this is a 
  table caption even though it is inside the \Environment{figure} environment.

  The optional argument of \Macro{captionaboveof} creates an entry in the
  list of tables. Without the optional argument, the caption specified in the 
  final mandatory argument would have been used for the list of tables too. 
  Although this caption text is sufficient for the environment itself, it 
  would not be very meaningful in the list of tables. Therefore, a different 
  title is used for the list of tables using the optional argument.
  \autoref{fig:\LabelBase.captionaboveof} shows the result of the example code.%
\begin{figure}
%  \centering
%  \caption
  \KOMAoptions{captions=bottombeside}%
  \setcapindent{0pt}%
  \begin{captionbeside}
    [{Example: Using \Macro{captionaboveof} inside another floating
      environment}]
    {Example: Using \Macro{captionaboveof} inside another floating
      environment\label{fig:\LabelBase.captionaboveof}}
    [l]
  \begin{minipage}[b]{.66\linewidth}%
  \raisebox{\depth}{\fbox{\KOMAoptions{captions=oneline}%
    \begin{minipage}{.5\dimexpr\linewidth-2\fboxsep-2\fboxrule\relax}
      \centering
      \rule{4cm}{5cm}
      \caption[Example: A rectangle]{A rectangle}\label{fig:\LabelBase.rechteck}
    \end{minipage}%
    \begin{minipage}{.5\dimexpr\linewidth-2\fboxsep-2\fboxrule\relax}
      \centering
      \captionaboveof{table}[Example: Measure of the rectangle in
        figure~\ref{fig:\LabelBase.rechteck}]{Rectangle size}
      \label{tab:\LabelBase.rechteck}
      \begin{tabular}{ll}
        Width:  & 4\,cm\\
        Height: & 5\,cm
      \end{tabular}
    \end{minipage}}}%
  \end{minipage}%
  \end{captionbeside}%
\end{figure}%
\end{Example}%
You can produce a non-floating table with a caption in the same way as the
table inside a figure environment in the example above. In such a case, a
\Environment{minipage} environment should also be used to avoid page breaks
between the caption and the table. In addition, you should embed the 
\Environment{minipage} environment in a \Environment{flushleft} environment 
both to achieve a pleasing separation between the surrounding text and to 
avoid the paragraph indentation of the \Environment{minipage} environment.%
\EndIndexGroup


\begin{Declaration}
  \begin{Environment}{captionbeside}
    \OParameter{short title}
    \Parameter{caption text}
    \OParameter{placement}
    \OParameter{width}
    \OParameter{offset}
  \end{Environment}
  \labelsuffix[*]%
  \begin{Environment}{captionbeside}
    \OParameter{short title}
    \Parameter{caption text}
    \OParameter{placement}
    \OParameter{width}
    \OParameter{offset}\PValue{*}%
  \end{Environment}
\end{Declaration}
In addition\ChangedAt{v2.8q}{\Class{scrbook}\and \Class{scrreprt}\and
  \Class{scrartcl}} to captions above and below the figure, you will often
find captions, in particular for small figures, which are placed beside the
figure. The bottom edge of the caption is normally aligned with the bottom of
the figure. Of course, you can achieve the same thing in the standard classes
with some fiddling and the use of two \Macro{parbox} commands. However,
\KOMAScript{} offers a special environment for this which you can use within
the floating environments. The first optional parameter,
\PName{short title}\important{\PName{short title}\\\PName{caption text}}, and
the required parameter \PName{caption text} have the same meaning as the
corresponding parameters of \DescRef{\LabelBase.cmd.caption},
\DescRef{\LabelBase.cmd.captionabove} or
\DescRef{\LabelBase.cmd.captionbelow}. The \PName{caption text} is
placed beside the content of the environment in this case.

The \PName{placement}\important{\PName{placement}} parameter can determine
whether the \PName{caption text} is placed on the left or the right. Exactly
one of the following letters is allowed:
\begin{labeling}[~--]{\quad\PValue{o}}\setlength{\itemsep}{-1\parsep plus 1ex}%
	\item[\PValue{l}] left 
	\item[\PValue{r}] right 
	\item[\PValue{i}] inner margin in two-sided printing 
	\item[\PValue{o}] outer margin in two-sided printing
\end{labeling}
The default placement is to the right of the content of the environment. You
can change this default\ChangedAt{v3.00}{\Class{scrbook}\and
  \Class{scrreprt}\and \Class{scrartcl}} using the
\DescRef{\LabelBase.option.captions}\IndexOption{captions}%
\important{\DescRef{\LabelBase.option.captions}} option (see
\DescPageRef{\LabelBase.option.captions}) with values like
\PValue{innerbeside}\IndexOption[indexmain]{captions~=\textKValue{innerbeside}},
\PValue{leftbeside}\IndexOption[indexmain]{captions~=\textKValue{leftbeside}},
\PValue{outerbeside}\IndexOption[indexmain]{captions~=\textKValue{outerbeside}},
and
\PValue{rightbeside}\IndexOption[indexmain]{captions~=\textKValue{rightbeside}}.
If either \PValue{o} or \PValue{i} are used you may need to run {\LaTeX} twice
to obtain the correct placement.

Normally\important{\PName{width}}, the content of the environment and the
\PName{caption text} fill the entire available text width. However, you can
specify a different width using the optional parameter \PName{width}. This can
be greater than the current text width.

When specifying a \PName{width}\important{\PName{width}\\\PName{offset}}, the
width used is usually centred with respect to the body text. Using the
optional parameter \PName{offset}, you can shift the environment relative to
the left margin. A positive value corresponds to a shift to the right, whereas
a negative value corresponds to a shift to the left. An \PName{offset} of
0\Unit{pt} gives you a left-aligned output.

If\important{\PValue{*}} you add a star to the optional \PName{offset}
parameter, the value represents a shift relative to the right margin on
left-hand pages in a two-sided layout. A positive value corresponds to a shift
towards the outer margin, whereas a negative value corresponds to a shift
towards the inner margin. An \PName{offset} of 0\Unit{pt} means alignment with
the inner margin. This variant may require two {\LaTeX} runs to achieve the
correct offset.

The default vertical alignment is bottom. This means that the bottommost base
lines of the contents of the environment and of the caption are aligned.
You\ChangedAt{v3.00}{\Class{scrbook}\and \Class{scrreprt}\and
	\Class{scrartcl}} can change this setting using the
\DescRef{\LabelBase.option.captions}\IndexOption{captions}%
\important{\DescRef{\LabelBase.option.captions}} option (see
\DescPageRef{\LabelBase.option.captions}) with the value
\PValue{topbeside}\IndexOption[indexmain]{captions~=\textKValue{topbeside}},
\PValue{centeredbeside}%
\IndexOption[indexmain]{captions~=\textKValue{centeredbeside}}, or
\PValue{bottombeside}%
\IndexOption[indexmain]{captions~=\textKValue{bottombeside}}.
With the setting \PValue{topbeside}, the topmost base lines of the environment
contents and caption will be aligned. With \PValue{centeredbeside}, they will
be centred vertically. In this context, note that the base line of a figure is
usually the bottom of the figure. You can change this using, for example,
\Macro{raisebox}\IndexCmd{raisebox}.

\begin{Example}
  You can find an example using the \Environment{captionbeside} environment in
  \autoref{fig:\LabelBase.captionbeside}. This figure was typeset with:
\begin{lstcode}
  \begin{figure}
    \begin{captionbeside}[Example: Figure beside description]%
      {A figure description which is neither above nor
       below, but beside the figure}[i][\linewidth][%
      [i][\linewidth][%
        \dimexpr\marginparwidth+\marginparsep\relax]*
      \fbox{%
        \parbox[b][5\baselineskip][c]{.25\textwidth}
        {%
          \hspace*{\fill}\KOMAScript
          \hspace*{\fill}\par
        }%
      }
    \end{captionbeside}
    \label{fig:\LabelBase.captionbeside}
  \end{figure}
\end{lstcode}
  \begin{figure}
    \begin{captionbeside}[Example: Figure beside description]%
      {A figure description which is neither above nor 
       below, but beside the figure}[i][\linewidth][2em]*
      \fbox{%
        \parbox[b][5\baselineskip][c]{.25\textwidth}{%
          \hspace*{\fill}\KOMAScript\hspace*{\fill}\par}}
    \end{captionbeside}
    \label{fig:\LabelBase.captionbeside}
  \end{figure}
  The total width is thus the currently available width of
  \PValue{\Macro{linewidth}}. However, this width is shifted
  $\Length{marginparwidth} + \Length{marginparsep}$ to the outside. The
  caption text or description is placed on the inner side beside the figure.
  The figure itself is shifted 2\Unit{em} into the outer margin.

\iffalse% Umbruchkorrekturtext
  The \Macro{dimexp} macro is an \eTeX{} command. This should not be a
  problem at all because \KOMAScript{} itself requires \eTeX{} and every almost
  up-to-date \LaTeX{} distribution uses \eTeX already.
\fi

  \hyperref[fig:\LabelBase.captionbesidecentered]%
           {Figure~\ref*{fig:\LabelBase.captionbesidecentered}}
  shows a centred caption with:
\begin{lstcode}
  \KOMAoption{captions}{centeredbeside}
\end{lstcode}
\iffalse % Umbruchalternativen
  Even a typographical novice will immediately realize that you should
  usually avoid this alignment.
\else
  This is certainly not a recommended solution.
\fi

  \begin{figure}
    \begin{captionbeside}[Example: Description centered beside figure]%
      {A figure description which is neither above nor 
       below, but centred beside the figure}[i][\linewidth][2em]*
      \fbox{%
        \parbox[b][5\baselineskip][c]{.25\textwidth}{%
          \hspace*{\fill}\KOMAScript\hspace*{\fill}\par}}
    \end{captionbeside}
    \label{fig:\LabelBase.captionbesidecentered}
  \end{figure}

  In contrast, you can use the top-aligned format seen in
  \autoref{fig:\LabelBase.captionbesidetop}. To illustrate how to shift the
  baseline using \Macro{raisebox}, here is a complete example. You can apply
  this not only to a substitute figure, as previously shown, but also, for
  example, to \Macro{includegraphics} (see \cite{package:graphics}).
\begin{lstcode}
  \documentclass[captions=topbeside]{scrbook}
  \usepackage[english]{babel}
  \usepackage{graphics}
  \begin{document}
  \chapter{An Example}
  \begin{figure}
    \begin{captionbeside}%
      [Example: Figure title top beside]%
      {A figure description which is neither above nor 
       below, but top beside the figure}%
      [i][\linewidth][%
        \dimexpr\marginparwidth+\marginparsep\relax
      ]*
      \raisebox{%
        \dimexpr\baselineskip-\totalheight\relax
      }{%
        \includegraphics{examplepicture}%
      }%
    \end{captionbeside}
    \label{fig:\LabelBase.captionbesidetop}
  \end{figure}
  \end{document}
\end{lstcode}
  \begin{figure}
    \KOMAoption{captions}{topbeside}
    \begin{captionbeside}[Example: Figure title top beside]%
      {A figure description which is neither above nor 
        below, but top beside the figure}%
      [i][\linewidth]
      \raisebox{\dimexpr\baselineskip-\totalheight}{%
        \fbox{%
          \parbox[b][5\baselineskip][c]{.25\textwidth}{%
            \hspace*{\fill}\KOMAScript\hspace*{\fill}\par}}%
      }%
    \end{captionbeside}
    \label{fig:\LabelBase.captionbesidetop}
  \end{figure}
\end{Example}
%
\EndIndexGroup


\begin{Declaration}
  \begin{Environment}{captionofbeside}
    \Parameter{float type}
    \OParameter{short title}
    \Parameter{caption text}
    \OParameter{placement}
    \OParameter{width}
    \OParameter{offset}
    \begin{Body}\BodyDots\end{Body}
  \end{Environment}
  \labelsuffix[*]%
  \begin{Environment}{captionofbeside}
    \Parameter{float type}
    \OParameter{short title}
    \Parameter{caption text}
    \OParameter{placement}
    \OParameter{width}
    \OParameter{offset}\PValue{*}%
    \begin{Body}\BodyDots\end{Body}
  \end{Environment}
\end{Declaration}
As\ChangedAt{v3.10}{\Class{scrbook}\and \Class{scrreprt}\and
  \Class{scrartcl}} is the case for \DescRef{\LabelBase.cmd.caption}, there is
a variant for \DescRef{\LabelBase.cmd.captionof} where the \PName{float type}
is not defined by using it within a floating environment of this type, so you
can specify a suitable environment for \DescRef{\LabelBase.env.captionbeside}
with \Environment{captionofbeside}. In contrast to
\DescRef{\LabelBase.env.captionbeside}, the \PName{float type} must be
specified as an additional, first argument.%
%
\EndIndexGroup

\begin{Declaration}
  \FloatStyle{komaabove}%
  \FloatStyle{komabelow}
\end{Declaration}%
If you use\OnlyAt{\Package{float}} the \Package{float}\IndexPackage{float}
package, the appearance of the float environments is solely defined by the
\emph{float} style. This includes whether captions appear above or below. In
the \Package{float} package, there is no predefined style which gives you the
same output and offers the same setting options (see below) as \KOMAScript{}.
Therefore \KOMAScript{} defines the two additional styles, \PValue{komaabove}
and \PValue{komabelow}.  When using the \Package{float} package, you can
activate these styles just like the styles
\PValue{plain}\IndexFloatstyle{plain}, \PValue{boxed}\IndexFloatstyle{boxed},
or \PValue{ruled}\IndexFloatstyle{ruled} defined in \Package{float}. For
details refer to \cite{package:float}. The style \PValue{komaabove} inserts
\DescRef{\LabelBase.cmd.caption}, \DescRef{\LabelBase.cmd.captionabove}, and
\DescRef{\LabelBase.cmd.captionbelow} above, whereas \PValue{komabelow}
inserts them below the float content.
%
\EndIndexGroup


\begin{Declaration}
  \Macro{captionformat}
\end{Declaration}%
In {\KOMAScript} there are various ways to change the formatting of
the caption text. The definition of different font styles has already
been explained above. The delimiter or delimiters between the label and
actual caption text is specified by the macro \Macro{captionformat}.
In contrast to all other \Macro{\dots}format commands, this is not the counter
but only the items which follow it. The original definition is:
\begin{lstcode}
  \newcommand*{\captionformat}{:\ }
\end{lstcode}
You can change this too with \Macro{renewcommand}.
\begin{Example}
  For some inexplicable reason, you want a dash surrounded by spaces
  instead of a colon followed by a space as a label delimiter. You
  therefore define:
\begin{lstcode}
  \renewcommand*{\captionformat}{~--~}
\end{lstcode}
This definition should be put in the preamble of your document.
\end{Example}
%
\EndIndexGroup


\begin{Declaration}
  \Macro{figureformat}%
  \Macro{tableformat}
\end{Declaration}%
It has already been mentioned that \DescRef{\LabelBase.cmd.captionformat} does
not contain formatting for the label itself. You should not alter this by
redefining the commands for the output of the \Macro{thefigure} or
\Macro{thetable} counters. Such a redefinition would have unwanted side
effects on the output of \Macro{ref}, the table of contents, the list of
figures, and the list of tables. Instead, {\KOMAScript} offers two
\Macro{\dots format} commands. These have the following defaults:
\begin{lstcode}
  \newcommand*{\figureformat}{\figurename~\thefigure\autodot}
  \newcommand*{\tableformat}{\tablename~\thetable\autodot}
\end{lstcode}
They can also be customised to your requirements with \Macro{renewcommand}.
\begin{Example}
  From time to time, it is required to have captions without labels or
  delimiters. With {\KOMAScript} the following definitions suffice to
  achieve this:
\begin{lstcode}
  \renewcommand*{\figureformat}{}
  \renewcommand*{\tableformat}{}
  \renewcommand*{\captionformat}{}
\end{lstcode}
  It should be noted, however, that although no numbering is output, the
  internal counters are nevertheless incremented. This becomes especially
  important if this redefinition is applied only to selected
  \Environment{figure} or \Environment{table} environments.
\end{Example}
%
\EndIndexGroup


\begin{Declaration}
  \Macro{setcapindent}\Parameter{indent}%
  \Macro{setcapindent*}\Parameter{xindent}%
  \Macro{setcaphanging}
\end{Declaration}%
As mentioned previously, in the standard classes the captions are set in a
non-hanging style. In other words, in multi-line captions the second and
subsequent lines start directly beneath the label. The standard classes
provide no direct mechanism to change this behaviour. In {\KOMAScript}, on the
contrary, beginning at the second line all lines are indented by the width of
the label so that the caption text is aligned.

You can change this behaviour, which corresponds to using
\Macro{setcaphanging}, at any time with the \Macro{setcapindent} or
\Macro{setcapindent*} command. Here the parameter \PName{indent} determines
the indentation of the second and subsequent lines. If you want a line break
after the label and before the caption text, then you can define the
indentation \PName{xindent} of the caption text with the starred version of
the command instead: \Macro{setcapindent*}.

A negative value of \PName{indent}, on the other hand, results in a line break
before the caption text, and only the first line of the caption text, not
subsequent lines, is indented by the absolute value of \PName{indent}.

Whether one-line captions are set the same way as multi-line captions or are
treated separately is specified with the option
\DescRef{\LabelBase.option.captions}. For details please refer to the
explanations of these option on
\DescPageRef{\LabelBase.option.captions.oneline}.

\begin{Example}
  The illustrations~\ref{fig:\LabelBase.caption.first} to
  \ref{fig:\LabelBase.caption.last} show the effects of different settings. As
  you can see, using a fully hanging indentation with a narrow column width is
  awkward. To illustrate, the source code for the second figure is given here
  with a modified caption text:
\begin{lstcode}
  \begin{figure}
    \setcapindent{1em}
    \fbox{\parbox{.95\linewidth}{\centering{\KOMAScript}}}
    \caption{Example with partially indented caption
             starting from the second line}
  \end{figure}
\end{lstcode}
  As you can see, the formatting can also be changed locally within the
  \Environment{figure} environment\IndexEnv{figure}. The change then affects
  only the current figure. Subsequent figures once again use the default
  settings or global settings that you set, for example, in the preamble. This
  also, of course, applies to tables.
  \begin{figure}
    \typeout{^^J--- Ignore underfull and overfull \string\hbox:}
    \addtokomafont{caption}{\small}
    \addtokomafont{captionlabel}{\bfseries}
    \centering%
    \begin{minipage}{.9\linewidth}
      \begin{minipage}[t]{.48\linewidth}\sloppy
        \fbox{\parbox{.95\linewidth}{\centering{\KOMAScript}}}
        \caption[Example: Default caption]%
        {\sloppy With the default setting, like using
          \Macro{setcaphanging}}
        \label{fig:\LabelBase.caption.first}
      \end{minipage}
      \hspace{.02\linewidth}
      \begin{minipage}[t]{.48\linewidth}\sloppy
        \setcapindent{1em}
        \fbox{\parbox{.95\linewidth}{\centering{\KOMAScript}}}
        \caption[Example: Caption with partially hanging indention]%
        {With partially hanging indentation starting from the second line
         by using \Macro{setcapindent}\PParameter{1em}}
      \end{minipage}
    \end{minipage}

    \vspace*{2ex}\noindent%
    \begin{minipage}{.9\linewidth}
      \begin{minipage}[t]{.48\linewidth}\sloppy
        \setcapindent*{1em}
        \fbox{\parbox{.95\linewidth}{\centering{\KOMAScript}}}
        \caption[Example: Caption with hanging indention and line break]%
        {With hanging indentation starting from the second line and line
          break before the description using
          \Macro{setcapindent*}\PParameter{1em}}
      \end{minipage}
      \hspace{.02\linewidth}
      \begin{minipage}[t]{.48\linewidth}\sloppy
        \setcapindent{-1em}
        \fbox{\parbox{.95\linewidth}{\centering{\KOMAScript}}}
        \caption[Example: Caption with indention in the second line]%
        {With indentation only in the second line and a line break
          before the description using
          \Macro{setcapindent}\PParameter{-1em}}
                \label{fig:\LabelBase.caption.last}
      \end{minipage}
    \end{minipage}
    \typeout{^^J--- Don't ignore underfull and overfull
      \string\hbox:^^J}
  \end{figure}
\end{Example}
%
\EndIndexGroup


\begin{Declaration}
  \Macro{setcapwidth}\OParameter{justification}\Parameter{width}%
  \Macro{setcapdynwidth}\OParameter{justification}\Parameter{width}%
  \Macro{setcapmargin}\OParameter{left margin}\Parameter{margin}%
  \Macro{setcapmargin*}\OParameter{inside margin}\Parameter{margin}
\end{Declaration}
Using\ChangedAt{v2.8q}{%
  \Class{scrbook}\and\Class{scrreprt}\and\Class{scrartcl}} these three
commands, you can specify the width and justification of the caption
text. Normally, the entire text or column width is available for the caption.

With\important{\Macro{setcapwidth}} the \Macro{setcapwidth} command, you can
decrease this \PName{width}. The mandatory argument determines the maximum
\PName{width} of the caption. As an optional argument, you can supply exactly
one letter which specifies the horizontal justification. The possible
justifications are given in the following list:
\begin{labeling}[~--]{\quad\PValue{o}}%
  \itemsep=.5\topsep plus .5\topsep minus .5\topsep
  \rightskip=1em
\item[\PValue{l}] left-justified
\item[\PValue{c}] centred
\item[\PValue{r}] right-aligned
\item[\PValue{i}] aligned to the inner margin in two-sided printing
\item[\PValue{o}] aligned to the outer margin in two-sided printing
\end{labeling}
Inside and outside justification corresponds to left-aligned and right-aligned
justification, respectively, in one-sided printing.
Within\textnote{Attention!} \Package{longtable}\IndexPackage{longtable}%
\important{\Package{longtable}} tables, inside and outside justification does
not work correctly. In particular, the captions on subsequent pages of such
tables are aligned according to the format of the caption on the first page of
the table. This is a conceptual problem in the implementation of the
\Package{longtable} package.

Note\ChangedAt{v3.20}{\Class{scrbook}\and \Class{scrreprt}\and
  \Class{scrartcl}}\textnote{Attention!} that \Macro{setcapwidth} immediately
sets the width to the value of the \PName{width} parameter at the time of the
assignment, as \Macro{setlength} does. If\important{\Macro{setcapdynwidth}}
you instead want to use the current value of \PName{width} when the caption is
set, you should use \Macro{setcapdynwidth}. There can be significant
differences in the result if, for example, you use lengths like
\Length{linewidth} or other commands as \PName{width} arguments.

With\important{\Macro{setcapmargin}} the \Macro{setcapmargin} command, instead
of specifying the width of the caption text, you can specify a \PName{margin}
to be set next to the caption text in addition to the normal text margin. If
you want margins with different widths on the left and right sides, you can
use the optional argument to specify a \PName{left margin} that differs from
\PName{margin}. Instead of a \PName{left margin}, the starred version
\Macro{setcapmargin*} defines an \PName{inside margin} in a two-sided layout.
The same problem arises here with with inside and outside justification inside
\Package{longtable}\IndexPackage{longtable}\important{\Package{longtable}}
tables that occurs with \Macro{setcapwidth}. Furthermore, using
\Macro{setcapmargin} or \Macro{setcapmargin*} activates the
\OptionValueRef{\LabelBase}{captions}{nooneline} option (see
\DescPageRef{\LabelBase.option.captions.nooneline}) for captions which are
typeset with this margin setting.

You\textnote{Hint!} can also specify negative values for \PName{margin} and
\PName{left margin} or \PName{inside margin}. This has the effect of making
the caption protrude into the margin.%
\iffalse\par% Anhang wurde entfernt.
Experts\textnote{Hint!} and advanced users can find a a somewhat tricky
application of \Macro{setcapwidth} in
\iffree{\cite{book:komascript}}{\autoref{cha:floattricks},
  \autopageref{cha:floattricks}}.%
\fi%
%
\EndIndexGroup

\begin{Declaration}
  \Macro{setcaptionalignment}\OParameter{float type}\Parameter{alignment}
\end{Declaration}
Normally\ChangedAt{v3.25}{\Class{scrbook}\and \Class{scrreprt}\and
  \Class{scrartcl}}, multi-line\textnote{alignment of multi-line captions}
captions are fully justified. This corresponds to
\Macro{setcaptionalignment}\PParameter{j}. Sometimes, however, you want a
different alignment, for example ragged right. An appropriate change is
possible at any time with \Macro{setcaptionalignment}. You can specify exactly
one of the letters listed in \autoref{tab:maincls.captionalignment} for the
\PName{alignment}. If you specify an unknown \PName{alignment}, you will
receive an error message.
%
\begin{table}
%  \centering
  \KOMAoptions{captions=topbeside}%
  \setcapindent{0pt}%
  \begin{captionbeside}
    [{Alignments for multi-line captions of floating environments}]
    {\label{tab:maincls.captionalignment}%
      Alignments for multi-line captions of floating environments}
    [l]
    \begin{tabular}[t]{ll}
      \toprule
      c & centred \\
      j & fully justified \\
      l & left justified \\
      r & right justified \\
      C & centred with \Package{ragged2e} \\
      J & fully justified with \Package{ragged2e} \\
      L & left justified with \Package{ragged2e} \\
      R & right justified with \Package{ragged2e} \\
      \bottomrule
    \end{tabular}
  \end{captionbeside}
\end{table}

The four possibilities with the 
\Package{ragged2e}\important{\Package{ragged2e}}\IndexPackage{ragged2e}
package are only available if that package was loaded before you use
\Macro{setcaptionalignment}. Otherwise, they are converted to the
corresponding options without \Package{ragged2e}. For safety reasons, a
warning is issued in this case.

When using this command without\textnote{without optional parameter} the
optional parameter, the result depends on whether the call occurs inside our
outside of a floating environment. Within a floating environment, the
alignment is then set for this floating environment. Outside, on the other
hand, the optional parameter is assumed to be empty.

Using this command with an empty\textnote{empty optional parameter} optional
parameter, or outside a floating environment and also without any optional
parameter, sets the general alignment. This is used whenever the current
floating environment type does not define an alignment.

If you only want to set the alignment of a particular type of floating
environment\textnote{with \PName{float type}} without changing the 
\PName{alignment} for other types of floating environments, then the type of
floating environment, e.\,g., \PValue{figure} or \PValue{table}, is given as
the optional parameter \PName{float type}.
%
\begin{Example}
  You want captions to be centred under the images even if they are
  multi-line. To text this for a single image, use\textnote{in the
    floating environment}:
\begin{lstcode}
  \begin{figure}
    \centering
    \setcaptionalignment{c}
    \includegraphics{example-image}
    \caption{\blindtext}
  \end{figure}
\end{lstcode}
  Since you are satisfied with the result, you move the\textnote{in the
    preamble}
\begin{lstcode}
  \setcaptionalignment{c}
\end{lstcode}
  command to the document preamble. Thereafter, however, you decide you do not
  like this change for table captions at all. Therefore, you use\textnote{only
    for figures}
\begin{lstcode}
  \setcaptionalignment[figure]{c}
\end{lstcode}
  to limit the centring to figures.

  A little later, you realize that the centring is not so suitable. Instead,
  you now prefer to have ragged right alignment. So you change the alignment
  again:
\begin{lstcode}
  \setcaptionalignment[figure]{l}
\end{lstcode}
  However, you do not like the fact that the lines are very different in
  length. This is caused by the lack of hyphenation, causing long words to
  wrap completely onto the next line, leaving large gaps. You want to allow
  hyphenation as needed. This is easy to achieve with the help of the
  \Package{ragged2e}\important{\Package{ragged2e}}\IndexPackage{ragged2e}
  package. However, it is not enough to use
\begin{lstcode}
  \usepackage{ragged2e}
\end{lstcode}
  to load the package. Using the \Option{newcommands} option when loading the
  package does not help. Instead, the \PName{alignment} must also be changed:
\begin{lstcode}
  \usepackage{ragged2e}
  \setcaptionalignment[figure]{L}
\end{lstcode}
  Note\textnote{Attention!} the upper-case letter for the \PName{alignment}.
\end{Example}
\EndIndexGroup


\begin{Declaration}
  \Option{origlongtable}
\end{Declaration}%
\BeginIndex{Package}{longtable}%
If you do not want the table captions produced by the \Package{longtable}
package (see \cite{package:longtable}) to be redefined by the {\KOMAScript}
classes, activate the \Option{origlongtable} option. This\textnote{Attention!}
option must be used in the optional argument of
\DescRef{\LabelBase.cmd.documentclass}. It cannot be used as a setting of
\DescRef{\LabelBase.cmd.KOMAoptions} or \DescRef{\LabelBase.cmd.KOMAoptions}.
%
\EndIndexGroup
%
\EndIndexGroup


\begin{Declaration}
  \OptionVName{listof}{setting}
\end{Declaration}
Normally\ChangedAt{v3.00}{\Class{scrbook}\and \Class{scrreprt}\and
  \Class{scrartcl}} lists of floating environments\,---\,like
tables\Index{tables>list of} or figures\Index{figures>list of}\,---\,are
neither numbered nor included in the table of contents. You can find more
information about this behaviour in \autoref{sec:\LabelBase.toc}. As an
alternative to the settings \OptionValueRef{\LabelBase}{toc}{nolistof}%
\IndexOption{toc~=\textKValue{nolistof}},
\OptionValueRef{\LabelBase}{toc}{listof}\IndexOption{toc~=\textKValue{listof}},
and \OptionValueRef{\LabelBase}{toc}{listofnumbered}%
\IndexOption{toc~=\textKValue{listofnumbered}} mentioned there, you can also
look at this behaviour from perspective of the lists themselves.  Therefore
you can achieve the same results with the settings
\OptionValue{listof}{notoc}, \OptionValue{listof}{totoc}, and
\OptionValue{listof}{numbered}.

By default, the headings in the lists of floating environments use the topmost
level below \DescRef{\LabelBase.cmd.part}. This is the chapter level in
\Class{scrbook} and \Class{scrreprt} and the section level in
\Class{scrartcl}. By contrast,\ChangedAt{v3.06}{\Class{scrbook}\and
  \Class{scrreprt}\and \Class{scrartcl}}
\OptionValue{listof}{leveldown}\important{\OptionValue{listof}{leveldown}}
uses the next lower level to be used instead.
\important{\OptionValue{listof}{standardlevel}}%
\ChangedAt{v3.15}{\Class{scrbook}\and \Class{scrreprt}\and \Class{scrartcl}}%
\OptionValue{listof}{standardlevel} switches back to the default sectioning 
level, if necessary.
\begin{Example}
  In a book, you want to put the list of figures and the list of tables as
  sub-lists into a common list named ``Figures and Tables''. Simply use:
\begin{lstcode}
  \KOMAoption{listof}{leveldown}
\end{lstcode}
  to use the section instead of the chapter level for both lists, and put the
  following at the appropriate place in your document:
\begin{lstcode}
  \addchap*{Figures and Tables}
  \listoffigures
  \listoftables
\end{lstcode}
  You can find more information about the \DescRef{\LabelBase.cmd.addchap*}
  command in \autoref{sec:\LabelBase.structure} on
  \DescPageRef{\LabelBase.cmd.addchap*}.
\end{Example}

Normally\ChangedAt{v2.8q}{%
  \Class{scrbook}\and \Class{scrreprt}\and \Class{scrartcl}} the lists of
floating environments use a fixed-width space for the caption number of the
entries. At the same time, all entries are somewhat indented. This behaviour
corresponds to the 
\OptionValue{listof}{graduated}\IndexOption{listof~=\textKValue{graduated}} 
setting.

If the numbers become very wide, for example because you have many tables or
figures, the space provided may at some point no longer be sufficient.
\KOMAScript{} offers the setting \OptionValue{listof}{flat}%
\important{\OptionValue{listof}{flat}}\IndexOption{listof~=\textKValue{flat}}
for lists of floating environments, comparable to
\OptionValueRef{\LabelBase}{toc}{flat}\IndexOption{toc~=\textKValue{flat}} for
the table of contents. The width required to print the number is
determined automatically and the space is adjusted accordingly. See the
\OptionValueRef{\LabelBase}{toc}{flat} option, \autoref{sec:\LabelBase.toc},
\DescPageRef{\LabelBase.option.toc.flat} for more information about side
effects and how it works. Note\textnote{Attention!} again that you need more
than one \LaTeX{} run before the lists of floating environments reach
their final state.

The \OptionValue{listof}{entryprefix}\ChangedAt{v3.06}{\Class{scrbook}\and
  \Class{scrreprt}\and \Class{scrartcl}} setting automatically activates
\OptionValue{listof}{flat} too.
Normally\important{\OptionValue{listof}{entryprefix}}, it does not make sense
to add a prefix such as ``figure'' or ``table'' to each entry in the lists of
floating environments because, of course, only figures are included in the
list of figures and only tables are included in the list of tables. Such a
prefix provides no additional information and is thus omitted by default.
However, you can add such prefixes using the
\OptionValue{listof}{entryprefix} option. With this, all entries in the same
list will get the same prefix. The prefix depends on the file extension of the
auxiliary file that is used for the corresponding list. For the list of
figures, the file extension is ``\File{lof}'' and therefore
\Macro{listoflofentryname} is used. For the list of tables, the file extension
is ``\File{lot}'' and \Macro{listoflotentryname} is used.

For the \Class{scrbook} and \Class{scrreprt}\OnlyAt{\Class{scrbook}\and
  \Class{scrreprt}} classes, \KOMAScript{} adds a vertical space to the lists
of floating environments whenever a new chapter starts. This behaviour, which
also exists in the standard classes, groups the lists by chapters. In
\KOMAScript{}, it corresponds to setting \ChangedAt{v3.00}{\Class{scrbook}\and
  \Class{scrreprt}\and \Class{scrartcl}}%
\OptionValue{listof}{chaptergapsmall}%
\LabelOptionValue{listof}{chaptergapsmall}%
\important{\OptionValue{listof}{chaptergapsmall}}%
\IndexOption{listof~=\textKValue{chaptergapsmall}}. In this case, is uses a
fixed vertical space of 10\Unit{pt}.
With\important{\OptionValue{listof}{chaptergapline}} the
\OptionValue{listof}{chaptergapline}%
\IndexOption{listof~=\textKValue{chaptergapline}} option, you instead get a
vertical space the height of one standard text line. With
\OptionValue{listof}{nochaptergap}%
\IndexOption{listof~=\textKValue{nochaptergap}}%
\important{\OptionValue{listof}{nochaptergap}}, you can completely remove the
vertical space. The\important{\OptionValue{listof}{chapterentry}}
\OptionValue{listof}{chapterentry}%
\IndexOption{listof~=\textKValue{chapterentry}} option is a special feature.
Instead of a vertical space, the table of contents entry for the chapter is
inserted into the lists of floating environments. Note\textnote{Attention!}
that this happens even if the chapter does not contain a floating environment.
You can find a method where only chapters containing floating environments
appear in the respective lists at \cite{https://komascript.de/comment/5070}.
You can achieve a more direct control over what appears in the lists of
floating environments with the \DescRef{\LabelBase.option.chapteratlists}
option, which is explained in \autoref{sec:\LabelBase.structure}, on
\DescPageRef{\LabelBase.option.chapteratlists}.

You can find an overview of all settings for the \Option{listof} option in
\autoref{tab:\LabelBase.listof}.

\begin{desclist}
  \desccaption[{Available values for the \Option{listof} option}]{%
    Available values for the \Option{listof} option for modifying the format
    and contents of the lists of floating
    environments\label{tab:\LabelBase.listof}%
  }{%
    Available values for the \Option{listof} option (\emph{continued})%
  }%
  \entry{\PValue{chapterentry}, \PValue{withchapterentry}}{%
    Indicates the beginning of chapters in the lists of floating environments
    with copies of their table-of-contents entries.%
    \IndexOption{listof~=\textKValue{chapterentry}}}%
  \entry{\PValue{chaptergapline}, \PValue{onelinechaptergap}}{%
    Indicates the beginning of chapters in the lists of floating environments
    with a vertical space the height of one standard text line.%
    \IndexOption{listof~=\textKValue{chaptergapline}}}%
  \entry{\PValue{chaptergapsmall}, \PValue{smallchaptergap}}{%
    Indicates the beginning of chapters in the lists of floating environments
    with a small vertical space.%
    \IndexOption{listof~=\textKValue{chaptergapsmall}}}%
  \entry{\PValue{entryprefix}}{%
    \ChangedAt{v3.06}{\Class{scrbook}\and \Class{scrreprt}\and
      \Class{scrartcl}}%
    Adds a prefix before the number of each floating-environment list entry.
    The prefix is usually language-dependent, e.\,g., in English ``Figure'' is
    used for the list of figures and ``Table'' for the list of tables, each
    followed by a white space.%
    \IndexOption{listof~=\textKValue{entryprefix}}}%
  \entry{\PValue{flat}, \PValue{left}}{%
    Prints the lists of floating environments in tabular form. The caption
    numbers are the first column, the caption texts the second column, and the
    page numbers the third column. The space reserved for the caption numbers
    depends on the previous \LaTeX{} run.%
    \IndexOption{listof~=\textKValue{flat}}}%
  \entry{\PValue{graduated}, \PValue{indent}, \PValue{indented}}{%
    Prints the lists of floating environments in a hierarchical form. The
    space reserved for the caption numbers is limited.%
    \IndexOption{listof~=\textKValue{graduated}}}%
  \entry{\PValue{leveldown}}{%
    Uses headings that are one level lower in the sectioning hierarchy than
    the default for lists of floating environments.%
    \IndexOption{listof~=\textKValue{leveldown}}}%
  \entry{\PValue{indenttextentries}, \PValue{indentunnumbered},
    \PValue{numberline}}{%
    \ChangedAt{v3.12}{\Class{scrbook}\and \Class{scrreprt}\and
      \Class{scrartcl}}%
    The \PValue{numberline} property (see \autoref{sec:tocbasic.toc},
    \DescPageRef{tocbasic.cmd.setuptoc}) is set for the lists of floating
    environments such as figures and tables. As a result, unnumbered entries
    are left-aligned with the text of numbered entries of the same level.
    However, the \KOMAScript{} classes themselves do not put unnumbered
    entries in these lists. This option therefore affects only entries that
    are generated not by the \KOMAScript{} classes themselves but with the
    help of \DescRef{tocbasic.cmd.addxcontentsline} (see
    \autoref{sec:tocbasic.toc}, \DescPageRef{tocbasic.cmd.addxcontentsline}).%
    \IndexOption{toc~=\textKValue{numberline}}}%
  \entry{\PValue{leftaligntextentries}, \PValue{leftalignunnumbered},
    \PValue{nonumberline}}{%
    \ChangedAt{v3.12}{\Class{scrbook}\and \Class{scrreprt}\and
      \Class{scrartcl}}%
    The \PValue{nonumberline} property (see \autoref{sec:tocbasic.toc},
    \DescPageRef{tocbasic.cmd.setuptoc}) is set for the lists of floating
    environments. This will place unnumbered entries left-aligned with the
    number of numbered entries. However, the \KOMAScript{} classes themselves
    do not put unnumbered entries in these lists. This option therefore
    affects only entries that are generated not by the \KOMAScript{} classes
    themselves but with the help of \DescRef{tocbasic.cmd.addxcontentsline}
    (see \autoref{sec:tocbasic.toc},
    \DescPageRef{tocbasic.cmd.addxcontentsline}).%
    \IndexOption{toc~=\textKValue{nonumberline}}}%
  \entry{\PValue{nochaptergap}, \PValue{ignorechapter}}{%
    The beginnings of chapters are not marked in the lists of floating
    environments.%
    \IndexOption{listof~=\textKValue{nochaptergap}}}%
  \entry{\PValue{notoc}, \PValue{nottotoc}, \PValue{plainheading}}{%
    The lists of floating environments do not generate entries in the table of
    contents.%
    \IndexOption{listof~=\textKValue{nottotoc}}}%
  \entry{\PValue{numbered}, \PValue{totocnumbered}, \PValue{tocnumbered},
    \PValue{numberedtoc}, \PValue{numberedtotoc}}{%
    The lists of floating environments receive numbered entries in the table
    of contents.%
    \IndexOption{listof~=\textKValue{numbered}}}%
  \entry{\PValue{standardlevel}}{%
    The lists use the normal sectioning level.%
    \IndexOption{listof~=\textKValue{standardlevel}}}%
  \entry{\PValue{totoc}, \PValue{toc}, \PValue{notnumbered}}{%
    The lists of floating environment generate entries in the table of
    contents, but their headings are unnumbered.%
    \IndexOption{listof~=\textKValue{totoc}}}%
\end{desclist}
%
\EndIndexGroup


\begin{Declaration}
  \Macro{listoftables}%
  \Macro{listoffigures}
\end{Declaration}
These commands generate a list of tables or figures. Changes affecting these
lists will require two \LaTeX{} runs to take effect. The layout of the lists
can be influenced by the \DescRef{\LabelBase.option.listof}%
\important{\DescRef{\LabelBase.option.listof}} option with the values
\PValue{graduated} or \PValue{flat} (see
\DescPageRef{\LabelBase.option.listof}). In addition, the \PValue{listof} and
\PValue{listofnumbered} values of the \DescRef{\LabelBase.option.toc}%
\important{\DescRef{\LabelBase.option.toc}} option (see
\autoref{sec:\LabelBase.toc}), as well as the \PValue{totoc} and
\PValue{numbered} values of the \DescRef{\LabelBase.option.listof} option
described above indirectly affect these lists.

As\textnote{Hint!} a rule, you will find the lists of the floating
environments immediately after the table of contents. In some documents, they
go into the appendix. However, the author of this guide prefers them
immediately after the table of contents.%
%
\EndIndexGroup


\LoadCommonFile{marginpar}% \section{Marginal Notes}


\section{Appendix}
\seclabel{appendix}
\BeginIndexGroup
\BeginIndex{}{Appendix}

The appendix of a document mostly consists of supplements to the document.
Typical parts of an appendix include a bibliography, an index, and a glossary.
However you should not start an appendix for these parts alone because their
format already distinguishes them from the main document. But if there are
additional elements in the appendix, such as quoted third-party documents,
endnotes, figures, or tabulars, the standard elements such as the bibliography
should also be part of the appendix.


\begin{Declaration}
  \Macro{appendix}
\end{Declaration}%
The appendix is started in the standard as well as the {\KOMAScript} classes
with \Macro{appendix}. Among other things, this command changes the chapter
numbering to upper-case letters while ensuring that the rules according to
\cite{DUDEN} for numbering the sectioning levels are followed (for
German-speaking regions). These rules are explained in more detail in the
description of the \DescRef{\LabelBase.option.numbers} option in
\autoref{sec:\LabelBase.structure}, \DescPageRef{\LabelBase.option.numbers}.

The format of the chapter headings\OnlyAt{\Class{scrbook}\and
  \Class{scrreprt}} in the appendix is influenced by the
\DescRef{\LabelBase.option.chapterprefix}%
\important{\DescRef{\LabelBase.option.chapterprefix}\\
  \DescRef{\LabelBase.option.appendixprefix}} and
\DescRef{\LabelBase.option.appendixprefix} options. See
\autoref{sec:\LabelBase.structure},
\DescPageRef{\LabelBase.option.appendixprefix} for more information.

Please\textnote{Attention!} note that \Macro{appendix} is a command,
\emph{not} an environment! This command does not expect an argument. Chapters
and sections in the appendix use \DescRef{\LabelBase.cmd.chapter} and
\DescRef{\LabelBase.cmd.section}, just as in the main text.%
%
\EndIndexGroup
%
\EndIndexGroup


\section{Bibliography}
\seclabel{bibliography}
\BeginIndexGroup
\BeginIndex{}{bibliography}

The bibliography makes external sources accessible. As a rule, the
bibliography is created from an external file with a database-like structure
using the \BibTeX{} program. You can use the \BibTeX{} style to change both
the format of the entries and their sorting. If you use an additional
bibliography package like \Package{natbib}\IndexPackage{natbib},
\Package{babelbib}\IndexPackage{babelbib}, or
\Package{biblatex}\IndexPackage{biblatex}, \KOMAScript{}'s influence over the
bibliography disappears. In such cases, you must follow the manual of the
relevant bibliography package. You can find general information about
bibliographies in \cite{lshort}.


\begin{Declaration}
  \OptionVName{bibliography}{setting}
\end{Declaration}
For a start,\ChangedAt{v3.00}{\Class{scrbook}\and \Class{scrreprt}\and 
  \Class{scrartcl}} you can select any predefined bibliography style in
\PName{setting}. There are two such bibliography styles predefined in
\KOMAScript{}. You should not confuse them with the styles used by
\BibTeX\Index{BibTeX=\BibTeX}, which you select with
\Macro{bibstyle}. While \BibTeX{} determines both the sorting and the
contents of the bibliography, \KOMAScript{} influences only some basic
features of the bibliography and few properties of the entry format.

The\important{\OptionValue{bibliography}{oldstyle}}
\OptionValue{bibliography}{oldstyle}%
\IndexOption{bibliography~=\textKValue{oldstyle}} option selects a compact
format for bibliography entries. In this case, using the
\DescRef{maincls-experts.cmd.newblock}\IndexCmd{newblock} command results in
only a small glue between the entries. The name comes from the fact that this
is the most common classic form of a bibliography.
In\important{\OptionValue{bibliography}{openstyle}} contrast, the
\OptionValue{bibliography}{openstyle}%
\IndexOption{bibliography~=\textKValue{openstyle}} setting selects a more
modern and open kind of bibliography. The name comes from the fact that the
\DescRef{maincls-experts.cmd.newblock} command inserts a paragraph break. The
bibliography entries appear more structured. They are less compact and seem
more relaxed or open. Information about defining new bibliography styles can
be found in the description of the \DescRef{maincls-experts.cmd.newbibstyle}
command in \autoref{sec:maincls-experts.bibliography},
\DescPageRef{maincls-experts.cmd.newbibstyle}.

In addition to the formatting style, you can select one more feature using
\PName{setting}. The bibliography is a kind of list of contents. But instead
of listing the contents of the document itself, it references external works.
With this reasoning, you could argue that the bibliography is a separate
chapter or section and therefore deserves a chapter or section number. The
\OptionValue{bibliography}{numbered}%
\important{\OptionValue{bibliography}{numbered}}%
\IndexOption{bibliography~=\textKValue{numbered}} setting does exactly that,
including creating an entry in the table of contents. In my opinion, a
traditional, annotated source list should by this reasoning be a separate
chapter too. Moreover, the bibliography is ultimately nothing you've written
yourself and so merits at most an unnumbered entry in the table of contents,
which is achieved with the \OptionValue{bibliography}{totoc}%
\IndexOption{bibliography~=\textKValue{totoc}} option. The default setting,
where the bibliography is produced as an unnumbered chapter and does not
receive an entry in the table of contents corresponds to
\OptionValue{bibliography}{nottotoc}%
\IndexOption{bibliography~=\textKValue{nottotoc}}. For more information, see
the \DescRef{\LabelBase.option.toc} option in \autoref{sec:\LabelBase.toc},
especially the \PValue{bibliographynumbered}, \PValue{bibliography},
and\PValue{nobibliography} values for this option on
\DescPageRef{\LabelBase.option.toc.bibliography}.

Sometimes\ChangedAt{v3.12}{\Class{scrbook}\and \Class{scrreprt}\and
  \Class{scrartcl}} a document made using \Class{scrbook} or \Class{scrreprt}
will have a bibliography for every chapter rather than one bibliography for
the whole document. In that case, it makes sense for the bibliography
itself\important{\OptionValue{bibliography=leveldown}} to be not a chapter but
a section. You can achieve this using the
\OptionValue{bibliography}{leveldown}%
\IndexOption{bibliography~=\textKValue{leveldown}} option. You can also use
this if you want the bibliography to appear with other lists under a common
heading, therefore this option is also available with \Class{scrartcl}.

You can find a summary of available values for the \Option{bibliography}
option in \autoref{tab:\LabelBase.bibliography}. Note, however, that you can
define new values with
\DescRef{maincls-experts.cmd.newbibstyle}\IndexCmd{newbibstyle}.

\begin{table}
  \caption[{Available values for the \Option{bibliography} option}]{%
    Predefined values for the \Option{bibliography} option for setting the
    bibliography format}
  \label{tab:\LabelBase.bibliography}
  \begin{desctabular}
    \pventry{leveldown}{%
      \ChangedAt{v3.12}{\Class{scrbook}\and \Class{scrreprt}\and
        \Class{scrartcl}}%
      The bibliography uses a sectioning command one level lower than
      the default.%
      \IndexOption{bibliography~=\textKValue{leveldown}}}%
    \entry{\PValue{notoc}, \PValue{nottotoc}, \PValue{plainheading}}{%
      The bibliography receives neither an entry in the table of contents nor
      a number.%
      \IndexOption{bibliography~=\textKValue{nottotoc}}}%
    \entry{\PValue{numbered}, \PValue{tocnumbered}, \PValue{totocnumbered},
      \PValue{numberedtoc}, \PValue{numberedtotoc}}{%
      The bibliography receives an entry in the table of contents and a
      number.%
      \IndexOption{bibliography~=\textKValue{numbered}}}%
    \pventry{oldstyle}{%
      The bibliography uses the classic, compact formation, where
      \DescRef{maincls-experts.cmd.newblock}\IndexCmd{newblock} generates only
      a horizontal glue.%
      \IndexOption{bibliography~=\textKValue{oldstyle}}}%
    \pventry{openstyle}{%
      The bibliography uses the structured, open format where
      \DescRef{maincls-experts.cmd.newblock}\IndexCmd{newblock} generates a
      paragraph break.%
      \IndexOption{bibliography~=\textKValue{openstyle}}}%
    \pventry{standardlevel}{%
      \ChangedAt{v3.12}{\Class{scrbook}\and \Class{scrreprt}\and
        \Class{scrartcl}}%
      The bibliography uses the standard sectioning command level.%
      \IndexOption{bibliography~=\textKValue{standardlevel}}}%
    \entry{\PValue{toc}, \PValue{totoc}, \PValue{notnumbered}}{%
      The bibliography receives an entry in the table of contents but no
      number.%
      \IndexOption{bibliography~=\textKValue{totoc}}}%
  \end{desctabular}
\end{table}
%
\EndIndexGroup


\begin{Declaration}
  \Macro{setbibpreamble}\Parameter{preamble}
\end{Declaration}%
You can use the \Macro{setbibpreamble} command to set a preamble for the
bibliography\Index{bibliography}. This preamble must be placed before the
command for issuing the bibliography. However, it need not be directly 
in front of it. For example, it could be placed at the beginning of
the document. Like\textnote{Attention!} the 
\OptionValueRef{\LabelBase}{bibliography}{totoc} and
\OptionValueRef{\LabelBase}{bibliography}{totocnumbered} options, this command
only works if you have not loaded a package which prevents this from happening
by redefining the \Environment{thebibliography} environment. Although the
\Package{natbib} package\IndexPackage{natbib} uses undocumented, internal 
{\KOMAScript} macros, \Macro{setbibpreamble} could still work with the current
version of \Package{natbib} (see \cite{package:natbib}).
\begin{Example}
  You want to point out that the bibliographical references are sorted
  alphabetically by the names of the authors. You therefore use the following
  command:
\begin{lstcode}
  \setbibpreamble{References are in alphabetical order.
    References with more than one author are sorted
    according to the first author.\par\bigskip}
\end{lstcode}
  The \Macro{bigskip}\IndexCmd{bigskip} command ensures that the
  preamble and the first reference are separated by a large vertical
  space.%
\end{Example}%
%
\EndIndexGroup


\begin{Declaration}
  \Macro{BreakBibliography}\Parameter{interruption code}
\end{Declaration}
This\textnote{Attention!}\ChangedAt{v3.00}{\Class{scrbook}\and
  \Class{scrreprt}\and \Class{scrartcl}} command exists only if the
environment \Environment{thebibliography} has not been redefined by another
package. With this instruction, you can insert a break into the
bibliography. The \PName{interruption code} will be expanded inside a
group. Such a break may be, for example, a heading using
\DescRef{\LabelBase.cmd.minisec}.  Unfortunately there is currently no way to
have this command automatically generated, for example by using a special
entry in the \BibTeX{} database.  Therefore, it can currently only be used by
users who edit the bibliography directly. For this reason, its usefulness is
very limited.%
%
\EndIndexGroup


\begin{Declaration}
  \Macro{AfterBibliographyPreamble}\Parameter{code}%
  \Macro{AtEndBibliography}\Parameter{code}
\end{Declaration}
In\ChangedAt{v3.00}{\Class{scrbook}\and \Class{scrreprt}\and \Class{scrartcl}}
some cases, it may be useful to add some \PName{code} after the bibliography
preamble or just before the end of the bibliography. This is possible with the
help of these two instructions.
\begin{Example}
  You want to set the bibliography not justified but ragged right. You can
  achieve this with:
\begin{lstcode}
  \AfterBibliographyPreamble{\raggedright}
\end{lstcode}
  You can put this instruction anywhere before the bibliography. However, it is
  recommended to do so in the preamble of the document or a separate package.
\end{Example}
The\textnote{Attention!} functionality of this instruction depends on
cooperation with packages modifying the bibliography, if you use such a
package\iffree{}{(see \autoref{sec:maincls-experts.coexistence},
\DescPageRef{maincls-experts.cmd.AfterBibliographyPreamble})}.%
%
\EndIndexGroup
%
\EndIndexGroup


\section{Index}
\seclabel{index}
\BeginIndexGroup

For general information about making a keyword index, see \cite{lshort},
\cite{makeindex}, and \cite{xindy}. Using a package that redefines
commands or environments for the index \KOMAScript{}'s ability to influence
the index may disappear. This applies, for example, when using the
\Package{index}\IndexPackage{index} package, but not when using the
\Package{splitidx}\IndexPackage{splitidx} package (see
\cite{package:splitindex}).

\begin{Declaration}
  \OptionVName{index}{setting}%
\end{Declaration}
\ChangedAt{v3.00}{\Class{scrbook}\and \Class{scrreprt}\and \Class{scrartcl}}%
By default or with \OptionValue{index}{default}%
\IndexOption{index~=\textKValue{default}}, the index is an unnumbered chapter
(\Class{scrbook} or \Class{scrreprt}) or section (\Class{scrartcl}) without an
entry in the table of contents. Since\important{\OptionValue{index}{totoc}}
the index usually comes last in a book or similar document, it does not
actually need an entry in the table of contents. Nevertheless, if you want to
do this, for example because you are working with a multi-index keyword
dictionary such as the one that appears in this \iffree{guide}{book}, you can
create this with the
\OptionValue{index}{totoc}\IndexOption{index~=\textKValue{totoc}} option.
You\ChangedAt{v3.18}{\Class{scrbook}\and \Class{scrreprt}\and
  \Class{scrartcl}} can even number the index using the
\OptionValue{index}{numbered}\IndexOption{index~=\textKValue{numbered}}
setting. See also the \DescRef{\LabelBase.option.toc} option with the
\PValue{index} or \PValue{indexnumbered} values in
\autoref{sec:\LabelBase.toc} starting at
\DescPageRef{\LabelBase.option.toc.index}.

For example, if you create multiple indexes with
\Package{splitidx}\IndexPackage{splitidx} (see \cite{package:splitindex}),
it may be useful to group them under a common heading. To make this possible,
\OptionValue{index}{leveldown}%
\ChangedAt{v3.18}{\Class{scrbook}\and \Class{scrreprt}\and
  \Class{scrartcl}}\important{\OptionValue{index}{leveldown}} places the index
one sectioning level deeper than usual. In \Class{scrbook} and
\Class{scrreprt}, it is no longer a chapter but a section; with
\Class{scrartcl}, a subsection. The \OptionValue{index}{standardlevel}%
\ChangedAt{v3.18}{\Class{scrbook}\and \Class{scrreprt}\and
  \Class{scrartcl}}\important{\OptionValue{index}{standardlevel}} option is 
the counterpart to this and cancels any instance of 
\OptionValue{index}{leveldown} used previously.

You can find a summary of the available values for the \PName{setting} of
\Option{index} in \autoref{tab:maincls.index}.

\begin{table}
  \caption{Available values for the \Option{index} option}
  \label{tab:maincls.index}
  \begin{desctabular}
    \pventry{leveldown}{%
      \ChangedAt{v3.18}{\Class{scrbook}\and \Class{scrreprt}\and
        \Class{scrartcl}}%
      The index is moved down by one sectioning level.%
      \IndexOption{index~=\textKValue{leveldown}}%
    }%
    \entry{\PValue{notoc}, \PValue{nottotoc}, \PValue{plainheading}}{%
      The index does not receive an entry in the table of contents.%
      \IndexOption{index~=\textKValue{nottotoc}}}%
    \entry{\PValue{numbered}, \PValue{tocnumbered}, \PValue{totocnumbered},
      \PValue{numberedtoc}, \PValue{numberedtotoc}}{%
      \ChangedAt{v3.18}{\Class{scrbook}\and \Class{scrreprt}\and
        \Class{scrartcl}}%
      The index receives an entry in the table of contents and is numbered.%
      \IndexOption{index~=\textKValue{numbered}}}%
    \pventry{standardlevel}{%
      \ChangedAt{v3.18}{\Class{scrbook}\and \Class{scrreprt}\and
        \Class{scrartcl}}%
      The index is at the usual sectioning level.%
      \IndexOption{index~=\textKValue{standardlevel}}%
    }%
    \entry{\PValue{toc}, \PValue{totoc}, \PValue{notnumbered}}{%
      The index receives an entry in the table of contents without being
      numbered.%
      \IndexOption{index~=\textKValue{totoc}}}%
  \end{desctabular}
\end{table}
%
\EndIndexGroup


\begin{Declaration}
  \Macro{setindexpreamble}\Parameter{preamble}
\end{Declaration}%
As with the bibliography, you can also provide a preamble to the
index. This is often the case if you have more than one index or if
you mark different kinds of references by highlighting the page
numbers in different ways.
\begin{Example}
  You have a document in which terms are both defined and used. The
  page numbers of definitions are in bold. Of course you want to make
  your reader aware of this fact. Thus you insert a preamble for the
  index:
\begin{lstcode}
  \setindexpreamble{All page numbers printed in \textbf{bold}
    refer to definitions of terms. Page numbers printed
    normally refer to pages where the term is used.\par\bigskip}
\end{lstcode}
\end{Example}

Note\textnote{Attention!} that the page style is changed for the first page
of the index. The page style that is applied is defined in the macro
\DescRef{\LabelBase.cmd.indexpagestyle} (see
\autoref{sec:\LabelBase.pagestyle},
\DescPageRef{\LabelBase.cmd.indexpagestyle}).

The usual \LaTeX{} packages and additional programs are responsible for
creating, sorting, and outputting the index.%
\iftrue % Umbruchoptimierung
  \iffree{}{ %
    For example, the \Package{imakeidx}\IndexPackage{imakeidx} package
    (see \cite{package:imakeidx}), which among other things automates the
    invocation of the \File{makeindex} or \File{xindy} program, is highly
    recommended. This eliminates a common source of error, particularly for
    \LaTeX{} beginners.%
  }
  {\KOMAScript}, like the standard classes, provides only the basic macros
  and environments for them.%
\fi
%
\EndIndexGroup
%
\EndIndexGroup
%
\EndIndexGroup

\endinput

%%% Local Variables:
%%% mode: latex
%%% mode: flyspell
%%% ispell-local-dictionary: "english"
%%% TeX-master: "../guide"
%%% End:
