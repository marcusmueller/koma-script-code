% ======================================================================
% preface.tex
% Copyright (c) Markus Kohm, 2008-2018
%
% This file is part of the LaTeX2e KOMA-Script bundle.
%
% This work may be distributed and/or modified under the conditions of
% the LaTeX Project Public License, version 1.3c of the license.
% The latest version of this license is in
%   http://www.latex-project.org/lppl.txt
% and version 1.3c or later is part of all distributions of LaTeX
% version 2005/12/01 or later and of this work.
%
% This work has the LPPL maintenance status "author-maintained".
%
% The Current Maintainer and author of this work is Markus Kohm.
%
% This work consists of all files listed in manifest.txt.
% ----------------------------------------------------------------------
% preface.tex
% Copyright (c) Markus Kohm, 2008-2018
%
% Dieses Werk darf nach den Bedingungen der LaTeX Project Public Lizenz,
% Version 1.3c, verteilt und/oder veraendert werden.
% Die neuste Version dieser Lizenz ist
%   http://www.latex-project.org/lppl.txt
% und Version 1.3c ist Teil aller Verteilungen von LaTeX
% Version 2005/12/01 oder spaeter und dieses Werks.
%
% Dieses Werk hat den LPPL-Verwaltungs-Status "author-maintained"
% (allein durch den Autor verwaltet).
%
% Der Aktuelle Verwalter und Autor dieses Werkes ist Markus Kohm.
%
% Dieses Werk besteht aus den in manifest.txt aufgefuehrten Dateien.
% ======================================================================

\KOMAProvidesFile{typearea.tex}
                 [$Date$
                  Vorwort zu Version 3.25]

\addchap{Vorwort zu \KOMAScript~3.25}

Die Anleitung zu \KOMAScript~3.25 profitiert wieder einmal davon, dass nahezu
zeitgleich mit dieser Version auch eine überarbeitete Neuauflage der
Print-Ausgabe \cite{book:komascript} und der eBook-Ausgabe
\cite{ebook:komascript} erscheinen wird. Das führte zu vielen Verbesserungen,
die sich auch auf die freie Anleitung auswirken.

Eine dieser Verbesserungen ist die Verlinkung von Befehlen, Umgebungen,
Optionen etc. jeweils auf die entsprechende Erklärung innerhalb dieser
Anleitung. Um innerhalb der Erklärung selbst nicht dazu verleitet zu werden,
erneut an den Anfang der Erklärung zu springen, und so quasi Leserekursionen
zu vermeiden, wurde die Verlinkung allerdings nur dann durchgeführt, wenn sie
tatsächlich von der aktuellen Stelle weg führt.

Viele der Verbesserungen gehen auch auf die fleißigen, freiwilligen
Korrekturleser zurück. Ihnen sei an dieser Stelle ausdrücklich gedankt.

\iffree{Leser dieser freien Bildschirm-Version müssen allerdings weiterhin mit
  gewissen Einschränkungen leben. So sind einige Informationen --
  hauptsächliche solche für fortgeschrittene Anwender oder die dazu geeignet
  sind, aus einem Anwender einen fortgeschrittenen Anwender zu machen -- der
  Buchfassung vorbehalten. Das führt auch dazu, dass weiterhin einige Links in
  dieser Anleitung lediglich zu einer Seite führen, auf der genau diese
  Tatsache erwähnt ist. Darüber hinaus ist die freie Version nur eingeschränkt
  zum Ausdruck geeignet. Der Fokus liegt vielmehr auf der Verwendung am
  Bildschirm parallel zur Arbeit an einem Dokument. Sie hat auch weiterhin
  keinen optimierten Umbruch, sondern ist quasi ein erster Entwurf, bei dem
  sowohl der Absatz- als auch der Seitenumbruch in einigen Fällen durchaus
  dürftig ist. Entsprechende Optimierungen sind den Buchausgaben
  vorbehalten.}{}

Nicht nur zur Anleitung erfahre ich inzwischen eher wenig Kritik. Auch zu den
Klassen und Paketen gibt es kaum noch Nachfragen nach neuen Möglichkeiten. Für
mich selbst bedeutet das, dass das Wissen um die Wünsche der Anwender
stagniert. Schon seit einigen Jahren habe ich daher hauptsächlich Dinge
implementiert, von denen ich annahm, dass sie nützlich sein könnten. Die
Rückmeldungen, die ich zu diesen neuen Möglichkeiten erhalten habe,
beschränken sich aber überwiegend auf Kritik daran, dass uralte \emph{Hacks},
die sich auf undokumentierte Eigenschaften von \KOMAScript{} stützen, in
einigen Fällen nicht mehr funktionieren. Freude darüber, dass derart unsaubere
Notlösungen nicht mehr notwendig sind, wurde hingegen kaum geäußert. Daher
habe ich beschlossen, Erweiterungen und Verbesserungen an \KOMAScript{} mehr
und mehr auf solche Dinge zu beschränken, die von Anwendern explizit
nachgefragt werden. Oder sollte es gar sein, dass \KOMAScript{} nach bald
25~Jahren schlicht einen Stand erreicht hat, in dem es alle Wünsche erfüllt?

Die rückläufige Entwicklung bei Fehlermeldungen ist leider ebenfalls nicht nur
erfreulich. Inzwischen ist häufig zu beobachten, dass diejenigen, die ein
Problem entdecken, dieses nicht mehr unmittelbar an mich melden, sondern sich
lediglich in irgendwelchen Internet-Foren darüber auslassen. Oftmals finden
sich in diesen Foren dann mehr oder weniger geschickte Notlösungen. Das ist
zwar grundsätzlich erfreulich, führt aber leider in der Regel dazu, dass das
Problem nie gemeldet und daher auch nie wirklich beseitigt wird. Dass solche
Notlösungen irgendwann selbst zu einem Problem werden können, versteht sich
von selbst und wurde bereits im vorherigen Absatz erwähnt.

Vereinzelt finden sich dankenswerter Weise Dritte, die mich auf solche Themen
hinweisen. Dies betrifft aber nur einzelne Beiträge in sehr wenigen Foren. Ein
direkter Kontakt zu demjenigen, bei dem das Problem aufgetreten ist, ist dann
meist nicht möglich, obwohl er teilweise wünschenswert wäre.

Daher sei noch einmal ausdrücklich darum gebeten, mir vermeintliche Bugs
wahlweise in Deutsch oder Englisch unmittelbar zu melden. Dabei ist
sprachliche Perfektion weniger wichtig. Die Meldung sollte halbwegs
verständlich und das Problem nachvollziehbar sein. Ein möglichst kurzes
Code-Beispiel ist in der Regel unabhängig von der darin verwendeten Sprache zu
verstehen. Im direkten Kontakt sind mir bei Bedarf auch Rückfragen
möglich. Bitte verlassen Sie sich nicht darauf, dass irgendwer das Problem
schon irgendwann melden wird. Gehen Sie davon aus, dass es nur behoben wird,
wenn Sie es selbst melden. Näheres zu Fehlermeldungen ist im ersten Kapitel
\iffree{der Anleitung}{des Buches} zu finden.

\bigskip\noindent
Markus Kohm, Neckarhausen bei Regen im März 2018
\endinput

%%% Local Variables: 
%%% mode: latex
%%% coding: utf-8
%%% TeX-master: "../guide.tex"
%%% End: 

