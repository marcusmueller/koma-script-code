% ======================================================================
% scrlfile.tex
% Copyright (c) Markus Kohm, 2001-2020
%
% This file is part of the LaTeX2e KOMA-Script bundle.
%
% This work may be distributed and/or modified under the conditions of
% the LaTeX Project Public License, version 1.3c of the license.
% The latest version of this license is in
%   http://www.latex-project.org/lppl.txt
% and version 1.3c or later is part of all distributions of LaTeX 
% version 2005/12/01 or later and of this work.
%
% This work has the LPPL maintenance status "author-maintained".
%
% The Current Maintainer and author of this work is Markus Kohm.
%
% This work consists of all files listed in manifest.txt.
% ----------------------------------------------------------------------
% scrlfile.tex
% Copyright (c) Markus Kohm, 2001-2020
%
% Dieses Werk darf nach den Bedingungen der LaTeX Project Public Lizenz,
% Version 1.3c, verteilt und/oder veraendert werden.
% Die neuste Version dieser Lizenz ist
%   http://www.latex-project.org/lppl.txt
% und Version 1.3c ist Teil aller Verteilungen von LaTeX
% Version 2005/12/01 oder spaeter und dieses Werks.
%
% Dieses Werk hat den LPPL-Verwaltungs-Status "author-maintained"
% (allein durch den Autor verwaltet).
%
% Der Aktuelle Verwalter und Autor dieses Werkes ist Markus Kohm.
% 
% Dieses Werk besteht aus den in manifest.txt aufgefuehrten Dateien.
% ======================================================================
%
% Chapter about scrlfile of the KOMA-Script guide
% Maintained by Markus Kohm
%
% ----------------------------------------------------------------------------
%
% Kapitel über scrlfile in der KOMA-Script-Anleitung
% Verwaltet von Markus Kohm
%
% ============================================================================

\KOMAProvidesFile{scrlfile.tex}%
                 [$Date$
                  KOMA-Script guide (chapter: scrlfile)]

\chapter{Paketabhängigkeiten mit \Package{scrlfile} 
  beherrschen}
\labelbase{scrlfile}

\BeginIndexGroup
\BeginIndex{Package}{scrlfile}

Die Einführung von \LaTeXe{} brachte 1994 eine Menge Neuerungen im Umgang mit
\LaTeX-Erweiterungen. So stehen dem Paketautor heute eine ganze Reihe von
Befehlen zur Verfügung, um festzustellen, ob ein anderes Paket oder eine
bestimmte Klasse verwendet wird und ob dabei bestimmte Optionen zur Anwendung
kommen. Der Paketautor kann selbst andere Pakete laden oder diesen Optionen
mit auf den Weg geben für den Fall, dass sie später noch geladen werden. Es
bestand daher die Hoffnung, dass es künftig unerheblich wäre, in welcher
Reihenfolge Pakete geladen werden. Diese Hoffnung hat sich leider nicht
erfüllt.

%\section{Die Sache mit den Paketabhängigkeiten}
%\seclabel{dependency}
%\begin{Explain}
Immer\textnote{Problem} häufiger definieren unterschiedliche Pakete den
gleichen Befehl neu oder um. Dabei ist es dann sehr entscheidend, in welcher
Reihenfolge die Pakete geladen werden. Manchmal ist das für den Anwender kaum
zu überschauen. Teilweise ist es auch notwendig, in irgendeiner Form auf das
Laden eines anderen Pakets zu reagieren.

Nehmen\textnote{Beispiel} wir als einfaches Beispiel das Laden des
\Package{longtable}-Pakets bei Verwendung von \KOMAScript{}. Das
\Package{longtable}-Paket definiert seine eigene Form von
Tabellenüberschriften. Diese passen perfekt zu den Standardklassen. Sie passen
aber überhaupt nicht zu den Voreinstellungen für die Tabellenüberschriften von
\KOMAScript{} und reagieren auch nicht auf die entsprechenden Möglichkeiten
der Konfiguration. Um dieses Problem zu lösen, müssen die Befehle von
\Package{longtable}, die für die Tabellenüberschriften zuständig sind, von
\KOMAScript{} umdefiniert werden. Allerdings sind die \KOMAScript{}-Klassen
bereits abgearbeitet, wenn das Paket geladen wird.

Ursprünglich\textnote{ursprünglicher Ansatz} bestand die einzige %
\iffalse% Umbruchkorrektur
Möglichkeit, dieses Problem zu lösen %
\else%
Lösungsmöglichkeit %
\fi%
darin, die Umdefinierung mit Hilfe von \Macro{AtBeginDocument} auf einen
späteren Zeitpunkt zu verschieben. Will der Anwender die entsprechende
Anweisung jedoch selbst umdefinieren, so sollte er dies eigentlich ebenfalls
in der Präambel tun. Das kann er jedoch nicht, weil \KOMAScript{} ihm dabei in
die Quere kommt. Er müsste die Umdefinierung also ebenfalls mit Hilfe von
\Macro{AtBeginDocument} durchführen.

Aber\textnote{Potential} eigentlich müsste \KOMAScript{} die Abarbeitung gar
nicht auf den Zeitpunkt von \Macro{begin}\PParameter{document} verschieben. Es
würde genügen, wenn sie bis unmittelbar nach dem Laden von \Package{longtable}
verzögert werden könnte. Leider fehlen entsprechende Anweisungen im
\LaTeX-Kern. Das Paket \Package{scrlfile} bringt hier Abhilfe.

Ebenso\textnote{weitere Anwendungen} wäre es denkbar, dass man vor dem Laden
eines bestimmten Pakets gerne die Bedeutung eines Makros in einem Hilfsmakro
retten und nach dem Laden des Pakets wieder restaurieren will. Auch das geht
mit \Package{scrlfile}.

Die Anwendung von \Package{scrlfile} ist nicht auf die Abhängigkeit von
Paketen beschränkt. Auch Abhängigkeiten von anderen Dateien, die mit
\Macro{input} oder \Macro{InputIfFileExists} geladen werden, können
berücksichtigt werden.%
\iffalse% Stimmt so nur noch bedingt
So kann beispielsweise dafür gesorgt werden, dass das nicht unkritische Laden
einer Datei wie \File{french.ldf} automatisch zu einer Warnung führt.%
\fi

Obwohl das Paket in erster Linie für andere Paketautoren interessant sein
dürfte, gibt es durchaus auch Anwendungen für normale \LaTeX-Benutzer. Deshalb
sind in diesem Kapitel auch für beide Gruppen Beispiele aufgeführt.
%\end{Explain}

\begin{Declaration}
  \Option{withdeprecated}
\end{Declaration}
Mit\ChangedAt{v3.32}{\Package{scrlfile}} \LaTeX{} 2020-10-01 steht ein
komplett neuer \emph{Hook}-Mechanismus zur Verfügung. Statt Makros des
\LaTeX-Kerns umzudefinieren, macht \Package{scrlfile} ab Version 3.32 davon
auch Gebrauch. Dazu lädt es das interne Paket
\Package{scrlfile-hook}\IndexPackage[indexmain]{scrlfile-hook}. Bei älteren
\LaTeX-Versionen wird dagegen das interne Paket
\Package{scrlfile-patcholdkernel}%
\IndexPackage[indexmain]{scrlfile-patcholdkernel} verwendet. Da mit dem neuen
Mechanismus in \LaTeX{} jedoch keine identische Funktionalität zu erreichen
war, wurden einige Anweisungen von \Package{scrlfile} als veraltet
markiert. Darüber hinaus wurde die Gelegenheit genutzt, den Wildwuchs an
Befehlsvarianten zu ordnen. Sollte ein Anwender eine Klasse oder Pakete
verwenden, die noch auf den dadurch nicht mehr unterstützten Befehlen
basieren, so kann vor \Package{scrlfile} vor dem Laden der Klasse mit
\begin{lstcode}
  \RequirePackage[withdeprecated]{scrlfile}
\end{lstcode}
geladen werden. \Package{scrlfile} emuliert dann zusätzlich einige der
veralteten Befehle in einer Weise, die in den meisten Fällen ausreichend sein
sollte. In jedem Fall sollte jedoch der Autor der entsprechenden Klasse oder
des entsprechenden Pakets verständigt werden, damit dieser eine Anpassung an
die aktuelle Version von \Package{scrlfile} oder direkt an \LaTeX{} ab Version
2020-10-01 vornimmt.%
%
\EndIndexGroup


\section{Aktionen vor und nach dem Laden}
\seclabel{macros}

Mit \Package{scrlfile} können vor und nach dem Laden von Dateien Aktionen
ausgelöst werden. Bei den dazu verwendeten Befehlen wird zwischen allgemeinen
Dateien, Klassen und Paketen unterschieden.


\begin{Declaration}
  \Macro{BeforeFile}\Parameter{Datei}\OParameter{Label}\Parameter{Anweisungen}
  \Macro{AfterFile}\Parameter{Datei}\OParameter{Label}\Parameter{Anweisungen}
\end{Declaration}%
Mit Hilfe von \Macro{BeforeFile} kann dafür gesorgt werden, dass die
\PName{Anweisungen} vor dem nächsten Laden einer bestimmten \PName{Datei}
ausgeführt werden. Vergleichbar arbeitet \Macro{AfterFile}. Nur werden die
\PName{Anweisungen} hier erst nach dem Laden der \PName{Datei}
ausgeführt. Wird die Datei nie geladen, so werden die \PName{Anweisungen} in
beiden Fällen natürlich auch nie ausgeführt. Bei \PName{Datei} sind etwaige
Dateiendungen wie bei \Macro{input} als Teil des Dateinamens anzugeben.

Um die Funktionalität für \LaTeX{} vor Version 2020-10-01 bereitstellen zu
können, bedient sich \Package{scrlfile-patcholdlatex} der bekannten
\LaTeX-Anweisung \Macro{InputIfFileExists}. Diese\textnote{Achtung!} wird
hierzu umdefiniert.  Falls die Anweisung nicht die erwartete Definition hat,
gibt \Package{scrlfile-patcholdlatex} eine Warnung aus.  Dies geschieht für
den Fall, dass die Anweisung bereits von einem inkompatiblen Paket umdefiniert
wurde.

Bei\ChangedAt{v3.32}{\Package{scrlfile}}\IndexCmd{AddToHook} \LaTeX{} ab
Version 2020-10-01 wird von \Package{scrlfile-hook} hingegen
\Macro{AddToHook}\PParameter{file/before/\PName{Datei}}\OParameter{Label}\Parameter{Anweisungen}
beziehungsweise
\Macro{AddToHook}\PParameter{file/after/\PName{Datei}}\OParameter{Label}\Parameter{Anweisungen}
verwendet. Näheres zur Bedeutung des optionalen Arguments \PName{Label} ist
der zugehörigen Anleitung zur \LaTeX-Kern-Anweisung \Macro{AddToHook} zu
entnehmen.
  
Die Anweisung \Macro{InputIfFileExists} wird von \LaTeX{} immer verwendet,
wenn eine Datei geladen werden soll. Dies geschieht unabhängig davon, ob die
Datei mit \Macro{LoadClass}, \Macro{documentclass}, \Macro{usepackage},
\Macro{RequirePackage}, \Macro{include} oder vergleichbaren Anweisungen
geladen wird. Lediglich
% Umbruchkorrektur: listings korrigieren
\begin{lstcode}
  \input foo
\end{lstcode}
lädt die Datei \texttt{foo} ohne Verwendung von \Macro{InputIfFileExists}. Sie
sollten daher stattdessen immer
% Umbruchkorrektur: listings korrigieren
\begin{lstcode}
  \input{foo}
\end{lstcode}
verwenden. Beachten Sie die Klammern um den Dateinamen!%
%
\EndIndexGroup


\begin{Declaration}
  \Macro{BeforeClass}\Parameter{Klasse}\OParameter{Label}\Parameter{Anweisungen}
  \Macro{BeforePackage}\Parameter{Paket}\OParameter{Label}\Parameter{Anweisungen}
\end{Declaration}%
Diese beiden Befehle arbeiten vergleichbar zu
\DescRef{\LabelBase.cmd.BeforeFile} mit dem einen Unterschied, dass die
\PName{Klasse} beziehungsweise das \PName{Paket} mit seinem Namen und nicht
mit seinem Dateinamen angegeben wird. Die Endungen »\File{.cls}« und
»\File{.sty}« entfallen hier also.

Es ist zu beachten\important{Achtung!}, dass hier von \Package{scrlfile-hook}
ebenfalls \PValue{file} und nicht \PValue{class} respektive \PValue{package}
für den Hook verwendet wird. Nur so ist sichergestellt, dass die Ausführung
der \PName{Anweisungen} bereits im Kontext der Klasse beziehungsweise des
Pakets erfolgt. Zur Bedeutung des optionalen Arguments \PName{label} sei
wiederum auf die Anleitung der \LaTeX-Kernanweisung
\Macro{AddToHook}\IndexCmd{AddToHook} verwiesen.%
\EndIndexGroup


\begin{Declaration}
  \Macro{AfterClass}\Parameter{Klasse}\OParameter{Label}\Parameter{Anweisungen}
  \Macro{AfterClass*}\Parameter{Klasse}\OParameter{Label}\Parameter{Anweisungen}
  \Macro{AfterAtEndOfClass}\Parameter{Klasse}\OParameter{Label}\Parameter{Anweisungen}
  \Macro{AfterAtEndOfClass*}\Parameter{Klasse}\OParameter{Label}\Parameter{Anweisungen}
  \Macro{AfterPackage}\Parameter{Paket}\OParameter{Label}\Parameter{Anweisungen}
  \Macro{AfterPackage*}\Parameter{Paket}\OParameter{Label}\Parameter{Anweisungen}
  \Macro{AfterAtEndOfPackage}\Parameter{Paket}\OParameter{Label}\Parameter{Anweisungen}
  \Macro{AfterAtEndOfPackage*}\Parameter{Paket}\OParameter{Label}\Parameter{Anweisungen}
\end{Declaration}%
Die\important[i]{\Macro{AfterClass}\\\Macro{AfterPackage}} Anweisungen
\Macro{AfterClass} und \Macro{AfterPackage} arbeiten weitgehend wie
\DescRef{\LabelBase.cmd.AfterFile}, mit dem winzigen Unterschied, dass die
\PName{Klasse} beziehungsweise das \PName{Paket} mit seinem Namen und nicht
mit seinem Dateinamen angegeben wird. Die Endungen »\File{.cls}« und
»\File{.sty}« entfallen hier also.

Bei\important[i]{\Macro{AfterClass*}\\\Macro{AfterPackage*}} den
Sternvarianten \Macro{AfterClass*} und \Macro{AfterPackage*} gibt es eine
zusätzliche Funktionalität. Wurde oder wird die entsprechende Klasse oder das
entsprechende Paket bereits geladen, so werden die \PName{Anweisungen} nicht
nach dem nächsten Laden, sondern unmittelbar ausgeführt.

Code,\ChangedAt{v3.09}{\Package{scrlfile}}%
\important[i]{\Macro{AfterAtEndOfClass}\\\Macro{AfterAtEndOfPackage}}
dessen Ausführung per \Macro{AtEndOfClass}\IndexCmd{AtEndOfClass} oder
\Macro{AtEndOfPackage}\IndexCmd{AtEndOfPackage} innerhalb der Klasse
respektive des Pakets verzögert wird, wird allerdings erst danach
ausgeführt. Mit \Macro{AfterAtEndOfClass} beziehungsweise
\Macro{AfterAtEndOfPackage} wird dagegen sichergestellt, dass
\PName{Anweisungen} erst nach solchem Code ausgeführt wird.

Auch\ChangedAt{v3.32}{\Package{scrlfile}}%
\important[i]{\Macro{AfterAtEndOfClass*}\\\Macro{AfterAtEndOfPackage*}}
hierzu gibt es Sternvarianten, die dafür sorgen, dass \PName{Anweisungen}
sofort ausgeführt werden, falls die Klasse beziehungsweise das Paket bereits
vollständig geladen ist. Wird eine Klasse oder ein Paket gerade geladen, so
wird die Ausführung wie bei der Version ohne Stern verzögert.

Es ist zu beachten, dass \Package{scrlfile-hook} für \Macro{AfterClass},
\Macro{AfterPackage} und deren Sternvarianten einen \PValue{file/after}-Hook
verwendet. Dagegen setzen \Macro{AfterAtEndOfClass} und dessen Sternvariante
einen \PValue{class/after}-Hook und \Macro{AfterAtEndOfPackage} und dessen
Sternvariante entsprechend einen \PValue{package/after}-Hook. Zur Bedeutung
des optionalen Arguments \PName{label} sei wiederum auf die Anleitung der
\LaTeX-Kernanweisung \Macro{AddToHook}\IndexCmd{AddToHook} verwiesen.

\begin{Example}
  Als Beispiel für Paket- oder Klassenautoren will ich zunächst
  erklären, wie \KOMAScript{} selbst Gebrauch von den neuen
  Anweisungen macht. Dazu findet sich beispielsweise in \Class{scrbook}
  Folgendes:
\begin{lstcode}
  \AfterPackage{hyperref}{%
    \@ifpackagelater{hyperref}{2001/02/19}{}{%
      \ClassWarningNoLine{scrbook}{%
        You are using an old version of the hyperref 
        package!\MessageBreak%
        This version has a buggy hack in many 
        drivers,\MessageBreak%
        causing \string\addchap\space to behave 
        strangely.\MessageBreak%
        Please update hyperref to at least version
        6.71b}%
    }%
  }
\end{lstcode}
  Alte Versionen von \Package{hyperref} definierten ein Makro von
  \Class{scrbook} in einer Weise um, die mit neueren Versionen
  von \KOMAScript{} nicht mehr funktioniert. Neuere Versionen von
  \Package{hyperref} unterlassen dies, wenn sie eine neuere Version
  von \KOMAScript{} erkennen. Für den Fall, dass \Package{hyperref}
  zu einem späteren Zeitpunkt geladen wird, sorgt also \Class{scrbook}
  dafür, dass unmittelbar nach dem Laden des Pakets überprüft wird, ob
  es sich um eine verträgliche Version handelt. Falls dies nicht der
  Fall ist, wird eine Warnung ausgegeben.

  An anderer Stelle findet sich in drei der \KOMAScript-Klassen folgendes:
\begin{lstcode}
  \AfterPackage{caption2}{%
    \renewcommand*{\setcapindent}{%
\end{lstcode}% }}
  Nach dem Laden von \Package{caption2} und nur falls das Paket
  geladen wird, wird hier die \KOMAScript{} eigene Anweisung
  \DescRef{maincls.cmd.setcapindent} umdefiniert. Der Inhalt der Umdefinierung
  ist für dieses Beispiel unerheblich. Es sei nur erwähnt, dass
  \Package{caption2} die Kontrolle über die
  \DescRef{maincls.cmd.caption}-Anweisung übernimmt und daher die normale
  Definition von \DescRef{maincls.cmd.setcapindent} keinerlei Wirkung mehr
  hätte. Die Umdefinierung verbessert dann die Zusammenarbeit mit dem
  veralteten \Package{caption2}.

  Es gibt aber auch Beispiele für den sinnvollen Einsatz der neuen
  Anweisungen durch normale Anwender. Angenommen, Sie erstellen ein
  Dokument, aus dem sowohl eine PS-Datei mit \LaTeX{} und dvips als auch
  eine PDF-Datei mit \mbox{pdf\LaTeX} erstellt werden soll. Das Dokument soll
  außerdem Hyperlinks aufweisen. Im Tabellenverzeichnis haben Sie
  Einträge, die über mehrere Zeilen gehen. Nun gibt es zwar mit
  \mbox{pdf\LaTeX} bei der PDF-Ausgabe keine Probleme, da dort Links
  umbrochen werden können. Bei Verwendung des
  \Package{hyperref}-Treibers für dvips oder
  \mbox{hyper\TeX} ist dies jedoch nicht
  möglich. In diesem Fall hätten Sie gerne, dass bei
  \Package{hyperref} die Einstellung \Option{linktocpage} verwendet
  wird. Die Entscheidung, welcher Treiber geladen wird, wird von
  \Package{hyperref} automatisch getroffen.

  Alles weitere kann nun \DescRef{\LabelBase.cmd.AfterFile} überlassen werden:
\begin{lstcode}[moretexcs={hypersetup}]
  \documentclass{article}
  \usepackage[ngerman]{babel}
  \usepackage{scrlfile}
  \AfterFile{hdvips.def}{\hypersetup{linktocpage}}
  \AfterFile{hypertex.def}{\hypersetup{linktocpage}}
  \usepackage{hyperref}
  \begin{document}
  \listoffigures
  \clearpage
  \begin{figure}
    \caption{Dies ist ein Beispiel mit einer
      Abbildungsunterschrift, die mehrere Zeile
      umfasst und bei der trotzdem auf die
      Verwendung des optionalen Arguments verzichtet
      wurde.}
  \end{figure}
  \end{document}
\end{lstcode}
  Egal, ob nun der \Package{hyperref}-Treiber \Option{hypertex} oder
  \Option{dvips} zur Anwendung kommt, wird die dann nützliche Einstellung
  \Option{linktocpage} verwendet. Wenn Sie jedoch mit \mbox{pdf\LaTeX} eine
  PDF-Datei erstellen, wird darauf verzichtet, da dann der
  \Package{hyperref}-Treiber \File{hpdftex.def} verwendet wird. Das bedeutet,
  dass weder die Treiberdatei \File{hdvips.def} noch \File{hypertex.def}
  geladen wird.
\end{Example}
\iffalse% Umbruchkorrekturtext (der besser nicht mehr verwendet wird!)
Übrigens\textnote{Tipp!} können Sie das Laden von \Package{scrlfile} und die
obigen Anweisungen \DescRef{\LabelBase.cmd.AfterFile} auch in Ihre private
\File{hyperref.cfg} (siehe oben) einfügen. Verwenden Sie dabei jedoch zum
Laden des Pakets besser \Macro{RequirePackage} anstelle von
\Macro{usepackage} (siehe \cite{latex:clsguide}). Die neuen Zeilen müssen in
obigem Beispiel unmittelbar nach der \Macro{ProvidesFile}-Zeile, also
unbedingt vor der Ausführung der Optionen \Option{dvips} oder \Option{pdftex},
eingefügt werden.%
\fi%
Übrigens\textnote{Tipp!} kann \Package{scrlfile} auch bereits vor
\DescRef{maincls.cmd.documentclass}\IndexCmd{documentclass} geladen werden. In
diesem Fall ist allerdings \Macro{RequirePackage}\IndexCmd{RequirePackage}
anstelle von \DescRef{maincls.cmd.usepackage} zu verwenden (siehe
\cite{latex:clsguide}).%
\EndIndexGroup


\begin{Declaration}
  \Macro{BeforeClosingMainAux}\Parameter{Anweisungen}
  \Macro{AfterReadingMainAux}\Parameter{Anweisungen}
\end{Declaration}
\iffalse% Umbruchkorrektur
Diese Anweisungen unterscheiden sich in einem Detail von den zuvor erklärten
Anweisungen. Jene ermöglichen Aktionen vor und nach dem Laden von Dateien. Das
ist hier nicht der Fall. %
\fi%
Paketautoren haben des Öfteren das Problem, dass sie Anweisungen in die
\File{aux}-Datei schreiben wollen, nachdem die letzte Seite des Dokuments
ausgegeben wurde. Dazu wird -- in Unkenntnis der dadurch verursachten Probleme
-- häufig Code wie der folgende eingesetzt:
\begin{lstcode}[escapechar=\$]
  \AtEndDocument{%$\textnote{Nicht nachmachen!}$
    \if@filesw
      \write\@auxout{\protect\writethistoaux}%
    \fi
  } 
\end{lstcode}
Dies ist jedoch keine wirkliche Lösung. Wurde die letzte
Seite vor \Macro{end}\PParameter{document} bereits ausgegeben, so führt obiges
zu keiner Ausgabe in die \File{aux}-Datei. Würde man zur Lösung dieses
Problems nun ein \Macro{immediate} vor \Macro{write} setzen, so hätte man das
umgekehrte Problem: wurde die letzte Seite bei
\Macro{end}\PParameter{document} noch nicht ausgegeben, so wird
\Macro{writethistoaux} zu früh in die \File{aux}-Datei geschrieben. Man sieht
daher häufig auch Lösungsversuche wie:
\begin{lstcode}[escapechar=\$]
  \AtEndDocument{%$\textnote{Nicht nachmachen!}$
    \if@filesw
      \clearpage
      \immediate\write\@auxout{\protect\writethistoaux}%
    \fi
  } 
\end{lstcode}
Diese Lösung hat jedoch den Nachteil, dass damit die Ausgabe der letzten Seite
erzwungen wird. Eine Anweisung wie
\begin{lstcode}
  \AtEndDocument{%
    \par
    \vspace*{\fill}%
    Hinweis am Ende des Dokuments.
    \par
  }
\end{lstcode}
führt dann nicht mehr dazu, dass der Hinweis am Ende der letzten Seite des
Dokuments ausgegeben wird, sie würde stattdessen am Ende der nächsten Seite
ausgegeben. Gleichzeitig würde \Macro{writethistoaux} wieder eine Seite zu
früh in die \File{aux}-Datei geschrieben.

Die beste Lösung des Problems wäre nun, wenn man unmittelbar in die
\File{aux}-Datei schreiben könnte, nachdem das finale
\DescRef{maincls.cmd.clearpage} innerhalb von \Macro{end}\PParameter{document}
ausgeführt, aber bevor die \File{aux}-Datei geschlossen wird. Dies ist das
Ziel von \Macro{BeforeClosingMainAux}:
\begin{lstcode}
  \BeforeClosingMainAux{%
    \if@filesw
      \immediate\write\@auxout{\protect\writethistoaux}%
    \fi
  }
\end{lstcode}
Das ist auch dann erfolgreich, wenn das finale \DescRef{maincls.cmd.clearpage}
innerhalb von \Macro{end}\PParameter{document} tatsächlich zu keiner Ausgabe
einer Seite mehr führt oder wenn -- sei es korrekt verwendet oder in
Unkenntnis der oben erläuterten Probleme -- \DescRef{maincls.cmd.clearpage}
innerhalb einer \Macro{AtEndDocument}-Anweisung zum Einsatz kam.

Es gibt jedoch für \Macro{BeforeClosingMainAux} eine Einschränkung: Im
Argument \PName{Anweisungen} sollte keine Satzanweisung verwendet
werden. Es darf also mit \Macro{BeforeClosingMainAux} kein zusätzliches
Material gesetzt werden! Wird diese Einschränkung nicht beachtet, so ist das
Ergebnis ebenso unvorhersehbar wie bei den gezeigten Problemen mit
\Macro{AtEndDocument}.

Die Anweisung \Macro{AfterReadingMainAux}\ChangedAt{v3.03}{\Package{scrlfile}}
führt sogar \PName{Anweisungen} nach dem Schließen und Einlesen der
\File{aux}-Datei innerhalb von \Macro{end}\PParameter{document} aus. Dies ist
nur in einigen wenigen, sehr seltenen Fällen sinnvoll, beispielsweise, wenn
man statistische Informationen in die \File{log}-Datei schreiben will, die
erst nach dem Einlesen der \File{aux}-Datei gültig sind, oder zur
Implementierung zusätzlicher
\mbox{\emph{Rerun}}-\mbox{Auf"|forderungen}. Satzanweisungen sind an dieser
Stelle noch kritischer zu betrachten als bei \Macro{BeforeClosingMainAux}.%
%
\EndIndexGroup


\section{Dateien beim Einlesen ersetzen}
\seclabel{replace}

In den bisherigen Abschnitten wurden Anweisungen erklärt, mit denen es möglich
ist, vor oder nach dem Einlesen einer bestimmten Datei, eines bestimmten
Pakets oder einer Klasse Aktionen auszuführen. Es ist mit \Package{scrlfile}
aber auch möglich, eine ganz andere Datei als die angeforderte einzulesen.

\begin{Declaration}[0]
  \Macro{ReplaceInput}\Parameter{Dateiname}\Parameter{Ersatzdatei}
\end{Declaration}
Mit\ChangedAt{v2.96}{\Package{scrlfile}} dieser Anweisung wird eine Ersetzung
der Datei mit dem als erstes angegebenen \PName{Dateiname} definiert. Wenn
\LaTeX{} anschließend angewiesen wird, diese Datei zu laden, wird stattdessen
\PName{Ersatzdatei} geladen. Die Definition der Ersatzdatei wirkt sich auf
alle Dateien aus, die vom Anwender oder intern von \LaTeX{} mit Hilfe von
\Macro{InputIfFileExists} geladen werden.

Bei\ChangedAt{v3.32}{\Package{scrlfile}} Verwendung eines \LaTeX-Version bis
2020-04-01 muss dass Paket \Package{scrlfile-patcholdlatex} dazu die Anweisung
\Macro{InputIfFileExists} umdefinieren. Bei Verwendung von \LaTeX{} ab Version
2020-10-01 wird hingegen von \Package{scrlfile-hook} die interne
\LaTeX-Anweisung \Macro{declare@file@substitution} verwendet. Das
\LaTeX{}-Team bittet darum, eine derartige Dateiersetzung nur vorzunehmen,
wenn es keine andere Möglichkeit gibt, zu dem gewünschten Ergebnis zu
gelangen, beispielsweise wenn eine solche Ersetzung zum Erhalt der
Kompatibilität unabdingbar ist und dabei die \PName{Ersatzdatei} die gleiche
Funktionalität bereitstellt.

\begin{Example}
  Sie wollen, dass anstelle der Datei \File{\Macro{jobname}.lot}, die Datei
  \File{\Macro{jobname}.tol} geladen wird. Dazu verwenden Sie:
\begin{lstcode}
  \ReplaceInput{\jobname.lot}{\jobname.tol}
\end{lstcode}
  Wenn Sie nun zusätzlich \File{\Macro{jobname}.tol} auch noch durch
  \File{\Macro{jobname}.tlo} ersetzen:
\begin{lstcode}
  \ReplaceInput{\jobname.tol}{\jobname.tlo}
\end{lstcode}
  dann wird \File{\Macro{jobname}.lot} am Ende durch
  \File{\Macro{jobname}.tlo} ersetzt. Es wird also die komplette
  Ersetzungskette abgearbeitet.

  Einer Ersetzung im Kreis:
\begin{lstcode}
  \ReplaceInput{\jobname.lot}{\jobname.tol}
  \ReplaceInput{\jobname.tol}{\jobname.lot}
\end{lstcode}
  würde jedoch zu einem Fehler führen. Es ist also nicht möglich, eine einmal
  ersetzte Datei wieder durch ihren Ursprung zu ersetzen.
\end{Example}

Theoretisch wäre es auch möglich, auf diesem Wege ein Paket durch ein anderes
oder eine Klasse durch eine andere zu ersetzen. Es wird jedoch empfohlen, für
das Ersetzen eines Pakets oder einer Klasse die nachfolgenden Anweisungen zu
verwenden. Bei älteren \LaTeX-Versionen ist dies sogar zwingend.%
\EndIndexGroup


\begin{Declaration}
  \Macro{ReplaceClass}\Parameter{Klasse}\Parameter{Ersatzklasse}
  \Macro{ReplacePackage}\Parameter{Paket}\Parameter{Ersatzpaket}
\end{Declaration}
Eine\ChangedAt{v2.96}{\Package{scrlfile}}\textnote{Achtung!} Klasse oder ein
Paket sollte niemals mit Hilfe der oben erklärten Anweisung
\DescRef{\LabelBase.cmd.ReplaceInput} ersetzt werden. Stattdessen sollte für
Paketersetzungen \Macro{ReplacePackage} und für Klassenersetzungen
\Macro{ReplaceClass} verwendet werden. Es ist zu beachten, dass wie bei
\Macro{documentclass} und \Macro{usepackage} der Name der Klasse oder des
Pakets und nicht deren kompletter Dateiname anzugeben ist.

Die Ersetzung funktioniert für Klassen, die mit \Macro{documentclass},
\Macro{LoadClassWithOptions} oder \Macro{LoadClass} geladen werden. Für Pakete
funktioniert die Ersetzung beim Laden mit \Macro{usepackage},
\Macro{RequirePackageWithOptions} und \Macro{RequirePackage}.

Es\textnote{Achtung!} ist zu beachten, dass die \PName{Ersatzklasse} oder das
\PName{Ersatzpaket} mit denselben Optionen geladen wird, mit denen die
ursprünglich geforderte Klasse oder das ursprünglich geforderte Paket geladen
würden. Wird ein Paket oder eine Klasse durch ein Paket oder eine Klasse
ersetzt, die eine geforderte Option nicht unterstützt, würde das zu den
üblichen Warnungen und Fehlern führen.

Damit Klassen mit \Macro{ReplaceClass} ersetzt werden können, ist es natürlich
erforderlich \Package{scrlfile} vor der Klasse zu laden. Dazu ist
\Macro{RequirePackage}\IndexCmd{RequirePackage} anstelle von
\DescRef{maincls.cmd.usepackage} zu verwenden (siehe \cite{latex:clsguide}).

Bei\ChangedAt{v3.32}{\Package{scrlfile}} Verwendung eines \LaTeX-Version bis
2020-04-01 muss dass Paket \Package{scrlfile-patcholdlatex} die internen
Anweisung \Macro{@onefilewithoptions} und \Macro{@loadwithoptions}
umdefinieren. Bei Verwendung von \LaTeX{} ab Version 2020-10-01 wird hingegen
von \Package{scrlfile-hook} die interne \LaTeX-Anweisung
\Macro{declare@file@substitution} verwendet. Das \LaTeX{}-Team bittet darum,
eine derartige Ersetzung nur vorzunehmen, wenn es keine andere Möglichkeit
gibt, zu dem gewünschten Ergebnis zu gelangen, beispielsweise wenn eine solche
Ersetzung zum Erhalt der Kompatibilität unabdingbar ist und dabei die
\PName{Ersatzpaket} beziehungsweise \PName{Ersatzklasse} die gleiche
Funktionalität bereitstellt.%
\EndIndexGroup


\begin{Declaration}
  \Macro{UnReplaceInput}\Parameter{Datei}
  \Macro{UnReplacePackage}\Parameter{Paket}
  \Macro{UnReplaceClass}\Parameter{Klasse}
\end{Declaration}
Eine\ChangedAt{v3.12}{\Package{scrlfile}} Ersetzung kann auch wieder
aufgehoben werden. Dabei sollten Ersetzungen von Dateien immer mit
\Macro{UnReplaceInput}, Ersetzungen von Paketen mit \Macro{UnReplacePackage}
und Ersetzungen von Klassen mit \Macro{UnReplaceClass} aufgehoben werden. Nach
der Aufhebung der Ersetzung führen Ladebefehle für die entsprechende
\PName{Datei}, das entsprechende \PName{Paket} oder die entsprechende
\PName{Klasse} dann wieder dazu, dass die \PName{Datei}, das \PName{Paket}
oder die \PName{Klasse} selbst anstelle der Ersatzdatei, des Ersatzpakets oder
der Ersatzklasse geladen wird.%
\EndIndexGroup


\section{Dateien gar nicht erst einlesen}
\seclabel{prevent}

Gerade\ChangedAt{v3.08}{\Package{scrlfile}} in Klassen und Paketen, die
innerhalb von Firmen oder Instituten verwendet werden, findet man häufig, dass
sehr viele Pakete nur deshalb geladen werden, weil die Anwender diese Pakete
oft verwenden. Wenn es dann mit einem dieser automatisch geladenen Pakete zu
einem Problem kommt, muss man irgendwie das Laden des problematischen Pakets
verhindern. Auch hier bietet \Package{scrlfile} eine einfache Lösung.

\begin{Declaration}
  \Macro{PreventPackageFromLoading}\OParameter{Stattdessencode}
                                   \Parameter{Paketliste}
  \Macro{PreventPackageFromLoading*}\OParameter{Stattdessencode}
                                    \Parameter{Paketliste}
\end{Declaration}
Wird diese Anweisung\ChangedAt{v3.08}{\Package{scrlfile}} vor dem
Laden eines Pakets mit \Macro{usepackage}\IndexCmd{usepackage},
\Macro{RequirePackage}\IndexCmd{RequirePackage} oder
\Macro{RequirePackageWithOptions}\IndexCmd{RequirePackageWithOptions}
aufgerufen, so wird das Laden des Pakets effektiv verhindert, falls es in
der \PName{Paketliste} zu finden ist.
%
\begin{Example}
  Angenommen, Sie arbeiten in einer Firma, in der alle Dokumente mit
  Latin~Modern erzeugt werden. In der Firmenklasse, \Class{firmenci}, befinden
  sich daher die Zeilen:
\begin{lstcode}
  \RequirePackage[T1]{fontenc}
  \RequirePackage{lmodern}
\end{lstcode}
  Nun wollen Sie zum ersten Mal ein Dokument mit
  X\kern-.125em\lower.5ex\hbox{\reflectbox{E}}\LaTeX{} oder Lua\LaTeX{}
  setzen. Da beim hierbei empfohlenen Paket \Package{fontspec} ohnehin
  Latin-Modern voreingestellt ist und das Laden von \Package{fontenc} eher
  störend wäre, wollen Sie das Laden beider Pakete verhindern. Sie laden die
  Klasse deshalb nun in Ihrem eigenen Dokument wie folgt:
\begin{lstcode}
  \RequirePackage{scrlfile}
  \PreventPackageFromLoading{fontenc,lmodern}
  \documentclass{firmenci}
\end{lstcode}
\end{Example}
Wie im Beispiel zu sehen ist, kann man das Paket \Package{scrlfile} auch
bereits vor der Klasse laden. In diesem Fall muss das Laden dann aber mit
Hilfe von \Macro{RequirePackage}\IndexCmd{RequirePackage} erfolgen, da
\Macro{usepackage} vor \Macro{documentclass} verboten ist (siehe
\cite{latex:clsguide}).

Wird eine leere \PName{Paketliste} angegeben oder wird ein Paket angegeben,
das bereits geladen ist, gibt \Macro{PreventPackageFromLoading} eine
Warnung aus, während
\Macro{PreventPackageFromLoading*}\ChangedAt{v3.12}{\Package{scrlfile}}
lediglich einen entsprechenden Hinweis in die Log-Datei schreibt.

Das optionale Argument\ChangedAt{v3.12}{\Package{scrlfile}} kann verwendet
werden, wenn anstelle des Ladens des Pakets etwas anderes getan werden
soll. Innerhalb des \PName{Stattdessencode}s dürfen jedoch keine anderen
Pakete und keine Dateien geladen werden. Zum Laden eines anderen Pakets siehe
\DescRef{\LabelBase.cmd.ReplacePackage} in \autoref{sec:scrlfile.replace} auf
\DescPageRef{\LabelBase.cmd.ReplacePackage}. Beachten\textnote{Achtung!} Sie
bitte auch, dass der \PName{Stattdessencode} mehrfach ausgeführt wird, falls
Sie versuchen, das Paket mehrfach zu laden!%
\EndIndexGroup


\begin{Declaration}
  \Macro{StorePreventPackageFromLoading}\Parameter{\textMacro{Anweisung}}
  \Macro{ResetPreventPackageFromLoading}
\end{Declaration}
\Macro{Anweisung} wird mit
\Macro{StorePreventPackageFromLoading}\ChangedAt{v3.08}{\Package{scrlfile}}
als die aktuelle Liste der Pakete definiert, für die das Laden verhindert
werden soll. Dagegen setzt
\Macro{ResetPreventPackageFromLoading}\ChangedAt{v3.08}{\Package{scrlfile}}
die Liste der Pakete, für die das Laden verhindert werden soll, zurück. Danach
können wieder alle Pakete geladen werden.
\begin{Example}
  Angenommen, Sie sind innerhalb eines Pakets unbedingt auf das Laden eines
  anderen Pakets angewiesen und wollen nicht, dass der Anwender das Laden
  dieses Pakets mit
  \DescRef{\LabelBase.cmd.PreventPackageFromLoading}%
  \IndexCmd{PreventPackageFromLoading}
  verhindern kann. Also setzen Sie die Paketliste für die Ausnahmen zuvor
  zurück:
\begin{lstcode}
  \ResetPreventPackageFromLoading
  \RequirePackage{foo}
\end{lstcode}
  Allerdings hat dies den Nachteil, dass ab diesem Zeitpunkt die komplette
  Ausnahmeliste des Anwenders verloren ist. Also speichern Sie die Liste
  zunächst zwischen und reaktivieren sie später wieder:
\begin{lstcode}
  \newcommand*{\Users@PreventList}{}%
  \StorePreventPackageFromLoading\Users@PreventList
  \ResetPreventPackageFromLoading
  \RequirePackage{foo}
  \PreventPackageFromLoading{\Users@PreventList}
\end{lstcode}
  Es ist zu beachten, dass\textnote{Achtung!} \Macro{Users@PreventList} durch 
  die Anweisung \Macro{StorePreventPackageFromLoading} auch definiert
  werden würde, wenn diese bereits anderweitig definiert wäre. Eine vorhandene
  Definition würde also ohne Rücksicht überschrieben werden. In diesem
  Beispiel wurde deshalb mit einem vorherigen \Macro{newcommand*}
  sichergestellt, dass in dem Fall zur Sicherheit eine Fehlermeldung
  ausgegeben wird.
\end{Example}
An dieser Stelle sei darauf hingewiesen, dass Sie bei Manipulationen an der mit
\Macro{StorePreventPackageFromLoading} zwischengespeicherten Liste selbst die
Verantwortung für eine korrekte Wiederherstellbarkeit tragen. So muss die
Liste unbedingt mit Komma separiert sein, sollte keine Leerzeichen oder
Gruppenklammern enthalten und muss voll expandierbar sein.%

Beachten\textnote{Achtung!} Sie bitte, dass
\Macro{ResetPreventPackageFromLoading} den \PName{Stattdessencode} für ein
Paket nicht löscht, sondern nur vorübergehend dessen Ausführung nicht mehr
erfolgt.%
\EndIndexGroup


\begin{Declaration}
  \Macro{UnPreventPackageFromLoading}\Parameter{Paketliste}
  \Macro{UnPreventPackageFromLoading*}\Parameter{Paketliste}
\end{Declaration}
Statt\ChangedAt{v3.12}{\Package{scrlfile}} die Liste der Pakete, für die das
Laden verhindert werden soll, komplett zurückzusetzen, kann man auch
einzelne oder mehrere Pakete gezielt von dieser Liste entfernen. Die
Sternvariante des Befehls löscht außerdem den \PName{Stattdessencode}, der für
das Paket gespeichert ist. Falls die Verhinderungsliste beispielsweise aus
einer gespeicherten Liste wiederhergestellt wird, wird dann der
\PName{Stattdessencode} trotzdem nicht mehr ausgeführt.%
%
\begin{Example}
  Angenommen, Sie wollen zwar verhindern, dass ein Paket \Package{foo}
  geladen wird, wollen aber nicht, dass ein eventuell bereits gespeicherter
  \PName{Stattdessencode} ausgeführt wird. Stattdessen soll nur Ihr neuer
  \PName{Stattdessencode} ausgeführt werden. Dies ist wie folgt möglich:
\begin{lstcode}
  \UnPreventPackageFromLoading*{foo}
  \PreventPackageFromLoading[%
    \typeout{Stattdessencode}%
  ]{foo}
\end{lstcode}
  Für die Anweisung \Macro{UnPreventPackageFromLoading*} ist es unerheblich,
  ob das Paket zuvor überhaupt vom Laden ausgenommen war.

  Natürlich können Sie die Anweisung indirekt auch nutzen, um den
  \PName{Stattdessencode} aller Pakete zu löschen:
\begin{lstcode}
  \StorePreventPackageFromLoading\TheWholePreventList
  \UnPreventPackageFromLoading*{\TheWholePreventList}
  \PreventPackageFromLoading{\TheWholePreventList}
\end{lstcode}
  Die Pakete werden dann zwar noch immer nicht geladen, ihr
  \PName{Stattdessencode} existiert aber nicht mehr und wird nicht mehr
  ausgeführt.%
\end{Example}%
\EndIndexGroup
%\ExampleEndFix % Am Ende des Kapitels nicht ausführen
%
\EndIndexGroup

%%% Local Variables: 
%%% mode: latex
%%% coding: utf-8
%%% TeX-master: "../guide"
%%% End: 
