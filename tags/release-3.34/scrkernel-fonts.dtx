% \iffalse meta-comment
% ======================================================================
% scrkernel-fonts.dtx
% Copyright (c) Markus Kohm, 2002-2021
%
% This file is part of the LaTeX2e KOMA-Script bundle.
%
% This work may be distributed and/or modified under the conditions of
% the LaTeX Project Public License, version 1.3c of the license.
% The latest version of this license is in
%   http://www.latex-project.org/lppl.txt
% and version 1.3c or later is part of all distributions of LaTeX 
% version 2005/12/01 or later and of this work.
%
% This work has the LPPL maintenance status "author-maintained".
%
% The Current Maintainer and author of this work is Markus Kohm.
%
% This work consists of all files listed in manifest.txt.
% ----------------------------------------------------------------------
% scrkernel-fonts.dtx
% Copyright (c) Markus Kohm, 2002-2021
%
% Dieses Werk darf nach den Bedingungen der LaTeX Project Public Lizenz,
% Version 1.3c, verteilt und/oder veraendert werden.
% Die neuste Version dieser Lizenz ist
%   http://www.latex-project.org/lppl.txt
% und Version 1.3c ist Teil aller Verteilungen von LaTeX
% Version 2005/12/01 oder spaeter und dieses Werks.
%
% Dieses Werk hat den LPPL-Verwaltungs-Status "author-maintained"
% (allein durch den Autor verwaltet).
%
% Der Aktuelle Verwalter und Autor dieses Werkes ist Markus Kohm.
% 
% Dieses Werk besteht aus den in manifest.txt aufgefuehrten Dateien.
% ======================================================================
% \fi
%
% \CharacterTable
%  {Upper-case    \A\B\C\D\E\F\G\H\I\J\K\L\M\N\O\P\Q\R\S\T\U\V\W\X\Y\Z
%   Lower-case    \a\b\c\d\e\f\g\h\i\j\k\l\m\n\o\p\q\r\s\t\u\v\w\x\y\z
%   Digits        \0\1\2\3\4\5\6\7\8\9
%   Exclamation   \!     Double quote  \"     Hash (number) \#
%   Dollar        \$     Percent       \%     Ampersand     \&
%   Acute accent  \'     Left paren    \(     Right paren   \)
%   Asterisk      \*     Plus          \+     Comma         \,
%   Minus         \-     Point         \.     Solidus       \/
%   Colon         \:     Semicolon     \;     Less than     \<
%   Equals        \=     Greater than  \>     Question mark \?
%   Commercial at \@     Left bracket  \[     Backslash     \\
%   Right bracket \]     Circumflex    \^     Underscore    \_
%   Grave accent  \`     Left brace    \{     Vertical bar  \|
%   Right brace   \}     Tilde         \~}
%
% \iffalse
%%% From File: $Id$
%<identify>%%%            (run: identify)
%<option>%%%            (run: option)
%<body>%%%            (run: body)
%<10pt>%%%            (run: 10pt)
%<11pt>%%%            (run: 11pt)
%<11pt>%%%            (run: 12pt)
%<*dtx>
\ifx\ProvidesFile\undefined\def\ProvidesFile#1[#2]{}\fi
\begingroup
  \def\filedate$#1: #2-#3-#4 #5${\gdef\filedate{#2/#3/#4}}
  \filedate$Date$
  \def\filerevision$#1: #2 ${\gdef\filerevision{r#2}}
  \filerevision$Revision: 1827 $
  \edef\reserved@a{%
    \noexpand\endgroup
    \noexpand\ProvidesFile{scrkernel-fonts.dtx}%
                          [\filedate\space\filerevision\space
                           KOMA-Script source
                           (font size)]
  }%
\reserved@a
% \changes{v3.09}{2011/02/23}{Neues Paket \texttt{scrfontsizes}}
%</dtx>
%<*package&generator&identify>
%<package&generator&identify>\ProvidesPackage{scrfontsizes}[%
%!KOMAScriptVersion
%<package&generator&identify>  package (font size file generator)]
%</package&generator&identify>
%<*dtx>
\documentclass[parskip=half-]{scrdoc}
\usepackage[english,ngerman]{babel}
\CodelineIndex
\RecordChanges
\GetFileInfo{scrkernel-fonts.dtx}
\title{\KOMAScript{} \partname\ \texttt{\filename}%
  \footnote{Dies ist Version \fileversion\ von Datei \texttt{\filename}.}}
\date{\filedate}
\author{Markus Kohm}

\begin{document}
  \maketitle
  \tableofcontents
  \DocInput{\filename}
\end{document}
%</dtx>
% \fi
%
% \selectlanguage{ngerman}
%
% \changes{v2.95}{2002/06/25}{%
%   erste Version aus der Aufteilung von \texttt{scrclass.dtx}}
%
% \section{Schriftauswahl}
%
% Dieser Bereich befasst sich mit allem, was zur Schriftauswahl
% gehört. Die einzelnen Elemente, für die eine Schriftauswahl möglich
% ist, sind jedoch in den Dateien definiert, in denen diese Elemente
% auftreten.
%
% \StopEventually{\PrintIndex\PrintChanges}
%
% \iffalse
%<*option>
% \fi
%
% \subsection{Der Generator für Schriftgrößendateien}
%
% \iffalse
%<*generator>
% \fi
% Ab Version 3.09 gibt es einen Generator für neue
% Schriftgrößendateien. Dieser richtet sich vor allem an Klassenautoren und
% bietet die Möglichkeit, eine eigene berechnete Schriftgrößendatei als
% Grundlage für eine spezifisch angepasste Schriftgrößendateie zu erstellen.
% Optionen besitzt dieser jedoch nicht. Daher reduziert sich der Optionenteil
% auf ein einfaches
%    \begin{macrocode}
\ProcessOptions\relax
%    \end{macrocode}
% \iffalse
%</generator>
% \fi
%
% \subsection{Option zur Auswahl der Größe der Grundschrift}
%
% \iffalse
%<*class|extend>
% \fi
% \begin{option}{10pt}
% \begin{option}{11pt}
% \begin{option}{12pt}
% Diese Optionen existieren nicht mehr als explizite Optionen, sondern
% werden in \texttt{scrtarea.dtx} über \cs{DeclareOption*}
% ausgewertet.
% \begin{macro}{\@ptsize}
% \changes{v2.96}{2006/08/11}{\cs{@ptsize} hat temporär eine andere
%   Bedeutung}%^^A
% \changes{v3.27}{2019/03/25}{\cs{providecommand} statt \cs{newcommand}}%^^A
% Wir definieren \cs{@ptsize} als Grundschriftgrößee minus 10\,pt.
% Hier wird die Standardeinstellung 11\,pt für alle Klassen außer der
% Briefklasse gewählt. Bis zum Einstellen tatsächlichen Einstellen der
% Schriftgröße wird in \cs{@ptsize} allerdings die tatsächliche Größe
% angegeben, damit sich Rundungsfehler nicht so leicht summieren.
%    \begin{macrocode}
%<*!extend>
\providecommand*\@ptsize{%
%<!letter>  11%
%<letter>  12%
}
%</!extend>
%    \end{macrocode}
% \end{macro}
% \end{option}
% \end{option}
% \end{option}
%
% \begin{macro}{\@pt@scan}
% \changes{v2.6}{2000/01/04}{neu (intern)}%^^A
% \begin{macro}{\@pt@@scan}
% \changes{v2.6}{2000/01/04}{neu (intern)}%^^A
% \changes{v2.97c}{2007/05/12}{\cs{KOMA@UseObsolete} durch
%     \cs{KOMA@UseObsoleteOption} ersetzt}
% \changes{v2.98c}{2008/03/22}{Verwendung von \cs{KOMA@UseObsoleteOption}%^^A
%     korrigiert}
% \changes{v2.97d}{2007/10/03}{\cs{PackageInfo} durch
%     \cs{PackageInfoNoLine} ersetzt}
% \changes{v3.12}{2013/03/04}{Alle Schriftgrößen-Optionen außer
%     \texttt{10pt}, \texttt{11pt} und \texttt{12pt} sind überholt.}
% \changes{v3.28}{2019/11/18}{\cs{ifstr} umbenannt in \cs{Ifstr}}%^^A
% \changes{v3.28}{2019/11/18}{\cs{ifnumber} umbenannt in \cs{Ifnumber}}%^^A
% Das Macro \cs{@pt@scan} wird benötigt, um die Schriftgröße nach der alten
% Methode zu erkennen.
%    \begin{macrocode}
%<*!extend>
\newcommand*{\@pt@scan}{%
  \expandafter\@pt@@scan\CurrentOption pt\@pt@@scan%
}
\newcommand*{\@pt@@scan}{}
\def\@pt@@scan #1pt#2\@pt@@scan{%
  \Ifstr{#2}{pt}{%
    \Ifnumber{#1}{%
      \Ifstr{#1}{10}{%
        \KOMA@UseStandardOption{\PackageInfoNoLine{\KOMAClassName}}%
                               {#1#2}{fontsize=#1#2}%
      }{%
        \Ifstr{#1}{11}{%
          \KOMA@UseStandardOption{\PackageInfoNoLine{\KOMAClassName}}%
                                 {#1#2}{fontsize=#1#2}%
        }{%
          \Ifstr{#1}{12}{%
            \KOMA@UseStandardOption{\PackageInfoNoLine{\KOMAClassName}}%
                                   {#1#2}{fontsize=#1#2}%
          }{%
            \KOMA@UseDeprecatedOption{\PackageWarningNoLine{\KOMAClassName}}%
                                     {#1#2}{fontsize=#1#2}%
          }%
        }%
      }
    }{\@headlines}%
  }{\@headlines}%
}
%</!extend>
%    \end{macrocode}
% \end{macro}
% \end{macro}
%
% \begin{macro}{\@fontsizefilebase}
% \changes{v2.96}{2006/08/11}{neu (intern)}%^^A
% \changes{v3.00}{2008/05/01}{Definition für \textsf{scrextend} vorgezogen}
% Dieses Makro speichert den Präfix der primären
% Schriftgrößenoptionsdateien. Es wird nur definiert, wenn es nicht bereits
% definiert ist. Damit können Wrapperklassen komplett andere Größen vorgeben,
% ohne mit \cs{ReplaceInput} arbeiten zu müssen.
%    \begin{macrocode}
\providecommand*{\@fontsizefilebase}{scrsize}
%    \end{macrocode}
% \end{macro}
%
% \changes{v3.17}{2015/03/10}{die Schriftgröße wird in der internen
%   Optionenliste gespeichert}%^^A
% \begin{option}{fontsize}
% \changes{v2.96}{2006/08/11}{Funktion komplett geändert}%^^A
% \changes{v2.98c}{2008/03/26}{lädt nach Möglichkeit die zugehörige
%     Schriftgrößendatei}%^^A
% \changes{v3.12}{2013/03/05}{Status wird mit \cs{FamilyKeyStateProcessed}%^^A
%     gesetzt.}
% Hier nun die Option, die tatsächlich verwendet wird. Es sei darauf
% hingewiesen, dass damit auch Schriftgrößen wie \texttt{10.3pt} denkbar
% wären.
%    \begin{macrocode}
\KOMA@key{fontsize}{%
%<*!extend>
  \scr@ifundefinedorrelax{changefontsizes}{%
    \@defaultunits\@tempdima#1pt\relax\@nnil
    \edef\@ptsize{#1}%
  }{%
%</!extend>
    \expandafter\@defaultunits\expandafter\@tempdima#1 pt\relax\@nnil
    \edef\@tempa{#1}%
    \setlength{\@tempdimb}{\@tempdima}%
    \edef\@tempb{\strip@pt\@tempdimb}%
    \addtolength{\@tempdimb}{-10\p@}%
    \edef\@ptsize{\strip@pt\@tempdimb}%
    \edef\@tempa{%
      \noexpand\makeatletter
      \noexpand\InputIfFileExists{\@fontsizefilebase\@tempa.clo}{%
%<!extend>        \noexpand\ClassInfo{\KOMAClassName}{%
%<extend>        \noexpand\PackageInfo{scrextend}{%
          File `\@fontsizefilebase\@tempa.clo' used to setup font sizes}%
      }{%
        \noexpand\InputIfFileExists{\@fontsizefilebase\@tempb pt.clo}{%
%<!extend>          \noexpand\ClassInfo{\KOMAClassName}{%
%<extend>          \noexpand\PackageInfo{scrextend}{%
            File `\@fontsizefilebase\@tempb pt.clo' used instead of%
            \noexpand\MessageBreak
            file `\@fontsizefilebase\@tempa.clo' to setup font sizes}%
        }{%
          \noexpand\changefontsizes{#1}%
        }%
      }%
      \noexpand\catcode`\noexpand\@=\the\catcode`\@
    }%
%<extend>  \scr@ifundefinedorrelax{changefontsizes}{%
%<extend>    \expandafter\AtEndOfPackage\expandafter{\@tempa}%
%<extend>  }{%
    \@tempa
  }%
  \FamilyKeyStateProcessed
  \KOMA@kav@xreplacevalue{.%
%<class>    \KOMAClassFileName
%<extend>    scrextend.\scr@pkgextension
  }{fontsize}{#1}%
}
%    \end{macrocode}
% \end{option}
%
% \iffalse
%</class|extend>
% \fi
%
% \iffalse
%</option>
%<*body>
% \fi
%
% \subsection{Sicherstellen der Umgebung des Generators}
%
% \iffalse
%<*generator>
% \fi
% Der Generator benötigt die Anweisung \cs{changefontsizes}, die entweder von
% einer \KOMAScript-Klasse oder vom Paket \texttt{scrextend} bereitgestellt
% wird:
%    \begin{macrocode}
\@ifundefined{changefontsizes}{\RequirePackage{scrextend}}{}
%    \end{macrocode}
% \iffalse
%</generator>
% \fi
%
% \subsection{Einlesen der Schriftgrößendatei}
%
% \changes{v2.6}{2000/01/04}{Einlesen der Schriftgrößen-Options-Datei
%   funktioniert nun mit diversen Größen}%^^A
% \changes{v2.98c}{2008/03/26}{Eigene \texttt{scrsize}-Dateien definiert}
%
% \iffalse
%</body>
%<*10pt|11pt|12pt>
% \fi
%
% Es werden min. drei Schriftgrößendateien benötigt. Ab Version 2.98c
% verwendet \KOMAScript{} vorzugsweise eigene Dateien, die auch noch zu einem
% späteren Zeitpunkt geladen werden können.  Ansonsten sind diese Dateien
% weitestgehend dem Quellcode der Standardklassen \texttt{classes.dtx}
% entnommen.
%
%    \begin{macrocode}
\ProvidesFile{%
%<10pt>  scrsize10pt.clo%
%<11pt>  scrsize11pt.clo%
%<12pt>  scrsize12pt.clo%
}[\KOMAScriptVersion\space font size class option %
%<10pt>  (10pt)%
%<11pt>  (11pt)%
%<12pt>  (12pt)%
]
%    \end{macrocode}
%
% \begin{macro}{\normalsize}
% \changes{v2.98c}{2008/03/26}{in eigener \texttt{scrsize}-Datei definiert}
% \begin{macro}{\small}
% \changes{v2.98c}{2008/03/26}{in eigener \texttt{scrsize}-Datei definiert}
% \begin{macro}{\footnotesize}
% \changes{v2.98c}{2008/03/26}{in eigener \texttt{scrsize}-Datei definiert}
% \begin{macro}{\scriptsize}
% \changes{v2.98c}{2008/03/26}{in eigener \texttt{scrsize}-Datei definiert}
% \begin{macro}{\tiny}
% \changes{v2.98c}{2008/03/26}{in eigener \texttt{scrsize}-Datei definiert}
% \begin{macro}{\large}
% \changes{v2.98c}{2008/03/26}{in eigener \texttt{scrsize}-Datei definiert}
% \begin{macro}{\Large}
% \changes{v2.98c}{2008/03/26}{in eigener \texttt{scrsize}-Datei definiert}
% \begin{macro}{\LARGE}
% \changes{v2.98c}{2008/03/26}{in eigener \texttt{scrsize}-Datei definiert}
% \begin{macro}{\huge}
% \changes{v2.98c}{2008/03/26}{in eigener \texttt{scrsize}-Datei definiert}
% \begin{macro}{\Huge}
% \changes{v2.98c}{2008/03/26}{in eigener \texttt{scrsize}-Datei definiert}
% Siehe \texttt{classes.dtx}.
%    \begin{macrocode}
\def\normalsize{%
%<*10pt>
  \@setfontsize\normalsize\@xpt\@xiipt
  \abovedisplayskip 10\p@ \@plus2\p@ \@minus5\p@
  \abovedisplayshortskip \z@ \@plus3\p@
  \belowdisplayshortskip 6\p@ \@plus3\p@ \@minus3\p@
%</10pt>
%<*11pt>
  \@setfontsize\normalsize\@xipt{13.6}%
  \abovedisplayskip 11\p@ \@plus3\p@ \@minus6\p@
  \abovedisplayshortskip \z@ \@plus3\p@
  \belowdisplayshortskip 6.5\p@ \@plus3.5\p@ \@minus3\p@
%</11pt>
%<*12pt>
  \@setfontsize\normalsize\@xiipt{14.5}%
  \abovedisplayskip 12\p@ \@plus3\p@ \@minus7\p@
  \abovedisplayshortskip \z@ \@plus3\p@
  \belowdisplayshortskip 6.5\p@ \@plus3.5\p@ \@minus3\p@
%</12pt>
  \belowdisplayskip \abovedisplayskip
  \let\@listi\@listI
}
\def\small{%
%<*10pt>
  \@setfontsize\small\@ixpt{11}%
  \abovedisplayskip 8.5\p@ \@plus3\p@ \@minus4\p@
  \abovedisplayshortskip \z@ \@plus2\p@
  \belowdisplayshortskip 4\p@ \@plus2\p@ \@minus2\p@
  \def\@listi{\leftmargin\leftmargini
    \topsep 4\p@ \@plus2\p@ \@minus2\p@
    \parsep 2\p@ \@plus\p@ \@minus\p@
    \itemsep \parsep}%
%</10pt>
%<*11pt>
  \@setfontsize\small\@xpt\@xiipt
  \abovedisplayskip 10\p@ \@plus2\p@ \@minus5\p@
  \abovedisplayshortskip \z@ \@plus3\p@
  \belowdisplayshortskip 6\p@ \@plus3\p@ \@minus3\p@
  \def\@listi{\leftmargin\leftmargini
    \topsep 6\p@ \@plus2\p@ \@minus2\p@
    \parsep 3\p@ \@plus2\p@ \@minus\p@
    \itemsep \parsep}%
%</11pt>
%<*12pt>
  \@setfontsize\small\@xipt{13.6}%
  \abovedisplayskip 11\p@ \@plus3\p@ \@minus6\p@
  \abovedisplayshortskip \z@ \@plus3\p@
  \belowdisplayshortskip 6.5\p@ \@plus3.5\p@ \@minus3\p@
  \def\@listi{\leftmargin\leftmargini
    \topsep 9\p@ \@plus3\p@ \@minus5\p@
    \parsep 4.5\p@ \@plus2\p@ \@minus\p@
    \itemsep \parsep}%
%</12pt>
  \belowdisplayskip \abovedisplayskip
}
\def\footnotesize{%
%<*10pt>
  \@setfontsize\footnotesize\@viiipt{9.5}%
  \abovedisplayskip 6\p@ \@plus2\p@ \@minus4\p@
  \abovedisplayshortskip \z@ \@plus\p@
  \belowdisplayshortskip 3\p@ \@plus\p@ \@minus2\p@
  \def\@listi{\leftmargin\leftmargini
    \topsep 3\p@ \@plus\p@ \@minus\p@
    \parsep 2\p@ \@plus\p@ \@minus\p@
    \itemsep \parsep}%
%</10pt>
%<*11pt>
  \@setfontsize\footnotesize\@ixpt{11}%
  \abovedisplayskip 8\p@ \@plus2\p@ \@minus4\p@
  \abovedisplayshortskip \z@ \@plus\p@
  \belowdisplayshortskip 4\p@ \@plus2\p@ \@minus2\p@
  \def\@listi{\leftmargin\leftmargini
    \topsep 4\p@ \@plus2\p@ \@minus2\p@
    \parsep 2\p@ \@plus\p@ \@minus\p@
    \itemsep \parsep}%
%</11pt>
%<*12pt>
   \@setfontsize\footnotesize\@xpt\@xiipt
   \abovedisplayskip 10\p@ \@plus2\p@ \@minus5\p@
   \abovedisplayshortskip \z@ \@plus3\p@
   \belowdisplayshortskip 6\p@ \@plus3\p@ \@minus3\p@
   \def\@listi{\leftmargin\leftmargini
               \topsep 6\p@ \@plus2\p@ \@minus2\p@
               \parsep 3\p@ \@plus2\p@ \@minus\p@
               \itemsep \parsep}%
%</12pt>
  \belowdisplayskip \abovedisplayskip
}
%<*10pt>
\def\scriptsize{\@setfontsize\scriptsize\@viipt\@viiipt}
\def\tiny{\@setfontsize\tiny\@vpt\@vipt}
\def\large{\@setfontsize\large\@xiipt{14}}
\def\Large{\@setfontsize\Large\@xivpt{18}}
\def\LARGE{\@setfontsize\LARGE\@xviipt{22}}
\def\huge{\@setfontsize\huge\@xxpt{25}}
\def\Huge{\@setfontsize\Huge\@xxvpt{30}}
%</10pt>
%<*11pt>
\def\scriptsize{\@setfontsize\scriptsize\@viiipt{9.5}}
\def\tiny{\@setfontsize\tiny\@vipt\@viipt}
\def\large{\@setfontsize\large\@xiipt{14}}
\def\Large{\@setfontsize\Large\@xivpt{18}}
\def\LARGE{\@setfontsize\LARGE\@xviipt{22}}
\def\huge{\@setfontsize\huge\@xxpt{25}}
\def\Huge{\@setfontsize\Huge\@xxvpt{30}}
%</11pt>
%<*12pt>
\def\scriptsize{\@setfontsize\scriptsize\@viiipt{9.5}}
\def\tiny{\@setfontsize\tiny\@vipt\@viipt}
\def\large{\@setfontsize\large\@xivpt{18}}
\def\Large{\@setfontsize\Large\@xviipt{22}}
\def\LARGE{\@setfontsize\LARGE\@xxpt{25}}
\def\huge{\@setfontsize\huge\@xxvpt{30}}
\let\Huge=\huge
%</12pt>
\normalsize
%    \end{macrocode}
% \begin{Length}{\footnotesep}
% \changes{v2.98c}{2008/03/26}{in eigener \texttt{scrsize}-Datei definiert}
% \begin{Length}{\footins}
% \changes{v2.98c}{2008/03/26}{in eigener \texttt{scrsize}-Datei definiert}
% \begin{Length}{\floatsep}
% \changes{v2.98c}{2008/03/26}{in eigener \texttt{scrsize}-Datei definiert}
% \begin{Length}{\textfloatsep}
% \changes{v2.98c}{2008/03/26}{in eigener \texttt{scrsize}-Datei definiert}
% \begin{Length}{\intextsep}
% \changes{v2.98c}{2008/03/26}{in eigener \texttt{scrsize}-Datei definiert}
% \begin{Length}{\dblfloatsep}
% \changes{v2.98c}{2008/03/26}{in eigener \texttt{scrsize}-Datei definiert}
% \begin{Length}{\sbltextfloatsep}
% \changes{v2.98c}{2008/03/26}{in eigener \texttt{scrsize}-Datei definiert}
% \begin{Length}{\@fptop}
% \changes{v2.98c}{2008/03/26}{in eigener \texttt{scrsize}-Datei definiert}
% \begin{Length}{\@fpsep}
% \changes{v2.98c}{2008/03/26}{in eigener \texttt{scrsize}-Datei definiert}
% \begin{Length}{\@fpbot}
% \changes{v2.98c}{2008/03/26}{in eigener \texttt{scrsize}-Datei definiert}
% \begin{Length}{\@dblfptop}
% \changes{v2.98c}{2008/03/26}{in eigener \texttt{scrsize}-Datei definiert}
% \begin{Length}{\@dblfpsep}
% \changes{v2.98c}{2008/03/26}{in eigener \texttt{scrsize}-Datei definiert}
% \begin{Length}{\@dblfpbot}
% \changes{v2.98c}{2008/03/26}{in eigener \texttt{scrsize}-Datei definiert}
% \begin{Length}{\partopsep}
% \changes{v2.98c}{2008/03/26}{in eigener \texttt{scrsize}-Datei definiert}
% Siehe \texttt{classes.dtx}.
%    \begin{macrocode}
%<*10pt>
\setlength\footnotesep    {6.65\p@}
\setlength{\skip\footins} {9\p@ \@plus 4\p@ \@minus 2\p@}
\setlength\floatsep       {12\p@ \@plus 2\p@ \@minus 2\p@}
\setlength\textfloatsep   {20\p@ \@plus 2\p@ \@minus 4\p@}
\setlength\intextsep      {12\p@ \@plus 2\p@ \@minus 2\p@}
\setlength\dblfloatsep    {12\p@ \@plus 2\p@ \@minus 2\p@}
\setlength\dbltextfloatsep{20\p@ \@plus 2\p@ \@minus 4\p@}
\setlength\@fptop         {0\p@ \@plus 1fil}
\setlength\@fpsep         {8\p@ \@plus 2fil}
\setlength\@fpbot         {0\p@ \@plus 1fil}
\setlength\@dblfptop      {0\p@ \@plus 1fil}
\setlength\@dblfpsep      {8\p@ \@plus 2fil}
\setlength\@dblfpbot      {0\p@ \@plus 1fil}
\setlength\partopsep      {2\p@ \@plus 1\p@ \@minus 1\p@}
%</10pt>
%<*11pt>
\setlength\footnotesep    {7.7\p@}
\setlength{\skip\footins} {10\p@ \@plus 4\p@ \@minus 2\p@}
\setlength\floatsep       {12\p@ \@plus 2\p@ \@minus 2\p@}
\setlength\textfloatsep   {20\p@ \@plus 2\p@ \@minus 4\p@}
\setlength\intextsep      {12\p@ \@plus 2\p@ \@minus 2\p@}
\setlength\dblfloatsep    {12\p@ \@plus 2\p@ \@minus 2\p@}
\setlength\dbltextfloatsep{20\p@ \@plus 2\p@ \@minus 4\p@}
\setlength\@fptop         {0\p@ \@plus 1fil}
\setlength\@fpsep         {8\p@ \@plus 2fil}
\setlength\@fpbot         {0\p@ \@plus 1fil}
\setlength\@dblfptop      {0\p@ \@plus 1fil}
\setlength\@dblfpsep      {8\p@ \@plus 2fil}
\setlength\@dblfpbot      {0\p@ \@plus 1fil}
\setlength\partopsep      {3\p@ \@plus 1\p@ \@minus 1\p@}
%</11pt>
%<*12pt>
\setlength\footnotesep    {8.4\p@}
\setlength{\skip\footins} {10.8\p@ \@plus 4\p@ \@minus 2\p@}
\setlength\floatsep       {12\p@ \@plus 2\p@ \@minus 4\p@}
\setlength\textfloatsep   {20\p@ \@plus 2\p@ \@minus 4\p@}
\setlength\intextsep      {14\p@ \@plus 4\p@ \@minus 4\p@}
\setlength\dblfloatsep    {14\p@ \@plus 2\p@ \@minus 4\p@}
\setlength\dbltextfloatsep{20\p@ \@plus 2\p@ \@minus 4\p@}
\setlength\@fptop         {0\p@ \@plus 1fil}
\setlength\@fpsep         {10\p@ \@plus 2fil}
\setlength\@fpbot         {0\p@ \@plus 1fil}
\setlength\@dblfptop      {0\p@ \@plus 1fil}
\setlength\@dblfpsep      {10\p@ \@plus 2fil}
\setlength\@dblfpbot      {0\p@ \@plus 1fil}
\setlength\partopsep      {3\p@ \@plus 2\p@ \@minus 2\p@}
%</12pt>
%    \end{macrocode}
% \begin{macro}{\@listi}
% \changes{v2.98c}{2008/03/26}{in eigener \texttt{scrsize}-Datei definiert}
% \begin{macro}{\@listii}
% \changes{v2.98c}{2008/03/26}{in eigener \texttt{scrsize}-Datei definiert}
% \begin{macro}{\@listiii}
% \changes{v2.98c}{2008/03/26}{in eigener \texttt{scrsize}-Datei definiert}
% \begin{macro}{\@listiv}
% \changes{v2.98c}{2008/03/26}{in eigener \texttt{scrsize}-Datei definiert}
% \begin{macro}{\@listv}
% \changes{v2.98c}{2008/03/26}{in eigener \texttt{scrsize}-Datei definiert}
% \begin{macro}{\@listvi}
% \changes{v2.98c}{2008/03/26}{in eigener \texttt{scrsize}-Datei definiert}
% Siehe \texttt{classes.dtx}.
%    \begin{macrocode}
\def\@listi{\leftmargin\leftmargini
%<*10pt>
            \parsep 4\p@ \@plus2\p@ \@minus\p@
            \topsep 8\p@ \@plus2\p@ \@minus4\p@
            \itemsep4\p@ \@plus2\p@ \@minus\p@}
%</10pt>
%<*11pt>
            \parsep 4.5\p@ \@plus2\p@ \@minus\p@
            \topsep 9\p@   \@plus3\p@ \@minus5\p@
            \itemsep4.5\p@ \@plus2\p@ \@minus\p@}
%</11pt>
%<*12pt>
            \parsep 5\p@  \@plus2.5\p@ \@minus\p@
            \topsep 10\p@ \@plus4\p@   \@minus6\p@
            \itemsep5\p@  \@plus2.5\p@ \@minus\p@}
%</12pt>
\let\@listI\@listi
\def\@listii {\leftmargin\leftmarginii
              \labelwidth\leftmarginii
              \advance\labelwidth-\labelsep
%<*10pt>
              \topsep    4\p@ \@plus2\p@ \@minus\p@
              \parsep    2\p@ \@plus\p@  \@minus\p@
%</10pt>
%<*11pt>
              \topsep    4.5\p@ \@plus2\p@ \@minus\p@
              \parsep    2\p@   \@plus\p@  \@minus\p@
%</11pt>
%<*12pt>
              \topsep    5\p@   \@plus2.5\p@ \@minus\p@
              \parsep    2.5\p@ \@plus\p@    \@minus\p@
%</12pt>
              \itemsep   \parsep}
\def\@listiii{\leftmargin\leftmarginiii
              \labelwidth\leftmarginiii
              \advance\labelwidth-\labelsep
%<10pt>              \topsep    2\p@ \@plus\p@\@minus\p@
%<11pt>              \topsep    2\p@ \@plus\p@\@minus\p@
%<12pt>              \topsep    2.5\p@\@plus\p@\@minus\p@
              \parsep    \z@
              \partopsep \p@ \@plus\z@ \@minus\p@
              \itemsep   \topsep}
\def\@listiv {\leftmargin\leftmarginiv
              \labelwidth\leftmarginiv
              \advance\labelwidth-\labelsep}
\def\@listv  {\leftmargin\leftmarginv
              \labelwidth\leftmarginv
              \advance\labelwidth-\labelsep}
\def\@listvi {\leftmargin\leftmarginvi
              \labelwidth\leftmarginvi
              \advance\labelwidth-\labelsep}
%    \end{macrocode}
% \iffalse
%</10pt|11pt|12pt>
%<*body>
% \fi
%
% \iffalse
%<*class|extend>
% \fi
%
% \begin{macro}{\changefontsizes}
% \changes{v2.96}{2006/08/11}{neue Anweisung (benötigt \eTeX)}%^^A
% \changes{v3.08b}{2011/02/22}{drei falsche \cs{def} durch \cs{edef}
%     ersetzt}%^^A
% \changes{v3.17}{2015/02/23}{wird nur definiert, wenn nicht vorhanden}%^^A
% \begin{macro}{\simple@changefontsizes}
% \changes{v3.17}{2015/02/23}{auf speziellen Wunsch von Falk}%^^A
% \changes{v3.17}{2015/03/25}{\cs{par@updaterelative} hinzugefügt}%^^A
% Diese Anweisung bietet berechnete Schriftgrößen als Fallbacklösung an. Das
% optionale erste Argument ist dabei der Grundlinienabstand. Das zweite
% Argument ist die gewünschte Grundschriftgröße.
% \begin{macro}{\scr@setlength}
% \changes{v3.12}{2013/10/08}{neue Anweisung (intern)}%^^A
% Setzt die Länge in |#1| auf den Wert |#2| plus |#3| minus |#4|, wobei
% allerdings für |#2|, |#3| und |#4| Mindestwerte von 1\,pt verwendet werden.
%    \begin{macrocode}
\newcommand*\scr@setlength[4]{%
  \expandafter\ifnum\scr@v@is@lt{3.12}\relax
    \setlength{#1}{#2 \@plus#3 \@minus#4}%
  \else
    \ifdim #2<\ifdim #2=\z@ \z@ \else \p@\fi
      \ifdim #3<\ifdim #3=\z@ \z@ \else \p@\fi
        \ifdim #4<\ifdim #4=\z@ \z@ \else \p@\fi
          \setlength{#1}{\p@ \@plus\p@ \@minus\p@}%
        \else
          \setlength{#1}{\p@ \@plus\p@ \@minus#4}%
        \fi
      \else
        \ifdim #4<\ifdim #4=\z@ \z@ \else \p@\fi
          \setlength{#1}{\p@ \@plus#3 \@minus\p@}%
        \else
          \setlength{#1}{\p@ \@plus#3 \@minus#4}%
        \fi
      \fi
    \else
      \ifdim #3<\ifdim #3=\z@ \z@ \else \p@\fi
        \ifdim #4<\ifdim #4=\z@ \z@ \else \p@\fi
          \setlength{#1}{#2 \@plus\p@ \@minus\p@}%
        \else
          \setlength{#1}{#2 \@plus\p@ \@minus#4}%
        \fi
      \else
        \ifdim #4<\ifdim #4=\z@ \z@ \else \p@\fi
          \setlength{#1}{#2 \@plus#3 \@minus\p@}%
        \else
          \setlength{#1}{#2 \@plus#3 \@minus#4}%
        \fi
      \fi
    \fi
  \fi
}
%    \end{macrocode}
% \end{macro}
% \changes{v3.17}{2015/03/10}{e-\TeX-Warnung entfernt}%^^A
%    \begin{macrocode}
\providecommand*{\simple@changefontsizes}[2][1.2\@tempdima]{%
  \KOMA@kav@removekey{.%
%<class>    \KOMAClassFileName
%<extend>    scrextend.\scr@pkgextension
  }{fontsize}%
  \@defaultunits\@tempdima#2pt\relax\@nnil
  \setlength{\@tempdimc}{\@tempdima}%
  \addtolength{\@tempdimc}{-10\p@}%
  \edef\@ptsize{\strip@pt\@tempdimc}%
  \@defaultunits\@tempdimb#1pt\relax\@nnil
  \setlength{\@tempdimc}{\dimexpr (100\@tempdimb / \@tempdima * \p@)}%
  \edef\@tempb{\the\@tempdimc}%
  \setlength{\@tempdimc}{\@tempdima}%
  \def\@tempa##1##2##3##4##5\@nnil{\def##1{##2.##3##4}}%
  \expandafter\@tempa\expandafter\@tempb\@tempb\@nnil%
%    \end{macrocode}
% Ab hier ist \cs{@tempdimc} die gewünschte Grundschriftgröße in pt und
% \cs{@tempb} der Faktor für den Grundlinienabstand auf 2 Stellen hinter dem
% Komma. Zunächst ist auch \cs{@tempdima} noch die geforderte
% Grundschriftgröße und \cs{@tempdimb} der geforderte
% Grundlinienabstand. Allerdings verändern sich \cs{@tempdima} und
% \cs{@tempdimb} im Laufe der nachfolgenden Definitionen abhängig von der zu
% definierenden Schriftgröße.
%
% Berechnungen für und Definition von \cs{normalsize}:
%    \begin{macrocode}
  \expandafter\ifnum\scr@v@is@lt{3.12}\relax
    \setlength{\abovedisplayskip}{%
      \@tempdima \@plus .25\@tempdima \@minus .58\@tempdima}%
    \setlength{\abovedisplayshortskip}{\z@ \@plus .25\@tempdima}%
    \setlength{\belowdisplayshortskip}{%
      .55\@tempdima \@plus .3\@tempdima \@minus .25\@tempdima}%
  \else
    \scr@setlength{\abovedisplayskip}%
                  {.8333\@tempdimb}{.1667\@tempdimb}{.5\@tempdimb}%
    \scr@setlength{\abovedisplayshortskip}{\z@}{.25\@tempdimb}{\z@}%
    \scr@setlength{\belowdisplayshortskip}%
                  {.5\@tempdimb}{.25\@tempdimb}{.25\@tempdimb}%
  \fi
  \setlength{\belowdisplayskip}{\abovedisplayskip}%
  \edef\normalsize{%
    \noexpand\@setfontsize\noexpand\normalsize
    {\the\@tempdima}{\the\@tempdimb}%
    \abovedisplayskip \the\abovedisplayskip
    \abovedisplayshortskip \the\abovedisplayshortskip
    \belowdisplayskip \the\belowdisplayskip
    \belowdisplayshortskip \the\belowdisplayshortskip
    \let\noexpand\@listi\noexpand\@listI
  }%
%    \end{macrocode}
% Berechnungen für und Definition von \cs{small}:
%    \begin{macrocode}
  \setlength{\@tempdima}{0.9125\@tempdimc}%
  \setlength{\@tempdimb}{\@tempb\@tempdima}%
  \expandafter\ifnum\scr@v@is@lt{3.12}\relax
    \setlength{\abovedisplayskip}{%
      \@tempdima \@plus .25\@tempdima \@minus .58\@tempdima}%
    \setlength{\abovedisplayshortskip}{\z@ \@plus .25\@tempdima}%
    \setlength{\belowdisplayshortskip}{%
      .55\@tempdima \@plus .3\@tempdima \@minus .25\@tempdima}%
  \else
    \scr@setlength{\abovedisplayskip}%
                  {.8333\@tempdimb}{.1667\@tempdimb}{.5\@tempdimb}%
    \scr@setlength{\abovedisplayshortskip}{\z@}{.25\@tempdimb}{\z@}%
    \scr@setlength{\belowdisplayshortskip}%
                  {.5\@tempdimb}{.25\@tempdimb}{.25\@tempdimb}%
  \fi
  \setlength{\belowdisplayskip}{\abovedisplayskip}%
  \edef\small{%
    \noexpand\@setfontsize\noexpand\small
    {\the\@tempdima}{\the\@tempdimb}%
    \abovedisplayskip \the\abovedisplayskip
    \abovedisplayshortskip \the\abovedisplayshortskip
    \belowdisplayskip \the\belowdisplayskip
    \belowdisplayshortskip \the\belowdisplayshortskip
    \let\noexpand\@listi\noexpand\@listi@small
  }%
%    \end{macrocode}
% Berechnungen für und Definition von \cs{footnotesize}:
%    \begin{macrocode}
  \setlength{\@tempdima}{.83334\@tempdimc}%
  \setlength{\@tempdimb}{\@tempb\@tempdima}%
  \expandafter\ifnum\scr@v@is@lt{3.12}\relax
    \setlength{\abovedisplayskip}{%
      \@tempdima \@plus .25\@tempdima \@minus .58\@tempdima}%
    \setlength{\abovedisplayshortskip}{\z@ \@plus .25\@tempdima}%
    \setlength{\belowdisplayshortskip}{%
      .55\@tempdima \@plus .3\@tempdima \@minus .25\@tempdima}%
  \else
    \scr@setlength{\abovedisplayskip}%
                  {.8333\@tempdimb}{.1667\@tempdimb}{.5\@tempdimb}%
    \scr@setlength{\abovedisplayshortskip}{\z@}{.25\@tempdimb}{\z@}%
    \scr@setlength{\belowdisplayshortskip}%
                  {.5\@tempdimb}{.25\@tempdimb}{.25\@tempdimb}%
  \fi
  \setlength{\belowdisplayskip}{\abovedisplayskip}%
  \edef\footnotesize{%
    \noexpand\@setfontsize\noexpand\footnotesize
    {\the\@tempdima}{\the\@tempdimb}%
    \abovedisplayskip \the\abovedisplayskip
    \abovedisplayshortskip \the\abovedisplayshortskip
    \belowdisplayskip \the\belowdisplayskip
    \belowdisplayshortskip \the\belowdisplayshortskip
    \let\noexpand\@listi\noexpand\@listi@footnotesize
  }%
%    \end{macrocode}
% Berechnungen für und Definition von \cs{scriptsize}:
%    \begin{macrocode}
  \setlength{\@tempdima}{.66667\@tempdimc}%
  \setlength{\@tempdimb}{\@tempb\@tempdima}%
  \edef\scriptsize{%
    \noexpand\@setfontsize\noexpand\scriptsize
    {\the\@tempdima}{\the\@tempdimb}%
  }%
%    \end{macrocode}
% Berechnungen für und Definition von \cs{tiny}:
%    \begin{macrocode}
  \setlength{\@tempdima}{.5\@tempdimc}%
  \setlength{\@tempdimb}{\@tempb\@tempdima}%
  \edef\tiny{%
    \noexpand\@setfontsize\noexpand\tiny
    {\the\@tempdima}{\the\@tempdimb}%
  }%
%    \end{macrocode}
% Berechnungen für und Definition von \cs{large}:
%    \begin{macrocode}
  \setlength{\@tempdima}{1.2\@tempdimc}%
  \setlength{\@tempdimb}{\@tempb\@tempdima}%
  \edef\large{%
    \noexpand\@setfontsize\noexpand\large
    {\the\@tempdima}{\the\@tempdimb}%
  }%
%    \end{macrocode}
% Berechnungen für und Definition von \cs{Large}:
%    \begin{macrocode}
  \setlength{\@tempdima}{1.44\@tempdimc}%
  \setlength{\@tempdimb}{\@tempb\@tempdima}%
  \edef\Large{%
    \noexpand\@setfontsize\noexpand\Large
    {\the\@tempdima}{\the\@tempdimb}%
  }%
%    \end{macrocode}
% Berechnungen für und Definition von \cs{LARGE}:
%    \begin{macrocode}
  \setlength{\@tempdima}{1.728\@tempdimc}%
  \setlength{\@tempdimb}{\@tempb\@tempdima}%
  \edef\LARGE{%
    \noexpand\@setfontsize\noexpand\LARGE
    {\the\@tempdima}{\the\@tempdimb}%
  }%
%    \end{macrocode}
% Berechnungen für und Definition von \cs{huge}:
%    \begin{macrocode}
  \setlength{\@tempdima}{2.074\@tempdimc}%
  \setlength{\@tempdimb}{\@tempb\@tempdima}%
  \edef\huge{%
    \noexpand\@setfontsize\noexpand\huge
    {\the\@tempdima}{\the\@tempdimb}%
  }%
%    \end{macrocode}
% Berechnungen für und Definition von \cs{Huge}:
%    \begin{macrocode}
  \setlength{\@tempdima}{2.488\@tempdimc}%
  \setlength{\@tempdimb}{\@tempb\@tempdima}%
  \edef\Huge{%
    \noexpand\@setfontsize\noexpand\Huge
    {\the\@tempdima}{\the\@tempdimb}%
  }%
%    \end{macrocode}
% Wechsel zu \cs{normalsize} und Berechnung weiterer schriftabhängigen
% Abstände für diese Schriftgröße.
%    \begin{macrocode}
  \normalsize
%<!extend>  \expandafter\ifnum\scr@v@is@ge{3.17}\@nameuse{par@updaterelative}\fi
%    \end{macrocode}
% Ab hier kann nun \cs{f@size} für die Schriftgröße und \cs{f@baselineskip}
% für den normalen Grundlinienabstand (ohne \cs{baselinestretch}) verwendet
% werden. Allerdings ist das keine Länge!
%    \begin{macrocode}
  \expandafter\ifnum\scr@v@is@lt{3.12}\relax
    \setlength{\footnotesep}{.7\@tempdimc}%
    \setlength{\skip\footins}{.9\@tempdimc \@plus .3333\@tempdimc \@minus
      .6\@tempdimc}%
    \setlength{\floatsep}{\@tempdimc \@plus .1667\@tempdimc \@minus
      .3333\@tempdimc}%
    \setlength{\textfloatsep}{1.6667\@tempdimc \@plus .1667\@tempdimc \@minus
      .3333\@tempdimc}%
    \setlength{\intextsep}{\@tempb\@tempdimc \@plus .3333\@tempdimc \@minus
      .3333\@tempdimc}%
    \setlength{\dblfloatsep}{\@tempb\@tempdimc \@plus .1667\@tempdimc \@minus
      .3333\@tempdimc}%
    \setlength{\dbltextfloatsep}{\textfloatsep}%
    \setlength{\@fptop}{0\p@ \@plus 1fil}%
    \setlength{\@fpsep}{.8333\@tempdimc \@plus 2fil}%
    \setlength{\@fpbot}{\@fptop}%
    \setlength{\@dblfptop}{0\p@ \@plus 1fil}%
    \setlength{\@dblfpsep}{.8333\@tempdimc \@plus 2fil}%
    \setlength{\@dblfpbot}{\@fptop}%
  \fi
%    \end{macrocode}
% Berechnungen für und Definition von \cs{@listi} und \cs{@listI}:
%    \begin{macrocode}
  \expandafter\ifnum\scr@v@is@lt{3.12}\relax
    \setlength{\topsep}{.8333\@tempdimc \@plus .3333\@tempdimc \@minus
      .5\@tempdimc}%
    \setlength{\parsep}{.4167\@tempdimc \@plus .2083\@tempdimc \@minus \p@}%
  \else
    \setlength{\@tempdimb}{\f@baselineskip}%
    \scr@setlength{\parsep}%
                  {.3333\@tempdimb}{.1667\@tempdimb}{.0833\@tempdimb}%
    \scr@setlength{\topsep}%
                  {.6667\@tempdimb}{.1667\@tempdimb}{.3333\@tempdimb}%
  \fi
  \@tempswafalse
  \begingroup
    \def\@list@extra{\aftergroup\@tempswatrue}%
    \csname @listi\endcsname
  \endgroup
  \edef\@listi{\leftmargin\leftmargini
    \topsep \the\topsep
    \parsep \the\parsep
    \itemsep \parsep
    \if@tempswa\noexpand\@list@extra\fi
  }%
  \let\@listI\@listi
%    \end{macrocode}
% Berechnungen für und Definition von \cs{@listi@small}:
%    \begin{macrocode}
  \expandafter\ifnum\scr@v@is@lt{3.12}\relax
    \setlength{\topsep}{.75\@tempdimc \@plus .25\@tempdimc \@minus
      .41667\@tempdimc}%
    \setlength{\parsep}{.375\@tempdimc \@plus .16667\@tempdimc \@minus \p@}%
  \else
    \scr@setlength{\parsep}%
                  {.1667\@tempdimb}{.0833\@tempdimb}{.0833\@tempdimb}%
    \scr@setlength{\topsep}%
                  {.3333\@tempdimb}{.1667\@tempdimb}{.1667\@tempdimb}%
  \fi
  \@tempswafalse
  \begingroup
    \def\@list@extra{\aftergroup\@tempswatrue}%
    \csname @listi\endcsname
  \endgroup
  \edef\@listi@small{\leftmargin\leftmargini
    \topsep \the\topsep
    \parsep \the\parsep
    \itemsep \parsep
    \if@tempswa\noexpand\@list@extra\fi
  }%
%    \end{macrocode}
% Berechnungen für und Definition von \cs{@listi@footnotesize}:
%    \begin{macrocode}
  \expandafter\ifnum\scr@v@is@lt{3.12}\relax
    \setlength{\topsep}{.5\@tempdimc \@plus .16667\@tempdimc \@minus
      .16667\@tempdimc}%
    \setlength{\parsep}{.25\@tempdimc \@plus .16667\@tempdimc \@minus \p@}%
  \else
    \scr@setlength{\parsep}%
                  {.125\@tempdimb}{.0625\@tempdimb}{.0625\@tempdimb}%
    \scr@setlength{\topsep}%
                  {.25\@tempdimb}{.125\@tempdimb}{.125\@tempdimb}%
  \fi
  \@tempswafalse
  \begingroup
    \def\@list@extra{\aftergroup\@tempswatrue}%
    \csname @listi\endcsname
  \endgroup
  \edef\@listi@footnotesize{\leftmargin\leftmargini
    \topsep \the\topsep
    \parsep \the\parsep
    \itemsep \parsep
    \if@tempswa\noexpand\@list@extra\fi
  }%
%    \end{macrocode}
% Berechnungen für und Definition von \cs{@listii}:
%    \begin{macrocode}
  \expandafter\ifnum\scr@v@is@lt{3.12}\relax
    \setlength{\topsep}{.4167\@tempdimc \@plus .2083\@tempdimc \@minus \p@}%
    \setlength{\parsep}{.2083\@tempdimc \@plus \p@ \@minus \p@}%
  \else
    \scr@setlength{\parsep}%
                  {.1667\@tempdimb}{.0833\@tempdimb}{.0833\@tempdimb}%
    \scr@setlength{\topsep}%
                  {.3333\@tempdimb}{.1667\@tempdimb}{.0833\@tempdimb}%
  \fi
  \@tempswafalse
  \begingroup
    \def\@list@extra{\aftergroup\@tempswatrue}%
    \csname @listii\endcsname
  \endgroup
  \edef\@listii{\leftmargin\leftmarginii
    \labelwidth=\dimexpr \leftmargin-\labelsep
    \topsep \the\topsep
    \parsep \the\parsep
    \itemsep \parsep
    \if@tempswa\noexpand\@list@extra\fi
  }%
%    \end{macrocode}
% Berechnungen für und Definition von \cs{@listiii}:
%    \begin{macrocode}
  \expandafter\ifnum\scr@v@is@lt{3.12}\relax
    \setlength{\topsep}{.2083\@tempdimc \@plus \p@ \@minus \p@}%
  \else
    \scr@setlength{\topsep}%
                  {.1667\@tempdimb}{.0833\@tempdimb}{.0833\@tempdimb}%
  \fi
  \setlength{\partopsep}{\z@ \@plus\z@ \@minus\p@}%
  \@tempswafalse
  \begingroup
    \def\@list@extra{\aftergroup\@tempswatrue}%
    \csname @listiii\endcsname
  \endgroup
  \edef\@listiii{\leftmargin\leftmarginiii
    \labelwidth=\dimexpr \leftmargin-\labelsep
    \topsep \the\topsep
    \parsep \z@
    \partopsep \the\partopsep
    \itemsep \topsep
    \if@tempswa\noexpand\@list@extra\fi
  }%
%    \end{macrocode}
% Berechnungen für und Definition von \cs{@listiv}:
%    \begin{macrocode}
  \@tempswafalse
  \begingroup
    \def\@list@extra{\aftergroup\@tempswatrue}%
    \csname @listiv\endcsname
  \endgroup
  \edef\@listiv{\leftmargin\leftmarginiv
    \labelwidth=\dimexpr \leftmargin-\labelsep\relax
    \if@tempswa\noexpand\@list@extra\fi
  }%
%    \end{macrocode}
% Berechnungen für und Definition von \cs{@listv}:
%    \begin{macrocode}
  \@tempswafalse
  \begingroup
    \def\@list@extra{\aftergroup\@tempswatrue}%
    \csname @listv\endcsname
  \endgroup
  \edef\@listv{\leftmargin\leftmarginv
    \labelwidth=\dimexpr \leftmargin-\labelsep\relax
    \if@tempswa\noexpand\@list@extra\fi
  }%
%    \end{macrocode}
% Berechnungen für und Definition von \cs{@listvi}:
%    \begin{macrocode}
  \@tempswafalse
  \begingroup
    \def\@list@extra{\aftergroup\@tempswatrue}%
    \csname @listvi\endcsname
  \endgroup
  \edef\@listvi{\leftmargin\leftmarginvi
    \labelwidth=\dimexpr \leftmargin-\labelsep\relax
    \if@tempswa\noexpand\@list@extra\fi
  }%
  \@listi
%    \end{macrocode}
% Einige Platzierungsabstände sind ebenfalls schriftgrößenabhängig. ^^A
% \changes{v2.97c}{2007/07/04}{\cs{intextsep} setzen}%^^A
% \changes{v2.97c}{2007/07/04}{\cs{dblfloatsep} setzen}%^^A
% \changes{v2.97c}{2007/07/04}{\cs{partopsep} setzen}%^^A
% \changes{v3.12}{2013/10/08}{Berechnung einiger Größen verändert}%^^A
% Ab Version~3.12 wird deren Berechnung korrigiert. Viele der Größen sind nun
% nicht mehr von der Schriftgröße, sondern dem Grundlinienabstand abhängig.
%    \begin{macrocode}
  \expandafter\ifnum\scr@v@is@lt{3.12}\relax
    \setlength{\@tempdimb}{\@tempb\@tempdima}%
    \setlength{\@tempdima}{\dimexpr \@tempdimb-\@tempdimc}%
    \setlength\intextsep{\@tempdimb \@plus.2\@tempdima \@minus.2\@tempdima}%
    \setlength\dblfloatsep\intextsep
    \setlength\partopsep{.2\@tempdimc \@plus.1\@tempdimc \@minus.1\@tempdimc}%
  \else
    \setlength{\@tempdimb}{\f@baselineskip}%
    \setlength{\footnotesep}{.555\@tempdimb}%
    \scr@setlength{\skip\footins}%
                  {.75\@tempdimb}{.3333\@tempdimb}{.1667\@tempdimb}%
    \scr@setlength{\floatsep}%
                  {\@tempdimb}{.1667\@tempdimb}{.1667\@tempdimb}%
    \scr@setlength{\textfloatsep}%
                  {1.6667\@tempdimb}{.1667\@tempdimb}{.3333\@tempdimb}%
    \setlength{\intextsep}{\floatsep}%
    \setlength{\dblfloatsep}{\floatsep}%
    \setlength{\dbltextfloatsep}{\textfloatsep}%
    \setlength{\@fptop}{\z@ \@plus 1fil}%
    \setlength{\@fpsep}{.6667\@tempdimb \@plus 2fil}%
    \setlength{\@fpbot}{\@fptop}%
    \setlength{\@dblfptop}{\@fptop}%
    \setlength{\@dblfpsep}{\@fpsep}%
    \setlength{\@dblfpbot}{\@fptop}%
    \scr@setlength{\partopsep}{.2\@tempdimb}{.1\@tempdimb}{.1\@tempdimb}%
%    \end{macrocode}
% Hinweis: \cs{topsep} und \cs{parsep} werden hier nicht gesetzt, weil dies
% bereits durch das \cs{@listi} zurvor erfolgte.
%    \begin{macrocode}
  \fi
}
\scr@ifundefinedorrelax{changefontsizes}{%
  \let\changefontsizes\simple@changefontsizes
}{%
%<class>  \ClassWarning{\KOMAClassName}{%
%<package&extend>  \PackageWarning{scrextend}{%
    \string\changefontsizes\space already defined.\MessageBreak
    I hope, the definition is compatible,\MessageBreak
    because I do not change it%
  }%
}
%    \end{macrocode}
% \end{macro}%^^A \simple@changefontsizes
% \end{macro}%^^A \changefontsizes
% \iffalse
%</class|extend>
%<*generator>
% \fi
% \begin{macro}{\generatefontfile}
% \changes{v3.09}{2011/02/23}{Neu}%^^A
% \changes{v3.17}{2015/03/26}{\cs{par@updaterelative} added}%^^A
% \changes{v3.28}{2019/11/18}{\cs{ifstr} umbenannt in \cs{Ifstr}}%^^A
% Dies ist die zentrale Anweisung des neuen Generators. Es wird eine neue
% Schriftgrößendatei ausgegeben. Der Name dieser Datei setzt sich aus dem
% ersten obligaorischen Argument gefolgt von der Schriftgröße und der
% Erweiterung \texttt{.clo} zusammen. Da Schriftgrößendateien, die mit
% "`\texttt{scrsize}"' beginnen, \KOMAScript{} vorbehalten sind wird dieses
% erste Argument verweigert.
%    \begin{macrocode}
\newcommand*{\generatefontfile}[1]{%
  \Ifstr{#1}{scrsize}{%
    \ClassError{\KOMAClassName}{%
      Font file name `scrsize' not allowed%
    }{%
      Font files with name `scrsize<fontsize>.clo' are reserved for
      KOMA-Script.\MessageBreak
      You should use another prefix and rename \string\@fontsizefilebase\space
      either before\MessageBreak
      loading a KOMA-Script class using one of \string\documentclass,
      \string\LoadClass,\MessageBreak
      or \string\LoadClassWithOptions\space or before loading package
      `scrextend' using another\MessageBreak
      class.}%
    \@@@generatefontfile
  }{%
%    \end{macrocode}
% \changes{v3.20}{2016/04/12}{\cs{@ifnextchar} replaced by
%     \cs{kernel@ifnextchar}}%^^A
% Das zweite Argument ist optional. Ist es angegeben, so gibt es den
% gewünschten Zeilenabstand an. Ist es nicht angegeben, so wird der
% Zeilenabstand |\changefontsized| überlassen. Das dritte Argument ist
% schließlich die gewünschte Schriftgröße.
%    \begin{macrocode}
    \kernel@ifnextchar [{\@generatefontfile{#1}}{\@@generatefontfile{#1}}%
  }%
}
%    \end{macrocode}
% \begin{macro}{\@generatefontfile}
% \changes{v3.09}{2011/02/23}{Neu (intern)}%^^A
% \begin{macro}{\@@generatefontfile}
% \changes{v3.09}{2011/02/23}{Neu (intern)}%^^A
% Diese beiden Hilfsmakros werden benötigt, um Aufrufe von |\generatefontfile|
% mit oder ohne optionales, zweites Argument zu behandeln. In jedem Fall wird
% die gewünschte Schriftgröße innerhalb einer Gruppe mit Hilfe von
% |\changefontsizes| bestimmt und dann eine Datei mit den dabei ermittelten
% Einstellungen ausgegeben.
%    \begin{macrocode}
\newcommand*{\@generatefontfile}{}
\def\@generatefontfile#1[#2]#3{%
  \begingroup
    \@ifundefined{setparsizes}{\setlength{\parskip}{0pt}}%
                              {\setparsizes{0pt}{0pt}{0pt}}%
    \changefontsizes[{#2}]{#3}%
    \generate@fontfile{#1}{#3}%
  \endgroup
}
\newcommand\@@generatefontfile[2]{%
  \begingroup
    \@ifundefined{setparsizes}{\setlength{\parskip}{0pt}}%
                              {\setparsizes{0pt}{0pt}{0pt}}%
    \changefontsizes{#2}%
    \generate@fontfile{#1}{#2}%
  \endgroup
}
%    \end{macrocode}
% \end{macro}
% \end{macro}
% \begin{macro}{\@@@generatorfontfile}
% \changes{v3.09}{2011/02/23}{Neu (intern)}%^^A
% Dieses Hilfsmakro wird hingegen im Fehlerfall aufgerufen, um die Ausgabe
% einer Schriftgrößendatei zu verhindern.
%    \begin{macrocode}
\newcommand*{\@@@generatefontfile}[2][]{}
%    \end{macrocode}
% \end{macro}
% \begin{macro}{\@fontfile}
% \changes{v3.09}{2011/02/23}{Neu (intern)}%^^A
% Da später die Verwendung von |\generatefontfile| noch auf die Präambel
% beschränkt werden wird, ist es kein Problem hier |\@mainaux| für die Ausgabe
% zu missbrauchen. Das funktioniert aber nur, weil |\@mainaux| erst innerhalb
% von |\begin{document}| tatsächlich geöffnet wird. Die Dateisparsamkeit an
% dieser Stelle ist also eigentlich ein ziemlich unsauberer Hack!
%    \begin{macrocode}
\newcommand*{\@fontfile}{}
\let\@fontfile\@mainaux
%    \end{macrocode}
% \begin{macro}{\generate@fontfile}
% \changes{v3.09}{2011/02/23}{Neu (intern)}%^^A
% \changes{v3.16a}{2015/02/11}{\cs{footnotesize} korrigiert}
% Es wird nun eine Schriftgrößendatei ausgegeben. Dabei werden die aktuellen
% Einstellungen verwendet. Dabei wird die Tatsache ausgenutzt, dass diese
% Anweisung immer innerhalb einer Gruppe aufgerufen wird. Das erste Argument
% ist der Dateinamenpräfix (also so etwas wie "`\texttt{scrsize}"', das zweite
% Argument die gewünschte Schriftgröße. Es ist zu beachten, dass alle Werte in
% der Datei in pt angegeben werden, auch wenn die gewünschte Schriftgröße eine
% ganz andere Einheit besitzt. Das macht schlicht die Arbeit etwas einfacher.
%    \begin{macrocode}
\newcommand*{\generate@fontfile}[2]{%
  \def\@ind{\space\space}%
  \immediate\openout\@fontfile #1#2.clo
  \immediate\write\@fontfile{\@percentchar^^J%
    \@percentchar\space This is file `#1#2.clo', generated with^^J%
    \@percentchar\space scrfontsizes
    \csname ver@scrfontsizes.\scr@pkgextension\endcsname,^^J%
    \@percentchar\space Copyright (c) Markus Kohm.^^J%
    \@percentchar^^J%
    \string\ProvidesFile{#1#2.clo}[%
    \the\year/\ifnum\month<10 0\fi\the\month/\ifnum\day<10 0\fi\the\day%
    \space generated font size class option (#2)]}%
  \immediate\write\@fontfile{%
    \string\def\string\normalsize{\@percentchar^^J%
      \@ind\string\@setfontsize\string\normalsize{\f@size}{\f@baselineskip}%
      \@percentchar^^J%
      \@ind\string\abovedisplayskip\space \the\abovedisplayskip^^J%
      \@ind\string\abovedisplayshortskip\space \the\abovedisplayshortskip^^J%
      \@ind\string\belowdisplayskip\space \the\belowdisplayskip^^J%
      \@ind\string\belowdisplayshortskip\space \the\belowdisplayshortskip^^J%
      \@ind\string\let\string\@listi\string\@listI^^J%
    }\@percentchar%
  }%
  \begingroup\small\@listi
    \immediate\write\@fontfile{%
      \string\def\string\small{\@percentchar^^J%
        \@ind\string\@setfontsize\string\small{\f@size}{\f@baselineskip}%
        \@percentchar^^J%
        \@ind\string\abovedisplayskip\space \the\abovedisplayskip^^J%
        \@ind\string\abovedisplayshortskip\space \the\abovedisplayshortskip^^J%
        \@ind\string\belowdisplayskip\space \the\belowdisplayskip^^J%
        \@ind\string\belowdisplayshortskip\space \the\belowdisplayshortskip^^J%
        \@ind\string\def\string\@listi{\@percentchar^^J%
          \@ind\@ind\string\leftmargin\string\leftmargini^^J%
          \@ind\@ind\string\topsep \the\topsep^^J%
          \@ind\@ind\string\parsep \the\parsep^^J%
          \@ind\@ind\string\itemsep \string\parsep^^J%
        \@ind}\@percentchar^^J%
      }\@percentchar%
    }%
  \endgroup
  \begingroup\footnotesize\@listi
    \immediate\write\@fontfile{%
      \string\def\string\footnotesize{\@percentchar^^J%
        \@ind\string\@setfontsize\string\footnotesize{\f@size}{\f@baselineskip}%
        \@percentchar^^J%
        \@ind\string\abovedisplayskip\space \the\abovedisplayskip^^J%
        \@ind\string\abovedisplayshortskip\space \the\abovedisplayshortskip^^J%
        \@ind\string\belowdisplayskip\space \the\belowdisplayskip^^J%
        \@ind\string\belowdisplayshortskip\space \the\belowdisplayshortskip^^J%
        \@ind\string\def\string\@listi{\@percentchar^^J%
          \@ind\@ind\string\leftmargin\string\leftmargini^^J%
          \@ind\@ind\string\topsep \the\topsep^^J%
          \@ind\@ind\string\parsep \the\parsep^^J%
          \@ind\@ind\string\itemsep \string\parsep^^J%
        \@ind}\@percentchar^^J%
      }\@percentchar%
    }%
  \endgroup
  \begingroup\scriptsize
    \immediate\write\@fontfile{%
      \string\def\string\scriptsize{%
        \string\@setfontsize\string\scriptsize{\f@size}{\f@baselineskip}%
      }\@percentchar%
    }%
  \endgroup
  \begingroup\tiny
    \immediate\write\@fontfile{%
      \string\def\string\tiny{%
        \string\@setfontsize\string\tiny{\f@size}{\f@baselineskip}%
      }\@percentchar%
    }%
  \endgroup
  \begingroup\large
    \immediate\write\@fontfile{%
      \string\def\string\large{%
        \string\@setfontsize\string\large{\f@size}{\f@baselineskip}%
      }\@percentchar%
    }%
  \endgroup
  \begingroup\Large
    \immediate\write\@fontfile{%
      \string\def\string\Large{%
        \string\@setfontsize\string\Large{\f@size}{\f@baselineskip}%
      }\@percentchar%
    }%
  \endgroup
  \begingroup\LARGE
    \immediate\write\@fontfile{%
      \string\def\string\LARGE{%
        \string\@setfontsize\string\LARGE{\f@size}{\f@baselineskip}%
      }\@percentchar%
    }%
  \endgroup
  \begingroup\huge
    \immediate\write\@fontfile{%
      \string\def\string\huge{%
        \string\@setfontsize\string\huge{\f@size}{\f@baselineskip}%
      }\@percentchar%
    }%
  \endgroup
  \begingroup\Huge
    \immediate\write\@fontfile{%
      \string\def\string\Huge{%
        \string\@setfontsize\string\Huge{\f@size}{\f@baselineskip}%
      }\@percentchar%
    }%
  \endgroup
  \immediate\write\@fontfile{\string\normalsize}%
  \immediate\write\@fontfile{%
    \string\setlength{\string\footnotesep}{\the\footnotesep}\@percentchar
  }%
  \immediate\write\@fontfile{%
    \string\setlength{\string\skip\string\footins}{\the\skip\footins}\@percentchar
  }%
  \immediate\write\@fontfile{%
    \string\setlength{\string\floatsep}{\the\floatsep}\@percentchar
  }%
  \immediate\write\@fontfile{%
    \string\setlength{\string\textfloatsep}{\the\textfloatsep}\@percentchar
  }%
  \immediate\write\@fontfile{%
    \string\setlength{\string\intextsep}{\the\intextsep}\@percentchar
  }%
  \immediate\write\@fontfile{%
    \string\setlength{\string\dblfloatsep}{\the\dblfloatsep}\@percentchar
  }%
  \immediate\write\@fontfile{%
    \string\setlength{\string\dbltextfloatsep}{\the\dbltextfloatsep}\@percentchar
  }%
  \immediate\write\@fontfile{%
    \string\setlength{\string\@fptop}{\the\@fptop}\@percentchar
  }%
  \immediate\write\@fontfile{%
    \string\setlength{\string\@fpsep}{\the\@fpsep}\@percentchar
  }%
  \immediate\write\@fontfile{%
    \string\setlength{\string\@fpbot}{\the\@fpbot}\@percentchar
  }%
  \immediate\write\@fontfile{%
    \string\setlength{\string\@dblfptop}{\the\@dblfptop}\@percentchar
  }%
  \immediate\write\@fontfile{%
    \string\setlength{\string\@dblfpsep}{\the\@dblfpsep}\@percentchar
  }%
  \immediate\write\@fontfile{%
    \string\setlength{\string\@dblfpbot}{\the\@dblfpbot}\@percentchar
  }%
  \immediate\write\@fontfile{%
    \string\setlength{\string\partopsep}{\the\partopsep}\@percentchar
  }%
  \begingroup
    \@listi
    \immediate\write\@fontfile{%
      \string\def\string\@listi{\@percentchar^^J%
        \@ind\string\leftmargin\string\leftmargini^^J%
        \@ind\string\topsep \the\topsep^^J%
        \@ind\string\parsep \the\parsep^^J%
        \@ind\string\itemsep \string\parsep^^J%
      }\@percentchar
    }%
  \endgroup
  \immediate\write\@fontfile{%
    \string\let\string\@listI\string\@listi
  }%
  \begingroup
    \@listii
    \immediate\write\@fontfile{%
      \string\def\string\@listii{\@percentchar^^J%
        \@ind\string\leftmargin\string\leftmarginii^^J%
        \@ind\string\labelwidth\string\leftmarginii^^J%
        \@ind\string\advance\string\labelwidth-\string\labelsep^^J%
        \@ind\string\topsep \the\topsep^^J%
        \@ind\string\parsep \the\parsep^^J%
        \@ind\string\itemsep \string\parsep^^J%
      }\@percentchar
    }%
  \endgroup
  \begingroup
    \@listiii
    \immediate\write\@fontfile{%
      \string\def\string\@listiii{\@percentchar^^J%
        \@ind\string\leftmargin\string\leftmarginiii^^J%
        \@ind\string\labelwidth\string\leftmarginiii^^J%
        \@ind\string\advance\string\labelwidth-\string\labelsep^^J%
        \@ind\string\topsep \the\topsep^^J%
        \@ind\string\parsep \the\parsep^^J%
        \@ind\string\partopsep \the\partopsep^^J%
       \@ind\string\itemsep \string\topsep^^J%
      }\@percentchar
    }%
  \endgroup
  \begingroup
    \@listiv
    \immediate\write\@fontfile{%
      \string\def\string\@listiv{\@percentchar^^J%
        \@ind\string\leftmargin\string\leftmarginiv^^J%
        \@ind\string\labelwidth\string\leftmarginiv^^J%
        \@ind\string\advance\string\labelwidth-\string\labelsep^^J%
      }\@percentchar
    }%
  \endgroup
  \begingroup
    \@listv
    \immediate\write\@fontfile{%
      \string\def\string\@listv{\@percentchar^^J%
        \@ind\string\leftmargin\string\leftmarginv^^J%
        \@ind\string\labelwidth\string\leftmarginv^^J%
        \@ind\string\advance\string\labelwidth-\string\labelsep^^J%
      }\@percentchar
    }%
  \endgroup
  \begingroup
    \@listvi
    \immediate\write\@fontfile{%
      \string\def\string\@listvi{\@percentchar^^J%
        \@ind\string\leftmargin\string\leftmarginvi^^J%
        \@ind\string\labelwidth\string\leftmarginvi^^J%
        \@ind\string\advance\string\labelwidth-\string\labelsep^^J%
      }\@percentchar
    }%
  \endgroup
  \immediate\write\@fontfile{%
    \string\@ifundefined{@list@extra}{}{\@percentchar^^J%
      \string\expandafter\string\ifnum\string\scr@v@is@ge{3.17}%
        \string\par@updaterelative\string\fi^^J%
      \@ind\string\l@addto@macro{\string\@listi}{\string\@list@extra}%
      \@percentchar^^J%
      \@ind\string\let\string\@listI=\string\@listi^^J%
      \@ind\string\l@addto@macro{\string\@listii}{\string\@list@extra}%
      \@percentchar^^J%
      \@ind\string\l@addto@macro{\string\@listiii}{\string\@list@extra}%
      \@percentchar^^J%
      \@ind\string\l@addto@macro{\string\footnotesize}{\string\protect
        \string\add@extra@listi{ftns}}\@percentchar^^J%
      \@ind\string\l@addto@macro{\string\small}{\string\protect
        \string\add@extra@listi{sml}}\@percentchar^^J%
    }\@percentchar^^J%
    \string\@listi^^J%
    \string\endinput^^J%
    \@percentchar^^J%
    \@percentchar\space End of file `#1#2.clo'.
  }
  \immediate\closeout\@fontfile
}
\@onlypreamble\generatefontfile
%    \end{macrocode}
% \end{macro}
% \end{macro}
% \end{macro}
% \iffalse
%</generator>
%<*class|extend>
% \fi
% \end{macro}
% \end{macro}
% \end{macro}
% \end{macro}
% \end{macro}
% \end{macro}
% \end{Length}
% \end{Length}
% \end{Length}
% \end{Length}
% \end{Length}
% \end{Length}
% \end{Length}
% \end{Length}
% \end{Length}
% \end{Length}
% \end{Length}
% \end{Length}
% \end{Length}
% \end{Length}
% \end{macro}
% \end{macro}
% \end{macro}
% \end{macro}
% \end{macro}
% \end{macro}
% \end{macro}
% \end{macro}
% \end{macro}
% \end{macro}
%
% Zunächst wird die gewünschte Größe, die in \cs{@ptsize} abgelegt ist, als
% Länge ermittelt. In \cs{@tempa} wird die ursprünglich gewünschte Größe
% abgelegt, damit zwischen \texttt{10pt}, \texttt{10} und \texttt{10dd}
% unterschieden werden kann. In \cs{@tempb} wird hingegen die Größe in pt
% abgelegt, damit die Standard-Dateien verwendet werden können. Dann wird
% \cs{@ptsize} schon einmal richtig eingestellt.
%    \begin{macrocode}
%<*!extend>
\expandafter\@defaultunits\expandafter\@tempdima\@ptsize pt\relax\@nnil
\edef\@tempa{\@ptsize}%
\setlength{\@tempdimb}{\@tempdima}%
\edef\@tempb{\strip@pt\@tempdimb}%
\addtolength{\@tempdimb}{-10\p@}%
\edef\@ptsize{\strip@pt\@tempdimb}%
%    \end{macrocode}
% Jetzt wird zunächst versucht, ob eine Datei
% "`\texttt{\cs{fontsizefilebase}\cs{@tempa}.clo}"' vorhanden ist, also eine,
% bei der die ursprünliche Einheit im Namen angegeben ist. Wenn das der Fall
% ist, wird diese verwendet. Wenn nicht wird das Ganze für \cs{@tempb} um die
% Einheit pt erweitert wiederholt.
%    \begin{macrocode}
\InputIfFileExists{\@fontsizefilebase\@tempa.clo}{%
  \ClassInfo{\KOMAClassName}{%
    File `\@fontsizefilebase\@tempa.clo' used to setup font sizes}%
  \KOMA@kav@removekey{.\KOMAClassFileName}{fontsize}%
  \KOMA@kav@xadd{.\KOMAClassFileName}{fontsize}{\@tempa}%
}{%
  \InputIfFileExists{\@fontsizefilebase\@tempb pt.clo}{%
    \ClassInfo{\KOMAClassName}{%
      File `\@fontsizefilebase\@tempb pt.clo' used instead of\MessageBreak
      file `\@fontsizefilebase\@tempa.clo' to setup font sizes}%
    \KOMA@kav@removekey{.\KOMAClassFileName}{fontsize}%
    \KOMA@kav@xadd{.\KOMAClassFileName}{fontsize}{\@tempb pt}%
  }{%
%    \end{macrocode}
% Wurde bis hier noch keine Datei geladen, wird nun versucht, die
% Größendateien der Standardklassen zu verwenden.
%    \begin{macrocode}
%<*book>
    \InputIfFileExists{bk\@tempb.clo}{%
      \ClassInfo{\KOMAClassName}{%
        File `bk\@tempb.clo' used to setup font sizes}%
      \KOMA@kav@removekey{.\KOMAClassFileName}{fontsize}%
      \KOMA@kav@xadd{.\KOMAClassFileName}{fontsize}{\@tempb}%
    }{%
%</book>
      \InputIfFileExists{size\@tempb.clo}{%
        \ClassInfo{\KOMAClassName}{%
          File `size\@tempb.clo' used to setup font sizes}%
        \KOMA@kav@removekey{.\KOMAClassFileName}{fontsize}%
        \KOMA@kav@xadd{.\KOMAClassFileName}{fontsize}{\@tempb}%
      }{%
%    \end{macrocode}
% Sind auch diese Dateien nicht vorhanden, so wird als Fallback eine
% Berechnung der Schriftgrößen verwendet. Es sei darauf hingewiesen, dass
% dabei auch \cs{@tempa}, \cs{@tempb} und \cs{@ptsize} neu gesetzt werden.
%    \begin{macrocode}
        \edef\@tempa{%
          \noexpand\changefontsizes{\@tempa}%
          \noexpand\KOMA@kav@xadd{.\KOMAClassFileName}{fontsize}{\@tempa}%
        }\@tempa
%    \end{macrocode}
% Damit sollten nun die Schriftgrößen eingestellt sein.
%    \begin{macrocode}
      }%
%<book>    }%
  }%
}
%</!extend>
%</class|extend>
%</body>
%    \end{macrocode}
% 
%
% \subsection{"`Alte"' Font-Auswahlbefehle}
%
% \begin{option}{enabledeprecatedfontcommands}
% \changes{v3.20}{2015/10/14}{neue (veraltete) Option}%^^A
% Diese Option ist bereits bei ihrer Definition veraltet. Sie stellt quasi die
% nächste Eskalation für die veralteten Font-Auswahlbefehle dar. Man kann mit
% ihrer Hilfe notfalls noch einmal die alten Befehle zurück holen. Dabei wird
% in der Voreinstellung weiterhin eine Warnung ausgegeben.
% \begin{macro}{\scr@defineobsoletefonts}
% \changes{v3.20}{2015/10/14}{neue (veraltete) Anweisung (intern)}%^^A
% Über diese Anweisung wird eingestellt, ob die veralteten Befehle mit
% Fehlermeldung, Warnung, Info oder gar nicht gemeldet werden.
% Negative Werte bedeuten, dass die Befehle nicht definiert werden. Null steht
% für interne Fehlermeldung. Eins ist eine Warnung. Zwei ist eine Info und
% darüber hinaus werden die Befehle einfach definiert. Voreingestellt ist
% derzeit noch Null. In naher Zukunft wird das aber -1 werden.
%    \begin{macrocode}
%<*option>
%<*class>
\DeclareOption{enabledeprecatedfontcommands}{%
  \ClassWarningNoLine{\KOMAClassName}{%
    deprecated option `enabledeprecatedfontcommands'.\MessageBreak
    Note, that this option was already depreacted when\MessageBreak
    it has been defined. Support for old font commands\MessageBreak
    has been removed from KOMA-Script more than one\MessageBreak
    decade ago. It is not recommended to use them any\MessageBreak
    longer. Therefore usage of this class option also\MessageBreak
    is not recommended%
  }%
  \let\scr@defineobsoletefonts\@ne
}
\scr@ifundefinedorrelax{scr@defineobsoletefonts}{%
  \let\scr@defineobsoletefonts\z@
}{}
\BeforePackage{tex4ht}{%
  \let\scr@defineobsoletefonts\thr@@
}
%</class>
%</option>
%    \end{macrocode}
% \end{macro}%^^A \scr@degineobsoletefonts
% \end{option}%^^A enabledeprecatedfontscommands
%
% \begin{macro}{\scr@DeclareOldFontCommand}
% \changes{v3.12}{2013/10/31}{neu}%^^A
% \changes{v3.20}{2015/10/14}{veraltet eskaliert}%^^A
% Diese Anweisung arbeitet prinzipiell wie die \LaTeX-Kern-Anweisung
% \cs{DeclareOldFontCommand}, wirft aber zusätzlich mit Warnungen um sich, um
% dem Anwender die Verwendung der alten Befehle zu versauern.
%    \begin{macrocode}
%<*body>
%<*class|extend>
%<*!extend>
\newcommand*{\scr@DeclareOldFontCommand}[3]{%
  \ifnum\scr@defineobsoletefonts<\z@\else
    \ifnum\scr@defineobsoletefonts>\tw@
      \DeclareOldFontCommand{#1}{#2}{#3}%
    \else
      \DeclareOldFontCommand{#1}{%
        \scr@ErrorWarningInfo{#1}{#2}#2%
      }{%
        \scr@ErrorWarningInfo{#1}{#3}#3%
      }%
    \fi
  \fi
}
%    \end{macrocode}
% \begin{macro}{\scr@ErrorWarningInfo}
% \changes{v3.20}{2015/10/14}{neu (intern)}%^^A
% \changes{v3.22}{2016/09/28}{message fixed}%^^A
%    \begin{macrocode}
\DeclareRobustCommand*{\scr@ErrorWarningInfo}[2]{%
%    \end{macrocode}
% Wenn \cs{scr@defineobsoletefonts} Null ist, wird eine Fehlermeldung
% ausgegeben.
%    \begin{macrocode}
  \ifnum \scr@defineobsoletefonts=\z@
    \ClassError{\KOMAClassName}{undefined old font command `\string#1'}{%
      You should note that since 1994 LaTeX2e provides a new font selection
      scheme\MessageBreak
      called NFSS2 with several new, combinable font commands.
      KOMA-Script\MessageBreak
      classes had defined the old font commands like `\string#1'
      only for compatibility\MessageBreak
      with old LaTeX 2.09 document styles of Script 2.0. Nevertheless,
      these\MessageBreak
      commands are deprecated and undocumented at least since 2003. Since
      2013\MessageBreak
      KOMA-Script classes warned about soon removement of these deprecated
      commands.\MessageBreak
      Now, after two decades of LaTeX2e and NFSS2, these commands will not
      work any\MessageBreak
      more. If loading a package results in this error message, you should
      contact\MessageBreak
      the author of that package and ask him to replace the deprecated font
      command\MessageBreak
      `\string#1', e.g., by `\detokenize{#2}`. Otherwise you should
      reconfigure\MessageBreak
      or replace the package. If you have used the old font command
      `\string#1' yourself,\MessageBreak
      you should replace it, e.g., by `\detokenize{#2}'.\MessageBreak
      To make it work for now, you can use the already also deprecated class
      option\MessageBreak
      `enabledeprecatedfontcommands'.%
    }%
  \else
    \ifcase \scr@defineobsoletefonts
%    \end{macrocode}
% Bei Eins gibt es eine Warnung:
%    \begin{macrocode}
    \or
      \expandafter\ClassWarning
%    \end{macrocode}
% Bei Zwei gibt es eine Info:
%    \begin{macrocode}
    \or
      \expandafter\ClassInfo
%    \end{macrocode}
% Sonst gibt es nichts:
%    \begin{macrocode}
    \else
      \expandafter\@gobbletwo
    \fi
    {\KOMAClassName}{deprecated old font command `\string#1' used.\MessageBreak
      You should note, that since 1994 LaTeX2e provides a\MessageBreak
      new font selection scheme called NFSS2 with several\MessageBreak
      new, combinable font commands. New KOMA-Script classes\MessageBreak
      defined the old font commands like `\string#1' only for\MessageBreak
      compatibility with LaTeX 2.09 document styles of\MessageBreak
      Script 2.0. These commands are deprecated and\MessageBreak
      undocumented at least since 2003. Since 2013,\MessageBreak
      KOMA-Script classes warned about soon removement of\MessageBreak
      these deprecated commands. Now, after two decades of\MessageBreak
      LaTeX2e, NFSS2, and KOMA-Script these commands will\MessageBreak
      not work any longer. If loading a package results in\MessageBreak
      this message you should contact the author of that\MessageBreak
      package and ask him to replace the depracted font\MessageBreak
      command `\string#1', e.g., by `\detokenize{#2}'.\MessageBreak
      Otherwise you should reconfigure or replace the\MessageBreak
      package. If you have used the old font command\MessageBreak
      `\string #1' yourself you should replace it, e.g., by\MessageBreak
      `\detokenize{#2}'%
    }%
  \fi
}
%</!extend>
%    \end{macrocode}
% \end{macro}%^^A \scr@OldFontErrorWarningInfo
% \end{macro}%^^A \scr@DeclareOldFontCommand
%
% \begin{macro}{\rm}
% \changes{v3.12}{2013/10/13}{Anweisung ist veraltet}%^^A
% \begin{macro}{\sf}
% \changes{v3.12}{2013/10/13}{Anweisung ist veraltet}%^^A
% \begin{macro}{\tt}
% \changes{v3.12}{2013/10/13}{Anweisung ist veraltet}%^^A
% \begin{macro}{\bf}
% \changes{v3.12}{2013/10/13}{Anweisung ist veraltet}%^^A
% \begin{macro}{\it}
% \changes{v3.12}{2013/10/13}{Anweisung ist veraltet}%^^A
% \begin{macro}{\sl}
% \changes{v3.12}{2013/10/13}{Anweisung ist veraltet}%^^A
% \begin{macro}{\sc}
% \changes{v3.12}{2013/10/13}{Anweisung ist veraltet}%^^A
% \begin{macro}{\sfb}
% \changes{v2.2c}{1995/05/25}{nicht mehr mathematisch}%^^A
% \changes{v2.3a}{1995/07/08}{keine Unterscheidung mehr für den
%      Kompatibilitätsmodus}%^^A
% \changes{v3.12}{2013/10/13}{Anweisung ist veraltet}%^^A
% Um die Umstellung von \LaTeX\ auf \LaTeXe\ zu erleichtern, gibt es
% die alten Font-Auswahlbefehle \cs{rm}, \cs{sf}, \cs{tt}, \cs{bf},
% \cs{it}, \cs{sl} und \cs{sc} auch in den Klassen. Es ist zu
% beachten, dass diese Befehle hier nach dem alten
% Fontauswahlverfahren arbeiten. Es werden also immer alle Parameter
% zugleich geändert. Somit ist zu empfehlen, dass statt dieser Befehle
% zukünftig in der Regel die neuen \cs{text\dots}-Befehle verwendet
% werden.
%
% Der aus der \textsf{Script 2.0}-Familie bekannte Befehl \cs{sfb} ist
% hier ebenfalls definiert.
%    \begin{macrocode}
%<*!extend>
\scr@DeclareOldFontCommand{\rm}{\normalfont\rmfamily}{\mathrm}
\scr@DeclareOldFontCommand{\sf}{\normalfont\sffamily}{\mathsf}
\scr@DeclareOldFontCommand{\tt}{\normalfont\ttfamily}{\mathtt}
\scr@DeclareOldFontCommand{\bf}{\normalfont\bfseries}{\mathbf}
\scr@DeclareOldFontCommand{\it}{\normalfont\itshape}{\mathit}
\scr@DeclareOldFontCommand{\sl}{\normalfont\slshape}{\@nomath\sl}
\scr@DeclareOldFontCommand{\sc}{\normalfont\scshape}{\@nomath\sc}
\scr@DeclareOldFontCommand{\sfb}{\normalfont\sffamily\bfseries}{%
  \@nomath\sfb}
%</!extend>
%    \end{macrocode}
% \end{macro}
% \end{macro}
% \end{macro}
% \end{macro}
% \end{macro}
% \end{macro}
% \end{macro}
% \end{macro}
%
% \begin{macro}{\cal}
% \changes{v2.3}{1995/06/25}{hier statt im \LaTeX-Kern}%^^A
% Der Befehl ist in \LaTeX{} nicht mehr definiert. Deshalb wird er nun
% neu und gleich robust deklariert. Dafür fällt \cs{pcal} weg.
% \changes{v2.3g}{1996/01/14}{überflüssige Klammerpaare entfernt}%^^A
% \begin{macro}{\mit}
% \changes{v2.3}{1995/06/25}{hier statt im \LaTeX-Kern}%^^A
% Der Befehl ist in \LaTeX{} nicht mehr definiert. Deshalb wird er nun
% neu und gleich robust deklariert. Dafür fällt \cs{pmit} weg.
% \changes{v2.3g}{1996/01/14}{überflüssige Klammerpaare entfernt}%^^A
%
% Die beiden Befehle \cs{cal} und \cs{mit} war bis zur \LaTeX-Version
% vom 1.12.1994 patch level 3 noch im Kernal definiert. Jetzt sind die
% beiden den Klassen überlassen. Hier ist die Definition aus den
% Standard-classes übernommen.
%    \begin{macrocode}
%<*!extend>
\DeclareRobustCommand*{\cal}{\@fontswitch\relax\mathcal}
\DeclareRobustCommand*{\mit}{\@fontswitch\relax\mathnormal}
%</!extend>
%    \end{macrocode}
% \end{macro}
% \end{macro}
%
% \iffalse
%</class|extend>
% \fi
%
% \subsection{Setzen der Schrift eines Elements}
%
% \iffalse
%<*scrkbase>
% \fi
%
% \begin{macro}{\IfExistskomafont}
% \changes{v3.15}{2014/11/21}{neue Anweisung}%^^A
% Führe das zweite Argument aus, wenn ein Fontelement existiert, sonst das
% dritte.
%    \begin{macrocode}
\newcommand*{\IfExistskomafont}[1]{%
  \scr@ifundefinedorrelax{scr@fnt@#1}{%
    \scr@ifundefinedorrelax{scr@fnt@instead@#1}{\@secondoftwo}{\@firstoftwo}%
  }{%
    \@firstoftwo
  }%
}
%    \end{macrocode}
% \end{macro}
%
% \begin{macro}{\IfIsAliaskomafont}
% \changes{v3.25}{2017/12/08}{neue Anweisung}%^^A
% Führe das zweite Argument aus, wenn ein Fontelement ein Alias is, sonst das
% dritte.
%    \begin{macrocode}
\newcommand*{\IfIsAliaskomafont}[1]{%
  \scr@ifundefinedorrelax{scr@fnt@#1}{%
    \scr@ifundefinedorrelax{scr@fnt@instead@#1}{\@secondoftwo}{\@firstoftwo}%
  }{%
    \@secondoftwo
  }%
}
%    \end{macrocode}
% \end{macro}
%
% \begin{macro}{\setkomafont}
% \changes{v2.8o}{2001/09/14}{neu}%^^A
% \changes{v3.05a}{2010/03/24}{nach \textsf{scrkbase} verschoben}
% Mit Hilfe dieses Makros kann die Schriftart von
% \KOMAScript-Elementen gesetzt werden.
%    \begin{macrocode}
\newcommand*{\setkomafont}[2]{%
  \@ifundefined{scr@fnt@#1}{%
    \@ifundefined{scr@fnt@instead@#1}{%
      \PackageError{scrkbase}{%
        font of element `#1' can't be set}{%
        You've told me to redefine the font selection of the
        element,\MessageBreak%
        but either no such element is known by
        KOMA-Script\MessageBreak%
        or the element does not use a special font selection%
      }%
    }{%
      \PackageInfo{scrkbase}{%
        You've told me to redefine the font selection of the\MessageBreak
        element `#1' that is an alias of element\MessageBreak
        `\csname scr@fnt@instead@#1\endcsname'%
      }%
      \expandafter\setkomafont\expandafter{%
        \csname scr@fnt@instead@#1\endcsname}{#2}%
    }%
  }{%
    \expandafter\expandafter\expandafter\def\csname scr@fnt@#1\endcsname{#2}%
  }%
  \@ifundefined{scr@fnt@wrn@#1}{}{%
    \PackageWarning{scrkbase}{%
      \csname scr@fnt@wrn@#1\endcsname{#1}%
    }%
  }%
}
%    \end{macrocode}
% \end{macro}
% \begin{macro}{\addtokomafont}
% \changes{v2.8p}{2001/09/22}{neu}%^^A
% \changes{v3.05a}{2010/03/24}{nach \textsf{scrkbase} verschoben}
% Dieses Makro funktioniert fast wie obiges, allerdings fügt es der
% vorhandenen Definition etwas an. Dazu wird \cs{l@addto@macro} aus
% \textsf{scrkbase} verwendet.
%    \begin{macrocode}
\newcommand*{\addtokomafont}[2]{%
  \@ifundefined{scr@fnt@#1}{%
    \@ifundefined{scr@fnt@instead@#1}{%
      \PackageError{scrkbase}{%
        font of element `#1' can't be extended%
      }{%
        You've told me to extend the font selection of the
        element,\MessageBreak
        but either no such element is known by
        KOMA-Script\MessageBreak
        or the element does not use a special font selection%
      }%
    }{%
      \PackageInfo{scrkbase}{%
        You've told me to extend the font selection of the\MessageBreak
        element `#1' that is an alias of element\MessageBreak
        `\csname scr@fnt@instead@#1\endcsname'%
      }%
      \expandafter\addtokomafont\expandafter{%
        \csname scr@fnt@instead@#1\endcsname}{#2}%
    }%
  }{%
    \expandafter\expandafter\expandafter\l@addto@macro
    \csname scr@fnt@#1\endcsname{#2}%
  }%
  \@ifundefined{scr@fnt@wrn@#1}{}{%
    \PackageWarning{scrkbase}{%
      \csname scr@fnt@wrn@#1\endcsname{#1}%
    }%
  }%
}
%    \end{macrocode}
% \end{macro}
%
% \begin{macro}{\usekomafont}
% \changes{v2.8p}{2001/09/28}{neu}%^^A
% \changes{v2.96}{2006/08/18}{Benutzung eines Fontalias korrigiert}%^^A
% \changes{v3.05a}{2010/03/24}{nach \textsf{scrkbase} verschoben}
% Mit Hilfe dieses Makros kann auf die Schriftart von
% \KOMAScript-Elementen umgeschaltet werden.
%    \begin{macrocode}
\newcommand*{\usekomafont}[1]{%
  \@ifundefined{scr@fnt@#1}{%
    \@ifundefined{scr@fnt@instead@#1}{%
      \PackageError{scrkbase}{%
        font of element `#1' can't be used%
      }{%
        You've told me to use the font selection of the
        element,\MessageBreak
        but either no such element is known by
        KOMA-Script\MessageBreak
        or the element does not use a special font selection%
      }%
    }{%
      \PackageInfo{scrkbase}{%
        You've told me to use the font selection of the
        element\MessageBreak
        `#1' that is an alias of element `\csname
        scr@fnt@instead@#1\endcsname'\MessageBreak
      }%
      \expandafter\expandafter\expandafter\usekomafont
      \expandafter\expandafter\expandafter{%
        \csname scr@fnt@instead@#1\endcsname}%
    }%
  }{%
    \@nameuse{scr@fnt@#1}%
  }%
}
%    \end{macrocode}
% \end{macro}
%
% \selectlanguage{english}
% \begin{macro}{\usesizeofkomafont}
% \changes{v3.12}{2013/03/30}{new}
% \changes{v3.17}{2015/04/03}{new more robust implementation}
% \changes{v3.24}{2017/05/06}{new more robust implementation}
% \changes{v3.34}{2021/05/17}{new implementation due to \LaTeX{} kernel
%   change}
% \begin{macro}{\usefamilyofkomafont}
% \changes{v3.12}{2013/03/30}{new}
% \changes{v3.17}{2015/04/03}{new more robust implementation}
% \changes{v3.24}{2017/05/06}{new more robust implementation}
% \begin{macro}{\useseriesofkomafont}
% \changes{v3.12}{2013/03/30}{new}
% \changes{v3.17}{2015/04/03}{new more robust implementation}
% \changes{v3.24}{2017/05/06}{new more robust implementation}
% \begin{macro}{\useshapeofkomafont}
% \changes{v3.12}{2013/03/30}{new}
% \changes{v3.16a}{2015/02/20}{fix of macro name}
% \changes{v3.17}{2015/04/03}{new more robust implementation}
% \changes{v3.24}{2017/05/06}{new more robust implementation}
% \begin{macro}{\useencodingofkomafont}
% \changes{v3.12}{2013/03/30}{new}
% \changes{v3.17}{2015/04/03}{new more robust implementation}
% \changes{v3.24}{2017/05/06}{new more robust implementation}
% \begin{macro}{\usefontofkomafont}
% \changes{v3.12}{2013/03/30}{new}
% \changes{v3.17}{2015/04/03}{new more robust implementation}
% \changes{v3.24}{2017/05/06}{new more robust implementation}
% \changes{v3.34}{2021/05/17}{new implementation due to \LaTeX{} kernel
%   change}
% Sometimes not the whole font but only single elements of a font is
% wanted.
%    \begin{macrocode}
\newcommand*{\usesizeofkomafont}[1]{%
  \begingroup
    \scr@prepareforkomafont
    \sbox\@tempboxa{%
      \usekomafont{#1}{%
        \selectfont
        \global\let\g@scr@f@size\f@size
        \global\let\g@scr@f@baselineskip\f@baselineskip
        \global\let\g@scr@f@linespread\f@linespread
      }%
    }%
  \endgroup
  \linespread{\g@scr@f@linespread}%
  \fontsize{\g@scr@f@size}{\g@scr@f@baselineskip}%
  \selectfont
}
\newcommand*{\usefamilyofkomafont}{\use@ofkomafont{family}}
\newcommand*{\useseriesofkomafont}{\use@ofkomafont{series}}
\newcommand*{\useshapeofkomafont}{\use@ofkomafont{shape}}
\newcommand*{\useencodingofkomafont}{\use@ofkomafont{encoding}}
\newcommand*{\usefontofkomafont}[1]{%
  \begingroup
    \scr@prepareforkomafont
    \sbox\@tempboxa{%
      \usekomafont{#1}{%
        \selectfont
        \global\let\g@scr@f@encoding\f@encoding
        \global\let\g@scr@f@family\f@family
        \global\let\g@scr@f@series\f@series
        \global\let\g@scr@f@shape\f@shape
        \global\let\g@scr@f@size\f@size
        \global\let\g@scr@f@baselineskip\f@baselineskip
        \global\let\g@scr@f@linespread\f@linespread
      }%
    }%
  \endgroup
  \linespread{\g@scr@f@linespread}%
  \fontsize{\g@scr@f@size}{\g@scr@f@baselineskip}%
  \usefont{\g@scr@f@encoding}{\g@scr@f@family}{\g@scr@f@series}{\g@scr@f@shape}%
}%
%    \end{macrocode}
% \begin{macro}{\use@ofkomafont}
% \changes{v3.12}{2013/03/30}{new (internal)}
% \changes{v3.17}{2015/04/03}{new more robust implementation}
% \changes{v3.24}{2017/05/06}{new more robust implementation}
% \changes{v3.34}{2021/05/17}{new implementation due to \LaTeX{} kernel
%   change}
% Helper macro for all commands above despite the first and the last one.
%    \begin{macrocode}
\newcommand*{\use@ofkomafont}[2]{%
  \begingroup
    \scr@prepareforkomafont
    \sbox\@tempboxa{%
      \usekomafont{#2}{%
        \selectfont
        \global\expandafter\let\csname g@scr@f@#1\expandafter\endcsname
                               \csname f@#1\endcsname
      }%
    }%
  \endgroup
  \@nameuse{font#1}{\csname g@scr@f@#1\endcsname}%
  \selectfont
}
%    \end{macrocode}
% \selectlanguage{ngerman}
% \begin{macro}{\g@scr@usefont}
% \changes{v3.17}{2015/04/03}{neu (intern)}%^^A
% \changes{v3.24}{2017/05/06}{entfernt}%^^A
% \end{macro}%^^A \g@scr@usefont
% \begin{macro}{\g@scr@f@encoding}
% \changes{v3.24}{2017/05/06}{neu (intern)}%^^A
% \begin{macro}{\g@scr@f@family}
% \changes{v3.24}{2017/05/06}{neu (intern)}%^^A
% \begin{macro}{\g@scr@f@series}
% \changes{v3.24}{2017/05/06}{neu (intern)}%^^A
% \begin{macro}{\g@scr@f@shape}
% \changes{v3.24}{2017/05/06}{neu (intern)}%^^A
% \begin{macro}{\g@scr@f@size}
% \changes{v3.24}{2017/05/06}{neu (intern)}%^^A
% \begin{macro}{\g@scr@f@baselineskip}
% \changes{v3.24}{2017/05/06}{neu (intern)}%^^A
% \begin{macro}{\g@scr@f@linespread}
% \changes{v3.24}{2017/05/06}{neu (intern)}%^^A
%    \begin{macrocode}
\newcommand*{\g@scr@f@encoding}{}\let\g@scr@f@encoding\f@encoding
\newcommand*{\g@scr@f@family}{}\let\g@scr@f@family\f@family
\newcommand*{\g@scr@f@series}{}\let\g@scr@f@series\f@series
\newcommand*{\g@scr@f@shape}{}\let\g@scr@f@shape\f@shape
\newcommand*{\g@scr@f@size}{}\let\g@scr@f@size\f@size
\newcommand*{\g@scr@f@baselineskip}{}\let\g@scr@f@baselineskip\f@baselineskip
\newcommand*{\g@scr@f@linespread}{}\let\g@scr@f@linespread\f@linespread
%    \end{macrocode}
% \end{macro}%^^A \g@scr@f@linespread
% \end{macro}%^^A \g@scr@f@family
% \end{macro}%^^A \g@scr@f@series
% \end{macro}%^^A \g@scr@f@shape
% \end{macro}%^^A \g@scr@f@size
% \end{macro}%^^A \g@scr@f@baselineskip
% \end{macro}%^^A \g@scr@f@linespread
% \begin{macro}{\scr@komafontrelaxlist}
% \changes{v3.17}{2015/04/03}{neu (intern)}%^^A
% \changes{v3.24}{2017/05/05}{\cs{uppercase} und \cs{lowercase} entfernt}%^^A
% \changes{v3.24}{2017/05/05}{\cs{MakeUppercase} und \cs{MakeLowercase} nach
%     \cs{scr@komafontonearglist} verschoben}%^^A
% Hilfsmakro mit einer |\do|-Liste aller Makros, die in obigen Font-Makros vor
% der Anwendung des Fonts |\relax| verwenden sollen.
%    \begin{macrocode}
\newcommand*{\scr@komafontrelaxlist}{%
  \do\normalcolor 
}
%    \end{macrocode}
% \end{macro}%^^A \scr@komafontrelaxlist
% \begin{macro}{\scr@komafontgobblelist}
% \changes{v3.19}{2015/08/20}{neu (intern)}%^^A
% Wie \cs{scr@komafontrelaxlist} nur dass |\@gobble| statt |\relax| verwendet
% wird.
%    \begin{macrocode}
\newcommand*{\scr@komafontgobblelist}{%
  \do\color
}
%    \end{macrocode}
% \end{macro}%^^A \scr@komafontgobblelist
% \begin{macro}{\scr@komafontonearglist}
% \changes{v3.24}{2017/05/05}{neu (intern)}%^^A
%   Hilfsmakro mit einer |\do|-Liste aller Makros, die in obige Font-Makros
%   vor der Anwendung des Fonts |\@firstofone| verwenden sollen.
%    \begin{macrocode}
\newcommand*{\scr@komafontonearglist}{%
  \do\MakeUppercase \do\MakeLowercase
}
%    \end{macrocode}
% \end{macro}
% \begin{macro}{\scr@prepareforkomafont}
% \changes{v3.17}{2015/04/03}{neu (intern)}%^^A
% \changes{v3.19}{2015/08/20}{abarbeiten von \cs{scr@komafontgobblelist}}%^^A
% \changes{v3.24}{2017/05/05}{abarbeiten von \cs{scr@komafontonearglist}}%^^A
% Hilfsmakro, das die Vorbereitungen für die obigen Font-Makros trifft. In der
% Voreinstellung ist das lediglich die Anwendung der |\do|-Liste, so dass die
% dort angegebenen Makros alle |\relax| bzw. |\@gobble| werden. Da die
% Anweisung dazu |\do| umdefiniert, sollte die Anweisung immer in einer Gruppe
% eingeschlossen werden.
%    \begin{macrocode}
\newcommand*{\scr@prepareforkomafont}{%
  \long\def\do##1{\let##1\relax}\scr@komafontrelaxlist
  \long\def\do##1{\let##1\@gobble}\scr@komafontgobblelist
  \long\def\do##1{\let##1\@firstofone}\scr@komafontonearglist
}
%    \end{macrocode}
% \end{macro}%^^A \scr@prepareforkomafont
% \begin{macro}{\addtokomafontrelaxlist}
% \changes{v3.17}{2015/04/03}{Neu}
% Diese Anweisung erlaubt es genau ein Makro der obigen |\do|-Liste
% hinzuzufügen.
%    \begin{macrocode}
\newcommand*{\addtokomafontrelaxlist}[1]{%
  \l@addto@macro\scr@komafontrelaxlist{\do#1}%
}
%    \end{macrocode}
% \end{macro}%^^A \addtokomafontrelaxlist
% \begin{macro}{\addtokomafontgobblelist}
% \changes{v3.19}{2015/08/20}{Neu}
% Diese Anweisung erlaubt es genau ein Makro der obigen |\do|-Liste
% hinzuzufügen.
%    \begin{macrocode}
\newcommand*{\addtokomafontgobblelist}[1]{%
  \l@addto@macro\scr@komafontgobblelist{\do#1}%
}
%    \end{macrocode}
% \end{macro}%^^A \addtokomafontgobblelist
% \begin{macro}{\addtokomafontonearglist}
% \changes{v3.24}{2017/05/05}{Neu}
%   Diese Anweisung erlaubt es genau ein Makro der obigen |\do|-Liste
%   hinzuzufügen
%    \begin{macrocode}
\newcommand*{\addtokomafontonearglist}[1]{%
  \l@addto@macro\scr@komafontonearglist{\do#1}%
}
%    \end{macrocode}
% \end{macro}%^^A \addtokonafontonearglist
% \end{macro}
% \end{macro}
% \end{macro}
% \end{macro}
% \end{macro}
% \end{macro}
% \end{macro}
%
% Um dann ein Element zu definieren, definiert man zunächst ein
% Makro, das die Schriftart enthält. Dann definiert man
% \cs{scr@fnt@\emph{Elementname}}. Wobei der Inhalt dieses Makros das
% Makro ist, das die Schriftart speichert. Damit ist das Element
% definiert. Soll ein Element durch ein anderes Element gesteuert
% werden, so definiert man stattdessen
% \cs{scr@fnt@instead@\emph{Elementname}}, wobei der Inhalt des Makros
% dann der Name jenes anderen Elements ist. Soll beim Ändern der
% Schrift eines Elements eine zusätzliche Warnung ausgegeben werden,
% so kann der Text dieser Warnung in
% \cs{scr@fnt@wrn@\emph{Elementname}} abgelegt werden. Die Warnung wird dann
% als Befehl mit einem Argument definiert. Also alles
% eigentlich ganz einfach. Damit es noch einfacher wird, ein paar 
% Hilfsmakros:
%
% \begin{macro}{\newkomafont}
% \changes{v2.95}{2004/07/21}{neu}%^^A
% \changes{v3.05a}{2010/03/24}{nach \textsf{scrkbase} verschoben}
% \changes{v3.11c}{2013/02/18}{die Warnung hat ein Argument}%^^A
% \changes{v3.25}{2017/12/08}{Warnung, wenn bereits als Alias definiert}%^^A
% Mit \cs{newkomafont} wird ein neues Element definiert. Das erste optionale
% Argument ist dabei die optionale Warnung. Ist dieses nicht gesetzt oder
% \cs{relax}, so wird keine Warnung definiert. Das erste obligatorische
% Argument ist der Name des Elements. Als letztes folgt die Voreinstellung für
% die Schrift dieses Elements. Als Fontmakro wird übrigens
% \cs{@\emph{Elementname}font} definiert.
%    \begin{macrocode}
\newcommand*{\newkomafont}[3][\relax]{%
  \scr@ifundefinedorrelax{scr@fnt@instead@#2}{}{%
    \PackageWarning{scrkbase}{%
      Making stand-alone element `#2' from\MessageBreak
      alias to `\@nameuse{scr@fnt@instead@#2}'
    }%
    \expandafter\let\csname scr@fnt@instead@#2\endcsname\relax
  }%
  \expandafter\newcommand\expandafter*\csname @#2font\endcsname{#3}%
  \expandafter\expandafter\expandafter\newcommand
  \expandafter\expandafter\expandafter*%
  \expandafter\csname scr@fnt@#2\expandafter\endcsname\expandafter{%
    \csname @#2font\endcsname%
  }%
  \ifx\relax#1\relax\else
    \expandafter\newcommand\expandafter*\csname scr@fnt@wrn@#2\endcsname[1]{#1}%
  \fi
}
%    \end{macrocode}
% \begin{macro}{\aliaskomafont}
% \changes{v2.95}{2004/07/21}{neu}%^^A
% \changes{v3.05a}{2010/03/24}{nach \textsf{scrkbase} verschoben}%^^A
% \changes{v3.25}{2017/12/08}{Warnung, wenn bereits als Font definiert}%^^A
% Mit \cs{aliaskomafont} wird hingegen einfach ein Alias definiert. Dieser
% Befehl kennt zwei Argumente: der Name des Elements und der Names des
% Elements, das stattdessen verwendet werden soll.
%    \begin{macrocode}
\newcommand*{\aliaskomafont}[2]{%
  \scr@ifundefinedorrelax{scr@fnt@#1}{}{%
    \PackageWarning{scrkbase}{%
      Redefining stand alone element `#1' as\MessageBreak
      alias to `#2'%
    }
    \expandafter\let\csname scr@fnt@#1\endcsname\relax
  }%
  \scr@ifundefinedorrelax{scr@fnt@wrn@#1}{}{%
    \expandafter\let\csname scr@fnt@wrn@#1\encsname\relax
  }
  \expandafter\newcommand\expandafter*\csname scr@fnt@instead@#1\endcsname{%
    #2%
  }%
}
%    \end{macrocode}
% \end{macro}
% \end{macro}
%
% \iffalse
%</scrkbase>
% \fi
%
% \iffalse
%</body>
% \fi
%
% \Finale
%
\endinput
%
% end of file `scrkernel-fonts.dtx'
%%% Local Variables:
%%% mode: doctex
%%% TeX-master: t
%%% End:
