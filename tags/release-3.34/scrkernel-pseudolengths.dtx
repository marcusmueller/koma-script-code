% \CheckSum{223}
% \iffalse meta-comment
% ======================================================================
% scrkernel-pseudolengths.dtx
% Copyright (c) Markus Kohm, 2006-2019
%
% This file is part of the LaTeX2e KOMA-Script bundle.
%
% This work may be distributed and/or modified under the conditions of
% the LaTeX Project Public License, version 1.3c of the license.
% The latest version of this license is in
%   http://www.latex-project.org/lppl.txt
% and version 1.3c or later is part of all distributions of LaTeX 
% version 2005/12/01 or later and of this work.
%
% This work has the LPPL maintenance status "author-maintained".
%
% The Current Maintainer and author of this work is Markus Kohm.
%
% This work consists of all files listed in manifest.txt.
% ----------------------------------------------------------------------
% scrkernel-pseudolengths.dtx
% Copyright (c) Markus Kohm, 2006-2019
%
% Dieses Werk darf nach den Bedingungen der LaTeX Project Public Lizenz,
% Version 1.3c, verteilt und/oder veraendert werden.
% Die neuste Version dieser Lizenz ist
%   http://www.latex-project.org/lppl.txt
% und Version 1.3c ist Teil aller Verteilungen von LaTeX
% Version 2005/12/01 oder spaeter und dieses Werks.
%
% Dieses Werk hat den LPPL-Verwaltungs-Status "author-maintained"
% (allein durch den Autor verwaltet).
%
% Der Aktuelle Verwalter und Autor dieses Werkes ist Markus Kohm.
% 
% Dieses Werk besteht aus den in manifest.txt aufgefuehrten Dateien.
% ======================================================================
% \fi
%
% \CharacterTable
%  {Upper-case    \A\B\C\D\E\F\G\H\I\J\K\L\M\N\O\P\Q\R\S\T\U\V\W\X\Y\Z
%   Lower-case    \a\b\c\d\e\f\g\h\i\j\k\l\m\n\o\p\q\r\s\t\u\v\w\x\y\z
%   Digits        \0\1\2\3\4\5\6\7\8\9
%   Exclamation   \!     Double quote  \"     Hash (number) \#
%   Dollar        \$     Percent       \%     Ampersand     \&
%   Acute accent  \'     Left paren    \(     Right paren   \)
%   Asterisk      \*     Plus          \+     Comma         \,
%   Minus         \-     Point         \.     Solidus       \/
%   Colon         \:     Semicolon     \;     Less than     \<
%   Equals        \=     Greater than  \>     Question mark \?
%   Commercial at \@     Left bracket  \[     Backslash     \\
%   Right bracket \]     Circumflex    \^     Underscore    \_
%   Grave accent  \`     Left brace    \{     Vertical bar  \|
%   Right brace   \}     Tilde         \~}
%
% \iffalse
%%% From File: $Id$
%<option>%%%            (run: option)
%<body>%%%            (run: body)
%<*dtx>
% \fi
\ifx\ProvidesFile\undefined\def\ProvidesFile#1[#2]{}\fi
\begingroup
  \def\filedate$#1: #2-#3-#4 #5${\gdef\filedate{#2/#3/#4}}
  \filedate$Date$
  \def\filerevision$#1: #2 ${\gdef\filerevision{r#2}}
  \filerevision$Revision: 1872 $
  \edef\reserved@a{%
    \noexpand\endgroup
    \noexpand\ProvidesFile{scrkernel-pseudolengths.dtx}%
                          [\filedate\space\filerevision\space
                           KOMA-Script source
                          (pseudo length)]
  }%
\reserved@a
% \iffalse
\documentclass{scrdoc}
\usepackage[english,ngerman]{babel}
\CodelineIndex
\RecordChanges
\GetFileInfo{scrkernel-pseudolengths.dtx}
\title{\KOMAScript{} \partname\ \texttt{\filename}%
  \footnote{Dies ist Version \fileversion\ von Datei \texttt{\filename}.}}
\date{\filedate}
\author{Markus Kohm}

\begin{document}
  \maketitle
  \tableofcontents
  \DocInput{\filename}
\end{document}
%</dtx>
% \fi
%
% \selectlanguage{ngerman}
%
% \changes{v2.95}{2006/03/22}{%
%   erste Version aus der Aufteilung von \textsf{scrclass.dtx}}
%
% \section{Pseudo-Längen für Briefe}
%
% Pseudo-Längen wurden für \textsf{scrlttr2} eingeführt, um Längenregister zu
% sparen. Tatsächlich hat das \LaTeX-Team kurz darauf entschieden, dass
% künftig \eTeX{} die empfohlene Maschine für \LaTeX{} sein soll, wodurch die
% Einführung von Pseudo-Längen nachträglich überflüssig wurde. Es gibt sie nun
% aber einmal und aus Gründen der Kompatiblität wird es sie für Briefe auch
% weiterhin geben.
%
% \StopEventually{\PrintIndex\PrintChanges}
%
% \iffalse
%<*letter>
% \fi
%
% \iffalse
%<*option>
% \fi
%
% \subsection{Option}
% Das Prinzip der Pseudo-Längen wird nicht durch Optionen beeinflusst.
%
%
% \iffalse
%</option>
%<*body>
% \fi
%
% \subsection{Anweisungen für Pseudo-Längen}
%
% Zu Berechnung der Feldpositionen werden verschiedene Dimensions-Variablen
% benötigt. Um jedoch nicht unnötig viele Längenregister zu verbrauchen, wird
% hier mit einer Pseudo-Längen-Arithmetik gearbeitet.
%
% \selectlanguage{english}%^^A
% \begin{macro}{\Ifplength}
% \changes{v3.27}{2019/03/29}{new}%^^A
% \begin{macro}{\if@plength}
% \changes{v3.27}{2019/03/29}{new (internal)}%^^A
% Test whether or not a given pseudo-length (\texttt{\#1}) already exists. If
% so execute the second argument, otherwise the third. It's somehow
% inconsistent to use \cs{Ifplength} instead of \cs{ifplength}, but as it does
% not act as a \TeX{} switch but a \LaTeX{} command it is better not to use
% \cs{if\dots}.
%    \begin{macrocode}
\newcommand*{\if@plength}[1]{%
  \@ifundefined{ltr@len@#1}{\@secondoftwo}{\@firstoftwo}%
}
\newcommand*{\Ifplength}{\if@plength}
%    \end{macrocode}
% \end{macro}%^^A\Ifplength
% \end{macro}%^^A\if@plength
% \selectlanguage{ngerman}%^^A
%
% \begin{macro}{\@newplength}
% \changes{v2.8q}{2001/10/06}{Neu}%^^A
% \changes{v2.8q}{2001/10/06}{Kompatibilität zu \texttt{scrlettr.cls}%^^A
%   aufgegeben}
% \changes{v3.27}{2019/03/29}{Fehlermeldung angepasst}%^^A
% Dieses Makro definiert eine neue Pseude-Länge und setzt sie auf den
% Wert 0. Eine Pseudo-Länge ist ein Makro, das eine Länge enthält.
%    \begin{macrocode}
\newcommand*{\@newplength}[1]{%
  \@ifundefined{ltr@len@#1}{%
    \expandafter\let\csname ltr@len@#1\endcsname=\z@%
  }{%
%<class>    \ClassError{scrlttr2%
%<package>    \PackageError{scrletter%
    }{%
      pseudo-length \expandafter\string\csname ltr@len@#1\endcsname
      already defined%
      }{%
      You've tried to define a new pseudo-length using\MessageBreak
      \string\@newplength\space
      or \string\newplength.\MessageBreak
      Please, try another name}%
    }%
  }
%    \end{macrocode}
% \end{macro}
%
% \begin{macro}{\newplength}
% \changes{v3.26}{2018/04/26}{Neu}
% Anweisung auf Anwenderebene verfügbar.
%    \begin{macrocode}
\newcommand*{\newplength}{\@newplength}
%    \end{macrocode}
% \end{macro}
%
% \begin{macro}{\useplength}
% \changes{v2.8q}{2001/10/06}{Neu}%^^A
% \changes{v2.97c}{2007/09/17}{\cs{dimexpr} statt \cs{number} verwendet,
% wenn \eTeX{} verwendet wird}
% \changes{v3.25}{2017/11/29}{\eTeX{} ist zwingend}
% \changes{v3.25}{2017/11/29}{\cs{dimexpr} durch \cs{glueexpr} ersetzt}%^^A
% Natürlich muss auf eine solche Pseudo-Länge auch wieder lesend
% zugeriffen werden können. Wir definieren dies hier so, dass es
% außerdem möglich ist, ein Vorzeichen oder einen Faktor zu verwenden. Dafür
% wird allerdings \eTeX{} benötigt.
%    \begin{macrocode}
\newcommand*{\useplength}[1]{%
  \glueexpr \csname ltr@len@#1\endcsname\relax}%
%    \end{macrocode}
% \end{macro}
%
% \begin{macro}{\setlengthtoplength}
% \changes{v2.8q}{2001/10/06}{Neu}%^^A
% Eine echte Länge soll auf das Vielfache einer Pseudo-Länge gesetzt
% werden. Dabei wird der Faktor als optionaler Parameter übergeben.
% Der erste obligatorische Parameter ist die Länge, der zweite
% obligatorische Parameter ist die Pseudo-Länge.
%    \begin{macrocode}
\newcommand*{\setlengthtoplength}[3][]{%
  \setlength{#2}{\useplength{#3}}%
  \setlength{#2}{#1#2}}
%    \end{macrocode}
% \end{macro}
%
% \begin{macro}{\@setplength}
% \changes{v2.8q}{2001/10/06}{Neu}%^^A
% \changes{v3.25}{2017/11/29}{\cs{glueexpr} eingefügt}%^^A
% Natürlich wird ebenfalls eine Möglichkeit benötigt, einer
% Pseudo-Länge einen Wert zuzuweisen. Dies geschieht unter lokaler
% Anwendung einer Länge, damit sichergestellt ist, dass eine
% Längenangabe übergeben wird. Dabei wird das zuvor definierte Makro
% \cs{setlengthtoplength} verwendet.
%    \begin{macrocode}
\newcommand*{\@setplength}[3][]{%
  \begingroup%
    \setlength{\@tempskipa}{\glueexpr #3\relax}%
    \setlength{\@tempskipa}{#1\@tempskipa}%
    \edef\@tempa{\noexpand\endgroup%
      \noexpand\expandafter\noexpand\renewcommand\noexpand\expandafter*%
      \noexpand\csname ltr@len@#2\noexpand\endcsname{\the\@tempskipa}%
    }%
  \@tempa
}
%    \end{macrocode}
% \end{macro}
%
% \begin{macro}{\setplength}
% \changes{v3.26}{2018/04/26}{Neu}%^^A
% Anweisung auf Anwenderebene verfügbar.
%    \begin{macrocode}
\newcommand*{\setplength}{\@setplength}
%    \end{macrocode}
% \end{macro}
%
% \begin{macro}{\setplengthtowidth}
% \changes{v3.26}{2018/04/26}{Neu}
% \begin{macro}{\setplengthtoheight}
% \changes{v3.26}{2018/04/26}{Neu}
% \begin{macro}{\setplengthtodepth}
% \changes{v3.26}{2018/04/26}{Neu}
% \begin{macro}{\setplengthtototalheight}
% \changes{v3.26}{2018/04/26}{Neu}
% Man kann auch die Breite, Höhe, Tiefe oder Höhe + Tiefe eines Inhalts
% verwenden, um eine Pseudolänge zu definieren.
%    \begin{macrocode}
\newcommand*{\setplengthtowidth}[3][]{%
  \begingroup
    \settowidth{\@tempdima}{#3}%
    \setlength{\@tempdima}{#1\@tempdima}%
    \edef\@tempa{\noexpand\endgroup
      \noexpand\expandafter\noexpand\renewcommand\noexpand\expandafter*%
      \noexpand\csname ltr@len@#2\noexpand\endcsname{\the\@tempdima}%
    }%
  \@tempa
}  
\newcommand*{\setplengthtoheight}[3][]{%
  \begingroup
    \settoheight{\@tempdima}{#3}%
    \setlength{\@tempdima}{#1\@tempdima}%
    \edef\@tempa{\noexpand\endgroup
      \noexpand\expandafter\noexpand\renewcommand\noexpand\expandafter*%
      \noexpand\csname ltr@len@#2\noexpand\endcsname{\the\@tempdima}%
    }%
  \@tempa
}  
\newcommand*{\setplengthtodepth}[3][]{%
  \begingroup
    \settodepth{\@tempdima}{#3}%
    \setlength{\@tempdima}{#1\@tempdima}%
    \edef\@tempa{\noexpand\endgroup
      \noexpand\expandafter\noexpand\renewcommand\noexpand\expandafter*%
      \noexpand\csname ltr@len@#2\noexpand\endcsname{\the\@tempdima}%
    }%
  \@tempa
}  
\newcommand*{\setplengthtototalheight}[3][]{%
  \begingroup
    \settoheight{\@tempdima}{#3}%
    \settodepth{\@tempdimb}{#3}%
    \addtolength{\@tempdima}{\@tempdimb}%
    \setlength{\@tempdima}{#1\@tempdima}%
    \edef\@tempa{\noexpand\endgroup
      \noexpand\expandafter\noexpand\renewcommand\noexpand\expandafter*%
      \noexpand\csname ltr@len@#2\noexpand\endcsname{\the\@tempdima}%
    }%
  \@tempa
}  
%    \end{macrocode}
% \end{macro}
% \end{macro}
% \end{macro}
% \end{macro}
%
% \begin{macro}{\addtolengthplength}
% \changes{v2.8q}{2001/10/06}{Neu}%^^A
% Wenn wir schon dabei sind, soll statt dem Setzen auch das Addieren
% funktionieren. Hier ist die Erschwernis, dass der Faktor nur für
% die Pseudo-Länge und nicht für das Endergebnis gelten soll.
%    \begin{macrocode}
\newcommand*{\addtolengthplength}[3][]{%
  \begingroup%
    \setlengthtoplength[{#1}]{\@tempskipa}{#3}%
    \edef\@tempa{\endgroup%
      \noexpand\addtolength{#2}{\the\@tempskipa}}%
    \@tempa%
  }
%    \end{macrocode}
%
% \begin{macro}{\@addtoplength}
% \changes{v2.8q}{2001/10/06}{Neu}%^^A
% \changes{v3.25}{2017/11/29}{\cs{glueexpr} eingefügt}%^^A
% Ebenso wird ein Makro benötigt, mit dem zu einer Pseudo-Länge das
% Vielfache eines Wertes addiert werden kann.
%    \begin{macrocode}
\newcommand*{\@addtoplength}[3][]{%
  \begingroup%
    \setlength{\@tempskipa}{\glueexpr #3\relax}%
    \setlength{\@tempskipa}{#1\@tempskipa}%
    \addtolengthplength{\@tempskipa}{#2}%
    \edef\@tempa{\noexpand\endgroup%
      \noexpand\expandafter\noexpand\renewcommand\noexpand\expandafter*%
      \noexpand\csname ltr@len@#2\noexpand\endcsname{\the\@tempskipa}%
      }%
    \@tempa%
  }
%    \end{macrocode}
% \end{macro}
% \end{macro}
%
% \begin{macro}{\addtoplength}
% \changes{v3.26}{2018/04/26}{Neu}
% Anweisung auf Anwenderebene verfügbar.
%    \begin{macrocode}
\newcommand*{\addtoplength}{\@addtoplength}
%    \end{macrocode}
% \end{macro}
%
% \iffalse
%</body>
% \fi
%
% \iffalse
%</letter>
% \fi
%
% \Finale
%
\endinput
%
% end of file `scrkernel-pseudolengths.dtx'
%%% Local Variables:
%%% mode: doctex
%%% TeX-master: t
%%% End:
