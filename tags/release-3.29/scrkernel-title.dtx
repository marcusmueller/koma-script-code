% \CheckSum{981}
% \iffalse meta-comment
% ======================================================================
% scrkernel-title.dtx
% Copyright (c) Markus Kohm, 2002-2019
%
% This file is part of the LaTeX2e KOMA-Script bundle.
%
% This work may be distributed and/or modified under the conditions of
% the LaTeX Project Public License, version 1.3c of the license.
% The latest version of this license is in
%   http://www.latex-project.org/lppl.txt
% and version 1.3c or later is part of all distributions of LaTeX 
% version 2005/12/01 or later and of this work.
%
% This work has the LPPL maintenance status "author-maintained".
%
% The Current Maintainer and author of this work is Markus Kohm.
%
% This work consists of all files listed in manifest.txt.
% ----------------------------------------------------------------------
% scrkernel-title.dtx
% Copyright (c) Markus Kohm, 2002-2019
%
% Dieses Werk darf nach den Bedingungen der LaTeX Project Public Lizenz,
% Version 1.3c, verteilt und/oder veraendert werden.
% Die neuste Version dieser Lizenz ist
%   http://www.latex-project.org/lppl.txt
% und Version 1.3c ist Teil aller Verteilungen von LaTeX
% Version 2005/12/01 oder spaeter und dieses Werks.
%
% Dieses Werk hat den LPPL-Verwaltungs-Status "author-maintained"
% (allein durch den Autor verwaltet).
%
% Der Aktuelle Verwalter und Autor dieses Werkes ist Markus Kohm.
% 
% Dieses Werk besteht aus den in manifest.txt aufgefuehrten Dateien.
% ======================================================================
% \fi
%
% \CharacterTable
%  {Upper-case    \A\B\C\D\E\F\G\H\I\J\K\L\M\N\O\P\Q\R\S\T\U\V\W\X\Y\Z
%   Lower-case    \a\b\c\d\e\f\g\h\i\j\k\l\m\n\o\p\q\r\s\t\u\v\w\x\y\z
%   Digits        \0\1\2\3\4\5\6\7\8\9
%   Exclamation   \!     Double quote  \"     Hash (number) \#
%   Dollar        \$     Percent       \%     Ampersand     \&
%   Acute accent  \'     Left paren    \(     Right paren   \)
%   Asterisk      \*     Plus          \+     Comma         \,
%   Minus         \-     Point         \.     Solidus       \/
%   Colon         \:     Semicolon     \;     Less than     \<
%   Equals        \=     Greater than  \>     Question mark \?
%   Commercial at \@     Left bracket  \[     Backslash     \\
%   Right bracket \]     Circumflex    \^     Underscore    \_
%   Grave accent  \`     Left brace    \{     Vertical bar  \|
%   Right brace   \}     Tilde         \~}
%
% \iffalse
%%% From File: $Id$
%<option>%%%            (run: option)
%<body>%%%            (run: body)
%<*dtx>
% \fi
\ifx\ProvidesFile\undefined\def\ProvidesFile#1[#2]{}\fi
\begingroup
  \def\filedate$#1: #2-#3-#4 #5${\gdef\filedate{#2/#3/#4}}
  \filedate$Date$
  \def\filerevision$#1: #2 ${\gdef\filerevision{r#2}}
  \filerevision$Revision: 1638 $
  \edef\reserved@a{%
    \noexpand\endgroup
    \noexpand\ProvidesFile{scrkernel-title.dtx}%
                          [\filedate\space\filerevision\space
                           KOMA-Script source (title pages)]
  }%
\reserved@a
% \iffalse
\documentclass[parskip=half-]{scrdoc}
\usepackage[english,ngerman]{babel}
\CodelineIndex
\RecordChanges
\GetFileInfo{scrkernel-title.dtx}
\title{\KOMAScript{} \partname\ \texttt{\filename}%
  \footnote{Dies ist Version \fileversion\ von Datei
    \texttt{\filename}.}}
\date{\filedate}
\author{Markus Kohm}

\begin{document}
  \maketitle
  \tableofcontents
  \DocInput{\filename}
\end{document}
%</dtx>
% \fi
%
% \selectlanguage{ngerman}
%
% \changes{v2.95}{2002/06/26}{%
%   erste Version aus der Aufteilung von \texttt{scrclass.dtx}}
%
% \section{Die Titelei}
%
% Die Titelei ist gegenüber den Standardklassen erheblich
% erweitert. Trotzdem kann ein Titel einer Standardklasse unmittelbar
% mit einer \KOMAScript-Klasse gesetzt werden. Briefe besitzen allerdings
% keine Titelei im eigentlichen Sinne.
%
% \StopEventually{\PrintIndex\PrintChanges}
%
% \iffalse
%<*!letter>
% \fi
%
% \iffalse
%<*option>
% \fi
%
%
% \subsection{Optionen der Titelei}
%
% \begin{option}{titlepage}
% \changes{v2.95c}{2006/08/21}{als \textsf{keyval}-Option}%^^A
% \changes{v3.12}{2013/11/21}{neuer Wert \texttt{firstiscover}}%^^A
% \changes{v3.17}{2015/03/10}{interne Speicherung des Werts}%^^A
% \changes{v3.28}{2019/11/18}{\cs{ifstr} umbenannt in \cs{Ifstr}}%^^A
% \begin{option}{notitlepage}
% \changes{v2.95c}{2006/08/21}{Option ist obsolet}%^^A
% \changes{v3.01a}{2008/11/21}{standard statt obsolet}%^^A
% Es werden zei Arten von Titeln unterstützt. Da sind zum einen eigene
% Titelseiten, zum anderen sogenannte Titelköpfe, also Titel, die nicht eine
% eigene Seite erhalten, sondern am Anfang einer Seite stehen. Zwischen den
% beiden Arten wird mit einer Option umgeschaltet. Ab Version~3.12 unterstützt
% \KOMAScript{} außerdem die Einstellung \texttt{titlepage=firstiscover} bei
% der die erste Titelseite als Umschlagseite gesetzt wird.
% \begin{macro}{\if@titlepage}
% \begin{macro}{\@titlepagetrue}
% \begin{macro}{\@titlepagefalse}
% Die gewählte Einstellung wird in einem Schalter gespeichert.
%    \begin{macrocode}
%<*class>
\newif\if@titlepage
%<article>\@titlepagefalse
%<report|book>\@titlepagetrue
%</class>
%    \end{macrocode}
% Im Falle von \textsf{scrextend} ist die Option Teil der Erweiterung.
%    \begin{macrocode}
%<*package&extend>
\scr@ext@activateable{title}{%
  \scr@ifundefinedorrelax{if@titlepage}{%
    \expandafter\newif\csname if@titlepage\endcsname
    \@titlepagefalse
  }{}%
  \scr@ifundefinedorrelax{if@titlepageiscoverpage}{%
    \expandafter\newif\csname if@titlepageiscoverpage\endcsname
    \@titlepageiscoverpagefalse
  }{}%
%</package&extend>
%    \end{macrocode}
% \end{macro}
% \end{macro}
% \end{macro}
%    \begin{macrocode}
%<*class|package>
%<class>\newif\if@titlepageiscoverpage
\KOMA@key{titlepage}[true]{%
%<!(package&extend)>  \Ifstr{#1}{firstiscover}{%
%<package&extend>  \Ifstr{##1}{firstiscover}{%
    \@titlepagetrue
    \@titlepageiscoverpagetrue
    \FamilyKeyStateProcessed
    \KOMA@kav@replacevalue{.%
%<class>      \KOMAClassFileName
%<package&extend>      scrextend.\scr@pkgextension
    }{titlepage}{firstiscover}%
  }{%
    \def\FamilyElseValue{, `firstiscover'}%
%<!(package&extend)>     \KOMA@set@ifkey{titlepage}{@titlepage}{#1}%
%<package&extend>     \KOMA@set@ifkey{titlepage}{@titlepage}{##1}%
    \ifx\FamilyKeyState\FamilyKeyStateProcessed
      \KOMA@kav@remove{.%
%<class>      \KOMAClassFileName
%<package&extend>      scrextend.\scr@pkgextension
      }{titlepage}{firstiscover}%
      \KOMA@kav@replacebool{.%
%<class>      \KOMAClassFileName
%<package&extend>      scrextend.\scr@pkgextension
      }{titlepage}{@titlepage}%
      \@titlepageiscoverpagefalse
    \fi
  }%
}
\KOMA@kav@xadd{.%
%<class>  \KOMAClassFileName
%<package&extend>  scrextend.\scr@pkgextension
}{titlepage}{\if@titlepage true\else false\fi}
%<*extend>
%    \end{macrocode}
% Diese Erweiterung soll dann auch tatsächlich sofort aktiviert werden:
%    \begin{macrocode}
  \def\scr@ext@immediate@title{%
    \scr@ext@activate{title}%
    \let\scr@ext@immediate@title\relax
  }%
}
%</extend>
%</class|package>
\KOMA@DeclareStandardOption%
%<package>  [scrextend]%
  {notitlepage}{titlepage=false}
%    \end{macrocode}
% \end{option}
% \end{option}
%
% \begin{option}{abstract}
% \changes{v2.95c}{2006/08/21}{neue \textsf{keyval}-Option}%^^A
% \begin{option}{abstracton}
% \changes{v2.95c}{2006/08/21}{Option ist obsolet}%^^A
% \changes{v3.01a}{2008/11/21}{standard statt obsolet}%^^A
% \changes{v3.27}{2019/03/13}{deprecated statt standard}%^^A
% \begin{option}{abstractoff}
% \changes{v2.95c}{2006/08/21}{Option ist obsolet}%^^A
% \changes{v3.01a}{2008/11/21}{standard statt obsolet}%^^A
% \changes{v3.27}{2019/03/13}{deprecated statt standard}%^^A
% Obwohl die Zusammenfassung nicht unbedingt auf einer Titelseite steht,
% gehört sie doch zur Titelei und wird daher hier mit behandelt. Die
% Zusammenfassung kann mit einem standardmäßigen Titel versehen werden. Der
% Titel kann aber auch unterdrückt werden.
% \iffalse
%<*class>
%<*article|report>
% \fi
% Bei Büchern gibt es keine Zusammenfassung als gesonderte Umgebung. Als
% Ersatz kann bei Büchern auf \cs{addchap*} zurückgegriffen werden.
% \begin{macro}{\if@abstrt}
% \changes{v3.17}{2015/03/10}{wird von \cs{KOMA@ifkey} definiert}%^^A
% \begin{macro}{\@abstrttrue}
% \begin{macro}{\@abstrtfalse}
% Die Enscheidung, ob der Titel der Zusammenfassung gesetzt werden soll, wird
% in einem Schalter gespeichert.
%    \begin{macrocode}
\KOMA@ifkey{abstract}{@abstrt}
\KOMA@DeclareDeprecatedOption{abstracton}{abstract=true}
\KOMA@DeclareDeprecatedOption{abstractoff}{abstract=false}
%    \end{macrocode}
% \end{macro}
% \end{macro}
% \end{macro}
% \end{option}
% \end{option}
% \end{option}
% \iffalse
%</article|report>
%</class>
% \fi
%
%
% \iffalse
%</option>
%<*body>
% \fi
%
%
% \subsection{Definitionen der Titelei}
%
% \begin{macro}{\extratitle}
% \begin{macro}{\@extratitle}
% \begin{macro}{\frontispiece}
% \changes{v3.25}{2017/11/21}{neu}
% \begin{macro}{\@frontispiece}
% \changes{v3.25}{2017/11/21}{neu (intern)}
% \begin{macro}{\titlehead}
% \begin{macro}{\@titlehead}
% \begin{macro}{\subject}
% \begin{macro}{\@subject}
% \begin{macro}{\subtitle}
% \changes{v2.97c}{2007/06/20}{neue Möglichkeit}%^^A
% \begin{macro}{\@subtitle}
% \changes{v2.97c}{2007/06/20}{neu (intern)}%^^A
% \begin{macro}{\publishers}
% \begin{macro}{\@publishers}
% \begin{macro}{\uppertitleback}
% \begin{macro}{\@uppertitleback}
% \begin{macro}{\lowertitleback}
% \begin{macro}{\@lowertitleback}
% \begin{macro}{\dedication}
% \begin{macro}{\@dedication}
% Da der Titel im \KOMAScript{} Paket wesentlich mehr Angaben erlaubt
% als bei den Standardklassen, gibt es natürlich auch einige Befehle,
% mit denen diese gesetzt werden können. Alle zusätzlichen Angaben
% sind optional und können auch weggelassen werden. In diesem Fall
% werden Leerfelder verwendet. Bis auf \cs{subject} sind alle
% Zusatzbefehle \cs{long} deklariert, können also ganze Absätze
% enthalten.
%    \begin{macrocode}
%<package&extend>\scr@ext@addto@activateable{title}{%
\newcommand*{\@extratitle}{}%
\newcommand{\extratitle}[1]{\gdef\@extratitle{%
%<extend>    ##1%
%<!extend>    #1%
}}%
\newcommand*{\@frontispiece}{}%
\newcommand{\frontispiece}[1]{\gdef\@frontispiece{%
%<extend>    ##1%
%<!extend>    #1%
}}%
\newcommand*{\@titlehead}{}%
\newcommand{\titlehead}[1]{\gdef\@titlehead{%
%<extend>    ##1%
%<!extend>    #1%
}}%
\newcommand*{\@subject}{}%
\newcommand*{\subject}[1]{\gdef\@subject{%
%<extend>    ##1%
%<!extend>    #1%
}}%
\newcommand*{\subtitle}[1]{\gdef\@subtitle{%
%<extend>    ##1%
%<!extend>    #1%
}}%
\newcommand*{\@subtitle}{}%
\newcommand*{\@publishers}{}%
\newcommand{\publishers}[1]{\gdef\@publishers{%
%<extend>    ##1%
%<!extend>    #1%
}}%
\newcommand*{\@uppertitleback}{}%
\newcommand{\uppertitleback}[1]{\gdef\@uppertitleback{%
%<extend>    ##1%
%<!extend>    #1%
}}%
\newcommand*{\@lowertitleback}{}%
\newcommand{\lowertitleback}[1]{\gdef\@lowertitleback{%
%<extend>    ##1%
%<!extend>    #1%
}}%
\newcommand*{\@dedication}{}%
\newcommand{\dedication}[1]{\gdef\@dedication{%
%<extend>    ##1%
%<!extend>    #1%
}}%
%    \end{macrocode}
% \end{macro}
% \end{macro}
% \end{macro}
% \end{macro}
% \end{macro}
% \end{macro}
% \end{macro}
% \end{macro}
% \end{macro}
% \end{macro}
% \end{macro}
% \end{macro}
% \end{macro}
% \end{macro}
% \end{macro}
% \end{macro}
% \end{macro}
% \end{macro}
%
% \begin{macro}{\next@tpage}
% \changes{v2.3b}{1995/07/24}{\cs{null} entfernt.}
% \changes{v2.3g}{1996/01/14}{\cs{newpage} durch \cs{clearpage} ersetzt}
% \changes{v3.11c}{2013/02/14}{Fußnoten der letzten Titelseite ausgeben}%^^A
% \changes{v3.11c}{2013/02/14}{Fußnoten für die nächste Titelseite
%     initialisieren}%^^A
% \changes{v3.20}{2015/10/14}{\cs{setparsizes} bzw. entsprechende Angaben
%     für die Längen eingefügt}%^^A
% \begin{macro}{\next@tdpage}
% \changes{v3.11c}{2013/02/14}{neue Anweisung (intern)}%^^A
% \changes{v3.12}{2013/11/21}{\cs{titlepage@restore} hinzugefügt}
% Innerhalb eines Titels auf die nächste Seite bzw. nächste rechte Seite
% umschalten.
%    \begin{macrocode}
\newcommand*{\next@tpage}{%
  \@thanks\global\let\@thanks\@empty
  \clearpage
  \csname titlepage@restore\endcsname
%<!extend>      \setparsizes{\z@}{\z@}{\z@\@plus 1fil}\par@updaterelative
%<extend>      \parskip\z@ \parindent\z@ \parfillskip\z@\@plus 1fil
  \thispagestyle{empty}%
  \let\footnote\thanks
  \setcounter{footnote}{0}%
}
\newcommand*{\next@tdpage}{%
  \next@tpage\if@twoside \ifodd\value{page}\else\null\next@tpage\fi\fi
}
%    \end{macrocode}
% \end{macro}
% \end{macro}
% \begin{macro}{\maketitle}
% \changes{v3.11c}{2013/02/14}{Schmutztitel bei Titelkopf in eigene
%   Anweisung verlagert}%^^A
% \changes{v3.11c}{2013/02/14}{Verwendung von \cs{next@tdpage}}%^^A
% Mit dieser Anweisung wird der Titel generiert, dessen Inhalt zuvor
% gesetzt wurde. Es muss unterschieden werden, ob der Titel auf einer
% eigenen Titelseite oder als Seitenkopf erstellt werden soll. Der
% \KOMAScript-Titel ist u.\,U. sehr groß. Im Fall dass diverse
% Zusatzelemente verwendet werden, sollte eigentlich immer eine
% Titelseite verwendet werden.%
% \changes{v2.1a}{1994/10/29}{das Hilfskonstrukt \cs{@maketitle}
%   wurde aus der Unterscheidung herausgenommen (für den Fall, dass
%   \cs{maketitle} von einem Paket überladen wird)}%^^A
% \changes{v2.3a}{1995/07/08}{\texttt{plus} durch \cs{@plus}
%   ersetzt}%^^A
% \changes{v2.3a}{1995/07/08}{\texttt{fill} durch \cs{fill}
%   ersetzt}%^^A
% \changes{v2.3d}{1995/08/19}{\cs{fill} durch \texttt{fill}
%   ersetzt}%^^A
% \changes{v2.3a}{1995/07/08}{\cs{vfil} durch \cs{vfill} ersetzt}%^^A
% \changes{v2.3b}{1995/07/24}{Umbruch bei \cs{@extratitle} in der
%   beidseitigen Titelseite korrigiert}%^^A
% \changes{v2.3g}{1996/01/14}{\cs{footnote} funktioniert nun auch
%   im Titel}%^^A
% \changes{v2.4}{1996/02/25}{\cs{footnote} funktioniert nun
%   wirklich im Titel}%^^A
% \changes{v2.4h}{1996/11/09}{egal ob der Titel auf eine Extraseite
%   kommt oder nicht, \cs{@title} wird mit \cs{sectfont} gesetzt,
%   wobei die Größe neuerdings danach eingestellt wird}%^^A
% \changes{v2.4l}{1997/02/06}{symbolische Fußnoten und
%   Fußnotensymbolbreiten von 0\,pt zur besseren Zentrierung des
%   Autors sind auch bei Titelseiten angebracht}%^^A
% \changes{v2.8p}{2001/09/22}{\cs{titlefont} an Stelle von
%   \cs{sectfont}}%^^A
% \changes{v2.95c}{2006/08/21}{Absatzgrundeinstellungen}%^^A
% \changes{v2.97c}{2007/08/31}{Vakatseiten in der Titelei immer im
%   Seitenstil \texttt{empty}}%^^A
% \changes{v3.04a}{2009/07/14}{die Unterscheidung, ob Titelseite oder
%   Titelkopf sollte erst in \cs{maketitle} und nicht schon bei dessen
%   Definition erfolgen}%^^A
% \changes{v3.12}{2013/05/29}{Verwendung neuer Font-Elemente
%   \texttt{titlehead}, \texttt{author}, \texttt{date}, \texttt{publishers},
%   \texttt{dedication}}%^^A
% \changes{v3.25}{2017/11/21}{Verwendung von \cs{@frontispiece}}%^^A
%    \begin{macrocode}
%<package&extend>\let\maketitle\relax\let\@maketitle\relax
\newcommand*\maketitle[1][1]{%
%    \end{macrocode}
% Da ab Version~3.12 auch mehrere \cs{maketitle} möglich sind, muss ggf. für
% eine korrekte Definition von \cs{and} gesorgt werden:
%    \begin{macrocode}
  \expandafter\ifnum \csname scr@v@3.12\endcsname>\scr@compatibility\relax
  \else
    \def\and{%
      \end{tabular}%
      \hskip 1em \@plus.17fil%
      \begin{tabular}[t]{c}%
    }%
  \fi
%    \end{macrocode}
% Es folgt die eigentliche Definition, die davon abhängt, ob wir uns im
% doppelseitigen Modus befinden oder nicht.
%    \begin{macrocode}
  \if@titlepage
    \begin{titlepage}
      \setcounter{page}{%
%<extend>        ##1%
%<!extend>        #1%
      }%
%    \end{macrocode}
% \changes{v3.12}{2013/11/21}{Berücksichtigung von
%   \texttt{titlepage!=firstiscover}}%^^A
% \changes{v3.27}{2019/03/18}{immer \cs{global} beim Löschen von
%   \cs{@thanks}}%^^A
% Ab Version~3.12 gibt es hier eine Sonderbehandlung für die erste Titelseite,
% falls es sich dabei um eine Umschlagsseite handeln soll. In diesem Fall
% werden die Maße der Seite verändert, was später per \cs{titlepage@restore}
% ggf. wieder zurückgenommen wird.
%    \begin{macrocode}
      \if@titlepageiscoverpage
        \edef\titlepage@restore{%
          \noexpand\endgroup
          \noexpand\global\noexpand\@colht\the\@colht
          \noexpand\global\noexpand\@colroom\the\@colroom
          \noexpand\global\vsize\the\vsize
          \noexpand\global\noexpand\@titlepageiscoverpagefalse
          \noexpand\let\noexpand\titlepage@restore\noexpand\relax
        }%
        \begingroup
        \topmargin=\dimexpr \coverpagetopmargin-1in\relax
        \oddsidemargin=\dimexpr \coverpageleftmargin-1in\relax
        \evensidemargin=\dimexpr \coverpageleftmargin-1in\relax
        \textwidth=\dimexpr
        \paperwidth-\coverpageleftmargin-\coverpagerightmargin\relax
        \textheight=\dimexpr
        \paperheight-\coverpagetopmargin-\coverpagebottommargin\relax
        \headheight=0pt
        \headsep=0pt
        \footskip=\baselineskip
        \@colht=\textheight
        \@colroom=\textheight
        \vsize=\textheight
        \columnwidth=\textwidth
        \hsize=\columnwidth
        \linewidth=\hsize
      \else
        \let\titlepage@restore\relax
      \fi
      \let\footnotesize\small
      \let\footnoterule\relax
      \let\footnote\thanks
      \renewcommand*\thefootnote{\@fnsymbol\c@footnote}%
      \let\@oldmakefnmark\@makefnmark
      \renewcommand*{\@makefnmark}{\rlap\@oldmakefnmark}%
      \ifx\@extratitle\@empty
        \ifx\@frontispiece\@empty
        \else
          \if@twoside\mbox{}\next@tpage\fi
          \noindent\@frontispiece\next@tdpage
        \fi
      \else
        \noindent\@extratitle
        \ifx\@frontispiece\@empty
        \else
          \next@tpage
          \noindent\@frontispiece
        \fi
        \next@tdpage  
      \fi
%<!extend>      \setparsizes{\z@}{\z@}{\z@\@plus 1fil}\par@updaterelative
%<extend>      \parskip\z@ \parindent\z@ \parfillskip\z@\@plus 1fil
      \ifx\@titlehead\@empty \else
        \begin{minipage}[t]{\textwidth}%
          \usekomafont{titlehead}{\@titlehead\par}%
        \end{minipage}\par
      \fi
      \null\vfill
      \begin{center}
        \ifx\@subject\@empty \else
          {\usekomafont{subject}{\@subject\par}}%
          \vskip 3em
        \fi
        {\usekomafont{title}{\huge \@title\par}}%
        \vskip 1em
        {\ifx\@subtitle\@empty\else\usekomafont{subtitle}{\@subtitle\par}\fi}%
        \vskip 2em
        {%
          \usekomafont{author}{%
            \lineskip 0.75em
            \begin{tabular}[t]{c}
              \@author
            \end{tabular}\par
          }%
        }%
        \vskip 1.5em
        {\usekomafont{date}{\@date \par}}%
        \vskip \z@ \@plus3fill
        {\usekomafont{publishers}{\@publishers \par}}%
        \vskip 3em
      \end{center}\par
      \@thanks\global\let\@thanks\@empty
      \vfill\null
      \if@twoside
%    \end{macrocode}
% \changes{v3.12}{2013/11/21}{Rückseite wird nur bei Bedarf ausgegeben.}%^^A
% Ab Version~3.12 wird die Rückseite der Titelseite nur noch bei Bedarf
% ausgegeben. Das gilt aber nur, wenn keine Kompatibilität mit früheren
% Versionen verlangt ist.
%    \begin{macrocode}
        \@tempswatrue
        \expandafter\ifnum \@nameuse{scr@v@3.12}>\scr@compatibility\relax
        \else
          \ifx\@uppertitleback\@empty\ifx\@lowertitleback\@empty
            \@tempswafalse
          \fi\fi
        \fi
        \if@tempswa
          \next@tpage
          \begin{minipage}[t]{\textwidth}
            \@uppertitleback
          \end{minipage}\par
          \vfill
          \begin{minipage}[b]{\textwidth}
            \@lowertitleback
          \end{minipage}\par
          \@thanks\global\let\@thanks\@empty
        \fi
%    \end{macrocode}
% \changes{v3.27}{2019/01/01}{Warnung für nicht leere Titelrückseite bei
%   einseitigen Dokumenten}%^^A
% Bei einseitigen Dokumenten werden eventuell gesetzte Inhalte von
% \cs{uppertitleback} oder \cs{lowertitleback} nicht ausgegeben.
%    \begin{macrocode}
      \else
        \ifx\@uppertitleback\@empty\else
%<package>          \PackageWarning{scrextend}{%
%<class>          \ClassWarning{\KOMAClassName}{%
            non empty \string\uppertitleback\space ignored
            by \string\maketitle\MessageBreak
            in `twoside=false' mode%
          }%
        \fi
        \ifx\@lowertitleback\@empty\else
%<package>          \PackageWarning{scrextend}{%
%<class>          \ClassWarning{\KOMAClassName}{%
            non empty \string\lowertitleback\space ignored
            by \string\maketitle\MessageBreak
            in `twoside=false' mode%
          }%
        \fi          
      \fi
      \ifx\@dedication\@empty 
      \else
        \next@tdpage\null\vfill
        {\centering\usekomafont{dedication}{\@dedication \par}}%
        \vskip \z@ \@plus3fill
        \@thanks\global\let\@thanks\@empty
%    \end{macrocode}
% \changes{v3.12}{2013/10/31}{Nach der Widmung kein \cs{next@tdpage},
%     sondern nur \cs{cleardoubleemptypage}}%^^A
%    \begin{macrocode}
        \cleardoubleemptypage
      \fi
%    \end{macrocode}
% Falls wir uns noch immer auf einer Umschlagseite befinden, muss diese hier
% beendet werden.
%    \begin{macrocode}
      \ifx\titlepage@restore\relax\else\clearpage\titlepage@restore\fi
    \end{titlepage}
%    \end{macrocode}
% \changes{v2.3a}{1995/07/08}{Verwendung von \cs{@makefnmark} in
%   \cs{@makefntext}}%^^A
% \changes{v2.3a}{1995/07/08}{Definition von \cs{@makefnmark} und
%   \cs{@makefntext} unabhängig von math}%^^A
% \changes{v2.3e}{1995/08/30}{optionales Argument bei der
%   einseitigen Version erlauben und ignorieren}%^^A
% \changes{v2.3g}{1996/01/14}{mehrfach \cs{null} bei
%   \cs{next@tpage} ergänzt}%^^A
% \changes{v2.4l}{1997/02/06}{es ist nicht mehr notwendig \cs{\@makefntext}
%   umzudefinieren, stattdessen wird nur \cs{\@makefnmark} umdefiniert}%^^A
% \changes{v2.8d}{2001/07/05}{\cs{titlepagestyle} statt
%   \texttt{plain}}%^^A
% \changes{v3.12a}{2014/03/03}{Leerzeichenfehler behoben}%^^A
% \changes{v3.12a}{2014/03/03}{Schmutztitel wird nur ausgegeben, wenn er
%   nicht leer ist}%^^A
% \changes{v3.12a}{2014/03/03}{\cs{next@tdpage} aus \cs{@maketitle} und
%   \cs{@makeextrattitle} entfernt}%^^A
% \changes{v3.19}{2015/08/25}{Seitenstil nur ändern wenn nicht leer}%^^A
% \changes{v3.27}{2019/03/18}{\cs{@thanks} nach Benutzung löschen}%^^A
%    \begin{macrocode}
  \else
    \par
    \@tempcnta=%
%<extend>    ##1%
%<!extend>    #1%
    \relax\ifnum\@tempcnta=1\else
%<class>      \ClassWarning{\KOMAClassName}{%
%<package>      \PackageWarning{scrextend}{%
        Optional argument of \string\maketitle\space ignored\MessageBreak
        in `titlepage=false' mode%
      }%
    \fi
    \ifx\@uppertitleback\@empty\else
%<package>     \PackageWarning{scrextend}{%
%<class>      \ClassWarning{\KOMAClassName}{%
        non empty \string\uppertitleback\space ignored
        by \string\maketitle\MessageBreak
        in `titlepage=false' mode%
      }%
    \fi
    \ifx\@lowertitleback\@empty\else
%<package>      \PackageWarning{scrextend}{%
%<class>      \ClassWarning{\KOMAClassName}{%
        non empty \string\lowertitleback\space ignored
        by \string\maketitle\MessageBreak
        in `titlepage=false' mode%
      }%
    \fi          
    \begingroup
      \let\titlepage@restore\relax
      \renewcommand*\thefootnote{\@fnsymbol\c@footnote}%
      \let\@oldmakefnmark\@makefnmark
      \renewcommand*{\@makefnmark}{\rlap\@oldmakefnmark}%
      \next@tdpage
      \if@twocolumn
        \ifnum \col@number=\@ne
          \ifx\@extratitle\@empty
            \ifx\@frontispiece\@empty\else\if@twoside\mbox{}\fi\fi
          \else
            \@makeextratitle
          \fi
          \ifx\@frontispiece\@empty
            \ifx\@extratitle\@empty\else\next@tdpage\fi
          \else
            \next@tpage
            \@makefrontispiece
            \next@tdpage
          \fi
          \@maketitle
        \else
          \ifx\@extratitle\@empty
            \ifx\@frontispiece\@empty\else\if@twoside\mbox{}\fi\fi
          \else
            \twocolumn[\@makeextratitle]%
          \fi
          \ifx\@frontispiece\@empty
            \ifx\@extratitle\@empty\else\next@tdpage\fi
          \else
            \next@tpage
            \twocolumn[\@makefrontispiece]%
            \next@tdpage
          \fi
          \twocolumn[\@maketitle]%
        \fi
      \else
        \ifx\@extratitle\@empty
          \ifx\@frontispiece\@empty\else \mbox{}\fi
        \else
          \@makeextratitle
        \fi
        \ifx\@frontispiece\@empty
          \ifx\@extratitle\@empty\else\next@tdpage\fi
        \else
          \next@tpage
          \@makefrontispiece
          \next@tdpage
        \fi
        \@maketitle
      \fi
      \ifx\titlepagestyle\@empty\else\thispagestyle{\titlepagestyle}\fi
      \@thanks\global\let\@thanks\@empty
    \endgroup
  \fi
  \setcounter{footnote}{0}%
%    \end{macrocode}
% \changes{v3.12}{2013/11/21}{die Titelei wird nur noch bei Kompatibilität
%     zu früheren Versionen zerstört}%^^A
%    \begin{macrocode}
  \expandafter\ifnum \csname scr@v@3.12\endcsname>\scr@compatibility\relax
    \let\thanks\relax
    \let\maketitle\relax
    \let\@maketitle\relax
%    \end{macrocode}
% \changes{v2.3g}{1996/01/14}{Verwendung von \cs{global}\cs{let}
%      statt \cs{gdef}, um Speicher zu sparen}
% \changes{v2.3g}{1996/01/14}{\cs{@date} und \cs{title}
%      u.\,ä. ebenfalls löschen, um Speicher zu sparen}
%    \begin{macrocode}
    \global\let\@thanks\@empty
    \global\let\@author\@empty
    \global\let\@date\@empty
    \global\let\@title\@empty
    \global\let\@subtitle\@empty
    \global\let\@extratitle\@empty
    \global\let\@frontispiece\@empty    
    \global\let\@titlehead\@empty
    \global\let\@subject\@empty
    \global\let\@publishers\@empty
    \global\let\@uppertitleback\@empty
    \global\let\@lowertitleback\@empty
    \global\let\@dedication\@empty
    \global\let\author\relax
    \global\let\title\relax
    \global\let\extratitle\relax
    \global\let\titlehead\relax
    \global\let\subject\relax
    \global\let\publishers\relax
    \global\let\uppertitleback\relax
    \global\let\lowertitleback\relax
    \global\let\dedication\relax
    \global\let\date\relax
  \fi
  \global\let\and\relax
}%
%    \end{macrocode}
% \begin{macro}{\@maketitle}
% \changes{v2.95c}{2006/08/21}{Absatzgrundeinstellungen}%^^A
% \changes{v3.11c}{2012/02/14}{Schmutztitel in eigene Anweisung verlagert}%^^A
% \changes{v3.12a}{2014/03/03}{\cs{next@tdpage} aus \cs{@maketitle}
%     entfernt}%^^A
% \begin{macro}{\@makeextratitle}
% \changes{v3.11c}{2013/02/14}{neue Anweisung (intern)}%^^A
% \changes{v3.12a}{2014/03/03}{\cs{next@tdpage} aus \cs{@makeextrattitle}
%     entfernt}%^^A
% Damit wird im Fall des Titelkopfes die eigentliche Arbeit geleistet.
% \begin{macro}{\@makefrontispiece}
% \changes{v3.25}{2017/11/21}{neu}%^^A
%    \begin{macrocode}
\newcommand*{\@makeextratitle}{%
  \ifx\@extratitle\@empty \else
    \noindent\@extratitle\par
  \fi
}
\newcommand*{\@makefrontispiece}{%
  \ifx\@frontispiece\@empty \else
    \noindent\@frontispiece\par
  \fi
}
\newcommand*{\@maketitle}{%
  \global\@topnum=\z@
%<!extend>  \setparsizes{\z@}{\z@}{\z@\@plus 1fil}\par@updaterelative
%<extend>  \parskip\z@ \parindent\z@ \parfillskip\z@\@plus 1fil
  \ifx\@titlehead\@empty \else
    \begin{minipage}[t]{\textwidth}
      \usekomafont{titlehead}{\@titlehead\par}%
    \end{minipage}\par
  \fi
  \null
  \vskip 2em%
  \begin{center}%
    \ifx\@subject\@empty \else
      {\usekomafont{subject}{\@subject \par}}%
      \vskip 1.5em
    \fi
    {\usekomafont{title}{\huge \@title \par}}%
    \vskip .5em
    {\ifx\@subtitle\@empty\else\usekomafont{subtitle}\@subtitle\par\fi}%
    \vskip 1em
    {%
      \usekomafont{author}{%
        \lineskip .5em%
        \begin{tabular}[t]{c}
          \@author
        \end{tabular}\par
      }%
    }%
    \vskip 1em%
    {\usekomafont{date}{\@date \par}}%
    \vskip \z@ \@plus 1em
    {\usekomafont{publishers}{\@publishers \par}}%
    \ifx\@dedication\@empty \else
      \vskip 2em
      {\usekomafont{dedication}{\@dedication \par}}%
    \fi
  \end{center}%
  \par
  \vskip 2em
}%
%    \end{macrocode}
% \end{macro}
% \end{macro}
% \end{macro}
% \end{macro}
%
% \begin{macro}{\coverpagetopmargin}
% \changes{v3.12}{2013/11/21}{neu}%^^A
% Oberer Rand für Umschlagsseiten.
%    \begin{macrocode}
\newcommand*{\coverpagetopmargin}{%
  \dimexpr \topmargin+1in\relax
}
%    \end{macrocode}
% \end{macro}
%
% \begin{macro}{\coverpagebottommargin}
% \changes{v3.12}{2013/11/21}{neu}%^^A
% Unterer Rand für Umschlagsseiten.
%    \begin{macrocode}
\newcommand*{\coverpagebottommargin}{%
  2\dimexpr\coverpagetopmargin\relax
}
%    \end{macrocode}
% \end{macro}
%
% \begin{macro}{\coverpageleftmargin}
% \changes{v3.12}{2013/11/21}{neu}%^^A
% Linker Rand für Umschlagsseiten.
%    \begin{macrocode}
\newcommand*{\coverpageleftmargin}{%
  \dimexpr \evensidemargin+1in\relax
}
%    \end{macrocode}
% \end{macro}
%
% \begin{macro}{\coverpagerightmargin}
% \changes{v3.12}{2013/11/21}{neu}%^^A
% Oberer Rand für Umschlagsseiten.
%    \begin{macrocode}
\newcommand*{\coverpagerightmargin}{\coverpageleftmargin}
%    \end{macrocode}
% \end{macro}
%
%
% \subsection{Umgebung für die Titelseite}
%
% \begin{environment}{titlepage}
% Die Titelseite bedarf ebenfalls einer gesonderter Umgebung. Beim
% zweispaltigen Layout soll die Titelseite z.\,B. einspaltig gedruckt
% werden.
%    \begin{macrocode}
%<package&extend>\scr@ifundefinedorrelax{titlepage}{%
\newenvironment{titlepage}{%
%<report|book>  \cleardoublepage
  \if@twocolumn
    \@restonecoltrue\onecolumn
  \else
    \@restonecolfalse\newpage
  \fi
  \thispagestyle{empty}%
  \if@compatibility
    \setcounter{page}{0}%
  \fi
}{%
  \if@restonecol\twocolumn \else \newpage \fi
}%
%<package&extend>}{}%
%    \end{macrocode}
% \end{environment}
%
% Im Fall von \textsf{scrextend} muss jetzt das Kürzel \emph{title} beendet
% werden:
%    \begin{macrocode}
%<package&extend>}\csname scr@ext@immediate@title\endcsname
%    \end{macrocode}
%
% \subsection{Fonts für den Titel}
%
% \begin{macro}{\titlefont}
% \changes{v2.8p}{2001/09/22}{neu}%^^A
% Wie oben zu sehen ist, wird in der Titelei ein eigenes Schriftmakro
% verwendet. Dieses ist als internes Makro zu verstehen. Der Anwender
% sollte stattdessen auf das entsprechende Element zugreifen (siehe
% unten).
% \begin{macro}{\subject@font}
% \changes{v2.8q}{2002/01/14}{neu (intern)}%^^A
% \changes{v2.95}{2002/06/26}{jetzt auch bei \textsf{scrbook},
%     \textsf{scrreprt} und \textsf{scrartcl}}
% Ebenso verhält es sich mit den \emph{Subject} im Titel. Hier ist
% aber von vornherein ein internes Makro definiert.
%    \begin{macrocode}
\newcommand*\titlefont{\sectfont}%
\newcommand*{\subject@font}{\normalfont\normalcolor\bfseries\Large}%
%    \end{macrocode}
% \end{macro}
% \end{macro}
%
% \begin{KOMAfont}{subtitle}
% \changes{v2.97c}{2007/06/20}{neu}%^^A
%    \begin{macrocode}
\newkomafont{subtitle}{\usekomafont{title}\large}%
%    \end{macrocode}
% \end{KOMAfont}
%
% \begin{KOMAfont}{titlehead}
% \changes{v3.12}{2013/05/29}{neu}%^^A
%    \begin{macrocode}
\newkomafont{titlehead}{}%
%    \end{macrocode}
% \end{KOMAfont}
%
% \begin{KOMAfont}{author}
% \changes{v3.12}{2013/05/29}{neu}%^^A
%    \begin{macrocode}
\newkomafont{author}{\Large}
%    \end{macrocode}
% \end{KOMAfont}
%
% \begin{KOMAfont}{date}
% \changes{v3.12}{2013/05/29}{neu}%^^A
%    \begin{macrocode}
\newkomafont{date}{\Large}
%    \end{macrocode}
% \end{KOMAfont}
%
% \begin{KOMAfont}{publishers}
% \changes{v3.12}{2013/05/29}{neu}%^^A
%    \begin{macrocode}
\newkomafont{publishers}{\Large}
%    \end{macrocode}
% \end{KOMAfont}
%
% \begin{KOMAfont}{dedication}
% \changes{v3.12}{2013/05/29}{neu}%^^A
%    \begin{macrocode}
\newkomafont{dedication}{\Large}
%    \end{macrocode}
% \end{KOMAfont}
%
% \begin{macro}{\scr@fnt@title}
% \changes{v2.8o}{2001/09/14}{neues Element \texttt{title}}
% \begin{macro}{\scr@fnt@subject}
% \changes{v2.8q}{2002/01/14}{neues Element \texttt{subject}}
% \changes{v2.95}{2002/06/26}{jetzt auch bei \textsf{scrbook},
%      \textsf{scrreprt} und \textsf{scrartcl}}
% Die beiden Elemente, deren Schrift geändert werden kann.
%    \begin{macrocode}
\newcommand*{\scr@fnt@title}{\titlefont}%
\newcommand*{\scr@fnt@subject}{\subject@font}%
%    \end{macrocode}
% \end{macro}
% \end{macro}
%
%
% \subsection{Umgebung für die Zusammenfassung}
%
% \begin{environment}{abstract}
% \textsf{scrartcl} und \textsf{scrreprt} bieten die Möglichkeit einer
% Zusammenfassung, eines sogenannten \emph{Abstracts}. Wenn eine
% Titelseite verlangt wurde, wird auch die Zusammenfassung auf eine
% eigene Seite gesetzt. \textsf{scrbook} kennt keine Zusammenfassung
% dieser Art. In Büchern werden Zusammenfassungen üblicherweise mit
% Kapitelcharakter gesetzt.
% \iffalse
%<*class>
%<*article|report>
% \fi
% \changes{v2.3a}{1995/07/08}{\cs{@endparpenalty} zur Verhinderung
%     eines Seitenumbruchs nach dem Abstract-Kopf eingefügt}%^^A
% \changes{v2.3g}{1996/01/14}{\cs{@beginparpenalty} zur
%     Verhinderung eines Seitenumbruchs vor dem Abstract-Kopf
%     eingefügt}%^^A
% \changes{v2.7a}{2001/01/04}{statt \cs{section*} wird nun
%     \cs{addsec*} verwendet, damit die Kolumnentitel korrekt
%     behandelt werden}%^^A
% \changes{v3.04a}{2009/07/14}{die Unterscheidung, ob Titelseite oder
%     Titelkopf sollte erst in der Umgebung und nicht schon bei deren
%     Definition erfolgen}%^^A
%    \begin{macrocode}
\newenvironment{abstract}{%
  \if@titlepage
    \titlepage
    \null\vfil
    \@beginparpenalty\@lowpenalty
    \if@abstrt
      \begin{center}
        \normalfont\sectfont\nobreak\abstractname
        \@endparpenalty\@M
      \end{center}
    \fi
  \else
    \if@twocolumn\if@abstrt
        \addsec*{\abstractname}
      \fi
    \else
      \if@abstrt
        \small
        \begin{center}
          {\normalfont\sectfont\nobreak\abstractname
            \vspace{-.5em}\vspace{\z@}}%
        \end{center}
      \fi
      \quotation
    \fi
  \fi
}{%
  \if@titlepage
    \par\vfil\null\endtitlepage
  \else
    \if@twocolumn\else\endquotation\fi
  \fi
}
%    \end{macrocode}
% \end{environment}%^^A abstract
%
% \begin{macro}{\abstractname}
% Name der Zusammenfassung.
%    \begin{macrocode}
\newcommand*\abstractname{Abstract}
%    \end{macrocode}
% \end{macro}%^^A \abstractname
%
% \iffalse
%</article|report>
%</class>
% \fi
%
%
% \iffalse
%</body>
% \fi
%
% \iffalse
%</!letter>
% \fi
%
% \Finale
%
\endinput
%
% end of file `scrkernel-title.dtx'
%%% Local Variables:
%%% mode: doctex
%%% TeX-master: t
%%% End:
