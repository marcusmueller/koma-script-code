% \iffalse^^A meta-comment
% ======================================================================
% komamarks.dtx
% Copyright (c) Markus Kohm, 2019
%
% This file is part of the LaTeX2e KOMA-Script bundle.
%
% This work may be distributed and/or modified under the conditions of
% the LaTeX Project Public License, version 1.3c of the license.
% The latest version of this license is in
%   http://www.latex-project.org/lppl.txt
% and version 1.3c or later is part of all distributions of LaTeX
% version 2005/12/01 and of this work.
%
% This work has the LPPL maintenance status "author-maintained".
%
% The Current Maintainer and author of this work is Markus Kohm.
%
% This work consists of all files listed in manifest.txt.
% ----------------------------------------------------------------------
% komamarks.dtx
% Copyright (c) Markus Kohm, 2019
%
% Diese Datei ist Teil der LaTeX2e KOMA-Script-Sammlung.
%
% Dieses Werk darf nach den Bedingungen der LaTeX Project Public Lizenz,
% Version 1.3c.
% Die neuste Version dieser Lizenz ist
%   http://www.latex-project.org/lppl.txt
% und Version 1.3c ist Teil aller Verteilungen von LaTeX
% Version 2005/12/01 und dieses Werks.
%
% Dieses Werk hat den LPPL-Verwaltungs-Status "author-maintained"
% (allein durch den Autor verwaltet).
%
% Der Aktuelle Verwalter und Autor dieses Werkes ist Markus Kohm.
%
% Dieses Werk besteht aus den in manifest.txt aufgefuehrten Dateien.
% ======================================================================
% \fi^^A meta-comment
%
% \CharacterTable
%  {Upper-case    \A\B\C\D\E\F\G\H\I\J\K\L\M\N\O\P\Q\R\S\T\U\V\W\X\Y\Z
%   Lower-case    \a\b\c\d\e\f\g\h\i\j\k\l\m\n\o\p\q\r\s\t\u\v\w\x\y\z
%   Digits        \0\1\2\3\4\5\6\7\8\9
%   Exclamation   \!     Double quote  \"     Hash (number) \#
%   Dollar        \$     Percent       \%     Ampersand     \&
%   Acute accent  \'     Left paren    \(     Right paren   \)
%   Asterisk      \*     Plus          \+     Comma         \,
%   Minus         \-     Point         \.     Solidus       \/
%   Colon         \:     Semicolon     \;     Less than     \<
%   Equals        \=     Greater than  \>     Question mark \?
%   Commercial at \@     Left bracket  \[     Backslash     \\
%   Right bracket \]     Circumflex    \^     Underscore    \_
%   Grave accent  \`     Left brace    \{     Vertical bar  \|
%   Right brace   \}     Tilde         \~}
%
% \iffalse^^A meta-comment
%%% From File: $Id$
%<identify>%%%            (run: identify)
%<init>%%%            (run: init)
%<options>%%%            (run: options)
%<processoptions>%%%            (run: processoptions)
%<body>%%%            (run: body)
%<final>%%%            (run: final)
%<*dtx>
\ifx\ProvidesFile\undefined\def\ProvidesFile#1[#2]{}\fi
\begingroup
  \def\filedate$#1: #2-#3-#4 #5${\gdef\filedate{#2/#3/#4}}
  \filedate$Date$
  \def\filerevision$#1: #2 ${\gdef\filerevision{#2}}
  \filerevision$Revision$
  \edef\reserved@a{%
    \noexpand\endgroup
    \noexpand\ProvidesFile{komamarks.dtx}%
                          [\filedate\space\filerevision\space
                           KOMA-Script package source
  }%
\reserved@a
%</dtx>
%<*identify|doc>
%<package>\NeedsTeXFormat{LaTeX2e}[1995/12/01]
%<package>\ProvidesPackage{komamarks}[%
%<doc>\ProvidesFile{komamarks.tex}[%
% In a future time following will be replaced by !KOMASCRIPTVERSION:
%!KOMAMARKSVERSION
%<package>  package
%</identify|doc>
%<*dtx|identify|doc>
  (providing extra marks)]
%</dtx|identify|doc>
%<*dtx>
\ifx\documentclass\undefined
  \input scrdocstrip.tex
  \@@input scrkernel-version.dtx
  \@@input scrstrip.inc
  \KOMAdefVariable{COPYRIGHTFROM}{2012}
  \KOMAdefVariable{KOMAMARKSVERSION}{%
    \space\space\filedate\space v0.\filerevision\space KOMA-Script
  }
  \generate{\usepreamble\defaultpreamble
    \file{komamarks.sty}{%
      \from{komamarks.dtx}{package,trace,identify}%
      \from{komamarks.dtx}{package,trace,init}%
      \from{komamarks.dtx}{package,trace,options}%
      \from{komamarks.dtx}{package,trace,processoptions}%
      \from{komamarks.dtx}{package,trace,body}%
      \from{komamarks.dtx}{package,trace,final}%
      \from{scrlogo.dtx}{trace,logo}%
    }%
  }

  \@@input scrstrop.inc
\else
  \let\endbatchfile\relax
\fi
\endbatchfile
  \documentclass{scrdoc}
  \addtolength{\textwidth}{-1em}
  \addtolength{\marginparwidth}{2em}
  \addtolength{\oddsidemargin}{2em}
  \usepackage[ngerman,english]{babel}
  \usepackage{url,babelbib}\bibliographystyle{babalpha-fl}
  \usepackage{listings}
  \usepackage{scrhack}
  \usepackage{xpatch}
  \xpretocmd\DescribeMacro{\ifhmode\else\bigskip\noindent\fi}{}{}
  \xpretocmd\DescribeEnv{\ifhmode\else\bigskip\noindent\fi}{}{}
  \xpretocmd\DescribeOption{\ifhmode\else\bigskip\noindent\fi}{}{}

  \CodelineIndex
  \RecordChanges
  \GetFileInfo{komamarks.dtx}
  \title{The \KOMAScript{} package \texttt{komamarks}%
    \footnote{This is revision \fileversion\ of file \texttt{\filename}.}}
  \date{\filedate}
  \author{Markus Kohm}
  
  \newenvironment{Explain}{\par}{\par}
  \newcommand*{\length}{}
  \let\length\Length
  \let\endlength\endLength
  \let\Macro\cs
  \let\Length\Macro
  \let\Package\textsf
  \let\Class\Package
  \let\File\texttt
  \let\Option\texttt
  \newcommand*{\KOption}[1]{\Option{#1}\texttt{=}}
  \newcommand*{\OptionValue}[2]{\Option{#1}\texttt{=}\PValue{#2}}
  \let\Counter\texttt
  \let\Environment\texttt
  \let\ShowOutput\quote
  \let\endShowOutput\endquote
  \let\Pagestyle\texttt
  \newcommand*{\Parameter}[1]{\texttt{\marg{#1}}\linebreak[1]}
  \newcommand*{\OParameter}[1]{\texttt{\oarg{#1}}\linebreak[1]}
  \newcommand*{\MParameter}[2]{\texttt{(\meta{#1},\meta{#2})}\linebreak[1]}
  \providecommand\PParameter[1]{\mbox{\texttt{\{#1\}}}\linebreak[1]}
  \let\PName\meta
  \let\PValue\texttt
  \providecommand*{\autoref}[1]{\expandafter\AUTOREF#1:}
  \providecommand*{\AUTOREF}{}
  \makeatletter
  \def\AUTOREF#1:#2:{%
    \edef\@tempa{#1}%
    \edef\@tempb{tab}\ifx\@tempa\@tempb table~\fi
    \edef\@tempb{sec}\ifx\@tempa\@tempb section~\fi
    \ref{#1:#2}%
  }
  \providecommand*{\IndexCmd}[2][]{}
  \providecommand*{\textnote}[2][]{}
  \providecommand*\eTeX{\leavevmode\hbox{$\varepsilon$}-\TeX}
  \providecommand*\NTS{%
    \leavevmode\hbox{$\cal N\kern-0.35em\lower0.5ex\hbox{$\cal T$}%
      \kern-0.2emS$}}

  \lstnewenvironment{lstcode}{\lstset{language=[LaTeX]TeX}}{}
  \makeatother
  \sloppy% YOU SHOULD NOT DO THIS!!!

  \begin{document}
  \maketitle
  \DocInput{\filename}
  \bibliography{guide}
  \PrintChanges
  \PrintIndex
  \end{document}
%</dtx>
% \fi^^A meta-comment
%
% \selectlanguage{english}
%
% \changes{v0.0}{2019/08/08}{Start of new package}%^^A
%
% \tableofcontents
% \section{User Manual}
% This is an experimental beta version without user manual.
%
% \StopEventually{}
%
%
% \section{Implementation of \Package{komamarks}}
%
% This section is for developers only.
%
%\iffalse%^^A meta-comment
%<*package>
%\fi%^^A meta-comment
%
%
% \subsection{Options}
% \label{sec:kommarks.options}
%
% This package currently does not provide any options. Nevertheless
% processing the options is recommended:
%    \begin{macrocode}
%<*processoptions>
\ProcessOptions\relax
%</processoptions>
%    \end{macrocode}
%
%
% \subsection{Handling for new marks similar to \LaTeX}
% \label{sec:komamarks.newmarks}
%
% \LaTeX{} provides \cs{markright}, \cs{markboth}, \cs{leftmark} and
% \cs{rightmark} using \TeX's \cs{mark}, \cs{firstmark} and
% \cs{botmark}. Let's do the same for \eTeX's \cs{marks}, \cs{firstmarks},
% \cs{botmarks} and add some new features like \Package{scrlayer} does.
%
% \begin{macro}{\marksboth}
% \begin{macro}{\marksright}
% \begin{macro}{\@marksright}
% These are similar to \cs{markboth} and \cs{markright} but uses \cs{marks}
% instead of \cs{mark} and therefore has one new mandator argument: the marks
% register. We can not use \LaTeX's \cs{@markright}, but a local version of
% it.
%    \begin{macrocode}
%<*body>
\DeclareRobustCommand*{\marksboth}[3]{%
  \begingroup
    \let\label\relax \let\index\relax \let\glossary\relax
    \ExecuteDoHook{marks/relax}%
    \unrestored@protected@edef\@themark{{#2}{#3}}%
    \global\expandafter\let\csname @themarks\the#1\endcsname\@themark
    \@temptokena \expandafter{\@themark}%
    \marks#1{\the\@temptokena}%
  \endgroup
  \if@nobreak\ifvmode\nobreak\fi\fi
}
\def\unrestored@protected@edef{%
   \let\protect\@unexpandable@protect
   \edef
}
\DeclareRobustCommand*{\marksright}[2]{%
  \begingroup
    \let\label\relax \let\index\relax \let\glossary\relax
    \ExecuteDoHook{marks/relax}%
    \@ifundefined{@themarks\the#1}{%
      \def\@themark{{}{}}%
    }{%
      \expandafter\let\expandafter\@themark\csname @themarks\the#1\endcsname
    }%
    \expandafter\@marksright\@themark {#2}%
    \global\expandafter\let\csname @themarks\the#1\endcsname\@themark
    \@temptokena \expandafter{\@themark}%
    \marks#1{\the\@temptokena}%
  \endgroup
  \if@nobreak\ifvmode\nobreak\fi\fi
}
\newcommand*{\@marksright}[3]{%
  \@temptokena {#1}%
  \unrestored@protected@edef\@themark{{\the\@temptokena}{#3}}%
}
%</body>
%    \end{macrocode}
% \end{macro}
% \end{macro}
% \end{macro}
%
% Note, we can use \cs{ExecuteDoHook} to make it easy to extend relaxing some
% commands only, if we habe \Package{scrbase} loaded. So do it while
% initialization:
%    \begin{macrocode}
%<*init>
\RequirePackage{scrbase}
%</init>
%    \end{macrocode}
% 
% \begin{macro}{\leftfirstmarks}
% \begin{macro}{\lefttopmarks}
% \begin{macro}{\leftbotmarks}
% \begin{macro}{\leftmarks}
% \begin{macro}{\leftfirstmarks}
% \begin{macro}{\lefttopmarks}
% \begin{macro}{\leftbotmarks}
% \begin{macro}{\leftmarks}
% These are similar to \cs{leftmark} resp. \cs{rightmark} but using
% \cs{firstmarks}, \cs{topmarks} or \cs{botmarks} and therefore need one
% argument.
%    \begin{macrocode}
%<*body>
\newcommand*{\leftfirstmarks}[1]{%
  \expandafter\@leftmark\firstmarks#1\@empty\@empty
}
\newcommand*{\lefttopmarks}[1]{%
  \expandafter\@leftmark\topmarks#1\@empty\@empty
}
\newcommand*{\leftbotmarks}[1]{%
  \expandafter\@leftmark\botmarks#1\@empty\@empty
}
\newcommand*{\leftmarks}{\leftbotmarks}
\newcommand*{\rightfirstmarks}[1]{%
  \expandafter\@rightmark\firstmarks#1\@empty\@empty
}
\newcommand*{\righttopmarks}[1]{%
  \expandafter\@rightmark\topmarks#1\@empty\@empty
}
\newcommand*{\rightbotmarks}[1]{%
  \expandafter\@rightmark\botmarks#1\@empty\@empty
}
\newcommand*{\rightmarks}{\rightfirstmarks}
%</body>
%    \end{macrocode}
% \end{macro}
% \end{macro}
% \end{macro}
% \end{macro}
% \end{macro}
% \end{macro}
% \end{macro}
% \end{macro}
%
% \subsection{Handling Marks inside \Package{multicol}'s \texttt{multicols}
% Environment}
% \label{sec:komamarks.multicols}
%
% Package \Package{multicol} needs special treatment for marks and we have to
% patch it to do the same for the \eTeX{} marks.
%
% Not yet implemented.
%
%    \begin{macrocode}
%<*init>
\RequirePackage{scrlfile}
%</init>
%<*body>
\AfterPackage*{multicol}{%
  \PackageWarning{komamarks}{%
    komamarks does not yet support multicols
  }%
}
%</body>
%    \end{macrocode}
%
%
% \subsection{Multimarks}
% \label{sec:komamarks.multimarks}
%
% Usually \LaTeX's marks consist of a left and a right mark. But what about
% having marks als sequence of items? In this case we need to setup the
% n-th mark item in such a sequence and to get the n-th mark item in such a
% sequence. Here it would be nice to have \Package{expl3}, i.\,e.,
% \File{l3seq} instead of only \LaTeXe's \cs{@tfor}.
%
% Not yet implemented completely and therefore currently not usable.
%
% \begin{macro}{\marksall}
% Like \cs{marksboth} but instead of two mark argument (for the left and the
% right mark) it has only one mark argument, which is a token list with a
% token for each mark.
%    \begin{macrocode}
%<*body>
\DeclareRobustCommand*{\marksall}[2]{%
  \begingroup
    \let\label\relax \let\index\relax \let\glossary\relax
    \ExecuteDoHook{marks/relax}%
    \unrestored@protected@edef\@themark{#2}%
    \global\expandafter\let\csname @themarks\the#1\endcsname\@themark
    \@temptokena \expandafter{\@themark}%
    \marks#1{\the\@temptokena}%
  \endgroup
  \if@nobreak\ifvmode\nobreak\fi\fi
}
%</body>
%    \end{macrocode}
% \end{macro}
%
% \begin{macro}{\marksnth}
% Like \cs{marksright} but instead of changing the right (second) mark it
% changes the n-th mark.
%    \begin{macrocode}
%<*body>
\DeclareRobustCommand*{\marksnth}[3]{%
  \begingroup
    \let\label\relax \let\index\relax \let\glossary\relax
    \ExecuteDoHook{marks/relax}%
    \@tempcnta=\z@
    \let\@themark\@empty
    \@ifundefined{@themarks\the#1}{}{%
      \expandafter\expandafter\expandafter\@tfor
      \expandafter\expandafter\expandafter\reserved@a
      \expandafter\expandafter\expandafter:%
      \expandafter\expandafter\expandafter=%
      \csname @themarks\the#1\endcsname\do {%
        \advance\@tempcnta \@ne
        \ifnum \number #2=\@tempcnta
          \unrestored@protected@edef\@themark{\@themark{#3}}%
        \else
          \unrestored@protected@edef\@themark{\@themark{\reserved@a}}%
        \fi
      }%
    }%
    \@whilenum \number #2>\@tempcnta\do {%
      \advance\@tempcnta \@ne
      \ifnum \number #2=\@tempcnta
        \unrestored@protected@edef\@themark{\@themark{#3}}%
      \else
        \unrestored@protected@edef\@themark{\@themark{}}%
      \fi
    }%
    \global\expandafter\let\csname @themarks\the#1\endcsname\@themark
    \@temptokena \expandafter{\@themark}%
    \marks#1{\the\@temptokena}%
  \endgroup
  \if@nobreak\ifvmode\nobreak\fi\fi
}
%</body>
%    \end{macrocode}
% \end{macro}
%
% \begin{macro}{\marksnthplus}
% Like \cs{marksnth} but with one more argument, that is used till the end of
% the mark sequence.
%    \begin{macrocode}
%<*body>
\DeclareRobustCommand*{\marksnthplus}[4]{%
  \begingroup
    \let\label\relax \let\index\relax \let\glossary\relax
    \ExecuteDoHook{marks/relax}%
    \@tempcnta=\z@
    \let\@themark\@empty
    \@ifundefined{@themarks\the#1}{}{%
      \expandafter\expandafter\expandafter\@tfor
      \expandafter\expandafter\expandafter\reserved@a
      \expandafter\expandafter\expandafter:%
      \expandafter\expandafter\expandafter=%
      \csname @themarks\the#1\endcsname\do {%
        \advance\@tempcnta \@ne
        \ifnum \number #2=\@tempcnta
          \unrestored@protected@edef\@themark{\@themark{#3}}%
        \else
          \ifnum \number #2>\@tempcnta
            \unrestored@protected@edef\@themark{\@themark{\reserved@a}}%
          \else
            \unrestored@protected@edef\@themark{\@themark{#4}}%
          \fi
        \fi
      }%
    }%
    \@whilenum \number #2>\@tempcnta\do {%
      \advance\@tempcnta \@ne
      \ifnum \number #2=\@tempcnta
        \unrestored@protected@edef\@themark{\@themark{#3}}%
      \else
        \unrestored@protected@edef\@themark{\@themark{}}%
      \fi
    }%
    \global\expandafter\let\csname @themarks\the#1\endcsname\@themark
    \@temptokena \expandafter{\@themark}%
    \marks#1{\the\@temptokena}%
  \endgroup
  \if@nobreak\ifvmode\nobreak\fi\fi
}
%</body>
%    \end{macrocode}
% \end{macro}
%
% \begin{macro}{\marksnthcleartail}
% Like \cs{marksnth} but removes all items above the n-th.
%    \begin{macrocode}
%<*body>
\DeclareRobustCommand*{\marksnthcleartail}[3]{%
  \begingroup
    \let\label\relax \let\index\relax \let\glossary\relax
    \ExecuteDoHook{marks/relax}%
    \@tempcnta=\z@
    \let\@themark\@empty
    \@ifundefined{@themarks\the#1}{}{%
      \expandafter\expandafter\expandafter\@tfor
      \expandafter\expandafter\expandafter\reserved@a
      \expandafter\expandafter\expandafter:%
      \expandafter\expandafter\expandafter=%
      \csname @themarks\the#1\endcsname\do {%
        \advance\@tempcnta \@ne
        \ifnum \number #2=\@tempcnta
          \unrestored@protected@edef\@themark{\@themark{#3}}%
        \else
          \ifnum \number #2>\@tempcnta
            \unrestored@protected@edef\@themark{\@themark{\reserved@a}}%
          \fi
        \fi
      }%
    }%
    \@whilenum \number #2>\@tempcnta\do {%
      \advance\@tempcnta \@ne
      \ifnum \number #2=\@tempcnta
        \unrestored@protected@edef\@themark{\@themark{#3}}%
      \else
        \unrestored@protected@edef\@themark{\@themark{}}%
      \fi
    }%
    \global\expandafter\let\csname @themarks\the#1\endcsname\@themark
    \@temptokena \expandafter{\@themark}%
    \marks#1{\the\@temptokena}%
  \endgroup
  \if@nobreak\ifvmode\nobreak\fi\fi
}
%</body>
%    \end{macrocode}
% \end{macro}
%
% \begin{macro}{\nthmarknotdefined}
% This macro usually expands to nothing, if the n-th mark is not yet
% defined.
%    \begin{macrocode}
%<*body>
\newcommand*{\nthmarknotdefined}{}
%</body>
%    \end{macrocode}
% \end{macro}
%
% \begin{macro}{\nthtopmarks}
% \begin{macro}{\nthfirstmarks}
% \begin{macro}{\nthbotmarks}
% \begin{macro}{\@nthofm}
% Like \cs{leftfirstmarks} or \cs{rightfirstmarks} etc. but not using
% \cs{@leftmark} or \cs{@rightmark} that is \cs{@firstoftwo} or
% \cs{@secondoftwo} but \cs{@nthofm}.
%    \begin{macrocode}
%<*body>
\newcommand*{\nthtopmarks}[2]{%
  \expandafter\@nthofm\expandafter{\topmarks#1}{#2}%
}
\newcommand*{\nthfirstmarks}[2]{%
  \expandafter\@nthofm\expandafter{\firstmarks#1}{#2}%
}
\newcommand*{\nthbotmarks}[2]{%
  \expandafter\@nthofm\expandafter{\botmarks#1}{#2}%
}
\newcommand{\@nthofm}[2]{%
  \begingroup
    \@tempcnta=\z@
    \let\reserved@a\nthmarknotdefined
    \@tfor\reserved@b:=#1\do{%
      \advance\@tempcnta\@ne
      \ifnum \number #2=\@tempcnta
        \let\reserved@a\reserved@b
      \fi
    }%
  \expandafter\def\expandafter\reserved@a\expandafter{\expandafter\endgroup
    \reserved@a}%
  \reserved@a
}
%</body>
%    \end{macrocode}
% \end{macro}
% \end{macro}
% \end{macro}
% \end{macro}
% 
%
%\iffalse%^^A meta-comment
%</package>
%\fi%^^A meta-comment
% \Finale
%
\endinput
%
% end of file `komamarks.dtx'

%%% Local Variables:
%%% mode: doctex
%%% ispell-local-dictionary: "en_GB"
%%% flyspell-mode: t
%%% TeX-master: t
%%% End:
