% ======================================================================
% scrwfile.tex
% Copyright (c) Markus Kohm, 2010-2019
%
% This file is part of the LaTeX2e KOMA-Script bundle.
%
% This work may be distributed and/or modified under the conditions of
% the LaTeX Project Public License, version 1.3c of the license.
% The latest version of this license is in
%   http://www.latex-project.org/lppl.txt
% and version 1.3c or later is part of all distributions of LaTeX 
% version 2005/12/01 or later and of this work.
%
% This work has the LPPL maintenance status "author-maintained".
%
% The Current Maintainer and author of this work is Markus Kohm.
%
% This work consists of all files listed in manifest.txt.
% ----------------------------------------------------------------------
% scrwfile.tex
% Copyright (c) Markus Kohm, 2010-2019
%
% Dieses Werk darf nach den Bedingungen der LaTeX Project Public Lizenz,
% Version 1.3c, verteilt und/oder veraendert werden.
% Die neuste Version dieser Lizenz ist
%   http://www.latex-project.org/lppl.txt
% und Version 1.3c ist Teil aller Verteilungen von LaTeX
% Version 2005/12/01 oder spaeter und dieses Werks.
%
% Dieses Werk hat den LPPL-Verwaltungs-Status "author-maintained"
% (allein durch den Autor verwaltet).
%
% Der Aktuelle Verwalter und Autor dieses Werkes ist Markus Kohm.
% 
% Dieses Werk besteht aus den in manifest.txt aufgefuehrten Dateien.
% ======================================================================
%
% Chapter about scrwfile of the KOMA-Script guide
% Maintained by Markus Kohm
%
% ----------------------------------------------------------------------
%
% Kapitel ueber scrwfile in der KOMA-Script-Anleitung
% Verwaltet von Markus Kohm
%
% ============================================================================

\KOMAProvidesFile{scrwfile.tex}%
                 [$Date$
                  KOMA-Script guide (chapter: scrwfile)]

% Date of the translated German file: 2019-11-18

\translator{Markus Kohm\and Jana Schubert\and Karl Hagen}

\chapter{Economising and Replacing Files with \Package{scrwfile}}
\labelbase{scrwfile}
\BeginIndexGroup
\BeginIndex{Package}{scrwfile}

One of the problems not solved by the introduction of \eTeX{} is the fact that
\TeX{} can support only 18 open write handles. This number seems quite large
at first, but many of these handles are already reserved. \TeX{} itself uses
handle 0 for the log file. \LaTeX{} needs handle 1 for \Macro{@mainaux},
handle 2 for \Macro{@partaux}, one handle for \Macro{tableofcontents}, one
handle for \Macro{listoffigures}, one handle for \Macro{listoftables}, and one
handle for \Macro{makeindex}. Every other such list generates another handle,
and packages like \Package{hyperref} or \Package{minitoc} require write
handles too.

The bottom line is that eventually you may get the following error message:
\begin{lstcode}
  ! No room for a new \write .
  \ch@ck ...\else \errmessage {No room for a new #3}
                                                    \fi
\end{lstcode}

For some time, the simplest solution to this problem has been to use
\mbox{Lua\LaTeX} instead of \mbox{pdf\LaTeX} or \XeLaTeX. This eliminates the
restriction, and the maximum number of files you can have open for writing is
then limited only by the operating system. In reality, you usually so not need
to worry about it any more.

The fact that \LaTeX{} always opens a new file for writing every table of
contents, list of figures, list of tables, etc. has another disadvantage. Such
lists are not only output by their respective commands, they also could not be
output a second time because the associated auxiliary file\footnote{The term
  \emph{auxiliary file} here refers not to the main \File{aux} file but to the
  other internal files used indirectly via the \File{aux} file, e.\,g. the
  \File{toc} file, the \File{lof} file, or the \File{lot} file.} is empty
after the respective command until the end of the document.

The \Package{scrwfile} package makes a fundamental change to the \LaTeX{}
kernel, which can solve both problems not only for \mbox{Lua\LaTeX} but also
for \mbox{pdf\LaTeX} or \XeLaTeX.

\section{Fundamental Changes to the \LaTeX{} Kernel}
\seclabel{kernelpatches}

\LaTeX{} classes use the \LaTeX{} kernel command
\Macro{@starttoc}\IndexCmd{@starttoc} to allocate a new file handle, such as
for \Macro{tableofcontents} or \Macro{listoffigures}. This command not only
loads the associated auxiliary file but also reopens it for writing. If
entries to these lists are added using \Macro{addcontentsline}, however, the
system does not write directly to these auxiliary files. Instead, \LaTeX{}
writes \Macro{@writefile}\IndexCmd{@writefile} commands to the \File{aux}
file. Only while reading the \File{aux} file at the end of the document do
those \Macro{@writefile} commands become actual write operations in the
auxiliary files. Additionally, \LaTeX{} does not close the auxiliary files
explicitly. Instead it relies on \TeX{} to close all open files at the end.

This procedure ensures that the auxiliary files are only written to within
\Macro{end}\PParameter{document}, but they remain open throughout the entire
\LaTeX{} run. \Package{scrwfile} takes is starting point here, by redefining
\Macro{@starttoc} and \Macro{@writefile}.

Of course\textnote{Attention!} changes to the \LaTeX{} kernel always have the
potential to cause incompatibilities with other packages. Those primarily
affected may be those which also redefine \Macro{@starttoc} or
\Macro{@writefile}. In some cases, it may help to change the order the
packages are loaded. If you encounter such a problem, please contact the
\KOMAScript{} author.

\section{The Single-File Method}
\seclabel{singlefilefeature}

As soon as the package is loaded with
\begin{lstcode}
  \usepackage{scrwfile}
\end{lstcode}
\Package{scrwfile} redefines \Macro{@starttoc}\IndexCmd{@starttoc} so that it
no longer allocates a write handle or opens a file for writing.
\Macro{@writefile} is redefined so that immediately before closing the
\File{aux} file in \Macro{end}\PParameter{document}, it writes not to the
usual auxiliary files but to a single new file with extension \File{wrt}.
After reading the \File{aux} file, this \File{wrt} file will be processed once
for each of the auxiliary files that \Macro{@writefile} writes information to.
However these auxiliary files do not all have to be open at the same time.
Instead, only one is open at a time and is explicitly closed afterwards. Since
\LaTeX{} reuses an internal write file, \Package{scrwfile} doesn't need its
own write handle for this type of table of contents or list of floating
environments.

Because of this behaviour, even if you have only one table of contents, once
you load \Package{scrwfile} you will have access to a write file handle for
bibliographies, indexes, glossaries, and similar lists that do not use
\Macro{@starttoc}. Additionally, you can create any number of tables of
contents and other lists that use \Macro{@starttoc}\IndexCmd{@starttoc}.

\section{The File Cloning Method}
\seclabel{clonefilefeature}

Since \Macro{@writefile}\IndexCmd{@writefile} has already been modified for
the single-file method described in the previous system so that it no longer
writes directly to the corresponding auxiliary file, a further possibility
suggests itself. When copying the \Macro{@writefile} statements into the
\File{wrt} file, you can also copy them to other destinations.

\begin{Declaration}
  \Macro{TOCclone}%
  \OParameter{list heading}\Parameter{source extension}%
  \Parameter{destination extension}
\end{Declaration}
This cloning of file entries copies entire tables of contents or other lists.
For this, you only need to specify the extension of the auxiliary file whose
entries should be copied and the extension of a destination file. The entries
are then copied there. Of course, you can also write additional entries to
this cloned file.

You can manage the \PName{destination extention} using
\hyperref[cha:tocbasic]{\Package{tocbasic}}%
\important{\hyperref[cha:tocbasic]{\Package{tocbasic}}} (see
\autoref{cha:tocbasic}). If such a file is already under the control of
\hyperref[cha:tocbasic]{\Package{tocbasic}}, a warning will be issued.
Otherwise, a new list for this extension will be created using
\hyperref[cha:tocbasic]{\Package{tocbasic}}. You can set the heading for this
list with the optional argument \PName{list heading}.

You can then output this new content list, for example, with the command
\Macro{listof\PName{destination extension}}. The content-list
attributes\important{\hyperref[cha:tocbasic]{\Package{tocbasic}}}
\PValue{leveldown}, \PValue{numbered}, \PValue{onecolumn}, and \PValue{totoc}
(see \DescRef{tocbasic.cmd.setuptoc} in \autoref{sec:tocbasic.toc},
\DescPageRef{tocbasic.cmd.setuptoc}) are automatically copied to the
destination list if they were already set in the source list. The
\PValue{nobabel} attribute is always set for cloned content lists, because the
language-selection commands in the source content list are already copied
anyway.

\begin{Example}
  Suppose you want a short table of contents with only the chapter level in
  addition to the normal the table of contents:
\begin{lstcode}
  \usepackage{scrwfile}
  \TOCclone[Summary Contents]{toc}{stoc}
\end{lstcode}
  This creates a new table of contents with the heading ``Summary Contents''.
  The new table of contents uses an auxiliary file with the extension
  \File{stoc}. All entries to the auxiliary file with extension \File{toc}
  will also be copied to this new auxiliary file.

  In order to have the new short table of contents display only the chapter
  entries, we use:
\begin{lstcode}
  \addtocontents{stoc}{\protect\value{tocdepth}=0}
\end{lstcode}
  Although\textnote{Attention!} normally you cannot write to an auxiliary file
  before \Macro{begin}\PParameter{document}, the code above works in the
  preamble after loading \Package{scrwfile}. Owing to the unconventional way
  of changing the \DescRef{maincls.counter.tocdepth} counter within the TOC
  file, this change only applies to this content list.

  Later in the document, we then output the content list with the file
  extension \File{stoc} with:
\begin{lstcode}[moretexcs={listofsttoc}]
  \listofstoc
\end{lstcode}
  and this shows only the parts and chapters of the document.

  Things become a bit more difficult if the summary contents are to be
  listed in the table of contents. This would seem to be possible with
\begin{lstcode}
  \addtocontents{toc}{%
    \protect\addxcontentsline
      {stoc}{chapter}{\protect\contentsname}%
  }
\end{lstcode}
However, since all entries in \File{toc} are also copied to \File{stoc}, this
entry would also be copied from the summary contents. So we cannot generate
the entry from the content list. Because we use the
\Package{tocbasic}\important{\Package{tocbasic}} package, we can use the
following:
\phantomsection\xmpllabel{cmd.BeforeStartingTOC}
\begin{lstcode}
  \BeforeStartingTOC[toc]{%
    \addcontentslinedefault{stoc}{chapter}
                    {\protect\contentsname}%
  }
\end{lstcode}
Of course, this assumes that the \File{toc} file is under the control
of the \Package{tocbasic} package, which is indeed the case for all
\KOMAScript{} classes.  See \autoref{sec:tocbasic.toc} on
\DescPageRef{tocbasic.cmd.BeforeStartingTOC} for more information about
\DescRef{tocbasic.cmd.BeforeStartingTOC}.%
\end{Example}
Incidentally, the \DescRef{tocbasic.cmd.addxcontentsline} command used in the
examples is also documented in \autoref{cha:tocbasic},
\DescPageRef{tocbasic.cmd.addxcontentsline}.%
\EndIndexGroup


\section{Note on the State of Development}
\seclabel{draft}

Although this package has already been tested by many users and is often in
production use, its development is still ongoing. Therefore, it is
theoretically possible that there might be changes, especially to the internal
functionality. It is likely that the package will be extended in the future.
Some code for such extensions is already in the package. However, as there are
no user commands that make use of these features, they are currently
undocumented.

\section{Known Package Incompatibilities}
\seclabel{incompatible}

As mentioned in \autoref{sec:scrwfile.kernelpatches}, \Package{scrwfile}
redefines some commands of the \LaTeX{} kernel. This happens not only while
loading the package, but indeed at various times while the document is
processed, for example just before reading the \File{aux} file.
This\textnote{Attention!} results in incompatibility with packages that also
redefine these commands at run time.

The \Package{titletoc}\important{Package{titletoc}}\IndexPackage{titletoc}
package is an example for such an incompatibility. That package redefines
\Macro{@writefile} under some conditions at run time. If you use both
\Package{scrwfile} and \Package{titletoc}, there is no warranty for the
correct behaviour of either one. This is neither an error of
\Package{titletoc} nor of \Package{scrwfile}.%
%
\EndIndexGroup

%%% Local Variables:
%%% mode: latex
%%% mode: flyspell
%%% coding: us-ascii
%%% ispell-local-dictionary: "en_GB"
%%% TeX-master: "../guide"
%%% End: 
