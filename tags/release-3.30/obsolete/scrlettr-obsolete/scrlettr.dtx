% \CheckSum{1888}
% \iffalse meta-comment
% ======================================================================
% scrlettr.dtx
% Copyright (c) Axel Kielhorn and Markus Kohm, 1995-2020
%
% This file is part of the LaTeX2e KOMA-Script bundle.
%
% This work may be distributed and/or modified under the conditions of
% the LaTeX Project Public License, version 1.3c of the license.
% The latest version of this license is in
%   http://www.latex-project.org/lppl.txt
% and version 1.3c or later is part of all distributions of LaTeX 
% version 2005/12/01 or later.
%
% This work has the LPPL maintenance "not maintained" and is deprecated!
% It has been replaced by KOMA-Script class scrlttr2.
%
% The authors of this work are Axel Kielhorn and Markus Kohm.
%
% This work consists of the files `scrlettr.dtx', `scrlettr.ins', and 
% `README'.
% ----------------------------------------------------------------------
% scrlettr.dtx
% Copyright (c) Axel Kielhorn and Markus Kohm, 1995-2020
%
% Dieses Werk darf nach den Bedingungen der LaTeX Project Public Lizenz,
% Version 1.3c, verteilt und/oder veraendert werden.
% Die neuste Version dieser Lizenz ist
%   http://www.latex-project.org/lppl.txt
% und Version 1.3c ist Teil aller Verteilungen von LaTeX
% Version 2005/12/01 oder spaeter.
%
% Dieses Werk hat den LPPL-Verwaltungs-Status "not maintained"
% (nicht verwaltet), ist veraltet und wurde durch die KOMA-Script-Klasse
% scrlttr2 ersetzt.
%
% Die Autoren dieses Werkes sind Axel Kielhorn und Markus Kohm.
% 
% Dieses Werk besteht aus den Dateien `scrlettr.dtx', `scrlettr.ins' und 
% `README'.
% ======================================================================
% \fi
%
% \CharacterTable
%  {Upper-case    \A\B\C\D\E\F\G\H\I\J\K\L\M\N\O\P\Q\R\S\T\U\V\W\X\Y\Z
%   Lower-case    \a\b\c\d\e\f\g\h\i\j\k\l\m\n\o\p\q\r\s\t\u\v\w\x\y\z
%   Digits        \0\1\2\3\4\5\6\7\8\9
%   Exclamation   \!     Double quote  \"     Hash (number) \#
%   Dollar        \$     Percent       \%     Ampersand     \&
%   Acute accent  \'     Left paren    \(     Right paren   \)
%   Asterisk      \*     Plus          \+     Comma         \,
%   Minus         \-     Point         \.     Solidus       \/
%   Colon         \:     Semicolon     \;     Less than     \<
%   Equals        \=     Greater than  \>     Question mark \?
%   Commercial at \@     Left bracket  \[     Backslash     \\
%   Right bracket \]     Circumflex    \^     Underscore    \_
%   Grave accent  \`     Left brace    \{     Vertical bar  \|
%   Right brace   \}     Tilde         \~}
%
% \iffalse
%%% From: scrlettr.dtx
%<*dtx>
            \ProvidesFile{scrlettr.dtx}
%</dtx>
%<scrlettr>\NeedsTeXFormat{LaTeX2e}[1995/12/01]
%<driver>\ProvidesFile{scrlettr.drv}
%<scrlettr>\ProvidesClass{scrlettr}
%<*scrlettr|dtx|driver>
              [2002/05/24 v2.6e
%<scrlettr>               OBSOLETE
               LaTeX2e KOMA
%</scrlettr|dtx|driver>
%<scrlettr>               document class]
%
%<*driver|dtx>
              Script bundle]
%</driver|dtx>
%<*driver>
\documentclass{ltxdoc}
\IfFileExists{typearea.sty}{%
  \usepackage[a4paper,BCOR15mm,DIV12]{typearea}%
}{}
\usepackage[german]{babel}
\DoNotIndex{\.,\@@par,\@M,\@abstrtfalse,\@abstrttrue,\@addtoreset}
\DoNotIndex{\@afieldfalse,\@afieldtrue,\@afterheading}
\DoNotIndex{\@afterindentfalse,\@arabic,\@badmath,\@beginparpenalty}
\DoNotIndex{\@biglocfalse,\@bigloctrue,\@bsphack}
\DoNotIndex{\@car,\@cdr,\@centercr,\@cite,\@dblfloat,\@dotsep}
\DoNotIndex{\@dottedtocline,\@empty,\@endparpenalty,\@endpart,\@esphack}
\DoNotIndex{\@finclfalse,\@fincltrue,\@float,\@fontswitch,\@foldfalse}
\DoNotIndex{\@foldtrue,\@fslfalse,\@fsltrue,\@gobbletwo}
\DoNotIndex{\@hangfrom,\@highpenalty,\@hinclfalse,\@hincltrue,\@hslfalse}
\DoNotIndex{\@hsltrue,\@idxitem,\@ifnextchar,\@ifundefined,\@ifstar}
\DoNotIndex{\@itempenalty}
\DoNotIndex{\@latex@warning,\@m,\@mainmatterfalse,\@mainmattertrue}
\DoNotIndex{\@medpenalty,\@minus,\@mkboth,\@mparswitchfalse,\@mparswitchtrue}
\DoNotIndex{\@ne,\@nil,\@nobreakcr,\@nobreakfalse,\@nobreaktrue}
\DoNotIndex{\@nobreakvspace,\@nobreakvspacex,\@noitemerr,\@nomath,\@normalcr}
\DoNotIndex{\@openbibfalse,\@openbibtrue,\@openrightfalse,\@openrighttrue}
\DoNotIndex{\@plus,\@processto,\@reffalse,\@reftrue,\@restonecolfalse}
\DoNotIndex{\@restonecoltrue}
\DoNotIndex{\@subjfalse,\@subjtrue}
\DoNotIndex{\@tempa,\@tempboxa,\@tempdima,\@tempswafalse,\@tempswatrue}
\DoNotIndex{\@titlepagefalse,\@titlepagetrue,\@tocrmarg,\@topnewpage}
\DoNotIndex{\@topnum,\@twocolumnfalse,\@twocolumntrue,\@twosidefalse}
\DoNotIndex{\@twosidetrue}
\DoNotIndex{\@whiledim,\@whilenum,\@xnewline,\@xproc,\\,\ }
\DoNotIndex{\addcontentsline,\addpenalty,\addtocontents,\addtolength}
\DoNotIndex{\addvspace,\advance,\Alph,\alph,\arabic,\Ask,\AtBeginDocument}
\DoNotIndex{\begin,\begingroup,\bfseries,\bgroup,\box,\bullet}
\DoNotIndex{\c@figure,\c@page,\c@secnumdepth,\c@table,\c@tocdepth}
\DoNotIndex{\cal,\cdot,\centering,\changes,\ClassWarningNoLine}
\DoNotIndex{\cleardoublepage,\clearpage}
\DoNotIndex{\cmd,\col@number,\CurrentOption,\CodelineIndex,\csname}
\DoNotIndex{\day,\dblfloatpagefraction,\dbltopfraction,\Decisionfalse}
\DoNotIndex{\Decisiontrue,\DeclareOldFontCommand}
\DoNotIndex{\DeclareOption,\def,\defpar,\DisableCrossrefs}
\DoNotIndex{\divide,\documentclass,\DoNotIndex,\dotfill}
\DoNotIndex{\iden,\ifdim,\else,\fi,\egroup,\empty,\em,\EnableCrossrefs,\end}
\DoNotIndex{\end@dblfloat,\endcsname,\endletter}
\DoNotIndex{\end@float,\endgroup,\endlist,\endquotation,\endtitlepage}
\DoNotIndex{\everypar,\ExecuteOptions,\expandafter}
\DoNotIndex{\fboxrule,\fboxsep,\fontsize,\frenchspacing}
\DoNotIndex{\gdef,\global}
\DoNotIndex{\hangindent,\hbox,\hfil,\hfill,\hrule,\hsize,\hskip,\hspace,\hss}
\DoNotIndex{\if@tempswa,\ifcase,\or,\fi,\fi}
\DoNotIndex{\ifnum,\ifodd,\ifvmode,\ifx,\fi,\fi,\fi,\fi}
\DoNotIndex{\ignorespaces,\input,\InputIfFileExists,\item,\itshape,\j,\ja}
\DoNotIndex{\kern,\LARGE,\Large,\leavevmode,\leftmark,\leftskip,\let}
\DoNotIndex{\lineskip,\list,\long}
\DoNotIndex{\m@ne,\m@th,\marginpar,\marginparpush,\markboth,\markright}
\DoNotIndex{\mathbf,\mathcal}
\DoNotIndex{\mathit,\mathnormal,\mathrm,\mathsf,\mathtt,\MessageBreak,\month}
\DoNotIndex{\newblock,\newcommand,\newcount,\newcounter,\newdimen}
\DoNotIndex{\newenvironment,\newlength,\newpage,\nobreak,\noindent}
\DoNotIndex{\normalfont,\normallineskip,\normalsize,\null,\number}
\DoNotIndex{\numberline}
\DoNotIndex{\OldMakeindex,\OnlyDescription,\overfullrule}
\DoNotIndex{\p@,\PackageError,\PackageInfo,\PackageWarningNoLine}
\DoNotIndex{\pagenumbering,\pagestyle,\par,\paragraph,\parbox}
\DoNotIndex{\PassOptionsToPackage,\pcal,\penalty,\pmit,\PrintChanges}
\DoNotIndex{\PrintIndex,\ProcessOptions,\protect,\providecommand}
\DoNotIndex{\ProvidesClass}
\DoNotIndex{\raggedbottom,\raggedleft,\raggedright,\refstepcounter,\relax}
\DoNotIndex{\renewcommand,\RequirePackage,\reset@font,\reversemarginpar}
\DoNotIndex{\rightmargin,\rightmark,\rightskip,\rmfamily,\Roman,\roman,\rule}
\DoNotIndex{\sc@septext,\sc@temp,\sc@@temp,\scshape,\secdef,\setbox}
\DoNotIndex{\setcounter,\setlength}
\DoNotIndex{\settowidth,\sfcode,\sffamily,\skip,\sloppy,\slshape,\space}
\DoNotIndex{\string,\strip,\strut,\subjectoff,\subjecton}
\DoNotIndex{\ta@temp,\the,\thispagestyle,\triangleright,\ttfamily,\twocolumn}
\DoNotIndex{\typein,\typeout}
\DoNotIndex{\undefined,\underline,\unhbox,\usecounter,\usepackage}
\DoNotIndex{\vadjust,\vfil,\vfill,\vspace}
\DoNotIndex{\wd,\xdef,\y,\year,\yes,\z@}
\CodelineIndex
% Kopiert aus scrguide2.tex
\newcommand{\ExampleName}{Beispiel}
\newcommand{\ClassName}{Klasse}
\newcommand{\PackageName}{Paket}
\newcommand{\EnvironmentName}{Umgebung}
\newcommand{\OptionName}{Option}
\newcommand{\MacroName}{Befehl}
\newcommand{\CounterName}{Z\"ahler}
\newcommand{\CounterSortName}{Zaehler}
\newcommand{\LengthName}{L\"ange}
\newcommand{\LengthSortName}{Laenge}
\newcommand{\PagestyleName}{Seitenstil}
\newcommand{\StyleName}{Stil}
\newcommand{\FileName}{Datei}
\newcommand{\TitleText}{Das\ \KOMAScript\ Paket}
\newcommand{\ManualName}{Anleitung}
\newcommand{\ManualFromText}{Autoren der Anleitung:}
\newcommand{\NoteName}{Notizen}
\DeclareRobustCommand*{\Class}[1]{\textsf{#1}}
\DeclareRobustCommand*{\Package}[1]{\textsf{#1}}
\DeclareRobustCommand*{\File}[1]{\texttt{#1}}
\DeclareRobustCommand{\Script}{\textsc{Script}}
\DeclareRobustCommand{\ScriptII}{\textsc{Script-2}}
\ifx\KOMAScript\undefined%
  \DeclareRobustCommand{\KOMAScript}{\textsf{K\kern.05em O\kern.05em%
      M\kern.05em A\kern.1em-\kern.1em Script}}
\fi
\makeatletter
\DeclareRobustCommand{\BibTeX}{B\kern-.05em%
  \hbox{$\m@th$%
    \csname S@\f@size\endcsname \fontsize\sf@size\z@
    \math@fontsfalse\selectfont
    I\kern-.025emB}%
  \kern-.08em%
  \-\TeX%
}
\makeatother
\newcommand*{\Var}[1]{\ensuremath{\mathit{#1}}}
\newcommand*{\Const}[1]{\ensuremath{\mathrm{#1}}}
\newcommand*{\Unit}[1]{\ensuremath{\,\mathrm{#1}}}
\DeclareRobustCommand*{\Macro}[1]{\mbox{\texttt{\char`\\#1}}}
\DeclareRobustCommand*{\Option}[1]{\mbox{\texttt{#1}}}
\DeclareRobustCommand*{\Environment}[1]{\mbox{\texttt{#1}}}
\DeclareRobustCommand*{\Counter}[1]{\mbox{\texttt{#1}}}
\DeclareRobustCommand*{\Length}[1]{\mbox{\texttt{\char`\\#1}}}
\DeclareRobustCommand*{\EMail}[1]{\textless #1\textgreater}
\DeclareRobustCommand*{\TextEMail}[1]{{\small\EMail{#1}}}
\newenvironment{Example}{%
  \begin{labeling}{{\sectfont\ExampleName:\ }}
  \item[{\sectfont\ExampleName:\ }]}
  {\end{labeling}}
\newenvironment{Explain}{%
  \small\sffamily
}{\normalcolor\par}
\newcommand*{\sectfont}{\normalfont\normalcolor\bfseries}
\newenvironment{labeling}[2][]
  {\def\sc@septext{#1}
   \list{}{\settowidth{\labelwidth}{#2#1}
     \leftmargin\labelwidth \advance\leftmargin by \labelsep
     \let\makelabel\labelinglabel}}
  {\endlist}
\newcommand\labelinglabel[1]{#1\hfil\sc@septext}
\newenvironment{Declaration}%
    {\par\small\addvspace{2\baselineskip plus .5\baselineskip}%
     \vspace{-\baselineskip}%
     \noindent\hspace{-1em}%
     \begin{tabular}{|l|}\hline\ignorespaces}%
    {\\\hline\end{tabular}\nobreak\par\nobreak
     \vspace{1.5\baselineskip}\vspace{-\baselineskip}%
     \noindent\ignorespacesafterend}
\newcommand{\PName}[1]{\mbox{\textit{#1}}}% Parametername
\newcommand{\PValue}[1]{\texttt{#1}}% Parametername
\newcommand{\Parameter}[1]{% Parameter/Argument
  \texttt{\{}\PName{#1}\texttt{\}}}
\newcommand{\OParameter}[1]{\texttt{[%] Parameter/Argument optional
    }\PName{#1}\texttt{%[
    ]}}
\newcommand{\AParameter}[1]{\texttt{(%) Parameter/Argument alternative
    }\PName{#1}\texttt{%(
    )}}
\newcommand{\PParameter}[1]{\texttt{\{% Parameter/Argument as/als
    #1% part of commands / Befehlsbestandteil
    \}}}
\newcommand*{\Index}[2][indexrm]{\index{#2|#1}}
\newcommand*{\BeginIndex}[3][indexit]{\csname Index#2\endcsname[(%)
  #1]{#3}%
  \ignorespaces}
\newcommand*{\EndIndex}[3][indexit]{\csname Index#2\endcsname[%(
  )#1]{#3}}
\newcommand*{\IndexCmd}[2][indexrm]{%
  \Index[#1]{#2=\Macro{#2}}}
\newcommand*{\IndexEnv}[2][indexrm]{%
  \Index[#1]{\EnvironmentName>#2=\Environment{#2}}%
  \Index[#1]{#2=\Environment{#2} (\EnvironmentName)}}
\newcommand*{\IndexOption}[2][indexrm]{%
  \Index[#1]{\OptionName>#2=\Option{#2}}%
  \Index[#1]{#2=\Option{#2} (\OptionName)}}
\newcommand*{\IndexPackage}[2][indexrm]{%
  \Index[#1]{\PackageName>#2=\Package{#2}}%
  \Index[#1]{#2=\Package{#2} (\PackageName)}}
\newcommand*{\IndexClass}[2][indexrm]{%
  \Index[#1]{\ClassName>#2=\Class{#2}}%
  \Index[#1]{#2=\Class{#2} (\ClassName)}}
\newcommand*{\IndexFile}[2][indexrm]{%
  \Index[#1]{\FileName>#2=\File{#2}}%
  \Index[#1]{#2=\File{#2} (\FileName)}}
\newcommand*{\IndexCounter}[2][indexrm]{%
  \Index[#1]{\CounterSortName=\CounterName>#2=\Counter{#2}}%
  \Index[#1]{#2=\Counter{#2} (\CounterName)}}
\newcommand*{\IndexLength}[2][indexrm]{%
  \Index[#1]{\LengthSortName=\LengthName>#2=\Length{#2}}%
  \Index[#1]{#2=\Length{#2} (\LengthName)}}
\newcommand*{\IndexPagestyle}[2][indexrm]{%
  \Index[#1]{\PagestyleName>#2=\PValue{#2}}%
  \Index[#1]{#2=\PValue{#2} (\PagestyleName)}}
\newcommand*{\IndexFloatstyle}[2][indexrm]{%
  \Index[#1]{float-\StyleName=\emph{float}-\StyleName>#2=\PValue{#2}}%
  \Index[#1]{#2=\PValue{#2} (\emph{float}-\StyleName)}}
\newcommand*{\indexrm}[1]{\textrm{\hyperpage{#1}}}
\newcommand*{\indexit}[1]{\textit{\hyperpage{#1}}}
\newcommand*{\indexbf}[1]{\textbf{\hyperpage{#1}}}
\newcommand*{\indexsl}[1]{\textsl{\hyperpage{#1}}}
\newcommand*{\indexsf}[1]{\textsf{\hyperpage{#1}}}
\newcommand*{\indexsc}[1]{\textsc{\hyperpage{#1}}}
\providecommand*{\hyperpage}[1]{#1}
%
\begin{document}
 \DocInput{scrlettr.dtx}
\end{document}
%</driver>
% \fi
%
% \GetFileInfo{scrlettr.dtx}
% \RecordChanges
%
% \makeatletter
% \def\macro{\begingroup
%    \catcode`\\12
%    \MakePrivateLetters \m@cro@ 0}
% \def\environment{\begingroup
%    \catcode`\\12
%    \MakePrivateLetters \m@cro@ 1}
% \def\option{\begingroup
%    \catcode`\\12
%    \MakePrivateLetters \m@cro@ 2}
% \long\def\m@cro@#1#2{\endgroup \topsep\MacroTopsep \trivlist
%   \edef\saved@macroname{\string#2}%
%   \ifcase #1%
%     \edef\saved@@macroname{\expandafter\@gobble\saved@macroname}
%   \or
%     \edef\saved@@macroname{\expandafter\@gobble\saved@macroname}
%   \else
%     \let\saved@@macroname\saved@macroname
%   \fi
%   \def\makelabel##1{\llap{##1}}%
%   \if@inlabel
%     \let\@tempa\@empty \count@\macro@cnt
%     \loop \ifnum\count@>\z@
%       \edef\@tempa{\@tempa\hbox{\strut}}\advance\count@\m@ne \repeat
%     \edef\makelabel##1{\llap{\vtop to\baselineskip
%                                {\@tempa\hbox{##1}\vss}}}%
%     \advance \macro@cnt \@ne
%   \else  \macro@cnt\@ne  \fi
%   \edef\@tempa{\noexpand\item[%
%      \ifcase #1%
%        \noexpand\PrintMacroName
%      \or
%        \noexpand\PrintEnvName
%      \else
%        \noexpand\PrintOptionName
%      \fi
%      {\string#2}]}%
%   \@tempa
%   {\advance\c@CodelineNo\@ne
%    \ifcase #1%
%       \SpecialMainIndex{#2}\nobreak
%       \DoNotIndex{#2}%
%    \or
%       \SpecialMainEnvIndex{#2}\nobreak
%    \else
%       \SpecialMainOptionIndex{#2}\nobreak
%    \fi
%    }%
%   \ignorespaces}
% \let\endoption\endtrivlist
% \@ifundefined{PrintOptionName}
%    {\def\PrintOptionName#1{\strut \MacroFont #1\ }}{}
% \def\SpecialMainOptionIndex#1{\@bsphack
%     \special@index{#1\actualchar{\string\ttfamily\space#1}
%            (option)\encapchar main}%
%     \special@index{options:\levelchar{\string\ttfamily\space#1}\encapchar
%            main}\@esphack}
% \def\changes@#1#2#3{%
%   \protected@edef\@tempa{\noexpand\glossary{#1\levelchar
%                          \ifx\saved@macroname\@empty
%                            \space
%                            \actualchar
%                            \generalname
%                          \else
%                            \saved@@macroname
%                            \actualchar
%                            \string\verb\quotechar*\verbatimchar%
%                            \saved@macroname
%                            \verbatimchar
%                          \fi
%                          :\levelchar #3}}%
%   \@tempa\endgroup\@esphack}
% \makeatother
%
% \title{Die ehemalige Brief-Klasse aus der
%        \textsf{KOMA-Script}-Sammlung\thanks{Diese Datei
%        hat die Versionsnummer \fileversion, letzte "Anderung vom
%        \filedate.}}
% \author{Frank Neukam\and Markus Kohm}
% \date{\filedate}
% \maketitle
%
% \tableofcontents
%
% \changes{v2.0e}{1994/10/28}{Erste Version, die \texttt{docstrip} verwendet.}
% \changes{v2.3b}{1996/01/14}{Diverse \cs{newcommand} mit und ohne
%                             Parameter durch \cs{newcommand*} ersetzt.}
% \changes{v2.3b}{1996/01/14}{Diverse \cs{renewcommand} mit und ohne
%                             Parameter durch \cs{renewcommand*}
%                             ersetzt.}
% \changes{v2.6e}{2002/05/24}{\texttt{scraddr} ist in einer eigenen
%                             \texttt{dtx}-Datei zu finden.}
% \changes{v2.6e}{2002/05/24}{\texttt{dir.tex}, \texttt{phone.tex},
%   \texttt{addrconv.bst}, \texttt{birthday.bst}, \texttt{email.bst},
%   \texttt{addrconv.tex}, \texttt{birthday.tex}, \texttt{email.tex}
%   sind obsolet und gel\"oscht, stattdessen sollte das
%   \texttt{adrconv}-Paket von Axel Kielhorn verwendet werden.}
%
% \part{Anleitung}
%
% \section{Generelles}
%
% \subsection{Rechtliches}
% Es wird keinerlei Haftung "ubernommen f"ur irgendwelche Sch"aden,
% die aus der Benutzung der Programme und Dateien des hier
% beschriebenen Paketes folgen.
%
% \subsection{Das \textsf{KOMA-Script} Paket}
%
% Das gesamte \textsf{KOMA-Script} Paket besteht aus mehreren Teilen.
% Der Teil |scrclass.dtx| beinhaltet die Haupt-classes |scrartcl.cls|,
% |scrreprt.cls| und |scrbook.cls| und |scrlttr2.cls| sowie das von
% diesen ben"otigte package |typearea.sty|.
%
% Die urspr"unglich in |komascr.dtx| enthaltene Brief-Klasse
% |scrlettr.cls| liegt nun hier als |scrlettr.dtx| separat vor. Eine
% Anleitung zu dieser Klasse existiert jedoch nur noch
% eingeschr"ankt. Da die Klasse nicht mehr unterst"utzt wird. Sie ist
% nun also \emph{unsupported}.
%
% Die alte Anleitung wurde jedoch in diese Datei integriert.
% \changes{v2.6c}{2001/10/09}{Die Klasse ist obsolet}
% \changes{v2.6c}{2001/10/09}{Die Anleitung ist integriert}
%
% \DeleteShortVerb{\|}
%
% \section{"Uberblick}
% 
% Die Dokumentenklasse \Class{scrlettr} ist eine erweiterte und an
% europ"aische Verh"altnisse angepasste Version der originalen
% \LaTeX{}-Briefklasse \Class{letter}. Urspr"unglich wurde sie von
% Axel Kielhorn entwickelt, erfuhr aber durch Markus Kohm einige
% Ver"anderungen.
% 
% Hervorzuhebende Eigenschaften von \Class{scrlettr} gegen"uber
% \Class{letter} sind die Anpassung an das A4 Papierformat, die
% erweiterte Sprachunterst"utzung und ein umfangreicherer
% Befehlssatz, mit dem auch komplexere W"unsche umsetzbar sind.
% 
% Bevor alle Befehle der Klasse \Class{scrlettr} vorgestellt werden,
% soll mit Hilfe eines Minimalbeispiels ein erster "Uberblick "uber
% Aufbau und Funktion eines Briefes gegeben werden.
% 
% \begin{Example}
% Ein mit nur den n"otigsten Befehlen erstellter
% \Class{scrlettr}-Brief sieht beispielsweise so aus.
% \begin{small}\begin{verbatim}
%   \documentclass[10pt]{scrlettr}
%   \usepackage{ngerman}
%   \name{\KOMAScript{}-Gruppe}
%   \address{Klassengasse 1\\12345 \LaTeX{}hausen}
%   \signature{Euer \KOMAScript{}-Team}
%   \begin{document}
%     \begin{letter}{Die \KOMAScript{}-Nutzer\\
%                    Irgendwo\\weltweit}
% 
%       \opening{Liebe \KOMAScript{}-Nutzer,}
%       das \KOMAScript{}-Team m"ochte Euch mit ein paar
%       Informationen ...
% 
%       \closing{Viel Spa"s}
%     \end{letter}
%   \end{document}
% \end{verbatim}\end{small}
% Wie zu sehen ist, werden Informationen, die unabh"angig vom
% einzelnen Brief sind, getrennt definiert, wie beispielsweise der
% Absender mittels \Macro{name}. Die einzelnen briefspezifische
% Daten werden innerhalb der \Environment{letter}-Umgebung
% angegeben. Nat"urlich k"onnen durch mehrmaliges Nutzen der
% \Environment{letter}-Umgebung mehrere Briefe in einem Dokument
% erzeugt werden.
% 
% \begin{Explain}
%   Hierbei ist jedoch zu beachten, dass \TeX\ Z"ahler grunds"atzlich
%   \emph{global} verwaltet. Es ist also notwendig, alle Z"ahler vor
%   einer neuen \Environment{letter}-Umgebung zur"uckzusetzen. Einzige
%   Ausnahme ist der Seitenz"ahler. Dieser wird bei jedem Aufruf von
%   \Macro{begin}\PParameter{letter} wieder auf 1 zur"uckgesetzt.
% \end{Explain}
% \end{Example}
% 
% \section{Brief"ubergreifende Befehle}
% \begin{Declaration}
%   \Macro{name}\Parameter{Absendername}\\
%   \Macro{address}\Parameter{Adresse des Absenders}\\
%   \Macro{signature}\Parameter{Unterschrift}
% \end{Declaration}\ignorespaces
% \BeginIndex{Cmd}{name}\BeginIndex{Cmd}{address}\ignorespaces
% \BeginIndex{Cmd}{signature}\ignorespaces
% Der Befehl \Macro{name} nimmt den Namen des
% Absenders\Index[indexit]{Absender} auf und gibt diesen im
% voreingestellten Seitenstil f"ur die erste und die folgenden Seiten
% aus. Weiterhin wird dieser Text als
% Unterschrift\Index[indexit]{Unterschrift} gesetzt, wenn
% \Macro{signature} nicht angegeben wurde, da dieser Befehl optional
% ist, und somit nicht immer angegeben werden muss. Die
% Absenderadresse wird mit dem \Macro{address}-Befehl gesetzt.
% \EndIndex{Cmd}{name}\EndIndex{Cmd}{address}\EndIndex{Cmd}{signature}
% 
% \begin{Declaration}
%   \Macro{backaddress}\Parameter{Absender}\\
%   \Macro{specialmail}\Parameter{Versandart}\\
%   \Macro{addrfieldon}\\
%   \Macro{addrfieldoff}
% \end{Declaration}\ignorespaces
% \BeginIndex{Cmd}{backaddress}\BeginIndex{Cmd}{specialmail}\ignorespaces
% \BeginIndex{Cmd}{addrfieldon}\BeginIndex{Cmd}{addrfieldoff}\ignorespaces
% Der Befehl \Macro{backaddress} erzeugt "uber dem
% Adressfeld\Index{Adressfeld} des Empf"angers einen einzeiligen
% Eintrag, der auch in Briefumschl"agen mit Sichtfenster zu sehen
% ist. Es bietet sich somit als Angabe des Absenders an.
% Versandhinweise\Index[indexit]{Versandart}, wie beispielsweise
% \emph{Einschreiben} k"onnen mit \Macro{specialmail} gesetzt werden.
% 
% Mit dem Befehl \Macro{addrfieldoff} werden weder das Adressfeld
% noch das \Option{locfield} gesetzt. Alle Angaben "uber Empf"anger-
% und R"uckadresse, die Versandart und die Erg"anzungen aus
% \Macro{location} werden ignoriert und \emph{nicht} gesetzt. Da
% \Macro{addrfieldon} und \Macro{addrfieldoff} als Befehle
% implementiert sind, ist es m"oglich, sie f"ur verschiedene Briefe
% eines Dokuments je nach Bedarf anzuwenden. Voreingestellt ist
% \Macro{addrfieldon}.
% \EndIndex{Cmd}{backaddress}\EndIndex{Cmd}{specialmail}
% \EndIndex{Cmd}{addrfieldon}\EndIndex{Cmd}{addrfieldoff}
% 
% \begin{Declaration}
%   \Macro{location}\Parameter{zus"atzliche Adresstext}\\
%   \Macro{place}\Parameter{Ort}\\
%   \Macro{date}\Parameter{Datum}
% \end{Declaration}\ignorespaces
% \BeginIndex{Cmd}{location}\BeginIndex{Cmd}{place}\ignorespaces
% \BeginIndex{Cmd}{date}\ignorespaces
% Der Befehl \Macro{location} schreibt sein Argument in ein Textfeld
% rechts neben dem Adressfeld. Es kann beispielsweise Daten einer
% Abteilung oder die Zweigstelle eine Firma aufnehmen. Mit dem
% Befehl \Macro{place} wird der Ort des Absenders eingestellt. Der
% Befehl \Macro{date} ist nur wichtig, wenn der Brief l"anger in
% Quellform gespeichert werden soll und man nicht m"ochte, dass die
% Datumsinformation\Index[indexit]{Datum} des Originalbriefes
% verloren geht. In normalen Anwendungsf"allen wird das Datum aus dem
% Systemdatum beim LaTeX{}-Lauf ermittelt.
% 
% \begin{Example}
% Sie m"ochten einen monatlichen Rundbrief an die Mitglieder eines
% Vereins versenden. Hierbei spielt das genaue Datum keine gro"se
% Rolle. Ein Vermerk in der Form \glqq Vereinssitz im M"arz
% 2001\grqq\ erreichen Sie durch folgende Definitionen.
% \begin{small}\begin{verbatim}
%   \place{Vereinssitz}
%   \date{im M"arz 2001}
% \end{verbatim}\end{small}
% \end{Example}
% \EndIndex{Cmd}{location}\EndIndex{Cmd}{place}\EndIndex{Cmd}{date}
% 
% \begin{Declaration}
%   \Option{wlocfield}\\
%   \Option{slocfield}
% \end{Declaration}
% \BeginIndex{Option}{wlocfield}\BeginIndex{Option}{slocfield}
% Die Breite des Textfeldes mit dem \Macro{location}-Eintrag ist
% standardm"a"sig die H"alfte des freien Raums neben dem Adressfeld.
% Dies entspricht der Einstellung der Option \Option{slocfield}. Mit
% der Klassenoptionen \Option{wlocfield} stehen dem Textfeld zwei
% Drittel der freien Textbreite neben dem Adressfeld zur Verf"ugung.
% \EndIndex{Option}{wlocfield}\EndIndex{Option}{slocfield}
% 
% \section{Briefspezifische Befehle}
% 
% \begin{Declaration}
%   \Environment{letter}\\
%   \Macro{title}\Parameter{"Uberschrift}\\
%   \Macro{subject}\Parameter{Kurzinhalt oder Betreff}\\
%   \Macro{subjecton}\\
%   \Macro{subjectoff}\\
%   \Macro{opening}\Parameter{Anrede}
% \end{Declaration}
% \BeginIndex{Env}{letter}\BeginIndex{Cmd}{title}
% \BeginIndex{Cmd}{subject}\BeginIndex{Cmd}{subjecton}
% \BeginIndex{Cmd}{subjectoff}
% \BeginIndex{Cmd}{opening}
% Das zentrale Element ist die \Environment{letter}-Umgebung. Ihr
% obligatorisches Argument nimmt die Anschrift des
% Empf"angers\Index[indexit]{Empfaenger=Empf"anger} des Briefes auf.
% Notwendige Zeilenumbr"uche m"ussen selbst"andig eingef"ugt werden.
% Dabei muss beachtet werden, dass das Textfeld nur eine bestimmte
% Breite zul"asst, und dann die Zeile automatische umbrochen wird.
% 
% \begin{Example}
% Sie m"ochten einen Brief an die Deutsche Anwendervereinigung \TeX\
% e.\,V. schreiben. Dieser Brief m"u"ste durch folgende Zeilen
% eingeleitet werden:
% \begin{small}\begin{verbatim}
%   \begin{letter}{DANTE, Deutschsprachige
%                  Anwendervereingung TeX e.\,V.\\
%                  Postfach 101840\\
%                  69008 Heidelberg}
% \end{verbatim}\end{small}
% \end{Example}
% 
% Gew"ohnlich m"ochte man dem Empf"anger eines Briefes schnell das
% Anliegen des Briefes mitteilen. Dazu gibt es zwei M"oglichkeiten.
% Mit dem \Macro{title} kann einem Brief eine
% "Uberschrift\Index[indexit]{Briefueberschrift=Brief"uberschrift}
% vorangestellt werden. Dies erzeugt eine zentrierte in
% \Macro{LARGE} gesetzte "Uberschrift. F"ur gew"ohnlich soll der
% Hinweis auf den Inhalt des Briefes nicht so gro"sspurig ausfallen.
% Vielmehr wird eine schlichte Betreffzeile\Index[indexit]{Betreff}
% mit einer kurzen Zusammenfassung des Schreibens erwartet. Dies
% geschieht mit dem Makro \Macro{subject}.
% 
% \begin{Example}
% Anl"asslich eines Firmenjubil"aums m"ochten Sie ein Preisauschreiben
% f"ur alle Kunden ausrichten. Damit Ihre Kunden sofort sehen, dass
% es sich bei Ihrem Brief nicht um eine Rechnung handelt, m"ochten
% Sie deutlich auf den Inhalt Ihres Briefes aufmerksam machen.
% \begin{small}\begin{verbatim}
%   \title{Preisausschreiben}
% \end{verbatim}\end{small}
% Eine schlichte Betreffzeile erzeugen Sie dagegen mit folgender
% Zeile.
% \begin{small}\begin{verbatim}
%   \subject{Lagerverkauf}
% \end{verbatim}\end{small}
% \end{Example}
% Dar"uber hinaus kann mit dem Befehl \Macro{subjecton} vor der
% Betreffzeile noch den Eintrag \emph{Betr.:} gesetzt werden.
% Voreingestellt ist \Macro{subjectoff}, so dass nur die blo"se
% Betreffzeile gesetzt wird.
% 
% Der Brief beginnt mit dem \Macro{opening}-Befehl. Erst nach diesem
% Befehl werden die Angaben im Referenzfeld\Index{Referenzfeld}, Betreff
% und Empf"anger\Index{Empfaenger=Empf"anger} gesetzt. Diesem Befehl wird
% die Anrede des Briefpartners "ubergeben.
% \EndIndex{Env}{letter}\EndIndex{Cmd}{title}
% \EndIndex{Cmd}{subject}\EndIndex{Cmd}{subjecton}\EndIndex{Cmd}{subjectoff}
% \EndIndex{Cmd}{opening}
% 
% \begin{Declaration}
%   \Macro{closing}\Parameter{Gru"sformel}\\
%   \Macro{ps}\Parameter{Postskriptum}\\
%   \Macro{cc}\Parameter{Verteiler}\\
%   \Macro{ccnameseparator}\\
%   \Macro{ccname}\\
%   \Macro{encl}\Parameter{Anlagen}\\
%   \Macro{enclnameseparator}\\
%   \Macro{enclname}
% \end{Declaration}
% \BeginIndex{Cmd}{closing}\BeginIndex{Cmd}{ps}
% Nachdem der Brief geschrieben wurde, wird mit \Macro{closing} das
% Briefende eingeleitet. Diesem Befehl wird die
% Gru"sformel\Index[indexit]{Grussformel=Gru"sformel} (zum Beispiel
% \glqq Mit freundlichen Gr"u"sen\grqq) "ubergeben.
% 
% Manchmal ist es auch erw"unscht, nach dem eigentlichen Brief noch
% Informationen unterzubringen, die nicht dem engsten Briefanliegen
% entsprechen aber dennoch von Interesse f"ur den Empf"anger sein
% k"onnen. Diese Informationen werden dem Befehl \Macro{ps}
% "ubergeben.
% 
% \begin{Example}
% Sie m"ochten Ihre Kunden darauf hinweisen, dass Ihr Gesch"aft im
% August Betriebsurlaub macht und deshalb geschlossen bleibt. Da
% diese Information nicht in unmittelbaren Zusammenhang mit einer
% Rechnung steht, m"ochten Sie dies als
% Postscript\Index[indexit]{Postscriptum} schreiben.
% \begin{small}\begin{verbatim}
%   \closing{Mit freundlichen Gr"u"sen,}
%   \ps{Bitte beachten Sie, dass wir vom 01.08. bis 31.08.
%   Betriebsurlaub machen. Unser Gesch"aft bleibt in dieser
%   Zeit geschlossen. Wir sind ab dem 01.09. in alter
%   Frische wieder f"ur Sie da.}
% \end{verbatim}\end{small}
% \end{Example}
% H"aufig liegen gesch"aftlichen Briefen
% Anlagen\Index[indexit]{Anlagen} bei, beispielsweise
% Vertragsentw"urfe, Rechnungskopien oder "ahnliches. Darauf kann der
% Adressat mit dem Befehl \Macro{encl} aufmerksam gemacht werden.
% Gehen Kopien eines Briefes an die Empf"anger eines
% Verteilers\Index[indexit]{Verteiler}, so kann auch dies im
% Abschluss des Briefes bekannt gegeben werden. Dazu dient der
% Befehl \Macro{cc}.
% \EndIndex{Cmd}{closing}\EndIndex{Cmd}{ps}
% \BeginIndex{Cmd}{cc}\BeginIndex{Cmd}{ccnameseparator}
% \BeginIndex{Cmd}{ccname}
% \BeginIndex{Cmd}{encl}\BeginIndex{Cmd}{enclnameseparator}
% \BeginIndex{Cmd}{enclname}
% \begin{Example}
% Die W"ahrungsumstellung auf den Euro stellt Ihr kleines Unternehmen
% vor gro"se Schwierigkeiten. Um Zahlungsausf"alle zu vermeiden,
% ziehen Sie alle Verbindlichkeiten f"ur den Monat Dezember vor und
% m"ochten diese bereits im November auszahlen. Au"serdem sollen alle
% F"alligkeiten im neuen Jahr erst ab der dritten Januarwoche
% beglichen werden. Diesen Beschluss der Gesch"aftsleitung m"ochten
% Sie als Kopie an die Buchhaltung und den Betriebsrat schicken.
% \begin{small}\begin{verbatim}
%   \encl{Beschluss "uber die Zahlungsmodalit"aten beim
%   "Ubergang auf den Euro}
%   \cc{Buchhaltung\\Betriebsrat}
% \end{verbatim}\end{small}
% \end{Example}
% Der Aufruf \Macro{cc} setzt vor dem "ubergebenen Argument noch den
% \Macro{ccname} \glqq Kopie an\grqq\ und den Trenner
% \Macro{ccnameseparator} \glqq :\ \grqq\ (Doppelpunkt gefolgt von
% einem Leerzeichen). "Ahnlich funktioniert der Befehl \Macro{encl}.
% Der \Macro{enclname} lautet \glqq Anlagen\grqq\ -- der Trenner
% \Macro{enclnameseparator} ist genauso definiert, wie der f"ur
% Verteiler.
% 
% \begin{Example}
% Da Sie im obigen Beispiel nur eine Anlage verschicken, ist es
% besser, den Singular zu verwenden. Sie m"ochten, dass der Verteiler
% auch als solcher benannt wird und au"serdem halten Sie den
% Doppelpunkt f"ur "uberfl"ussig. Alle Ihre W"unsche k"onnen erf"ullt
% werden:
% \begin{small}\begin{verbatim}
%   \renewcommand*{\ccnameseparator}{\ }
%   \renewcommand*{\enclnameseparator}{\ccnameseparator}
%   \renewcommand*{\ccname}{Verteiler}
%   \renewcommand*{\enclname}{Anlage}
% \end{verbatim}\end{small}
% \end{Example}
% \EndIndex{Cmd}{cc}\EndIndex{Cmd}{ccnameseparator}\EndIndex{Cmd}{ccname}
% \EndIndex{Cmd}{encl}\EndIndex{Cmd}{enclnameseparator}\EndIndex{Cmd}{enclname}
% 
% \subsection{Das Referenzfeld}\Index[indexit]{Referenzfeld}
% 
% \begin{Declaration}
%   \Macro{yourref}\Parameter{Ihr Zeichen}\\
%   \Macro{yourmail}\Parameter{Ihr Schreiben vom}\\
%   \Macro{myref}\Parameter{Unser Zeichen}\\
%   \Macro{customer}\Parameter{Kundennummer}\\
%   \Macro{invoice}\Parameter{Rechnungsnummer}
% \end{Declaration}
% \BeginIndex{Cmd}{yourref}\BeginIndex{Cmd}{yourmail}\BeginIndex{Cmd}{myref}
% \BeginIndex{Cmd}{customer}\BeginIndex{Cmd}{invoice}
% In Gesch"aftsbriefen werden h"aufig Informationen wie
% Aktenzeichen\Index[indexit]{Aktenzeichen}, Rech\-nungs- oder
% Kundennummer\Index[indexit]{Rechnungsnummer}\Index[indexit]{Kundennummer}
% oder ein Hinweis auf das Schreiben, das beantwortet wird,
% ben"otigt. Um diese Anforderungen realisieren zu k"onnen, sind in
% der \Class{scrletter}-Klasse einige Makros implementiert.
% 
% \begin{Example}
% Sie m"ochten einen Brief Ihres Gesch"aftspartners Maier beantworten,
% den dieser am 14. August 2000 geschrieben hat. Ihr Zeichen ist
% {\emph{xyz}} Herr Maier hat die Kundennummer {\em{maier007}} und
% beschwerte sich in seinem Brief, dass in der Rechnung mit der
% Nummer {\em{197200/01}} kein Mehrwertsteueranteil angegeben war.
% Das Zeichen von Herrn Maier lautet {\em{maier}}. Sie schreiben
% also folgenden Brief:
% \begin{small}\begin{verbatim}
%   \documentclass[10pt,a4paper]{scrlettr}
%   \usepackage{ngerman}
%   \name{Firma xyz}
%   \address{Industriegasse 12\\23987 Stahlhausen}
%   \signature{Herr Schmidt\\ Reklamationen}
%     \begin{document}
%       \begin{letter}{Herr Maier\\Wiesenweg 37\\ Blumental}
%       \yourref{maier}
%       \yourmail{14.08.2000}
%       \myref{xyz}
%       \customer{maier007}
%       \invoice{197200/01}
%       \opening{Sehr geehrter Herr Maier,}
%       vielen Dank f"ur Ihr Schreiben vom 14. August.
%       Wir bedauern unseren Fehler und senden Ihnen
%       anbei eine korrigierte Rechnung.
%       \closing{Mit freundlichen Gr"u"sen}
%       \end{letter}
%     \end{document}
% \end{verbatim}\end{small}
% Sie sehen, dass Ihre Angaben im Referenzfeld\Index{Referenzfeld}
% zwischen der Empf"angeradresse und dem eigentlichen Brieftext
% gesetzt werden.
% \end{Example}
% \EndIndex{Cmd}{yourref}\EndIndex{Cmd}{yourmail}\EndIndex{Cmd}{myref}
% \EndIndex{Cmd}{customer}\EndIndex{Cmd}{invoice}
% 
% \begin{Declaration}
%   \Macro{refitemi}\Parameter{Eigenes Feld 1}\\
%   \Macro{refitemii}\Parameter{Eigenes Feld 2}\\
%   \Macro{refitemiii}\Parameter{Eigenes Feld 3}\\
%   \Macro{refitemnamei}\Parameter{Bezeichnung des eigenen Feldes 1}\\
%   \Macro{refitemnameii}\Parameter{Bezeichnung des eigenen Feldes 2}\\
%   \Macro{refitemnameiii}\Parameter{Bezeichnung des eigenen Feldes 3}
% \end{Declaration}
% \BeginIndex{Cmd}{refitemi}\BeginIndex{Cmd}{refitemii}
% \BeginIndex{Cmd}{refitemiii}
% \EndIndex{Cmd}{refitemi}\EndIndex{Cmd}{refitemii}\EndIndex{Cmd}{refitemiii}
% \BeginIndex{Cmd}{refitemnamei}\BeginIndex{Cmd}{refitemnameii}
% \BeginIndex{Cmd}{refitemnameiii}
% Neben den bereits vordefinierten Makros stehen noch bis zu drei
% frei definierbare Makros zur Verf"ugung, um das Referenzfeld
% verschiedenen Anforderungen gem"a"s anpassen zu k"onnen.
% 
% \begin{Example}
% Angenommen, Sie stehen mit Herrn Maier in Verhandlungen "uber die
% Abnahme von 10.000 St"uck Ihres schlimmsten Ladenh"uters. Dann
% ben"otigen Sie nat"urlich keine Rechnungsnummer. Statt dessen
% m"ochten Sie im Referenzfeld ein Aktenzeichen\Index{Aktenzeichen}
% vermerken. Dazu definieren zu mit \Macro{refitemnamei} den ersten
% frei w"ahlbaren Referenzeintrag als \emph{Aktenzeichen}. Diesem
% Eintrag k"onnen Sie dann das Aktenzeichen \emph{123/01mai}
% zuweisen.
% \begin{small}\begin{verbatim}
%   \refitemnamei{Aktenzeichen}
%   \refitemi{123/01mai}
% \end{verbatim}\end{small}
% \end{Example}
% \EndIndex{Cmd}{refitemnamei}\EndIndex{Cmd}{refitemnameii}
% \EndIndex{Cmd}{refitemnameiii}
% 
% \section{Seitenstile}
% 
% \begin{Declaration}
%   \Macro{firsthead}\Parameter{Kopfdefinition}\\
%   \Macro{firstfoot}\Parameter{Fu"sdefinition}\\
%   \Macro{nexthead}\Parameter{Kopfdefinition}\\
%   \Macro{nextfoot}\Parameter{Fu"sdefinition}
% \end{Declaration}
% \BeginIndex{Cmd}{firsthead}\BeginIndex{Cmd}{firstfoot}
% \BeginIndex{Cmd}{nexthead}\BeginIndex{Cmd}{nextfoot}
% Die \Class{scrlettr}-Klasse erm"oglicht es, den
% Seitenstil\Index{Seitenstil} eines Dokuments an die eigenen
% Bed"urfnisse anzupassen. Dazu kann getrennt f"ur die erste und alle
% folgenden Seiten der Fu"s und der Kopf frei definiert werden. Diese
% Definition muss \emph{vor} dem Aufruf von
% \Macro{pagestyle}\Parameter{\dots} erfolgen.
% \begin{Example}
% Sie m"ochten in der Fu"szeile der ersten Seite eines Briefes die
% Bankverbindung notieren. In den Fu"szeilen der folgenden Seiten
% m"ochten Sie auf Ihren verantwortungsbewussten Umgang mit der
% Umwelt hinweisen.
% \begin{small}\begin{verbatim}
%   \firstfoot{Bankverbindung:\hfill $\bullet$\hfill Deutsche
%   Bank AG\hfill $\bullet$\hfill BLZ: 999\,720\,00\hfill
%   $\bullet$\hfill Konto: 123\,456\,890}
%   \nextfoot{\centerline{Dieses Schreiben wird
%   ausschlie"slich auf chlorfrei gebleichten Papier gedruckt.}}
%   \pagestyle{firstpage}
% \end{verbatim}\end{small}
% \end{Example}
% Der Aufruf des Steitenstils
% \Macro{pagestyle}\PParameter{firstpage}\IndexPagestyle[indexit]{firstpage}
% sorgt daf"ur, dass auf allen Seiten Kopf und Fu"s wie auf der ersten
% Seite gestaltet werden.
% 
% Voreingestellt ist der Stil \PValue{plain}\IndexPagestyle{plain}.
% Wird der Stil \PValue{headings}\IndexPagestyle{headings} ohne
% eigene Definition der Kopf- und Fu"szeilen verwendet, so sind die
% Fu"szeilen grunds"atzlich leer. Die Kopfzeile der ersten Seite
% enth"alt zentriert den Absendernamen eine
% Trennlinie\Index{Trennlinie} und die Absenderadresse. Der Kopf der
% folgenden Seiten besteht aus den linksb"undig gesetzten
% Absendernamen in der ersten Zeile. In einer zweiten Zeile steht
% der Empf"angername, das Datum und die Seitenzahl. Wird der Stil
% \PValue{empty}\IndexPagestyle{empty} gew"ahlt, so bleiben Kopf-
% und Fu"szeile auf allen Seiten leer.
% 
% Auf diese Weise k"onnen sehr individulle Briefb"ogen erstellt
% werden.
% \iffalse
% Der Phantasie sind lediglich durch die zur Verf"ugung
% stehenden Zeichens"atze Grenzen gesetzt.
% \fi
% \EndIndex{Cmd}{firsthead}\EndIndex{Cmd}{firstfoot}
% \EndIndex{Cmd}{nexthead}\EndIndex{Cmd}{nextfoot}
% 
% \begin{Declaration}
%   \Option{twoside}
% \end{Declaration}
% \BeginIndex{Option}{twoside}
% Zweiseitig gedruckte Briefe werden durch die Angabe der Option
% \Option{twoside} unterst"utzt. Im Gegensatz zu den "ubrigen
% \KOMAScript -Klassen "andert sich hier der Satzspiegel nicht,
% sondern es wird lediglich sichergestellt, dass ein neuer Brief
% immer auf einer ungeraden \emph{Druck}seite beginnt. Dar"uber
% hinaus wird eine Warnung ausgegeben, um darauf hinzuweisen, dass
% es sich nicht wirklich um ein zweiseitiges Layout handelt.
% \EndIndex{Option}{twoside}
% 
% \begin{Declaration}
%   \Macro{foldmarkson}\\
%   \Macro{foldmarksoff}
% \end{Declaration}
% \BeginIndex{Cmd}{foldmarkson}\BeginIndex{Cmd}{foldmarksoff}
% Die Faltmarken\Index[indexit]{Faltmarken} k"onnen mit dem Befehl
% \Macro{foldmarkson} eingeschaltet und mit dem Befehl
% \Macro{foldmarksoff} ausgestellt werden. Dies ist f"ur jeden Brief
% eines Dokuments getrennt m"oglich. Voreingestellt ist
% \Macro{foldsmarkson}. Diese Faltmarken werden von \Macro{opening}
% gesetzt. Die Schalter und evtl. "Anderungen der Ma"se (vgl.
% Abschnitt~\ref{list:laengen}) m"ussen also vor diesem Makro
% aufgerufen werden.
% \EndIndex{Cmd}{foldmarkson}\EndIndex{Cmd}{foldmarksoff}
% 
% 
% \section{Unterst"utzung verschiedener Sprachen}
% 
% \subsection{Sprachauswahl und -umschaltung}
% \Index[indexit]{Sprachumschaltung}\Index[indexit]{Sprachauswahl}
% 
% Die \Class{scrlettr}-Klasse unterst"utz viele Sprachen. Dazu z"ahlen
% neben Deutsch auch "Osterreichisch, Englisch (britisch und
% amerikanisch), Franz"osich, Italienisch und Spanisch. Zwischen den
% Sprachen wird bei Verwendung des \Package{babel}-Pakets mit dem
% Befehl \Macro{selectlanguage}\Parameter{Sprachauswahl} gewechselt.
% 
% \begin{Declaration}
%   \Macro{captionsenglish}\\
%   \Macro{captionsUSenglish}\\
%   \Macro{captionsamerican}\\
%   \Macro{captionsbritish}\\
%   \Macro{captionsUKenglish}\\
%   \Macro{captionsgerman}\\
%   \Macro{captionsaustrian}\\
%   \Macro{captionsfrench}\\
%   \Macro{captionsitalian}\\
%   \Macro{captionsspanish}
% \end{Declaration}
% \BeginIndex{Cmd}{captionsenglish}\BeginIndex{Cmd}{captionsUSenglish}
% \BeginIndex{Cmd}{captionsamerican}\BeginIndex{Cmd}{captionsbritish}
% \BeginIndex{Cmd}{captionsUKenglish}\BeginIndex{Cmd}{captionsgerman}
% \BeginIndex{Cmd}{captionsaustrian}\BeginIndex{Cmd}{captionsfrench}
% \BeginIndex{Cmd}{captionsitalian}\BeginIndex{Cmd}{captionsspanish}
% Wird die Sprache eines Briefes gewechselt, so "andern sich
% automatisch die Eintragungen der automatisch gesetzten \glqq
% Caption\grqq -Texte wie \emph{Betreff}, \emph{Seite} oder
% \emph{Anlagen}. Sollte das verwendete Sprachumschaltpaket diese
% Texte nicht automatisch verwalten, so k"onnen die entsprechenden
% Befehle notfalls auch dirket verwendet werden.
% \EndIndex{Cmd}{captionsenglish}\EndIndex{Cmd}{captionsUSenglish}
% \EndIndex{Cmd}{captionsamerican}\EndIndex{Cmd}{captionsbritish}
% \EndIndex{Cmd}{captionsUKenglish}\EndIndex{Cmd}{captionsgerman}
% \EndIndex{Cmd}{captionsaustrian}\EndIndex{Cmd}{captionsfrench}
% \EndIndex{Cmd}{captionsitalian}\EndIndex{Cmd}{captionsspanish}
% 
% \begin{Declaration}
%   \Macro{dateenglish}\\
%   \Macro{dateUSenglish}\\
%   \Macro{dateamerican}\\
%   \Macro{datebritish}\\
%   \Macro{dateUKenglish}\\
%   \Macro{dategerman}\\
%   \Macro{dateaustrian}\\
%   \Macro{datefrench}\\
%   \Macro{dateitalian}\\
%   \Macro{datespanish}
% \end{Declaration}
% \BeginIndex{Cmd}{dateenglish}\BeginIndex{Cmd}{dateUSenglish}
% \BeginIndex{Cmd}{dateamerican}\BeginIndex{Cmd}{datebritish}
% \BeginIndex{Cmd}{dateUKenglish}\BeginIndex{Cmd}{dategerman}
% \BeginIndex{Cmd}{dateaustrian}\BeginIndex{Cmd}{datefrench}
% \BeginIndex{Cmd}{dateitalian}\BeginIndex{Cmd}{datespanish}
% Je nach verwendeter Sprache werden auch die
% Datumsangaben\Index{Datum} in unterschiedlicher Form umgesetzt.
% Die genauen Angaben k"onnen der Tabelle~\ref{TAB:Datum} entnommen
% werden.
% \begin{table}
% \centering
% \begin{tabular}{ll}
%   \verb \dateenglish   & 1/12/1993\\
%   \verb \dateUSenglish & 12/1/1993\\
%   \verb \dateamerican  & 12/1/1993\\
%   \verb \datebritish   & 1/12/1993\\
%   \verb \dateUKenglish & 1/12/1993\\
%   \verb \dategerman    & 1.\,12.\,1993\\
%   \verb \dateaustrian  & 1.\,12.\,1993\\
%   \verb \datefrench    & 1.\,12.\,1993\\
%   \verb \dateitalian   & 1.\,12.\,1993\\
%   \verb \datespanish   & 1.\,12.\,1993\\
% \end{tabular}
% \caption{Sprachabh"angige Ausgabeformate f"ur Datum}
% \label{TAB:Datum}
% \end{table}
% \EndIndex{Cmd}{dateenglish}\EndIndex{Cmd}{dateUSenglish}
% \EndIndex{Cmd}{dateamerican}\EndIndex{Cmd}{datebritish}
% \EndIndex{Cmd}{dateUKenglish}\EndIndex{Cmd}{dategerman}
% \EndIndex{Cmd}{dateaustrian}\EndIndex{Cmd}{datefrench}
% \EndIndex{Cmd}{dateitalian}\EndIndex{Cmd}{datespanish}
% 
% \begin{Declaration}
%   \Option{orgdate}\\
%   \Option{scrdate}
% \end{Declaration}
% \BeginIndex{Option}{orgdate}
% Sollen die Datumseinstellungen\Index{Datum} des \Package{babel}-
% oder \Package{ngerman}-Pakets oder eines eigenen
% Sprachumschaltpakets benutzt werden, so kann dies durch die
% Klassenoption \Option{orgdate} erreicht werden. Voreingestellt ist
% die Verwendung der \Class{scrlettr}-eigenen Definition
% (\Option{scrdate}).
% \EndIndex{Option}{orgdate}
% 
% \subsection{Sprachabh"angige Variablen}
% \Index[indexit]{sprachabhaengige Variablen=sprachabh"angige
% Variablen}
% \begin{Declaration}
%   \Macro{yourrefname}\\
%   \Macro{yourmailname}\\
%   \Macro{myrefname}\\
%   \Macro{customername}\\
%   \Macro{invoicename}\\
%   \Macro{subjectname}\\
%   \Macro{ccname}\\
%   \Macro{enclname}\\
%   \Macro{headtoname}\\
%   \Macro{datename}\\
%   \Macro{pagename}
% \end{Declaration}
% \BeginIndex{Cmd}{yourrefname}\BeginIndex{Cmd}{yourmailname}
% \BeginIndex{Cmd}{myrefname}
% \BeginIndex{Cmd}{customername}\BeginIndex{Cmd}{invoicename}
% \BeginIndex{Cmd}{subjectname}
% \BeginIndex{Cmd}{ccname}\BeginIndex{Cmd}{enclname}
% \BeginIndex{Cmd}{headtoname}
% \BeginIndex{Cmd}{datename}\BeginIndex{Cmd}{pagename}
% Die aufgef"uhrten Befehle enthalten die jeweils sprachtypischen
% \emph{Captiontexte}. Diese k"onnen f"ur die Realisierung einer
% weiteren Sprache oder aber auch zur eigenen freien Gestaltung
% angepasst werden. Dazu benutzt man den Befehl
% \Macro{renewcommand}.
% 
% \begin{Example}
% M"ochten Sie statt des Eintrags \glqq Ihr Schreiben vom\grqq\
% lieber \glqq Ihre Nachricht vom\grqq\ im Referenzfeld stehen
% haben, m"ussen Sie den Befehl \Macro{yourmailname} wie folgt
% umdefinieren.
% \begin{small}\begin{verbatim}
%   \renewcommand*{\yourmailname}{Ihre Nachricht vom}
% \end{verbatim}\end{small}
% Auf diese Weise k"onnen Sie nat"urlich auch alle Variablen den
% Vorgaben einer anderen Sprache anpassen.
% \end{Example}
% Es ist darauf zu achten, dass die Variablen erst \emph{nach}
% \Macro{begin}\PParameter{document} definiert werden. Der Aufruf
% \Macro{renewcommand*}\Parameter{\dots} muss daher zwingend nach
% \Macro{begin}\PParameter{document} oder mit Hilfe von
% \Macro{AtBeginDocument} erfolgen.
% \EndIndex{Cmd}{yourrefname}\EndIndex{Cmd}{yourmailname}
% \EndIndex{Cmd}{myrefname}
% \EndIndex{Cmd}{customername}\EndIndex{Cmd}{invoicename}
% \EndIndex{Cmd}{subjectname}
% \EndIndex{Cmd}{ccname}\EndIndex{Cmd}{enclname}\EndIndex{Cmd}{headtoname}
% \EndIndex{Cmd}{datename}\EndIndex{Cmd}{pagename}
% 
% 
% \section{Adressdateien}
% \label{sec:adressdateien}
% \begin{Declaration}
%   \Macro{adrentry}\Parameter{Name}\Parameter{Vorname}\Parameter{Adresse}
%   \Parameter{Telefon}\Parameter{F1}\Parameter{F2}
%   \Parameter{Kommentar}\Parameter{K"urzel}
% \end{Declaration}
% \BeginIndex{Cmd}{adrentry}
% \label{decl:adrentry}
% Mit der \Class{scrlettr}-Klasse k"onnen auch
% Adressdateien\Index[indexit]{Adressdatei} ausgewertet werden. Dies
% ist beispielsweise f"ur Serienbriefe sehr n"utzlich (siehe
% Abschnitt~\ref{subsec:Serienbriefe}). Eine Adressdatei muss die
% Endung \File{.adr} haben und besteht aus einer Reihe von
% \Macro{adrentry}-Eintr"agen. Ein solcher Eintrag besteht aus acht
% Elementen und kann beispielsweise wie folgt aussehen:
% \begin{small}\begin{verbatim}
%   \adrentry{Maier}
%            {Herbert}
%            {\Wiesenweg 37\\ 09091 Blumental}
%            {0\,23\,34 / 91\,12\,74}
%            {Bauunternehmer}
%            {}
%            {kauft alles}
%            {MAIER}
% \end{verbatim}\end{small}
% Die Elemente f"unf und sechs, \PValue{F1} und \PValue{F2}, k"onnen
% frei bestimmt werden. Denkbar w"aren neben Hinweisen auf das
% Geschlecht oder akademische Grade auch der Geburtstag oder das
% Eintrittsdatum in einen Verein.
% Um das "Uberschreiben von \TeX - oder \LaTeX -Befehlen zu
% vermeiden, ist es empfehlenswert, f"ur \emph{K"urzel} ausschlie"slich
% Gro"sbuchstaben zu verwenden.
% 
% \begin{Example}
% Herr Maier geh"ort zu Ihren engeren Gesch"aftspartnern. Da Sie eine
% rege Korrespondenz mit ihm pflegen, ist es Ihnen auf Dauer zu
% m"u"sig, jedesmal alle Empf"angerdaten aufs Neue einzugeben.
% \Class{scrlettr} nimmt Ihnen diese Arbeit ab. Angenommen, Sie
% haben Ihre Kundenkontakte in der Datei \File{partner.adr}
% gespeichert und Sie m"ochten Herrn Maier einen Brief schreiben,
% dann sparen Sie sich viel Tipparbeit, wenn Sie folgendes eingeben:
% \begin{small}\begin{verbatim}
%   \input{partner.adr}
%   \begin{letter}{\MAIER}
%     Der Brief ...
%   \end{letter}
% \end{verbatim}\end{small}
% Achten Sie bitte darauf, dass Ihr \TeX -System auch auf die
% \File{.adr}-Dateien zugreifen kann, da sonst eine Fehlermeldung
% von \Macro{input} verursacht wird. Entweder Sie legen die Brief-
% und Adressdateien im selben Verzeichnis an, oder Sie binden ein
% Adressverzeichnis fest in Ihr \TeX -System ein.
% \end{Example}
% \EndIndex{Cmd}{adrentry}
% 
% \subsection{Serienbriefe mit der \Class{scrlettr}-Klasse}
% \label{subsec:Serienbriefe}
% 
% Neben dem vereinfachten Zugriff auf Kundendaten k"onnen die
% \File{.adr}-Dateien auch f"ur
% Serienbriefe\Index[indexit]{Serienbriefe} genutzt werden. So ist
% es ohne die komplizierte Anbindung an Datenbanksysteme m"oglich,
% solche Massenpostsendungen zu erstellen.
% \begin{Example}
% Sie wollen einen Serienbrief an alle Mitglieder Ihres
% Anglervereins schicken, um zur n"achsten Mitgliederversammlung
% einzuladen.
% \begin{small}\begin{verbatim}
%   \documentclass{scrlettr}
%   \usepackage{ngerman}
%     \begin{document}
%     \def\adrentry#1#2#3#4#5#6#7#8{
%       \begin{letter}{#2 #1\\#3}
%         \opening{Liebe Vereinsmitglieder,}
%         unsere n"achste Mitgliederversammlung
%         findet am Montag,
%         dem 13.\, August 2001, statt.
% 
%         Folgende Punkte m"ussen besprochen werden...
%         \closing{Petri Heil,}
%       \end{letter}
%     }
%     \input{mitglieder.adr}
%     \end{document}
% \end{verbatim}\end{small}
% \end{Example}
% Nat"urlich kann der Briefinhalt auch von den Adressatenmerkmalen
% abh"angig gemacht werden. Als Bedingungsfelder k"onnen die frei
% bestimmbaren Elemente f"unf oder sechs eines
% \Macro{adrentry}-Eintrages genutzt werden.
% \begin{Example}
% Angenommen, Sie verwenden das Element f"unf, um das Geschlecht
% eines Vereinmitgliedes zu hinterlegen (\PValue{m/w}) und das
% sechste Element weist auf eine R"uckstand der Mitgliedsbeitr"age
% hin. Wollen Sie nun alle s"aumigen Mitglieder anschreiben und
% pers"onlich anreden, so hilft Ihnen folgendes Beispiel weiter:
% \begin{small}\begin{verbatim}
%   \def\adrentry#1#2#3#4#5#6#7#8{
%     \ifcase #6
%     % #6 > 0
%     % hier werden die s"aumigen Mitglieder herausgefiltert
%     \else
%       \begin{letter}{#2 #1\\#3}
%         \if #5m \opening{Lieber #2,} \fi
%         \if #5w \opening{Liebe #2,} \fi
% 
%         Leider mussten wir feststellen, dass du mit der Zah-
%         lung deiner Mitgliedsbeitr"age im R"uckstand bist.
% 
%         Wir m"ochten Dich bitten, den offenen Betrag von #6 DM
%         auf das Vereinskonto einzuzahlen.
%        \closing{Petri Heil,}
%       \end{letter}
%      \fi
%   }
% \end{verbatim}\end{small}
% \end{Example}
% Es ist also m"oglich, den Brieftext auf bestimmte Empf"angermerkmale
% gezielt abzustimmen und so den Eindruck eines pers"onlichen
% Schreibens zu erwecken. Die Anwendungsbreite ist lediglich durch
% die maximale Anzahl von zwei freien \Macro{adrentry}-Elementen
% begrenzt.
% 
% \subsection{Adressverzeichnisse und Telefonlisten erstellen}
% 
% Der Inhalt dieses Abschnitts ist komplett obsolet. Hier sei
% stattdessen auf das \Package{adrconv}-Paket von Axel Kielhorn
% verwiesen.
% 
% \section{Befehls- und Variablen"ubersicht}
% \subsection{Briefspezifische Befehle, die strukturbeschreibend sind
% oder eine Ausgabe erzeugen:}
% 
% \begin{labeling}[~]{\Macro{begin}\PParameter{letter}\Parameter{Adressat}}
% \item[\Macro{begin}\PParameter{letter}\Parameter{Adressat}] Markiert den
%   Beginn eines Briefes an \emph{Adressat} und beginnt eine neue
%   Seite\IndexEnv{letter}
% \item[\Macro{end}\PParameter{letter}] Markiert das Ende eines
%   Briefes\IndexEnv{letter}
% \item[\Macro{opening}\Parameter{Anrede}] Setzt alle Teile eines
%   Briefes oberhalb und einschlie"slich der
%   \emph{Anrede}\IndexCmd{opening}
% \item[\Macro{closing}\Parameter{Gru"sformel}] Setzt \emph{Gru"sformel}
%   und Unterschrift\IndexCmd{closing}\Index{Grussformel=Gru"sformel}
% \item[\Macro{ps}\Parameter{Postscriptum}] Setzt ein
%   \emph{Postscriptum}\IndexCmd{ps}\Index{Postscriptum}
% \item[\Macro{cc}\Parameter{Verteiler}] setzt eine Verteilerliste,
%   deren Eintr"age durch \verb \\ \ zu trennen sind
%   (vgl.\Macro{ccnameseparator} und
%   \Macro{ccname})\IndexCmd{cc}\Index{Verteiler}
% \item[\Macro{encl}\Parameter{Anlagen}] Setzt eine Anlagenliste, deren
%   Eintr"age durch \verb \\ \ zu trennen sind (vgl.
%   \Macro{enclnameseparator} und
%   \Macro{enclname})\IndexCmd{encl}\Index{Anlagen}
% \end{labeling}
% 
% \subsection{Befehle der Adressdateien:}
% 
% \Macro{adrchar} und \Macro{adrentry} siehe
% Abschnitt~\ref{decl:adrentry}
% 
% \subsection{Befehle zur Sprachumschaltung:}\Index{Sprachumschaltung}
% 
% \begin{labeling}[~]{\Macro{captionsUSenglish}}
% \item[\Macro{captionsenglish}] Umschaltung auf englische
%   Caption-Texte\IndexCmd{captionsenglish}
% \item[\Macro{captionsUSenglish}] Umschaltung auf amerikanische
%   Cation-Texte\IndexCmd{captionsUSenglish}
% \item[\Macro{captionsgerman}] Umschaltung auf deutsche
%   Caption-Texte\IndexCmd{captionsgerman}
% \item[\Macro{captionsfrench}] Umschaltung auf franz"osische
%   Caption-Texte\IndexCmd{captionsfrench}
% \item[\Macro{captionsitalian}] Umschaltung auf italienische
%   Caption-Texte\IndexCmd{captionsitalian}
% \item[\Macro{captionsaustrian}] Umschaltung auf "osterreichische
%   Caption-Texte\IndexCmd{captionsaustrian}
% \item[\Macro{captionsspanish}] Umschaltung auf spanische
%   Caption-Texte\IndexCmd{captionsspanish}
% \index{Datum}
% \item[\Macro{dateenglish}] Englisches Datum
%   (vgl. Tabelle~\ref{TAB:Datum})\IndexCmd{dateenglish}
% \item[\Macro{dateUSenglish}] Amerikanisches Datum
%   (vgl. Tabelle~\ref{TAB:Datum})\IndexCmd{dateUSenglish}
% \item[\Macro{dategerman}] Deutsches Datum
%   (vgl. Tabelle~\ref{TAB:Datum})\IndexCmd{dategerman}
% \item[\Macro{datefrench}] Franz"osisches Datum
%   (vgl. Tabelle~\ref{TAB:Datum})\IndexCmd{datefrench}
% \item[\Macro{dateitalian}] Italienisches Datum
%   (vgl. Tabelle~\ref{TAB:Datum})\IndexCmd{dateitalian}
% \item[\Macro{dateaustrian}] "Osterreichisches Datum
%   (vgl. Tabelle~\ref{TAB:Datum})\IndexCmd{dateaustrian}
% \item[\Macro{datespanish}] Spanisches Datum (vgl. Tabelle~\ref{TAB:Datum})\IndexCmd{datespanish}
% \end{labeling}
% 
% \subsection{Sprachabh"angige Variablen}\Index{sprachabhaengige
%   Variablen=sprachabh"angige Variablen}
% 
% Diese Varialblen d"urfen an jeder Stelle nach dem
% \Macro{begin}\PParameter{document}-Befehl aufgerufen werden. Sie
% k"onnen nur mit \Macro{renewcommand} ge"andert werden. Die
% untenstehende Aufstellung listet die voreingestellten Eintr"age f"ur
% die Sprachen Deutsch, Englisch, Franz"osisch, Italienisch und
% Spanisch auf. Die amerikanischen Caption-Texte entsprechen den
% englischen. Deutsche und "osterreichische Eintragungen sind
% identisch.
% 
% \begin{labeling}[~]{\Macro{customername}}
% \item[\Macro{yourrefname}] \small
%   Ihr Zeichen / Your ref. / Vos références / Vs./Rif. / Su ref.
% \IndexCmd{yourrefname}
% \item[\Macro{yourmailname}] \small
%   Ihr Schreiben vom / Your letter of / Votre lettre du / Vs.~lettera
%   del / Su carta de\IndexCmd{yourmailname}
% \item[\Macro{myrefname}] \small
%   Unser Zeichen / Our ref. / Nos références / Ns./Rif. / Nuestra
%   ref.\IndexCmd{myrefname}
% \item[\Macro{customername}] \small
%   Kundennummer / Customer no. / Numéro de client / Nr.~cliente / No.
%   de cliente\IndexCmd{customername}
% \item[\Macro{invoicename}] \small
%   Rechnungsnummer / Invoice no. / Numéro de facture / Nr.~fattura /
%   No. de factura\IndexCmd{invoicename}
% \item[\Macro{subjectname}] \small Betr. / Subject / Concernant /
%   Oggetto / Asunto\IndexCmd{subjectname}
% \item[\Macro{ccname}] \small
%   Kopie an / cc / Copia á / Per conoscenza / Copias\IndexCmd{ccname}
% \item[\Macro{enclname}] \small
%   Anlagen / encl / Annexes / Allegato / Adjunto\IndexCmd{enclname}
% \item[\Macro{headtoname}] \small
%   An / To / A / A / A\IndexCmd{headtoname}
% \item[\Macro{datename}] \small
%   Datum / Date / Date / Data / Fecha\IndexCmd{datename}
% \item[\Macro{pagename}] \small
%   Seite / Page / Page / Pagina / Página\IndexCmd{pagename}
% \end{labeling}
% 
% \subsection{Briefspezifische Variablen und deren Befehle zur Neu- und
% Umdefinierung}
% 
% Die in Klammern stehenden Variablen werden durch Aufruf folgender
% Makros ge"andert.
% \begin{labeling}[~]{\Macro{refitemnameiii}}
% \index{briefspezifische Variablen}
% \item[\Macro{name}] Name des Absenders
%   (\Macro{fromname})\IndexCmd{name}
% \item[\Macro{branch}] Branche des Absenders (
%   \Macro{frombranch})\IndexCmd{branch}
% \item[\Macro{signature}] Unterschrift, voreingestellt ist die
%   "Ubernahme des Wertes von \Macro{name}
%   (\Macro{fromsig})\IndexCmd{signature}
% \item[\Macro{address}] Absenderadresse
%   (\Macro{fromaddress})\IndexCmd{address}
% \item[\Macro{place}] Absenderort (\Macro{fromplace})\IndexCmd{place}
% \item[\Macro{location}] weitere Angabe zur Absenderadresse
%  (\Macro{fromlocation})\IndexCmd{location}
% \item[\Macro{backaddress}] Absenderadresse im Adressfeld
%  (\Macro{frombackaddress})\IndexCmd{backaddress}
% \item[\Macro{telephone}] Telefonnummer des Absenders
%  (\Macro{telephonenum})\IndexCmd{telefone}
% \item[\Macro{yourref}] Referenzfeldeintrag
%  (\Macro{varyourref})\IndexCmd{yourref}
% \item[\Macro{yourmail}] Referenzfeldeintrag
%  (\Macro{varyourmail}\IndexCmd{yourmail}
% \item[\Macro{myref}] Referenzfeldeintrag
%  (\Macro{varmymail})\IndexCmd{yourref}
% \item[\Macro{customer}] Referenzfeldeintrag
%  (\Macro{varcustomer})\IndexCmd{customer}
% \item[\Macro{invoice}] Referenzfeldeintrag
%  (\Macro{varinvoice})\IndexCmd{invoice}
% \item[\Macro{refitemi}] Referenzfeldeintrag, frei definierbar
%   (\Macro{varrefitemi})\IndexCmd{refitemi}
% \item[\Macro{refitemii}] Referenzfeldeintrag, frei definierbar
%   (\Macro{varrefitemii})\IndexCmd{refitemii}
% \item[\Macro{refitemiii}] Referenzfeldeintrag, frei definierbar
%   (\Macro{varrefitemiii})\IndexCmd{refitemiii}
% \item[\Macro{refitemnamei}] Bezeichnung eines frei definierbaren
%   Referenzfeldeintrags
%   (\Macro{varrefitemnamei})\IndexCmd{refitemnamei}
% \item[\Macro{refitemnameii}] Bezeichnung eines frei definierbaren
%   Referenzfeldeintrags
%   (\Macro{varrefitemnameii})\IndexCmd{refitemnameii}
% \item[\Macro{refitemnameiii}] Bezeichnung eines frei definierbaren
%   Referenzfeldeintrags
%   (\Macro{varrefitemnameiii})\IndexCmd{refitemnameiii}
% \item[\Macro{specialmail}] Versandart
%  (\Macro{@specialmail})\IndexCmd{specialmail}\Index{Versandart}
% \item[\Macro{title}] "Uberschrift (\Macro{@title})\IndexCmd{title}
% \item[\Macro{subject}] Betreff, sprachabh"angig
%   (\Macro{@subject})\IndexCmd{subject}\Index{Betreff}
% \item[\Macro{firsthead}] Kopfzeilendefinition f"ur erste Seite
%  (\Macro{@firsthead})\IndexCmd{firsthead}
% \item[\Macro{firstfoot}] Fu"szeilendefinition f"ur erste Seite
%  (\Macro{@firstfoot})\IndexCmd{firstfoot}
% \item[\Macro{nexthead}] Kopfzeile der folgenden Seiten
%  (\Macro{@nexthead})\IndexCmd{nexthead}
% \item[\Macro{nextfoot}] Fu"szeile der folgenden Seiten
%  (\Macro{@nextfoot})\IndexCmd{nextfoot}
% \end{labeling}
% 
% \subsection{Briefspezifische L"angenangaben} \label{list:laengen}
% \Index{Laengenangaben=L"angenangaben}
% 
% Voreingestellte L"angen sind in Klammern angegeben.
% \begin{labeling}[~]{\Length{foldvskipiii}}
% \item[\Length{foldhskip}] Abstand der Falzmarke vom linken
%   Papierrand  (3,5\Unit{mm})\IndexLength{foldhskip}
% \item[\Length{foldvskipi}] Abstand zwischen der ersten Falzmarke
%   und dem oberen Seitenrand (62\Unit{mm})\IndexLength{foldvskipi}
% \item[\Length{foldvskipii}] Abstand der zweiten Falzmarke von
%   der ersten Falzmarke (45\Unit{mm})\IndexLength{foldvskipii}
% \item[\Length{foldvskipiii}] Abstand der dritten Falzmark von
%   der zweiten Falzmarke (54\Unit{mm})\IndexLength{foldvskipiii}
% \item[\Length{addvskip}] Abstand des Adressfensters von der
%   Textbereichsoberkante (7,5\Unit{mm})\IndexLength{addvskip}
% \item[\Length{addrindent}] Abstand des Adressfensters vom linken
%   Rand des Textbereiches (0\Unit{mm})\IndexLength{addrindent}
% \item[\Length{addrwidth}] Breite des Adressfeldes
%   (70\Unit{mm})\IndexLength{addrwidth}
% \item[\Length{locwidth}] Breite des \glqq Location\grqq -Feldes
%   [$(\Length{textwidth}-\Length{addrwidth})/2$ bei Option
%   \Option{slocfield} oder
%   $(\Length{textwidth}-\Length{addrwidth})*2/3$ bei Verwendung der
%   Option \Option{wlocfield}]\IndexLength{locwidth}
% \item[\Length{refvskip}] Abstand zwischen dem Referenzfelde und der
%   Adressfeldunterkante (5,5\Unit{mm})\IndexLength{refvskip}
% \item[\Length{sigindent}] Abstand der Gru"sformel und der
%   Unterschrift vom linken Rand des Textbereiches
%   (0\Unit{mm})\IndexLength{sigindent}
% \end{labeling}
% 
% 
% \subsection{Befehle zum Setzen interner Abst"ande}
% Voreingestellte L"angen sind in Klammern angegeben.
% \begin{labeling}[~]{\Macro{setpresigskip}}
% \item[\Macro{setpresigskip}] Abstand zwischen der Gru"sformel und der
%   Unterschrift voreingestellt sind
%   (2\Length{baselineskip})\IndexCmd{setpresigskip}
%   \Index{Grussformel=Gru"sformel}
% \end{labeling}
% 
% \subsection{Schalter}
% 
% An den jeweiligen Schalternamen ist noch ein \PValue{on} bzw.
% \PValue{off} anzuh"angen.
% 
% \begin{labeling}[~]{\Macro{subjectafter}}
% \item[\Macro{foldmarks}] Faltmarken (Default = on)
%   \IndexCmd{foldmarkson}\IndexCmd{foldmarksoff}\Index{Faltmarken}
% \item[\Macro{addrfield}] Adress- und \glqq Location\grqq -Feld
%   (Default = on)\Index{Adressfeld}
% \item[\Macro{subject}] \glqq Betreff: \grqq\ vor \Macro{subject}
%  (Default = off)\IndexCmd{subjecton}\IndexCmd{subjectoff}
% \item[\Macro{subjectafter}] Betreff nach der Anrede setzen
%   (Default =  off)
%   \IndexCmd{subjectafteron}\Index{Betreff}\IndexCmd{subjectafteroff}
% \end{labeling}
% 
% \subsection{Klassenoptionen}
% 
% Die Standartoptionen (\Option{12pt}, \Option{oneside},
% \Option{final}, \Option{slocfield}) k"onnen durch explizite
% Optionsangaben "uberschrieben werden.
% \begin{labeling}[~]{\Option{10pt}, \Option{11pt}, \Option{12pt}}
% \item[\Option{10pt}, \Option{11pt}, \Option{12pt}] Option f"ur die
%   Schriftgr"o"se\IndexOption{10pt}\IndexOption{11pt}\IndexOption{12pt}
% \item[\Option{oneside}] einseitiges Layout\IndexOption{oneside}
% \item[\Option{twoside}] pseudo-doppelseitiges
%   Layout\IndexOption{twoside}\IndexOption{twoside}
% \item[\Option{draft}] Dokumente im Entwurfsstadium
%   setzen\IndexOption{draft}
% \item[\Option{final}] Dokumente in der Endfassung
%   setzen\IndexOption{final}
% \item[\Option{a4paper}] Papiergr"o"se\IndexOption{a4paper}
% \item[\Option{wlocfield}] gro"ses \glqq
%   Location\grqq-Feld\IndexOption{wlocfield}
% \item[\Option{slocfield}] kleines \glqq
%   Location\grqq-Feld\IndexOption{wlocfield}
% \item[\Option{orgdate}] eigene Datumsanpassungen oder die
%   eines externen Pakets verwenden\IndexOption{orgdate}
% \item[\Option{scrdate}] Datumsanpassungen des
%   \Class{scrlettr}-Pakets verwenden.\IndexOption{scrdate}
% \end{labeling}
% 
% \EndIndex{Class}{scrlettr}
% \section{Autoren}
% \label{sec:scrletter.autoren} Die folgenden Autoren waren an
% dieser Anleitung beteiligt oder haben die Vorlage daf"ur geliefert.
% \begin{itemize}
% \item Markus Kohm
% \item \textbf{Enrico Kunz} \TextEMail{enricokunz@web.de}
% \item Jens-Uwe Morawski
% \end{itemize}
%
% \MakeShortVerb{\|}
%
% \StopEventually{\PrintIndex\PrintChanges}
%
% \part{"`scrlettr"'-class}
%
% \section{Implementierung}
%
%    \begin{macrocode}
%<*scrlettr>
%    \end{macrocode}
%\iffalse
%    \begin{macrocode}
\ClassWarningNoLine{scrlettr}{%
  THIS CLASS IS OBSOLETE AND NOT LONGER SUPPORTED!\MessageBreak
  Since the new KOMA-Script letter class ``scrlttr2'' was\MessageBreak
  released, the use of ``scrlettr'' is obsolete.\MessageBreak
  You should not use this class for writing new letters.\MessageBreak
  You should use ``scrlttr2''.\MessageBreak
  All old commands of ``scrlettr'' are supported at\MessageBreak
  the new class ``scrlttr2''. But the length are not\MessageBreak
  and the typearea was changed - even using the\MessageBreak
  compatiblity option ``KOMAold''. So you may have\MessageBreak
  to do some changes}
%    \end{macrocode}
%\fi
%
% \changes{v2.0}{1993/12/01}{Letzte "`script\_l"'-\LaTeX~2.0-Version
%                            von Frank Neukam}
% \changes{v2.0-2e}{1994/10/08}{"`script\_l"'-\LaTeXe-Version
%                               von Axel Kielhorn}
% \changes{v2.0e}{1994/10/12}{Erste "`scrlettr"'-Version im
%                             \textsf{KOMA-Script} Paket}
% \changes{v2.3e}{1996/05/31}{Faltmarken korrigiert}
% Die Implementierung von |scrlettr| stammt im wesentlichen von Frank Neukam.
% Axel Kielhorn hat sie nach \LaTeXe portiert, wobei die von Roland T. Lichti
% modifizierte Version zugrunde gelegt wurde. Markus Kohm hat nur einige
% wenige, unwesentliche "Anderungen vorgenommen.
%
% \subsection{Optionen}
%
% \changes{v2.4c}{1997/11/25}{Neuer Schalter \cs{if@orgdate} f"ur die neuen
%                             Optionen \texttt{orgdate}, \texttt{scrdate}}
%    \begin{macrocode}
\newcommand*\@ptsize{}
\newif\if@bigloc
\newif\if@orgdate
%    \end{macrocode}
%
% \subsubsection{Standardoptionen}
%
% \changes{v2.1a}{1994/10/29}{Meldung bei \texttt{twoside}-Option in
%                             "`scrlettr"' ge"andert}
%  \begin{option}{10pt}
%  \begin{option}{11pt}
%  \begin{option}{12pt}
%  \begin{option}{oneside}
%  \begin{option}{twoside}
%  \begin{option}{draft}
%  \begin{option}{final}
%  \begin{option}{a4paper}
%  \changes{v2.6a}{2001/05/24}{Neue Standardoption}  
%    \begin{macrocode}
\DeclareOption{10pt}{\renewcommand*\@ptsize{0}}
\DeclareOption{11pt}{\renewcommand*\@ptsize{1}}
\DeclareOption{12pt}{\renewcommand*\@ptsize{2}}
\DeclareOption{oneside}{\@twosidefalse \@mparswitchfalse}
\DeclareOption{twoside}{\@twosidetrue \@mparswitchtrue%
    \ClassWarningNoLine{scrlettr}{This is no twoside-layout but openright!}
}
\DeclareOption{draft}{\overfullrule 5pt}
\DeclareOption{final}{\setlength\overfullrule{0pt}}
\DeclareOption{a4paper}{
  \setlength{\paperwidth}{210mm}
  \setlength{\paperheight}{297mm}}
%    \end{macrocode}
%  \end{option}
%  \end{option}
%  \end{option}
%  \end{option}
%  \end{option}
%  \end{option}
%  \end{option}
%  \end{option}
%
% \subsubsection{Optionen f"ur das Adre"sfenster}
%
%  \begin{option}{wlocfield}
%  \begin{option}{slocfield}
% Mit Hilfe der Optione |wlocfield| kann der Platz neben dem Adre"sfeld
% vergr"o"sert werden. Mit |slocfield| wird der normale, kleine Platz
% verwendet.
%    \begin{macrocode}
\DeclareOption{wlocfield}{\@bigloctrue}
\DeclareOption{slocfield}{\@biglocfalse}
%    \end{macrocode}
%  \end{option}
%  \end{option}
%
% \subsubsection{Optionen f"ur das Datum}
%
% \begin{option}{orgdate}
% \changes{v2.4c}{1997/11/25}{Neue Option}
% \begin{option}{scrdate}
% \changes{v2.4c}{1997/11/25}{Neue Option}
%    \begin{macrocode}
\DeclareOption{orgdate}{\@orgdatetrue}
\DeclareOption{scrdate}{\@orgdatefalse}
%    \end{macrocode}
% \end{option}
% \end{option}
%
% \subsubsection{Optionenwahl}
%
% Standard sind bei |scrlettr| die Optionen |12pt|, |oneside|, |final| und
% |slocfield|. Diese k"onnen durch explizite Optionsangabe "uberschrieben
% werden.
%    \begin{macrocode}
\ExecuteOptions{12pt,oneside,final,slocfield,a4paper}
\ProcessOptions
\input{size1\@ptsize.clo}
%    \end{macrocode}
%
% \subsection{"`Alte"' Fontauswahlbefehle}
%
%  \begin{macro}{\rm}
%  \begin{macro}{\sf}
%  \begin{macro}{\tt}
%  \begin{macro}{\bf}
%  \begin{macro}{\it}
%  \begin{macro}{\sl}
%  \begin{macro}{\sc}
%  \begin{macro}{\sfb}
% \changes{v2.3b}{1996/01/14}{nicht mehr mathematisch.}
% \changes{v2.3b}{1996/01/14}{Keine Unterscheidung mehr f"ur den
%                             Kompatibilit"atsmodus.}
% Die alten Font-Auswahlbefehle werden zwar noch unterst"utzt, sollten
% aber in der Regel nicht mehr verwendet werden, da sie nach dem alten
% Fontauswahlverfahren arbeiten. Im Kompatibilit"atsmodus wurde dies
% beim nicht standardgem"a"sen Befehl |\sfb| noch verst"arkt.
%    \begin{macrocode}
\DeclareOldFontCommand{\rm}{\normalfont\rmfamily}{\mathrm}
\DeclareOldFontCommand{\sf}{\normalfont\sffamily}{\mathsf}
\DeclareOldFontCommand{\tt}{\normalfont\ttfamily}{\mathtt}
\DeclareOldFontCommand{\bf}{\normalfont\bfseries}{\mathbf}
\DeclareOldFontCommand{\it}{\normalfont\itshape}{\mathit}
\DeclareOldFontCommand{\sl}{\normalfont\slshape}{\@nomath\sl}
\DeclareOldFontCommand{\sc}{\normalfont\scshape}{\@nomath\sc}
\DeclareOldFontCommand{\sfb}{\normalfont\sffamily\bfseries}{\@nomath\sfb}
%    \end{macrocode}
%  \end{macro}
%  \end{macro}
%  \end{macro}
%  \end{macro}
%  \end{macro}
%  \end{macro}
%  \end{macro}
%  \end{macro}
%
% \subsection{Font-Variablen}
%
%  \begin{macro}{\descfont}
%  \begin{macro}{\sectfont}
% \changes{v2.3b}{1996/01/14}{\cs{sectfont} wird nun verwendet}
%  \begin{macro}{\pnumfont}
%  \begin{macro}{\headfont}
%  \begin{macro}{\capfont}
%  \begin{macro}{\caplabelfont}
% Auch in der |scrlettr|-class werden verschiedene Font-Variablen verwendet.
%    \begin{macrocode}
\newcommand*\descfont{\sffamily\bfseries}
\newcommand*\sectfont{\sffamily\bfseries}
\newcommand*\pnumfont{\normalfont}
\newcommand*\headfont{\slshape}
\newcommand*\capfont{\normalfont}
\newcommand*\caplabelfont{\normalfont}
%    \end{macrocode}
%  \end{macro}
%  \end{macro}
%  \end{macro}
%  \end{macro}
%  \end{macro}
%  \end{macro}
%
% \subsection{Standard-Labels}
%
% \changes{v2.2b}{1995/02/16}{Sprachauswahl an german.sty Version 2.5b
%                             angepa"st}
%  \begin{macro}{\captionsenglish}
%  \begin{macro}{\captionsUSenglish}
%  \begin{macro}{\captionsamerican}
% \changes{v2.4c}{1997/11/25}{american identisch mit USenglish definiert.}
%  \begin{macro}{\captionsbritish}
% \changes{v2.4c}{1997/11/25}{british identisch mit english definiert.}
%  \begin{macro}{\captionsUKenglish}
% \changes{v2.4c}{1997/11/25}{UKenglish identisch mit english definiert.}
%  \begin{macro}{\captionsgerman}
% \changes{v2.2a}{1994/01/26}{{\cmd\subjectname} korrigiert}
%  \begin{macro}{\captionsaustrian}
%  \begin{macro}{\captionsngerman}
% \changes{v2.5}{1999/09/08}{ngerman neu und identisch mit german.}
% \changes{v2.5b}{2000/01/03}{ngerman korrigiert.}
%  \begin{macro}{\captionsfrench}
%  \begin{macro}{\captionsitalian}
% \changes{v2.3e}{1996/05/31}{Sprachspende von Simone Naldi}
%  \begin{macro}{\captionsspanish}
% \changes{v2.4c}{1997/11/25}{Sprachspende von Ralph J. Hangleiter}
% Bei |scrlettr| gibt es eine gro"se Anzahl von Label-Variablen, die es
% in den Standard-classes nicht gibt, die also auch nicht in
% Sprachanpassungen wie |german.sty| oder |german3.sty| vorhanden sind.
% Deshalb ist es notwendig diese Labels hier neu zu definieren. Weil es
% in fr"uheren Versionen zu Problemen damit gekommen ist, werden diese
% jedoch erst bei |\begin{document}| definiert. Sollen sie nachtr"aglich
% ge"andert werden, mu"s dies \emph{nach} |\begin{document}| geschehen.
%
%    \begin{macrocode}
\AtBeginDocument{
 \def\captionsenglish{%
  \def\yourrefname{Your ref.}
  \def\yourmailname{Your letter of}
  \def\myrefname{Our ref.}
  \def\customername{Customer no.}
  \def\invoicename{Invoice no.}
  \def\subjectname{Subject}
  \def\ccname{cc}
  \def\enclname{encl}
  \def\headtoname{To}
  \def\datename{Date}
  \def\pagename{Page}}
 \let\captionsUSenglish=\captionsenglish
 \let\captionsamerican=\captionsUSenglish
 \let\captionsbritish=\captionsenglish
 \let\captionsUKenglish=\captionsenglish
 \def\captionsgerman{%
  \def\yourrefname{Ihr Zeichen}
  \def\yourmailname{Ihr Schreiben vom}
  \def\myrefname{Unser Zeichen}
  \def\customername{Kundennummer}
  \def\invoicename{Rechnungsnummer}
  \def\subjectname{Betr.}
  \def\ccname{Kopien an}
  \def\enclname{Anlagen}
  \def\headtoname{An}
  \def\datename{Datum}
  \def\pagename{Seite}}
 \let\captionsaustrian=\captionsgerman
 \let\captionsngerman=\captionsgerman
 \def\captionsfrench{%
  \def\yourrefname{Vos r\'ef\'erences}
  \def\yourmailname{Votre lettre du}
  \def\myrefname{Nos r\'ef\'erences}
  \def\customername{Num\'ero de client}
  \def\invoicename{Num\'ero de facture}
  \def\subjectname{Concernant}
  \def\ccname{Copie \`a}
  \def\enclname{Annexes}
  \def\headtoname{A}
  \def\datename{Date}
  \def\pagename{Page}}
 \def\captionsitalian{%
  \def\yourrefname{Vs./Rif.}
  \def\yourmailname{Vs.~lettera del}
  \def\myrefname{Ns./Rif.}
  \def\customername{Nr.~cliente}
  \def\invoicename{Nr.~fattura}
  \def\subjectname{Oggetto}
  \def\ccname{Per conoscenza}
  \def\enclname{Allegato}
  \def\headtoname{A}
  \def\datename{Data}
  \def\pagename{Pagina}}
 \def\captionsspanish{%
  \def\yourrefname{Su ref.}
   \def\yourmailname{Su carta de}
   \def\myrefname{Nuestra ref.}
   \def\customername{No. de cliente}
   \def\invoicename{No. de factura}
   \def\subjectname{Asunto}
   \def\ccname{Copias}
   \def\enclname{Adjunto}
   \def\headtoname{A}
   \def\datename{Fecha}
   \def\pagename{P\'agina}}
%    \end{macrocode}
%  \end{macro}
%  \end{macro}
%  \end{macro}
%  \end{macro}
%  \end{macro}
%  \end{macro}
%  \end{macro}
%  \end{macro}
%  \end{macro}
%  \end{macro}
%  \end{macro}
%
%  \begin{macro}{\dateenglish}
%  \begin{macro}{\dateUSenglish}
%  \begin{macro}{\dateamerican}
% \changes{v2.4c}{1997/11/25}{american identisch mit USenglish definiert.}
%  \begin{macro}{\datebritish}
% \changes{v2.4c}{1997/11/25}{british identisch mit english definiert.}
%  \begin{macro}{\dateUKenglish}
% \changes{v2.4c}{1997/11/25}{UKenglish identisch mit english definiert.}
%  \begin{macro}{\dategerman}
%  \begin{macro}{\dateaustrian}
%  \begin{macro}{\datengerman}
% \changes{v2.5}{1999/09/08}{ngerman neu und identisch mit german.}
%  \begin{macro}{\datefrench}
%  \begin{macro}{\dateitalian}
% \changes{v2.3e}{1996/05/31}{Ich hoffe, da"s das stimmt}
%  \begin{macro}{\datespanish}
% \changes{v2.4c}{1997/11/25}{Ich hoffe, da"s das stimmt}
% \changes{v2.4c}{1997/11/25}{Datumsumschaltung nicht mehr zwingend.}
% Dar"uber hinaus ist auch das Datumsformat sprachabh"angig. Dies wird
% hier ebenfalls beachtet.
%    \begin{macrocode}
 \if@orgdate
  \ifx\dateenglish\undefined
   \def\dateenglish{\def\today{\number\day/\number\month/\number\year}}
  \fi
 \else
  \def\dateenglish{\def\today{\number\day/\number\month/\number\year}}
  \def\dateUSenglish{\def\today{\number\month/\number\day/\number\year}}
  \let\datebritish=\dateenglish
  \let\dateUKenglish=\dateenglish
  \let\dateamerican=\dateUSenglish
  \def\dategerman{\def\today{\number\day.\,\number\month.\,\number\year}}
  \let\dateaustrian=\dategerman
  \let\datengerman=\dategerman
  \let\datefrench=\dategerman
  \let\dateitalian=\dategerman
  \let\datespanish=\dategerman
 \fi
%    \end{macrocode}
%  \end{macro}
%  \end{macro}
%  \end{macro}
%  \end{macro}
%  \end{macro}
%  \end{macro}
%  \end{macro}
%  \end{macro}
%  \end{macro}
%  \end{macro}
%  \end{macro}
% Zum Schlu"s findet noch die eigentliche Auswahl statt. Diese orientiert
% sich nun an der Auswahl nach german.sty 2.5b und verwendet keine festen
% Sprachzuordnungen mehr. Daf"ur sind nun keine Erweiterungen f"ur andere
% Sprachen mehr m"oglich.
% \changes{v2.2c}{1995/03/20}{Im Sprachenvergleich fehlten die "`="' hinter
%                             {\cmd\language}}
% \changes{v2.4c}{1997/11/25}{Sprachauswahl um american, british, UKenglish
%                             und spanish erweitert}
% \changes{v2.5}{1999/09/08}{Sprachauswahl um ngermen erweitert}
% \changes{v2.5b}{2000/01/20}{Reaktivierung der Sprache geschieht nun
%                             via \cs{languagename}, soweit dies
%                             m"oglich ist}
% \changes{v2.5e}{2000/07/14}{Workaround f"ur Sprache nohyphenation durch
%                             Format mit Babel-Erweiterung aber kein
%                             Babel package geladen}
%    \begin{macrocode}
 \captionsenglish
 \dateenglish
 \ifx\languagename\undefined
  \ClassWarningNoLine{scrlettr}{\string\languagename\space not
    defined, using \string\language.\MessageBreak
    This may result in use of wrong language!\MessageBreak
    You should use a compatible language
    package\MessageBreak
    (e.g. `Babel', `german', `french', ...)}
  \ifx\l@american\undefined\else\ifnum\language=\l@american
  \captionsamerican
  \dateamerican
  \fi\fi
  \ifx\l@british\undefined\else\ifnum\language=\l@british
  \captionsbritish
  \datebritish
  \fi\fi
  \ifx\l@UKenglish\undefined\else\ifnum\language=\l@UKenglish
  \captionsUKenglish
  \dateUKenglish
  \fi\fi
  \ifx\l@USenglish\undefined\else\ifnum\language=\l@USenglish
  \captionsUSenglish
  \dateUSenglish
  \fi\fi
  \ifx\l@austrian\undefined\else\ifnum\language=\l@austrian
  \captionsaustrian
  \dateaustrian
  \fi\fi
  \ifx\l@german\undefined\else\ifnum\language=\l@german
  \captionsgerman
  \dategerman
  \fi\fi
  \ifx\l@ngerman\undefined\else\ifnum\language=\l@ngerman
  \captionsngerman
  \datengerman
  \fi\fi
  \ifx\l@french\undefined\else\ifnum\language=\l@french
  \captionsfrench
  \datefrench
  \fi\fi
  \ifx\l@italian\undefined\else\ifnum\language=\l@italian
  \captionsitalian
  \dateitalian
  \fi\fi
  \ifx\l@spanish\undefined\else\ifnum\language=\l@spanish
  \captionsspanish
  \datespanish
  \fi\fi
 \else
   \edef\@tempa{nohyphenation}
   \ifx\languagename\@tempa
     \ClassWarningNoLine{scrlettr}
       {You've selected language ``\languagename''.\MessageBreak
        Maybe your LaTeX format contains Babel extension\MessageBreak
        but you have not selected a language using\MessageBreak
        Babel package.\MessageBreak
        Please select another language!\MessageBreak
        Only as a workaround english captions and date\MessageBreak
        will be used}
   \else
     \expandafter\selectlanguage\expandafter{\languagename}
     \ClassInfo{scrlettr}{used language is \languagename}
   \fi
 \fi
}
%    \end{macrocode}
%
% \subsection{Seitenspiegel}
%
% |scrlettr| arbeitet mit einem festen Seitenspiegel, der f"ur Briefe
% im A4-Format ausgelegt ist.
% \changes{v2.2c}{1995/05/25}{{\cmd\headheight} und {\cmd\textheight}
%                             ge"andert}
%
%    \begin{macrocode}
\oddsidemargin  0in
\evensidemargin 0in
\marginparwidth 0.9in
\marginparsep   0.1in
\marginparpush  0.45\baselineskip
\topmargin      -19mm % mk 941012, was: -15.5mm
\headheight     23mm  % mk 950411, was: 22mm
\headsep        8mm   % mk 941012, was: 9mm
\footskip       20mm  % mk 941012, was: 30mm
\textheight     226mm % mk 950411, was: 217mm
\textwidth      159.2mm

\parskip        0.5\baselineskip % mk 941012
\parindent      0pt
\smallskipamount=0.5\parskip
\medskipamount  =\parskip
\bigskipamount  =2\parskip
\footnotesep    0.8\baselineskip
\skip\footins   0.75\baselineskip plus 2pt minus 4pt
\skip\@mpfootins =\skip\footins
\columnsep      1cc
\columnseprule  0pt
%    \end{macrocode}
%
% \subsection{Flie"sumgebungen}
%
% Es folgen die Einstellungen f"ur Abbildungen
%    \begin{macrocode}
\floatsep       1\baselineskip plus 2pt minus 2pt
\textfloatsep   20pt plus 2pt minus 4pt
\intextsep      1\baselineskip plus 2pt minus 2pt
\dblfloatsep    1\baselineskip plus 2pt minus 2pt
\dbltextfloatsep 20pt plus 2pt minus 4pt
\@fptop 0pt plus 1fil
\@fpsep 0.7\baselineskip plus 2fil
\@fpbot 0pt plus 1fil
\@dblfptop 0pt plus 1fil
\@dblfpsep 0.7\baselineskip plus 2fil
\@dblfpbot 0pt plus 1fil
%    \end{macrocode}
% und f"ur Flie"stabellen
%    \begin{macrocode}
\arraycolsep    5pt
\tabcolsep      6pt
\arrayrulewidth 0.4pt
\doublerulesep  2pt
\fboxsep        3pt
\fboxrule       0.4pt
\tabbingsep \labelsep
%    \end{macrocode}
%
% \subsection{Listen-Umgebungen}
%
% Die Einstellungen f"ur Listenumgebungen entsprechen im Wesentlichen
% den gewohnten.
%    \begin{macrocode}
\topsep         0.25\baselineskip
\partopsep      0pt
\itemsep        0.25\baselineskip
\parsep         0.25\baselineskip % ak: 1\baselineskip
\labelsep       .5em
\leftmargini    2.5em
\leftmarginii   2.2em
\leftmarginiii  1.87em
\leftmarginiv   1.7em
\leftmarginv    1em
\leftmarginvi   1em
\leftmargin\leftmargini
\labelwidth\leftmargini
\advance\labelwidth-\labelsep
\def\@listI{\leftmargin\leftmargini}
\let\@listi\@listI
\def\@listii{\leftmargin\leftmarginii
 \labelwidth\leftmarginii\advance\labelwidth-\labelsep}
\def\@listiii{\leftmargin\leftmarginiii
 \labelwidth\leftmarginiii\advance\labelwidth-\labelsep}
\def\@listiv{\leftmargin\leftmarginiv
 \labelwidth\leftmarginiv\advance\labelwidth-\labelsep}
\def\@listv{\leftmargin\leftmarginv
 \labelwidth\leftmarginv\advance\labelwidth-\labelsep}
\def\@listvi{\leftmargin\leftmarginvi
 \labelwidth\leftmarginvi\advance\labelwidth-\labelsep}
\@listi

\@lowpenalty    51
\@medpenalty    151
\@highpenalty   301
\@beginparpenalty -\@lowpenalty
\@endparpenalty -\@lowpenalty
\@itempenalty   -\@lowpenalty

\def\theenumi{\arabic{enumi}}
\def\theenumii{\alph{enumii}}
\def\theenumiii{\roman{enumiii}}
\def\theenumiv{\Alph{enumiv}}
\def\labelenumi{\theenumi.}
\def\labelenumii{\theenumii)}
\def\labelenumiii{\theenumiii.}
\def\labelenumiv{\theenumiv.}
\def\p@enumii{\theenumi}
\def\p@enumiii{\theenumi\theenumii)}
\def\p@enumiv{\p@enumiii\theenumiii}
\def\labelitemi{$\bullet$}
\def\labelitemii{\bf --}
\def\labelitemiii{$\triangleright$}
\def\labelitemiv{$\cdot$}
%    \end{macrocode}
%
% Es sind alle im \textsf{KOMA-Script} Paket "ublichen Listenumgebungen
% vorhanden:
%
% \subsubsection{"`description"'-Umgebung}
%    \begin{macrocode}
\newenvironment{description}
               {\list{}{\labelwidth\z@ \itemindent-\leftmargin
                        \let\makelabel\descriptionlabel}}
               {\endlist}
\newcommand\descriptionlabel[1]{\hspace\labelsep
                                \descfont #1}
%    \end{macrocode}
%
% \subsubsection{"`labeling"'-Umgebung}
%    \begin{macrocode}
\newenvironment{labeling}[2][]
  {\def\sc@septext{#1}
    \list{}{\settowidth{\labelwidth}{#2#1}
            \leftmargin\labelwidth \advance\leftmargin by \labelsep
            \let\makelabel\labelinglabel}}
  {\endlist}
\newcommand\labelinglabel[1]{#1\hfil\sc@septext}
%    \end{macrocode}
%
% \subsubsection{"`verse"'-Umgebung}
%    \begin{macrocode}
\newenvironment{verse}
               {\let\\=\@centercr
                \list{}{\itemsep      \z@
                        \itemindent   -1.5em%
                        \listparindent\itemindent
                        \rightmargin  \leftmargin
                        \advance\leftmargin 1.5em}%
                \item[]}
               {\endlist}
%    \end{macrocode}
%
% \subsubsection{"`quotation"'- und "`quote"'-Umgebung}
%    \begin{macrocode}
\newenvironment{quotation}
               {\list{}{\listparindent 1em%
                        \itemindent    \listparindent
                        \rightmargin   \leftmargin
                        \parsep        \z@ \@plus\p@}%
                \item[]}
               {\endlist}
\newenvironment{quote}
               {\list{}{\rightmargin\leftmargin}%
                \item[]}
               {\endlist}
%    \end{macrocode}
%
% \subsection{Feld-Variablen}
%
%  \begin{macro}{\fromname}
%  \begin{macro}{\frombranch}
%  \begin{macro}{\fromsig}
% \changes{v2.2b}{1995/02/16}{Direkte Verwendung von {\cmd\fromname}}
%  \begin{macro}{\fromaddress}
%  \begin{macro}{\fromplace}
%  \begin{macro}{\fromlocation}
%  \begin{macro}{\frombackaddress}
%  \begin{macro}{\telephonenum}
%  \begin{macro}{\varyourref}
%  \begin{macro}{\varyourmail}
%  \begin{macro}{\varmyref}
%  \begin{macro}{\varcustomer}
%  \begin{macro}{\varinvoice}
%  \begin{macro}{\varrefitemi}
%  \begin{macro}{\varrefitemii}
%  \begin{macro}{\varrefitemiii}
%  \begin{macro}{\varrefitemnamei}
%  \begin{macro}{\varrefitemnameii}
%  \begin{macro}{\varrefitemnameiii}
%  \begin{macro}{\@specialmail}
%  \begin{macro}{\@title}
%  \begin{macro}{\@subject}
% Es gibt in |scrlettr| verschiedene Feld-Variablen, die alle mit
% einer Zeichenkette belegt werden k"onnen, aber als leer
% initialisiert werden.
%    \begin{macrocode}
\def\fromname{}
\def\frombranch{}                       % RTL 21.04.94
\def\fromsig{\fromname}
\def\fromaddress{}
\def\fromplace{}
\def\fromlocation{}
\def\frombackaddress{}
\def\telephonenum{}
\def\varyourref{}
\def\varyourmail{}
\def\varmyref{}
\def\varcustomer{}
\def\varinvoice{}
\def\varrefitemi{}
\def\varrefitemii{}
\def\varrefitemiii{}
\def\varrefitemnamei{}
\def\varrefitemnameii{}
\def\varrefitemnameiii{}
\def\@specialmail{}
\def\@title{}
\def\@subject{}
%    \end{macrocode}
%  \end{macro}
%  \end{macro}
%  \end{macro}
%  \end{macro}
%  \end{macro}
%  \end{macro}
%  \end{macro}
%  \end{macro}
%  \end{macro}
%  \end{macro}
%  \end{macro}
%  \end{macro}
%  \end{macro}
%  \end{macro}
%  \end{macro}
%  \end{macro}
%  \end{macro}
%  \end{macro}
%  \end{macro}
%  \end{macro}
%  \end{macro}
%  \end{macro}
%
%  \begin{macro}{\name}
%  \begin{macro}{\branch}
%  \begin{macro}{\signature}
%  \begin{macro}{\adress}
%  \begin{macro}{\place}
%  \begin{macro}{\location}
%  \begin{macro}{\backaddress}
%  \begin{macro}{\telephone}
%  \begin{macro}{\yourref}
%  \begin{macro}{\yourmail}
%  \begin{macro}{\myref}
%  \begin{macro}{\customer}
%  \begin{macro}{\invoice}
%  \begin{macro}{\refitemi}
%  \begin{macro}{\refitemii}
%  \begin{macro}{\refitemiii}
%  \begin{macro}{\refitemnamei}
%  \begin{macro}{\refitemnameii}
%  \begin{macro}{\refitemnameiii}
%  \begin{macro}{\specialmail}
%  \begin{macro}{\title}
%  \begin{macro}{\subject}
% F"ur jedes dieser Felder wird ein Befehl definiert, mit dem der
% Inhalt gesetzt werden kann, ohne da"s zu |\renewcommand| oder
% |\def| gegriffen werden mu"s. F"ur |\fromaddress| und
% |\fromlocation| kann mit |\address| bzw.
% |\location| auch eine komplexe \LaTeX-Definition
% gesetzt werden. F"ur die benutzerspezifischen Felder |\varrefitemi|,
% |\varrefitemii|, |\varrefitemiii| kann au"serdem jeweils ein Titel
% gesetzt werden.
%    \begin{macrocode}
\def\name#1     {\def\fromname{#1}}
\def\branch#1   {\def\frombranch{#1}}              % RTL 21.04.94
\def\signature#1{\def\fromsig{#1}}
\long\def\address#1{\def\fromaddress{#1}}
\def\place#1    {\def\fromplace{#1}}
\long\def\location#1{\def\fromlocation{#1}}
\def\backaddress#1{\def\frombackaddress{#1}}
\def\telephone#1{\def\telephonenum{#1}}
\def\yourref#1  {\def\varyourref{#1}}
\def\yourmail#1 {\def\varyourmail{#1}}
\def\myref#1    {\def\varmyref{#1}}
\def\customer#1 {\def\varcustomer{#1}}
\def\invoice#1  {\def\varinvoice{#1}}
\def\refitemi#1 {\def\varrefitemi{#1}}
\def\refitemii#1{\def\varrefitemii{#1}}
\def\refitemiii#1{\def\varrefitemiii{#1}}
\def\refitemnamei#1{\def\varrefitemnamei{#1}}
\def\refitemnameii#1{\def\varrefitemnameii{#1}}
\def\refitemnameiii#1{\def\varrefitemnameiii{#1}}
\def\specialmail#1{\def\@specialmail{#1}}
\def\title#1    {\def\@title{#1}}
\def\subject#1  {\def\@subject{{\bf #1}}}
%    \end{macrocode}
%  \end{macro}
%  \end{macro}
%  \end{macro}
%  \end{macro}
%  \end{macro}
%  \end{macro}
%  \end{macro}
%  \end{macro}
%  \end{macro}
%  \end{macro}
%  \end{macro}
%  \end{macro}
%  \end{macro}
%  \end{macro}
%  \end{macro}
%  \end{macro}
%  \end{macro}
%  \end{macro}
%  \end{macro}
%  \end{macro}
%  \end{macro}
%  \end{macro}
%
%  \begin{macro}{\firsthead}
%  \begin{macro}{\firstfoot}
%  \begin{macro}{\nexthead}
%  \begin{macro}{\nextfoot}
% Bei der Gelegenheit werden auch gleich die Befehle zum Setzen
% der Felder f"ur die Kopf- und Fu"szeilen der ersten und aller
% weiterer Seiten eines Briefes definiert.
%    \begin{macrocode}
\long\def\firsthead#1{\def\@firsthead{#1}}
\long\def\firstfoot#1{\def\@firstfoot{#1}}
\long\def\nexthead#1{\def\@nexthead{#1}}
\long\def\nextfoot#1{\def\@nextfoot{#1}}
%    \end{macrocode}
%  \end{macro}
%  \end{macro}
%  \end{macro}
%  \end{macro}
%
% \subsection{Faltmarken, Adre"sfeld, Titel}
%
% Es werden Schalter f"ur Faltmarken, Adre"sfeld und Betreff definiert.
%    \begin{macrocode}
\newif\if@fold
\newif\if@afield
\newif\if@subj
\newif\if@subjafter
%    \end{macrocode}
%
%  \begin{macro}{\foldmarkson}
%  \begin{macro}{\foldmarksoff}
% Die Faltmarken k"onnen mit |\foldmarkson| ein- und mit |\foldmarksoff|
% abgeschaltet werden. Dies ist f"ur jeden Brief getrennt m"oglich.
%    \begin{macrocode}
\def\foldmarkson        {\@foldtrue}
\def\foldmarksoff       {\@foldfalse}
%    \end{macrocode}
%  \end{macro}
%  \end{macro}
%
%  \begin{macro}{\addrfieldon}
%  \begin{macro}{\addrfieldoff}
% Das Adre"sfeld f"ur Fensterumschl"age kann mit |\addrfieldon| ein- und
% mit |\addrfieldoff| abgeschaltet werden. Dies ist f"ur jeden Brief
% getrennt m"oglich.
%    \begin{macrocode}
\def\addrfieldon        {\@afieldtrue}
\def\addrfieldoff       {\@afieldfalse}
%    \end{macrocode}
%  \end{macro}
%  \end{macro}
%
%  \begin{macro}{\subjecton}
%  \begin{macro}{\subjectoff}
% Ein Betreff-Titel kann mit |\subjecton| ein- und mit |\subjectoff|
% abgeschaltet werden. Dies ist f"ur jeden Brief getrennt m"oglich.
%    \begin{macrocode}
\def\subjecton          {\@subjtrue}
\def\subjectoff         {\@subjfalse}
%    \end{macrocode}
%  \end{macro}
%  \end{macro}
%
%  \begin{macro}{\subjectafteron}
%  \begin{macro}{\subjectafteroff}
% \changes{v2.4a}{1997/06/06}{Neue Befehle \cs{subjectafteron} und
%                             \cs{subjectafteroff}.}
% In England und Frankreich scheint es teilweise "ublich zu sein, den Betreff
% erst nach der Anrede und daf"ur zentriert zu setzen\footnote{Ich kann mir
% nur m"uhsahm eine Bemerkung dazu verkneifen.}. Deshalb gibt es zwei neue
% Befehle mit denen man f"ur jeden Brief getrennt etwas anderes einschalten
% kann.
%    \begin{macrocode}
\newcommand*\subjectafteron {\@subjaftertrue}
\newcommand*\subjectafteroff{\@subjafterfalse}
%    \end{macrocode}
%  \end{macro}
%  \end{macro}
%
% Zu Berechnung der Feldpositionen werden verschiedene
% Dimensions-Variablen ben"otigt.
%    \begin{macrocode}
\newdimen\sc@temp
\newdimen\sc@@temp
\newdimen\foldhskip
\newdimen\foldvskipi
\newdimen\foldvskipii
\newdimen\foldvskipiii
\newdimen\addrvskip
\newdimen\addrindent
\newdimen\addrwidth
\newdimen\addrheight
\newdimen\locwidth
\newdimen\refvskip
\newdimen\sigindent
%    \end{macrocode}
%
% F"ur alle Positionen gibt es Standardwerte. Diese Stellen sind leider
% extrem druckerabh"angig. Es ist daher "au"serst wichtig, da"s der
% Druckertreiber richtig eingestellt wird. Dazu kann die Testseite aus
% dem Standardpaket verwendet werden.
%
% Die Breite des "`locfield"'s h"angt von den Optionen |wlocfield| und
% |slocfield| ab und wird entweder zu $2/3$ oder $1/2$ der Restbreite
% neben dem Adre"sfeld berechnet.
%
%    \begin{macrocode}
\foldhskip      3.5mm
\foldvskipi     62mm    % war: 65mm RTL
\foldvskipii    45mm    % mk 960531 (war: 40.5mm)
\foldvskipiii   54mm    % mk 960531 (war: 58.5mm)
\addrvskip      7.5mm
\addrindent     0mm
\addrwidth      70mm
\addrheight     35mm
\locwidth\textwidth
\advance\locwidth by -\addrwidth
\if@bigloc                           % mk 940330
    \advance\locwidth by \locwidth
    \divide\locwidth by 3
\else
    \divide\locwidth by 2
\fi
\refvskip       5.5mm
\sigindent      0mm
%    \end{macrocode}
%
% \subsection{Serienbriefe und Adre"sdateien}
%
%  \begin{macro}{\adrentry}
% \changes{v2.2b}{1995/05/25}{{\cmd\ifx} sicherer gemacht}
%  \begin{macro}{\adrchar}
% Serienbriefe werden mit Hilfe der Funktionen |\adrentry| und
% |\adrchar| und einer Adressdatei realisiert. Dar"uber hinaus
% werden mit diesen Befehlen Abk"urzungen f"ur Adressen definiert.
%    \begin{macrocode}
\def\adrentry#1#2#3#4#5#6#7#8{\def\@tempa{#1}\ifx \@tempa\@empty \else
 \def\@tempa{#2}\ifx \@tempa\@empty
  \expandafter\def\csname #8\endcsname{#1\\#3}
 \else
  \expandafter\def\csname #8\endcsname{#2 #1\\#3}
 \fi \fi}
\def\adrchar#1{}
%    \end{macrocode}
%  \end{macro}
%  \end{macro}
%
%
% \subsection{Die Brief-Umgebung}
%
%  \begin{environment}{letter}
%  \begin{macro}{\stopletter}
% Diese Umgebung ist etwas anders definiert, als normalerweise von
% \LaTeX gewohnt. Mit ihr k"onnen innerhalb eines Dokuments beliebig
% viele Briefe erzeugt werden.
%    \begin{macrocode}
\long\def\letter#1{\newpage
 \if@twoside\ifodd\c@page\else \thispagestyle{empty}\null\newpage \fi\fi
 \c@page\@ne \interlinepenalty=200 \@processto{#1}}
\def\stopletter{}
\def\endletter{\stopletter\@@par\pagebreak\@@par}
\long\def\@processto#1{\expandafter\@xproc #1\\@@@\ifx\toaddress\@empty
 \else\expandafter\@yproc #1@@@\fi}
\long\def\@xproc #1\\#2@@@{\def\toname{#1}\def\toaddress{#2}}
\long\def\@yproc #1\\#2@@@{\def\toaddress{#2}}
%    \end{macrocode}
%  \end{macro}
%  \end{environment}
%
%  \begin{macro}{\stopbreaks}
%  \begin{macro}{\startbreaks}
% Innerhalb von Briefen wird der Umbruch etwas anders gehandhabt.
% Dadurch soll der Brieftext besser positioniert werden.
%    \begin{macrocode}
\def\stopbreaks{\interlinepenalty \@M
 \def\par{\@@par\nobreak}\let\\=\@nobreakcr
 \let\vspace\@nobreakvspace}
\def\@nobreakvspace{\@ifstar{\@nobreakvspacex}{\@nobreakvspacex}}
\def\@nobreakvspacex#1{\ifvmode\nobreak\vskip #1\relax\else
 \@bsphack\vadjust{\nobreak\vskip #1}\@esphack\fi}
\def\@nobreakcr{\vadjust{\penalty\@M}\@ifstar{\@xnewline}{\@xnewline}}
\def\startbreaks{\let\\=\@normalcr
 \interlinepenalty 200\def\par{\@@par\penalty 200}}
%    \end{macrocode}
%  \end{macro}
%  \end{macro}
%
%  \begin{macro}{\@foldmarks}
% F"ur die Faltmarken m"ussen nicht nur Abst"ande definiert werden, es
% wird auch ein Befehl ben"otigt, um die Faltmarken zu setzen.
%    \begin{macrocode}
\def\@foldmarks{\if@fold \bgroup
 \reversemarginpar\vspace{\foldvskipi}
 \marginpar{\hspace{\foldhskip}\rule{2mm}{.2pt}} \vspace{\foldvskipii}
 \marginpar{\hspace{\foldhskip}\rule{4mm}{.2pt}} \vspace{\foldvskipiii}
 \marginpar{\hspace{\foldhskip}\rule{2mm}{.2pt}}
 \vspace{-\foldvskipiii}\vspace{-\foldvskipii}\vspace{-\foldvskipi}
 \egroup \fi}
%    \end{macrocode}
%  \end{macro}
%
%  \begin{macro}{\@addrfield}
% Dasselbe gilt auch f"ur das Adre"sfeld f"ur Fensterumschl"age. Die
% R"ucksendeadresse und die Briefart wird dabei unterstrichen.
%    \begin{macrocode}
\def\@addrfield{\bgroup
 \setbox0\vbox{\hsize\addrwidth
  \ifx\frombackaddress\@empty \else
  \underline{\scriptsize \sf \frombackaddress} \fi}
 \setbox1\vbox{\hsize\addrwidth
  \ifx\@specialmail\@empty \else
   \underline{\@specialmail} \fi}
 \setbox2\vbox{\hsize\addrwidth
  \toname \\ \toaddress}
 \vskip\addrvskip \hskip\addrindent
 \vbox to \addrheight{%
  \ifx\frombackaddress\@empty \else \box0 \fi
  \ifx\@specialmail\@empty \else \vfil\box1 \fi
  \vfil\box2\vfil}\egroup}
%    \end{macrocode}
%  \end{macro}
%
%  \begin{macro}{\@locfield}
% Das "`locfield"' wird mittels einer |minipage| gesetzt.
%    \begin{macrocode}
\def\@locfield{\begin{minipage}[b]{\locwidth}\fromlocation\end{minipage}}
%    \end{macrocode}
%  \end{macro}
%
%  \begin{macro}{\@datefield}
% Je nachdem, welche Felder definiert sind, mu"s eine gesch"aftsm"a"siger
% Standardbrief oder ein eher pers"onliches Aussehen gew"ahlt werden.
% \changes{v2.3e}{1996/05/31}{Abstand zwischen Ort und Datum eingef"ugt}
%    \begin{macrocode}
\newif\if@ref
\def\@datefield{\@reffalse
 \ifx\varyourref\@empty \else \@reftrue \fi
 \ifx\varyourmail\@empty \else \@reftrue \fi
 \ifx\varmyref\@empty \else \@reftrue \fi
 \ifx\varcustomer\@empty \else \@reftrue \fi
 \ifx\varinvoice\@empty \else \@reftrue \fi
 \ifx\varrefitemi\@empty \else \@reftrue \fi
 \ifx\varrefitemii\@empty \else \@reftrue \fi
 \ifx\varrefitemiii\@empty \else \@reftrue \fi
 \vskip -\parskip
 \vskip \refvskip
 \if@ref
  \ifx\varyourref\@empty \else
   \settowidth{\sc@temp}{\varyourref}
   \setbox0\hbox{\scriptsize \sf \yourrefname}
   \ifdim \sc@temp > \wd0 \sc@@temp\sc@temp \else \sc@@temp\wd0 \fi
   \parbox[t]{\sc@@temp}{\noindent \box0\par \varyourref}\hfill
  \fi
  \ifx\varyourmail\@empty \else
   \settowidth{\sc@temp}{\varyourmail}
   \setbox0\hbox{\scriptsize \sf \yourmailname}
   \ifdim \sc@temp > \wd0 \sc@@temp\sc@temp \else \sc@@temp\wd0 \fi
   \parbox[t]{\sc@@temp}{\noindent \box0\par \varyourmail}\hfill
  \fi
  \ifx\varmyref\@empty \else
   \settowidth{\sc@temp}{\varmyref}
   \setbox0\hbox{\scriptsize \sf \myrefname}
   \ifdim \sc@temp > \wd0 \sc@@temp\sc@temp \else \sc@@temp\wd0 \fi
   \parbox[t]{\sc@@temp}{\noindent \box0\par \varmyref}\hfill
  \fi
  \ifx\varcustomer\@empty \else
   \settowidth{\sc@temp}{\varcustomer}
   \setbox0\hbox{\scriptsize \sf \customername}
   \ifdim \sc@temp > \wd0 \sc@@temp\sc@temp \else \sc@@temp\wd0 \fi
   \parbox[t]{\sc@@temp}{\noindent \box0\par \varcustomer}\hfill
  \fi
  \ifx\varinvoice\@empty \else
   \settowidth{\sc@temp}{\varinvoice}
   \setbox0\hbox{\scriptsize \sf \invoicename}
   \ifdim \sc@temp > \wd0 \sc@@temp\sc@temp \else \sc@@temp\wd0 \fi
   \parbox[t]{\sc@@temp}{\noindent \box0\par \varinvoice}\hfill
  \fi
  \ifx\varrefitemi\@empty \else
   \settowidth{\sc@temp}{\varrefitemi}
   \setbox0\hbox{\scriptsize \sf \varrefitemnamei}
   \ifdim \sc@temp > \wd0 \sc@@temp\sc@temp \else \sc@@temp\wd0 \fi
   \parbox[t]{\sc@@temp}{\noindent \box0\par \varrefitemi}\hfill
  \fi
  \ifx\varrefitemii\@empty \else
   \settowidth{\sc@temp}{\varrefitemii}
   \setbox0\hbox{\scriptsize \sf \varrefitemnameii}
   \ifdim \sc@temp > \wd0 \sc@@temp\sc@temp \else \sc@@temp\wd0 \fi
   \parbox[t]{\sc@@temp}{\noindent \box0\par \varrefitemii}\hfill
  \fi
  \ifx\varrefitemiii\@empty \else
   \settowidth{\sc@temp}{\varrefitemiii}
   \setbox0\hbox{\scriptsize \sf \varrefitemnameiii}
   \ifdim \sc@temp > \wd0 \sc@@temp\sc@temp \else \sc@@temp\wd0 \fi
   \parbox[t]{\sc@@temp}{\noindent \box0\par \varrefitemiii}\hfill
  \fi
  \ifx\@date\@empty \else
   \settowidth{\sc@temp}{\@date}
   \setbox0\hbox{\scriptsize \sf \datename}
   \ifdim \sc@temp > \wd0 \sc@@temp\sc@temp \else \sc@@temp\wd0 \fi
   \parbox[t]{\sc@@temp}{\noindent \box0\par \@date}
  \fi
  \par
  \vspace{2.5\baselineskip}
 \else
  {\ifx\@date\@empty \else \raggedleft\fromplace\ \@date\par \fi}
  \vspace{1.5\baselineskip}
 \fi}
%    \end{macrocode}
%  \end{macro}
%
%  \begin{macro}{\maketitle}
% \changes{v2.3b}{1996/01/14}{Verwendung von \cs{sectfont} f"ur den
%                             \cs{title}.}
%  \begin{macro}{\@subjfield}
% \changes{v2.6d}{2001/10/19}{\cs{centerline} durch \cs{centering}
%                             ersetzt} 
% Der eigentliche Titel wir mit |\maketitle| gesetzt. Dies sollte
% jedoch ebenso wie der Betreff \emph{nicht} manuell geschehen.
%    \begin{macrocode}
\def\maketitle{\ifx\@title\@empty \else
 {\centering \LARGE \sectfont \@title\par}
 \vspace{1\baselineskip} \fi}
\def\@subjfield{{%
 \if@subjafter%
  \centering%
 \else%
  \if@subj\ifx\@subject\@empty\else\subjectname:\ \fi\fi
  \@subject\par\nobreak\vspace{1\baselineskip}%
 \fi}}
%    \end{macrocode}
%  \end{macro}
%  \end{macro}
%
%  \begin{macro}{\opening}
% Denn beim Er"offnungsgru"s mit |\opening| wird neben s"amtlichen
% Feldern auch der Titel und der Betreff gesetzt.
%    \begin{macrocode}
\def\opening#1{\thispagestyle{firstpage} \null \@foldmarks
 \if@afield \@addrfield \hfill \@locfield \par \fi
 \@datefield
 \maketitle
 \if@subjafter\else\@subjfield\fi
 #1\par%
 \vspace{0.5\baselineskip}% added by unknown
 \nobreak%
 \if@subjafter\@subjfield\fi}
%    \end{macrocode}
%  \end{macro}
%
%  \begin{macro}{\closing}
% \changes{v2.2b}{1995/02/16}{{\cmd\fromsig} wird nicht mehr auf {\cmd\empty}
%                            getestet sondern direkt eingesetzt}
% \changes{v2.3b}{1996/01/14}{Abstand zwischen Schlu"sgru"s und Signatur
%                            erh"oht}
% \changes{v2.5d}{2000/06/10}{Abstand zwischen Schlu"sgru"s und
%                            Signatur variabel}
% Au"ser dem Er"offnungsgru"s mit |\opening| gibt es nat"urlich auch
% einen Schlu"sgru"s mit |\closing|, bei dem au"serdem die
% Unterschrift u."a. gesetzt wird.
%  \begin{macro}{\presig@skip}
% \changes{v2.5d}{2000/06/10}{Neues Makro, das den Abstand zwischen
%                            Schlu"sgru"s und Signatur beinhaltet}
%  \begin{macro}{\setpresigskip}
% \changes{v2.5d}{2000/06/10}{Neues Makro, das den Abstand zwischen
%                            Schlu"sgru"s und Signatur setzt}
% Das Makro |\presig@skip| beinhaltet dabei den Abstand zwischen Schlu"sgru"s
% und Signatur. Da der Wert "uber ein Benutzerinterface per |\setpresigskip|
% gesetzt wird, mu"s hier kein kostbares L"angenregister (skip oder dim)
% verschwendet werden, sondern ein Makro reicht aus. In |\setpresigskip| wird
% allerdings noch sichergestellt, da"s die Fehlermeldung bei falschem
% Parameter m"oglichst nicht erst bei Verwendung auftritt.
%    \begin{macrocode}
\newcommand*\presig@skip{}
\newcommand*{\setpresigskip}[1]{%
  \begingroup%
    \setlength{\@tempdima}{#1}%
  \endgroup%
  \edef\presig@skip{#1}}
\setpresigskip{2\baselineskip}
\long\def\closing#1{\par\nobreak\vspace{0.5\baselineskip}
 \stopbreaks \noindent \sc@temp\textwidth \advance\sc@temp by
 -\sigindent \hspace{\sigindent}%
 \parbox{\sc@temp}{\raggedright\ignorespaces #1\mbox{}\\[\presig@skip]
 \fromsig\strut}\par
 \vspace{1.5\baselineskip}}
%    \end{macrocode}
%  \end{macro}
% \end{macro}
% \end{macro}
%
%  \begin{macro}{\ps}
% Ebenfalls m"oglich ist ein Postscriptum. Dieser Befehl erwartet
% das Postscriptum nicht als Argument, sondern schaltet zum Nachtext
% um.
%    \begin{macrocode}
\def\ps{\par\startbreaks}
%    \end{macrocode}
%  \end{macro}
%
%  \begin{macro}{\cc}
% Dar"uber hinaus gibt es noch einen Verteiler. Dieser wird
% sauber formatiert.
% \changes{v2.6}{2001/01/03}{\cs{def} durch \cs{newcommand*} ersetzt.}
%  \begin{macro}{\ccnameseparator}
% \changes{v2.6}{2001/01/03}{Die Zeichen nach \cs{ccname} kann frei gew"ahlt
%                            werden. Voreinstellung ist wie bisher Doppelpunkt
%                            gefolgt von einem Leerzeichen.}
%    \begin{macrocode}
\newcommand*{\ccnameseparator}{: }
\newcommand*{\cc}[1]{\par\noindent\parbox[t]{\textwidth}
 {\@hangfrom{\ccname\ccnameseparator}%
 \ignorespaces #1\strut}\par}
%    \end{macrocode}
%  \end{macro}
%  \end{macro}
%
%  \begin{macro}{\encl}
% Auch Anlagen werden ordentlich gesetzt.
% \changes{v2.4a}{1997/06/06}{Der Doppelpunkt bei den Anlagen wird nur noch
%                             gesetzt, wenn \cs{enclname} nicht leer ist.}
% \changes{v2.6}{2001/01/03}{\cs{def} durch \cs{newcommand*} ersetzt.}
%  \begin{macro}{\enclnameseparator}
% \changes{v2.6}{2001/01/03}{Die Zeichen nach \cs{enclname} kann frei gew"ahlt
%                            werden. Voreinstellung ist wie bisher Doppelpunkt
%                            gefolgt von einem Leerzeichen.}
%    \begin{macrocode}
\newcommand*{\enclnameseparator}{: }
\newcommand*{\encl}[1]{\par\noindent%
 \parbox[t]{\textwidth}{%
  \ifx\enclname\@empty\else\@hangfrom{\enclname\enclnameseparator}\fi%
  \ignorespaces #1\strut}\par}
%    \end{macrocode}
%  \end{macro}
%  \end{macro}
%
%  \begin{macro}{\footnoterule}
% Fu"snoten werden durch eine Linie abgetrennt. Sie werden wie im gesamten
% \textsf{KOMA-Script} Paket gewohnt formatiert.
%    \begin{macrocode}
\def\footnoterule{\kern-1\p@
 \hrule width 0.4\columnwidth
 \kern .6\p@}
%    \end{macrocode}
%  \end{macro}
%  \begin{macro}{\deffootnote}
% \changes{v2.4b}{1997/08/15}{Neues Makro zur Definition der Gestalt von
%                             Fu"snoten.}
% Dieses Makro erlaubt einen optionalen und erwartet drei weitere Parameter.
% Der erste, optionale gibt den Einzug der ersten Zeile des Fu"snotentextes
% vom linken Rand an. Die Fu"snotenmarkierungen werden rechtsb"undig in diesen
% Einzug gesetzt. Der zweite, also erste nicht optionale Parameter gibt den
% Einzug jeder weiteren Zeile des Fu"snotentextes vom linken Rand an. Fehlt
% der optionale Parameter so ist er gleich diesem. Der dritte, also zweite
% nicht optionale Parameter gibt den zus"atzlichen Einzug jedes weiteren
% Absatzes einer Fu"snote an. Der letzte Parameter schlie"slich bestimmt die
% Ausgabe der Fu"snotenmarkierung in der Fu"snote. Diese wird zus"atzlich in 
% eine \verb|\hbox| gesetzt.
%    \begin{macrocode}
\newcommand\deffootnote[4][]{%
  \long\def\@makefntext##1{%
    \edef\@tempa{#1}\ifx\@tempa\@empty
      \@setpar{\@@par
        \@tempdima = \hsize
        \addtolength{\@tempdima}{-#2}
        \parshape \@ne #2 \@tempdima}%
    \else
      \@setpar{\@@par
        \@tempdima = \hsize
        \addtolength{\@tempdima}{-#1}
        \@tempdimb = \hsize
        \addtolength{\@tempdimb}{-#2}
        \parshape \tw@ #1 \@tempdima #2 \@tempdimb}%
    \fi
    \par
    \parindent #3\noindent
    \hbox to \z@{\hss\@@makefnmark}##1}
%    \end{macrocode}
%  \begin{macro}{\@@makefnmark}
% \changes{v2.4b}{1997/08/15}{Neues Makro zum Setzen der Fu"snotenmarkierung
%                             im Text}
%    \begin{macrocode}
  \def\@@makefnmark{\hbox{#4}}
%    \end{macrocode}
%  \end{macro}
%    \begin{macrocode}
}
%    \end{macrocode}
%  \end{macro}
%
%  \begin{macro}{\deffootnotemark}
% \changes{v2.4b}{1997/08/15}{Neues Makro zur Definition der
%                             Fu"snotenmarkierung im Text}
%    \begin{macrocode}
\newcommand*\deffootnotemark[1]{\def\@makefnmark{\hbox{#1}}}
%    \end{macrocode}
%  \end{macro}
%
%  \begin{macro}{\thefootnotemark}
% \changes{v2.4b}{1997/08/15}{Neues Makro, damit \cs{@thefnmark} auf
%                             Anwenderebene verf"ugbar wird.}
%    \begin{macrocode}
\def\thefootnotemark{\@thefnmark}
%    \end{macrocode}
%  \end{macro}
%  \begin{macro}{\textsuperscript}
% \changes{v2.4b}{1997/08/15}{Neues Makro, damit \cs{@textsuperscript} auf
%                             Anwenderebene verf"ugbar wird.}
%    \begin{macrocode}
\let\textsuperscript\@textsuperscript
%    \end{macrocode}
%  \end{macro}
% \changes{v2.4b}{1997/08/15}{Verwendung der neuen Makros zur
%                             Fu"snotengestaltgebung.}
%  \begin{macro}{\@makefnmark}
%    \begin{macrocode}
\deffootnote[1em]{1.5em}{1em}
  {\textsuperscript{\thefootnotemark}}
\deffootnotemark{\textsuperscript{\thefootnotemark}}
%    \end{macrocode}
%  \end{macro}
%
% Aufz"ahlungen werden mit arabischen Zahlen numeriert.
%    \begin{macrocode}
\def\theequation{\arabic{equation}}
%    \end{macrocode}
% \subsection{Verteilung von Text- und Flie"sumgebungen}
%    \begin{macrocode}
\setcounter{topnumber}{2}
\def\topfraction{.7}
\setcounter{bottomnumber}{1}
\def\bottomfraction{.3}
\setcounter{totalnumber}{3}
\def\textfraction{.2}
\def\floatpagefraction{.5}
\setcounter{dbltopnumber}{2}
\def\dbltopfraction{.7}
\def\dblfloatpagefraction{.5}
%    \end{macrocode}
%
% \subsection{Seitenstil}
%
%  \begin{macro}{\pagestyle}
% \changes{v2.4a}{1997/06/06}{\cs{fromname} wird nur noch dann gesetzt, wenn
%                             es nicht leer ist, wodurch \cs{name} genau wie
%                             bei der Standard-Letter-Class optional ist.}
% Neben den "ublichen Seitenstilen |plain|, |headings|, |myheadings|, |empty|
% gibt es bei |scrlettr| noch einen Stil f"ur die erste Seite eines Briefs.
%    \begin{macrocode}
\def\@firsthead{%
  \parbox[b]{\textwidth}
    {\begin{center}
       \ifx\fromname\@empty\else
         \textsc{\ignorespaces\fromname}\\[-8pt]
         \rule{\textwidth}{.4pt}\\
       \fi
       \ifx\fromaddress\@empty\else
         \ignorespaces\fromaddress
       \fi
     \end{center}}}
\def\@firstfoot{}
\def\@nexthead{%
  \parbox[b]{\textwidth}
  {\ifx\fromname\@empty\else
     \textsc{\ignorespaces\fromname}\\[8pt]
   \fi
   \headtoname\ \ignorespaces\toname\hfill\datename:\ \@date\hfill
   \pagename\ \thepage\\[-8pt]
   \rule{\textwidth}{.4pt}}}
\def\@nextfoot{}

\def\ps@plain{%
  \def\@evenhead{\sffamily\bfseries\fontsize{10pt}{12}\selectfont
  \hfil-- \thepage\ --\hfil}%
 \let\@oddhead\@evenhead%
 \def\@evenfoot{}%
 \def\@oddfoot{}}
\def\ps@firstpage{%
 \def\@evenhead{\@firsthead}%
 \let\@oddhead\@evenhead%
 \def\@evenfoot{\@firstfoot}%
 \let\@oddfoot\@evenfoot}
\def\ps@headings{%
 \def\@evenhead{\@nexthead}%
 \let\@oddhead\@evenhead%
 \def\@evenfoot{\@nextfoot}%
 \let\@oddfoot\@evenfoot}
%    \end{macrocode}
%  \end{macro}
%
% \subsection{Voreinstellungen}
%
% Voreingestellt sind Faltmarken,
%    \begin{macrocode}
\@foldtrue
%    \end{macrocode}
% Adre"sfeld,
%    \begin{macrocode}
\@afieldtrue
%    \end{macrocode}
% kein Betrefftitel,
%    \begin{macrocode}
\@subjfalse
%    \end{macrocode}
% Betreff vor der der Anrede,
%    \begin{macrocode}
\@subjafterfalse
%    \end{macrocode}
% Einfache Kopfzeilen,
%    \begin{macrocode}
\ps@plain
%    \end{macrocode}
% Seitennumerierung mit arabischen Zahlen,
%    \begin{macrocode}
\pagenumbering{arabic}
%    \end{macrocode}
% kein Abgleich des unteren Randes
%    \begin{macrocode}
\raggedbottom
%    \end{macrocode}
% und europ"aische Handhabung des Leerschritts. Letzteres sollte eigentlich
% von der Sprachanpassung (z.B. |german.sty| oder |german3.sty| erledigt)
% werden und wird in einer der n"achsten Versionen vermutlich entfernt.
%    \begin{macrocode}
\frenchspacing
%    \end{macrocode}
%
% \subsection{Fast das Ende}
%
% \begin{macro}{\KOMAScript}
% Das \KOMAScript-Logo wird in allen \KOMAScript-Paketen und -Klassen
% definiert, falls es nicht bereits definiert ist. Dabei werden die
% Versalien moderat gesperrt. Es wird jedoch darauf verzichtet, die
% Versalien etwa einen Punkt kleiner zu setzen, da das Logo aktiv
% ausgezeichnet erscheinen soll.
%    \begin{macrocode}
\@ifundefined{KOMAScript}{%
  \DeclareRobustCommand{\KOMAScript}{\textsf{K\kern.05em O\kern.05em%
      M\kern.05em A\kern.1em-\kern.1em Script}}}{}
%    \end{macrocode}
% \end{macro}
%    \begin{macrocode}
%</scrlettr>
%    \end{macrocode}
%
% \IndexPrologue{\clearpage
%                \section*{Index}
%                \markboth{Index}{Index}
%                Die kursiven Zahlen geben die Seiten an, auf denen
%                der entsprechende Eintrag beschrieben ist.
%                Die unterstrichenden Zahlen geben die Stelle der
%                Definition des Eintrags an.
%                Alle anderen Zahlen benennen Stellen, an denen der
%                entsprechende Eintrag verwendet ist.
%                \vspace{1em}\noindent}
%
% \GlossaryPrologue{\section*{"Anderungsverzeichnis}
%                   \markboth{"Anderungsverzeichnis}{"Anderungsverzeichnis}
%                   \addcontentsline{toc}{section}{"Anderungsverzeichnis}
%                   Die erste Version des \textsf{KOMA-Script} Pakets
%                   stammt vom 7.\,Juli~1994. Es werden nur die
%                   "Anderungen ab diesem Zeitpunkt dokumentiert.\par%
%                   \vspace{1em}\noindent}
%
% \Finale
%
\endinput
%
% Ende der Datei `scrlettr.dtx'
%
%%% Local Variables:
%%% mode: doctex
%%% TeX-master: t
%%% End:
