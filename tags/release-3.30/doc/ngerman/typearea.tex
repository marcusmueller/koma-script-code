% ======================================================================
% typearea.tex
% Copyright (c) Markus Kohm, 2001-2019
%
% This file is part of the LaTeX2e KOMA-Script bundle.
%
% This work may be distributed and/or modified under the conditions of
% the LaTeX Project Public License, version 1.3c of the license.
% The latest version of this license is in
%   http://www.latex-project.org/lppl.txt
% and version 1.3c or later is part of all distributions of LaTeX 
% version 2005/12/01 or later and of this work.
%
% This work has the LPPL maintenance status "author-maintained".
%
% The Current Maintainer and author of this work is Markus Kohm.
%
% This work consists of all files listed in manifest.txt.
% ----------------------------------------------------------------------
% typearea.tex
% Copyright (c) Markus Kohm, 2001-2019
%
% Dieses Werk darf nach den Bedingungen der LaTeX Project Public Lizenz,
% Version 1.3c, verteilt und/oder veraendert werden.
% Die neuste Version dieser Lizenz ist
%   http://www.latex-project.org/lppl.txt
% und Version 1.3c ist Teil aller Verteilungen von LaTeX
% Version 2005/12/01 oder spaeter und dieses Werks.
%
% Dieses Werk hat den LPPL-Verwaltungs-Status "author-maintained"
% (allein durch den Autor verwaltet).
%
% Der Aktuelle Verwalter und Autor dieses Werkes ist Markus Kohm.
% 
% Dieses Werk besteht aus den in manifest.txt aufgefuehrten Dateien.
% ======================================================================
%
% Chapter about typearea of the KOMA-Script guide
% Maintained by Markus Kohm
%
% ----------------------------------------------------------------------
%
% Kapitel über typearea in der KOMA-Script-Anleitung
% Verwaltet von Markus Kohm
%
% ======================================================================


\KOMAProvidesFile{typearea.tex}
                 [$Date$
                  KOMA-Script guide (chapter: typearea)]
\chapter{Satzspiegelberechnung mit \Package{typearea.sty}}
\labelbase{typearea}

\BeginIndexGroup%
\BeginIndex{Package}{typearea}%
Viele \LaTeX-Klassen\iffree{, darunter auch die Standardklassen,}{} bieten dem
Anwender eine weitgehend feste Auf"|teilung von Rändern und Textbereich. Bei
den Standardklassen ist die konkrete Auf"|teilung in engen Grenzen von der
gewählten Schriftgröße abhängig. Darüber hinaus gibt es Pakete wie
\Package{geometry}\IndexPackage{geometry} (siehe \cite{package:geometry}), die
dem Anwender die volle Kontrolle, aber auch die Verantwortung für die
Einstellungen des Textbereichs und der Ränder überlassen.

\KOMAScript{} geht mit dem Paket \Package{typearea} einen etwas anderen
Weg. Hier werden basierend auf einer in der Typografie etablierten
Konstruktion Einstellmöglichkeiten und Automatismen geboten, die es dem
Anwender erleichtern, eine gute Wahl zu treffen.

\iffalse% Umbruchoptimierung!!!
  Es wird darauf hingewiesen, dass sich \Package{typearea} des Pakets
  \Package{scrbase} bedient. Letzteres Paket ist im Expertenteil
  \iffree{dieser Anleitung}{dieses Buches} in \autoref{cha:scrbase} ab
  \autopageref{cha:scrbase} erklärt. 
  Die Mehrzahl der dort dokumentierten Anweisungen richtet sich jedoch nicht
  an Anwender, sondern an Klassen- und Paketautoren.%
\fi

\section{Grundlagen der Satzspiegelkonstruktion}
\seclabel{basics}

\begin{Explain}
  Betrachtet man eine einzelne Seite eines Buches oder eines anderen
  Druckwerkes, so besteht diese auf den ersten Blick aus den Rändern%
  \iffalse% Umbruchkorrekturtext
  \footnote{Der Autor und der Lektor haben an dieser Stelle überlegt, ob eine
    Seite nicht nur einen umlaufenden Rand hat und daher von »dem Rand« die
    Rede sein müsste. Da jedoch \LaTeX{} diesen einen Rand logisch in mehrere
    Ränder unterteilt, die getrennt bestimmt werden, ist hier auch von »den
    Rändern« die Rede.}%
  \fi%
  , einem Kopfbereich, einem Textkörper und einem Fußbereich.  Genauer
  betrachtet, kommt noch ein Abstand zwischen Kopfbereich und Textkörper sowie
  zwischen Textkörper und Fußbereich hinzu.  Der Textkörper heißt in der
  Fachsprache der Typografen und Setzer \emph{Satzspiegel}. Die Auf"|teilung
  dieser Bereiche sowie ihre Anordnung zueinander und auf dem Papier nennen
  wir \emph{Satzspiegeldefinition} oder
  \emph{Satzspiegelkonstruktion}.\Index[indexmain]{Satzspiegel}
  
  In der Literatur werden verschiedene Algorithmen und heuristische Verfahren
  zur Konstruktion eines guten Satzspiegels vorgeschlagen und diskutiert%
  \iffree{ \cite{DANTE:TK0402:MJK}}{. Einen Überblick über einige Möglichkeiten
    und der dabei angenommenen Grundsätze gibt
    \autoref{cha:typeareaconstruction}}%
  . Häufig findet man ein Verfahren, das mit verschiedenen Diagonalen
  und deren Schnittpunkten arbeitet. Das gewünschte Ergebnis ist, dass das
  Seitenverhältnis des Satzspiegels dem Seitenverhältnis \emph{der Seite}
  entspricht. Bei einem einseitigen\Index{einseitig} Dokument sollen außerdem
  der linke und der rechte Rand gleich breit sein, während der obere zum
  unteren Rand im Verhältnis 1:2 stehen sollte. Bei einem
  doppelseitigen\Index{doppelseitig} Dokument, beispielsweise einem Buch, ist
  hingegen zu beachten, dass der gesamte innere Rand genauso groß sein sollte
  wie jeder der beiden äußeren Ränder. Eine einzelne Seite steuert dabei
  jeweils nur die Hälfte des inneren Randes bei.
  
  Im vorherigen Absatz wurde \emph{die Seite} erwähnt und
  hervorgehoben. Irrtümlich wird oftmals angenommen, das Format der Seite wäre
  mit dem Format des Papiers gleichzusetzen.\Index[indexmain]{Seiten>Format}
  \Index[indexmain]{Papier>Format} Betrachtet man jedoch ein gebundenes
  Druckerzeugnis, so ist zu erkennen, dass ein Teil des Papiers in der
  Bindung\Index[indexmain]{Bindung} verschwindet und nicht mehr als Seite zu
  sehen ist. Für den Satzspiegel ist jedoch nicht entscheidend, welches Format
  das Papier hat, sondern, was der Leser für einen Eindruck vom Format der
  Seite bekommt. Damit ist klar, dass bei der Berechnung des Satzspiegels der
  Teil, der durch die Bindung versteckt wird, aus dem Papierformat
  herausgerechnet und dann zum inneren Rand hinzugefügt werden muss. Wir
  nennen diesen Teil
  \emph{Bindekorrektur}.\Index[indexmain]{Bindekorrektur}%
  \textnote{Bindekorrektur}
  Die Bindekorrektur ist also rechnerischer Bestandteil des
  \emph{Bundstegs}\Index{Bundsteg}, nicht jedoch des sichtbaren
  inneren Randes.

  Die Bindekorrektur ist vom jeweiligen Produktionsvorgang abhängig und kann
  nicht allgemein festgelegt werden. Es handelt sich dabei also um einen
  Parameter, der für jeden Produktionsvorgang neu festzulegen ist. Im
  professionellen Bereich spielt dieser Wert nur eine geringe Rolle, da
  ohnehin auf größere Papierbögen gedruckt und entsprechend geschnitten
  wird. Beim Schneiden wird dann wiederum sichergestellt, dass obige
  Verhältnisse für die sichtbare Doppelseite eingehalten sind.

  Wir wissen nun also, wie die einzelnen Teile zueinander stehen.  Wir wissen
  aber noch nicht, wie breit und hoch der Satzspiegel ist.  Kennen wir eines
  dieser beiden Maße, so ergeben sich zusammen mit dem Papierformat und dem
  Seitenformat oder der Bindekorrektur alle anderen Maße durch Lösung mehrerer
  mathematischer Gleichungen:
  % Umbruchoptiomierung!!!
  \iffree{%
  \begin{align*}
    \Var{Satzspiegelhöhe}\Index{Satzspiegel} : \Var{Satzspiegelbreite} &=
      \Var{Seitenhöhe}\Index{Seite} : \Var{Seitenbreite} \\
    \Var{oberer~Rand}\Index{Rand} : \Var{unterer~Rand} &= 
      \text{1} : \text{2} \\
    \Var{linker~Rand} : \Var{rechter~Rand} &= \text{1} : \text{1} \\
%
    \Var{innerer~Randanteil} : \Var{äußerer~Rand} &= \text{1} : \text{2} \\
%
    \Var{Seitenbreite} &= 
      \Var{Papierbreite}\Index{Papier} - 
      \Var{Bindekorrektur}\Index{Bindekorrektur}\\
%
    \Var{oberer~Rand} + \Var{unterer~Rand} &=
      \Var{Seitenhöhe} - \Var{Satzspiegelhöhe} \\ 
%
    \Var{linker~Rand} + \Var{rechter Rand} &=
      \Var{Seitenbreite} - \Var{Satzspiegelbreite} \\
%
    \Var{innerer Randanteil} + \Var{äußerer Rand} &=
      \Var{Seitenbreite} - \Var{Satzspiegelbreite} \\
%
    \Var{innerer~Randanteil} + \Var{Bindekorrektur} &= 
      \Var{Bundsteg}\index{Bundsteg}
  \end{align*}
  }{%
    % Umbruchkorrektur: Abstände verändert!
    \begin{prepareformargin}
      \vskip -.5\baselineskip plus .25\baselineskip
    \begin{align*}
      \Var{Satzspiegelhöhe}\Index{Satzspiegel} : \Var{Satzspiegelbreite} &=
      \Var{Seitenhöhe}\Index{Seite} : \Var{Seitenbreite} \\[-2pt]
      \Var{oberer~Rand}\Index{Rand} : \Var{unterer~Rand} &= 
        \text{1} : \text{2} \\[-2pt]
      \Var{linker~Rand} : \Var{rechter~Rand} &= \text{1} : \text{1} \\[-2pt]
      \Var{innerer~Randanteil} : \Var{äußerer~Rand} &= 
        \text{1} : \text{2}\\[-2pt]
      \Var{Seitenbreite} &= 
        \Var{Papierbreite}\Index{Papier} - 
        \Var{Bindekorrektur}\Index{Bindekorrektur} \\[-2pt]
      \Var{oberer~Rand} + \Var{unterer~Rand} &=
        \Var{Seitenhöhe} - \Var{Satzspiegelhöhe} \\[-2pt]
      % 
      \Var{linker~Rand} + \Var{rechter Rand} &=
        \Var{Seitenbreite} - \Var{Satzspiegelbreite} \\[-2pt]
      % 
      \Var{innerer Randanteil} + \Var{äußerer Rand} &=
        \Var{Seitenbreite} - \Var{Satzspiegelbreite} \\
      % 
%    \end{align*}
%  \end{prepareformargin}\pagebreak
%  \begin{prepareformargin}\vskip-\abovedisplayskip
%    \begin{align*}
    \Var{innerer~Randanteil} + \Var{Bindekorrektur} &= 
      \Var{Bundsteg}\index{Bundsteg}
  \end{align*}
    \end{prepareformargin}
    \vskip .5\baselineskip minus .25\baselineskip
  }%
  \Index[indexmain]{Rand}%
  Dabei gibt es \Var{linker~Rand} und \Var{rechter~Rand} nur im einseitigen
  Druck. Entsprechend gibt es \Var{innerer~Randanteil} und \Var{äußerer~Rand}
  nur im doppelseitigen Druck. 

  In den Gleichungen wird mit \Var{innerer~Randanteil} gearbeitet, weil der
  komplette innere Rand ein Element der vollständigen Doppelseite ist. Zu
  einer Seite gehört also nur die Hälfte des inneren Randes:
  \Var{innerer~Randanteil}.

  Die Frage nach der Breite des Satzspiegels wird in der Literatur
  ebenfalls diskutiert. Die optimale Satzspiegelbreite ist von
  verschiedenen Faktoren abhängig:
  \iffree{\begin{itemize}}{\begin{itemize}[% Umbuchkorrektur
    topsep=.5\baselineskip plus .25\baselineskip minus .33\baselineskip,
    partopsep=.5\baselineskip plus .25\baselineskip minus .33\baselineskip,
    itemsep=.33\baselineskip plus .25\baselineskip minus .25\baselineskip,
    parsep=0pt
  ]}
  \item Größe, Laufweite und Art der verwendeten Schrift,
  \item verwendeter Durchschuss,
  \item Länge der Worte,
  \item verfügbarer Platz.
  \end{itemize}
  Der Einfluss der Schrift wird deutlich, wenn man sich bewusst macht,
  wozu Serifen dienen. Serifen\Index[indexmain]{Serifen} sind kleine
  Striche an den Linienenden der Buchstaben. Buchstaben, die mit
  vertikalen Linien auf die Grundlinie der Textzeile treffen,
  lösen diese eher auf, als dass sie das Auge auf der Linie halten.
  Genau bei diesen Buchstaben liegen die Serifen horizontal auf
  der Grundlinie und verstärken damit die Zeilenwirkung der Schrift.
  Das Auge kann der Textzeile nicht nur beim Lesen der Worte, sondern
  insbesondere auch beim schnellen Zurückspringen an den Anfang der
  nächsten Zeile besser folgen. Damit darf die Zeile bei einer Schrift
  mit Serifen genau genommen länger sein als bei einer Schrift ohne Serifen.
  
  Unter dem
  Durchschuss\Index[indexmain]{Durchschuss}\textnote{Durchschuss}
  versteht man den Abstand zwischen Textzeilen. Bei \LaTeX{} ist ein
  Durchschuss von etwa 20\,\% der Schriftgröße voreingestellt. Mit Befehlen
  wie \Macro{linespread}\IndexCmd{linespread} oder besser mit Hilfe von
  Paketen wie \Package{setspace}\IndexPackage{setspace} (siehe
  \cite{package:setspace}) kann der Durchschuss verändert werden. Ein großer
  Durchschuss erleichtert dem Auge die Verfolgung einer Zeile.  Bei sehr
  großem Durchschuss wird das Lesen aber dadurch gestört, dass das Auge
  zwischen den Zeilen weite Wege zurücklegen muss.  Daneben wird sich der
  Leser des entstehenden Streifeneffekts sehr deutlich und unangenehm
  bewusst. Der Graueindruck der Seite ist in diesem Fall gestört. Dennoch
  dürfen bei großem Durchschuss die Zeilen länger sein.
  
  Auf der Suche nach konkreten Werten für gute
  Zeilenlängen\Index[indexmain]{Zeilenlaenge=Zeilenlänge} findet man in der
  Literatur je nach Autor unterschiedliche Angaben. Teilweise ist dies auch in
  der Muttersprache des Autors begründet. Das Auge springt nämlich
  üblicherweise von Wort zu Wort, wobei kurze Wörter diese Aufgabe
  erleichtern. Über alle Sprachen und Schriftarten hinweg kann man sagen, dass
  eine Zeilenlänge von 60 bis 70 Zeichen, einschließlich Leer- und
  Satzzeichen, einen brauchbaren Kompromiss darstellt. Ein gut gewählter
  Durchschuss wird dabei vorausgesetzt. Bei den Voreinstellungen von \LaTeX{}
  braucht man sich über Letzteres normalerweise keine Sorgen zu
  machen. Größere Zeilenlängen darf man nur Gewohnheitslesern zumuten, die
  täglich viele Stunden lesend zubringen. Aber auch dann sind Zeilenlängen
  jenseits von 80 Zeichen unzumutbar. In jedem Fall ist dann der Durchschuss
  anzupassen. 5\,\% bis 10\,\% zusätzlich sind dabei als Faustregel
  empfehlenswert. Bei Schriften wie Palatino, die bereits bei einer normalen
  Zeilenlänge nach 5\,\% mehr Durchschuss verlangt, können es auch mehr sein.
  
  Bevor wir uns an die konkrete Konstruktion machen, fehlen jetzt nur noch
  Kleinigkeiten, die man wissen sollte. \LaTeX{} beginnt die erste Zeile des
  Textbereichs einer Seite nicht am oberen Rand des Textbereichs, sondern
  setzt die Grundlinie der Zeile mit einem definierten Mindestabstand zum
  oberen Rand des Textbereichs. Des Weiteren verfügt \LaTeX{} über die beiden
  Befehle \DescRef{maincls.cmd.raggedbottom}\IndexCmd{raggedbottom} und
  \DescRef{maincls.cmd.flushbottom}\IndexCmd{flushbottom}. Der erste dieser
  Befehle legt fest, dass die letzte Zeile einer jeden Seite dort liegen soll,
  wo sie eben zu liegen kommt. Das kann dazu führen, dass sich die Position
  der letzten Zeile von Seite zu Seite vertikal um nahezu eine Zeile verändern
  kann -- bei Zusammentreffen des Seitenendes mit Überschriften, Abbildungen,
  Tabellen oder Ähnlichem auch mehr. Im doppelseitigen Druck ist das in der
  Regel unerwünscht. Mit dem zweiten Befehl,
  \DescRef{maincls.cmd.flushbottom}, wird hingegen festgelegt, dass die letzte
  Zeile immer am unteren Rand des Textbereichs zu liegen kommt. Um diesen
  vertikalen Ausgleich zu erreichen, muss \LaTeX{} gegebenenfalls dehnbare
  vertikale Abstände über das erlaubte Maß hinaus strecken. Ein solcher
  Abstand ist beispielsweise der Absatzabstand. Dies gilt in der Regel auch,
  wenn man gar keinen Absatzabstand verwendet. Um nicht bereits auf normalen
  Seiten, auf denen der Absatzabstand das einzige dehnbare vertikale Maß
  darstellt, eine Dehnung zu erzwingen, sollte die Höhe des Textbereichs ein
  Vielfaches der Textzeilenhöhe zuzüglich des Abstandes der ersten Zeile vom
  oberen Rand des Textbereichs sein.

\iffalse% Umbruchkorrektur
  Damit sind alle Grundlagen der Satzspiegelberechnung, die bei
  {\KOMAScript} eine Rolle spielen, erklärt.
\else
  Soweit die Grundlagen.
\fi
\iffalse% Umbruchkorrektur
  Es folgen die beiden von \KOMAScript{} angebotenen Konstruktionen.
\else 
\iffalse% Umbruchkorrektur
  Wir können also mit den konkreten Konstruktionen beginnen. 
\else
  In den folgenden beiden Abschnitten werden die von {\KOMAScript} angebotenen
  Konstruktionen im Detail vorgestellt.
\fi
\fi
\end{Explain}


\section{Satzspiegelkonstruktion durch Teilung}
\seclabel{divConstruction}

\begin{Explain}
  Der einfachste Weg, um zu erreichen, dass der Textbereich dasselbe
  Seitenverhältnis aufweist wie die Seite, ist folgender:%
  \begin{itemize}
  \item Zunächst zieht man an der Innenseite des Papiers den Teil
    \Var{BCOR}\important{\Var{BCOR}}, der für die
    Bindekorrektur\Index{Bindekorrektur} benötigt wird, ab und teilt die
    restliche Seite vertikal in eine Anzahl \Var{DIV} gleich hoher Streifen.
  \item Dann teilt man die Seite horizontal in die gleiche Anzahl
    \Var{DIV}\important{\Var{DIV}} gleich breiter Streifen.
  \item Nun verwendet man den obersten horizontalen Streifen als oberen und
    die beiden untersten horizontalen Streifen als unteren Rand. Im
    doppelseitigen Druck verwendet man außerdem den innersten vertikalen
    Streifen als inneren und die beiden äußersten vertikalen Streifen als
    äußeren Rand.
  \item Zum inneren Rand gibt man dann noch \Var{BCOR} hinzu.
  \end{itemize}
  Was nun innerhalb der Seite noch übrig bleibt, ist der
  Textbereich.\Index{Text>Bereich} Die Breite bzw. Höhe der Ränder und des
  Textbereichs resultiert damit automatisch aus der Anzahl \Var{DIV} der
  Streifen. Da für die Ränder insgesamt jeweils drei Streifen benötigt werden,
  muss \Var{DIV} zwingend größer als drei sein. Damit der Satzspiegel
  horizontal und vertikal jeweils mindestens doppelt so viel Platz wie die
  Ränder einnimmt, sollte \Var{DIV} sogar mindestens 9 betragen. Mit diesem
  Wert ist die Konstruktion auch als \emph{klassische Neunerteilung} bekannt
  (siehe \autoref{fig:typearea.nineparts}).

  \begin{figure}
%    \centering
    \KOMAoption{captions}{bottombeside}%
    \setcapindent{0pt}%
    \setlength{\columnsep}{.6em}%
    \begin{captionbeside}[{%
        Doppelseite mit der Rasterkonstruktion für die klassische
        Neunerteilung nach Abzug einer Bindekorrektur%
      }]{%
        \label{fig:typearea.nineparts}%
        \hspace{0pt plus 1ex}%
        Doppelseite mit der Rasterkonstruktion für die klassische
        Neunerteilung nach Abzug einer Bindekorrektur%
      }
      [l]
    \setlength{\unitlength}{.25mm}%
    \definecolor{komalight}{gray}{.75}%
    \definecolor{komamed}{gray}{.6}%
    \definecolor{komadark}{gray}{.3}%
    \begin{picture}(420,297)
      % BCOR
      \put(198,0){\color{komalight}\rule{24\unitlength}{297\unitlength}}
      \multiput(198,2)(0,20){15}{\thinlines\line(0,1){10}}
      \multiput(222,2)(0,20){15}{\thinlines\line(0,1){10}}
      % Das Papier
      \put(0,0){\thicklines\framebox(420,297){}}
%      \put(210,0){\thicklines\framebox(210,297){}}
      % Der Satzspiegel
      \put(44,66){\color{komamed}\rule{132\unitlength}{198\unitlength}}
      \put(244,66){\color{komamed}\rule{132\unitlength}{198\unitlength}}
      % Hilfslinien
      \multiput(0,33)(0,33){8}{\thinlines\line(1,0){198}}
      \multiput(222,33)(0,33){8}{\thinlines\line(1,0){198}}
      \multiput(22,0)(22,0){8}{\thinlines\line(0,1){297}}
      \multiput(244,0)(22,0){8}{\thinlines\line(0,1){297}}
      % Beschriftung
      \put(198,0){\color{white}\makebox(24,297)[c]{%
          \rotatebox[origin=c]{-90}{Bindekorrektur}}}
      \put(44,66){\color{white}\makebox(132,198)[c]{Satzspiegel links}}
      \put(244,66){\color{white}\makebox(132,198)[c]{Satzspiegel rechts}}
      % Kästchennummern
      \makeatletter
      \multiput(1,27)(0,33){9}{\footnotesize\makebox(0,0)[l]{\the\@multicnt}}
      \multiput(177,291)(-22,0){9}{%
        \footnotesize\makebox(0,0)[l]{\the\@multicnt}}
      \multiput(419,27)(0,33){9}{%
        \footnotesize\makebox(0,0)[r]{\the\@multicnt}}
      \multiput(243,291)(22,0){8}{%
        \footnotesize\makebox(0,0)[r]{\the\numexpr\@multicnt+1\relax}}
      \makeatother
    \end{picture}
    \end{captionbeside}
%    \caption{Doppelseite mit der Rasterkonstruktion für die klassische
%      Neunerteilung nach Abzug einer Bindekorrektur}
%    \label{fig:typearea.nineparts}
  \end{figure}

  Bei {\KOMAScript} ist diese Art der Konstruktion im Paket \Package{typearea}
  realisiert, wobei der untere Rand weniger als eine Textzeile kleiner
  ausfallen kann, um die im vorherigen Abschnitt erwähnte Nebenbedingung für
  die Satzspiegelhöhe einzuhalten und damit die dort erwähnte Problematik in
  Bezug auf \DescRef{maincls.cmd.flushbottom} zu mindern. Dabei sind für
  A4-Papier je nach Schriftgröße unterschiedliche Werte für \Var{DIV}
  voreingestellt, die \autoref{tab:typearea.div},
  \autopageref{tab:typearea.div} zu entnehmen sind. Bei Verzicht auf
  Bindekorrektur, wenn also \Var{BCOR} = 0\Unit{pt} gilt, ergeben sich in etwa
  die Satzspiegelmaße aus \autoref{tab:typearea.typearea},
  \autopageref{tab:typearea.typearea}.

  Neben den voreingestellten Werten kann man \Var{BCOR} und \Var{DIV} direkt
  beim Laden des Pakets als Option angeben (siehe
  \autoref{sec:typearea.typearea} ab
  \autopageref{sec:typearea.typearea}). Zusätzlich existiert ein Befehl, mit
  dem man einen Satzspiegel explizit berechnen kann und dem man die beiden
  Werte als Parameter übergibt (\iftrue% Umbruchoptimierung!!!!
  siehe ebenfalls
  \autoref{sec:typearea.typearea}, \fi\DescPageRef{typearea.cmd.typearea}).

  Das \Package{typearea}-Paket bietet außerdem die Möglichkeit, den optimalen
  \Var{DIV}-Wert automatisch zu bestimmen. Dieser ist von der Schrift und dem
  Durchschuss abhängig, der zum Zeitpunkt der Satzspiegelberechnung
  eingestellt ist%
  \iffalse% Umbruchoptimierung!!!!
  . Siehe hierzu ebenfalls \autoref{sec:typearea.typearea}, %
  \else\ (siehe ebenfalls \autoref{sec:typearea.typearea}, %
  \fi%
  \DescPageRef{typearea.option.DIV.calc}\iffalse\else)\fi.%
\end{Explain}%


\section{Satzspiegelkonstruktion durch Kreisschlagen}
\seclabel{circleConstruction}

\begin{Explain}
  Neben der zuvor beschriebenen Satzspiegelkonstruktion\Index{Satzspiegel}
  gibt es in der Literatur noch eine eher klassische oder sogar
  mittelalterliche Methode. Bei diesem Verfahren will man die gleichen Werte
  nicht nur in Form des Seitenverhältnisses wiederfinden; man geht außerdem
  davon aus, dass das Optimum dann erreicht wird, wenn die Höhe des
  Textbereichs der Breite der Seite entspricht. Das genaue Verfahren ist
  beispielsweise in \cite{JTsch87} nachzulesen.

  Als Nachteil dieses spätmittelalterlichen Buchseitenkanons ergibt sich, dass
  die Breite des Textbereichs nicht mehr von der Schriftart abhängt. Es wird
  also nicht mehr der zur Schrift passende Textbereich gewählt, stattdessen
  muss der Autor oder Setzer unbedingt die zum Textbereich passende Schrift
  wählen.
%
\iffalse
% Umbruchkorrekturtext
  Dies ist als zwingend zu betrachten.
\fi

  Im \Package{typearea}-Paket wird diese Konstruktion dahingehend abgewandelt,
  dass durch Auswahl eines ausgezeichneten -- normalerweise unsinnigen --
  \Var{DIV}-Wertes oder einer speziellen, symbolischen Angabe derjenige
  \Var{DIV}-Wert ermittelt wird, bei dem der resultierende Satzspiegel dem
  spätmittelalterlichen Buchseitenkanon am nächsten kommt. Es wird also
  wiederum auf die Satzspiegelkonstruktion durch Teilung zurückgegriffen.
%
\iffalse
% Umbruchkorrekturtext
  Bei Wahl einer guten Schrift stimmt dieses Ergebnis nicht selten mit der
  Suche nach dem optimalen \Var{DIV}-Wert überein. Siehe hierzu ebenfalls
  \autoref{sec:typearea.typearea}, \DescPageRef{typearea.option.DIV.calc}.
%
\fi
\end{Explain}

\LoadCommonFile{options} % \section{Frühe oder späte Optionenwahl}

\LoadCommonFile{compatibility} % \section{Kompatibilität zu früheren Versionen von \KOMAScript}

\section{Einstellung des Satzspiegels und der \texorpdfstring{Seitenauf"|teilung}{Seitenaufteilung}}
\seclabel{typearea}

Das Paket \Package{typearea} bietet zwei unterschiedliche
Benutzerschnittstellen, um auf die Satzspiegelkonstruktion Einfluss zu
nehmen. Die wichtigste Möglichkeit ist die Angabe von Optionen. Wie
in \autoref{sec:\LabelBase.options} erwähnt, können die Optionen dabei auf
unterschiedlichen Wegen gesetzt werden.

In\textnote{Hinweis!} diesem Abschnitt wird die Klasse \Class{protokol}
verwendet werden. Es handelt sich dabei nicht um eine \KOMAScript-Klasse,
sondern um eine hypothetische Klasse. Diese\iffree{ Anleitung}{s Buch} geht
von dem Idealfall aus, dass für jede Aufgabe eine dafür passende Klasse zur
Verfügung steht.


\begin{Declaration}
  \OptionVName{BCOR}{Korrektur}
\end{Declaration}%
Mit Hilfe der Option
\OptionVName{BCOR}{Korrektur}\ChangedAt{v3.00}{\Package{typearea}} geben Sie
den absoluten Wert der
Bindekorrektur\Index{Bindekorrektur}\textnote{Bindekorrektur} an, also die
Breite des Bereichs, der durch die Bindung von der Papierbreite verloren
geht. Dieser Wert wird in der Satzspiegelkonstruktion automatisch
berücksichtigt und bei der Ausgabe wieder dem inneren beziehungsweise linken
Rand zugeschlagen. Als \PName{Korrektur} können Sie jede von \TeX{}
verstandene Maßeinheit angeben.

\begin{Example}
  Angenommen, Sie erstellen einen Finanzbericht. Das Ganze soll
  einseitig in A4 gedruckt und anschließend in eine Klemmmappe
  geheftet werden. Die Klemme der Mappe verdeckt 7,5\Unit{mm}.
  Der Papierstapel ist sehr dünn, deshalb gehen beim Knicken und
  Blättern durchschnittlich höchstens weitere 0,75\Unit{mm}
  verloren. Sie schreiben dann also:
\begin{lstcode}
  \documentclass[a4paper]{report}
  \usepackage[BCOR=8.25mm]{typearea}
\end{lstcode}
  mit \OptionValue{BCOR}{8.25mm} als Option für \Package{typearea} oder
\begin{lstcode}
  \documentclass[a4paper,BCOR=8.25mm]{report}
  \usepackage{typearea}
\end{lstcode}
  zur Angabe von \OptionValue{BCOR}{8.25mm} als globale Option.

  Bei Verwendung einer \KOMAScript-Klasse sollte das explizite Laden von
  \Package{typearea} entfallen:
\begin{lstcode}
  \documentclass[BCOR=8.25mm]{scrreprt}
\end{lstcode}
  Die Option \Option{a4paper} konnte bei \Class{scrreprt} entfallen, da diese
  der Voreinstellung bei allen \KOMAScript-Klassen entspricht.

  Setzt man die Option erst später auf einen neuen Wert, verwendet man also
  beispielsweise
\begin{lstcode}
  \documentclass{scrreprt}
  \KOMAoptions{BCOR=8.25mm}
\end{lstcode}
  so werden bereits beim Laden der Klasse \Class{scrreprt}
  Standardeinstellungen vorgenommen. Beim Ändern der Einstellung mit Hilfe
  einer der Anweisung \DescRef{\LabelBase.cmd.KOMAoptions} oder
  \DescRef{\LabelBase.cmd.KOMAoption} wird dann automatisch ein neuer
  Satzspiegel mit neuen Randeinstellungen berechnet.
\end{Example}

Bitte beachten Sie unbedingt\textnote{Achtung!}, dass diese Option bei
Verwendung einer der \KOMAScript-Klassen wie im Beispiel als Klassenoption
oder per \DescRef{\LabelBase.cmd.KOMAoptions} beziehungsweise
\DescRef{\LabelBase.cmd.KOMAoption} nach dem Laden der Klasse übergeben werden
muss. Weder sollte das Paket \Package{typearea} bei Verwendung einer
\KOMAScript-Klasse explizit per \DescRef{\LabelBase.cmd.usepackage} geladen,
noch die Option dabei als optionales Argument angegeben
werden. Wird\textnote{automatische Neuberechnung} die Option per
\DescRef{\LabelBase.cmd.KOMAoptions} oder \DescRef{\LabelBase.cmd.KOMAoption}
nach dem Laden des Pakets geändert, so werden Satzspiegel und Ränder
automatisch neu berechnet.%
%
\EndIndexGroup


\begin{Declaration}
  \OptionVName{DIV}{Faktor}
\end{Declaration}%
Mit Hilfe der Option
\OptionVName{DIV}{Faktor}\ChangedAt{v3.00}{\Package{typearea}} wird
festgelegt, in wie viele Streifen die Seite horizontal und vertikal bei der
Satzspiegelkonstruktion eingeteilt wird. Die genaue Konstruktion ist
\autoref{sec:typearea.divConstruction} zu entnehmen. Wichtig zu wissen ist,
dass gilt: Je größer der \PName{Faktor}, desto größer wird der Textbereich und
desto kleiner die Ränder. Als \PName{Faktor} kann jeder ganzzahlige Wert ab 4
verwendet werden. Bitte beachten Sie jedoch, dass sehr große Werte dazu führen
können, dass Randbedingungen der Satzspiegelkonstruktion, je nach Wahl der
weiteren Optionen, verletzt werden. So kann die Kopfzeile im Extremfall auch
außerhalb der Seite liegen. Bei Verwendung der Option
\OptionVName{DIV}{Faktor} sind Sie für die Einhaltung der Randbedingungen
sowie eine nach typografischen Gesichtspunkten günstige Zeilenlänge selbst
verantwortlich.

In \autoref{tab:typearea.typearea} finden Sie für das Seitenformat A4 ohne
Bindekorrektur die aus einigen \Option{DIV}-Faktoren resultierenden,
theoretischen Satzspiegelgrößen. Dabei werden die weiteren von der
Schriftgröße abhängigen Nebenbedingungen nicht berücksichtigt.

\begin{table}
%  \centering
  \KOMAoptions{captions=topbeside}%
  \setcapindent{0pt}%
  \begin{captionbeside}
%  \caption
  [{Satzspiegelma"se in Abh"angigkeit von \Option{DIV} bei A4}]
  {\label{tab:typearea.typearea}Satzspiegelma"se in Abh"angigkeit von \Option{DIV}
    bei A4 ohne Berücksichtigung von \Length{topskip} oder \Option{BCOR}}
  [l]
  \begin{tabular}[t]{ccccc}
    \toprule
    & 
    \multicolumn{2}{c}{Satzspiegel} & \multicolumn{2}{c}{R"ander}\\
    %\raisebox{1.5ex}[0pt]{
      \Option{DIV}
    %} 
       & Breite & H"ohe  & oben  & innen \\
    \midrule
    6  & 105,00 & 148,50 & 49,50 & 35,00 \\
    7  & 120,00 & 169,71 & 42,43 & 30,00 \\
    8  & 131,25 & 185,63 & 37,13 & 26,25 \\
    9  & 140,00 & 198,00 & 33,00 & 23,33 \\
    10 & 147,00 & 207,90 & 29,70 & 21,00 \\
    11 & 152,73 & 216,00 & 27,00 & 19,09 \\
    12 & 157,50 & 222,75 & 24,75 & 17,50 \\
    13 & 161,54 & 228,46 & 22,85 & 16,15 \\
    14 & 165,00 & 233,36 & 21,21 & 15,00 \\
    15 & 168,00 & 237,60 & 19,80 & 14,00 \\
    \bottomrule
    \multicolumn{5}{r}{\small (alle Längen in mm)}
  \end{tabular}
  \end{captionbeside}
\end{table}

\begin{Example}
  Angenommen, Sie schreiben ein Sitzungsprotokoll. Sie verwenden dafür die
  Klasse \Class{protokol}. Das Ganze soll doppelseitig werden. In Ihrer Firma
  wird die Schriftart Bookman in 12\Unit{pt} verwendet. Diese Schriftart%
  \iffalse % Umbruchkorrekturtext
  , die zu den Standard-PostScript-Schriften gehört,%
  \fi\ wird in {\LaTeX} mit der
  Anweisung \verb|\usepackage{bookman}| aktiviert %
  \iffalse % Umbruchkorrekturtext
  . Die Schriftart Bookman %
  \else
  und %
  \fi läuft sehr weit, das heißt, die einzelnen Zeichen sind im Verhältnis zur
  Höhe relativ breit. Deshalb ist Ihnen die Voreinstellung für den
  \Option{DIV}-Wert in \Package{typearea} zu gering. Statt eines Werts von 12
  sind Sie nach gründlichem Studium dieses Kapitels einschließlich der
  weiterführenden Abschnitte überzeugt, dass der Wert 15 angebracht ist. Das
  Protokoll wird nicht gebunden, sondern gelocht und in einen Ordner
  abgeheftet. Eine Bindekorrektur ist deshalb nicht notwendig. Sie schreiben
  also:
\begin{lstcode}
  \documentclass[a4paper,twoside]{protokol}
  \usepackage{bookman}
  \usepackage[DIV=15]{typearea}
\end{lstcode}
  Als Sie fertig sind, macht man Sie darauf aufmerksam, dass die
  Protokolle neuerdings gesammelt und am Quartalsende alle zusammen
  als Buch gebunden werden. Die Bindung erfolgt als
  einfache Leimbindung%
  \iffalse% Umbruchkorrektur
  , weil den Band ohnehin nie wieder jemand
  anschaut und er nur wegen ISO\,9000 angefertigt wird. Für die Bindung
  einschließlich Biegefalz werden %
  \else%
  . Einschließlich Biegefalz werden dafür %
  \fi
  durchschnittlich 12\Unit{mm} benötigt. Sie ändern die Optionen von
  \Package{typearea} also entsprechend ab und verwenden die Klasse für
  Protokolle nach ISO\,9000:
\begin{lstcode}
  \documentclass[a4paper,twoside]{iso9000p}
  \usepackage{bookman}
  \usepackage[DIV=15,BCOR=12mm]{typearea}
\end{lstcode}
  Natürlich können Sie auch hier wieder eine \KOMAScript-Klasse
  verwenden:
\begin{lstcode}
  \documentclass[twoside,DIV=15,BCOR=12mm]{scrartcl}
  \usepackage{bookman}
\end{lstcode}
\iftrue% Umbruchoptimierung!!!!
\iftrue
  Die voreingestellte Option \Option{a4paper} konnte dabei entfallen.
\else
  Option \Option{a4paper} entspricht der Voreinstellung und konnte daher
  entfallen.
\fi
\else
  Die Option \Option{a4paper} konnte bei der Klasse \Class{scrartcl} entfallen,
  da diese der Voreinstellung bei allen \KOMAScript-Klassen entspricht.
\fi
\end{Example}

Bitte beachten Sie unbedingt\textnote{Achtung!}, dass die Option \Option{DIV}
bei Verwendung einer der \KOMAScript-Klassen wie im Beispiel als Klassenoption
oder per \DescRef{\LabelBase.cmd.KOMAoptions} beziehungsweise
\DescRef{\LabelBase.cmd.KOMAoption} nach dem Laden der Klasse übergeben werden
muss. Weder sollte das Paket \Package{typearea} bei Verwendung einer
\KOMAScript-Klasse explizit per \DescRef{\LabelBase.cmd.usepackage} geladen,
noch die Option dabei als optionales Argument angegeben
werden. Wird\textnote{automatische Neuberechnung} die Option per
\DescRef{\LabelBase.cmd.KOMAoptions} oder \DescRef{\LabelBase.cmd.KOMAoption}
nach dem Laden des Pakets geändert, so werden Satzspiegel und Ränder
automatisch neu berechnet.

\begin{Declaration}
  \OptionValue{DIV}{calc}
  \OptionValue{DIV}{classic}
\end{Declaration}
Wie\ChangedAt{v3.00}{\Package{typearea}} bereits in
\autoref{sec:typearea.divConstruction} erwähnt, gibt es nur für das
Papierformat A4 feste Voreinstellungen für den \Option{DIV}-Wert. Diese sind
\autoref{tab:typearea.div} zu entnehmen. Solche festen Werte haben allerdings
den Nachteil\textnote{Achtung!}, dass sie die Laufweite der verwendeten
Schrift nicht berücksichtigen. Das kann bei A4 und recht schmalen Schriften
sehr rasch zu unangenehm hoher Zeichenzahl je Zeile führen. Siehe hierzu die
Überlegungen in \autoref{sec:typearea.basics}. Wird ein anderes Papierformat
gewählt, so berechnet \Package{typearea} selbst einen guten \Option{DIV}-Wert.
Natürlich können Sie diese Berechnung auch für A4 wählen. Hierzu verwenden Sie
\OptionValue{DIV}{calc}\important{\OptionValue{DIV}{calc}} anstelle von
\OptionVNameRef{\LabelBase}{DIV}{Faktor}.  Selbstverständlich können Sie diese
Option auch explizit bei allen anderen Papierformaten angeben. Wenn Sie die
automatische Berechnung wünschen, ist diese Angabe sogar sinnvoll, da die
Möglichkeit besteht, in einer Konfigurationsdatei andere Voreinstellungen zu
setzen (siehe \autoref{sec:typearea-experts.cfg}). Eine explizit angegebene
Option \OptionValue{DIV}{calc} überschreibt diese Vorkonfiguration aber.

\begin{table}
%  \centering
  \KOMAoptions{captions=topbeside}%
  \setcapindent{0pt}%
  \begin{captionbeside}
  %\caption
    [{\Option{DIV}-Voreinstellungen f"ur A4}]
    {\label{tab:typearea.div}\Option{DIV}-Voreinstellungen f"ur A4}
    [l]
  \begin{tabular}[t]{lccc}
    \toprule
    Grundschriftgr"o"se: & 10\Unit{pt} & 11\Unit{pt} & 12\Unit{pt} \\
    \Option{DIV}:           &   8  &  10  &  12  \\
    \bottomrule
  \end{tabular}
  \end{captionbeside}
\end{table}

Die in \autoref{sec:typearea.circleConstruction} erwähnte klassische
Konstruktion, der mittelalterliche Buchseitenkanon, ist ebenfalls auswählbar.
Verwenden Sie in diesem Fall anstelle von
\OptionVNameRef{\LabelBase}{DIV}{Faktor} oder \OptionValue{DIV}{calc} einfach
\OptionValue{DIV}{classic}\important{\OptionValue{DIV}{classic}}. Es wird dann
ein \Option{DIV}-Wert ermittelt, der eine möglichst gute Näherung an den
mittelalterlichen Buchseitenkanon darstellt.

\begin{Example}
  In dem bei der Option \OptionVNameRef{\LabelBase}{DIV}{Faktor} aufgeführten
  Beispiel mit der Schriftart Bookman gab es ja genau das Problem, dass man
  einen zur Schriftart besser passenden \Option{DIV}-Wert haben wollte. Man
  könnte also in Abwandlung des ersten Beispiels auch einfach die Ermittlung
  dieses Wertes \Package{typearea} überlassen:
\begin{lstcode}
  \documentclass[a4paper,twoside]{protokol}
  \usepackage{bookman}
  \usepackage[DIV=calc]{typearea}
\end{lstcode}
\end{Example}
\iffree{\par%
  Bitte beachten Sie unbedingt\textnote{Achtung!}, dass diese Option bei
  Verwendung einer der \KOMAScript-Klassen %
  \iftrue wie im Beispiel \fi%
  als Klassenoption oder per \DescRef{\LabelBase.cmd.KOMAoptions}
  beziehungsweise \DescRef{\LabelBase.cmd.KOMAoption} nach dem Laden der
  Klasse übergeben werden muss. Weder sollte das Paket \Package{typearea} bei
  Verwendung einer \KOMAScript-Klasse explizit per
  \DescRef{\LabelBase.cmd.usepackage} geladen, noch die Option dabei als
  optionales Argument angegeben werden. Wird die Option per
  \DescRef{\LabelBase.cmd.KOMAoptions} oder
  \DescRef{\LabelBase.cmd.KOMAoption} nach dem Laden des Pakets geändert, so
  werden Satzspiegel und Ränder automatisch neu berechnet.%
}{%
  \vskip-1\ht\strutbox plus .75\ht\strutbox% Wegen Code im Beispiel am Ende
}%
%
\EndIndexGroup

\begin{Declaration}
  \OptionValue{DIV}{current}
  \OptionValue{DIV}{last}
\end{Declaration}
Wenn\ChangedAt{v3.00}{\Package{typearea}} Sie bis hier die Beispiele aufmerksam
verfolgt haben, wissen Sie eigentlich bereits, wie man die Berechnung eines
\Option{DIV}-Wertes in Abhängigkeit von der gewählten Schrift erreicht, wenn eine
\KOMAScript-Klasse zusammen mit einem Schriftpaket verwendet wird.

\begin{Explain}
  Das Problem dabei ist, dass die \KOMAScript-Klasse das Paket
  \Package{typearea} bereits selbst lädt. Die Übergabe der Optionen als
  optionale Argumente von \DescRef{\LabelBase.cmd.usepackage} ist also nicht
  möglich. Es würde auch nichts nützen, die Option \OptionValue{DIV}{calc} als
  optionales Argument von \DescRef{\LabelBase.cmd.documentclass}
  anzugeben. Diese Option würde direkt beim Laden des Pakets
  \Package{typearea} ausgewertet. Damit würden Satzspiegel und Ränder für die
  \LaTeX-Standardschrift und nicht für die später geladene Schrift berechnet.

  Selbstverständlich ist es möglich, mit \DescRef{\LabelBase.cmd.KOMAoptions}%
  \PParameter{\OptionValueRef{\LabelBase}{DIV}{calc}} oder
  \DescRef{\LabelBase.cmd.KOMAoption}%
  \PParameter{\DescRef{\LabelBase.option.DIV}}\PParameter{calc} nach dem Laden
  des Schriftpakets Satzspiegel und Ränder neu berechnen zu lassen. Dabei wird
  dann über den Wert \PValue{calc} direkt ein \Option{DIV}-Wert für eine gute
  Zeilenlänge eingefordert.

  Da es aber häufig praktischer ist, die Einstellung für die Option
  \Option{DIV} nicht erst nach dem Laden der Schrift vorzunehmen, sondern an
  herausgehobener Stelle, beispielsweise beim Laden der Klasse, bietet
  \Package{typearea} zwei weitere symbolische Werte für diese Option.
\end{Explain}

Mit \OptionValue{DIV}{current}\ChangedAt{v3.00}{\Package{typearea}}%
\important{\OptionValue{DIV}{current}} wird eine Neuberechnung von Satzspiegel
und Rändern angestoßen, wobei genau der \Option{DIV}-Wert verwendet wird, der
aktuell eingestellt ist. Dies ist weniger für die Neuberechnung des
Satzspiegels nach Wahl einer anderen Grundschrift von Interesse. Vielmehr ist
das dann nützlich, wenn man etwa nach Änderung des Durchschusses unter
Beibehaltung des Teilers \Var{DIV} die Randbedingung sicherstellen will, dass
\Length{textheight} abzüglich \Length{topskip} ein Vielfaches von
\Length{baselineskip} sein sollte.

Mit \OptionValue{DIV}{last}\ChangedAt{v3.00}{\Package{typearea}}%
\important{\OptionValue{DIV}{last}} wird eine Neuberechnung von Satzspiegel
und Rändern angestoßen, wobei genau dieselbe Einstellung wie bei der letzten
Berechnung verwendet wird.

\begin{Example}
  Gehen wir wieder davon aus, dass für die Schriftart Bookman ein Satzspiegel
  mit guter Zeilenlänge berechnet werden soll.  Gleichzeitig wird eine
  \KOMAScript-Klasse verwendet. Dies ist mit dem symbolischen Wert
  \PValue{last} und der Anweisung \DescRef{\LabelBase.cmd.KOMAoptions} sehr
  einfach möglich:
\begin{lstcode}
  \documentclass[BCOR=12mm,DIV=calc,twoside]
                {scrartcl}
  \usepackage{bookman}
  \KOMAoptions{DIV=last}
\end{lstcode}
  Wird später entschieden, dass ein anderer \Option{DIV}-Wert
  verwendet werden soll, so muss nur die Einstellung im optionalen Argument
  von \DescRef{\LabelBase.cmd.documentclass} geändert werden.
\end{Example}

Eine Zusammenfassung aller möglichen symbolischen Werte für die Option
\Option{DIV} finden Sie in \autoref{tab:symbolicDIV}. Es wird an dieser Stelle
darauf hingewiesen, dass auch die Verwendung des Pakets
\Package{fontenc}\IndexPackage{fontenc} dazu führen kann, dass \LaTeX{} eine
andere Schrift lädt.

\begin{table}
  \caption[{%
    Symbolische Werte für Option \DescRef{typearea.option.DIV} und das
    \PName{DIV}-Argument von \DescRef{typearea.cmd.typearea}%
  }]{%
    Mögliche symbolische Werte für die Option \DescRef{typearea.option.DIV}
    oder das \PName{DIV}-Argument der Anweisung
    \DescRef{typearea.cmd.typearea}\OParameter{BCOR}\Parameter{DIV}%
  }
  \label{tab:symbolicDIV}
  \begin{desctabular}
    \pventry{areaset}{Satzspiegel neu
      anordnen.\IndexOption{DIV~=\textKValue{areaset}}}%
    \pventry{calc}{Satzspiegelberechnung einschließlich Ermittlung eines guten
      \Option{DIV}-Wertes erneut
      durchführen.\IndexOption{DIV~=\textKValue{calc}}}%
    \pventry{classic}{Satzspiegelberechnung nach dem mittelalterlichen
      Buchseitenkanon (Kreisberechnung) erneut
      durchführen.\IndexOption{DIV~=\textKValue{classic}}}%
    \pventry{current}{Satzspiegelberechnung mit dem aktuell gültigen
      \Option{DIV}-Wert erneut
      durchführen.\IndexOption{DIV~=\textKValue{current}}}%
    \pventry{default}{Satzspiegelberechnung mit dem Standardwert für das
      aktuelle Seitenformat und die aktuelle Schriftgröße erneut durchführen.
      Falls kein Standardwert existiert, \PValue{calc}
      anwenden.\IndexOption{DIV~=\textKValue{default}}}%
    \pventry{last}{Satzspiegelberechnung mit demselben \PName{DIV}-Argument,
      das beim letzten Aufruf angegeben wurde, erneut
      durchführen.\IndexOption{DIV~=\textKValue{last}}}%
  \end{desctabular}
\end{table}

Häufig\textnote{Achtung!} wird die Satzspiegelneuberechnung im Zusammenhang
mit der Ver"-änderung des Zeilenabstandes
(\emph{Durchschuss})\Index{Durchschuss} benötigt. Da der Satzspiegel unbedingt
so berechnet werden sollte, dass eine ganze Anzahl an Zeilen in den
Textbereich passt, muss bei Verwendung eines anderen Durchschusses als des
normalen der Satzspiegel für diesen Zeilenabstand neu berechnet werden.

\begin{Example}
  Angenommen, für eine Diplomarbeit wird die Schriftgröße
  10\Unit{pt} bei eineinhalbzeiligem Satz \iffalse% Umbruchkorrektur
  zwingend \fi gefordert. {\LaTeX}
  setzt normalerweise bei 10\Unit{pt} mit 2\Unit{pt} Durchschuss,
  also 1,2-zeilig. Deshalb muss als zusätzlicher Dehnfaktor der Wert
  1,25 verwendet werden. Gehen wir außerdem davon aus, dass eine
  Bindekorrektur von 12\Unit{mm} benötigt wird. Dann könnte die Lösung
  dieses Problems wie folgt aussehen:
\begin{lstcode}
  \documentclass[10pt,twoside,BCOR=12mm,DIV=calc]
                {scrreprt}
  \linespread{1.25}
  \KOMAoptions{DIV=last}
\end{lstcode}\IndexCmd{linespread}
  Da \Package{typearea} selbst immer die Anweisung \Macro{normalsize} bei
  Berechnung eines neuen Satzspiegels ausführt, ist es nicht zwingend
  notwendig, nach \Macro{linespread} den gewählten Durchschuss mit
  \Macro{selectfont} zu aktivieren, damit dieser %
\iftrue % Umbruchkorrektur
  auch tatsächlich 
\fi %
  für die Neuberechnung verwendet wird.

  Das gleiche Beispiel sähe unter Verwendung des
  \Package{setspace}-Pakets\IndexPackage{setspace}%
  \important{\Package{setspace}}
  (siehe \cite{package:setspace}) wie folgt aus:
\begin{lstcode}
  \documentclass[10pt,twoside,BCOR=12mm,DIV=calc]
                {scrreprt}
  \usepackage[onehalfspacing]{setspace}
  \KOMAoptions{DIV=last}
\end{lstcode}
\end{Example}

Wie\textnote{Tipp!} man an dem Beispiel sieht, spart man sich mit dem
\Package{setspace}-Paket das Wissen um den korrekten Dehnungswert. Dies gilt
allerdings nur für die Standardschriftgrößen 10\Unit{pt}, 11\Unit{pt} und
12\Unit{pt}. Für alle anderen Schriftgrößen verwendet das Paket 
%
\iffalse % Umbruchkorrektur
einen näherungsweise passenden Dehnungswert.
%
\else %
eine Näherung.
\fi

An\textnote{Achtung!} dieser Stelle erscheint es mir angebracht, darauf
hinzuweisen, dass der Zeilenabstand für die Titelseite wieder auf den normalen
Wert zurückgesetzt werden sollte und außerdem auch die Verzeichnisse mit dem
normalen Zeilenabstand gesetzt werden sollten.
\begin{Example}
  Ein\IndexPackage{setspace} vollständiges Beispiel wäre also:
\begin{lstcode}
  \documentclass[10pt,twoside,BCOR=12mm,DIV=calc]
                {scrreprt}
  \usepackage[onehalfspacing]{setspace}
  \AfterTOCHead{\singlespacing}
  \KOMAoptions{DIV=last}
  \begin{document}
  \title{Titel}
  \author{Markus Kohm}
  \begin{spacing}{1}
    \maketitle
  \end{spacing}
  \tableofcontents
  \chapter{Ok}
  \end{document}
\end{lstcode}
  Siehe hierzu auch die Anmerkungen in \autoref{sec:typearea.tips}. Die
  Anweisung \DescRef{tocbasic.cmd.AfterTOCHead}\IndexCmd{AfterTOCHead} wird in
  \autoref{part:forExperts}, \autoref{cha:tocbasic} auf
  \DescPageRef{tocbasic.cmd.AfterTOCHead} vorgestellt.
\end{Example}
Außerdem sei darauf hingewiesen, dass Änderungen am Zeilenabstand auch
Auswirkungen auf Kopf und Fuß der Seite haben können. Dies kann sich
beispielsweise bei Verwendung von \Package{scrlayer-scrpage} auswirken und man
muss dann selbst entscheiden, ob man dort lieber den normalen Durchschuss oder
den veränderten haben will. Siehe dazu auch Option
\DescRef{scrlayer.option.singlespacing} in \autoref{cha:scrlayer}%
\important{\hyperref[cha:scrlayer]{\Package{scrlayer}}}%
\IndexPackage{scrlayer}\IndexOption{singlespacing} auf
\DescPageRef{scrlayer.option.singlespacing}.

Bitte beachten Sie\iffalse % Umbruchkorrektur
\ unbedingt\fi\textnote{Achtung!}, dass diese Optionen
\iffalse % Umbruchkorrektur
auch zur \else bei \fi Verwendung mit \DescRef{\LabelBase.cmd.KOMAoptions}
oder \DescRef{\LabelBase.cmd.KOMAoption} \iffalse % Umbruchkorrektur
nach dem Laden des Pakets vorgesehen sind und dann \fi eine
automatische\textnote{automatische Neuberechnung} Neuberechnung von
Satzspiegel und Rändern auslösen.%
%
\EndIndexGroup
\EndIndexGroup


\begin{Declaration}
  \Macro{typearea}\OParameter{BCOR}\Parameter{DIV}
  \Macro{recalctypearea}
\end{Declaration}%
\begin{Explain}
  Wird die Option \DescRef{\LabelBase.option.DIV} oder die Option
  \DescRef{\LabelBase.option.BCOR} nach dem Laden des Pakets
  \Package{typearea} gesetzt, so wird intern die Anweisung \Macro{typearea}
  aufgerufen. Dabei wird beim Setzen der Option
  \DescRef{\LabelBase.option.DIV} für \PName{BCOR} intern der symbolische Wert
  \PValue{current} verwendet\iffalse % Umbruchkorrektur
  , der aus Gründen der Vollständigkeit auch in \autoref{tab:symbolicBCOR} zu
  finden ist\else\ (siehe \autoref{tab:symbolicBCOR})\fi. Beim Setzen der
  Option \DescRef{\LabelBase.option.BCOR} wird für \PName{DIV} hingegen der
  symbolische Wert \PValue{last} verwendet. Wollen Sie, dass Satzspiegel und
  Ränder stattdessen mit dem symbolischen Wert \PValue{current} für
  \PName{DIV} neu berechnet werden, so können Sie direkt
  \Macro{typearea}\OParameter{BCOR}\PParameter{current} verwenden.
\end{Explain}

\begin{table}
  \caption[{%
    Symbolische \PName{BCOR}-Argumente für \DescRef{typearea.cmd.typearea}%
  }]{%
    Mögliche symbolische \PName{BCOR}-Argumente für
    \Macro{typearea}\OParameter{BCOR}\Parameter{DIV}%
  }
  \label{tab:symbolicBCOR}
  \begin{desctabular}
    \pventry{current}{Satzspiegelberechnung mit dem aktuell gültigen
      \Option{BCOR}-Wert erneut
      durchführen.\IndexOption{BCOR~=\textKValue{current}}}
  \end{desctabular}
\end{table}

Sollen die Werte sowohl von \PName{BCOR} als auch \PName{DIV} geändert werden,
so ist die Verwendung von \Macro{typearea} zu empfehlen, da hierbei die Ränder
und der Satzspiegel nur einmal neu berechnet werden. Bei
\DescRef{\LabelBase.cmd.KOMAoptions}%
\PParameter{\OptionVNameRef{\LabelBase}{DIV}{Faktor},%
  \OptionVNameRef{\LabelBase}{BCOR}{Korrektur}} werden hingegen Ränder und
Satzspiegel zunächst in Folge von Option \DescRef{\LabelBase.option.DIV} und
dann zusätzlich durch Option \DescRef{\LabelBase.option.BCOR} neu berechnet.

\begin{Explain}
  Der Befehl \Macro{typearea} ist derzeit so definiert, dass es auch möglich
  ist, mitten in einem Dokument den Satzspiegel zu wechseln. Dabei werden
  allerdings Annahmen über den Aufbau des \LaTeX-Kerns gemacht und interne
  Definitionen und Größen des \LaTeX-Kerns verändert. %
  \iffalse% Stimmt so nicht mehr wirklich.
  Da am \LaTeX-Kern nur noch zur Beseitigung von Fehlern notwendige %
  \else %
  Auch wenn am \LaTeX-Kern inzwischen wieder mehr %
  \fi %
  Änderungen vorgenommen werden,
  ist die Wahrscheinlichkeit hoch, dass dies in zukünftigen Versionen von
  \LaTeXe{} noch funktionieren wird. Eine Garantie dafür gibt es jedoch
  nicht. Die Verwendung innerhalb des Dokuments führt außerdem immer zu einem
  Seitenumbruch.
\end{Explain}

Da\important{\Macro{recalctypearea}} \DescRef{\LabelBase.cmd.KOMAoption}%
\PParameter{\hyperref[desc:\LabelBase.option.DIV.last]{\Option{DIV}}}%
\PParameter{\hyperref[desc:\LabelBase.option.DIV.last]{last}} oder
\DescRef{\LabelBase.cmd.KOMAoptions}%
\PParameter{\OptionValueRef{\LabelBase}{DIV}{last}} beziehungsweise
\Macro{typearea}\POParameter{current}\PParameter{last} für die Neuberechnung
des Satzspiegels und der Ränder recht häufig benötigt werden, gibt es dafür
die abkürzende Anweisung
\Macro{recalctypearea}\ChangedAt{v3.00}{\Package{typearea}}.
\iffalse % Umbruchkorrektur
\begin{Example}
  Wenn Ihnen die Schreibweisen
\begin{lstcode}
  \KOMAoptions{DIV=last}
\end{lstcode}
  oder
\begin{lstcode}
  \typearea[current]{last}
\end{lstcode}
  für die Neuberechnung von Satzspiegel und Rändern aufgrund der vielen
  Sonderzeichen zu umständlich ist, können Sie einfach
\begin{lstcode}
  \recalctypearea
\end{lstcode}
  verwenden.
\end{Example}%
\vskip -1\ht\strutbox plus .75\ht\strutbox% Ende Beispiel+Ende Erklärung
\fi
\EndIndexGroup


\begin{Declaration}
  \OptionVName{twoside}{Ein-Aus-Wert}
  \OptionValue{twoside}{semi}
\end{Declaration}%
Wie in \autoref{sec:typearea.basics} erklärt, hängt
die Randverteilung davon ab, ob ein Dokument ein- oder zweiseitig gesetzt
werden soll. Bei einseitigem Satz sind der linke und rechte Rand gleich
breit, während bei doppelseitigem Satz der innere Randanteil einer Seite nur
halb so groß ist wie der jeweilige äußere Rand. Um diese Unterscheidung
vornehmen zu können, muss \Package{typearea} mit Option \Option{twoside}
mitgeteilt werden, ob das Dokument doppelseitig gesetzt wird. Als
\iffree{\PName{Ein-Aus-Wert}}{\PName{Wert}} kann dabei einer der Standardwerte
für einfache Schalter aus \autoref{tab:truefalseswitch} verwendet werden. Wird
die Option ohne Wert-Angabe verwendet, so wird der Wert
\PValue{true}\important{\OptionValue{twoside}{true}} angenommen, also
doppelseitiger Satz verwendet. Deaktivieren der
Option\important{\OptionValue{twoside}{false}} führt zu einseitigem Satz.

\begin{table}
%  \centering
%  \caption
  \KOMAoptions{captions=topbeside}\setcapindent{0pt}
  \begin{captionbeside}%
    [{Standardwerte für einfache Schalter in \KOMAScript}]%
    {\label{tab:truefalseswitch}Standardwerte für alle einfachen Schalter in
      \KOMAScript}%
    [l]
  \begin{tabular}[t]{ll}
    \toprule
    Wert & Bedeutung \\
    \midrule
    \PValue{true} & aktiviert die Option\\
    \PValue{on}   & aktiviert die Option \\
    \PValue{yes}  & aktiviert die Option \\
    \PValue{false}& deaktiviert die Option \\
    \PValue{off}  & deaktiviert die Option \\
    \PValue{no}   & deaktiviert die Option \\
    \bottomrule
  \end{tabular}
  \end{captionbeside}
\end{table}

Außer den Werten aus \autoref{tab:truefalseswitch} kann auch noch der
Wert \PValue{semi}\ChangedAt{v3.00}{\Package{typearea}}%
\important{\OptionValue{twoside}{semi}} angegeben werden. Dieser Wert
\PValue{semi} führt zu doppelseitigem Satz mit einseitigen Rändern und
einseitigen, also nicht alternierenden
Marginalien. Eine\ChangedAt{v3.12}{\Package{typearea}} eventuelle
Bindekorrektur (siehe Option \DescRef{\LabelBase.option.BCOR},
\DescPageRef{typearea.option.BCOR}) wird jedoch ab \KOMAScript~Version 3.12
wie beim doppelseitigen Satz auf Seiten mit ungerader Nummer dem linken Rand
und auf Seiten mit gerader Nummer dem rechten Rand zugeschlagen. Wird auf
Kompatibilität zu einer früheren
Version\important{\OptionValueRef{\LabelBase}{version}{3.12}} zurückgeschaltet
(siehe \autoref{sec:typearea.compatibilityOptions},
\autopageref{sec:typearea.compatibilityOptions}), so ist die Bindekorrektur
dagegen auch bei \OptionValue{twoside}{semi} immer Teil des linken Randes.

Die Option kann %
\iffalse % Umbruchkorrektur
wahlweise \fi %
als Klassenoption bei \DescRef{\LabelBase.cmd.documentclass}, als Paketoption
bei \DescRef{\LabelBase.cmd.usepackage} oder %
\iffalse % Umbruchkorrektur
auch \fi %
nach dem Laden von
\Package{typearea} per \DescRef{\LabelBase.cmd.KOMAoptions} oder
\DescRef{\LabelBase.cmd.KOMAoption} gesetzt werden. Eine Verwendung dieser
Option nach dem Laden von \Package{typearea} führt\textnote{automatische
  Neuberechnung} automatisch zur Neuberechnung des Satzspiegels mit
\DescRef{\LabelBase.cmd.recalctypearea} (siehe
\DescPageRef{typearea.cmd.recalctypearea}). War vor der Option
doppelseitiger Satz aktiv, wird noch vor der Neuberechnung auf die nächste
ungerade Seite umbrochen.%
%
\EndIndexGroup


\begin{Declaration}
  \OptionVName{twocolumn}{Ein-Aus-Wert}
\end{Declaration}%
Für die Berechnung eines guten Satzspiegels mit Hilfe von
\OptionValueRef{\LabelBase}{DIV}{calc} ist es erforderlich zu wissen, ob das
Dokument ein- oder zweispaltig gesetzt wird. Da die Betrachtungen zur
Zeilenlänge aus \autoref{sec:typearea.basics} dann für jede einzelne Spalte
gelten, darf der Satzspiegel in doppelspaltigen Dokumenten bis zu doppelt so
breit sein wie in einspaltigen Dokumenten.

Um diese Unterscheidung vornehmen zu können, muss \Package{typearea} mit
Option \Option{twocolumn} mitgeteilt werden, ob das Dokument doppelspaltig
gesetzt wird. Als \PName{Ein-Aus-Wert} kann dabei einer der Standardwerte für
einfache Schalter aus \autoref{tab:truefalseswitch} verwendet werden. Wird die
Option ohne Wert-Angabe verwendet, so wird der Wert
\PValue{true}\important{\OptionValue{twocolumn}{true}} angenommen, also
doppelspaltiger Satz verwendet. Ein
Deaktivieren der Option führt wieder zum voreingestellten
einspaltigen Satz.

Die Option kann als Klassenoption bei \DescRef{\LabelBase.cmd.documentclass},
als Paketoption bei \DescRef{\LabelBase.cmd.usepackage} oder nach dem Laden
von \Package{typearea} per \DescRef{\LabelBase.cmd.KOMAoptions} oder
\DescRef{\LabelBase.cmd.KOMAoption} gesetzt werden. Eine Verwendung %
\iffalse % Umbruchoptimierung
dieser Option \fi%
nach dem Laden von \Package{typearea}
führt\textnote{automatische Neuberechnung} automatisch zur Neuberechnung des
Satzspiegels mittels \DescRef{\LabelBase.cmd.recalctypearea} (siehe
\DescPageRef{typearea.cmd.recalctypearea}).%
%
\EndIndexGroup


\begin{Declaration}
  \OptionVName{headinclude}{Ein-Aus-Wert}
  \OptionVName{footinclude}{Ein-Aus-Wert}
\end{Declaration}%
\begin{Explain}%
  Bisher wurde zwar erklärt, wie die
  Satzspiegelkonstruktion\Index{Satzspiegel} funktioniert und in welchem
  Verhältnis einerseits die Ränder\Index{Rand} zueinander stehen, andererseits
  der Textkörper zur Seite steht, aber eine entscheidende Frage blieb
  ausgeklammert: Was genau ist \emph{der Rand}?

  Auf den ersten Blick wirkt diese Frage trivial: Der Rand ist der Teil der
  Seite, der oben, unten, links und rechts frei bleibt. Doch das ist nur die
  halbe Wahrheit. Beim oberen und unteren Rand stellt sich die Frage, wie
  Kopf- und Fußzeile\Index{Seiten>Kopf}\Index{Seiten>Fuss=Fuß} zu behandeln
  sind.  Gehören diese beiden zum Textkörper oder zum jeweiligen Rand? Die
  Frage ist nicht einfach zu beantworten.  Eindeutig ist, dass ein leerer Fuß
  und ein leerer Kopf zum Rand zu rechnen sind. Schließlich können sie nicht
  vom restlichen Rand unterschieden werden. Ein Fuß, der nur die
  Paginierung\Index{Paginierung}\textnote{Paginierung}
  \iffalse % Beseitigung von Fußnoten
  \unskip\footnote{Unter der Paginierung versteht man die Angabe der
    Seitenzahl, wahlweise innerhalb oder außerhalb des Satzspiegels, meist im
    Kopf oder Fuß der Seite.} %
  \fi enthält, wirkt optisch ebenfalls eher wie Rand und sollte deshalb zu
  diesem gerechnet werden. Für die optische Wirkung ist dabei unwesentlich, ob
  der Fuß beim Lesen oder Überfliegen leicht als Fuß erkannt werden kann oder
  nicht. Entscheidend ist, wie eine wohlgefüllte Seite bei \emph{unscharfer
    Betrachtung} wirkt. Dazu bedient man sich beispielsweise seiner
  altersweitsichtigen Großeltern, denen man die Brille stibitzt und dann die
  Seite etwa einen halben Meter von der Nasenspitze entfernt hält. In
  Ermangelung erreichbarer Großeltern kann man sich auch damit behelfen, dass
  man die eigenen Augen auf Fernsicht stellt, die Seite aber nur mit
  ausgestreckten Armen hält. Brillenträger sind hier deutlich im Vorteil. Hat
  man eine Fußzeile, die neben der Paginierung weitere weitschweifige Angaben
  enthält, beispielsweise einen Copyright-Hinweis, so wirkt die Fußzeile eher
  wie ein etwas abgesetzter Teil des Textkörpers. Bei der Berechnung des
  Satzspiegels sollte das berücksichtigt werden.
  
  Bei der Kopfzeile sieht es noch schwieriger aus. In der Kopfzeile wird
  häufig der Kolumnentitel\Index{Kolumnentitel}\textnote{Kolumnentitel}
  \iffalse% Beseitigung von Fußnoten
  \unskip\footnote{Unter dem Kolumnentitel versteht man in der Regel die
    Wiederholung einer Überschrift mit Titelcharakter. Er steht häufig im
    Seitenkopf, seltener im Seitenfuß.} %
  \fi gesetzt. Arbeitet man mit einem lebenden Kolumnentitel, also der
  Wiederholung der ersten bzw.  zweiten Gliederungsebene in der Kopfzeile, und
  hat gleichzeitig sehr lange Überschriften, so erhält man automatisch sehr
  lange Kopfzeilen. In diesem Fall wirkt der Kopf wiederum wie ein abgesetzter
  Teil des Textkörpers und weniger wie leerer Rand.  Verstärkt wird dieser
  Effekt noch, wenn neben dem Kolumnentitel auch die Paginierung im Kopf
  erfolgt. Dadurch erhält man einen links und rechts abgeschlossenen Bereich,
  der kaum noch als leerer Rand wirkt.  Schwieriger ist es bei Paginierung im
  Fuß und Überschriften, deren Länge sehr stark schwankt. Hier kann der Kopf
  der einen Seite wie Textkörper wirken, der Kopf der anderen Seite aber
  eher wie Rand.  Keinesfalls sollte man die Seiten jedoch unterschiedlich
  behandeln.  Das würde zu vertikal springenden Köpfen führen und ist nicht
  einmal für ein Daumenkino geeignet. Ich rate in diesem Fall dazu, den Kopf
  zum Textkörper zu rechnen.

  Ganz einfach fällt die Entscheidung, wenn Kopf oder Fuß durch
  eine Linie vom eigentlichen Textkörper abgetrennt sind. Dadurch
  erhält man eine geschlossene Wirkung und der Kopf bzw. Fuß sollte
  unbedingt zum Textkörper gerechnet werden. Wie gesagt: Die durch die
  Trennlinie verbesserte Erkennung des Kopfes oder Fußes ist hier
  unerheblich. Entscheidend ist die unscharfe Betrachtung.\par
\end{Explain}

Das \Package{typearea}-Paket trifft die Entscheidung, ob ein Kopf oder Fuß zum
Textkörper gehört\important{\OptionValue{headinclude}{true}
  \OptionValue{headinclude}{false} \OptionValue{footinclude}{true}
  \OptionValue{footinclude}{false}} oder davon getrennt zum Rand gerechnet
werden muss, nicht selbst. Stattdessen kann mit den Optionen
\Option{headinclude} und \Option{footinclude} eingestellt werden, ob der Kopf
und der Fuß zum Textkörper gerechnet werden sollen. Die Optionen verstehen
dabei als \PName{Ein-Aus-Wert}\ChangedAt{v3.00}{\Package{typearea}} die
Standardwerte für einfache Schalter, die in \autoref{tab:truefalseswitch},
\autopageref{tab:truefalseswitch} angegeben sind. Man kann die Optionen auch
ohne Wertzuweisung verwenden. In diesem Fall wird \PValue{true} als
\PName{Ein-Aus-Wert} verwendet, also der Kopf oder Fuß zum Satzspiegel
gerechnet.

Wenn Sie unsicher sind, was die richtige Einstellung ist, lesen Sie bitte
obige Erläuterungen. Voreingestellt sind normalerweise
\OptionValue{headinclude}{false} und
\OptionValue{footinclude}{false}. Dies kann sich jedoch bei den
\KOMAScript-Klassen je nach Klassenoption oder bei Verwendung anderer
\KOMAScript-Pakete generell ändern (siehe \autoref{sec:maincls.options} und
\autoref{cha:scrlayer-scrpage}).

Bitte beachten Sie unbedingt\textnote{Achtung!}, dass diese Optionen bei
Verwendung einer der \KOMAScript-Klassen als Klassenoptionen oder per
\DescRef{\LabelBase.cmd.KOMAoptions} beziehungsweise
\DescRef{\LabelBase.cmd.KOMAoption} nach dem Laden der Klasse übergeben werden
müssen. Eine Änderung dieser Optionen nach dem Laden von \Package{typearea}
führt dabei nicht\textnote{keine automatische Neuberechnung} zu einer
automatischen Neuberechnung des Satzspiegels. Vielmehr wirkt sich die Änderung
erst bei der nächsten Neuberechnung des Satzspiegels aus. Zur Neuberechnung
des Satzspiegels siehe Option
\hyperref[desc:\LabelBase.option.DIV.last]{\Option{DIV}} mit den Werten
\hyperref[desc:\LabelBase.option.DIV.last]{\PValue{last}} oder
\hyperref[desc:\LabelBase.option.DIV.current]{\PValue{current}} (siehe
\DescPageRef{typearea.option.DIV.last}) oder die Anweisung
\DescRef{\LabelBase.cmd.recalctypearea} (siehe
\DescPageRef{typearea.cmd.recalctypearea}).%
%
\EndIndexGroup

\begin{Declaration}
  \OptionVName{mpinclude}{Ein-Aus-Wert}
\end{Declaration}
Neben\ChangedAt{v2.8q}{\Class{scrbook}\and \Class{scrreprt}\and
  \Class{scrartcl}} Dokumenten, bei denen der Kopf und der Fuß der Seite eher
zum Textbereich als zum Rand gehört, gibt es auch Dokumente, bei denen dies
für Randnotizen (siehe beispielsweise Befehl \DescRef{maincls.cmd.marginpar}
in \cite{l2kurz} oder \autoref{sec:maincls.marginNotes}) zutrifft. Mit der
Option \Option{mpinclude} kann genau dies erreicht werden.  Die Option
versteht dabei als \PName{Ein-Aus-Wert}\ChangedAt{v3.00}{\Package{typearea}}
die Standardwerte für einfache Schalter, die in \autoref{tab:truefalseswitch},
\autopageref{tab:truefalseswitch} angegeben sind. Man kann die Option auch
ohne Wertzuweisung verwenden. In diesem Fall wird \PValue{true} als
\PName{Ein-Aus-Wert} verwendet.

Der Effekt von
\OptionValue{mpinclude}{true}\important{\OptionValue{mpinclude}{true}} ist,
dass eine Breiteneinheit vom Textbereich weggenommen und als Bereich für die
Randnotizen verwendet wird. Mit
\OptionValue{mpinclude}{false}, was
der Voreinstellung entspricht, wird hingegen ein Teil des Randes für
Randnotizen verwendet. Dies ist, je nachdem ob einseitig oder doppelseitig
gearbeitet wird, ebenfalls eine Breiteneinheit oder auch eineinhalb
Breiteneinheiten. In der Regel ist die Verwendung von
\OptionValue{mpinclude}{true} nicht anzuraten und sollte Experten vorbehalten
bleiben.

\begin{Explain}
  In den meisten Fällen, in denen die Option \Option{mpinclude} sinnvoll ist,
  werden außerdem breitere Randnotizen benötigt. In sehr vielen Fällen sollte
  dabei aber nicht die gesamte Breite, sondern nur ein Teil davon dem
  Textbereich zugeordnet werden. Dies ist beispielsweise der Fall, wenn der
  Rand für Zitate verwendet wird. Solche Zitate werden üblicherweise im
  Flattersatz gesetzt, wobei die bündige Kante an den Textbereich
  anschließt. Da sich kein geschlossener optischer Eindruck ergibt, dürfen die
  flatternden Enden also durchaus teilweise in den Rand ragen. Man kann das
  einfach erreichen, indem man zum einen die Option \Option{mpinclude}
  verwendet. Zum anderen vergrößert man die Länge \Length{marginparwidth} nach
  der Berechnung des Satzspiegels noch mit Hilfe der
  \Macro{addtolength}-Anweisung. Um welchen Wert man vergrößern sollte, hängt
  vom Einzelfall ab und erfordert einiges Fingerspitzengefühl. Auch deshalb
  ist die Option \Option{mpinclude} eher etwas für Experten. Natürlich kann
  man auch festlegen, dass die Randnotizen beispielsweise zu einem Drittel in
  den Rand hineinragen sollen, und das wie folgt erreichen:
\begin{lstcode}
  \setlength{\marginparwidth}{1.5\marginparwidth}
\end{lstcode}

  Da es derzeit keine Option gibt, um mehr Platz für die Randnotizen innerhalb
  des Textbereichs vorzusehen, gibt es nur eine Möglichkeit, dies zu
  erreichen: %
\iftrue % Umbruchkorrektur
  Die Anpassung von \Length{textwidth}\IndexLength{textwidth} und
  \Length{marginparwidth}\IndexLength{marginparwidth} nach der Berechnung des
  Satzspiegels. Siehe dazu
  \DescRef{typearea-experts.cmd.AfterCalculatingTypearea} in
  \autoref{sec:typearea-experts.experts},
  \DescPageRef{typearea-experts.cmd.AfterCalculatingTypearea}.
%
\else
%
  Man verzichtet auf die Option \Option{mpinclude} oder setzt
  \Option{mpinclude} auf \PValue{false}, verringert nach der
  Satzspiegelberechnung die Breite des Textbereichs \Macro{textwidth} und
  setzt die Breite des Bereichs der Randnotizen auf den Wert, um den man die
  Breite des Textbereichs verringert hat. Leider lässt sich dieses Vorgehen
  nicht mit der automatischen Berechnung des \PName{DIV}-Wertes verbinden.
  Demgegenüber wird \Option{mpinclude} bei
  \OptionValueRef{\LabelBase}{DIV}{calc}\IndexOption{DIV=calc} (siehe
  \DescPageRef{typearea.option.DIV.calc}) berücksichtigt.
%
\fi
\end{Explain}

Bitte beachten Sie unbedingt\textnote{Achtung!}, dass diese Option bei
Verwendung einer der \KOMAScript-Klassen als Klassenoption oder per
\DescRef{\LabelBase.cmd.KOMAoptions} beziehungsweise oder
\DescRef{\LabelBase.cmd.KOMAoption} nach dem Laden der Klasse übergeben werden
muss. Eine Änderung dieser Option nach dem Laden von \Package{typearea} führt
nicht\textnote{keine automatische Neuberechnung} zu einer automatischen
Neuberechnung des Satzspiegels. Vielmehr wirkt sich die Änderung erst bei der
nächsten Neuberechnung des Satzspiegels aus. Zur Neuberechnung des
Satzspiegels siehe Option
\hyperref[desc:\LabelBase.option.DIV.last]{\Option{DIV}} mit den Werten
\hyperref[desc:\LabelBase.option.DIV.last]{\PValue{last}} oder
\hyperref[desc:\LabelBase.option.DIV.current]{\PValue{current}} (siehe
\DescPageRef{typearea.option.DIV.last}) oder die Anweisung
\DescRef{\LabelBase.cmd.recalctypearea} (siehe
\DescPageRef{typearea.cmd.recalctypearea}).%
%
\EndIndexGroup


\begin{Declaration}
  \OptionVName{headlines}{Zeilenanzahl}
  \OptionVName{headheight}{Höhe}
\end{Declaration}%
\BeginIndex{}{Kopf>Hoehe=Höhe}%
Es ist nun also bekannt, wie man Satzspiegel mit dem \Package{typearea}-Paket
berechnet und wie man dabei angibt, ob der Kopf oder Fuß zum Textkörper oder
zum Rand gehört. Insbesondere für den Kopf fehlt aber noch die Angabe, wie
hoch er denn eigentlich sein soll. Hierzu dienen die Optionen
\Option{headlines} und
\Option{headheight}\ChangedAt{v3.00}{\Package{typearea}}.

Die Option \Option{headlines}\important{\Option{headlines}} setzt man dabei
auf die Anzahl der Kopfzeilen.  Normalerweise arbeitet das
\Package{typearea}-Paket mit 1,25 Kopfzeilen. Dieser Wert stellt einen
Kompromiss dar. Zum einen ist er groß genug, um auch für eine unterstrichene
Kopfzeile (siehe \autoref{sec:maincls.pagestyle}) Platz zu bieten, zum anderen
ist er klein genug, um das Randgewicht nicht zu stark zu verändern, wenn mit
einer einfachen, nicht unterstrichenen Kopfzeile gearbeitet wird. Damit ist
der voreingestellte Wert in den meisten Standardfällen ein guter Wert.  In
einigen Fällen will oder muss man aber die Kopfhöhe genauer den tatsächlichen
Erfordernissen anpassen.

\begin{Example}
  Angenommen, es soll ein Text mit einem zweizeiligen Kopf erstellt
  werden. Normalerweise würde dies dazu führen, dass auf jeder Seite
  eine Warnung »\texttt{overfull} \Macro{vbox}« von {\LaTeX}
  ausgegeben würde. Um dies zu verhindern, wird das
  \Package{typearea}-Paket angewiesen, einen entsprechenden Satzspiegel
  zu berechnen:
\begin{lstcode}
  \documentclass[a4paper]{article}
  \usepackage[headlines=2.1]{typearea}
\end{lstcode}
  Es ist auch wieder möglich und bei Verwendung einer \KOMAScript-Klasse
  empfehlenswert, diese Option direkt an die Klasse zu übergeben:
\begin{lstcode}
  \documentclass[headlines=2.1]{scrartcl}
\end{lstcode}
  Befehle, mit denen dann der Inhalt der zweizeiligen Kopfzeile definiert
  werden kann, sind in \autoref{cha:scrlayer-scrpage} zu finden.
\end{Example}

In einigen Fällen ist es nützlich, wenn man die Kopfhöhe nicht in Zeilen,
sondern direkt als Längenwert angeben kann. Dies ist mit Hilfe der alternativ
verwendbaren Option \Option{headheight}\important{\Option{headheight}}
möglich. Als \PName{Höhe} sind alle Längen und Größen verwendbar, die \LaTeX{}
kennt. Es ist jedoch zu beachten, dass bei Verwendung einer \LaTeX-Länge wie
\Length{baselineskip} nicht deren Größe zum Zeitpunkt des Setzens der Option,
sondern zum Zeitpunkt der Berechnung des Satzspiegels und der Ränder
entscheidend ist. Außerdem\textnote{Achtung!} sollten \LaTeX-Längen wie
\Length{baselineskip} keinesfalls im optionalen Argument von
\DescRef{\LabelBase.cmd.documentclass} oder
\DescRef{\LabelBase.cmd.usepackage} verwendet werden.

Bitte beachten Sie unbedingt\textnote{Achtung!}, dass diese Optionen bei
Verwendung einer der \KOMAScript-Klassen als Klassenoptionen oder per
\DescRef{\LabelBase.cmd.KOMAoptions} beziehungsweise
\DescRef{\LabelBase.cmd.KOMAoption} nach dem Laden der Klasse übergeben werden
müssen. Eine Änderung dieser Optionen nach dem Laden von \Package{typearea}
führt nicht\textnote{keine automatische Neuberechnung} zu einer automatischen
Neuberechnung des Satzspiegels. Vielmehr wirkt sich die Änderung erst bei der
nächsten Neuberechnung des Satzspiegels aus. Zur Neuberechnung des
Satzspiegels siehe Option
\hyperref[desc:\LabelBase.option.DIV.last]{\Option{DIV}} mit den Werten
\hyperref[desc:\LabelBase.option.DIV.last]{\PValue{last}} oder
\hyperref[desc:\LabelBase.option.DIV.current]{\PValue{current}} (siehe
\DescPageRef{typearea.option.DIV.last}) oder die Anweisung
\DescRef{\LabelBase.cmd.recalctypearea} (siehe
\DescPageRef{typearea.cmd.recalctypearea}).%
%
\EndIndexGroup


\begin{Declaration}
  \OptionVName{footlines}{Zeilenanzahl}%
  \OptionVName{footheight}{Höhe}%
  \Length{footheight}%
\end{Declaration}%
\BeginIndex{}{Fuss=Fuß>Hoehe=Höhe}%
Wie schon für den Kopf fehlt aber noch die Angabe, wie hoch der Fuß sein
soll. Hierzu dienen die Optionen \Option{footlines} und
\Option{footheight}\ChangedAt{v3.12}{\Package{typearea}}. Allerdings ist die
Höhe des Fußes im Gegensatz zur Höhe des Kopfes keine Länge des \LaTeX-Kerns
selbst. Daher definiert \Package{typearea} zur Einführung eine neue Länge
\Length{footheight}\IndexLength[indexmain]{footheight}, falls diese noch nicht
existiert. Ob diese dann auch beispielsweise von Klassen und Paketen für die
Gestaltung von Kopf und Fuß verwendet wird, hängt von den verwendeten Klassen
und Paketen ab. Das \KOMAScript-Paket
\hyperref[cha:scrlayer-scrpage]{\Package{scrlayer-scrpage}}%
\important{\hyperref[cha:scrlayer-scrpage]{\Package{scrlayer-scrpage}}}%
\IndexPackage{scrlayer-scrpage} berücksichtigt \Length{footheight} und
arbeitet somit aktiv mit \Package{typearea} zusammen. Die \KOMAScript-Klassen
berücksichtigen \Length{footheight} hingegen nicht, da sie ohne
Paketunterstützung nur Seitenstile mit einzeiligen Seitenfüßen anbieten.

Die Option \Option{footlines}\important{\Option{footlines}} setzt man
vergleichbar zu \DescRef{\LabelBase.option.headlines} auf die Anzahl der
Fußzeilen.  Normalerweise arbeitet das \Package{typearea}-Paket mit 1,25
Fußzeilen. Dieser Wert stellt einen Kompromiss dar. Zum einen ist er groß
genug, um auch für eine über- und unterstrichene Fußzeile (siehe
\autoref{sec:maincls.pagestyle}) Platz zu bieten, zum anderen ist er klein
genug, um das Randgewicht nicht zu stark zu verändern, wenn mit einer
einfachen Fußzeile ohne Trennlinien gearbeitet wird. Damit ist der
voreingestellte Wert in den meisten Standardfällen ein guter Wert.  In einigen
Fällen will oder muss man aber die Fußhöhe genauer den tatsächlichen
Erfordernissen anpassen.

\begin{Example}
  Angenommen, im Fuß soll eine zweizeilige Copyright-Angabe gesetzt
  werden. Zwar gibt es in \LaTeX{} selbst keinen Test, ob der für den Fuß
  vorgesehene Platz dafür genügend Raum bietet, die Überschreitung der
  vorgesehenen Höhe resultiert aber wahrscheinlich in einer unausgeglichenen
  Verteilung von Satzspiegeln und Rändern. Außerdem führt beispielsweise das
  Paket \hyperref[cha:scrlayer-scrpage]{\Package{scrlayer-scrpage}}%
  \important{\hyperref[cha:scrlayer-scrpage]{\Package{scrlayer-scrpage}}}%
  \IndexPackage{scrlayer-scrpage}, mit dem ein solcher Fuß\-in\-halt gesetzt
  werden könnte, durchaus eine entsprechende Überprüfung durch und meldet
  gegebenenfalls auch Überschreitungen. Daher ist es sinnvoll, die benötigte
  größere Fußhöhe bereits bei der Berechnung des Satzspiegels anzugeben:
\begin{lstcode}
  \documentclass[a4paper]{article}
  \usepackage[footlines=2.1]{typearea}
\end{lstcode}
  Es ist auch wieder möglich und bei Verwendung einer \KOMAScript-Klasse
  empfehlenswert, diese Option direkt an die Klasse zu übergeben:
\begin{lstcode}
  \documentclass[footlines=2.1]{scrartcl}
\end{lstcode}
  Befehle, mit denen dann der Inhalt der zweizeiligen Fußzeile
  definiert werden kann, sind in \autoref{cha:scrlayer-scrpage} zu finden.
\end{Example}

In einigen Fällen ist es nützlich, wenn man die Fußhöhe nicht in Zeilen,
sondern direkt als Längenwert angeben kann. Dies ist mit Hilfe der alternativ
verwendbaren Option \Option{footheight}\important{\Option{footheight}}
möglich. Als \PName{Höhe} sind alle Längen und Größen verwendbar, die \LaTeX{}
kennt. Es ist jedoch zu beachten, dass bei Verwendung einer \LaTeX-Länge wie
\Length{baselineskip} nicht deren Größe zum Zeitpunkt des Setzens der Option,
sondern zum Zeitpunkt der Berechnung des Satzspiegels und der Ränder
entscheidend ist. Außerdem\textnote{Achtung!} sollten \LaTeX-Längen wie
\Length{baselineskip} keinesfalls im optionalen Argument von
\DescRef{\LabelBase.cmd.documentclass} oder
\DescRef{\LabelBase.cmd.usepackage} verwendet werden.

Bitte beachten Sie unbedingt\textnote{Achtung!}, dass diese Optionen bei
Verwendung einer der \KOMAScript-Klassen als Klassenoptionen oder per
\DescRef{\LabelBase.cmd.KOMAoptions} beziehungsweise
\DescRef{\LabelBase.cmd.KOMAoption} nach dem Laden der Klasse übergeben werden
müssen. Eine Änderung dieser Optionen nach dem Laden von \Package{typearea}
führt nicht\textnote{keine automatische Neuberechnung} zu einer automatischen
Neuberechnung des Satzspiegels. Vielmehr wirkt sich die Änderung erst bei der
nächsten Neuberechnung des Satzspiegels aus. Zur Neuberechnung des
Satzspiegels siehe Option
\hyperref[desc:\LabelBase.option.DIV.last]{\Option{DIV}} mit den Werten
\hyperref[desc:\LabelBase.option.DIV.last]{\PValue{last}} oder
\hyperref[desc:\LabelBase.option.DIV.current]{\PValue{current}} (siehe
\DescPageRef{typearea.option.DIV.last}) oder die Anweisung
\DescRef{\LabelBase.cmd.recalctypearea} (siehe
\DescPageRef{typearea.cmd.recalctypearea}).%
\EndIndexGroup


\begin{Declaration}
  \Macro{areaset}\OParameter{BCOR}\Parameter{Breite}\Parameter{Höhe}
\end{Declaration}%
Bis hier wurde nun eine Menge darüber erzählt, wie man einen guten
Satzspiegel\Index{Satzspiegel} für Standardanwendungen erstellt und wie das
\Package{typearea}-Paket dem Anwender diese Arbeit erleichtert, ihm aber
gleichzeitig Möglichkeiten der Einflussnahme bietet. Es gibt jedoch auch
Fälle, in denen der Textkörper eine bestimmte Größe exakt einhalten soll, ohne
dass dabei auf gute Satzspiegelkonstruktion oder auf weitere Nebenbedingungen
zu achten ist. Trotzdem sollen die Ränder so gut wie möglich verteilt und
dabei gegebenenfalls auch eine Bindekorrektur berücksichtigt werden. Das
\Package{typearea}-Paket bietet hierfür den Befehl \Macro{areaset}, dem man
neben der optionalen Bindekorrektur als Parameter die Breite und Höhe des
Textbereichs übergibt. Die Ränder und deren Verteilung werden dann automatisch
berechnet, wobei gegebenenfalls auch die Einstellungen der Paketoptionen
\DescRef{\LabelBase.option.headinclude} und
\DescRef{\LabelBase.option.footinclude} berücksichtigt werden. Die Optionen
\DescRef{\LabelBase.option.headlines}\IndexOption{headlines}\textnote{Achtung!},
\DescRef{\LabelBase.option.headheight}\IndexOption{headheight},
\DescRef{\LabelBase.option.footlines}\IndexOption{footlines} und
\DescRef{\LabelBase.option.footheight}\IndexOption{footheight} bleiben in
diesem Fall jedoch unberücksichtigt! Siehe dazu die weiterführenden
Informationen zu \DescRef{typearea-experts.cmd.areaset} auf
\DescPageRef{typearea-experts.cmd.areaset} in
\autoref{sec:typearea-experts.experimental}.

Die Voreinstellung für \PName{BCOR} ist 0\Unit{pt}. Soll hingegen die aktuelle,
beispielsweise per Option \DescRef{\LabelBase.option.BCOR}\IndexOption{BCOR}
eingestellte Bindekorrektur erhalten bleiben, sollte man den symbolischen Wert
\PValue{current} als optionales Argument verwenden.

\begin{Example}
  Angenommen, ein Text auf A4-Papier soll genau die Breite von 60
  Zeichen in der Typewriter-Schrift haben und exakt 30 Zeilen je Seite
  besitzen. Dann könnte mit folgender Präambel gearbeitet werden:
\begin{lstcode}
  \documentclass[a4paper,11pt]{article}
  \usepackage{typearea}
  \newlength{\CharsLX}% Breite von 60 Zeichen
  \newlength{\LinesXXX}% Hoehe von 30 Zeilen
  \settowidth{\CharsLX}{\texttt{1234567890}}
  \setlength{\CharsLX}{6\CharsLX}
  \setlength{\LinesXXX}{\topskip}
  \addtolength{\LinesXXX}{29\baselineskip}
  \areaset{\CharsLX}{\LinesXXX}
\end{lstcode}
  Der Faktor von 29 statt 30 ist damit begründet, dass die Grundlinie der
  obersten Zeile bereits am obersten Rand des um \Macro{topskip} verringerten
  Satzspiegels liegt, solange die Höhe der obersten Zeile kleiner als
  \Macro{topskip} ist. Die oberste Zeile benötigt damit keine Höhe. Die
  Unterlängen der untersten Zeile ragen dafür unter den Satzspiegel.

\iffalse % Umbruchkorrekturtext
  Soll stattdessen ein Gedichtband gesetzt werden, bei dem es nur
  darauf ankommt, dass der Textbereich genau quadratisch mit einer
  Seitenlänge von 15\Unit{cm} ist, wobei ein Binderand von
  1\Unit{cm} zu berücksichtigen ist, so kann dies wie folgt
  erreicht werden:%
\else %
\iffalse %
  Ein quadratischer Gedichtband im Format 15\Unit{cm}\,:\,15\Unit{cm} mit
  mit 1\Unit{cm} Binderand gesetzt werden, ginge so:%
\else %
  Soll stattdessen ein Gedichtband mit quadratischem Textbereich der
  Seitenlänge 15\Unit{cm} und
  einem Binderand von 1\Unit{cm} gesetzt
  werden, so ist Folgendes möglich:%
\fi %
\fi %
% Korrektur von Code am Ende eines Beispiels am Ende einer Erklärung
\begin{lstcode}[belowskip=-1.5\baselineskip]
  \documentclass{gedichte}
  \usepackage{typearea}
  \areaset[1cm]{15cm}{15cm}
\end{lstcode}%
\end{Example}%
\EndIndexGroup


\begin{Declaration}
  \OptionValue{DIV}{areaset}
\end{Declaration}%
In\ChangedAt{v3.00}{\Package{typearea}} seltenen Fällen ist es nützlich, wenn
man den aktuell eingestellten Satzspiegel neu ausrichten lassen kann. Dies ist
mit der Option
\OptionValue{DIV}{areaset}\important{\OptionValue{DIV}{areaset}} möglich,
wobei \DescRef{\LabelBase.cmd.KOMAoptions}\PParameter{DIV=areaset} der
Anweisung
\begin{lstcode}
  \areaset[current]{\textwidth}{\textheight}
\end{lstcode}
entspricht. Dasselbe Ergebnis erhält man%
\iffalse auch\fi% Umbruchkorrekturtext!
, wenn \OptionValueRef{\LabelBase}{DIV}{last} verwendet wird und der
Satzspiegel zuletzt per \DescRef{\LabelBase.cmd.areaset} eingestellt wurde.%
%
\EndIndexGroup

\iffalse% Umbruchkorrekturtext: Alternativen
\iftrue%
Wenn Sie konkrete Vorgaben bezüglich der Ränder zu erfüllen haben, ist
\Package{typearea} nicht geeignet. In diesem Fall ist die Verwendung des
Pakets \Package{geometry}\IndexPackage{geometry}\important{\Package{geometry}}
(siehe \cite{package:geometry}) empfehlenswert.%
\else%
Das Paket \Package{typearea} ist nicht dafür gedacht, bestimmte Randbreiten
einzustellen. Dafür ist das Paket \Package{geometry}\IndexPackage{geometry}
(siehe \cite{package:geometry}) empfehlenswert.%
\fi%
\fi


\section{Einstellung des Papierformats}%
\seclabel{paperTypes}%
\BeginIndexGroup

Das Papierformat ist ein \iffalse entscheidendes \fi% Umbruchkorrekturtext
Grundmerkmal eines Dokuments. Wie
bereits bei der Vorstellung der \iffalse unterstützten \fi% Umbruchkorrekturtext
Satzspiegelkonstruktionen (siehe
\autoref{sec:typearea.basics} bis \autoref{sec:typearea.circleConstruction} ab
\autopageref{sec:typearea.basics})
aufgezeigt, steht und fällt die Auf"|teilung der Seite und damit das gesamte
Dokumentlayout mit der Wahl des Papierformats. Während die
\LaTeX-Standardklassen auf einige wenige Formate festgelegt sind, unterstützt
\KOMAScript{} mit dem Paket \Package{typearea} selbst ausgefallene
Seitengrößen.


\begin{Declaration}
  \OptionVName{paper}{Format}
  \OptionVName{paper}{Ausrichtung}
\end{Declaration}%
Die Option \Option{paper}\ChangedAt{v3.00}{\Package{typearea}} ist das
zentrale Element der Formatauswahl\important[i]{%
  \begin{tabular}[t]{@{}l@{}l@{}}
    \KOption{paper} & \PValue{letter}, \PValue{legal}, \\
    & \PValue{executive}, \\
    & \PValue{A0}, \PValue{A1}, \PValue{A2} \dots\\
    & \PValue{B0}, \PValue{B1}, \PValue{B2} \dots\\
    & \PValue{C0}, \PValue{C1}, \PValue{C2} \dots\\
    & \PValue{D0}, \PValue{D1}, \PValue{D2} \dots\end{tabular}} %
bei \KOMAScript. Als \PName{Format} wird dabei zunächst das amerikanische
\Option{letter}, \Option{legal} und \Option{executive} unterstützt. Darüber
hinaus sind die ISO-Formate der Reihen A, B, C und D möglich, also
beispielsweise \PValue{A4} oder -- klein geschrieben -- \PValue{a4}. 

Querformate\important{%
  \begin{tabular}[t]{@{}l@{}l@{}}
    \KOption{paper} & \PValue{landscape}, \\
                    & \PValue{seascape}
  \end{tabular}%
} werden dadurch unterstützt, dass man die Option ein weiteres Mal mit dem
Wert \PValue{landscape}\Index{Papier>Ausrichtung} oder
\PValue{seascape}\ChangedAt{v3.02c}{\Package{typearea}} angibt. Dabei
unterscheiden sich \PValue{landscape} und \PValue{seascape} nur darin, dass
das Programm \File{dvips} bei \PValue{landscape} um -90\Unit{\textdegree}
dreht, während bei \PValue{seascape} um +90\Unit{\textdegree} gedreht
wird. Hilfreich ist \PValue{seascape} also vor allem dann, wenn ein
PostScript-Anzeigeprogramm die Seiten im Querformat auf dem Kopf stellt.
Damit der Unterschied eine Rolle spielt, darf auch die nachfolgend
beschriebene Option \DescRef{\LabelBase.option.pagesize}%
\IndexOption{pagesize}\important{\DescRef{\LabelBase.option.pagesize}} nicht
deaktiviert sein.

Zusätzlich kann das \PName{Format} auch in der Form
\PName{Breite}\texttt{:}\PName{Höhe}\ChangedAt{v3.01b}{\Package{typearea}}%
\important{\OptionVName{paper}{Breite\textup{:}Höhe}} beziehungsweise
\PName{Höhe}\texttt{:}\PName{Breite}\ChangedAt{v3.22}{\Package{typearea}}%
\important{\OptionVName{paper}{Höhe\textup{:}Breite}} angegeben
werden. Welcher Wert die \PName{Höhe} und welcher die \PName{Breite} ist,
richtet sich nach der Ausrichtung des Papiers. Mit
\OptionValue{paper}{landscape} oder \OptionValue{paper}{seascape} ist der
kleinere Wert die \PName{Höhe} und der größere Wert die \PName{Breite}. Mit
\OptionValue{paper}{portrait}\important{\OptionValue{paper}{portrait}} ist
dagegen der kleinere Wert die \PName{Breite} und der größere Wert die
\PName{Höhe}.

Es\textnote{Achtung!} wird darauf hingewiesen, dass bis Version~3.01a der
erste Wert immer die \PName{Höhe} und der zweite Wert die \PName{Breite}
war. Dagegen war von Version~3.01b bis Version~3.21a der erste Wert immer die
\PName{Breite} und der zweite Wert immer die \PName{Höhe}. Dies ist
insbesondere dann zu beachten, wenn mit einer entsprechenden
Kompatibilitätseinstellung (siehe Option \DescRef{\LabelBase.option.version}%
\IndexOption{version}\important{\DescRef{\LabelBase.option.version}},
\autoref{sec:typearea.compatibilityOptions},
\DescPageRef{typearea.option.version}) gearbeitet wird.

\begin{Example}
  Angenommen, es soll eine Karteikarte im Format ISO-A8 quer bedruckt
  werden. Dabei sollen die Ränder sehr klein gewählt werden. Außerdem
  wird auf eine Kopf- und eine Fußzeile verzichtet.
\begin{lstcode}
  \documentclass{article}
  \usepackage[headinclude=false,footinclude=false,%
              paper=A8,paper=landscape]{typearea}
  \areaset{7cm}{5cm}
  \pagestyle{empty}
  \begin{document}
  \section*{Definierte Papierformate}
  letter, legal, executive, a0, a1 \dots\ %
  b0, b1 \dots\ c0, c1 \dots\ d0, d1 \dots
  \end{document}
\end{lstcode}
  Haben die Karteikarten das Sonderformat (Breite:Höhe)
  5\Unit{cm}\,:\,3\Unit{cm}, so ist dies mit
\begin{lstcode}
  \documentclass{article}
  \usepackage[headinclude=false,footinclude=false,
              paper=landscape,paper=5cm:3cm]{typearea}
  \areaset{4cm}{2.4cm}
  \pagestyle{empty}
  \begin{document}
  \section*{Definierte Papierformate}
  letter, legal, executive, a0, a1 \dots\ %
  b0, b1 \dots\ c0, c1 \dots\ d0, d1 \dots
  \end{document}
\end{lstcode}
  möglich.
\end{Example}

In der Voreinstellung wird bei \KOMAScript{} mit A4-Papier in der Ausrichtung
portrait gearbeitet. Dies ist ein Unterschied\textnote{\KOMAScript{}
  vs. Standardklassen} zu den Standardklassen, bei denen in der Voreinstellung
das amerikanische Format letter verwendet wird.

Bitte beachten Sie unbedingt\textnote{Achtung!}, dass diese Option bei
Verwendung einer der \KOMAScript-Klassen als Klassenoption oder per
\DescRef{\LabelBase.cmd.KOMAoptions} beziehungsweise
\DescRef{\LabelBase.cmd.KOMAoption} nach dem Laden der Klasse übergeben werden
muss.  Eine Änderung des Papierformats oder der Papierausrichtung mit Hilfe
der Anweisung \DescRef{\LabelBase.cmd.KOMAoptions}\textnote{keine automatische
  Neuberechnung} oder \DescRef{\LabelBase.cmd.KOMAoption} nach dem Laden von
\Package{typearea} führt nicht zu einer automatischen Neuberechnung des
Satzspiegels. Vielmehr wirkt sich die Änderung erst bei der nächsten
Neuberechnung des Satzspiegels aus. Zur Neuberechnung des Satzspiegels siehe
Option \hyperref[desc:\LabelBase.option.DIV.last]{\Option{DIV}} mit den Werten
\hyperref[desc:\LabelBase.option.DIV.last]{\PValue{last}} oder
\hyperref[desc:\LabelBase.option.DIV.current]{\PValue{current}} (siehe
\DescPageRef{typearea.option.DIV.last}) oder die Anweisung
\DescRef{\LabelBase.cmd.recalctypearea} (siehe
\DescPageRef{typearea.cmd.recalctypearea}).%
\EndIndexGroup


\begin{Declaration}
  \OptionVName{pagesize}{Ausgabetreiber}
\end{Declaration}%
\begin{Explain}%
  Die oben genannten Mechanismen zur Auswahl des Papierformats \iffree{haben
    nur insofern einen Einfluss auf die Ausgabe, als interne {\LaTeX}-Maße
    gesetzt werden}{führen nur zum Setzen interner {\LaTeX}-Maße}. Das Paket
  \Package{typearea} verwendet diese \iffree{dann }{}bei der Auf"|teilung der
  Seite in Ränder und Textbereich. Die Spezifikation des
  DVI-Formats\Index{DVI} sieht aber an keiner Stelle Angaben zum Papierformat
  vor. %
  \iffree{%
    Wird direkt aus dem DVI-Format in eine Low-Level-Druckersprache wie PCL%
    \iftrue% Umbruchkorrektur
    \footnote{PCL ist eine Familie von Druckersprachen, die HP für seine
      Tinten- und Laserdrucker verwendet.}%
    \fi \ oder ESC/P2%
    \iftrue% Umbruchkorrektur
    \footnote{ESC/P2 ist die Druckersprache, die EPSON für seine 24-Nadel- und
      ältere Tinten- oder Laserdrucker benutzt.}%
    \fi \ beziehungsweise ESC/P-R%
    \iftrue% Umbruchkorrektur
    \footnote{ESC/P-R ist die Druckersprache, die EPSON aktuell für Tinten-
      und Laserdrucker benutzt.}%
    \fi \ ausgegeben, spielt dies normalerweise keine Rolle%
  }{%
    In früheren Zeiten, als DVI zum Ausruck noch direkt in
    Low-Level-Druckersprachen übertragen wurde, spielte dies keine Rolle%
  }, da auch bei diesen Ausgaben der 0-Bezugspunkt wie bei DVI links oben
  liegt. \iffree{Wird aber}{Heutzutage wird aber meist} in Sprachen wie
  \iffree{PostScript\Index{PostScript} oder }{}% Umbruchkorrekturtext
  PDF\Index{PDF} übersetzt, bei denen der 0-Bezugspunkt an anderer Stelle
  liegt und außerdem das Papierformat in der Ausgabedatei angegeben werden
  sollte\iffree{, so}{. Dabei} fehlt diese Information. Als Lösung des
  Problems verwendet der entsprechende Treiber eine voreingestellte
  Papiergröße, die der Anwender entweder per Option oder durch entsprechende
  Angabe in der {\TeX}-Quelldatei verändern kann. Bei Verwendung des
  DVI-Treibers \File{dvips} oder \File{dvipdfm} kann diese Angabe in Form
  einer \Macro{special}-Anweisung erfolgen. Bei Verwendung von {pdf\TeX},
  {lua\TeX}, {\XeTeX} oder {V\TeX} werden deren Papierformat-Längen
  entsprechend gesetzt.%
\end{Explain}
Mit der Option \Option{pagesize} kann eingestellt werden, für welchen
\PName{Ausgabetreiber} die Papiergröße in das Dokument geschrieben wird. Die
unterstützten \PName{Ausgabetreiber} sind \autoref{tab:typearea.outputdriver}
\iffalse auf \autopageref{tab:typearea.outputdriver}\fi zu
entnehmen. Voreingestellt\ChangedAt{v3.17}{\Package{typearea}} ist
\Option{pagesize}. Diese Verwendung der Option ohne Angabe eines Wertes
entspricht \OptionValue{pagesize}{auto}.
%
\begin{table}
  \caption{Ausgabetreiber für Option \OptionVName{pagesize}{Ausgabetreiber}}
  \begin{desctabular}
    \pventry{auto}{Falls die pdf\TeX-spezifischen Register
      \Macro{pdfpagewidth}\IndexLength{pdfpagewidth} und
      \Macro{pdfpageheight}\IndexLength{pdfpageheight} oder die
      lua\TeX-spezifischen Register \Macro{pagewidth}\IndexLength{pagewidth}
      und \Macro{pageheight}\IndexLength{pageheight} vorhanden sind, wird
      der Ausgabetreiber \PValue{pdftex} aktiviert. Zusätzlich wird auch der
      Ausgabetreiber \PValue{dvips}
      verwendet. Diese Einstellung ist grundsätzlich auch für \XeTeX{}
      geeignet.%
      \IndexOption{pagesize~=\textKValue{auto}}}%
    \pventry{automedia}{Dies entspricht dem Ausgabetreiber
      \PValue{auto}. Allerdings werden zusätzlich auch noch die
      \mbox{V\TeX}-spezifischen Register
      \Macro{mediawidth}\IndexLength{mediawidth} und
      \Macro{mediaheight}\IndexLength{mediaheight} gesetzt, falls diese
      definiert sind.%
      \IndexOption{pagesize~=\textKValue{automedia}}}%
    \entry{\PValue{false}, \PValue{no}, \PValue{off}}{%
      Die Papiergröße wird nicht an den Ausgabetreiber
      gemeldet.%
      \IndexOption{pagesize~=\textKValue{false}}}%
    \pventry{dvipdfmx}{\ChangedAt{v3.05a}{\Package{typearea}}Die Papiergröße
      wird als
      \Macro{special}\PParameter{pagesize=\PName{Breite},\PName{Höhe}} in die
      DVI-Datei geschrieben. Der Name des Ausgabetreibers kommt daher, dass
      das Programm \File{dvipdfmx} eine Pa\-pier\-for\-mat\-um\-schal\-tung
      über diese Anweisung auch innerhalb des Dokuments erlaubt.%
      \IndexOption{pagesize~=\textKValue{dvipdfmx}}}%
    \pventry{dvips}{Bei Verwendung innerhalb der Dokumentpräambel wird die
      Papiergröße über
      \Macro{special}\PParameter{pagesize=\PName{Breite},\PName{Höhe}} in das
      Dokument geschrieben. Da das Programm \File{dvips} keine
      Pa\-pier\-for\-mat\-um\-schal\-tung innerhalb des Dokuments unterstützt,
      wird bei Bedarf im Dokument ein recht unsauberer Hack verwendet, um die
      Umschaltung nach Möglichkeit dennoch zu
      erreichen. Pa\-pier\-for\-mat\-um\-schal\-tung nach der Dokumentprämbel
      bei gleichzeitiger Verwendung des Ausgabetreibers \PValue{dvips}
      erfolgen daher auf eigene Gefahr!%
      \IndexOption{pagesize~=\textKValue{dvips}}}%
    \entry{\PValue{pdftex}, \PValue{luatex}}{%
      \ChangedAt{v3.20}{\Package{typearea}}Die Papiergröße
      wird über die pdf\TeX-spezifischen Register
      \Macro{pdfpagewidth}\IndexLength{pdfpagewidth} und 
      \Macro{pdfpageheight}\IndexLength{pdfpageheight} oder die
      lua\TeX-spezifischen Register \Macro{pagewidth}\IndexLength{pagewidth}
      und \Macro{pageheight}\IndexLength{pageheight} gesetzt. Dies ist
      auch jederzeit innerhalb des Dokuments problemlos
      möglich.%
      \IndexOption{pagesize~=\textKValue{pdftex}}%
      \IndexOption{pagesize~=\textKValue{luatex}}}%
  \end{desctabular}
  \label{tab:typearea.outputdriver}
\end{table}

\begin{Example}
  Angenommen, es soll ein Dokument sowohl als DVI-Datei verwendet
  werden, als auch eine Online-Version im PDF-Format erstellt
  werden. Dann könnte die Präambel beispielsweise so beginnen:
  \begin{lstcode}[%
    aboveskip=.5\baselineskip plus .1\baselineskip minus .1\baselineskip,%
    belowskip=.4\baselineskip plus .1\baselineskip minus .1\baselineskip]
  \documentclass{article}
  \usepackage[paper=A4,pagesize]{typearea}
\end{lstcode}
  Wird nun für die Bearbeitung {pdf\TeX} verwendet \emph{und} die
  PDF-Ausgabe aktiviert, so werden die beiden Spezialgrößen
  \Macro{pdfpagewidth} und \Macro{pdfpageheight} entsprechend gesetzt.
  Wird jedoch eine DVI-Datei erzeugt -- egal ob mit {\LaTeX} oder
  {pdf\LaTeX} --, so wird ein \Macro{special} an den Anfang dieser
  Datei geschrieben.
\end{Example}%
\iffree{}{%
  \vskip-1\ht\strutbox plus .75\ht\strutbox% Wegen Beispiel am Ende der Erklärung
}%
\EndIndexGroup
%
\EndIndexGroup

% Umbruchverbesserung:
%\iffree{}{\clearpage}
\section{Tipps}
\seclabel{tips}

Insbesondere für die Erstellung von schriftlichen Arbeiten während des
Studiums findet man häufig Vorschriften\textnote{Satzvorschriften
  vs. gute~Typografie}, die einer typografischen Begutachtung nicht nur in
keiner Weise standhalten, sondern massiv gegen alle Regeln der Typografie
verstoßen. Ursache für solche Regeln ist oft typografische Inkompetenz
derjenigen, die sie herausgeben. Manchmal ist die Ursache auch im
Ausgangspunkt begründet, nämlich der Schreibmaschine. Mit einer
Schreibmaschine oder einer Textverarbeitung von 1980 ist es
ohne
erheblichen Aufwand kaum möglich, typografisch perfekte Ergebnisse zu
erzielen. Also wurden einst Vorschriften erlassen, die leicht erfüllbar
schienen und dem Korrektor trotzdem entgegenkommen. Dazu zählen dann
Randeinstellungen, die für einseitigen Druck mit einer Schreibmaschine zu
brauchbaren Zeilenlängen führen. Um nicht extrem kurze Zeilen zu erhalten, die
durch Flattersatz zudem verschlimmert werden, werden die Ränder schmal
gehalten und für Korrekturen stattdessen ein großer Durchschuss in Form von
eineinhalbzeiligem Satz vorgeschrieben.  Bevor moderne
Textverarbeitungssysteme verfügbar wurden, wäre -- außer mit {\TeX} --
einzeiliger Satz die einzige Alternative gewesen. Dabei wäre dann selbst das
Anbringen von Korrekturzeichen schwierig geworden. Als die Verwendung von
Computern für die Erstellung schriftlicher Arbeiten üblicher wurde, hat sich
manches Mal auch der Spieltrieb des einen oder anderen Studenten gezeigt, der
durch Verwendung einer Schmuckschrift seine Arbeit aufpeppen und so eine
bessere Note mit weniger Einsatz herausschinden wollte.  Nicht bedacht hat er
dabei, dass solche Schriften schlechter zu lesen und deshalb für den Zweck
ungeeignet sind. Damit hielten zwei Brotschriften Einzug in die Vorschriften,
die weder zusammenpassen noch im Falle von Times wirklich gut geeignet
sind. Times ist eine relativ enge Schrift, die Anfang des 20.~Jahrhunderts
speziell für schmale Spalten im englischen Zeitungssatz entworfen wurde. In
modernen Schnitten ist dies etwas entschärft. Dennoch passt die häufig
vorgeschriebene Times meist nicht zu den gleichzeitig gegebenen Randvorgaben.

{\LaTeX} setzt bereits von sich aus mit ausreichendem Durchschuss.
Gleichzeitig sind die Ränder bei sinnvollen Zeilenlängen groß genug, um Platz
für Korrekturen zu bieten. Dabei wirkt die Seite trotz einer Fülle von Text
großzügig angelegt.

Oft sind die typografisch mehr als fragwürdigen Satzvorschriften mit {\LaTeX}
auch außerordentlich schwierig umzusetzen. So kann eine feste Anzahl von
»Anschlägen« nur dann eingehalten werden, wenn keine proportionale Schrift
verwendet wird. Es gibt nur wenige gute nichtproportionale Schriften.
\iffalse % Umbruchkorrektur
Kaum ein Text, der mit einer derartigen Schrift gesetzt ist, wirkt
gleichmäßig. So wird häufig %
\else Häufig wird %
\fi %
versucht, durch ausladende Serifen beispielsweise beim kleinen »i« oder »l«
die unterschiedliche Breite der Zeichen auszugleichen. Dies kann nicht
funktionieren. Im Ergebnis wirkt der Text unruhig und zerrissen. Außerdem
verträgt sich eine solche Schrift kaum mit dem im deutschen Sprachraum
üblichen und allgemein vorzuziehenden Blocksatz. Gewisse Vorgaben können daher
bei Verwendung von {\LaTeX} nur ignoriert oder großzügig ausgelegt werden,
etwa indem man »60~Anschläge pro Zeile« nicht als feste, sondern als
durchschnittliche oder maximale Angabe interpretiert.

Wie ausgeführt, sind Satzvorschriften meist dazu gedacht, ein brauchbares
Ergebnis zu erhalten, auch wenn der Ausführende selbst nicht weiß, was dabei
zu beachten ist. Brauchbar bedeutet häufig: lesbar und korrigierbar. Nach
meiner Auf"|fassung wird ein mit {\LaTeX} und dem \Package{typearea}-Paket
gesetzter Text bezüglich des Satzspiegels diesen Anforderungen von vornherein
gerecht. Wenn Sie also mit Vorschriften konfrontiert sind, die offensichtlich
erheblich davon abweichen, so empfehle ich, dem Betreuer einen Textauszug
vorzulegen und nachzufragen, ob es gestattet ist, die Arbeit trotz der
Abweichungen in dieser Form zu liefern. Gegebenenfalls kann durch Veränderung
der Option
\DescRef{\LabelBase.option.DIV}\important{\DescRef{\LabelBase.option.DIV}} der
Satzspiegel moderat angepasst werden. Von der Verwendung von
\DescRef{\LabelBase.cmd.areaset} zu diesem Zweck rate ich jedoch
ab. Schlimmstenfalls verwenden Sie das nicht zu {\KOMAScript} gehörende
\Package{geometry}-Paket\important{\Package{geometry}}\IndexPackage{geometry}
(siehe \cite{package:geometry}) oder verändern Sie die Satzspiegelparameter
von {\LaTeX} selbst. Die von \Package{typearea} ermittelten Werte finden Sie
in der \File{log}-Datei Ihres Dokuments. Mit Hilfe von Option
\DescRef{typearea-experts.option.usegeometry}%
\important{\DescRef{typearea-experts.option.usegeometry}}, die Sie in
\autoref{part:forExperts} finden, kann außerdem die Zusammenarbeit von
\Package{typearea} und \Package{geometry} verbessert werden. Damit sollten
moderate Anpassungen möglich sein. Achten Sie jedoch unbedingt darauf, dass
die Proportionen des Textbereichs mit denen der Seite unter Berücksichtigung
der Bindekorrektur annähernd übereinstimmen.

Sollte es unbedingt erforderlich sein, den Text eineinhalbzeilig zu setzen, so
definieren Sie keinesfalls \Macro{baselinestretch} um. Dieses Vorgehen wird
zwar allzu häufig empfohlen, ist aber seit der Einführung von {\LaTeXe} im
Jahre 1994 obsolet.  Verwenden Sie schlimmstenfalls den Befehl
\Macro{linespread}. Ich empfehle das Paket
\Package{setspace}\important{\Package{setspace}}\IndexPackage{setspace}, das
nicht zu {\KOMAScript} gehört (siehe \cite{package:setspace}).  Auch sollten
Sie \Package{typearea} nach der Umstellung des Zeilenabstandes den Satzspiegel
für diesen Abstand berechnen lassen, jedoch für den Titel, besser auch für die
Verzeichnisse -- sowie das Literaturverzeichnis und den Index -- wieder auf
normalen Satz umschalten. Näheres dazu finden Sie bei der Erklärung zu
\OptionValueRef{\LabelBase}{DIV}{current}%
\important{\OptionValueRef{\LabelBase}{DIV}{current}}.

Das \Package{typearea}-Paket berechnet auch bei der Option
\OptionValueRef{\LabelBase}{DIV}{calc}%
\important{\OptionValueRef{\LabelBase}{DIV}{calc}} einen sehr großzügigen
Textbereich. Viele konservative Typografen werden feststellen, dass die
resultierende Zeilenlänge noch zu groß ist. Der berechnete
\DescRef{\LabelBase.option.DIV}-Wert ist ebenfalls in der \File{log}-Datei zum
jeweiligen Dokument zu finden. Sie können also leicht nach dem ersten
\LaTeX-Lauf einen kleineren Wert wählen.

Nicht\textnote{Was ist besser?} selten wird mir die Frage gestellt, warum ich
eigentlich kapitelweise auf einer Satzspiegelberechnung herumreite, während es
sehr viel einfacher wäre, nur ein Paket zur Verfügung zu stellen, mit dem man
die Ränder wie bei einer Textverarbeitung einstellen kann.  Oft wird auch
behauptet, ein solches Paket wäre ohnehin die bessere Lösung, da jeder selbst
wisse, wie gute Ränder zu wählen seien, und die Ränder von {\KOMAScript} wären
ohnehin nicht gut. Ich erlaube mir zum Abschluss dieses Kapitels ein passendes
Zitat von Hans Peter Willberg und Friedrich Forssmann, zwei der angesehensten
Typografen der Gegenwart (siehe \cite{TYPO:ErsteHilfe}):
\begin{quote}
  \phantomsection\label{sec:typearea.tips.cite}%
  \textit{Das\textnote{Zitat!} Selbermachen ist längst üblich, die Ergebnisse
    oft fragwürdig, weil Laien-Typografen nicht sehen, was nicht stimmt und
    nicht wissen können, worauf es ankommt. So gewöhnt man sich an falsche und
    schlechte Typografie.} [\dots] \textit{Jetzt könnte der Einwand kommen,
    Typografie sei doch Geschmackssache. Wenn es um Dekoration ginge, könnte
    man das Argument vielleicht gelten lassen, da es aber bei Typografie in
    erster Linie um Information geht, können Fehler nicht nur stören, sondern
    sogar Schaden anrichten.}
\end{quote}
%
\EndIndexGroup

%%% Local Variables:
%%% mode: latex
%%% coding: utf-8
%%% TeX-master: "../guide"
%%% End:
