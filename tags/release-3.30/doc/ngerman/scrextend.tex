% ======================================================================
% scrextend.tex
% Copyright (c) Markus Kohm, 2002-2019
%
% This file is part of the LaTeX2e KOMA-Script bundle.
%
% This work may be distributed and/or modified under the conditions of
% the LaTeX Project Public License, version 1.3c of the license.
% The latest version of this license is in
%   http://www.latex-project.org/lppl.txt
% and version 1.3c or later is part of all distributions of LaTeX 
% version 2005/12/01 or later and of this work.
%
% This work has the LPPL maintenance status "author-maintained".
%
% The Current Maintainer and author of this work is Markus Kohm.
%
% This work consists of all files listed in manifest.txt.
% ----------------------------------------------------------------------
% scrextend.tex
% Copyright (c) Markus Kohm, 2002-2019
%
% Dieses Werk darf nach den Bedingungen der LaTeX Project Public Lizenz,
% Version 1.3c, verteilt und/oder veraendert werden.
% Die neuste Version dieser Lizenz ist
%   http://www.latex-project.org/lppl.txt
% und Version 1.3c ist Teil aller Verteilungen von LaTeX
% Version 2005/12/01 oder spaeter und dieses Werks.
%
% Dieses Werk hat den LPPL-Verwaltungs-Status "author-maintained"
% (allein durch den Autor verwaltet).
%
% Der Aktuelle Verwalter und Autor dieses Werkes ist Markus Kohm.
% 
% Dieses Werk besteht aus den in manifest.txt aufgefuehrten Dateien.
% ======================================================================
%
% Package scrextend for Document Writers
% Maintained by Markus Kohm
%
% ----------------------------------------------------------------------
%
% Paket scrextend für Dokument-Autoren
% Verwaltet von Markus Kohm
%
% ======================================================================

\KOMAProvidesFile{scrextend.tex}
                 [$Date$
                  KOMA-Script package scrextend]

\chapter[tocentry={Grundlegende Fähigkeiten der
  \KOMAScript-Klassen\protect\linebreak
  mit Hilfe des Pakets \Package{scrextend} anderen Klassen erschließen},
  head={Fähigkeiten von \KOMAScript-Klassen mit \Package{scrextend}}]
{Grundlegende Fähigkeiten der \KOMAScript-Klassen mit Hilfe des Pakets
  \Package{scrextend} anderen Klassen erschließen}
\labelbase{scrextend}
\BeginIndexGroup
\BeginIndex{Package}{scrextend}%

Es gibt einige Möglichkeiten, die allen \KOMAScript-Klassen gemeinsam
sind. Dies betrifft in der Regel nicht nur die Klassen \Class{scrbook},
\Class{scrreprt} und \Class{scrartcl}, die als Ersatz für die Standardklassen
\Class{book}, \Class{report} und \Class{article} für Bücher, Berichte und
Artikel gedacht sind, sondern in weiten Teilen auch die \KOMAScript-Klasse
\Class{scrlttr2} für Briefe. Diese grundlegenden Möglichkeiten werden von
\KOMAScript{} teilweise auch durch Paket \Package{scrextend}
angeboten. Dieses Paket sollte nicht\textnote{Achtung!} mit
\KOMAScript-Klassen verwendet werden. Es ist ausschließlich zur Verwendung mit
anderen Klassen gedacht. Der Versuch, das Paket mit einer \KOMAScript-Klasse
zu laden, wird von \Package{scrextend} erkannt und mit einer Warnung
abgebrochen.

Dass \hyperref[cha:scrlttr2]{\Package{scrletter}}\IndexPackage{scrletter}%
\important{\hyperref[cha:scrlttr2]{\Package{scrletter}}} nicht nur mit
\KOMAScript-Klassen, sondern auch mit den Standardklassen verwendet werden
kann, liegt übrigens teilweise an \Package{scrextend}. Stellt
\hyperref[cha:scrlttr2]{\Package{scrletter}} nämlich fest, dass es nicht mit
einer \KOMAScript-Klasse verwendet wird, so lädt es automatisch
\Package{scrextend}. Damit stehen dann alle von
\hyperref[cha:scrlttr2]{\Package{scrletter}} aktiv genutzten Möglichkeiten der
\KOMAScript-Klassen zur Verfügung.

Es gibt natürlich keine Garantie, dass \Package{scrextend} mit jeder
beliebigen Klasse zusammenarbeitet. Das Paket ist primär für die Erweiterung
der Standardklassen und davon abgeleiteter Klassen gedacht. In jedem Fall
sollten Benutzer zunächst prüfen, ob die verwendete Klasse nicht selbst
entsprechende Möglichkeiten bereitstellt.

\iffalse% Umruchkorrektur
Neben den in diesem Kapitel beschriebenen Möglichkeiten gibt es einige
weitere, die hauptsächlich für Klassen- und Paketautoren gedacht sind. %
Diese sind in \autoref{cha:scrbase}, ab \autopageref{cha:scrbase} zu
finden. Das dort dokumentierte Paket
\hyperref[cha:scrbase]{\Package{scrbase}}%
\important{\hyperref[cha:scrbase]{\Package{scrbase}}}\IndexPackage{scrbase}
wird von allen \KOMAScript-Klassen und dem Paket \Package{scrextend}
verwendet.%
\else%
Einige grundlegende Möglichkeiten, die hauptsächlich für Klassen- und
Paketautoren interessant sind, werden bei \KOMAScript{} von
\hyperref[cha:scrbase]{\Package{scrbase}}%
\important{\hyperref[cha:scrbase]{\Package{scrbase}}}\IndexPackage{scrbase}
bereitgestellt. Das Paket wird von allen \KOMAScript-Klassen und den meisten
\KOMAScript-Paketen geladen. Dadurch sind dessen in \autoref{cha:scrbase}, ab
\autopageref{cha:scrbase} dokumentierte Möglichkeiten bei Verwendung von
\Package{scrextend} ebenfalls verfügbar.%
\fi

Auch das Paket \hyperref[cha:scrlfile]{\Package{scrlfile}}%
\important{\hyperref[cha:scrlfile]{\Package{scrlfile}}}\IndexPackage{scrlfile}
aus \autoref{cha:scrlfile} ab \autopageref{cha:scrlfile} wird von allen
\KOMAScript-Klassen und dem Paket \Package{scrextend} geladen.  Somit stehen
auch dessen Möglichkeiten bei Verwendung von \Package{scrextend} zur
Verfügung.

\iftrue % Umbruchkorrekturtext
Im Unterschied dazu wird das ebenfalls für Klassen- und Paketautoren gedachte
Paket \hyperref[cha:tocbasic]{\Package{tocbasic}}%
\important{\hyperref[cha:tocbasic]{\Package{tocbasic}}} (siehe
\autoref{cha:tocbasic} ab \autopageref{cha:tocbasic}) nur von den Klassen
\Class{scrbook}, \Class{scrreprt} und \Class{scrartcl} geladen, so dass die
dort definierten Möglichkeiten auch nur in diesen Klassen und nicht in
\Package{scrextend} zu finden sind. Natürlich spricht nichts dagegen,
\hyperref[cha:tocbasic]{\Package{tocbasic}} zusätzlich zu
\Package{scrextend} zu laden.%
\fi

\LoadCommonFile{options}% \section{Frühe oder späte Optionenwahl bei \KOMAScript}

\LoadCommonFile{compatibility}% \section{Kompatibilität zu früheren Versionen von \KOMAScript}


\section{Optionale, erweiterte Möglichkeiten}
\seclabel{optionalFeatures}

Das Paket \Package{scrextend} kennt optional verfügbare, erweiterte
Möglichkeiten. Das sind Möglichkeiten, die in der Grundeinstellung nicht
vorhanden sind, aber zusätzlich ausgewählt werden können. Diese
sind beispielsweise deshalb optional, weil sie potentiell in Konflikt mit den
Möglichkeiten der Standardklassen oder häufig benutzter Pakete stehen.

\begin{Declaration}
  \OptionVName{extendedfeature}{Möglichkeit}
\end{Declaration}
Mit dieser Option kann eine optionale Möglichkeit von \Package{scrextend}
ausgewählt werden. Diese Option steht nur während des Ladens von
\Package{scrextend} zur Verfügung. Anwender geben diese Option daher als
optionales Argument von
\DescRef{\LabelBase.cmd.usepackage}\PParameter{scrextend} an. %
\iffree{%
  Eine Übersicht über die verfügbaren optionalen Möglichkeiten bietet
  \autoref{tab:scrextend.optionalFeatures}.

  \begin{table}
    \caption[{Optional verfügbare, erweiterte Möglichkeiten
      von \Package{scrextend}}]{Übersicht über die optional verfügbaren,
      erweiterten Möglichkeiten von \Package{scrextend}}
    \label{tab:scrextend.optionalFeatures}
    \begin{desctabular}
      \entry{\PName{title}}{%
        die Titelseiten werden auf die Möglichkeiten der \KOMAScript-Klassen
        erweitert; dies betrifft neben den Anweisungen für die Titelseiten
        auch die Option \DescRef{\LabelBase.option.titlepage} (siehe
        \autoref{sec:scrextend.titlepage}, ab
        \autopageref{sec:scrextend.titlepage})%
      }
    \end{desctabular}
  \end{table}%
}{%
  \par%
  Die derzeit einzige optionale Möglichkeit ist
  \PValue{title}\IndexOption[indexmain]{extendedfeature~=\textKValue{title}}%
  \important{\OptionValue{extendedfeature}{title}}.
  Damit werden die Titelseiten auf die Möglichkeiten der \KOMAScript-Klassen
  erweitert, wie sie in \autoref{sec:scrextend.titlepage} ab
  \autopageref{sec:scrextend.titlepage} beschrieben sind.%
}%
%
\EndIndexGroup


\LoadCommonFile{draftmode}% \section{Entwurfsmodus}

\LoadCommonFile{fontsize} % \section{Wahl der Schriftgröße für das Dokument}

\LoadCommonFile{textmarkup}% \section{Textauszeichnungen}

\LoadCommonFile{titles}% \section{Dokumenttitel}

\LoadCommonFile{oddorevenpage}% \section{Erkennung von rechten und linken Seiten}


\section{Wahl eines vordefinierten Seitenstils}
\seclabel{pagestyle}

\iffalse % Umbruchkorrektur
Eine der allgemeinen Eigenschaften eines Dokuments ist der
Seitenstil\Index[indexmain]{Seiten>Stil}. Bei {\LaTeX} versteht man unter dem
Seitenstil in erster Linie den Inhalt der Kopf- und Fußzeilen. %
\fi%
Das Paket \Package{scrextend} definiert selbst keine Seitenstile, nutzt aber
Seitenstile des \LaTeX-Kerns.


\begin{Declaration}
  \Macro{titlepagestyle}
\end{Declaration}%
\BeginIndex{}{Titel>Seitenstil}%
Auf einigen Seiten wird mit Hilfe von
\DescRef{maincls.cmd.thispagestyle}\IndexCmd{thispagestyle} automatisch ein
anderer Seitenstil gewählt. Bei \Package{scrextend} betrifft dies bisher nur
die Titelseiten und auch dies nur, wenn mit
\OptionValueRef{\LabelBase}{extendedfeature}{title} gearbeitet wird (siehe
\autoref{sec:scrextend.optionalFeatures},
\DescPageRef{scrextend.option.extendedfeature}). Welcher Seitenstil in diesem
Fall für einen
Titelkopf\important{\OptionValueRef{\LabelBase}{titlepage}{false}}%
\IndexOption{titlepage~=\textKValue{false}}\Index{Titel>Kopf} verwendet wird,
ist im Makro \Macro{titlepagestyle} festgelegt. In der Voreinstellung ist das
der Seitenstil \DescRef{maincls.pagestyle.plain}\IndexPagestyle{plain}. Dieser
Seitenstil wird bereits im \LaTeX-Kern vordefiniert und sollte daher immer
verfügbar sein.%
\EndIndexGroup

\LoadCommonFile{interleafpage}% \section{Vakatseiten}

\LoadCommonFile{footnotes}% \section{Fußnoten}

\LoadCommonFile{dictum}% \section{Schlauer Spruch}

\LoadCommonFile{lists}% \section{Listen}

\LoadCommonFile{marginpar}% \section{Randnotizen}
%
\EndIndexGroup

\endinput

%%% Local Variables:
%%% mode: latex
%%% coding: utf-8
%%% TeX-master: "guide.tex"
%%% End:
