% ======================================================================
% scrdatetime.tex
% Copyright (c) Markus Kohm, 2001-2019
%
% This file is part of the LaTeX2e KOMA-Script bundle.
%
% This work may be distributed and/or modified under the conditions of
% the LaTeX Project Public License, version 1.3c of the license.
% The latest version of this license is in
%   http://www.latex-project.org/lppl.txt
% and version 1.3c or later is part of all distributions of LaTeX 
% version 2005/12/01 or later and of this work.
%
% This work has the LPPL maintenance status "author-maintained".
%
% The Current Maintainer and author of this work is Markus Kohm.
%
% This work consists of all files listed in manifest.txt.
% ----------------------------------------------------------------------
% scrdatetime.tex
% Copyright (c) Markus Kohm, 2001-2019
%
% Dieses Werk darf nach den Bedingungen der LaTeX Project Public Lizenz,
% Version 1.3c, verteilt und/oder veraendert werden.
% Die neuste Version dieser Lizenz ist
%   http://www.latex-project.org/lppl.txt
% und Version 1.3c ist Teil aller Verteilungen von LaTeX
% Version 2005/12/01 oder spaeter und dieses Werks.
%
% Dieses Werk hat den LPPL-Verwaltungs-Status "author-maintained"
% (allein durch den Autor verwaltet).
%
% Der Aktuelle Verwalter und Autor dieses Werkes ist Markus Kohm.
% 
% Dieses Werk besteht aus den in manifest.txt aufgefuehrten Dateien.
% ======================================================================
%
% Chapter about scrpage2 of the KOMA-Script guide
% Maintained by Markus Kohm
%
% ----------------------------------------------------------------------
%
% Kapitel ueber scrpage2 in der KOMA-Script-Anleitung
% Verwaltet von Markus Kohm
%
% ============================================================================

\KOMAProvidesFile{scrdatetime.tex}
                 [$Date$
                  KOMA-Script guide (chapter: scrdate, scrtime)]
\translator{Markus Kohm\and Gernot Hassenpflug\and Karl Hagen}

% Date of the translated German file: 2019-11-15

\chapter{The Day of the Week with \Package{scrdate}}
\labelbase{scrdate}
\BeginIndexGroup
\BeginIndex{Package}{scrdate}

Originally, the \Package{scrdate} package could only give the day of the week
for the current date. Nowadays, it offers this and more for any date in the
Gregorian calendar.

\begin{Declaration}
  \Macro{CenturyPart}\Parameter{year}\\%
  \Macro{DecadePart}\Parameter{year}%
\end{Declaration}%
The\ChangedAt{v3.05a}{\Package{scrdate}} \Macro{CenturyPart} command returns
the value of the century digits\,---\,thousands and hundreds\,---\,of a
\PName{year}. The \Macro{DecadePart} command, on the other hand, gives the
value of the remaining digits, i.\,e. the tens and the units. The
\PName{year} can have any number of digits. You can assign the value directly
to a counter or use it for calculations with
\Macro{numexpr}\IndexCmd{numexpr}. To output\textnote{Attention!} it as an
Arabic number, you should prefix it with \Macro{the}\IndexCmd{the}.

\begin{Example}
  You want to calculate and print the century of the current year.
\begin{lstcode}
  The year \the\year\ is year \the\DecadePart{\year}
  of the \engord{\numexpr\CenturyPart{\year}+1\relax} century.
\end{lstcode}
  The result would be:
  \begin{ShowOutput}
    The year \the\year\ is year \the\DecadePart{\year}
    of the \engordnumber{\numexpr\CenturyPart{\year}+1\relax} century.
  \end{ShowOutput}
  This example uses the \Package{engord}\IndexPackage{engord} package.
  See \cite{package:engord} for more information.
\end{Example}

Note\textnote{Attention!} that the counting method used here treats the year
2000 as year~0\,---\,and therefore the first year\,---\,of the 21st~century.
If necessary, however, you can make a correction with \Macro{numexpr}, as
shown for the ordinal number in the example.%
\EndIndexGroup


\begin{Declaration}
  \Macro{DayNumber}\Parameter{year}\Parameter{month}\Parameter{day}\\%
  \Macro{ISODayNumber}\Parameter{ISO-date}%
\end{Declaration}%
These\ChangedAt{v3.05a}{\Package{scrdate}} two commands return the value of
the day-of-the-week\Index{day>of the week} number for any date. They differ
only in the method of specifying the date. While the \Macro{DayNumber} command
requires the \PName{year}, \PName{month}, and \PName{day} as separate
parameters, the \Macro{ISODayNumber} command expects an \PName{ISO-date} as a
single argument, \PName{ISO-date}, using the ISO notation
\PName{year}\texttt{-}\PName{month}\texttt{-}\PName{day}. It does not matter
if the \PName{month} or \PName{day} have one or two digits. You can use the
result of both commands to assign directly to a counter or for calculations
using \Macro{numexpr}\IndexCmd{numexpr}. To print\textnote{Attention!} it as
an Arabic number, you should prefix it with \Macro{the}\IndexCmd{the}.

\begin{Example}
  You want to know the number of the day of the week of the 1st~May~2027.
\begin{lstcode}
  The 1st~May~2027 has \the\ISODayNumber{2027-5-1}
  as the number of the day of the week.
\end{lstcode}
  The result will be:
  \begin{ShowOutput}
    The 1st~May~2027 has \the\ISODayNumber{2027-5-1}
    as the number of the day of the week.
  \end{ShowOutput}
\end{Example}

It is particularly worth noting that you can even step a specified number of
days into the future or or the past from a given date.
\begin{Example}
  You want to know the number of the day of the week 12~days from now
  and 24~days before the 24th~December~2027.
\begin{lstcode}
  In 12~days, the number of the day of the week
  will be \the\DayNumber{\year}{\month}{\day+12}, and
  24~days before the 24th~December~2027 it will be
  \the\ISODayNumber{2027-12-24-24}.
\end{lstcode}
  The result could be, for example:
  \begin{ShowOutput}
    In 12~days, the number of the day of the week
    will be \the\DayNumber{\year}{\month}{\day+12}, and
    24~days before the 24th~December~2027 it will be
    \the\ISODayNumber{2027-12-24-24}.
  \end{ShowOutput}
\end{Example}

The days of the week are numbered as follows: Sunday\,=\,0, Monday\,=\,1,
Tuesday\,=\,2, Wednesday\,=\,3, Thursday\,=\,4, Friday\,=\,5, and
Saturday\,=\,6.%
%
\EndIndexGroup


\begin{Declaration}
  \Macro{DayNameByNumber}\Parameter{number of the day of the week}\\%
  \Macro{DayName}\Parameter{year}\Parameter{month}\Parameter{day}\\%
  \Macro{ISODayName}\Parameter{ISO-date}%
\end{Declaration}%
Usually\ChangedAt{v3.05a}{\Package{scrdate}} you are less interested in the
number of the day of the week than in its name. Therefore, the
\Macro{DayNameByNumber} command returns the name of the day of the week
corresponding to a day-of-the-week number. This number can be the result, for
example, of \Macro{DayNumber} or \Macro{ISODayNumber}. The two commands
\Macro{DayName} and \Macro{ISODayName} directly return the name of the day of
the week of a given date.

\begin{Example}
  You want to know the name of the day of the week of the 24th~December~2027.
\begin{lstcode}
  Please pay by \ISODayName{2027-12-24},
  24th~December~2027 the amount of \dots.
\end{lstcode}
  The result will be:
  \begin{ShowOutput}
    Please pay by \ISODayName{2027-12-24},
    24th~December~2027 the amount of \dots.
  \end{ShowOutput}
\end{Example}

Once again, it is particularly worth noting that you can perform calculations,
to a certain extent:
\begin{Example}
  You want to know the names of the day of the week 12~days from now
  and 24~days before the 24th~December~2027.
\begin{lstcode}
  In 12~days, the name of the day of the week
  will be \DayName{\year}{\month}{\day+12}, and
  24~days before the 24th~December~2027 it will be
  \ISODayName{2027-12-24-24}, while two weeks
  and three days after a Wednesday will be a
  \DayNameByNumber{3+2*7+3}.
\end{lstcode}
  The result could be, for example:
  \begin{ShowOutput}
    In 12~days, the name of the day of the week
    will be \DayName{\year}{\month}{\day+12}, and
    24~days before the 24th~December~2027 it will be
    \ISODayName{2027-12-24-24}, while two weeks
    and three days after a Wednesday will be a
    \DayNameByNumber{3+2*7+3}.
  \end{ShowOutput}
\end{Example}%
%
\EndIndexGroup


\begin{Declaration}
  \Macro{ISOToday}%
  \Macro{IsoToday}%
  \Macro{todaysname}%
  \Macro{todaysnumber}%
\end{Declaration}%
In the previous examples, the current date was always specified cumbersomely
using the \TeX{} registers \Macro{year}\IndexCmd{year},
\Macro{month}\IndexCmd{month}, and \Macro{day}\IndexCmd{day}. The
\Macro{ISOToday}\ChangedAt{v3.05a}{\Package{scrdate}} and \Macro{IsoToday}
commands directly return the current date in ISO-notation. These commands
differ only in the fact that \Macro{ISOToday} always outputs a two-digit month
and day, while \Macro{IsoToday} outputs single-digit numbers for values less
than 10. The \Macro{todaysname} command directly returns the name of the
current day of the week, while \Macro{todaysnumber} returns the number of the
current day of the week. You can find more information about using this value
in the explanations of \DescRef{scrdate.cmd.DayNumber} and
\DescRef{scrdate.cmd.ISODayNumber}.

\begin{Example}
  I want to show you on what day of the week this document was typeset:
\begin{lstlisting}
  This document was created on a \todaysname.
\end{lstlisting}
  This will result, for example, in:
  \begin{ShowOutput}
    This document was created on a \todaysname.
  \end{ShowOutput}
\end{Example}

For languages that have a case system for nouns, note that the package cannot
decline words. The terms are given in the form appropriate for displaying a
date in a letter, which is the nominative singular for the currently supported
languages. Given this limitation, the example above will not work correctly if
translated directly into some other languages.

\begin{Explain}
  The\textnote{Hint!} names of the weekdays in \Package{scrdate} all have
  initial capital letters. If you need the names completely in lower case, for
  example because that is the convention in the relevant language, simply wrap
  the command with the \LaTeX{} \Macro{MakeLowercase}\IndexCmd{MakeLowercase}%
  \important{\Macro{MakeLowercase}} command:
  % Umbruchkorrektur: listings
\begin{lstcode}
  \MakeLowercase{\todaysname}
\end{lstcode}
  This converts the whole argument into lower-case letters. Of course, you can
  also do this for
  \DescRef{scrdate.cmd.DayNameByNumber}\IndexCmd{DayNameByNumber},
  \DescRef{scrdate.cmd.DayName}\IndexCmd{DayName}, and
  \DescRef{scrdate.cmd.ISODayName}\IndexCmd{ISODayName} commands described
  above.%
\end{Explain}%
\EndIndexGroup


\begin{Declaration}
  \Macro{nameday}\Parameter{name}
\end{Declaration}%
Just as you can directly modify the output of \Macro{today} with
\DescRef{maincls.cmd.date}\IndexCmd{date}, so you can change the output of
\DescRef{scrdate.cmd.todaysname} to \PName{name} with \Macro{nameday}.
\begin{Example}
  You change the current date to a fixed value using
  \DescRef{maincls.cmd.date}. You are not interested in the actual name of the
  day, but want only to show that it is a workday. So you write:
\begin{lstlisting}
  \nameday{workday}
\end{lstlisting}
  After this, the previous example will result in:
  \begin{ShowOutput}\nameday{workday}
    This document was created on a \todaysname.
  \end{ShowOutput}
\end{Example}
There's no corresponding command to change the result of
\DescRef{scrdate.cmd.ISOToday}\IndexCmd{ISOToday} or
\DescRef{scrdate.cmd.IsoToday}\IndexCmd{IsoToday}.%
\EndIndexGroup


\begin{Declaration}
  \Macro{newdaylanguage}\Parameter{language}%
                        \Parameter{Monday}\Parameter{Tuesday}%
                        \Parameter{Wednesday}\Parameter{Thursday}%
                        \Parameter{Friday}\Parameter{Saturday}
                        \Parameter{Sunday}%
\end{Declaration}
Currently the \Package{scrdate} package recognizes the following languages:
\begin{itemize}\setlength{\itemsep}{.5\itemsep}
\item Croatian (\PValue{croatian}),
\item Czech (\PValue{czech}\ChangedAt{v3.13}{\Package{scrdate}}),
\item Danish (\PValue{danish}),
\item Dutch (\PValue{dutch}),
\item English (\PValue{american}\ChangedAt{v3.13}{\Package{scrdate}},
  \PValue{australian}, \PValue{british}, \PValue{canadian}, \PValue{english},
  \PValue{UKenglish}, and USenglish),
\item Finnish (\PValue{finnish}),
\item French (\PValue{acadian}, \PValue{canadien},
  \PValue{francais}\ChangedAt{v3.13}{\Package{scrdate}}, and \PValue{french}),
\item German (\PValue{austrian}\ChangedAt{v3.08b}{\Package{scrdate}},
  \PValue{german}, \PValue{naustrian}, \PValue{ngerman},
  \PValue{nswissgerman}, and
  \PValue{swissgerman}\ChangedAt{v3.13}{\Package{scrdate}}),
\item Italian (\PValue{italian}),
\item Norwegian (\PValue{norsk}),
\item Polish (\PValue{polish}\ChangedAt{v3.13}{\Package{scrdate}}),
\item Slovak (\PValue{slovak}),
\item Spanish (\PValue{spanish}),
\item Swedish (\PValue{swedish}).
\end{itemize}
You can also configure it for additional languages. To do so, the first
argument of \Macro{newdaylanguage} is the name of the language, and the other
arguments are the names of the corresponding days of the week.

In the current implementation, it does not matter whether you load
\Package{scrdate} before or after \Package{ngerman}\IndexPackage{ngerman},
\Package{babel}\IndexPackage{babel}, or similar packages. In each case the
correct language will be used provided it is supported.

\begin{Explain}
  To be more precise, as long as you select a language in a way that is
  compatible with \Package{babel}\IndexPackage{babel}, \Package{scrdate} will
  use the correct language. If this is not the case, you will get (US) English
  names.
\end{Explain}

Of course, if you create definitions for a language that was previously
unsupported, please mail them to the author of \KOMAScript{}. There is a good
chance that future versions of \KOMAScript{} will add support for that
language.%
\EndIndexGroup
%
\EndIndexGroup


\chapter{The Current Time with \Package{scrtime}}
\labelbase{scrtime}
\BeginIndexGroup
\BeginIndex{Package}{scrtime}

This package lets you find the current time. Starting with version~3.05, the
package also supports the option interface already familiar from the
\KOMAScript{} classes and various other \KOMAScript{} packages. See, for
example, \autoref{sec:typearea.options} for more information.

\begin{Declaration}%
  \Macro{thistime}\OParameter{delimiter}%
  \Macro{thistime*}\OParameter{delimiter}
\end{Declaration}%
\Macro{thistime} returns the current time\Index{time} in hours and minutes.
The delimiter between the values of hour, minutes and seconds can be given in
the optional argument. The default is the character ``\PValue{:}''.

\Macro{thistime*} works in almost the same way as \Macro{thistime}. The only
difference is that, unlike \Macro{thistime}, \Macro{thistime*} does not add a
leading zero to the minute field when its value is less than 10. Thus, with
\Macro{thistime} the minute field has always two places.
\begin{Example}
  The line
\begin{lstlisting}
  Your train departs at \thistime.
\end{lstlisting}
  results, for example, in:
  \begin{ShowOutput}
    Your train departs at \thistime.
  \end{ShowOutput}
  or:
  \begin{ShowOutput}
    Your train departs at 23:09.
  \end{ShowOutput}
  \bigskip
  In contrast to the previous example a line like:
\begin{lstlisting}
  This day is already \thistime*[\ hours and\ ] minutes old.
\end{lstlisting}
  results in:
  \begin{ShowOutput}
    This day is already \thistime*[\ hours and\ ] minutes old.
  \end{ShowOutput}
  or:
  \begin{ShowOutput}
    This day is already 12 hours and 25 minutes old.
  \end{ShowOutput}
\end{Example}
\EndIndexGroup


\begin{Declaration}%
 \Macro{settime}\Parameter{time}
\end{Declaration}%
\Macro{settime} sets the output of \DescRef{scrtime.cmd.thistime} and
\DescRef{scrtime.cmd.thistime*} to a fixed value. In this case, the optional
parameter of \DescRef{scrtime.cmd.thistime} or \DescRef{scrtime.cmd.thistime*}
is ignored, since the complete string returned by
\DescRef{scrtime.cmd.thistime} or \DescRef{scrtime.cmd.thistime*} has been
explicitly defined. \Macro{settime}.%
\EndIndexGroup


\begin{Declaration}
  \OptionVName{12h}{simple switch}%
\end{Declaration}%
\BeginIndex{Option}{24h}%
With the \Option{12h}\ChangedAt{v3.05a}{\Package{scrtime}} option, you can
select whether to print the time given by \DescRef{scrtime.cmd.thistime} and
\DescRef{scrtime.cmd.thistime*} in 12- or 24-hour format. The option accepts
the values for simple switches listed in \autoref{tab:truefalseswitch},
\autopageref{tab:truefalseswitch}. Using the option without a value is
equivalent to \OptionValue{12h}{true}, and therefore activates the
12-hour-format. The default, however, is \Option{24h}.

You can set this option globally in the optional argument of
\DescRef{typearea.cmd.documentclass}, as a package option in the optional
argument of \DescRef{typearea.cmd.usepackage}, or even after loading the
package using \DescRef{typearea.cmd.KOMAoptions} or
\DescRef{typearea.cmd.KOMAoption} (see, e.\,g. \autoref{sec:typearea.options},
\DescPageRef{typearea.cmd.KOMAoptions}). However the option no longer has any
effect on the if you call \DescRef{scrtime.cmd.settime}. After invoking this
command, the time is output only with the value and in the format specified
there.

For\textnote{Attention!} the sake of compatibility with earlier versions of
\Package{scrtime}, the option \Option{24h} will switch to 24-hour format if
used in the optional argument of \Macro{documentclass} or \Macro{usepackage}.
However, you should not use this option any longer.%
\EndIndexGroup
%
\EndIndexGroup

\endinput

%%% Local Variables: 
%%% mode: latex
%%% coding: us-ascii
%%% TeX-master: "../guide"
%%% End: 
