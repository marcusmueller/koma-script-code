% \iffalse meta-comment
% ======================================================================
% scrkernel-floats.dtx
% Copyright (c) Markus Kohm, 2002-2021
%
% This file is part of the LaTeX2e KOMA-Script bundle.
%
% This work may be distributed and/or modified under the conditions of
% the LaTeX Project Public License, version 1.3c of the license.
% The latest version of this license is in
%   http://www.latex-project.org/lppl.txt
% and version 1.3c or later is part of all distributions of LaTeX 
% version 2005/12/01 or later and of this work.
%
% This work has the LPPL maintenance status "author-maintained".
%
% The Current Maintainer and author of this work is Markus Kohm.
%
% This work consists of all files listed in manifest.txt.
% ----------------------------------------------------------------------
% scrkernel-floats.dtx
% Copyright (c) Markus Kohm, 2002-2021
%
% Dieses Werk darf nach den Bedingungen der LaTeX Project Public Lizenz,
% Version 1.3c, verteilt und/oder veraendert werden.
% Die neuste Version dieser Lizenz ist
%   http://www.latex-project.org/lppl.txt
% und Version 1.3c ist Teil aller Verteilungen von LaTeX
% Version 2005/12/01 oder spaeter und dieses Werks.
%
% Dieses Werk hat den LPPL-Verwaltungs-Status "author-maintained"
% (allein durch den Autor verwaltet).
%
% Der Aktuelle Verwalter und Autor dieses Werkes ist Markus Kohm.
% 
% Dieses Werk besteht aus den in manifest.txt aufgefuehrten Dateien.
% ======================================================================
% \fi
%
% \CharacterTable
%  {Upper-case    \A\B\C\D\E\F\G\H\I\J\K\L\M\N\O\P\Q\R\S\T\U\V\W\X\Y\Z
%   Lower-case    \a\b\c\d\e\f\g\h\i\j\k\l\m\n\o\p\q\r\s\t\u\v\w\x\y\z
%   Digits        \0\1\2\3\4\5\6\7\8\9
%   Exclamation   \!     Double quote  \"     Hash (number) \#
%   Dollar        \$     Percent       \%     Ampersand     \&
%   Acute accent  \'     Left paren    \(     Right paren   \)
%   Asterisk      \*     Plus          \+     Comma         \,
%   Minus         \-     Point         \.     Solidus       \/
%   Colon         \:     Semicolon     \;     Less than     \<
%   Equals        \=     Greater than  \>     Question mark \?
%   Commercial at \@     Left bracket  \[     Backslash     \\
%   Right bracket \]     Circumflex    \^     Underscore    \_
%   Grave accent  \`     Left brace    \{     Vertical bar  \|
%   Right brace   \}     Tilde         \~}
%
% \iffalse
%%% From File: $Id$
%<option>%%%            (run: option)
%<body>%%%            (run: body)
%<*dtx>
% \fi
\ifx\ProvidesFile\undefined\def\ProvidesFile#1[#2]{}\fi
\begingroup
  \def\filedate$#1: #2-#3-#4 #5${\gdef\filedate{#2/#3/#4}}
  \filedate$Date$
  \def\filerevision$#1: #2 ${\gdef\filerevision{r#2}}
  \filerevision$Revision: 1962 $
  \edef\reserved@a{%
    \noexpand\endgroup
    \noexpand\ProvidesFile{scrkernel-floats.dtx}%
                          [\filedate\space\filerevision\space
                           KOMA-Script source
                           (floats)]
  }%
\reserved@a
% \iffalse
\documentclass[parskip=half-]{scrdoc}
\usepackage[english,ngerman]{babel}
\CodelineIndex
\RecordChanges
\GetFileInfo{scrkernel-floats.dtx}
\title{\KOMAScript{} \partname\ \texttt{\filename}%
  \footnote{Dies ist Version \fileversion\ von Datei \texttt{\filename}.}}
\date{\filedate}
\author{Markus Kohm}

\begin{document}
  \maketitle
  \tableofcontents
  \DocInput{\filename}
\end{document}
%</dtx>
% \fi
%
% \selectlanguage{ngerman}
%
% \changes{v2.95}{2002/06/25}{%
%   erste Version aus der Aufteilung von \texttt{scrclass.dtx}}
%
% \section{Gleitumgebungen}
%
% \KOMAScript{} verfügt bezüglich der Gleitumgebungen über einige
% Features, die über das, was die Standardklassen bieten, hinaus
% geht. \marginline{Geplant!} Allerdings ist das hier verwendete Interface
% gnadenlos veraltet und wird deshalb irgendwann ersetzt. Dabei wird
% höchstwahrscheinlich grundsätzlich auf das \textsf{caption}-Paket aufgebaut
% werden. Man betrachte dieses Modul daher als grundsätzlich zur Disposition
% stehend.
%
% Die einzige Gleitumgebung, über die Briefe derzeit verfügen sind
% Randnotizen.
%
% \StopEventually{\PrintIndex\PrintChanges}
%
% \iffalse
%<*option>
% \fi
%
% \iffalse
%<*class>
%<*!letter>
% \fi
%
% \subsection{Einige Anweisungen und Einstellungen für die Optionen}
%
% Einige Anweisungen und Einstellungen werden bereits benötigt, bevor die
% Optionen abgearbeitet werden.
%
% \begin{macro}{\if@captionabove}
% \changes{v2.8h}{2001/07/19}{neu}%^^A
% \changes{v3.09}{2011/03/23}{Initialisierung vor den Optionen}%^^A
% Ob es sich bei \cs{caption} um eine Unter- oder Überschrift handelt und also
% \cs{abovecaptionskip} davor und \cs{belowcaptionskip} danach steht
% oder umgekehrt, wird über diesen Schalter gesteuert. Voreingestellt
% sind Unterschriften.
%    \begin{macrocode}
\newif\if@captionabove\@captionabovefalse
%    \end{macrocode}  
% \end{macro}
%
% \begin{macro}{\if@tablecaptionabove}
% \changes{v2.8h}{2001/07/19}{neu Option}%^^A
% \changes{v2.8k}{2001/08/15}{aus \texttt{typearea} in die Klassen
%     verschoben} 
% \changes{v2.98c}{2008/02/18}{Definition erfolgt etwas früher}%^^A
% \begin{macro}{\@tablecaptionabovetrue}
% \begin{macro}{\@tablecaptionabovefalse}
% Die Einstellung, ob es Tabellenüber- oder -unterschriften sind, wird in
% einem Schalter gespeichert.
%    \begin{macrocode}
\newif\if@tablecaptionabove\@tablecaptionabovefalse
%    \end{macrocode}
% \end{macro}
% \end{macro}
% \end{macro}
%
% \begin{macro}{\if@figurecaptionabove}
% \changes{v3.09}{2011/03/23}{neu Option}%^^A
% \begin{macro}{\@figurecaptionabovetrue}
% \changes{v3.09}{2011/03/23}{neu (intern)}%^^A
% \begin{macro}{\@figurecaptionabovefalse}
% \changes{v3.09}{2011/03/23}{neu (intern)}%^^A
% Die Einstellung, ob es Abbildungsüber- oder -unterschriften sind, wird in
% einem Schalter gespeichert.
%    \begin{macrocode}
\newif\if@figurecaptionabove\@figurecaptionabovefalse
%    \end{macrocode}
% \end{macro}
% \end{macro}
% \end{macro}
%
% \selectlanguage{english}
% \begin{macro}{\ifonelinecaptions}
% \changes{v2.8q}{2001/11/17}{new boolean}
% \changes{v2.98c}{2008/03/03}{earlier definition}
% \begin{macro}{\onelinecaptionstrue}
% \begin{macro}{\onelinecaptionsfalse}
% The boolean is used to distinguish whether or not use a special handling of
% captions with one text line only. The name of the boolean is traditionally
% but nevertheless should seen as internal.
%    \begin{macrocode}
\newif\ifonelinecaptions\onelinecaptionstrue
%    \end{macrocode}
% \end{macro}
% \end{macro}
% \end{macro}
% \selectlanguage{ngerman}
%
% \begin{macro}{\scap@pos}
% \changes{v2.8q}{2001/11/22}{neu (intern)}%^^A
% \changes{v2.98c}{2008/03/03}{Definition erfolgt etwas früher}%^^A
% Hier wird gespeichert, ob die Beschreibung bei |captionbeside| links oder
% rechts vom Rest stehen soll. Voreingestellt ist rechts davon
% (\emph{\textbf{r}ight}). Andere Möglichkeiten sind \emph{\textbf{l}eft},
% \emph{\textbf{o}uter}, \emph{\textbf{i}nner}.
%    \begin{macrocode}
\newcommand*{\scap@pos}{r}
%    \end{macrocode}
% \end{macro}
%
% \begin{macro}{\scap@valign}
% \changes{v2.8q}{2001/11/22}{neu (intern)}%^^A
% \changes{v2.98c}{2008/03/03}{Definition erfolgt etwas früher}%^^A
% Hier wird gespeichert, ob bei |captionbeside| die beiden Boxen (siehe
% Definition der Umgebung) \emph{\textbf{t}op}, \emph{\textbf{b}ottem} oder
% \emph{\textbf{c}enter} ausgerichtet werden sollen. Andere Buchstaben
% sind nicht erlaubt. Voreingestellt ist \emph{\textbf{b}ottem}.
%    \begin{macrocode}
\newcommand*{\scap@valign}{b}
%    \end{macrocode}
% \end{macro}
%
% \subsection{Optionen für Titel von Gleitumgebungen}
%
% \begin{option}{captions}
% \changes{v2.98c}{2008/02/18}{Neuer Zentralschalter für caption}%^^A
% \changes{v3.09}{2011/04/02}{Neue Werte (\texttt{heading},
%     \texttt{signature}, \texttt{figureheading},
%     \texttt{figuresignature})}%^^A
% \changes{v3.17}{2015/03/11}{Beachtung der Signalisierung mit
%     \cs{FamilyKeyState}}%^^A
% \changes{v3.17}{2015/03/11}{Verwendung der internen Werteliste}%^^A
% Dies ist nun die zentrale Option für Einstellungen zu \cs{caption}. Als
% besondere Schwierigkeit ergibt sich dabei jedoch, dass das Paket
% \textsf{caption} einige Sonderbehandlungen für \KOMAScript{} enthält, die
% darauf angewiesen sind, dass die alte Optionenschnittstelle verwendet
% wird. Erfreulicherweise funktioniert das aber ab Version~3.1g über eine
% Erweiterung der Schalter-Umschaltbefehle (|\@tablecaptionabovetrue|,
% |\@tablecaptionabovefalse|, |\onelinecaptionstrue|,
% |\onelinecaptionsfalse|), so dass lediglich dafür georgt werden muss,
% dass diese weiterhin verwendet werden.
%    \begin{macrocode}
\KOMA@key{captions}{%
  \KOMA@set@ncmdkey{captions}{@tempa}{%
    {heading}{0},{above}{0},{top}{0},%
    {signature}{1},{below}{1},{bot}{1},{bottom}{1},%
    {tableheading}{2},%
    {tableabove}{2},{abovetable}{2},{abovetabular}{2},{topattable}{2},%
    {tablesignature}{3},%
    {tablebelow}{3},{belowtable}{3},{belowtabular}{3},{bottomattable}{3},%
    {figureheading}{4},{figureabove}{4},{abovefigure}{4},{topatfigure}{4},%
    {figuresignature}{5},{figurebelow}{5},{belowfigure}{5},{bottomatfigure}{5},%
    {oneline}{6},%
    {nooneline}{7},%
    {rightbeside}{8},{besideright}{8},%
    {leftbeside}{9},{besideleft}{9},%
    {innerbeside}{10},{besideinner}{10},%
    {outerbeside}{11},{besideouter}{11},%
    {topbeside}{12},{besidetop}{12},%
    {centeredbeside}{13},{besidecentered}{13},%
    {middlebeside}{13},{besidemiddle}{13},%
    {bottombeside}{14},{besidebottom}{14}%
  }{#1}%
  \ifx\FamilyKeyState\FamilyKeyStateProcessed
    \ifcase \@tempa\relax % heading
      \@tablecaptionabovetrue
      \@figurecaptionabovetrue
      \@captionabovetrue
      \KOMA@kav@remove{.\KOMAClassFileName}{captions}{heading}%
      \KOMA@kav@remove{.\KOMAClassFileName}{captions}{signature}%
      \KOMA@kav@add{.\KOMAClassFileName}{captions}{heading}%
   \or % signature
      \@tablecaptionabovefalse
      \@figurecaptionabovefalse
      \@captionabovefalse
      \KOMA@kav@remove{.\KOMAClassFileName}{captions}{heading}%
      \KOMA@kav@remove{.\KOMAClassFileName}{captions}{signature}%
      \KOMA@kav@add{.\KOMAClassFileName}{captions}{signature}%
    \or % tableheading
      \@tablecaptionabovetrue
      \KOMA@kav@remove{.\KOMAClassFileName}{captions}{tableheading}%
      \KOMA@kav@remove{.\KOMAClassFileName}{captions}{tablesignature}%
      \KOMA@kav@add{.\KOMAClassFileName}{captions}{tableheading}%
    \or % tablesignature
      \@tablecaptionabovefalse
      \KOMA@kav@remove{.\KOMAClassFileName}{captions}{tableheading}%
      \KOMA@kav@remove{.\KOMAClassFileName}{captions}{tablesignature}%
      \KOMA@kav@add{.\KOMAClassFileName}{captions}{tablesignature}%
    \or % figureheading
      \@figurecaptionabovetrue
      \KOMA@kav@remove{.\KOMAClassFileName}{captions}{figureheading}%
      \KOMA@kav@remove{.\KOMAClassFileName}{captions}{figuresignature}%
      \KOMA@kav@add{.\KOMAClassFileName}{captions}{figureheading}%
    \or % figuresignature
      \@figurecaptionabovefalse
      \KOMA@kav@remove{.\KOMAClassFileName}{captions}{figureheading}%
      \KOMA@kav@remove{.\KOMAClassFileName}{captions}{figuresignature}%
      \KOMA@kav@add{.\KOMAClassFileName}{captions}{figuresignature}%
    \or % oneline
      \onelinecaptionstrue
      \KOMA@kav@remove{.\KOMAClassFileName}{captions}{oneline}%
      \KOMA@kav@remove{.\KOMAClassFileName}{captions}{nooneline}%
      \KOMA@kav@add{.\KOMAClassFileName}{captions}{oneline}%
    \or % nooneline
      \onelinecaptionsfalse
      \KOMA@kav@remove{.\KOMAClassFileName}{captions}{oneline}%
      \KOMA@kav@remove{.\KOMAClassFileName}{captions}{nooneline}%
      \KOMA@kav@add{.\KOMAClassFileName}{captions}{nooneline}%
    \or % rightbeside
      \renewcommand*{\scap@pos}{r}%
      \KOMA@kav@remove{.\KOMAClassFileName}{captions}{rightbeside}%
      \KOMA@kav@remove{.\KOMAClassFileName}{captions}{leftbeside}%
      \KOMA@kav@remove{.\KOMAClassFileName}{captions}{innerbeside}%
      \KOMA@kav@remove{.\KOMAClassFileName}{captions}{outerbeside}%
      \KOMA@kav@add{.\KOMAClassFileName}{captions}{rightbeside}%
    \or % leftbeside
      \renewcommand*{\scap@pos}{l}%
      \KOMA@kav@remove{.\KOMAClassFileName}{captions}{rightbeside}%
      \KOMA@kav@remove{.\KOMAClassFileName}{captions}{leftbeside}%
      \KOMA@kav@remove{.\KOMAClassFileName}{captions}{innerbeside}%
      \KOMA@kav@remove{.\KOMAClassFileName}{captions}{outerbeside}%
      \KOMA@kav@add{.\KOMAClassFileName}{captions}{leftbeside}%
    \or % innerbeside
      \renewcommand*{\scap@pos}{i}%
      \KOMA@kav@remove{.\KOMAClassFileName}{captions}{rightbeside}%
      \KOMA@kav@remove{.\KOMAClassFileName}{captions}{leftbeside}%
      \KOMA@kav@remove{.\KOMAClassFileName}{captions}{innerbeside}%
      \KOMA@kav@remove{.\KOMAClassFileName}{captions}{outerbeside}%
      \KOMA@kav@add{.\KOMAClassFileName}{captions}{innerbeside}%
    \or % outerbeside
      \renewcommand*{\scap@pos}{o}%
      \KOMA@kav@remove{.\KOMAClassFileName}{captions}{rightbeside}%
      \KOMA@kav@remove{.\KOMAClassFileName}{captions}{leftbeside}%
      \KOMA@kav@remove{.\KOMAClassFileName}{captions}{innerbeside}%
      \KOMA@kav@remove{.\KOMAClassFileName}{captions}{outerbeside}%
      \KOMA@kav@add{.\KOMAClassFileName}{captions}{outerbeside}%
    \or % topbeside
      \renewcommand*{\scap@valign}{t}%
      \KOMA@kav@remove{.\KOMAClassFileName}{captions}{topbeside}%
      \KOMA@kav@remove{.\KOMAClassFileName}{captions}{centeredbeside}%
      \KOMA@kav@remove{.\KOMAClassFileName}{captions}{bottombeside}%
      \KOMA@kav@add{.\KOMAClassFileName}{captions}{topbeside}%
    \or % centeredbeside
      \renewcommand*{\scap@valign}{c}%
      \KOMA@kav@remove{.\KOMAClassFileName}{captions}{topbeside}%
      \KOMA@kav@remove{.\KOMAClassFileName}{captions}{centeredbeside}%
      \KOMA@kav@remove{.\KOMAClassFileName}{captions}{bottombeside}%
      \KOMA@kav@add{.\KOMAClassFileName}{captions}{centeredbeside}%
    \or % bottombeside
      \renewcommand*{\scap@valign}{b}%
      \KOMA@kav@remove{.\KOMAClassFileName}{captions}{topbeside}%
      \KOMA@kav@remove{.\KOMAClassFileName}{captions}{centeredbeside}%
      \KOMA@kav@remove{.\KOMAClassFileName}{captions}{bottombeside}%
      \KOMA@kav@add{.\KOMAClassFileName}{captions}{bottombeside}%
    \fi
  \fi
}
\KOMA@kav@add{.\KOMAClassFileName}{captions}{signature}
\KOMA@kav@add{.\KOMAClassFileName}{captions}{tablesignature}
\KOMA@kav@add{.\KOMAClassFileName}{captions}{figuresignature}
\KOMA@kav@add{.\KOMAClassFileName}{captions}{oneline}
\KOMA@kav@add{.\KOMAClassFileName}{captions}{rightbeside}
\KOMA@kav@add{.\KOMAClassFileName}{captions}{bottombeside}
%    \end{macrocode}
% \end{option}
%
% \begin{option}{tablecaptionabove}
% \changes{v2.8h}{2001/07/19}{neue Option}%^^A
% \changes{v2.8k}{2001/08/15}{aus \texttt{typearea} in die Klassen
%     verschoben} 
% \changes{v2.98c}{2008/02/18}{obsolete} 
% \changes{v3.01a}{2008/11/20}{deprecated}%^^A
% \begin{option}{tablecaptionbelow}
% \changes{v2.8h}{2001/07/19}{neue Option}%^^A
% \changes{v2.8k}{2001/08/15}{aus \texttt{typearea} in die Klassen
%     verschoben} 
% \changes{v2.98c}{2008/02/18}{obsolete} 
% \changes{v3.01a}{2008/11/20}{deprecated}%^^A
% Es besteht nun die Wahl, ob \cs{caption} bei Tabellen für Unter-
% oder Überschriften voreingestellt ist. Die Defaulteinstellung ist
% aus Kompatibilitätsgründen die Tabellenunterschrift.
%    \begin{macrocode}
\KOMA@DeclareDeprecatedOption{tablecaptionabove}{captions=tableheading}
\KOMA@DeclareDeprecatedOption{tablecaptionbelow}{captions=tablesignature}
%    \end{macrocode}
% \end{option}
% \end{option}
%
% \selectlanguage{english}
% \begin{option}{origlongtable}
% \changes{v2.8l}{2001/08/16}{new option}
% If package \textsf{longtable} is loaded, the classes redefine table titles
% of that package. However, if the option has been set, this is prevented.
% \begin{macro}{\if@komalongtable}
% \changes{v2.8l}{2001/08/16}{new boolean}
% \begin{macro}{\@komalongtabletrue}
% \begin{macro}{\@komalongtablefalse}
% The wanted behaviour is stored in a boolean.
%    \begin{macrocode}
\newif\if@komalongtable\@komalongtabletrue
\DeclareOption{origlongtable}{\@komalongtablefalse}
%    \end{macrocode}
% \end{macro}
% \end{macro}
% \end{macro}
% \end{option}
%
% \begin{option}{onelinecaption}
% \changes{v2.8q}{2001/11/17}{new option}
% \changes{v2.98c}{2008/03/03}{obsolete} 
% \changes{v2.98c}{2008/03/03}{extra code for \textsf{caption2} removed}
% \changes{v3.01a}{2008/11/20}{deprecated}
% \begin{option}{noonelinecaption}
% \changes{v2.8q}{2001/11/17}{new option}
% \changes{v2.98c}{2008/03/03}{obsolete} 
% \changes{v2.98c}{2008/03/03}{extra code for \textsf{caption2} removed}
% \changes{v3.01a}{2008/11/20}{deprecated}%^^A
%    \begin{macrocode}
\KOMA@DeclareDeprecatedOption{onelinecaption}{captions=oneline}
\KOMA@DeclareDeprecatedOption{noonelinecaption}{captions=nooneline}
%    \end{macrocode}
% \end{option}
% \end{option}
% \selectlanguage{ngerman}
%
%
% \iffalse
%</!letter>
%</class>
%</option>
%<*body>
% \fi
%
% \subsection{Randnotizen}
%
% Randnotizen sind ebenfalls eine Art von Gleitumgebung.
%
% \begin{macro}{\marginline}
% \KOMAScript{} stellt zusätzlich ein Randzeilenmakro zur Verfügung,
% das je nach gerader oder ungerader Seite in doppelseitigen Layout
% rechts- oder linksbündig gesetzt werden muss.
% \changes{v2.4l}{1997/02/06}{Definition mit Hilfe des optionalen
%      Parameters von \cs{marginpar} deutlich vereinfacht}
% \changes{v2.8q}{2001/11/27}{fehlende Klammern ergänzt}%^^A
%    \begin{macrocode}
\newcommand\marginline[1]{%
  \marginpar[\raggedleft{#1}]{\raggedright #1}%
}
%    \end{macrocode}
% \end{macro}
%
% \iffalse
%<*class>
%<*!letter>
% \fi
%
% \subsection{Definitionen für Titel von Gleitumgebungen}
%
% \begin{macro}{\caption}
% \changes{v2.2c}{1995/05/25}{Fließtitel auf CJK umgestellt}%^^A
% Es ist möglich, die Fließumgebungen mit einem Titel und einer
% Beschreibung zu versehen. Diese wird vom ggf. etwas abgesetzt.
% \begin{Length}{abovecaptionskip}
% \begin{Length}{belowcaptionskip}
% Um wieviel ist in zwei Längenregistern gespeichert:
%    \begin{macrocode}
\newlength{\abovecaptionskip}
\newlength{\belowcaptionskip}
\setlength{\abovecaptionskip}{10\p@}
\setlength{\belowcaptionskip}{0\p@}
%    \end{macrocode}
% \end{Length}
% \end{Length}
% \begin{macro}{\captionformat}
% \changes{v2.3c}{1995/08/06}{neu}%^^A
% Trennung von Label und Text:
%    \begin{macrocode}
\newcommand*{\captionformat}{:\ }
%    \end{macrocode}
% \end{macro}
% \begin{macro}{\scr@caption}
% \changes{v2.8h}{2001/07/19}{neu (intern)}%^^A
% \begin{macro}{\captionabove}
% \changes{v2.8h}{2001/07/19}{neu}%^^A
% \begin{macro}{\captionbelow}
% \changes{v2.8h}{2001/07/19}{neu}%^^A
% Damit überhaupt zwischen Unter- und Überschriften unterschieden
% werden kann, muss zunächst das Originalmakro gesichert werden. Dies
% geschieht erst zu Beginn des Dokuments, damit Pakete ggf. noch ihre
% eigenen Definitionen von \cs{caption} vornehmen können. Das
% \textsf{hyperref}-Paket ist so ein Kandidat.
%    \begin{macrocode}
\newcommand{\scr@caption}{}
\AtBeginDocument{\let\scr@caption\caption}
%    \end{macrocode}
% \end{macro}
% Jetzt ist es möglich durch einfaches Umlegen des Schalters für die
% korrekte Darstellung zu sorgen:
%    \begin{macrocode}
\newcommand{\captionabove}{\@captionabovetrue\scr@caption}
\newcommand{\captionbelow}{\@captionabovefalse\scr@caption}
%    \end{macrocode}
% \end{macro}
% \end{macro}
%
%
% \begin{environment}{captionbeside}
% \changes{v2.8q}{2001/11/22}{neu}%^^A
% Um einiges komplizierter als die Beschreibung unter oder über die
% Tabelle oder Abbildung zu setzen, ist der Versuch, sie daneben zu
% setzen. Dies funktioniert frundsätzlich auch nicht mit
% \texttt{longtable}-Tabellen.
%
% \begin{macro}{\scap@caption}
% \changes{v2.8q}{2001/11/22}{neu (intern)}%^^A
% Die Beschreibung, die der Umgebung als Parameter übergeben wird,
% muss zwischengespeichert werden. Hier wird lediglich das Makro
% dafür reserviert.
%    \begin{macrocode}
\newcommand*{\scap@caption}{}
%    \end{macrocode}
% \end{macro}
%
% \begin{macro}{\scap@width}
% \changes{v2.8q}{2001/11/22}{neu (intern)}%^^A
% Die Breite, die Tabelle bzw. Abbildung einschließlich Beschreibung
% und insgesamt haben dürften. Voreingestellt ist die gesamte
% verfügbare Breite.
%    \begin{macrocode}
\newcommand*{\scap@width}{\linewidth}
%    \end{macrocode}
% \end{macro}
%
% \begin{macro}{\scap@move}
% \changes{v2.8q}{2001/11/22}{neu (intern)}%^^A
% Der Wert, um den die Tabelle bzw. Abbildung einschließlich
% Beschreibung nach rechts bzw. außen verschoben werden. 
% Voreingestellt ist keine Verschiebung.
%    \begin{macrocode}
\newcommand*{\scap@move}{\z@}
%    \end{macrocode}
% \end{macro}
%
% \begin{macro}{\scap@currentlabel}
% \changes{v2.8q}{2001/11/24}{neu (intern)}%^^A
% Da \cs{caption} innerhalb einer Umgebung (\cs{parbox} und
% \texttt{captionbeside}) aufgerufen wird, würde ein \cs{label}, das
% zwar innerhalb der Gleitumgebung aber nach der
% \texttt{captionbeside}-Umgebung eingefügt wird, nicht die
% Gleitumgebung, sondern die umgebende Referenzebene
% referenzieren. Damit das nicht geschieht, wird in der
% \texttt{captionbeside}-Umgebung in der \cs{parbox} die aktuelle
% Referenz \cs{@currentlabel} in dem globalen Makro
% \cs{scap@currentlabel} zwischengespeichert und am Ende der
% \texttt{captionbeside}-Umgebung mit Hilfe von \cs{aftergroup} nach
% dem Ende der Umgebung wiederhergestellt. Da die
% \texttt{captionbeside}-Umgebung ohnehin nicht geschachtelt werden
% kann, funktioniert das problemlos. Für diese Aktion wird hier ein
% globales Makro reserviert.
%    \begin{macrocode}
\newcommand*{\scap@currentlabel}{}
%    \end{macrocode}
% \end{macro}
%
% \begin{macro}{\ifscap@alt}
% \changes{v2.8q}{2001/11/22}{neuer Schalter}%^^A
% \begin{macro}{\scap@alttrue}
% \begin{macro}{\scap@altfalse}
% Der Schalter, der angibt, ob die Verschiebung \cs{scap@move} im
% doppelseitigen Modus aus linken Seiten vom rechten Rand aus nach
% links erfolgen soll.
%    \begin{macrocode}
\newif\ifscap@alt
%    \end{macrocode}
% \end{macro}
% \end{macro}
% \end{macro}
%
% \begin{macro}{\@captionbeside}
% \changes{v2.8q}{2001/11/22}{neu (intern)}%^^A
% \changes{v3.20}{2016/04/12}{\cs{@ifnextchar} replaced by
%     \cs{kernel@ifnextchar}}%^^A
% \begin{macro}{\@@captionbeside}
% \changes{v2.8q}{2001/11/22}{neu (intern)}%^^A
% \changes{v3.20}{2016/04/12}{\cs{@ifnextchar} replaced by
%     \cs{kernel@ifnextchar}}%^^A
% \begin{macro}{\@@@captionbeside}
% \changes{v2.8q}{2001/11/22}{neu (intern)}%^^A
% \changes{v3.20}{2016/04/12}{\cs{@ifnextchar} replaced by
%     \cs{kernel@ifnextchar}}%^^A
% \begin{macro}{\@@@@captionbeside}
% \changes{v2.8q}{2001/11/22}{neu (intern)}%^^A
% \changes{v3.20}{2016/04/12}{\cs{@ifstar} durch \cs{kernel@ifstar}
%     ersetzt}%^^A
% Die Umgebung soll einmal folgende Parameter besitzen:
% \begin{description}
%  \item[(optional)] Verzeichniseintrag
%  \item[(obligatorisch)] Beschreibung
%  \item[(optional)] lokaler Wert für \cs{scap@pos}
%  \item[(optional)] lokaler Wert für \cs{scap@width}
%  \item[(optional)] lokaler Wert für \cs{scap@move}
%  \item[(*)] lokaler Wert für \cs{ifscap@alt} (Stern gesetzt
%  entspricht \cs{scap@alttrue}
% \end{description}
% Wie man sieht, sollte man bei Angabe aller optionalen Parameter auch
% bewusst den Stern setzen oder darauf verzichten.
%
% Damit alle Parameter ausgewertet werden können, werden einige
% Hilfsmakros benötigt.
%    \begin{macrocode}
\newcommand*{\@captionbeside}{}
\long\def\@captionbeside[#1]#2{%
  \kernel@ifnextchar [%]
    {\@@captionbeside[{#1}]{#2}}
    {\c@pti@nb@sid@[{#1}]{#2}}%
}
\newcommand*{\@@captionbeside}{}
\long\def\@@captionbeside[#1]#2[#3]{%
  \def\scap@pos{#3}%
  \kernel@ifnextchar [%]
    {\@@@captionbeside[{#1}]{#2}}
    {\c@pti@nb@sid@[{#1}]{#2}}%
}
\newcommand*{\@@@captionbeside}{}
\long\def\@@@captionbeside[#1]#2[#3]{%
  \def\scap@width{#3}%
  \kernel@ifnextchar [%]
    {\@@@@captionbeside[{#1}]{#2}}
    {\setlength{\@tempdima}{#3}%
      \addtolength{\@tempdima}{-\linewidth}%
      \setlength{\@tempdima}{-.5\@tempdima}%
      \edef\scap@move{\the\@tempdima}%
      \c@pti@nb@sid@[{#1}]{#2}}%
}
\newcommand*{\@@@@captionbeside}{}
\long\def\@@@@captionbeside[#1]#2[#3]{%
  \def\scap@move{#3}%
  \kernel@ifstar
    {\scap@alttrue\c@pti@nb@sid@[{#1}]{#2}}
    {\scap@altfalse\c@pti@nb@sid@[{#1}]{#2}}%
}
%    \end{macrocode}
% \end{macro}
% \end{macro}
% \end{macro}
% \end{macro}
%
% \begin{macro}{\c@pti@nb@sid@b@x}
% \changes{v2.8q}{2001/11/22}{neu (intern)}%^^A
% Es wird noch eines der kostbaren Boxregister benötigt.
%    \begin{macrocode}
\newbox\c@pti@nb@sid@b@x
%    \end{macrocode}
% \end{macro}
%
% \begin{macro}{\c@pti@nb@sid@}
% \changes{v2.8q}{2001/11/22}{neu (intern)}%^^A
% \changes{v2.98c}{2008/03/11}{vertikale Ausrichtung korrigiert}%^^A
% \changes{v3.07a}{2010/11/04}{vertikale Ausrichtung im t- und b-Fall erneut
%     korrigiert}%^^A
% Hier wird nun die Umgebung eigentlich geöffnet, während zuvor nur
% die Parameter ausgewertet wurden.
%    \begin{macrocode}
\newcommand{\c@pti@nb@sid@}{}
\long\def\c@pti@nb@sid@[#1]#2{%
  \def\scap@caption{%
    \if c\scap@valign
      \setlength{\abovecaptionskip}{\z@}%
    \else \if t\scap@valign
        \setlength{\abovecaptionskip}{-\ht\strutbox}%
      \else \if b\scap@valign
          \setlength{\abovecaptionskip}{\z@}%
        \fi
      \fi
    \fi
    \setlength{\belowcaptionskip}{-\dp\strutbox}%
    \captionbelow[{#1}]{#2}%
  }%
  \begin{lrbox}{\c@pti@nb@sid@b@x}%
}
%    \end{macrocode}
% \end{macro}
%
% Jetzt endlich können wir uns auf die Umgebung stürzen. Die
% Umgebung zu starten ist einfach. Es wird ein Doppelargumentmakro
% ausgeführt.
%    \begin{macrocode}
\newenvironment*{captionbeside}{%
  \@dblarg\@captionbeside%
}{%
%    \end{macrocode}
% Komplizierter ist die Beendigung der Umgebung. Hier fällt der
% gesamte Satzaufwand an.
% \changes{v2.9a}{2002/06/27}{\cs{ifthispageodd} statt
%      \cs{if@thispageodd}}
% \changes{v3.28}{2019/11/19}{\cs{ifthispageodd} durch \cs{Ifthispageodd}
%   ersetzt}%^^A
%    \begin{macrocode}
  \end{lrbox}%
  \settowidth{\@tempdima}{\usebox\c@pti@nb@sid@b@x}%
  \setlength{\@tempdimb}{\scap@width}%
  \addtolength{\@tempdimb}{-\@tempdima}%
  \addtolength{\@tempdimb}{-\columnsep}%
  \ifdim \@tempdimb<2em
    \ClassWarning{\KOMAClassName}{%
      caption beside will be very, very small.\MessageBreak
      In most cases this will result in overfull and\MessageBreak
      underfull \string\hbox\space messages%
    }%
  \fi
  \onelinecaptionsfalse
  \setcapwidth[r]{\@tempdimb}%
  \let\@tempa\scap@pos
  \setlength{\@tempdimc}{\scap@move}%
  \if i\scap@pos\def\@tempa{l}\fi
  \if o\scap@pos\def\@tempa{r}\fi
  \if@twoside
    \Ifthispageodd{}{%
      \if o\scap@pos\def\@tempa{l}\fi
      \if i\scap@pos\def\@tempa{r}\fi
      \ifscap@alt
        \addtolength{\@tempdimc}{\scap@width}%
        \addtolength{\@tempdimc}{-\linewidth}%
        \setlength{\@tempdimc}{-\@tempdimc}%
      \fi
    }%
  \fi
  \hspace*{\@tempdimc}%
%    \end{macrocode}
% \changes{v2.9p}{2003/12/10}{\cs{@temdima} replaced by
%      \cs{wd}\cs{c@pti@nb@sid@b@x}, because \cs{@temdima} would be changed}
%    \begin{macrocode}   
  \if l\@tempa
    \rlap{%
      \parbox[\scap@valign]{\@tempdimb}{\scap@caption
        \global\let\scap@currentlabel\@currentlabel}%
      \nobreak\hskip\columnsep
      \parbox[\scap@valign]{\wd\c@pti@nb@sid@b@x}{\usebox\c@pti@nb@sid@b@x}%
    }%
  \else
    \rlap{%
      \parbox[\scap@valign]{\wd\c@pti@nb@sid@b@x}{\usebox\c@pti@nb@sid@b@x}%
      \nobreak\hskip\columnsep
      \parbox[\scap@valign]{\@tempdimb}{\scap@caption
        \global\let\scap@currentlabel\@currentlabel}%
    }%
  \fi
  \par
  \aftergroup\let\aftergroup\@currentlabel
  \aftergroup\scap@currentlabel
}
%    \end{macrocode}
% \end{environment}
%
%
% \begin{macro}{\capfont}
% \changes{v2.8c}{2001/06/29}{\cs{normalcolor} eingefügt}
% Die Schriftart der Titel kann eingestellt werden. Dazu dient dieses
% Makro, das als internes Makro zu verstehen ist,
% \begin{macro}{\caplabelfont}
% \changes{v2.8c}{2001/06/29}{\cs{normalcolor} eingefügt}
% Davon unabhängig kann auch die Schriftart der Titel-Titel
% ("`Abbildung X:"') eingestellt werden. Auch dieses Makro ist als
% internes Makro zu verstehen.
%    \begin{macrocode}
\newcommand*\capfont{\normalfont\normalcolor}
\newcommand*\caplabelfont{\normalfont\normalcolor}
%    \end{macrocode}
% \end{macro}
% \end{macro}
%
% \begin{macro}{\scr@fnt@caption}
% \changes{v2.8o}{2001/09/14}{neues Element \texttt{caption}}
% \begin{macro}{\scr@fnt@captionlabel}
% \changes{v2.8o}{2001/09/14}{neues Element \texttt{captionlabel}}
% Die eigentliche Änderung der Schriftart erfolgt über das Element
% \texttt{caption} bzw. \texttt{captionlabel}.
%    \begin{macrocode}
\newcommand*{\scr@fnt@caption}{\capfont}
\newcommand*{\scr@fnt@captionlabel}{\caplabelfont}
%    \end{macrocode}
% \end{macro}
% \end{macro}
%
% \begin{macro}{\@makecaption}
% \changes{v2.3a}{1995/07/08}{\cs{setbox}\cs{tempboxa}\cs{hbox}
%      durch \cs{sbox}\cs{@tempboxa} ersetzt}
% \changes{v2.3a}{1995/07/08}{Minipage-Flag zurückgesetzt (ist das
%      ok?)} 
% \changes{v2.3f}{1995/12/09}{Fontumschaltung einheitlich für lange
%      und  kurze Unter-/Überschriften}%^^A
% \changes{v2.5g}{1999/10/09}{neben den komplett hängenden
%      Titeln sind nun auch nicht hängende oder teilweise hängende
%      möglich}%^^A
% Bis Version 2.5f sah die Definition wie folgt aus.
% \begin{verbatim}
% \long\def\@makecaption#1#2{%
%   \vskip\abovecaptionskip
%   \sbox\@tempboxa{{\caplabelfont #1\captionformat}{\capfont #2}}%
%   \ifdim \wd\@tempboxa >\hsize
%     \@hangfrom{{\caplabelfont #1\captionformat}}{\capfont #2\par}
%   \else
%     \global \@minipagefalse
%     \hb@xt@\hsize{\hfil\box\@tempboxa\hfil}%
%   \fi
%   \vskip\belowcaptionskip}
% \end{verbatim}
% \begin{macro}{\ifdofullc@p}
% Nun jedoch wird das ganze etwas komplizierter. Als erstes wird ein
% Schalter benötigt, der angibt, ob auf die volle Länge des "`Labels"'
% eingezogen werden soll, oder nicht:
%    \begin{macrocode}
\newif\ifdofullc@p
%    \end{macrocode}
% wobei die Standardeinstellung der volle Einzug ist, um zum bisherigen
% Verhalten kompatibel zu bleiben.
%    \begin{macrocode}
\dofullc@ptrue
%    \end{macrocode}
% \end{macro}
% \begin{macro}{\if@capbreak}
% \changes{v2.8j}{2001/07/30}{neu}%^^A
% Außerdem gibt es einen Schalter, der angibt, ob nach dem Label ein
% Umbruch erfolgen soll. Dies ist normalerweise nicht der Fall.
%    \begin{macrocode}
\newif\if@capbreak\@capbreakfalse
%    \end{macrocode}
% \end{macro}
% \begin{macro}{\setcapindent}
% \changes{v2.8j}{2001/07/30}{\cs{if@capbreak} wird verwendet}
% \changes{v2.8l}{2001/08/17}{\cs{edef}\cs{cap@indent} statt
%   \cs{def}\cs{cap@indent}}
% \changes{v2.8q}{2001/11/17}{wird nach dem Laden von
%   \textsf{caption2} umdefiniert}
% \changes{v3.20}{2016/04/12}{\cs{@ifstar} durch \cs{kernel@ifstar} ersetzt}
% \changes{v3.34}{2021/03/22}{Sonderbehandlung für \textsf{caption2} entfernt}
% Dann wird weiter ein Befehl benötigt, mit dem der Benutzer die
% Weite des Einzugs selbst einstellen kann. Da intern für die Weite
% ein Makro verwendet werden soll, findet hier sicherheitshalber eine
% Zuweisung an eine Dimension statt. So sollten Fehlermeldungen hier
% und nicht erst bei Verwendung auftauchen. Natürlich muss hier auch
% der automatische Einzug abgeschaltet werden. In der Sternform wird
% außerdem dafür gesorgt, dass zwischen "`Label"' und Beschreibung
% ein Umbruch erfolgt. In der normalen Version wird stattdessen für
% Kompatibilität zu früheren Versionen gesorgt.
% \begin{macro}{\@setcapindent}
% \begin{macro}{\@@setcaptindent}
% Für die Implementierung der Sternform werden zwei interne Makros
% benötigt.
%    \begin{macrocode}
\newcommand*{\setcapindent}{\kernel@ifstar{\@@setcapindent}{\@setcapindent}}
\newcommand*{\@setcapindent}[1]{%
  \setlength{\@tempdima}{#1}%
  \ifdim\@tempdima>\hsize
    \addtolength{\@tempdima}{-\hsize}\@capbreaktrue
  \else
    \@capbreakfalse
  \fi
  \edef\cap@indent{\the\@tempdima}%
  \dofullc@pfalse
}
\newcommand*{\@@setcapindent}[1]{%
  \setlength{\@tempdima}{#1}%
  \edef\cap@indent{\the\@tempdima}%
  \dofullc@pfalse\@capbreaktrue
}
%    \end{macrocode}
% \end{macro}
% \end{macro}
% \end{macro}
% \begin{macro}{\setcaphanging}
% \changes{v2.8q}{2001/11/17}{wird nach dem Laden von
%   \textsf{caption2} umdefiniert}
% \changes{v3.34}{2021/03/22}{Sonderbehandlung für \textsf{caption2} entfernt}
% Andererseits soll es auch möglich sein, nachträglich wieder den
% automatischen Einzug einzuschalten. Dafür gibt es ein weiteres
% Makro.
%    \begin{macrocode}
\newcommand*{\setcaphanging}{\dofullc@ptrue\@capbreakfalse}
%    \end{macrocode}
% \end{macro}
%
% Nun aber zur neuen Definition der Titel. Zunächst geht alles wie
% gehabt.
% \changes{v2.5g}{1999/10/09}{\cs{@makecaption} per \cs{newcommand}
%   definiert statt mit \cs{long}\cs{def}}
% \changes{v2.8h}{2001/07/19}{Unterscheidung zwischen Über- und
%   Unterschriften}%^^A
% \changes{v2.8j}{2001/07/30}{\cs{@makecaption} verwendet
%   \cs{@@makecaption}}
% \changes{v2.8q}{2001/11/17}{die Sonderbehandlung für einzeilige
%   \emph{caption}s ist abschaltbar}
% \changes{v2.8q}{2001/11/17}{Titel können in ihrer Breite
%   beschränkt werden}%^^A
%    \begin{macrocode}
\newcommand{\@makecaption}[2]{%
  \if@captionabove
    \vskip\belowcaptionskip
  \else
    \vskip\abovecaptionskip
  \fi
  \@@makecaption\@firstofone{#1}{#2}%
  \if@captionabove
    \vskip\abovecaptionskip
  \else
    \vskip\belowcaptionskip
  \fi
}
%    \end{macrocode}
% \begin{macro}{\@@makecaption}
% Tatsächlich gesetzt wird der Titel mit diesem Makro. Dabei muss der
% erste Parameter entweder \cs{@gobble} oder \cs{@firstofone} sein.
%
% \begin{macro}{\cap@width}
% \changes{v2.8q}{2001/11/17}{neu (intern)}%^^A
% \begin{macro}{\cap@left}
% \changes{v2.8q}{2001/11/17}{neu (intern)}%^^A
% \begin{macro}{\cap@right}
% \changes{v2.8q}{2001/11/17}{neu (intern)}%^^A
% Die Breite der mehrzeiligen bzw. die maximale Breite der einzeiligen
% \emph{caption}s sowie das Makro das links und rechts von der Box mit
% der mehrzeiligen \emph{caption} aufgerufen wird. In der
% Voreinstellung wird für Zentrierung gesorgt.
%    \begin{macrocode}
\newcommand*{\cap@width}{\hsize}
\newcommand*{\cap@left}{\hfill}
\newcommand*{\cap@right}{\hfill}
%    \end{macrocode}
% \end{macro}
% \end{macro}
% \end{macro}
% \begin{macro}{\setcapwidth}
% \changes{v2.8q}{2001/11/17}{neu}
% \changes{v2.9a}{2002/06/27}{\cs{ifthispageodd} statt \cs{if@thispageodd}}
% \changes{v2.97}{2007/02/27}{Mindestbreite für captions ist 0\,pt}
% \changes{v3.28}{2019/11/19}{\cs{ifthispageodd} durch \cs{Ifthispageodd}
%   ersetzt}
% \changes{v3.34}{2021/03/22}{Sonderbehandlung für \textsf{caption2} entfernt}
% Hier wird \cs{cap@width} gesetzt. Dabei werden optional auch
% \cs{cap@left} und \cs{cap@right} angepasst.
%    \begin{macrocode}
\newcommand*{\setcapwidth}[2][]{%
  \begingroup%
    \setlength{\@tempdima}{#2}%
    \ifdim\@tempdima <\z@
      \@tempdima\z@
      \ClassWarning{\KOMAClassName}{%
        caption width less than 0pt not allowed.\MessageBreak
        Setting caption width to 0pt%
      }%
    \fi
    \edef\@tempa{\endgroup\def\noexpand\cap@width{\the\@tempdima}}%
  \@tempa
  \begingroup
    \edef\@tempa{#1}%
    \def\@tempb{c}\ifx\@tempa\@tempb
      \endgroup
      \def\cap@margin{\let\cap@left=\hfill\let\cap@right=\hfill}%
    \else
      \def\@tempb{l}\ifx\@tempa\@tempb
        \endgroup
        \def\cap@margin{\let\cap@left=\relax\let\cap@right\hfill}%
      \else
        \def\@tempb{r}\ifx\@tempa\@tempb
          \endgroup
          \def\cap@margin{\let\cap@left=\hfill\let\cap@right=\relax}%
        \else
          \def\@tempb{i}\ifx\@tempa\@tempb
            \endgroup
            \def\cap@margin{%
              \let\cap@left=\relax
              \let\cap@right=\hfill
              \if@twoside
                \Ifthispageodd{}{%
                  \let\cap@left=\hfill
                  \let\cap@right=\relax
                }%
              \fi
            }%
          \else
            \def\@tempb{o}\ifx\@tempa\@tempb
              \endgroup
              \def\cap@margin{%
                \let\cap@left=\hfill
                \let\cap@right=\relax
                \if@twoside
                  \Ifthispageodd{}{%
                    \let\cap@left=\relax
                    \let\cap@right=\hfill
                  }%
                \fi
              }%
            \else
              \ifx \@tempa\@empty\else
                \ClassError{\KOMAClassName}{%
                  unknown optional position argument%
                }{%
                  You've used `#1' as position argument at
                  \string\setcaptionwidth,\MessageBreak
                  but only ONE of `c', `l', `r', `i`, `o` ist
                  allowed.\MessageBreak
                  Argument will be ignored, if you'll continue.%
                }%
              \fi
              \endgroup
  \fi\fi\fi\fi\fi
}
%    \end{macrocode}
% \end{macro}%^^A \setcapwidth
% \begin{macro}{\setcapdynwidth}
% \changes{v3.20}{2016/03/31}{neu}
% Ulrike findet es blöd, dass \cs{setcapwidth} wie \cs{setlength} arbeitet und
% ist der Meinung, dass das Argument erst zur Laufzeit ausgewertet werden
% sollte. Also ein neuer Befehl genau dafür. Es sei allerdings darauf
% hingewiesen, dass Fehler im Argument dann ggf. erst bei jeder
% \cs{caption}-Anweisung und nicht bereits hier ausgegeben werden!
%    \begin{macrocode}
\newcommand*{\setcapdynwidth}[2][]{%
  \def\cap@width{#2}%
  \begingroup
    \edef\@tempa{#1}%
    \def\@tempb{c}\ifx\@tempa\@tempb
      \endgroup
      \def\cap@margin{%
        \ifdim \cap@width<\z@
          \ClassWarning{\KOMAClassName}{%
            caption width less than 0pt not allowed.\MessageBreak
            Setting caption width to 0pt%
          }%
          \let\cap@width\z@
        \fi
        \let\cap@left=\hfill\let\cap@right=\hfill}%
    \else
      \def\@tempb{l}\ifx\@tempa\@tempb
        \endgroup
        \def\cap@margin{%
          \ifdim \cap@width<\z@
            \ClassWarning{\KOMAClassName}{%
              caption width less than 0pt not allowed.\MessageBreak
              Setting caption width to 0pt%
            }%
            \let\cap@width\z@
          \fi
          \let\cap@left=\relax\let\cap@right\hfill}%
      \else
        \def\@tempb{r}\ifx\@tempa\@tempb
          \endgroup
          \def\cap@margin{%
            \ifdim \cap@width<\z@
              \ClassWarning{\KOMAClassName}{%
                caption width less than 0pt not allowed.\MessageBreak
                Setting caption width to 0pt%
              }%
              \let\cap@width\z@
            \fi
            \let\cap@left=\hfill\let\cap@right=\relax}%
        \else
          \def\@tempb{i}\ifx\@tempa\@tempb
            \endgroup
            \def\cap@margin{%
              \ifdim \cap@width<\z@
                \ClassWarning{\KOMAClassName}{%
                  caption width less than 0pt not allowed.\MessageBreak
                  Setting caption width to 0pt%
                }%
                \let\cap@width\z@
              \fi
              \let\cap@left=\relax
              \let\cap@right=\hfill
              \if@twoside
                \Ifthispageodd{}{%
                  \let\cap@left=\hfill
                  \let\cap@right=\relax
                }%
              \fi
            }%
          \else
            \def\@tempb{o}\ifx\@tempa\@tempb
              \endgroup
              \def\cap@margin{%
                \ifdim \cap@width<\z@
                  \ClassWarning{\KOMAClassName}{%
                    caption width less than 0pt not allowed.\MessageBreak
                    Setting caption width to 0pt%
                  }%
                  \let\cap@width\z@
                \fi
                \let\cap@left=\hfill
                \let\cap@right=\relax
                \if@twoside
                  \Ifthispageodd{}{%
                    \let\cap@left=\relax
                    \let\cap@right=\hfill
                  }%
                \fi
              }%
            \else
              \ifx \@tempa\@empty\else
                \ClassError{\KOMAClassName}{%
                  unknown optional position argument%
                }{%
                  You've used `#1' as position argument at
                  \string\setcaptionwidth,\MessageBreak
                  but only ONE of `c', `l', `r', `i`, `o` ist
                  allowed.\MessageBreak
                  Argument will be ignored, if you'll continue.%
                }%
              \fi
              \endgroup
  \fi\fi\fi\fi\fi
}
%    \end{macrocode}
% \end{macro}%^^A \setcapdynwidth
% \begin{macro}{\cap@margin}
% \changes{v2.8q}{2001/11/17}{neu (intern)}%^^A
% Dieses Makro beinhaltet nicht den Wert der Ränder, sondern das
% Makro, mit dem innerhalb von \cs{@@makecaption} die Makros
% \cs{cap@left} und \cs{cap@right} neu definiert werden. Dies
% geschieht nur, wenn die Ränder mit \cs{setcapmargin} gesetzt
% werden. Vordefiniert ist also keinerlei Aktion.
%    \begin{macrocode}
\newcommand*{\cap@margin}{}
\let\cap@margin=\relax%
%    \end{macrocode}
% \end{macro}
% \begin{macro}{\setcapmargin}
% \changes{v2.8q}{2001/11/17}{neu}
% \changes{v3.20}{2016/04/12}{\cs{@ifstar} durch \cs{kernel@ifstar} ersetzt}
% \begin{macro}{\@setcapmargin}
% \changes{v2.8q}{2001/11/17}{neu (intern)}
% \changes{v2.9a}{2002/06/27}{\cs{ifthispageodd} statt
%   \cs{if@thispageodd}}
% \changes{v3.28}{2019/11/19}{\cs{ifthispageodd} durch \cs{Ifthispageodd}
%   ersetzt}
% \changes{v3.34}{2021/03/22}{Sonderbehandlung für \textsf{caption2} entfernt}
% \begin{macro}{\@@setcapmargin}
% \changes{v2.8q}{2001/11/17}{neu (intern)}%^^A
% Statt die Breite der \emph{caption} zu setzen, können auch die
% Ränder gesetzt werden. In der Sternvariante werden im zweiseitigen
% Modus die Ränder auf linken Seiten vertauscht.
%    \begin{macrocode}
\newcommand*{\setcapmargin}{\kernel@ifstar{\@setcapmargin}{\@@setcapmargin}}
\newcommand*{\@setcapmargin}[2][\@tempdimb]{%
  \def\cap@margin{%
    \onelinecaptionsfalse
    \begingroup
      \setlength{\@tempdima}{\hsize}%
      \setlength{\@tempdimb}{#2}%
      \addtolength{\@tempdima}{-\@tempdimb}%
      \setlength{\@tempdimc}{#1}%
      \addtolength{\@tempdima}{-\@tempdimc}%
      \edef\@tempa{\endgroup
        \def\noexpand\cap@width{\the\@tempdima}%
        \noexpand\if@twoside
          \noexpand\Ifthispageodd{%
            \def\noexpand\cap@left{\noexpand\hspace*{\the\@tempdimc}}%
            \def\noexpand\cap@right{\noexpand\hspace*{\the\@tempdimb}}%
          }{%
            \def\noexpand\cap@right{\noexpand\hspace*{\the\@tempdimc}}%
            \def\noexpand\cap@left{\noexpand\hspace*{\the\@tempdimb}}%
          }%
        \noexpand\else
          \def\noexpand\cap@left{\noexpand\hspace*{\the\@tempdimc}}%
          \def\noexpand\cap@right{\noexpand\hspace*{\the\@tempdimb}}%
        \noexpand\fi
      }%
    \@tempa
  }%
}
\newcommand*{\@@setcapmargin}[2][\@tempdimb]{%
  \def\cap@margin{%
    \onelinecaptionsfalse
    \begingroup
      \setlength{\@tempdima}{\hsize}%
      \setlength{\@tempdimb}{#2}%
      \addtolength{\@tempdima}{-\@tempdimb}%
      \setlength{\@tempdimc}{#1}%
      \addtolength{\@tempdima}{-\@tempdimc}%
      \edef\@tempa{\endgroup
        \def\noexpand\cap@width{\the\@tempdima}%
        \def\noexpand\cap@left{\noexpand\hspace*{\the\@tempdimc}}%
        \def\noexpand\cap@right{\noexpand\hspace*{\the\@tempdimb}}%
      }%
    \@tempa
  }%
}
%    \end{macrocode}
% \end{macro}
% \end{macro}
% \end{macro}
% Jetzt aber zum Eigentlichen. Die Breite der \emph{caption} muss nur
% bestimmte werden, wenn Option \texttt{onelinecaption} verwendet
% wird. Das bedeutet auch, dass bei \texttt{noonelinecaption} das
% Problem mit dem Hochzählen der Fußnoten nicht auftritt.
% \changes{v2.8q}{2002/05/17}{Problem mit doppelten Fußnoten behoben}%
% \changes{v2.95}{2004/01/30}{Korrekturen für vertikalen Abstand und
%   \textsl{float}-Paket}%
% \changes{v2.95a}{2006/07/13}{diverse Korrekturen für den vertikalen
%   Abstand u.\,a. bei Verwendung des \textsl{color}-Pakets}%
% \changes{v2.97c}{2007/06/06}{\cs{unskip}\cs{strut} eingefügt, um die
%   vertikale Ausrichtung im \emph{above}-Fall in eingen Fällen zu
%   verbessern}%
% \changes{v3.02a}{2009/01/24}{diverse Gruppen- bzw. Argumentklammern ergänzt}%
% \changes{v3.11c}{2013/06/12}{\cs{capfont} durch
%   \cs{usekomafont{caption}}ersetzt}%
% \changes{v3.11c}{2013/06/12}{\cs{caplabelfont} durch
%   \cs{usekomafont{captionlabel}}ersetzt}%
% \changes{v3.12}{2013/06/12}{ein \cs{usekomafont} durch
%   \cs{usesizeofkomafont} ersetzt}
% \changes{v3.21}{2016/05/31}{Neuimplementierung}
%    \begin{macrocode}
\newcommand{\@@makecaption}[3]{%
%    \end{macrocode}
% Als erstes werden die Makros für den linken und rechten Rand generiert.
%    \begin{macrocode}
  \cap@margin
%    \end{macrocode}
% Danach wird der Test vorgenommen, ob der Text einzeilig oder
% mehrzeilig zu setzen ist und entsprechend gehandelt:
% \changes{v3.22}{2016/09/01}{Umdefinierung von \cs{stepcounter} und
%   \cs{refstepcounter} auf lokale Erhöhung des Zählers geändert, um Probleme
%   mit \textsf{isodate} zu umgehen}%^^A
%    \begin{macrocode}
  \begingroup
    \ifonelinecaptions
      \def\stepcounter##1{\advance\value{##1}\@ne}%
      \let\refstepcounter\stepcounter
      \let\label\@gobble
      \let\@footnotetext=\@gobble
      \let\hypertarget\@gobbletwo
%    \end{macrocode}
% \changes{v3.22}{2016/09/01}{Sonderbehandlung für \string\\ u.\,ä. im Text}
% Falls im Text ein \texttt{\string\\} oder \cs{newline} enthalten ist,
% wird ein Infotext ausgegeben und dafür gesorgt, dass mehrzeilig gearbeitet
% wird.
%    \begin{macrocode}
      \edef\@xnewline{%
        \unexpanded{\ClassInfo{\KOMAClassName}{%
            Workaround for not recommended usage of \string\\ or\MessageBreak
            \string\newline in caption text
          }%
          \hspace*{\linewidth}\hspace{1sp}%
        }%
        \unexpanded\expandafter{\@xnewline}%
      }%
      \edef\@xcentercr{%
        \unexpanded{\ClassInfo{\KOMAClassName}{%
            Workaround for not recommended usage of \string\\ or\MessageBreak
            \string\newline in caption text
          }%
          \hspace*{\linewidth}\hspace{1sp}%
        }%
        \unexpanded\expandafter{\@xcentercr}%
      }%
      \settowidth{\@tempdima}{\scr@@makesinglelinecaption{#1}{#2}{#3}}%
%    \end{macrocode}
% \selectlanguage{english}
% \changes{v3.34}{2021/03/22}{fix of maximum line width of single line
%   captions}
% Until \KOMAScript~3.34 the test here was wrong, because it used
% \cs{linewidth}, which is still the whole line width not the maximum caption
% width. From \KOMAScript~3.34 \cs{cap@width} is used.
% \selectlanguage{ngerman}
%    \begin{macrocode}
      \ifdim\@tempdima>\cap@width
        \aftergroup\scr@makemultilinecaption
      \else
        \aftergroup\scr@makesinglelinecaption
      \fi
    \else
      \aftergroup\scr@makemultilinecaption
    \fi
  \endgroup
  {#1}{#2}{#3}%
}
%    \end{macrocode}
% \begin{macro}{\scr@makesinglelinecaption}
% \changes{v3.21}{2016/05/31}{neu (intern)}
% \begin{macro}{\scr@@makesinglelinecaption}
% \changes{v3.21}{2016/05/31}{neu (intern)}
% Diesen Anweisungen übernehmen die komplette Ausgabe von einzeiligen Texten.
% Das erste Argumen hat \cs{@gobble} oder \cs{@firstofone} zu sein und
% bestimmt darüber ob das Label ausgegeben wird oder nicht. Der Text wird
% dagegen immer ausgegeben! Das vierte Argument im internen Makro ist entweder
% \cs{par} oder leer.
%    \begin{macrocode}
\newcommand{\scr@makesinglelinecaption}[3]{%
  {%
    \usefontofkomafont{caption}{%
      \hb@xt@\hsize{%
        \cap@left
        \scr@@makesinglelinecaption{#1}{#2}{#3}%
        \cap@right
      }\par\kern\dp\strutbox%
    }%
  }\kern-\dp\strutbox
}
\newcommand{\scr@@makesinglelinecaption}[3]{%
    \usekomafont{caption}{\strut\ignorespaces
      #1{{\usekomafont{captionlabel}{#2\captionformat}}}%
      \ignorespaces #3\unskip}%
}
%    \end{macrocode}
% \end{macro}%^^A \scr@@makesinglelinecaption
% \end{macro}%^^A \scr@makesinglelinecaption
% \begin{macro}{\scr@makemultilinecaption}
% \changes{v3.21}{2016/05/31}{neu (intern)}%^^A
%    \begin{macrocode}
\newcommand{\scr@makemultilinecaption}[3]{%
  {%
    \usefontofkomafont{caption}{%
      \hb@xt@\hsize{%
        \cap@left
        \vbox{\hsize\cap@width\linewidth\hsize
%    \end{macrocode}
% \changes{v3.25}{2017/11/15}{caption Ausrichtung hinzugefügt}%^^A
% Bevor wir irgend etwas anderes machen, führen wir an der Stelle die
% Ausrichtung für die gesamte \emph{caption} einschließlich Label aus. Dabei
% hat eine vorhandene spezielle Ausrichtung für den Gleitungebungstyp Vorrang
% vor der allgemeinen Ausrichtung für alle Gleitumgebungen:
%    \begin{macrocode}
          \scr@ifundefinedorrelax{@captype}{%
            \ClassWarning{\KOMAClassName}{%
              `\string\caption' outside float.\MessageBreak
              Seems you are using `\string\caption' outside a
              float.\MessageBreak
              Maybe you are using a package that uses
              `\string\@makecaption'\MessageBreak
              without setting `\string\@captype' before.\MessageBreak
              Because I cannot detect the caption type, I'm using\MessageBreak
              the empty one.%
            }%
            \let\@captype\@empty
          }{}%
          \scr@ifundefinedorrelax{ragged\@captype caption}{\raggedcaption}{%
            \@nameuse{ragged\@captype caption}%
          }%
%    \end{macrocode}
% Wenn der Schalter \cs{ifdofullc@p} gesetzt ist, wird der Text bezüglich des
% Labels vollständig hängend gesetzt:
% \changes{v3.25}{2017/11/15}{Verwendung von \cs{scr@endstrut} statt
%   \cs{unskip}\cs{strut}}%^^A
%    \begin{macrocode}
          \ifdofullc@p
            \@hangfrom{%
              \strut
              \scr@smashdp{%
                #1{%
                  \usekomafont{caption}{%
                    \usekomafont{captionlabel}{#2\captionformat}%
                  }%
                }%
              }%
            }%
            {\usekomafont{caption}{#3\scr@endstrut\par}}%
%    \end{macrocode}
% Anderenfalls soll entweder nach dem Label direkt ein Umbruch erfolgen und
% ein Einzug nur der ersten Zeile zu setzen:
%    \begin{macrocode}
          \else
            \ifdim\cap@indent<\z@
              \strut
              #1{%
                \usekomafont{caption}{%
                  \usekomafont{captionlabel}{#2\captionformat\par}%
                }%
              }%
              \noindent{\usekomafont{caption}{%
                  \hspace*{-\cap@indent}#3\scr@endstrut\par
              }}%
%    \end{macrocode}
% Oder es soll nach dem Label ein Umbruch erfolgen und danach alle Zeilen mit
% dem entsprechenden Einzug gesetzt werden:
%    \begin{macrocode}
            \else
              \if@capbreak
                \strut
                #1{%
                  \usekomafont{caption}{%
                    \usekomafont{captionlabel}{#2\captionformat\par}%
                  }%
                }%
                \noindent\@hangfrom{\hspace*{\cap@indent}}%
                {\usekomafont{caption}{#3\scr@endstrut\par}}%
%    \end{macrocode}
% \changes{v3.27a}{2019/10/21}{fehlende Verwendung von Element
%   \texttt{caption} ergänzt}%^^A
% Oder es soll gar kein Umbruch erfolgen aber trotzdem alles mit einem
% gewissen Einzug gesetzt werden:
%    \begin{macrocode}
              \else
                \@hangfrom{\scr@smashdp\strut\hspace*{\cap@indent}}%
                {%
                  \hspace{-\cap@indent}%
                  \scr@smashdp{%
                    #1{\usekomafont{caption}{%
                        \usekomafont{captionlabel}{#2\captionformat}%
                      }%
                    }%
                  }%
                  {\usekomafont{caption}{#3\scr@endstrut}\par}%
                }%
              \fi
            \fi
          \fi
%    \end{macrocode}
% Eigentlich fehlt hier nur die einfachste, gar nicht hängende Form:
%    \begin{macrocode}
%<*neverusethis>
%          #1{{\usekomafont{caption}{%
%              \usekomafont{captionlabel}{#2\captionformat}}}}%
%          {\usekomafont{capion}{\ignorespaces #3\scr@endstrut\par}%
%</neverusethis>
%    \end{macrocode}
% Die allerdings durch die Form mit Einzug 0 abgedeckt wird.
%    \begin{macrocode}
        }%
        \cap@right
      }\par\kern\dp\strutbox
    }%
  }\kern-\dp\strutbox
}
%    \end{macrocode}
% \begin{macro}{\setcaptionalignment}
% \changes{v3.25}{2017/11/15}{neu}%^^A
% \changes{v3.28}{2019/11/18}{\cs{ifstr} umbenannt in \cs{Ifstr}}%^^A
% Da es etwas lästig und fehleranfällig ist, die ganzen \cs{raggedcaption}
% und \cs{ragged\dots caption} von Hand zu definieren, gibt es dafür einen
% Befehl. Das erste, optionale Argument ist dabei die Gleitungebung für die
% die Definition erfolgen soll. Ist das Argument nicht leer, wird bei
% definiertem \cs{@captype} dieses als Gleitumgebung angenommen. Ist das
% Argument leer wird stattdessen die allgemeine Ausrichtung festgelegt. Das
% zweite Argument ist ein einzelner Buchstaben, der für die Ausrichtung steht.
%    \begin{macrocode}
\newcommand*\setcaptionalignment{%
  \kernel@ifnextchar [%]
    {\@setcaptionalignment}
    {%
      \scr@ifundefinedorrelax{@captype}{\@setcaptionalignment}{%
        \edef\reserved@a{\noexpand\@setcaptionalignment[{\@captype}]}\reserved@a
      }%
    }%
}%
\newcommand*\@setcaptionalignment[2][]{%
%    \end{macrocode}
% Wenn das optionale Argument nicht leer ist, sollte es eine bereits
% definierte Gleitumgebung sein. Anderenfalls wird vorsichtshalber zumindest
% eine Warnung ausgegeben. Die Ausrichtung wird aber trotzde, definiert.
%    \begin{macrocode}
  \IfArgIsEmpty{#1}{}{%
    \expandafter\ifnum 
    \scr@ifundefinedorrelax{#1}{0}{\scr@ifundefinedorrelax{end#1}{0}{1}}=0
      \ClassWarning{\KOMAClassName}{%
        float type `#1' defined not yet,\MessageBreak
        but setting ragged mode for captions of\MessageBreak
        float type `#1'%
      }%
    \fi
  }%
%    \end{macrocode}
% Danach gehen wir die Buchstaben durch. Erlaubt sind:
% \begin{description}
% \item[\texttt{j}] -- Blocksatz (\cs{ragged\dots caption} wird \cs{relax}.
% \item[\texttt{l}] -- Blocksatz (\cs{ragged\dots caption} wird
%                      \cs{raggedright}.
% \item[\texttt{r}] -- Blocksatz (\cs{ragged\dots caption} wird
%                      \cs{raggedleft}.
% \item[\texttt{c}] -- Blocksatz (\cs{ragged\dots caption} wird
%                      \cs{centering}.
% \end{description}
% Ist das Paket \textsf{ragged2e} geladen, werden stattdessen die Befehle
% \cs{LaTeXraggedright}, \cs{LaTeXraggedleft} und \cs{LaTeXcentering}
% verwendet. Außerdem führen dann die entsprechenden Großbuchstaben zu
% \cs{justifying}, \cs{RaggedRight}, \cs{RaggedLeft} und
% \cs{Centering}. Anderenfalls werden die Großbuchstaben wie die
% Kleinbuchstaben behandelt und geben zusätzlich eine Warnung aus.
%    \begin{macrocode}
  \def\reserved@a{#2}%
  \Ifstr{\reserved@a}{j}{%
    \expandafter\let\csname ragged#1caption\endcsname\@empty
  }{%
    \Ifstr{\reserved@a}{l}{%
      \scr@ifundefinedorrelax{LaTeXraggedright}{%
        \expandafter\let\csname ragged#1caption\endcsname\raggedright
      }{%
        \expandafter\let\csname ragged#1caption\endcsname\LaTeXraggedright
      }%
    }{%
      \Ifstr{\reserved@a}{r}{%
        \scr@ifundefinedorrelax{LaTeXraggedleft}{%
          \expandafter\let\csname ragged#1caption\endcsname\raggedleft
        }{%
          \expandafter\let\csname ragged#1caption\endcsname\LaTeXraggedleft
        }%
      }{%
        \Ifstr{\reserved@a}{c}{%
          \scr@ifundefinedorrelax{LaTeXcentering}{%
            \expandafter\let\csname ragged#1caption\endcsname\centering
          }{%
            \expandafter\let\csname ragged#1caption\endcsname\LaTeXcentering
          }%
        }{%
          \Ifstr{\reserved@a}{J}{%
            \scr@ifundefinedorrelax{justifying}{%
              \scr@caption@raggede@warning{J}{\justifying}%
              \expandafter\let\csname ragged#1caption\endcsname\@empty
            }{%
              \expandafter\let\csname ragged#1caption\endcsname\justifying
            }%
          }{%
            \Ifstr{\reserved@a}{L}{%
              \scr@ifundefinedorrelax{RaggedRight}{%
                \scr@caption@raggede@warning{L}{\RaggedRight}%
                \expandafter\let\csname ragged#1caption\endcsname\raggedright
              }{%
                \expandafter\let\csname ragged#1caption\endcsname\RaggedRight
              }%
            }{%
              \Ifstr{\reserved@a}{R}{%
                \scr@ifundefinedorrelax{RaggedLeft}{%
                  \scr@caption@raggede@warning{R}{\RaggedLeft}%
                  \expandafter\let\csname ragged#1caption\endcsname\raggedleft
                }{%
                  \expandafter\let\csname ragged#1caption\endcsname\RaggedLeft
                }%
              }{%
                \Ifstr{\reserved@a}{C}{%
                  \scr@ifundefinedorrelax{Centering}{%
                    \scr@caption@raggede@warning{C}{\Centering}%
                    \expandafter\let\csname ragged#1caption\endcsname\centering
                  }{%
                    \expandafter\let\csname ragged#1caption\endcsname\Centering
                  }%
                }{%
                  \Ifstr{\reserved@a}{}{%
                    \expandafter\let\csname ragged#1caption\endcsname\relax
                  }{%
                    \ClassError{\KOMAClassName}{unknown alignment char `#2'}{%
                      \string\setcaptionalignment\space supports
                      only:\MessageBreak 
                      \space\space c - centered captions,\MessageBreak
                      \space\space j - justified captions,\MessageBreak
                      \space\space l - left align captions,\MessageBreak
                      \space\space r - right aligned captions,\MessageBreak
                      \space\space C - centered captions with
                      `ragged2e',\MessageBreak 
                      \space\space J - justified captions with
                      `ragged2e',\MessageBreak 
                      \space\space L - left align captions with
                      `ragged2e',\MessageBreak 
                      \space\space R - right aligned captions with
                      `ragged2e',\MessageBreak 
                      or an empty argument to remove special
                      alignment.\MessageBreak 
                      If you'd continue, I'd not change the alignment.%
                    }%
                  }%
                }%
              }%
            }%
          }%
        }%
      }%
    }%
  }%
}
%    \end{macrocode}
% \begin{macro}{\scr@caption@raggede@warning}
% \changes{v3.25}{2017/11/15}{neu (intern)}%^^A
% Das ist die oben erwähnte Warnung.
%    \begin{macrocode}
\newcommand*{\scr@caption@raggede@warning}[2]{%
  \begingroup
    \lowercase{\def\reserved@a{#1}}%
    \ClassWarning{\KOMAClassName}{%
      `\string#2' undefined!\MessageBreak
      To avoid replacing caption alignment `#1' by `\reserved@a'\MessageBreak
      you should load package `ragged2e' before using\MessageBreak
      `\string\setcaptionalignment'%
    }%
  \endgroup
}
%    \end{macrocode}
% \begin{macro}{\raggedcaption}
% \changes{v3.25}{2017/11/15}{neu}%^^A
% Und natürlich wird eine Standardausrichtung benötigt. Ich bevorzuge hier die
% leere Ausrichtung, weil \LaTeX{} \cs{relax} nicht als definiert erkennt.
%    \begin{macrocode}
\newcommand*\raggedcaption{}
%    \end{macrocode}
% \end{macro}
% \end{macro}%^^A \scr@caption@raggede@warning
% \end{macro}%^^A \setcaptionalignment
% \end{macro}%^^A \scr@makemultilinecaption
% \end{macro}%^^A \@@makecaption
% \end{macro}%^^A \@makecaption
% \end{macro}%^^A \caption
%
% \begin{macro}{\LT@makecaption}
% \changes{v2.8j}{2001/07/30}{neu (intern)}
% \changes{v2.8k}{2001/08/15}{keine Umdefinierung, wenn das
%   \textsf{caption2}-Paket erkannt oder die Option
%   \texttt{origlongtable} gesetzt wurde}
% \changes{v2.8q}{2001/11/18}{Breite und Anordnung wird über
%   \KOMAScript-Makros bestimmt}
% \changes{v2.8q}{2001/11/18}{Verwendung von \cs{AfterPackage}}
% \changes{v2.9s}{2004/08/24}{vertikale Abstände durch verschieben von
%   \cs{vskip} korrigiert}
% \begin{macro}{\LT@array}
% \changes{v2.8q}{2001/11/18}{neu (intern)}
% \begin{macro}{\LT@caption}
% \changes{v2.8q}{2001/11/18}{neu (intern)}
% \begin{macro}{\scr@caption@before@longtable@warning}
% \changes{v2.98c}{2008/03/03}{neu (intern)}
% \changes{v3.34}{2021/03/22}{Sonderbehandlung für \textsf{caption2} entfernt}
% Ab Version 2.8j wird auf vielfachen Wunsch das
% \textsf{longtable}-Paket berücksichtigt. Da Pakete aber nach
% Klassen geladen werden, kann dies erst nach dem Laden des Pakets
% aktiviert werden. Die Definition entspricht der obigen.
%    \begin{macrocode}
\newcommand*{\scr@caption@before@longtable@warning}[1]{%
  \ClassWarning{\KOMAClassName}{%
    you've loaded package `#1' before package\MessageBreak
    `longtable'!\MessageBreak
    Because of this package `#1' isn't able to\MessageBreak
    control `\string\caption' at longtables. \space After loading
    \MessageBreak 
    package `#1' KOMA-Script also doesn't control\MessageBreak
    `\string\caption'.\MessageBreak
    I think, you should load package `caption' after\MessageBreak
    package `longtable'%
  }%
}
\AfterPackage{longtable}{%
  \if@komalongtable
    \renewcommand{\LT@makecaption}[3]{%
      \noalign{%
        \if@captionabove
          \vskip\belowcaptionskip
        \else
          \vskip\abovecaptionskip
        \fi
      }%
      \LT@mcol\LT@cols c{%
        \hbox to\z@{\hss\parbox[t]\linewidth{%
%    \end{macrocode}
% \changes{v3.25}{2017/11/15}{\cs{@captype} auf \texttt{table} gesetzt}%^^A
% Damit \cs{raggedtablecaption} ggf. verwendet wird, wird hier \cs{@captype}
% explizit auf \texttt{table} gesetzt.
%    \begin{macrocode}
            \def\@captype{table}%
            \@@makecaption{#1}{#2}{#3}%
            \endgraf
            \if@captionabove
              \vskip\abovecaptionskip
            \else
              \vskip\belowcaptionskip
            \fi
          }%
          \hss
        }%
      }%
    }%
    \newcommand*{\scr@LT@array}{}%
    \let\scr@LT@array=\LT@array
    \renewcommand*{\LT@array}{%
      \let\scr@caption=\LT@caption
      \def\captionabove{\noalign{\global\@captionabovetrue}%
        \scr@caption}%
      \def\captionbelow{\noalign{\global\@captionabovefalse}%
        \scr@caption}%
      \if@tablecaptionabove
        \let\LT@caption=\captionabove
      \else
        \let\LT@caption=\captionbelow
      \fi
      \scr@LT@array
    }%
    \ClassInfo{\KOMAClassName}{%
      longtable captions redefined%
    }%
  \fi
}
%    \end{macrocode}  
% \end{macro}
% \end{macro}
% \end{macro}
% \end{macro}
%
% \begin{macro}{\fs@komaabove}
% \changes{v2.8j}{2001/07/30}{neu (intern)}%^^A
% \begin{macro}{\fs@komabelow}
% \changes{v2.8j}{2001/07/30}{neu (intern)}%^^A
% \begin{macro}{\floatc@komascript}
% \changes{v2.8j}{2001/07/30}{neu (intern)}%^^A
% Ab Version 2.8j werden für das \textsf{float}-Paket zwei float-Stile
% Namens \texttt{komaabove} und \texttt{komabelow} definiert. Da
% möglicherweise auch andere Pakete darauf zurückgreifen, wird das
% unabhängig von der Existenz von \textsf{float} getan. Verwendet wird
% dies innerhalb von \KOMAScript{} jedoch nicht. Es werden außerdem
% interne Makros von \textsf{float} verwendet!
%    \begin{macrocode}
\newcommand*{\fs@komaabove}{%
  \let\@fs@capt\floatc@komascript
  \let\@fs@pre\@empty
  \def\@fs@mid{\vskip\abovecaptionskip}%
  \let\@fs@post\@empty
  \let\@fs@iftopcapt\iftrue}
\newcommand*{\fs@komabelow}{%
  \let\@fs@capt\floatc@komascript
  \let\@fs@pre\@empty
  \def\@fs@mid{\vskip\abovecaptionskip}%
  \let\@fs@post\@empty
  \let\@fs@iftopcapt\iffalse}
\newcommand{\floatc@komascript}{\@@makecaption\@firstofone}
%    \end{macrocode}
% \end{macro}
% \end{macro}
% \end{macro}
%
%
% \subsection{Verteilung der Gleitumgebungen}
%
% \begin{Counter}{topnumber}
% \begin{macro}{\topfraction}
% \begin{Counter}{bottomnumber}
% \begin{macro}{\bottomfraction}
% \begin{Counter}{totalnumber}
% \begin{macro}{\textfraction}
% \begin{macro}{\floatpagefraction}
% \begin{Counter}{dbltopnumber}
% \begin{macro}{\dbltopfraction}
% \begin{macro}{\dblfloatpagefraction}
% Für die Fließumgebungen müssen verschiedene Konstanten definiert
% werden, die Aufteilung zwischen Text und Fließumgebung und Ähnliches
% regeln.
%    \begin{macrocode}
\setcounter{topnumber}{2}
\renewcommand*\topfraction{.7}
\setcounter{bottomnumber}{1}
\renewcommand*\bottomfraction{.3}
\setcounter{totalnumber}{3}
\renewcommand*\textfraction{.2}
\renewcommand*\floatpagefraction{.5}
\setcounter{dbltopnumber}{2}
\renewcommand*\dbltopfraction{.7}
\renewcommand*\dblfloatpagefraction{.5}
%    \end{macrocode}
% \end{macro}
% \end{macro}
% \end{Counter}
% \end{macro}
% \end{macro}
% \end{Counter}
% \end{macro}
% \end{Counter}
% \end{macro}
% \end{Counter}
%
%
% \subsection{Titel ohne Gleitumgebung}
%
% \begin{macro}{\captionof}
% \changes{v3.05}{2009/07/08}{Neue Anweisung}%^^A
% \changes{v3.09}{2011/03/21}{Verwendung des internen \cs{scr@caption}}
% \changes{v3.09a}{2011/04/13}{Berücksichtigung von
%     \texttt{captions=tableheading} und \texttt{captions=figureheading}}
% \begin{macro}{\captionaboveof}
% \changes{v3.09}{2011/03/21}{Neue Anweisung}%^^A
% \begin{macro}{\captionbelowof}
% \changes{v3.09}{2011/03/21}{Neue Anweisung}%^^A
% Obwohl es inzwischen mehrere Pakete gibt, die diese Anweisung bereit
% stellen, kommt es doch immer wieder zu Fehlern, weil einige Anwender diese
% Pakete nicht nutzen. Also habe ich beschlossen, das selbst zu
% implementieren. Um die Kompatibilität mit \textsf{capt-of} nicht zu stören,
% wird die Definition vor dme Laden jenes Pakets zurück genommen. Mit dem
% Paket \textsf{caption} gibt es solche Probleme nicht, da dieses
% \cs{captionof} ohnehin per \cs{def} definiert. Allerdings berücksichtigt das
% \cs{captionof} von \textsf{caption} und \textsf{capt-of} eventuell die
% Einstellungen für darüber oder darunter nicht in gleicher Weise.
%    \begin{macrocode}
\newcommand*{\captionof}[1]{\def\@captype{#1}%
  \begingroup\expandafter\expandafter\expandafter\endgroup
  \expandafter\ifx\csname if@#1captionabove\endcsname\iftrue
    \@captionabovetrue
  \fi
  \scr@caption}
\newcommand*{\captionaboveof}{\@captionabovetrue\captionof}
\newcommand*{\captionbelowof}{\@captionabovefalse\captionof}
\BeforePackage{capt-of}{\let\captionof\relax}
%    \end{macrocode}
% \end{macro}
% \end{macro}
% \end{macro}
%
% \begin{environment}{captionofbeside}
% \changes{v3.10}{2011/09/02}{Neue Umgebung}%^^A
%   Analog zu \texttt{captionbeside} hier eine Umgebung mit Angabe des
%   Gleitumgebungstyps als erstes Argument.
%    \begin{macrocode}
\newenvironment{captionofbeside}[1]{%
  \def\@captype{#1}%
  \captionbeside
}{%
  \endcaptionbeside
}
%    \end{macrocode}
% \end{environment}
%
%
% \subsection{Standardbezeichner}
%
% \begin{macro}{\figurename}
% \begin{macro}{\tablename}
% So heißen die Dinger:
%    \begin{macrocode}
\newcommand*\figurename{Figure}
\newcommand*\tablename{Table}
%    \end{macrocode}
% \end{macro}
% \end{macro}
%
%
% \subsection{Abbildungsumgebung}
%
% \begin{environment}{figure}
% \changes{v2.2c}{1995/05/25}{Abbildungsausgabe auf CJK umgestellt}%^^A
% \changes{v2.8h}{2001/07/19}{\cs{caption} wird explizit zu
%     \cs{captionbelow}}%^^A
% \changes{v3.25}{2017/10/10}{Reimplementierung mit \textsf{tocbasic}}%^^A
% Zunächst werden \texttt{figure}s, also Abbildungen behandelt.
% \begin{Counter}{figure}
% \changes{v3.25}{2017/10/10}{Der Zähler wird implizit von
%     \cs{DeclareNewTOC} definiert.}%^^A
% \end{Counter}
% Der ganze Rest kann mit \cs{DeclareNewTOC} erledigt werden.
%    \begin{macrocode}
\DeclareNewTOC[%
  type=figure,
  float,
%<book|report>  counterwithin=chapter,
%    \end{macrocode}
% \begin{macro}{\fps@figure}
% Die Standardplatzierung für Abbildungen ist \emph{t}op,
% \emph{b}ottom, \emph{p}age.
%    \begin{macrocode}
  floatpos=tbp,
% \end{macro}
% \begin{macro}{\ftype@figure}
% Um die verschiedenen \emph{float}-Typen unterscheiden zu können,
% benötigen sie eine Nummer. Abbildungen bekommen hier willkürlich
% die Nummer 1.
%    \begin{macrocode}
  floattype=1,
%    \end{macrocode}
% \end{macro}
% \changes{v3.11c}{2013/03/01}{fix: one \cs{caption} change removed}%^^A
% \changes{v3.20}{2016/04/07}{Interne Umdefinierung von \cs{caption}}%^^A
%    \begin{macrocode}
  tocentryindent=1.5em,
  tocentrynumwidth=2.3em,
  atbegin=\def\caption{%
    \if@figurecaptionabove\expandafter\captionabove
    \else\expandafter\captionbelow\fi
  }
%    \end{macrocode}
% \begin{macro}{\ext@figure}
% Die Extension der Datei für das Abbildungsverzeichnis.
%    \begin{macrocode}
]{lof}
%    \end{macrocode}
% \end{macro}
% \begin{macro}{\thefigure}
% \changes{v2.97e}{2007/11/23}{Kapitelnummer nur im Hauptteil}
% \changes{v3.27}{2019/06/26}{\cs{relax} replaced by percent}
% \changes{v3.25}{2021/10/12}{use chapter number only of greater than zero}
% \changes{v3.25}{2021/10/12}{redefine \cs{thefigure} only if compatibility is
%   less or equal 2.97d}
%    \begin{macrocode}
%<*book|report>
\expandafter\ifnum\scr@v@is@le{2.97d}%
  \renewcommand*\thefigure{\thechapter.\@arabic\c@figure}
\fi
%</book|report>
%    \end{macrocode}
% \end{macro}
% \begin{macro}{\fnum@figure}
% \begin{macro}{\figureformat}
% \changes{v2.3c}{1995/08/06}{neu}%^^A
% Die Ausgabe der Abbildungsnummer wird implizit von \cs{DeclareNewTOC} als
% \iffalse
%<*ignore>
% \fi
%    \begin{macrocode}
\newcommand*{\fnum@figure}{\figureformat}
\newcommand*{\figureformat}{%
  \figurename~\thefigure
  \csname autodot\endcsname
}
%    \end{macrocode}
% \iffalse
%</ignore>
% \fi
% definiert.
% \end{macro}
% \end{macro}
% \end{environment}
%
%
% \subsection{Tabellengleitumgebung}
%
% \begin{environment}{table}
% \changes{v2.2c}{1995/05/25}{Tabellenausgabe auf CJK umgestellt}%^^A
% \changes{v2.8h}{2001/07/19}{bei Tabellen kann der Default für
%      \cs{caption} von Unter- auf Überschriften geändert werden} 
% \changes{v3.25}{2017/10/10}{Reimplementierung mit \textsf{tocbasic}}%^^A
% Auch verschiebbare Tabellen sind möglich. Das Prinzip ist das
% gleiche wie bei den Abbildungen.
% \begin{Counter}{table}
% \changes{v3.25}{2017/10/10}{Der Zähler wird implizit von
%     \cs{DeclareNewTOC} definiert.}%^^A
% \end{Counter}
% Der ganze Rest kann mit \cs{DeclareNewTOC} erledigt werden.
%    \begin{macrocode}
\DeclareNewTOC[%
  type=table,
  float,
%<book|report>  counterwithin=chapter,
%    \end{macrocode}
% \begin{macro}{\fps@table}
% Die Standardplatzierung für Tabellen ist \emph{t}op,
% \emph{b}ottom, \emph{p}age.
%    \begin{macrocode}
  floatpos=tbp,
% \end{macro}
% \begin{macro}{\ftype@table}
% Um die verschiedenen \emph{float}-Typen unterscheiden zu können,
% benötigen sie eine Nummer. Tabellen bekommen hier willkürlich
% die Nummer 2.
%    \begin{macrocode}
  floattype=2,
%    \end{macrocode}
% \end{macro}
% \changes{v3.20}{2016/04/07}{Interne Umdefinierung von \cs{caption}}%^^A
%    \begin{macrocode}
  tocentryindent=1.5em,
  tocentrynumwidth=2.3em,
  atbegin=\def\caption{%
    \if@tablecaptionabove\expandafter\captionabove
    \else\expandafter\captionbelow\fi
  }
%    \end{macrocode}
% \begin{macro}{\ext@table}
% Die Extension der Datei für das Abbildungsverzeichnis.
%    \begin{macrocode}
]{lot}
%    \end{macrocode}
% \end{macro}
% \begin{macro}{\thetable}
% \changes{v2.97e}{2007/11/23}{Kapitelnummer nur im Hauptteil}%^^A
% \changes{v3.27}{2019/06/26}{\cs{relax} replaced by percent}%^^A
% \changes{v3.25}{2021/10/12}{use chapter number only of greater than zero}
% \changes{v3.25}{2021/10/12}{redefine \cs{thefigure} only if compatibility is
%   less or equal 2.97d}
%    \begin{macrocode}
%<*book|report>
\expandafter\ifnum\scr@v@is@le{2.97d}%
  \renewcommand*\thetable{\thechapter.\@arabic\c@table}
\fi
%</book|report>
%    \end{macrocode}
% \end{macro}
% \begin{macro}{\fnum@table}
% \begin{macro}{\tableformat}
% \changes{v2.3c}{1995/08/06}{neu}%^^A
% Die Ausgabe der Abbildungsnummer wird implizit von \cs{DeclareNewTOC} als
% \iffalse
%<*ignore>
% \fi
%    \begin{macrocode}
\newcommand*{\fnum@table}{\tableformat}
\newcommand*{\tableformat}{%
  \tablename~\thetable
  \csname autodot\endcsname
}
%    \end{macrocode}
% \iffalse
%</ignore>
% \fi
% definiert.
% \end{macro}
% \end{macro}
% \end{environment}
%
% \iffalse
%</!letter>
%</class>
%</body>
% \fi
%
% \Finale
%
\endinput
%
% end of file `scrkernel-floats.dtx'
%%% Local Variables:
%%% mode: doctex
%%% TeX-master: t
%%% End:
