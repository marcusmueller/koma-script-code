% \iffalse^^A meta-comment
% ======================================================================
% scrlfile.dtx
% Copyright (c) Markus Kohm, 2002-2021
%
% This file is part of the work `scrlfile' which is part of the LaTeX2e
% KOMA-Script bundle.
%
% This work may be distributed and/or modified under the conditions of
% the LaTeX Project Public License, version 1.3c of the license.
% The latest version of this license is in
%   http://www.latex-project.org/lppl.txt
% and version 1.3c or later is part of all distributions of LaTeX 
% version 2005/12/01 or later and of this work.
%
% This work has the LPPL maintenance status "author-maintained".
%
% The Current Maintainer and author of this work is Markus Kohm.
%
% This work consists of all files listed in manifest.txt.
% ----------------------------------------------------------------------
% scrlfile.dtx
% Copyright (c) Markus Kohm, 2002-2021
%
% Diese Datei ist Teil des Werks `scrlfile', das wiederum Teil des
% LaTeX2e KOMA-Script Pakets ist.
%
% Dieses Werk darf nach den Bedingungen der LaTeX Project Public Lizenz,
% Version 1.3c, verteilt und/oder veraendert werden.
% Die neuste Version dieser Lizenz ist
%   http://www.latex-project.org/lppl.txt
% und Version 1.3c ist Teil aller Verteilungen von LaTeX
% Version 2005/12/01 oder spaeter und dieses Werks.
%
% Dieses Werk hat den LPPL-Verwaltungs-Status "author-maintained"
% (allein durch den Autor verwaltet).
%
% Der Aktuelle Verwalter und Autor dieses Werkes ist Markus Kohm.
% 
% Dieses Werk besteht aus den in manifest.txt aufgefuehrten Dateien.
% ======================================================================
%
%%% From File: $Id$
%<*dtx>
\ifx\ProvidesFile\undefined\def\ProvidesFile#1[#2]{}\fi
\begingroup
  \def\filedate$#1: #2-#3-#4 #5${\gdef\filedate{#2/#3/#4}}
  \filedate$Date$
  \def\filerevision$#1: #2 ${\gdef\filerevision{r#2}}
  \filerevision$Revision: 2631 $
  \edef\reserved@a{%
    \noexpand\endgroup
    \noexpand\ProvidesFile{scrlfile.dtx}%
                          [\filedate\space\filerevision\space
                           KOMA-Script package source
  }%
\reserved@a
%</dtx>
%<package>\ProvidesPackage{scrlfile}[%
%!KOMAScriptVersion
%<package>  package 
  (file load hooks)]
%<*dtx>
\ifx\documentclass\undefined
  \input scrdocstrip.tex
  \@@input scrkernel-version.dtx
  \@@input scrstrip.inc
  \KOMAdefVariable{COPYRIGHTFROM}{2002}
  \generate{\usepreamble\defaultpreamble
    \file{scrlfile.sty}{%
      \from{scrlfile.dtx}{package}%
    }%
  }
  \batchinput{scrlfile-hook.dtx}
  \batchinput{scrlfile-patcholdlatex.dtx}
  \@@input scrstrop.inc
\else
  \let\endbatchfile\relax
\fi
\endbatchfile
\documentclass[parskip=half-]{scrdoc}
\usepackage[english]{babel}
\CodelineIndex
\RecordChanges
\GetFileInfo{scrlfile.dtx}
\title{\KOMAScript{} \partname\ \texttt{\filename}%
  \thanks{This file is revision \fileversion\ of file \texttt{\filename}.}}  
\date{\filedate}
\author{Markus Kohm\thanks{mailto:komascript@gmx.info}}

\begin{document}
  \maketitle
  \begin{abstract}
    This package is the compatibility layer over \texttt{scrlfile-hook} and
    \texttt{scrlfile-patcholdlatex}. It provides package and class hook
    handling independent from the used \LaTeX{} release.
  \end{abstract}

  \tableofcontents

  \DocInput{\filename}
\end{document}
%</dtx>
% \fi^^A meta-comment
%
% \changes{v2.95}{2002/06/11}{First version split from hugh scrclass.dtx}
% \changes{v3.24}{2017/05/06}{standalone manual removed}
% \changes{v3.32}{2020/09/11}{new implementation based on
%   \texttt{scrlfile-hook} and \texttt{scrlfile-patcholdlatex}}
%
%
% \section{User Manual}
%
% For user manual see the \KOMAScript{} manual. Here you can find interim
% documentation only (before it is added to the \KOMAScript{} manual).
%
% \DescribeOption{withdeprecated}
% Package option \texttt{withdeprecated} enables the definition of deprecated
% commands \cs{AfterClass+}, \cs{AfterPackage+}, \cs{AfterClass!} and
% \cs{AfterPackage!}. All of them will be emulated by \cs{AfterAtEndOClass*}
% and \cs{AfterAtEndOfPackage*} so they are not completely compatible with the
% original deprecated commands.
%
% Users of classes or package, that use such deprecated commands can load
% \texttt{scrlfile} already before the class using:
% \begin{verbatim}
% \RequirePackage[withdeprecated]{scrlfile}
% \end{verbatim}
% or adding option \texttt{withdeprecated} to the global options of
% \cs{documentclass}. Note, this is not a key-value option. It cannot be
% changed after loading \texttt{scrlfile}.
%
%
% \StopEventually{\PrintIndex\PrintChanges}
%
% \section{Implementation of \textsf{scrlfile}}
%
% We either need to load one of these packages.
% \changes{v3.33}{2021/02/09}{do not test for existence of \cs{AddToHook} any
%   longer, but \cs{IfFormatAtLeastTF} and the format version}%
% We cannot do the decision by testing the existence of macro \cs{AddToHook},
% because package \textsf{latexrelease} does not rollback to prior format
% versions good enough. Instead of undefining the macro an so really reverting
% to the definition of \LaTeX{} kernels prior to 2020/10/01 it only redefines
% it to a do-nothing macro. Because of this and similar issues with
% \textsf{latexrelease} cannot (or only with provision) be used to simulate a
% real old \LaTeX{} kernel, i.e. for testing. So the only valid test for the
% existence of the wanted \cs{AddToHook} would be to test the definite
% definition of \cs{AddToHook} and the existence of the hooks we are
% using. But, because this is to much effort, we simply test the format
% version. Nevertheless, we know, this may fail!
%    \begin{macrocode}
\@ifundefined{IfFormatAtLeastTF}{%
  \RequirePackage{scrlfile-patcholdlatex}%
  \RequirePackage{xparse}%
}{%
  \IfFormatAtLeastTF{2020/10/01}{%
    \RequirePackage{scrlfile-hook}%
  }{%
    \RequirePackage{scrlfile-patcholdlatex}%
    \RequirePackage{xparse}%
  }%
}
%    \end{macrocode}
%
% \begin{option}{withdeprecated}
%   \changes{v3.32}{2020/09/11}{new option}
%   Depending on this option we implement addition deprecated commands.
%    \begin{macrocode}
\DeclareOption{withdeprecated}{%
%    \end{macrocode}
%
% \begin{macro}{\AfterClass+}
% \begin{macro}{\AfterClass!}
% \cs{scrlfile@AfterClass} is used to store the original \cs{AfterClass},
% before redefining \cs{AfterClass} to handle the plus and the exclamation
% mark variant.
%    \begin{macrocode}
  \RenewDocumentCommand\AfterClass{}{%
    \kernel@ifnextchar +%
      {\scrlfile@emulatedeprecated{Class}}%
      {%
        \kernel@ifnextchar !%
          {\scrlfile@emulatedeprecated{Class}}%
          \scrlfile@AfterClass
      }%
  }
%    \end{macrocode}
% \end{macro}
% \end{macro}
%
% \begin{macro}{\AfterPackage+}
% \begin{macro}{\AfterPackage!}
% \cs{scrlfile@AfterPackage} is used to store the original \cs{AfterPackage},
% before redefining \cs{AfterPackage} to handle the plus and the exclamation
% mark variant.
%    \begin{macrocode}
  \RenewDocumentCommand\AfterPackage{}{%
    \kernel@ifnextchar +%
      {\scrlfile@emulatedeprecated{Package}}%
      {%
        \kernel@ifnextchar !%
          {\scrlfile@emulatedeprecated{Package}}%
          \scrlfile@AfterPackage
      }%
  }
%    \end{macrocode}
% \end{macro}
% \end{macro}
%
% \begin{macro}{\scrlfile@emulatedeprecated}
% \changes{v3.32}{2020/09/11}{new internal command}
% This emulates \cs{AfterClass+}, \cs{AfterPackage+}, \cs{AfterClass!} and
% \cs{AfterPackage!} using \cs{AfterAtEndOfClass} or
% \cs{AfterAtEndOfPackage}. Note, that both, the plus and the exclamation mark
% variants, are emulated by the star variant of \cs{AfterAtEndOfClass}
% resp. \cs{AfterAtEndOfPackage}. So the emulation is not perfect.
%    \begin{macrocode}
  \newcommand*{\scrlfile@emulatedeprecated}[2]{%
    \PackageWarning{scrlfile}{%
      emulating deprecated \expandafter\string\csname After#1#2\endcsname
      \space by\MessageBreak
      \expandafter\string\csname AfterAtEndOf#1*\endcsname.\MessageBreak
      Note, this may fail, so you should not use\MessageBreak
      \expandafter\string\csname After#1#2\endcsname
    }%
    \csname AfterAtEndOf#1\endcsname*%
  }
%    \end{macrocode}
% \end{macro}
%
%    \begin{macrocode}
}
%    \end{macrocode}
% \end{option}
%
% Now, we just have to execute and process the options.
%    \begin{macrocode}
\ExecuteOptions{}
\ProcessOptions\relax
%    \end{macrocode}
%
% Last but not least this is a \KOMAScript{} package, so we also define the
% \KOMAScript{} logo:
%    \begin{macrocode}
\RequirePackage{scrlogo}
%    \end{macrocode}
%
% \Finale
%
% \endinput
%
% Local Variables:
% mode: doctex
% TeX-master: t
% End:
