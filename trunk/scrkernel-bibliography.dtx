% \CheckSum{421}
% \iffalse meta-comment
% ======================================================================
% scrkernel-bibliography.dtx
% Copyright (c) Markus Kohm, 2002-2019
%
% This file is part of the LaTeX2e KOMA-Script bundle.
%
% This work may be distributed and/or modified under the conditions of
% the LaTeX Project Public License, version 1.3c of the license.
% The latest version of this license is in
%   http://www.latex-project.org/lppl.txt
% and version 1.3c or later is part of all distributions of LaTeX 
% version 2005/12/01 or later and of this work.
%
% This work has the LPPL maintenance status "author-maintained".
%
% The Current Maintainer and author of this work is Markus Kohm.
%
% This work consists of all files listed in manifest.txt.
% ----------------------------------------------------------------------
% scrkernel-bibliography.dtx
% Copyright (c) Markus Kohm, 2002-2019
%
% Dieses Werk darf nach den Bedingungen der LaTeX Project Public Lizenz,
% Version 1.3c, verteilt und/oder veraendert werden.
% Die neuste Version dieser Lizenz ist
%   http://www.latex-project.org/lppl.txt
% und Version 1.3c ist Teil aller Verteilungen von LaTeX
% Version 2005/12/01 oder spaeter und dieses Werks.
%
% Dieses Werk hat den LPPL-Verwaltungs-Status "author-maintained"
% (allein durch den Autor verwaltet).
%
% Der Aktuelle Verwalter und Autor dieses Werkes ist Markus Kohm.
% 
% Dieses Werk besteht aus den in manifest.txt aufgefuehrten Dateien.
% ======================================================================
% \fi
%
% \CharacterTable
%  {Upper-case    \A\B\C\D\E\F\G\H\I\J\K\L\M\N\O\P\Q\R\S\T\U\V\W\X\Y\Z
%   Lower-case    \a\b\c\d\e\f\g\h\i\j\k\l\m\n\o\p\q\r\s\t\u\v\w\x\y\z
%   Digits        \0\1\2\3\4\5\6\7\8\9
%   Exclamation   \!     Double quote  \"     Hash (number) \#
%   Dollar        \$     Percent       \%     Ampersand     \&
%   Acute accent  \'     Left paren    \(     Right paren   \)
%   Asterisk      \*     Plus          \+     Comma         \,
%   Minus         \-     Point         \.     Solidus       \/
%   Colon         \:     Semicolon     \;     Less than     \<
%   Equals        \=     Greater than  \>     Question mark \?
%   Commercial at \@     Left bracket  \[     Backslash     \\
%   Right bracket \]     Circumflex    \^     Underscore    \_
%   Grave accent  \`     Left brace    \{     Vertical bar  \|
%   Right brace   \}     Tilde         \~}
%
% \iffalse
%%% From File: $Id$
%<option>%%%            (run: option)
%<body>%%%            (run: body)
%<*dtx>
% \fi
\ifx\ProvidesFile\undefined\def\ProvidesFile#1[#2]{}\fi
\begingroup
  \def\filedate$#1: #2-#3-#4 #5${\gdef\filedate{#2/#3/#4}}
  \filedate$Date$
  \def\filerevision$#1: #2 ${\gdef\filerevision{r#2}}
  \filerevision$Revision: 1872 $
  \edef\reserved@a{%
    \noexpand\endgroup
    \noexpand\ProvidesFile{scrkernel-bibliography.dtx}%
                          [\filedate\space\filerevision\space
                           KOMA-Script source
                           (bibliography)]%
  }%
\reserved@a
% \iffalse
\documentclass[parskip=half-]{scrdoc}
\usepackage[english,ngerman]{babel}
\usepackage[utf8]{inputenc}
\CodelineIndex
\RecordChanges
\GetFileInfo{scrkernel-bibliography.dtx}
\title{\KOMAScript{} \partname\ \texttt{\filename}%
  \footnote{Dies ist Version \fileversion\ von Datei
    \texttt{\filename}.}}
\date{\filedate}
\author{Markus Kohm}

\begin{document}
  \maketitle
  \tableofcontents
  \DocInput{\filename}
\end{document}
%</dtx>
% \fi
% \selectlanguage{ngerman}
%
% \changes{v2.95}{2002/06/30}{%
%   erste Version aus der Aufteilung von \texttt{scrclass.dtx}}
%
% \section{Literaturverzeichnis}
%
% Das Literaturverzeichnis unterscheidet sich bei \KOMAScript{} nicht
% wesentlich von den Standardklassen.
%
% Briefe haben kein Literaturverzeichnis.
%
% \StopEventually{\PrintIndex\PrintChanges}
%
% \iffalse
%<*!letter>
% \fi
%
% \iffalse
%<*option>
% \fi
% \subsection{Anweisungen, die für die Optionen benötigt werden}
%
% Einige Anweisungen werden über Optionen verändert. Dazu müssen sie aber
% bereits vor der Abarbeitung der Optionen definiert sein.
%
% \subsubsection{Literaturverzeichnisstile}
%
% \begin{macro}{\newbibstyle}
% \changes{v2.98c}{2008/02/15}{Neu}%^^A
% \changes{v3.17}{2015/03/12}{interne Speicherung des Werts für Option
%     \texttt{bibliography}}%^^A
% Neue Literaturverzeichnisstile müssen mit dieser Anweisung definiert werden,
% wenn sie von der Fehlermeldung korrekt angezeigt werden sollen. Das erste,
% optionale Argument ist der Name eines Elternstils. Das zweite
% Argument ist der Name des Stils. Das dritte Argument ist Code für den
% Stil. Darin sollten normalerweise Umdefinierungen für \cs{@openbib@code},
% \cs{newblock}, \cs{bib@beginhook}, \cs{bib@endhook} enthalten sein. Die
% genannten Makros werden am Anfang des Stils zurückgesetzt. Danach wird der
% Elternstil geladen und dann die eigenen Definitionen.
%    \begin{macrocode}
\newcommand{\newbibstyle}[3][]{%
  \scr@ifundefinedorrelax{scr@bibstyle@#2}{%
    \scr@ifundefinedorrelax{scr@bibstyle@#1}{%
      \ClassError{\KOMAClassName}{%
        parent bibliography style `#1' does not exist}{%
        You've told me to define the new bibliography style `#2'\MessageBreak
        to be a child of bibliography style `#1', but bibliography
        style\MessageBreak
        `#1' does not exist.\MessageBreak
        You have to remove the parent declaration or use an existing parent!
        See the KOMA-Script manual for more information.%
      }%
    }{%
      \@namedef{scr@bibstyle@#2}{%
        \let\@openbib@code\@empty
        \let\newblock\@empty
        \let\bib@beginhook\@empty
        \let\bib@endhook\@empty
%    \end{macrocode}
% \begin{macro}{\scr@current@bibstyle}
% \changes{v3.17}{2015/03/12}{Neu (intern)}%^^A
% Das Makro \cs{scr@current@bibstyle} wird benötigt, um den aktuellen Wert von
% Option \texttt{bibliography} korrigieren zu können. Es wird direkt vom Stil
% definiert, der auch die Korrektur des Optionenwerts durchführt.
%    \begin{macrocode}
        \@ifundefined{scr@current@bibstyle}{}{%
          \ifstr{\scr@current@bibstyle}{}{}{%
            \edef\reserved@a{%
              \noexpand\KOMA@kav@remove{.\KOMAClassFileName}{bibliography}%
                                       {\scr@current@bibstyle}%
            }\reserved@a
          }%
        }%
        \def\scr@current@bibstyle{#2}%
%    \end{macrocode}
% \end{macro}
%    \begin{macrocode}
        \KOMA@kav@xadd{.\KOMAClassFileName}{bibliography}{#2}%
        \csname scr@bibstyle@#1\endcsname
        #3%
      }%
      \g@addto@macro\scr@bibstylelist{, `#2'}%
    }%
  }{%
    \ClassError{\KOMAClassName}{bibliography style `#2' already exists}{%
      You've told me to define the new bibliography style `#2',\MessageBreak
      but this style already exists.\MessageBreak
      You have to use another bibliography style name.\MessageBreak
      See the KOMA-Script manual for more information.%
    }%
  }%
}
%    \end{macrocode}
% \begin{macro}{\scr@bibstyle@}
% \changes{v2.98c}{2008/02/15}{Neu (intern)}%
%   Nur ein Dummy, wenn kein Elternstil existiert.
%    \begin{macrocode}
\newcommand*{\scr@bibstyle@}{}
%    \end{macrocode}
% \end{macro}
% \begin{macro}{\bib@beginhook}
% \changes{v2.98c}{2008/02/15}{Neu (intern)}%
% \begin{macro}{\bib@endhook}
% \changes{v2.98c}{2008/02/15}{Neu (intern)}%
% Die beiden Haken, wie sie durch den Literaturverzeichnisstil vorgegeben
% werden.
%    \begin{macrocode}
\newcommand*{\bib@beginhook}{}
\newcommand*{\bib@endhook}{}
%    \end{macrocode}
% \end{macro}
% \end{macro}
% \begin{macro}{\scr@bibstylelist}
% \changes{v2.98c}{2008/02/15}{Neu (intern)}%
% Liste der mit \cs{newbibstyle} definierten und daher bekannten
% Literaturverzeichnisstile.
% \begin{macro}{\@openbib@code}
% \changes{v2.3g}{1996/01/14}{Neu (intern)}%^^A
% \changes{v2.98c}{2008/02/15}{wird nun früher definiert}%^^A
% Code, der am Anfang des Literaturverzeichnisses ausgeführt wird und
% normalerweise ursprünglich für die \texttt{openbib}-Funktionalität verwendet
% wurde. Inzwischen wird er für die komplette Stilauswahl verwendet.
% \begin{macro}{\newblock}
% \changes{v2.3g}{1996/01/14}{Grundeinstellung geändert}%^^A
% \changes{v2.98c}{2008/02/15}{wird nun früher definiert}%^^A
%    \begin{macrocode}
\newcommand*{\scr@bibstylelist}{\@gobble}
\newcommand*{\@openbib@code}{}
\newcommand*{\newblock}{}
%    \end{macrocode}
% \end{macro}
% \end{macro}
% \end{macro}
% \begin{macro}{\scr@bibstyle@openstyle}
% \changes{v2.98c}{2008/02/15}{Neu (intern)}%
% Vordefiniert sind zwei Stile.  Mit dem Bibliographystil \texttt{openstyle}
% kann eine alternative Darstellung der Literatureinträge aktiviert
% werden. Dabei fügt \cs{newblock} keinen horizontalen Abstand mehr ein,
% sondern beginnt einen neuen Absatz (in Form einer neuen Zeile).
% \begin{macro}{\scr@bibstyle@oldstyle}
% \changes{v2.98c}{2008/02/15}{Neu (intern)}%^^A
% Der Stil \texttt{oldstyle} entspricht hingegen der Voreinstellung.
%    \begin{macrocode}
\newbibstyle{openstyle}{%
  \renewcommand*{\@openbib@code}{%
    \advance\leftmargin\bibindent
    \itemindent -\bibindent
    \listparindent \itemindent
    \parsep \z@
  }%
  \renewcommand*{\newblock}{\par}%
}
\newbibstyle{oldstyle}{%
  \renewcommand*{\newblock}{\hskip .11em\@plus.33em\@minus.07em}%
}
\scr@bibstyle@oldstyle
%    \end{macrocode}
% \end{macro}
% \end{macro}
% \end{macro}
%
%
% \subsubsection{Überschrift des Literaturverzeichnisses}
%
% \begin{macro}{\bib@heading}
% \changes{v2.3h}{1995/01/21}{neu (intern)}%^^A
% \changes{v2.6c}{2000/06/10}{die Präambel wird gesetzt}%^^A
% \changes{v2.7b}{2001/01/05}{\cs{setchapterpreamble} wird verwendet}%^^A
% \changes{v2.7j}{2001/06/27}{Setzen der Präambel in die
%   \texttt{thebibliography}-Umgebung verschoben}%^^A
% \changes{v2.98c}{2008/02/15}{wird nun früher definiert}%^^A
% \changes{v3.10}{2011/08/31}{\cs{MakeMarkcase} wird beachtet}%^^A
% \changes{v3.12}{2012/02/06}{\cs{ifbib@leveldown} wird beachtet}%^^A
% \changes{v3.13b}{2014/10/10}{running head with \texttt{leveldown}%^^A
%   fixed}%^^A
% \changes{v3.25}{2017/10/13}{use always \cs{@mkright} and \cs{@mkdouble}}%^^A
% \changes{v3.25}{2017/10/13}{using \cs{bibliography@heading}}%^^A
% \begin{macro}{\bibliography@heading}
% \changes{v3.25}{2017/10/13}{neu (intern)}
% \changes{v3.26}{2018/09/03}{Verwendung von \cs{addsecmark}
%   oder \cs{subsectionmark} statt \cs{@mkright}}%^^A
% Befehl zum Setzen der Überschrift des Literaturverzeichnisses.
%    \begin{macrocode}
\newcommand*{\bib@heading}{%
%<article>  \bibliography@heading{\refname}%
%<book|report>  \bibliography@heading{\bibname}%
}
\newcommand*{\bibliography@heading}[1]{%
  \ifbib@leveldown
%<article>    \subsection*{#1}%
%<book|report>    \section*{#1}%
%<book|report>    \addsecmark{#1}%
%<*article>
    \begingroup
      \c@secnumdepth=-\maxdimen
      \subsectionmark{#1}%
    \endgroup
%</article>  
  \else
%<article>    \section*{#1}%
%<book|report>    \chapter*{#1}%
    \@mkdouble{\MakeMarkcase{#1}}%
  \fi
}%
%    \end{macrocode}
% \end{macro}
% \end{macro}%
%
%
% \subsection{Optionen für das Literaturverzeichnis}
%
% \begin{option}{bibliography}
% \changes{v2.98c}{2008/02/15}{Neu}%^^A
% \changes{v3.10}{2011/08/31}{\cs{MakeMarkcase} wird beachtet}%^^A
% \changes{v3.12}{2013/02/06}{neuer Wert \texttt{leveldown}}%^^A
% \changes{v3.12}{2013/02/06}{neuer Wert \texttt{standardlevel}}%^^A
% \changes{v3.12}{2013/03/05}{an die Änderungen in \textsf{scrbase}%^^A
%     angepasst}%^^A
% \changes{v3.12}{2013/09/28}{Verwendung von \cs{addxcontentsline} statt
%     \cs{addcontentsline}}%^^A
% \changes{v3.12}{2013/09/28}{Es werden keine Kapiteleinträge in andere
%     Verzeichnisse geschrieben.}%^^A
% \changes{v3.17}{2015/03/12}{interne Speicherung der Werte}%^^A
% \changes{v3.25}{2017/10/13}{use always \cs{@mkright} and
%     \cs{@mkdouble}}%^^A
% \changes{v3.25}{2017/10/13}{redefines \cs{bibliography@heading} instead of
%     \cs{bib@heading}}%^^A
%   Dies ist der Generalschlüssel für alle Optionen, die das
%   Literaturverzeichnis betreffen. Neu ist, dass verschiedene Stile definiert
%   werden können. Desweiteren kann man \texttt{totoc} und
%   \texttt{totocnumbered} wieder mit \texttt{nottotoc} aufgeben.
% \begin{macro}{\ifbib@leveldown}
% \changes{v3.12}{2013/02/06}{neuer Schalter (intern)}%^^A
%   Für den ab Version~3.12 neuen Wert \texttt{leveldown} und dessen
%   Gegenstück \texttt{standardlevel} wird außerdem ein Schalter benötigt.
%    \begin{macrocode}
\newif\ifbib@leveldown
\KOMA@key{bibliography}{%
  \scr@ifundefinedorrelax{scr@bibstyle@#1}{%
    \KOMA@set@ncmdkey{bibliography}{@tempa}{%
      {notoc}{0},{nottotoc}{0},{plainheading}{0},%
      {totoc}{1},{toc}{1},{notnumbered}{1},%
      {numbered}{2},{totocnumbered}{2},{tocnumbered}{2},{numberedtotoc}{2},%
      {numberedtoc}{2},%
      {leveldown}{3},%
      {standardlevel}{4}%
    }{#1}%
    \ifcase \@tempa\relax % notoc
      \KOMA@kav@remove{.\KOMAClassFileName}{toc}{nobibliography}%
      \KOMA@kav@remove{.\KOMAClassFileName}{toc}{bibliography}%
      \KOMA@kav@remove{.\KOMAClassFileName}{toc}{bibliographynumbered}%
      \KOMA@kav@add{.\KOMAClassFileName}{toc}{nobibliography}%
      \KOMA@kav@remove{.\KOMAClassFileName}{bibliography}{notoc}%
      \KOMA@kav@remove{.\KOMAClassFileName}{bibliography}{totoc}%
      \KOMA@kav@remove{.\KOMAClassFileName}{bibliography}{numbered}%
      \KOMA@kav@add{.\KOMAClassFileName}{bibliography}{notoc}%
      \renewcommand*{\bibliography@heading}[1]{%
        \ifbib@leveldown
%<article>          \subsection*{##1}%
%<book|report>          \section*{##1}%
          \@mkright{\MakeMarkcase{##1}}%
        \else
%<article>          \section*{##1}%
%<book|report>          \chapter*{##1}%
          \@mkdouble{\MakeMarkcase{##1}}%
        \fi
      }%
    \or% totoc
      \KOMA@kav@remove{.\KOMAClassFileName}{toc}{nobibliography}%
      \KOMA@kav@remove{.\KOMAClassFileName}{toc}{bibliography}%
      \KOMA@kav@remove{.\KOMAClassFileName}{toc}{bibliographynumbered}%
      \KOMA@kav@add{.\KOMAClassFileName}{toc}{bibliography}%
      \KOMA@kav@remove{.\KOMAClassFileName}{bibliography}{notoc}%
      \KOMA@kav@remove{.\KOMAClassFileName}{bibliography}{totoc}%
      \KOMA@kav@remove{.\KOMAClassFileName}{bibliography}{numbered}%
      \KOMA@kav@add{.\KOMAClassFileName}{bibliography}{totoc}%
      \renewcommand*{\bibliography@heading}[1]{%
%<*article>
        \ifbib@leveldown
%    \end{macrocode}
% \changes{v3.23}{2017/02/23}{Verwendung von \cs{ext@toc}}%^^A
% \changes{v3.25}{2017/10/13}{Verwendung von \cs{addsubsectiontocentry} statt
%   \cs{addxcontentsline}}%^^A
% Ab Version 3.23 wird das von \textsf{tocbasic} bereitgestellte
% \cs{ext@toc} statt dem hart codierten \texttt{toc} verwendet.
%    \begin{macrocode}
          \subsection*{%
                \addsubsectiontocentry{}{##1}%
                ##1%
              }%
          \@mkright{\MakeMarkcase{##1}}%
        \else
          \addsec{##1}%
          \@mkdouble{\MakeMarkcase{##1}}%
        \fi
%</article>
%<*book|report>
        \ifbib@leveldown
          \addsec{##1}%
        \else
          \if@chaptertolists
            \@chaptertolistsfalse\addchap{##1}\@chaptertoliststrue
          \else
            \addchap{##1}%
          \fi
          \@mkdouble{\MakeMarkcase{##1}}%
        \fi
%</book|report>
      }%
    \or% numbered
      \KOMA@kav@remove{.\KOMAClassFileName}{toc}{nobibliography}%
      \KOMA@kav@remove{.\KOMAClassFileName}{toc}{bibliography}%
      \KOMA@kav@remove{.\KOMAClassFileName}{toc}{bibliographynumbered}%
      \KOMA@kav@add{.\KOMAClassFileName}{toc}{bibliographynumbered}%
      \KOMA@kav@remove{.\KOMAClassFileName}{bibliography}{notoc}%
      \KOMA@kav@remove{.\KOMAClassFileName}{bibliography}{totoc}%
      \KOMA@kav@remove{.\KOMAClassFileName}{bibliography}{numbered}%
      \KOMA@kav@add{.\KOMAClassFileName}{bibliography}{numbered}%
      \renewcommand*{\bibliography@heading}[1]{%
%<*article>
        \ifbib@leveldown
          \subsection{##1}%
        \else
          \section{##1}%
          \@mkdouble{\MakeMarkcase{\sectionmarkformat ##1}}%
        \fi
%</article>
%<*book|report>
        \ifbib@leveldown
          \section{##1}%
        \else
          \if@chaptertolists
            \@chaptertolistsfalse\chapter{##1}\@chaptertoliststrue
          \else
            \chapter{##1}
          \fi
          \@mkdouble{\MakeMarkcase{\chaptermarkformat ##1}}%
        \fi
%</book|report>
      }
    \or% leveldown
      \KOMA@kav@remove{.\KOMAClassFileName}{bibliography}{leveldown}%
      \KOMA@kav@remove{.\KOMAClassFileName}{bibliography}{standardlevel}%
      \KOMA@kav@add{.\KOMAClassFileName}{bibliography}{leveldown}%
      \bib@leveldowntrue
    \or% standardlevel
      \KOMA@kav@remove{.\KOMAClassFileName}{bibliography}{leveldown}%
      \KOMA@kav@remove{.\KOMAClassFileName}{bibliography}{standardlevel}%
      \KOMA@kav@add{.\KOMAClassFileName}{bibliography}{standardlevel}%
      \bib@leveldownfalse
    \fi
  }{%
    \FamilyKeyStateProcessed
    \edef\reserved@a{%
      \noexpand\KOMA@kav@remove{.\KOMAClassFileName}{bibliography}%
                               {\scr@current@bibstyle}%
    }\reserved@a
    \csname scr@bibstyle@#1\endcsname
  }%
}
%    \end{macrocode}
% \end{macro}
% \end{option}
%
% \begin{option}{openbib}
% \changes{v2.3g}{1996/01/14}{\cs{if@openbib} (\cs{@openbibtrue},
%     \cs{@openbibfalse}) entfernt}
% \changes{v2.3g}{1996/01/14}{Definition der openbib Option neu
%     implementiert}%^^A
% \changes{v2.98c}{2008/02/15}{obsolete}%^^A
% \changes{v3.01a}{2008/11/21}{standard statt obsolet}%^^A
% Mit dieser Option kann eine alternative Darstellung der
% Literatureinträge aktiviert werden. Dabei fügt \cs{newblock} keinen
% horizontalen Abstand mehr ein, sondern beginnt einen neuen Absatz
% (in Form einer neuen Zeile).
%    \begin{macrocode}
\KOMA@DeclareStandardOption{openbib}{bibliography=openstyle}
%    \end{macrocode}
% \end{option}
%
% \begin{option}{bibtotoc}
% \changes{v2.3h}{1995/01/21}{neue Option}%^^A
% \changes{v2.6b}{2000/05/26}{\cs{@mkboth} eingefügt}
% \changes{v2.6c}{2000/06/10}{\cs{bibpreamble} eingefügt}
% \changes{v2.7}{2001/01/03}{Klammerung korrigiert}%^^A
% \changes{v2.7j}{2001/06/27}{Setzen der Präambel in die
%     thebibliography-Umgebung verschoben}%^^A
% \changes{v2.98c}{2008/02/15}{obsolete}%^^A
% \changes{v3.01a}{2008/11/20}{deprecated}%^^A
% \begin{option}{bibtotocnumbered}
% \changes{v2.3h}{1995/01/21}{neue Option}%^^A
% \changes{v2.6b}{2000/05/26}{\cs{@mkboth} eingefügt}
% \changes{v2.6c}{2000/06/10}{\cs{bibpreamble} eingefügt}
% \changes{v2.7}{2001/01/03}{Klammerung korrigiert}%^^A
% \changes{v2.7b}{2001/01/05}{\cs{setchapterpreamble} wird
%     verwendet}
% \changes{v2.7i}{2001/05/17}{\cs{chapter} statt \cs{addchap}}
% \changes{v2.7j}{2001/06/27}{Setzen der Präambel in die
%     thebibliography-Umgebung verschoben}%^^A
% \changes{v2.9p}{2003/07/07}{Nummerierung im Kolumnentitel}%^^A
% \changes{v2.98c}{2008/02/15}{obsolete}%^^A
% \changes{v3.01a}{2008/11/20}{deprecated}%^^A
% Die Option \texttt{bibtotoc} sorgt dafür, dass das
% Literaturverzeichnis im Inhaltsverzeichnis aufgeführt wird.
% Demgegenüber erweitert die Option \texttt{bibtotocnumbered} zu einem
% normalen Kapitel einschließlich Nummerierung.
%    \begin{macrocode}
\KOMA@DeclareDeprecatedOption{bibtotoc}{bibliography=totoc}
\KOMA@DeclareDeprecatedOption{bibtotocnumbered}{bibliography=totocnumbered}
%    \end{macrocode}
% \end{option}
% \end{option}
%
%
% \iffalse
%</option>
%<*body>
% \fi
%
%
% \subsection{Standardbezeichner}
%
% \begin{macro}{\refname}
% \begin{macro}{\bibname}
% Je nach Klasse wird \cs{refname} oder \cs{bibname} verwendet.
%    \begin{macrocode}
%<article>\newcommand*\refname{References}
%<book|report>\newcommand*\bibname{Bibliography}
%    \end{macrocode}
% \end{macro}
% \end{macro}
%
%
% \subsection{Definitionen für das Literaturverzeichnis}
%
% \begin{macro}{\setbibpreamble}
% \changes{v2.6c}{2000/06/10}{neu}%^^A
% \changes{v2.7j}{2001/05/27}{\cs{renewcommand} durch \cs{def}
%     ersetzt zur Erhöhung der \textsf{natbib}-Kompatibilität}
% \begin{macro}{\bibpreamble}
% \changes{v2.6c}{2000/06/10}{neue (intern)}%^^A
% \changes{v2.6d}{2000/07/20}{Makro funktioniert nachdem \cs{let}%^^A
%     durch \cs{newcommand*} ersetzt wurde}
% \changes{v2.8q}{2002/01/10}{wieder \cs{let}}
%  \cs{bibpreamble} kann vom Benutzer mit \cs{setbibpreamble} auf
%  einen Text gesetzt werden, der nach der Überschrift eingefügt
%  wird. Dasselbe Makro wird von \textsf{natbib} als \cs{relax}
%  definiert; aus Kompatibilitätsgründen wird hier vergleichbar
%  verfahren und \cs{let} bzw. \cs{def} statt \cs{renewcommand}
%  verwendet.
% \changes{v2.7j}{2001/05/27}{Setzen der Präambel in die
%     \texttt{thebibliography}-Umgebung verschoben}
%  Außerdem wird ab Version 2.7j die Präambel nicht mehr in
%  \cs{bib@heading} gesetzt, sondern in der
%  \texttt{thebibliography}-Umgebung. Dies dient ebenfalls der
%  Kompatibilität mit natbib. Trotz aller Kompatibilität ist
%  \cs{bibpreamble} als internes Makro zu betrachten.
%    \begin{macrocode}
\newcommand*{\bibpreamble}{\relax}
\let\bibpreamble\relax
\newcommand{\setbibpreamble}[1]{\def\bibpreamble{#1}}
%    \end{macrocode}
% \end{macro}
% \end{macro}
%
% \begin{Length}{bibindent}
% Der Einzug der Einträge bei \texttt{openbib}. Voreingestellt sind
% 1,5\,em.
%    \begin{macrocode}
\newdimen\bibindent
\bibindent=1.5em
%    \end{macrocode}
% \end{Length}
%
% \begin{environment}{thebibliography}
% \changes{v2.2a}{1995/02/01}{in \textsf{scrartcl} \cs{bibname}
%     vollständig durch \cs{refname} ersetzt}
% \changes{v2.3h}{1995/01/21}{Verwendung von \cs{bib@heading}}
% \changes{v2.8q}{2002/01/10}{Erkennung und gesonderte Behandlung
%     leerer Präambeln}%^^A
% \changes{v2.9o}{2003/03/06}{Vergleich \cs{bibpreamble} mit \cs{@empty}
%     korrigiert}
% \changes{v2.98c}{2008/02/15}{Neue Haken \cs{bib@before@hook} und
%     \cs{bib@after@hook} eingefügt}
% Nun werden noch die Umgebung für das Literaturverzeichnis und der
% Befehl für die einzelnen Einträge definiert.
%    \begin{macrocode}
\newenvironment{thebibliography}[1]{%
  \if@bib@break@
    \bib@beginhook
  \else
%<*book|report>
    \ifx\bibpreamble\relax\else\ifx\bibpreamble\@empty\else
        \setchapterpreamble{\bibpreamble}%
    \fi\fi
%</book|report>
    \bib@heading
%<*article>
    \ifx\bibpreamble\relax\else\ifx\bibpreamble\@empty\else
        \noindent\bibpreamble\par\nobreak
    \fi\fi
%</article>
    \bib@beginhook
    \bib@before@hook
  \fi
%    \end{macrocode}
% \begin{macro}{\BreakBibliography}
% \changes{v2.98c}{2008/02/15}{Neu}%^^A
% An dieser Stelle wird \cs{BreakBibliography} passend umdefiniert.
%    \begin{macrocode}
  \renewcommand{\BreakBibliography}[1]{%
    \@bib@break
    \end{thebibliography}%
    \begingroup ##1\endgroup
    \begin{thebibliography}{#1}
  }%
%    \end{macrocode}
% \end{macro}
%    \begin{macrocode}
% Hier geht es jetzt mit \texttt{thebibliography} weiter.
  \list{%
    \@biblabel{\@arabic\c@enumiv}%
  }{%
    \settowidth\labelwidth{\@biblabel{#1}}%
    \leftmargin\labelwidth
    \advance\leftmargin\labelsep
    \@openbib@code
    \if@bib@break@
      \edef\@tempa{%
        \noexpand\usecounter{enumiv}%
        \noexpand\setcounter{enumiv}{\the\c@enumiv}%
      }\@tempa
      \no@bib@break
    \else
      \usecounter{enumiv}%
    \fi
    \let\p@enumiv\@empty
    \renewcommand*\theenumiv{\@arabic\c@enumiv}%
  }%
  \sloppy\clubpenalty4000 \widowpenalty4000
  \sfcode`\.=\@m
}{%
  \if@bib@break@
    \def\@noitemerr{%
      \@latex@warning{%
        \string\BreakBibliography\space at begin of `thebibliography'
        environment\MessageBreak
        or following \string\BreakBibliography}%
    }%
  \else
    \def\@noitemerr{%
      \@latex@warning{%
        Empty `thebibliography' environment or
        \string\BreakBibliography\MessageBreak
        before end of `thebibliography' environment}%
    }%
  \fi
  \endlist
  \if@bib@break@
  \else
    \bib@after@hook
    \let\bib@before@hook\@empty
    \let\bib@after@hook\@empty
  \fi
  \bib@endhook
}
%    \end{macrocode}
% \begin{macro}{\if@bib@break@}
% \changes{v2.98c}{2008/02/15}{Neu (intern)}%^^A
% \begin{macro}{\no@bib@break}
% \changes{v2.98c}{2008/02/15}{Neu (intern)}%^^A
% \begin{macro}{\@bib@break}
% \changes{v2.98c}{2008/02/15}{Neu (intern)}%^^A
% Dieser Hilfsschalter arbeitet immer global!
%    \begin{macrocode}
\newif\if@bib@break@
\newcommand{\no@bib@break}{\global\let\if@bib@break@\iffalse}
\newcommand*{\@bib@break}{\global\let\if@bib@break@\iftrue}
%    \end{macrocode}
% \end{macro}
% \end{macro}
% \end{macro}
% \begin{macro}{\bib@before@hook}
% \changes{v2.98c}{2008/02/15}{Neu (intern)}%^^A
% \begin{macro}{\bib@after@hook}
% \changes{v2.98c}{2008/02/15}{Neu (intern)}%^^A
%    \begin{macrocode}
\newcommand*{\bib@before@hook}{}
\newcommand*{\bib@after@hook}{}
%    \end{macrocode}
% \end{macro}
% \end{macro}
% \end{environment}
%
% \begin{macro}{\BreakBibliography}
% \changes{v2.98c}{2008/02/15}{Neu}%^^A
% Diese Anweisung unterbricht normalerweise das Literaturverzeichnis für ein
% paar eigene Anweisungen (z. B. Unterüberschriften). Wenn jedoch nicht das
% \KOMAScript-Literaturverzeichnis verwendet wird, dann gibt es eine Warnung
% aus.
%    \begin{macrocode}
\newcommand{\BreakBibliography}[1]{%
  \ClassWarning{\KOMAClassName}{%
    You've used \string\BreakBibliography\space either outside of\MessageBreak
    the bibliography or `thebibliography' was redefined\MessageBreak
    incompatible, e.g., using a bibliography package.\MessageBreak
    \string\BreakBibliography\space ignored%
  }%
}
%    \end{macrocode}
% \end{macro}
%
% \begin{macro}{\AfterBibliographyPreamble}
% \changes{v2.98c}{2008/02/15}{Neu}%^^A
% \begin{macro}{\AtEndBibliography}
% \changes{v2.98c}{2008/02/15}{Neu}%^^A
% Zwei neue Anweisungen, um am Anfang des Literaturverzeichnisses, nach
% Überschrift und Präambel, und am Ende des Literaturverzeichnisses, nach
% Abschluss der Liste, Code einschleusen zu können. Das funktioniert natürlich
% nur so lange, wie keine Paket verwendet wird, das die Umgebung umdefiniert.
%    \begin{macrocode}
\newcommand{\AfterBibliographyPreamble}{\g@addto@macro\bib@before@hook}
\newcommand{\AtEndBibliography}{\g@addto@macro\bib@after@hook}
%    \end{macrocode}
% \end{macro}
% \end{macro}
%
% \iffalse
%</body>
% \fi
%
% \iffalse
%</!letter>
% \fi
%
% \Finale
%
\endinput
%
% end of file `scrkernel-bibliography.dtx'
%%% Local Variables:
%%% mode: doctex
%%% TeX-master: t
%%% End:
