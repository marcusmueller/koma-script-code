% ======================================================================
% preface.tex
% Copyright (c) Markus Kohm, 2008-2017
%
% This file is part of the LaTeX2e KOMA-Script bundle.
%
% This work may be distributed and/or modified under the conditions of
% the LaTeX Project Public License, version 1.3c of the license.
% The latest version of this license is in
%   http://www.latex-project.org/lppl.txt
% and version 1.3c or later is part of all distributions of LaTeX
% version 2005/12/01 or later and of this work.
%
% This work has the LPPL maintenance status "author-maintained".
%
% The Current Maintainer and author of this work is Markus Kohm.
%
% This work consists of all files listed in manifest.txt.
% ----------------------------------------------------------------------
% preface.tex
% Copyright (c) Markus Kohm, 2008-2017
%
% Dieses Werk darf nach den Bedingungen der LaTeX Project Public Lizenz,
% Version 1.3c, verteilt und/oder veraendert werden.
% Die neuste Version dieser Lizenz ist
%   http://www.latex-project.org/lppl.txt
% und Version 1.3c ist Teil aller Verteilungen von LaTeX
% Version 2005/12/01 oder spaeter und dieses Werks.
%
% Dieses Werk hat den LPPL-Verwaltungs-Status "author-maintained"
% (allein durch den Autor verwaltet).
%
% Der Aktuelle Verwalter und Autor dieses Werkes ist Markus Kohm.
%
% Dieses Werk besteht aus den in manifest.txt aufgefuehrten Dateien.
% ======================================================================

\KOMAProvidesFile{preface.tex}
                 [$Date$
                  Preface to version 3.22]
\translator{Markus Kohm\and Karl Hagen}

% Date of the translated German file: 2017-01-16

\addchap{Preface to \KOMAScript~3.22a}

Since the release of \KOMAScript{}~3, which included a manual that was
completely revised in structure and in content, I have received relatively
little feedback. Apparently, the decision to expand the chapter on letters
with countless examples and to make its content independent of other classes
has paid off. Therefore, although various sections, albeit often with modified
or omitted examples, are included multiple times in this
\iffree{manual}{book}, this procedure has been applied more consistently as of
\iffree{\KOMAScript~3.22}{the sixth edition} and has also been applied to
other chapters.

\iffree{The free \KOMAScript{} manual has also been redesigned and provided
  with countless new links. For example, clicking on a \KOMAScript{} macro,
  except in its immediate description, now takes you to the appropriate
  explanation. Cross-references to the matching section or page number still
  exist. This is due to the printed book and also serves as orientation.}{}

\iffree{Readers of this free, screen version, however, still have to live with
  some restrictions. So some information\,---\,mainly that for advanced users
  or capable of turning an ordinary user into an advanced one\,---\,is
  reserved for the printed book, which currently exists only in German. As a
  result, some links in this manual lead to a page that simply mentions this
  fact. In addition, the free version is scarcely suitable for making a
  hard-copy. The focus, instead, is on using it on screen, in parallel with
  the document you are working on. It still has no optimized wrapping but is
  almost a first draft, in which both the paragraph and page breaks are in
  some cases quite poor. Corresponding optimizations are reserved for the book
  edition \cite{book:komascript}.}{}

It is not just about the manual that I now receive little criticism. For the
classes and packages as well, there are hardly any requests for new features.
For me, this means that my knowledge about user desires stagnates. So for a
few years, I mostly implemented things that I thought could be useful.
However, the feedback that I have received about these new possibilities was
largely limited to complaints that old \emph{hacks} based on undocumented
\KOMAScript{} features sometimes no longer work. Little was said about the
happiness that such dirty workarounds were no longer necessary. Therefore, I
have decided to limit extensions and improvements to \KOMAScript{} more and
more to those things that are explicitly requested by users. Could it be that
\KOMAScript{}, after only 25 years, has reached the level that it fulfils all
desires?

Unfortunately, the declining number of error reports is not purely gratifying.
Over this period, I have often observed that those who discover a problem no
longer report it directly to me but work around it with the help of some
Internet forums. Often, there are more or less ingenious workarounds in these
forums. Although this is generally helpful, it unfortunately, as a rule,
causes the problem to remain unreported and therefore never really eliminated.
It goes without saying that such workarounds can sometimes become a problem
themselves, as mentioned in the previous paragraph.

Thankfully, there are third parties who occasionally point out such issues.
This applies to individual contributions in a very few forums. Direct contact
with the person for whom the problem occurred is in this case usually not
possible, although it would sometimes be desirable.

Therefore, let me please ask again explicitly that you report all suspected
bugs directly, either in German or in English. Linguistic perfection is less
important. The message should be reasonably understandable and the problem
comprehensible. A code example that is as short as possible is generally
independent of the language used. With direct contact, I can ask further
questions, if necessary. Please do not rely on anyone else to report the
problem at some point. Assume that it will only be fixed if you report it
yourself. More about error messages can be found in the first chapter of the
manual.

\bigskip\noindent
Markus Kohm, Neckarhausen in freezing temperatures, January, 2017.

\addchap{Preface to the English \KOMAScript{} Guide}

The translation of the German \KOMAScript{} guide is still a work in progress
and a never-ending story. I always try to have an English user guide with the
same descriptions as the German one. But as long as I have to do the primary
translation, these translations not only can but should be improved.

The translation of \autoref{cha:tocbasic} has still an example in German. A
proficient English speaker with basic TeX knowledge could improve the
translation. A large part of \autoref{cha:scrjura} has been translated by
Alexander Willand. The remaining part may need improvement.

So this English guide is complete but nevertheless not as good as the German
one. Currently there are only a few editors for the English guide, who improve
my translation for free. Many thanks to them for their very good job!
Nevertheless, additional editors or translators would be welcome!

\endinput

%%% Local Variables: 
%%% mode: latex
%%% mode: flyspell
%%% ispell-local-dictionary: "english"
%%% coding: us-ascii
%%% TeX-master: "../guide"
%%% End: 

