% \iffalse^^A meta-comment
% ======================================================================
% scrlfile-patcholdlatex.dtx
% Copyright (c) Markus Kohm, 2002-2020
%
% This file is part of the work `scrlfile' which is part of the LaTeX2e
% KOMA-Script bundle.
%
% This work may be distributed and/or modified under the conditions of
% the LaTeX Project Public License, version 1.3c of the license.
% The latest version of this license is in
%   http://www.latex-project.org/lppl.txt
% and version 1.3c or later is part of all distributions of LaTeX 
% version 2005/12/01 or later and of this work.
%
% This work has the LPPL maintenance status "author-maintained".
%
% The Current Maintainer and author of this work is Markus Kohm.
%
% This work consists of all files listed in manifest.txt.
% ----------------------------------------------------------------------
% scrlfile-patcholdlatex.dtx
% Copyright (c) Markus Kohm, 2002-2020
%
% Diese Datei ist Teil des Werks `scrlfile', das wiederum Teil des
% LaTeX2e KOMA-Script Pakets ist.
%
% Dieses Werk darf nach den Bedingungen der LaTeX Project Public Lizenz,
% Version 1.3c, verteilt und/oder veraendert werden.
% Die neuste Version dieser Lizenz ist
%   http://www.latex-project.org/lppl.txt
% und Version 1.3c ist Teil aller Verteilungen von LaTeX
% Version 2005/12/01 oder spaeter und dieses Werks.
%
% Dieses Werk hat den LPPL-Verwaltungs-Status "author-maintained"
% (allein durch den Autor verwaltet).
%
% Der Aktuelle Verwalter und Autor dieses Werkes ist Markus Kohm.
% 
% Dieses Werk besteht aus den in manifest.txt aufgefuehrten Dateien.
% ======================================================================
% \fi^^A meta-comment
%
% \CharacterTable
%  {Upper-case    \A\B\C\D\E\F\G\H\I\J\K\L\M\N\O\P\Q\R\S\T\U\V\W\X\Y\Z
%   Lower-case    \a\b\c\d\e\f\g\h\i\j\k\l\m\n\o\p\q\r\s\t\u\v\w\x\y\z
%   Digits        \0\1\2\3\4\5\6\7\8\9
%   Exclamation   \!     Double quote  \"     Hash (number) \#
%   Dollar        \$     Percent       \%     Ampersand     \&
%   Acute accent  \'     Left paren    \(     Right paren   \)
%   Asterisk      \*     Plus          \+     Comma         \,
%   Minus         \-     Point         \.     Solidus       \/
%   Colon         \:     Semicolon     \;     Less than     \<
%   Equals        \=     Greater than  \>     Question mark \?
%   Commercial at \@     Left bracket  \[     Backslash     \\
%   Right bracket \]     Circumflex    \^     Underscore    \_
%   Grave accent  \`     Left brace    \{     Vertical bar  \|
%   Right brace   \}     Tilde         \~}
%
% \iffalse
%%% From File: $Id$
%<*dtx>
\ifx\ProvidesFile\undefined\def\ProvidesFile#1[#2]{}\fi
\begingroup
  \def\filedate$#1: #2-#3-#4 #5${\gdef\filedate{#2/#3/#4}}
  \filedate$Date$
  \def\filerevision$#1: #2 ${\gdef\filerevision{r#2}}
  \filerevision$Revision$
  \edef\reserved@a{%
    \noexpand\endgroup
    \noexpand\ProvidesFile{scrlfile-patcholdlatex.dtx}%
                          [\filedate\space\filerevision\space
                           KOMA-Script package source
  }%
\reserved@a
%</dtx>
%<package>\ProvidesPackage{scrlfile-patcholdlatex}[%
%!KOMAScriptVersion
%<package>  package 
  (patching old LaTeX kernels)]
%<*dtx>
\ifx\documentclass\undefined
  \input scrdocstrip.tex
  \@@input scrkernel-version.dtx
  \@@input scrstrip.inc
  \KOMAdefVariable{COPYRIGHTFROM}{2002}
  \generate{\usepreamble\defaultpreamble
    \file{scrlfile-patcholdlatex.sty}{%
      \from{scrlfile-patcholdlatex.dtx}{package}%
    }%
  }
  \@@input scrstrop.inc
\else
  \let\endbatchfile\relax
\fi
\endbatchfile
\documentclass[parskip=half-]{scrdoc}
\usepackage[main=english]{babel}
\CodelineIndex
\RecordChanges
\GetFileInfo{scrlfile-patcholdlatex.dtx}
\title{\KOMAScript{} \partname\ \texttt{\filename}%
  \thanks{This file is revision \fileversion\ of file \texttt{\filename}.}}  
\date{\filedate}
\author{Markus Kohm\thanks{mailto:komascript@gmx.info}}

\begin{document}
  \maketitle
  \begin{abstract}
    This package provides hooks before and after loading files, packages or
    classes. It also provides a hook after the last \cs{clearpage} of the
    document. It allowes to replace files, packages and classes by other
    files, packages and classes. It is inteded to be used by package and class
    authors but may also be used by \LaTeX{} users.
  \end{abstract}

  \DocInput{\filename}
\end{document}
%</dtx>
% \fi^^A  meta-comment
%
% \changes{v2.95}{2002/06/11}{First version split from hugh scrclass.dtx}
% \changes{v3.24}{2017/05/06}{standalone manual removed}
% \changes{v3.32}{2020/09/02}{new (sub)-package}
%
% \section{The User Manual of \textsf{scrlfile-patcholdlatex}}
%
% \textsf{scrlfile-patcholdlatex} implements the part of \textsf{scrlfile}
% that patches old \LaTeX{} kernels.
%
% There isn't any user manual for the user level \LaTeXe{} commands in this
% file. Please see the manual of \textsf{scrlfile} for more information about
% \textsf{scrlfile-patcholdlatex}.
%
% \StopEventually{\PrintIndex\PrintChanges}
%
% \section{Implementation of \textsf{scrlfile-patcholdlatex}}
%
%
% \subsection{Patching \LaTeX{} Kernel Macros}
%
% \begin{macro}{\InputIfFileExists}
% \changes{v2.96}{2006/07/19}{replacement code added}%^^A
% \LaTeX{} uses this macro for loading files with
% \cs{input}\marg{filename}, \cs{include}, \cs{LoadClass},
% \cs{LoadClassWithOptions}, \cs{documentclass}, \cs{documentstyle},
% \cs{RequirePackage}, \cs{usepackage}. Only
% \cs{input}~\meta{filename} loads \meta{filename} without using
% \cs{InputIfFileExists}.
%
% Because of this, it's very easy to add stuff before and after
% loading a file. We simply change this macro and add to
% macros\,---\,one before loading the file and one after loading the
% file. But before we redefine the macro, we test, if the macro was
% defined like we expected. \LaTeX{} will warn, if not.
% \changes{v3.27}{2019/10/06}{\LaTeX{} 2019-10-01 changed the definition}%^^A
%
% \changes{v3.28}{2019/11/21}{support for \textsf{memoir}}%^^A
% With \textsf{memoir} their is a special case. It also redefines
% \cs{InputIfFileExists}. As shown in the source of the class it does so
% because \textsf{scrlfile} does it. However it does it not compatible with
% \textsf{scrlfile} and it does not emulate \textsf{scrlfile} as it do for
% other packages. So if you try to use \textsf{scrlfile} with memoir,
% \textsf{scrbase} has to take care to \textsf{memoir}. And in oposit to
% \textsf{scrlfile} \textsf{memoir} does not always fix the argument order bug
% of \LaTeX{} but do so only for the implementation of \LaTeX{} 2019-10-01.
% So in case of \textsf{memoir} we just patch every version of it without
% testing.
%    \begin{macrocode}
\@ifclassloaded{memoir}{%
  \PackageInfo{scrlfile}{class `memoir' detected.\MessageBreak
    Using a compatible redefinition of \string\InputIfFileExists,\MessageBreak
    but always robust and with the fix of the\MessageBreak
    argument order bug even for old versions of the\MessageBreak
    LaTeX kernel%
  }%
}{%
  \long\def\reserved@a#1#2{%
    \IfFileExists{#1}%
      {#2\@addtofilelist{#1}\@@input \@filef@und}}%
  \ifx\reserved@a\InputIfFileExists\else
    \expandafter\CheckCommand\expandafter*\expandafter\InputIfFileExists
    \expandafter{%
      \expandafter\protect\csname InputIfFileExists \endcsname
    }%
    \expandafter\CheckCommand \csname InputIfFileExists \endcsname[2]{%
      \IfFileExists {#1}{%
        \expandafter \@swaptwoargs \expandafter {\@filef@und }%
        {#2\@addtofilelist {#1}\@@input }%
      }%
    }%
  \fi
}
%    \end{macrocode}
% Now we can redefine the macro like described above:
%    \begin{macrocode}
\expandafter\renewcommand\expandafter*\expandafter\InputIfFileExists
\expandafter{\expandafter
  \protect\csname InputIfFileExists \endcsname}%
\expandafter\long\expandafter\def\csname InputIfFileExists \endcsname#1#2{%
  \begingroup\expandafter\expandafter\expandafter\endgroup
  \expandafter\ifx\csname #1-@alias\endcsname\relax
    \expandafter\@secondoftwo
  \else
    \scr@replacefile@msg{\csname #1-@alias\endcsname}{#1}%
    \expandafter\@firstoftwo
  \fi
  {%
    \expandafter\InputIfFileExists\expandafter{\csname
      #1-@alias\endcsname}{#2}%
  }%
  {\IfFileExists{#1}{%
      \expandafter\scr@input@withhook\expandafter{\@filef@und}{#1}{#2}%
    }}%
}
%    \end{macrocode}
% \begin{macro}{\scr@input@withhook}
% \changes{v3.27}{2019/03/28}{new (internal)}%^^A
% \changes{v3.28}{2019/11/21}{support for \textsf{memoir}}%^^A
% We need a new internal, because of expanding \cs{@file@und} before inputting
% the file.
%    \begin{macrocode}
\@ifclassloaded{memoir}{%
  \newcommand{\scr@input@withhook}[3]{%
    \scr@load@hook{before}{#2}%
    #3\@addtofilelist{#2}%
    \@ifundefined{m@matbeginf}{}{\m@matbeginf{#2}}%
    \@@input #1%
    \@ifundefined{m@matendf}{}{\m@matendf{#2}}%
    \@ifundefined{killm@matf}{}{\killm@matf{#2}}%
    \scr@load@hook{after}{#2}%
  }%
}{%
  \newcommand{\scr@input@withhook}[3]{%
    \scr@load@hook{before}{#2}%
    #3\@addtofilelist{#2}%
    \@@input #1%
    \scr@load@hook{after}{#2}%
  }%
}
%    \end{macrocode}
% \end{macro}
% \end{macro}
%
% \begin{macro}{\@loadwithoptions}
% \changes{v2.96}{2006/07/19}{new redefinition}
% \begin{macro}{\scr@loadwithoptions}
% \changes{v2.96}{2006/07/19}{new helper macro}
% \begin{macro}{\scr@@loadwithoptions}
% \changes{v2.97a}{2007/03/07}{new helper macro}
% \LaTeX{} uses this macro for loading classes and packages using
% \cs{LoadClassWithOptions} or \cs{RequirePackageWithOptions}. We simply say
% now: if the class or package should be replaced by another, try it again
% with the replacement name. The stack is used. This will result in an
% error if the replacements are nested too deep. This should be, because it
% will also result in an error using ring replacements.
%    \begin{macrocode}
\newcommand*{\scr@@loadwithoptions}{}
\newcommand*{\scr@loadwithoptions}{}
\let\scr@loadwithoptions\@loadwithoptions
\renewcommand*{\@loadwithoptions}[3]{%
  \begingroup\expandafter\expandafter\expandafter\endgroup
  \expandafter\ifx\csname #3.#1-@aliasname\endcsname\relax
    \def\scr@@loadwithoptions{\scr@loadwithoptions{#1}{#2}{#3}}%
  \else
    \scr@replacefile@msg{\csname #3.#1-@aliasname\endcsname.#1}{#3.#1}% 
    \def\scr@@loadwithoptions{%
      \@loadwithoptions{#1}{#2}{\csname #3.#1-@aliasname\endcsname}%
    }%
  \fi
  \scr@@loadwithoptions
}
%    \end{macrocode}
% \end{macro}
% \end{macro}
% \end{macro}
%
% \begin{macro}{\@onefilewithoption}
% \changes{v2.96}{2006/07/19}{new redefinition}
% \begin{macro}{\scr@onefilewithoption}
% \changes{v2.96}{2006/07/19}{new helper macro}
% \begin{macro}{\scr@@onefilewithoption}
% \changes{v2.98}{2007/12/18}{new helper macro}
% \LaTeX{} uses this macro for loading classes and packages using
% \cs{LoadClass}, \cs{documentclass}, \cs{RequirePackage}, and
% \cs{usepackage}. We simply say now: if the class or package should be
% replaced by another, try it again with the replacement name. The stack is
% used. This will result in an error if the replacements are nested too
% deep. This should be, because it will also result in an error using ring
% replacements.
%    \begin{macrocode}
\newcommand*{\scr@onefilewithoptions}{}
\let\scr@onefilewithoptions\@onefilewithoptions
\def\@onefilewithoptions#1[#2][#3]#4{%
  \begingroup\expandafter\expandafter\expandafter\endgroup
  \expandafter\ifx\csname #1.#4-@aliasname\endcsname\relax
%    \end{macrocode}
% Do this ridiculous definition and call outside of \cs{ifx}\dots\cs{fi}
% because of some errornous packages like some versions of
% \textsf{breakurl}. This does not avoid an error, but the message about
% unclosed \cs{if} will be outside of \textsf{scrlfile}.
%    \begin{macrocode}
    \def\scr@@onefilewithoptions{\scr@onefilewithoptions{#1}}%
  \else
    \scr@replacefile@msg{\csname #1.#4-@aliasname\endcsname.#4}{#1.#4}%
    \edef\scr@@onefilewithoptions{%
      \noexpand\@onefilewithoptions{\csname #1.#4-@aliasname\endcsname}}%
  \fi 
  \scr@@onefilewithoptions[{#2}][{#3}]{#4}%
  \scr@load@hook{lateafter}{#1.#4}%
}
%    \end{macrocode}
% \end{macro}
% \end{macro}
% \end{macro}
%
%  \begin{macro}{\scr@replacefile@msg}
% \changes{v2.96}{2006/07/19}{new helper command}
% \changes{v2.98}{2007/12/18}{spellfix}
% Last but not least \textsf{scrlfile} will write a message to the log file at
% every replacement.
%    \begin{macrocode}
\newcommand*{\scr@replacefile@msg}[2]{%
  \PackageInfo{scrlfile}{inputing `#1' instead of `#2'}%
}
%    \end{macrocode}
%  \end{macro}
%
% \begin{macro}{\scr@load@hook}
% \changes{v3.27a}{2019/10/16}{workaround for strange handling of
%   \cs{AtEndOfPackage} hook in \cs{@onefilewithoptions}}%^^A
% This macro simply calls a load-file-hook, either the \emph{before}
% or the \emph{after} one, if the hook exists. After calling the hook
% it will be deleted. First argument is the type of hook
% (\meta{before} or \meta{after}), second argument is a filename. 
%    \begin{macrocode}
\newcommand*{\scr@load@hook}[2]{%
%    \end{macrocode}
% \begin{macro}{\scrlfile@cmp@hook}
% \changes{v3.27a}{2019/10/23}{local macros renamed}%^^A
% \begin{macro}{\scrlfile@hookname}
% \changes{v3.27a}{2019/10/23}{local macros renamed}%^^A
% \changes{v3.29}{2020/06/01}{do the class and package load hooks only in the
%   document preamble}%^^A --> https://komascript.de/node/2274
% Workaround for strange handling of the \cs{AtEndOfPackage} hook in
% \cs{@onefilewithoptions} (becomes \cs{@empty}, if the package already has
% been loaded and is not loaded again). This workaround is not nice but
% smart. Before loading a file it defines a marker macro and undefines it
% later. To decide whether to define or remove the marker, two local macros
% are used \cs{scrlfile@hookname} and \cs{scrlfile@cmp@hook}. The first has
% the name of the current hook from the first argument. The other is a compare
% string. The first one is also valid while processing the hook, but only in
% the document preamble. The names are very unique, so we do not reserve them
% using \cs{newcommand} outside. The name of the marker is less
% unique. Nevertheless it is not reserved.
% Note: For recursive loading of files this does not work. But this also does
% not matter, because it is needed only for packages and classes, which are
% never loaded recursive. For the same reason from \KOMAScript{} 3.29 this is
% only used inside the document preamble. After the document preamble it can
% be dangerous, e.g., inside a |tabular| or |array|.
%    \begin{macrocode}
  \ifx\@begindocumenthook\@undefined\else
    \edef\scrlfile@hookname{#1}%
    \def\scrlfile@cmp@hook{before}%
    \ifx\scrlfile@hookname\scrlfile@cmp@hook
      \expandafter\def\csname #2-l@ading\endcsname{#2}%
    \fi
    \let\scrlfile@cmp@hook\@undefined
  \fi
  \@ifundefined{#2-@#1}{}{%
    \@nameuse{#2-@#1}%
    \expandafter\let\csname #2-@#1\endcsname\relax
  }%
%    \end{macrocode}
% The hook could destroy \cs{scrlfile@hookname} so define it again:
%    \begin{macrocode}
  \ifx\@begindocumenthook\@undefined\else
    \edef\scrlfile@hookname{#1}%
    \def\scrlfile@cmp@hook{after}%
    \ifx\scrlfile@hookname\scrlfile@cmp@hook
      \expandafter\let\csname #2-l@ading\endcsname\@undefined
    \fi
    \let\scrlfile@cmp@hook\@undefined
    \let\scrlfile@hookname\@undefined
  \fi
}
%    \end{macrocode}
%  \end{macro}
%  \end{macro}
%  \end{macro}
%
% \begin{macro}{\scr@set@load@hook}
% This command adds stuff to a load-file-hook, either the
% \emph{before} or the \emph{after} one. First argument is the type of
% hook (\meta{before} or \meta{after}), second is a filename. There's
% a third argument not parsed by the macro itself but
% \cs{g@addto@macro}. This third argument is the stuff, which will be
% added to the hook. Because of using \cs{g@addto@macro} the hook must
% be defined. So we define it, if it isn't.
% TODO: Handling/ignoring an optional argument to make it more compatible with
% \LaTeX's new \cs{AddToHook}.
%    \begin{macrocode}
\newcommand*{\scr@set@load@hook}[2]{%
  \@ifundefined{#2-@#1}{\@namedef{#2-@#1}{}}{}%
  \expandafter\g@addto@macro\csname #2-@#1\endcsname
}
%    \end{macrocode}
% \end{macro}
%
%
% \section{Before and After Commands}
%
% \begin{macro}{\BeforeFile}
% \begin{macro}{\AfterFile}
% Just add code to the \emph{before} or \emph{after} hook of a file. The
% arguments are:
% \begin{description}
% \item[\meta{file name}:] mandatory argument with the file name as given by
%   \cs{InputIfFileExists}.
% \item[\meta{code}:] this argument is not processed by \cs{BeforeFile} or
%   \cs{AfterFile} but the internal \cs{scr@set@load@hook}. It is the code
%   that should be added to either the \emph{before} or the \emph{after} hook.
% \end{description}
%    \begin{macrocode}
\newcommand*{\BeforeFile}{\scr@set@load@hook{before}}
\newcommand*{\AfterFile}{\scr@set@load@hook{after}}
%    \end{macrocode}
% \end{macro}%^^A \AfterFile
% \end{macro}%^^A \BeforeFile
%
% \begin{macro}{\BeforeClass}
% \changes{v3.20}{2016/04/12}{\cs{@ifnextchar} replaced by
%   \cs{kernel@ifnextchar}}%^^A
% \changes{v3.20}{2016/04/12}{\cs{@ifstar} durch \cs{kernel@ifstar}
%   ersetzt}%^^A
% This is quite simple. It only has to add code to the \emph{before} hook
% already known by \cs{BeforeFile}. The arguments are:
% \begin{description}
% \item[\meta{package name}:] mandatory argument.
% \item[\meta{code}:] this argument is processed not by \cs{BeforeClass} but
%   by the internal \cs{scr@set@load@hook} (see above), that will add the code
%   to the \emph{before} hook of file \meta{code}\texttt{.cls} (or whatever the
%   package extension is).
% \end{description}
%    \begin{macrocode}
\newcommand*{\BeforeClass}[1]{%
  \scr@set@load@hook{before}{#1.\@clsextension}%
}
%    \end{macrocode}
% \end{macro}%^^A \BeforeClass
%
% \begin{macro}{\AfterClass}
% \changes{v3.20}{2016/04/12}{\cs{@ifnextchar} replaced by
%   \cs{kernel@ifnextchar}}
% \changes{v3.20}{2016/04/12}{\cs{@ifstar} replaced by \cs{kernel@ifstar}}
% \changes{v3.32}{2020/09/02}{plus and exclamation mark variants removed}
% \begin{macro}{\AfterAtEndOfClass}
% \changes{v3.32}{2020/09/02}{star variant replaces the exclamation mark
%   variant of \cs{AfterClass}}
% These are more complicated. There are four forms:
% \begin{description}
% \item[\cs{AfterClass}] corresponds to \cs{AfterFile}. So the code will be
%   added to the usual \emph{after} hook and the code is exexuted before code
%   added by the class itself using \cs{AtEndOfClass}. Argument procession
%   will be done by \cs{@AfterClass}.
% \item[\cs{AfterAtEndOfClass}] This uses the \emph{lateafter} hook, which
%   is executed after the class has been loaded completely inclusive the
%   execution of \cs{AtEndOfClass} code. However, the code will never be
%   executed, if the class will not be loaded (again). But it does nothing,
%   if the class has already been loaded. The arguments are:
%   \begin{description}
%   \item[\meta{class name}] mandatory argument
%   \item[\meta{code}:] this argument is processed not by \cs{BeforeClass}
%     but by the internal \cs{scr@set@load@hook} (see above), that will add
%     the code to the \emph{lateafter} hook of file \meta{code}\texttt{.cls}
%     (or whatever the class extension is), or by \cs{@gobble}.
%   \end{description}
% \item[\cs{AfterClass*}] like \cs{AfterClass}, but if loading of the
%   class has already began, the code will be executed immediately. So, if
%   the command is used inside the class it is simply
%   \cs{@firstofone}. Argument procession will be done by \cs{@sAfterClass}.
% \item[\cs{AfterAtEndOfClass*}] like \cs{AfterClass*}, but the code will be
%   executed immediately only if the class already has been loaded
%   completely. Inside \cs{AtEndOfClass} the code is executed after every all
%   the \cs{AtEndOfClass} code. Argument procession will be done by
%   \cs{@xAfterClass}.
% \end{description}
%    \begin{macrocode}
\newcommand*{\AfterClass}{%
  \kernel@ifstar \@sAfterClass\@AfterClass
}
%    \end{macrocode}
% \begin{macro}{\@AfterClass}
% \cs{AfterClass}\marg{class name}\marg{code}.
% It uses \cs{scr@set@load@hook} to add \meta{code} to the
% \emph{after} hook. The second argument will be processed by
% \cs{scr@set@load@hook}.
%    \begin{macrocode}
\newcommand*{\@AfterClass}[1]{%
  \scr@set@load@hook{after}{#1.\@clsextension}%
}
%    \end{macrocode}
% \end{macro}%^^A \@AfterClass
% \begin{macro}{\@sAfterClass}
% \changes{v3.27a}{2019/10/16}{group removed}%^^A
% This is the internal implementation of:
% \cs{AfterClass*}\marg{class name}\marg{code}.
% It tests, whether or not \LaTeX{} thinks that the class
% has already been loaded, but this only means whether or not the class
% loading has already been started. If so, it immediately executes the next
% argument. If not, is uses \cs{@AfterClass} to execute the \meta{code}
% after loading the class.
%    \begin{macrocode}
\newcommand*{\@sAfterClass}[1]{%
  \@ifclassloaded{#1}{%
    \@secondoftwo
  }{%
    \@AfterClass
  }%
  {#1}%
}
%    \end{macrocode}
% \end{macro}%^^A \@sAfterClass
%    \begin{macrocode}
\newcommand*{\AfterAtEndOfClass}{%
  \kernel@ifstar \@sAfterAtEndOfClass\@AfterAtEndOfClass
}
%    \end{macrocode}
% \begin{macro}{\@AfterAtEndOfClass}
% \changes{v3.32}{2020/09/02}{new (internal)}
% Note that \cs{AfterAtEndOfClass} executed the \meta{code} after each try
% to load a class, even if the class has already been loaded and you try
% to load it again.
%    \begin{macrocode}
\newcommand*{\@AfterAtEndOfClass}[1]{%
  \scr@set@load@hook{lateafter}{#1.\@clsextension}%
}
%    \end{macrocode}
% \begin{macro}{\@sAfterAtEndOfClass}
% \changes{v3.09}{2011/03/09}{new (internal)}%^^A
% \changes{v3.27}{2019/08/26}{using inside class fixed}%^^A
% \changes{v3.27a}{2019/10/15}{renamed from \cs{@@@AfterClass}}%^^A
% \changes{v3.27a}{2019/10/16}{usage of \emph{l@ading} marker}%^^A
% \changes{v3.27a}{2019/10/16}{group removed}%^^A
% \changes{v3.32}{2020/09/02}{renamed from \cs{@xAfterClass}}
% This is the internal implementation of:
% \cs{AfterClass!}\marg{class name}\marg{code}. Note, that the exclamation
% mark has not already read. So the arguments are:
% \begin{description}
% \item[\texttt{!}:] static exclamation mark, to be ignored.
% \item[\meta{class name}] mandatory argument
% \item[\meta{code}:] this argument is processed not by \cs{BeforeClass} but
%   by the internal \cs{@sAfterClass} or \cs{@AfterClass} (see above).
% \end{description}
% There are two differences to \cs{@pAfterClass}:
% \begin{enumerate}
% \item In case of \cs{\meta{class name}.cls-l@ading} is defined (and not
%   \cs{relax}), while loading a class, it used \cs{AfterAtEndOfClass}
%   instead of \cs{AfterClass} to add \meta{code} to the \emph{lateafter}
%   hook instead of the \emph{after} hook.
% \item In case of \cs{\meta{class name}.cls-l@ading} is undefined (or
%   \cs{relax}) it also has to test, whether or not the class has already
%   been loaded. If so, it uses \cs{@sAfterClass} to execute the \meta{code}
%   immediately, otherwise it uses \cs{AfterAtEndOfClass} to add the
%   \meta{code} to the \emph{lateafter} hook.
% \end{enumerate}
% So it is more like (the not existing) \cs{AfterAtEndOfClass+} than
% \cs{AfterAtEndOfClass*}.
%    \begin{macrocode}
\newcommand*{\@sAfterAtEndOfClass}[1]{%
  \@ifundefined{#1.\@clsextension-l@ading}{%
    \@ifclassloaded{#1}{%
      \@secondoftwo
    }{%
      \@AfterAtEndOfClass
    }%
  }{%
    \@AfterAtEndOfClass
  }%
  {#1}%
}
%    \end{macrocode}
% \end{macro}%^^A \@sAfterAtEndOfClass
% \end{macro}%^^A \AfterAtEndOfClass
% \end{macro}%^^A \AfterClass
%
% \begin{macro}{\BeforePackage}
% This is quite simple. It only has to add code to the \emph{before} hook
% already known by \cs{BeforeFile}. The arguments are:
% \begin{description}
% \item[\meta{package name}:] mandatory argument.
% \item[\meta{code}:] this argument is processed not by \cs{BeforePackage} but
%   by the internal \cs{scr@set@load@hook} (see above), that will add the code
%   to the \emph{before} hook of file \meta{code}\texttt{.sty} (or whatever the
%   package extension is).
% \end{description}
%    \begin{macrocode}
\newcommand*{\BeforePackage}[1]{%
  \scr@set@load@hook{before}{#1.\@pkgextension}%
}
%    \end{macrocode}
% \end{macro}%^^A \BeforePackage
%
% \begin{macro}{\AfterPackage}
% \changes{v3.20}{2016/04/12}{\cs{@ifnextchar} replaced by
%   \cs{kernel@ifnextchar}}%^^A
% \changes{v3.20}{2016/04/12}{\cs{@ifstar} durch \cs{kernel@ifstar}
%   ersetzt}%^^A
% \begin{macro}{\AfterAtEndOfPackage}
% These are more complicated. There are four forms:
% \begin{description}
% \item[\cs{AfterPackage}] corresponds to \cs{AfterFile}. So the code will be
%   added to the usual \emph{after} hook and the code is exexuted before code
%   added by the package itself using \cs{AtEndOfPackage}. Argument procession
%   will be done by \cs{@AfterPackage}.
% \item[\cs{AfterAtEndOfPackage}] This uses the \emph{lateafter} hook, which
%   is executed after the package has been loaded completely inclusive the
%   execution of \cs{AtEndOfPackage} code. However, the code will never be
%   executed, if the package will not be loaded (again). But it does nothing,
%   if the package has already been loaded. The arguments are:
%   \begin{description}
%   \item[\meta{package name}] mandatory argument
%   \item[\meta{code}:] this argument is processed not by \cs{BeforePackage}
%     but by the internal \cs{scr@set@loaf@hook} (see above), that will add
%     the code to the \emph{lateafter} hook of file \meta{code}\texttt{.sty}
%     (or whatever the package extension is), or by \cs{@gobble}.
%   \end{description}
% \item[\cs{AfterPackage*}] like \cs{AfterPackage}, but if loading of the
%   package has already began, the code will be executed immediately. So, if
%   the command is used inside the package it is simply
%   \cs{@firstofone}..
% \item[\cs{AfterAtEndOfPackage*}] like \cs{AfterPackage*}, but the code will
%   be executed immediately only if the package already has been loaded
%   completely. Inside \cs{AtEndOfPackage} the code is executed after every
%   all the \cs{AtEndOfPackage} code.
% \end{description}
%    \begin{macrocode}
\newcommand*{\AfterPackage}{%
  \kernel@ifstar \@sAfterPackage\@AfterPackage
}
%    \end{macrocode}
% \begin{macro}{\@AfterPackage}
% This is the internal implementation of:
% \cs{AfterPackage}\marg{package name}\marg{code}.
% It uses \cs{scr@set@load@hook} to add \meta{code} to the
% \emph{after} hook. The second argument will be processed by
% \cs{scr@set@load@hook}.
%    \begin{macrocode}
\newcommand*{\@AfterPackage}[1]{%
  \scr@set@load@hook{after}{#1.\@pkgextension}%
}
%    \end{macrocode}
% \end{macro}%^^A \@AfterPackage
% \begin{macro}{\@sAfterPackage}
% \changes{v3.27a}{2019/10/16}{group removed}%^^A
% This is the internal implementation of:
% \cs{AfterPackage*}\marg{package name}\marg{code}.
% It tests, whether or not \LaTeX{} thinks that the package
% has already been loaded, but this only means whether or not the package
% loading has already been started. If so, it immediately executes the next
% argument. If not, is uses \cs{@AfterPackage} to execute the \meta{code}
% after loading the package.
%    \begin{macrocode}
\newcommand*{\@sAfterPackage}[1]{%
  \@ifpackageloaded{#1}{%
    \@secondoftwo
  }{%
    \@AfterPackage
  }%
  {#1}%
}
%    \end{macrocode}
% \end{macro}%^^A \@sAfterPackage
% \begin{macrocode}
\newcommand*{\AfterAtEndOfPackage}{%
  \kernel@ifstar \@sAfterAtEndOfPackage\@AfterAtEndOfPackage
}
% \end{macrocode}
% \begin{macro}{\@sAfterAtEndOfPackage}
% \changes{v3.09}{2011/03/09}{new (internal)
% \changes{v3.27}{2019/08/26}{using inside package fixed}
% \changes{v3.27a}{2019/10/15}{renamed from \cs{@@@AfterPackage}
% \changes{v3.27a}{2019/10/16}{usage of \emph{l@ading} marker}
% \changes{v3.27a}{2019/10/16}{group removed}
% \changes{v3.32}{2020/09/02}{renamed from \@xAfterPackage}
% This is the internal implementation of:
% \cs{AfterAtEndOfPackage*}\marg{package name}\marg{code}.
%    \begin{macrocode}
\newcommand*{\@sAfterAtEndOfPackage}[1]{%
  \@ifundefined{#1.\@pkgextension-l@ading}{%
    \@ifpackageloaded{#1}{%
      \@secondoftwo
    }{%
      \@AfterAtEndOfPackage
    }%
  }{%
    \@AfterAtEndOfPackage
  }%
  {#1}%
}
%    \end{macrocode}
% \end{macro}%^^A \@sAfterAtEndOfPackage
% \begin{macro}{\@AfterAtEndOfPackage}
% \changes{v3.32}{2020/09/02}{new (internal)}
% \begin{macro}{\@@AfterAtEndOfPackage}
% \changes{v3.27a}{2019/10/15}{new (internal)}
% \changes{v3.27a}{2019/10/16}{group removed}
% \changes{v3.32}{2020/09/02}{renamed from \cs{@AfterAtEndOfPackage}}
% Note that until 3.27a \cs{AfterAtEndOfPackage} executed the \meta{code}
% after each try to load a package, even if the package has already been
% loaded and you've try to load it again. Since 3.27a this is the
% functionality of \cs{@AfterAtEndOfPackage}. \cs{AfterAtEndOfPackage} now
% does not do anything for already loaded packages or while loading the
% package.
%    \begin{macrocode}
\newcommand*{\@AfterAtEndOfPackage}[1]{%
  \@ifpackageloaded{#1}{%
    \@ifundefined{#1.\@pkgextension-l@ading}{%
      \@gobbletwo
    }{%
      \@AfterAtEndOfPackage
    }%
  }{%
    \@AfterAtEndOfPackage
  }%
  {#1}%  
}
\newcommand*{\@@AfterAtEndOfPackage}[1]{%
  \scr@set@load@hook{lateafter}{#1.\@pkgextension}%
}
%    \end{macrocode}
% \end{macro}%^^A \@@AfterAtEndOfPackage
% \end{macro}%^^A \@AfterEndOfPackage
% \end{macro}%^^A \AfterAtEndOfPackage
% \end{macro}%^^A \AfterPackage
%  
%
% \section{File Substitution}
%
% \begin{macro}{\ReplaceInput}
% \changes{v2.96}{2006/07/19}{new command}
% \changes{v3.12}{2013/11/15}{implemented}
% Defining the replacement by definition of \cs{\#1-@alias}.
%    \begin{macrocode}
\newcommand*{\ReplaceInput}[2]{%
  \expandafter\edef\csname #1-@alias\endcsname{#2}%
}
%    \end{macrocode}
% \begin{macro}{\ReplaceFile}
% \changes{v2.96}{2006/07/19}{new command}
% \changes{v3.12}{2013/11/15}{throw error}
% Sorry, but I've implemented the wrong command some time ago.
%    \begin{macrocode}
\newcommand*{\ReplaceFile}{%
  \PackageError{scrlfile}{%
    usage of not documented command `\string\ReplaceFile'%
  }{%
    Command `\string\ReplaceFile' has been implemented accidentally in
    KOMA-Script 2.96\MessageBreak
    and replaced by `\string\ReplaceInput' in KOMA-Script 3.12.\MessageBreak
    You should replace `\string\ReplaceFile' by `\string\ReplaceInput' to
    avoid this error\MessageBreak
    message%
  }%
  \ReplaceInput
}
%    \end{macrocode}
% \end{macro}
% \end{macro}
%
% \begin{macro}{\UnReplaceInput}
% \changes{v3.12}{2013/11/15}{new command}
% Do not longer replace a file.
%    \begin{macrocode}
\newcommand*{\UnReplaceInput}[1]{%
  \ifcsname #1-@alias\endcsname
    \expandafter\let\csname #1-@alias\endcsname\relax
  \fi
}
%    \end{macrocode}
% \end{macro}
%
% \begin{macro}{\ReplacePackage}
% \changes{v2.96}{2006/07/19}{new command}
% \begin{macro}{\ReplaceClass}
% \changes{v2.96}{2006/07/19}{new command}
% \begin{macro}{\@replacefilewithext}
% \changes{v2.96}{2006/07/19}{new helper command}
% \changes{v3.32}{2020/09/09}{works global}
% Note that replacing a package or class will not replace the input of the
% file of the package or class but only the input of package or class.
%    \begin{macrocode}
\newcommand*{\@replacefilewithext}[3]{%
  \expandafter\xdef\csname #1.#3-@aliasname\endcsname{#2}%
}
\newcommand*{\ReplacePackage}[2]{%
  \@replacefilewithext{#1}{#2}\@pkgextension
}
\newcommand*{\ReplaceClass}[2]{%
  \@replacefilewithext{#1}{#2}\@clsextension
}
%    \end{macrocode}
% \end{macro}
% \end{macro}
% \end{macro}
%
% \begin{macro}{\UnReplacePackage}
% \changes{v3.12}{2013/11/15}{new command}
% \begin{macro}{\UnReplaceClass}
% \changes{v3.12}{2013/11/15}{new command}
% \begin{macro}{\@unreplacefilewithext}
% \changes{v3.12}{2013/11/15}{new helper command}%^^A
% \changes{v3.32}{2020/09/09}{works global}
% Do not nolonger replace a class or package.
%    \begin{macrocode}
\newcommand*{\@unreplacefilewithext}[2]{%
  \ifcsname #1.#2-@aliasname\endcsname
    \expandafter\global\expandafter\let\csname #1.#2-@aliasname\endcsname\relax
  \fi
}
\newcommand*{\UnReplacePackage}[1]{%
  \@unreplacefilewithext{#1}\@pkgextension
}
\newcommand*{\UnReplaceClass}[1]{%
  \@unreplacefilewithext{#1}\@clsextension
}
%    \end{macrocode}
% \end{macro}
% \end{macro}
% \end{macro}
%
%
% \section{Prevent Package from Loading}
%
% \begin{macro}{\PreventPackageFromLoading}
% \changes{v3.08}{2010/11/04}{New}
% \changes{v3.11a}{2012/06/13}{write fault error in \cs{MessageBreak} fixed}
% \changes{v3.12}{2012/06/15}{new star version allowes empty argument with
%   info instead of warning}
% \changes{v3.20}{2016/04/12}{\cs{@ifstar} replaced by \cs{kernel@ifstar}}
% \changes{v3.32}{2020/09/09}{works global}
% \begin{macro}{\@PreventPackageFromLoading}
% \changes{v3.12}{2012/06/15}{New (internal)}%^^A
% \changes{v3.20}{2016/04/12}{\cs{@ifnextchar} replaced by
%   \cs{kernel@ifnextchar}}%^^A
% \begin{macro}{\@@PreventPackageFromLoading}
% \changes{v3.12}{2013/11/22}{New (internal)}%^^A
% \changes{v3.12}{2013/11/22}{\cs{usepackage} before \cs{documentclass}
%   fixed}%^^A
% \begin{macro}{\scrlfile@excludedpackages}
% \changes{v3.08}{2010/11/04}{New (internal)}%^^A
% \begin{macro}{\scrlfile@saved@RequirePackage}
% \changes{v3.08}{2010/11/04}{New (internal)}%^^A
%    \begin{macrocode}
\newcommand*{\scrlfile@excludedpackages}{}
\let\scrlfile@excludedpackages\@empty
\newcommand*{\scrlfile@saved@RequirePackage}{}
\newcommand*{\PreventPackageFromLoading}{%
  \kernel@ifstar{\@PreventPackageFromLoading\PackageInfo}%
          {\@PreventPackageFromLoading\PackageWarning}%
}
\newcommand*{\@PreventPackageFromLoading}[1]{%
  \kernel@ifnextchar [%]
    {\@@PreventPackageFromLoading#1}{\@@PreventPackageFromLoading#1[]}%
}
\newcommand*{\@@PreventPackageFromLoading}{}
\def\@@PreventPackageFromLoading#1[#2]#3{%
%    \end{macrocode}
% We do nothing, if the argument was empty:
%    \begin{macrocode}
  \edef\reserved@b{\zap@space#3 \@empty}%
  \ifx \reserved@b\@empty
    #1{scrlfile}{%
      \string\PreventPackageFromLoading\space with empty packages\MessageBreak
      argument ignored%
    }%
  \else
%    \end{macrocode}
% If the argument was not empty we first need to patch \cs{RequirePackage} and
% maybe \cs{usepackage} too.
%    \begin{macrocode}
    \ifx\scrlfile@excludedpackages\@empty
      \global\let\scrlfile@saved@RequirePackage\RequirePackage
      \global\let\RequirePackage\scrlfile@RequirePackage
      \def\reserved@a##1##{%  
        \@latex@error
        {\noexpand \usepackage before \string\documentclass}%
        {\noexpand \usepackage may only appear in the document
          preamble, i.e.,\MessageBreak
          between \noexpand\documentclass and
          \string\begin{document}.}%
        \@gobble}%
      \ifx\reserved@a\usepackage\else
        \ifx\usepackage\scrlfile@saved@RequirePackage\else
          \PackageWarning{scrlfile}{%
            Maybe serious problem: unexpected definition of\MessageBreak
            \string\usepackage
          }%
        \fi
        \global\let\usepackage\RequirePackage
      \fi
    \fi
%    \end{macrocode}
% We have to test, if a package was already loaded. Only not yet loaded
% packages will be added to the exclusion list. Empty arguments will be
% ignored.
% \changes{v3.11a}{2012/06/11}{missing \cs{scrlfile@excludedpackages} at
%   redefinition of \cs{scrlfile@excludedpackages} added}%^^A
% \changes{v3.26}{2018/08/29}{spurious \cs{expandafter} removed}%^^A
%    \begin{macrocode}
    \@for\reserved@a:=\reserved@b\do {%
      \ifx\reserved@a\@empty
        #1{scrlfile}{%
          empty package argument for
          \string\PreventPackageFromLoading\MessageBreak
          ignored%
        }%
      \else
        \expandafter\@ifpackageloaded\expandafter{\reserved@a}{%
          #1{scrlfile}{%
            package `\reserved@a' already loaded.\MessageBreak
            Cannot prevent it from beeing loaded%
          }%
        }{%
          \ifx\scrlfile@excludedpackages\@empty
            \global\let\scrlfile@excludedpackages\reserved@a
          \else
            \xdef\scrlfile@excludedpackages{\scrlfile@excludedpackages,%
              \reserved@a}%
          \fi
          \if\relax\detokenize{#2}\relax\else
            \@ifundefined{scrlfile@exclude@package@\reserved@a @do}{%
              \@namedef{scrlfile@exclude@package@\reserved@a @do}{}%
            }{}%
            \expandafter\gdef
            \csname scrlfile@exclude@package@\reserved@a 
                    @do\expandafter\expandafter\expandafter\endcsname
            \expandafter\expandafter\expandafter{%
              \csname scrlfile@exclude@package@\reserved@a @do\endcsname 
              #2%
            }%
          \fi
        }%
      \fi
    }%
%    \end{macrocode}
% If the exclusion list is still empty (because all packages has been loaded
% already), remove the patch:
%    \begin{macrocode}
    \ifx\scrlfile@excludedpackages\@empty
      \ResetPreventPackageFromLoading
    \fi
  \fi
}
\@onlypreamble\PreventPackageFromLoading
%    \end{macrocode}
% \begin{macro}{\ResetPreventPackageFromLoading}
% \changes{v3.08}{2010/11/04}{New}
% \changes{v3.32}{2020/09/09}{works global}
% \begin{macro}{\StorePreventPackageFromLoading}
% \changes{v3.08}{2010/11/04}{New}
% These two are very simple
%    \begin{macrocode}
\newcommand*{\ResetPreventPackageFromLoading}{%
  \global\let\scrlfile@excludedpackages\@empty
  \ifx\RequirePackage\scrlfile@RequirePackage
    \ifx\usepackage\RequirePackage
      \global\let\usepackage\scrlfile@saved@RequirePackage
    \fi
    \global\let\RequirePackage\scrlfile@saved@RequirePackage
  \fi
}
\@onlypreamble\ResetPreventPackageFromLoading
\newcommand*{\StorePreventPackageFromLoading}[1]{%
  \let#1\scrlfile@excludedpackages
}
\@onlypreamble\StorePreventPackageFromLoading
%    \end{macrocode}
% \begin{macro}{\UnPreventPackageFromLoading}
% \changes{v3.12}{2013/11/22}{New}%^^A
% \changes{v3.20}{2016/04/12}{\cs{@ifstar} durch \cs{kernel@ifstar}
%     ersetzt}%^^A
% \changes{v3.32}{2020/09/09}{works global}
% \begin{macro}{\@UnPreventPackageFromLoading}
% \changes{v3.12}{2013/11/22}{New (internal)}%^^A
% This removes packages from the prevent exclude list. The star variant also
% undefines the instead code.
%    \begin{macrocode}
\newcommand*{\UnPreventPackageFromLoading}{%
  \kernel@ifstar {\@tempswatrue\@UnPreventPackageFromLoading}%
           {\@tempswafalse\@UnPreventPackageFromLoading}%
}
\newcommand*{\@UnPreventPackageFromLoading}[1]{%
  \edef\reserved@b{\zap@space#1 \@empty}%
  \if@tempswa
    \@for\reserved@a:=\reserved@b\do {%
      \ifx\reserved@a\@empty \else
        \ifcsname scrlfile@exclude@package@\reserved@a @do\endcsname
          \expandafter\let
          \csname scrlfile@exclude@package@\reserved@a @do\endcsname
          \undefined
        \fi
      \fi
    }%
  \fi
  \ifcsname scrlfile@excludedpackages\endcsname
    \let\reserved@a\scrlfile@excludedpackages
    \let\scrlfile@excludedpackages\@empty
    \expandafter\@for\expandafter\reserved@c
    \expandafter:\expandafter=\reserved@a
    \do{%
      \ifx\reserved@c\@empty\else
        \@expandtwoargs\in@{,\reserved@c,}{,\reserved@b,}%
        \ifin@
        \else
          \ifx\scrlfile@excludepackages\@empty
            \global\let\scrlfile@excludepackages\reserved@c
          \else
            \xdef\scrlfile@excludedpackages{%
              \scrlfile@excludedpackages,\reserved@c
            }%
          \fi
        \fi
      \fi
    }%
    \ifx\scrlfile@excludedpackages\@empty\ResetPreventPackageFromLoading\fi
  \fi
}
%    \end{macrocode}
% \end{macro}
% \end{macro}
% \begin{macro}{\scrlfile@RequirePackage}
% \changes{v3.08}{2010/11/04}{New (internal)}
% \changes{v3.12}{2013/11/22}{tolerate spaces}
% \changes{v3.12}{2013/11/22}{execute the optional argument of
%     \cs{PreventPackageFromLoading}}
% This macro will replace \cs{RequirePackage} whenever the exlude list is not
% empty
%    \begin{macrocode}
\newcommand*{\scrlfile@RequirePackage}[2][]{%
  \let\reserved@c\@empty
  \let\reserved@d\@empty
  \edef\reserved@b{\zap@space#2 \@empty}%
  \expandafter\@for\expandafter\reserved@b\expandafter:\expandafter=\reserved@b
  \do {%
    \begingroup
      \@tempswatrue
      \@for\reserved@a:=\scrlfile@excludedpackages\do {%
        \ifx\reserved@a\reserved@b
          \@tempswafalse
        \fi
      }%
      \if@tempswa
        \xdef\reserved@c{\reserved@c,\reserved@b}%
      \else
        \ifx\reserved@b\@empty\else
          \PackageInfo{scrlfile}{Excluding package `\reserved@b'\MessageBreak
            from loading}%
          \@ifundefined{scrlfile@exclude@package@\reserved@b @do}{%
          }{%
            \expandafter\g@addto@macro\expandafter\reserved@d
            \csname scrlfile@exclude@package@\reserved@b @do\endcsname
          }%
        \fi
      \fi
    \endgroup
  }%
  \ifx\reserved@c\@empty
    \let\reserved@a\scr@gobbleopt
  \else
    \edef\reserved@a{\noexpand\scrlfile@saved@RequirePackage[#1]{%
        \expandafter\@gobble\reserved@c\@empty}}%
  \fi
  \expandafter\reserved@d
  \reserved@a
}
%    \end{macrocode}
% \end{macro}
% \end{macro}
% \end{macro}
% \end{macro}
% \end{macro}
% \end{macro}
% \end{macro}
% \end{macro}
%
% \begin{macro}{\scr@gobbleopt}
% \changes{v3.08}{2010/11/04}{New (internal)}
% \changes{v3.08a}{2011/02/11}{\cs{@gobbleopt} renamed due to compaibility
%   problems with \texttt{ntheorem}}
% Eat the optional argument if it exists.
%    \begin{macrocode}
\providecommand{\scr@gobbleopt}[1][]{}
%    \end{macrocode}
% \end{macro}
%
% \begin{macro}{\kernel@ifstar}
% \changes{v3.20}{2016/04/12}{New (internal)}
% I hope, the \LaTeX{} kernel will define this command in future. But
% nevertheless, there are \LaTeX{} kernels in real live without this
% command. So I do it again if needed:
%    \begin{macrocode}
\providecommand*\kernel@ifstar[1]{\kernel@ifnextchar*{\@firstoftwo{#1}}}
%    \end{macrocode}
% \end{macro}%^^A \kernel@ifstar
%
%
% \section{Extra Document Hooks}
%
% \begin{macro}{\BeforeClosingMainAux}
% \changes{v2.95}{2002/09/03}{new}
% \begin{macro}{\AfterReadingMainAux}
% \changes{v3.03}{2009/03/25}{new}
% These are completly different. First we have to establish a new hook
% at \cs{end\{document\}}, just after closing the group and after the
% final \cs{clearpage}.
% TODO: Handling/ignoring an optional argument to make them more compatible
% with \LaTeX's new \cs{AddToHook}. 
%    \begin{macrocode}
\AtEndDocument{%
  \let\scr@saved@checkend=\@checkend
  \renewcommand*{\@checkend}[1]{%
    \def\reserved@a{#1}\def\reserved@b{document}%
    \scr@saved@checkend{#1}%
    \ifx\reserved@a\reserved@b
      \let\scr@saved@clearpage=\clearpage
      \renewcommand*{\clearpage}{\scr@saved@clearpage
        \scr@hook@bcma
        \scr@restore@newpage
        \let\clearpage=\scr@saved@clearpage}%
    \fi%
    \let\scr@saved@dofilelist\@dofilelist
    \def\@dofilelist{%
      \scr@hook@acma
      \let\@dofilelist\scr@saved@dofilelist
      \scr@saved@dofilelist
    }%
  }%
}
%    \end{macrocode}
% \begin{macro}{\scr@saved@checkend}
% \changes{v2.95}{2002/09/03}{new (internal)}
% \begin{macro}{\scr@saved@dofilelist}
% \changes{v3.03}{2009/03/25}{new}
% You may see above, that we need save-to macros.
%    \begin{macrocode}
\newcommand*{\scr@saved@checkend}{}
\newcommand*{\scr@saved@dofilelist}{}
%    \end{macrocode}
% \end{macro}
% \end{macro}
% \begin{macro}{\scr@redefine@newpage}
% \changes{v3.00}{2008/07/09}{new (internal)}
%   Some classes (e.g. \textsf{letter}) write output to the aux file.
%   Because of this, redefinition of \cs{newpage} will only be done, if
%   \cs{BeforeClosingMainAux} is used.
% \begin{macro}{\scr@saved@newpage}
% \changes{v3.00}{2008/07/09}{new (internal)}
%   And \cs{newpage} will be restored after the hook.
% \begin{macro}{\scr@restore@newpage}
%    \begin{macrocode}
\newcommand*{\scr@redefine@newpage}{%
  \let\scr@saved@newpage\newpage
  \def\scr@restore@newpage{%
    \renewcommand*{\newpage}{%
      \PackageWarningNoLine{%
        scrlfile%
      }{%
        \string\newpage\space at main aux file!\MessageBreak
        You have output material at the main aux file.\MessageBreak
        Note, that the \string\BeforeClosingMainAux\space commands\MessageBreak
        are processed before the output from aux file%
      }%
      \let\newpage\scr@saved@newpage
      \newpage
    }%
  }%
  \renewcommand*{\newpage}{%
    \PackageError{%
      scrlfile%
    }{%
      \string\newpage\space at \string\BeforeClosingMainAux
    }{%
      You should never use \string\newpage\space or
      \string\clear(double)page or other output material at the\MessageBreak
      argument of \string\BeforeClosingMainAux!\MessageBreak
      If you'll continue, the \string\newpage\space will be ignored to
      avoid corruption of\MessageBreak
      every prior use of \string\BeforeClosingMainAux.
    }%
  }%
}
\newcommand*{\scr@restore@newpage}{}
%    \end{macrocode}
% \end{macro}
% \end{macro}
% \end{macro}
% \begin{macro}{\scr@hook@bcma}
% \changes{v2.95}{2002/09/03}{new (internal)}
% \begin{macro}{\scr@hook@acma}
% \changes{v3.03}{2009/03/25}{new (internal)}
% We also need macros, that saves the commands at the hooks and are
% extendable using \cs{g@addto@macro}.
%    \begin{macrocode}
\newcommand*{\scr@hook@bcma}{}
\newcommand*{\scr@hook@acma}{}
%    \end{macrocode}
%  \end{macro}
%  \end{macro}
% Now we are able to write the user command. Here we also warn, if we detect
% one more \cs{newpage}\,---\,maybe at a \cs{clearpage} or
% \cs{cleardoublepage}.
%    \begin{macrocode}
\newcommand*{\BeforeClosingMainAux}{%
  \ifx\scr@hook@bcma\@empty
    \gdef\scr@hook@bcma{%
      \renewcommand*{\BeforeClosingMainAux}[1]{####1}%
      \let\protected@write\protected@immediate@write
      \scr@redefine@newpage
    }%
  \fi
  \g@addto@macro\scr@hook@bcma%
}
\newcommand*{\AfterReadingMainAux}{%
  \ifx\scr@hook@acma\@empty
    \gdef\scr@hook@acma{%
      \renewcommand*{\AfterReadingMainAux}[1]{####1}%
      \let\protected@write\protected@immediate@write
    }%
  \fi
  \g@addto@macro\scr@hook@acma%
}
%    \end{macrocode}
% \end{macro}
% \end{macro}
%
% \begin{macro}{\protected@immediate@write}
% Like \LaTeX{} kernel's |\protected@write| but using |\immediate\write|. In
% this case it is even not a good idea to protect |\thepage|!
%    \begin{macrocode}
\long\def\protected@immediate@write#1#2#3{%
  \begingroup
    #2%
    \let\protect\@unexpandable@protect
    \edef\reserved@a{\immediate\write#1{#3}}%
    \reserved@a
  \endgroup
  \if@nobreak\ifvmode\nobreak\fi\fi
}
%    \end{macrocode}
% \end{macro}
%
% \Finale
%
% \endinput
% Local Variables:
% mode: doctex
% TeX-master: t
% End:
