% \CheckSum{4850}
% \iffalse meta-comment
% ======================================================================
% scrkernel-sections.dtx
% Copyright (c) Markus Kohm, 2002-2019
%
% This file is part of the LaTeX2e KOMA-Script bundle.
%
% This work may be distributed and/or modified under the conditions of
% the LaTeX Project Public License, version 1.3c of the license.
% The latest version of this license is in
%   http://www.latex-project.org/lppl.txt
% and version 1.3c or later is part of all distributions of LaTeX 
% version 2005/12/01 or later and of this work.
%
% This work has the LPPL maintenance status "author-maintained".
%
% The Current Maintainer and author of this work is Markus Kohm.
%
% This work consists of all files listed in manifest.txt.
% ----------------------------------------------------------------------
% scrkernel-sections.dtx
% Copyright (c) Markus Kohm, 2002-2019
%
% Dieses Werk darf nach den Bedingungen der LaTeX Project Public Lizenz,
% Version 1.3c, verteilt und/oder veraendert werden.
% Die neuste Version dieser Lizenz ist
%   http://www.latex-project.org/lppl.txt
% und Version 1.3c ist Teil aller Verteilungen von LaTeX
% Version 2005/12/01 oder spaeter und dieses Werks.
%
% Dieses Werk hat den LPPL-Verwaltungs-Status "author-maintained"
% (allein durch den Autor verwaltet).
%
% Der Aktuelle Verwalter und Autor dieses Werkes ist Markus Kohm.
% 
% Dieses Werk besteht aus den in manifest.txt aufgefuehrten Dateien.
% ======================================================================
% \fi
%
% \CharacterTable
%  {Upper-case    \A\B\C\D\E\F\G\H\I\J\K\L\M\N\O\P\Q\R\S\T\U\V\W\X\Y\Z
%   Lower-case    \a\b\c\d\e\f\g\h\i\j\k\l\m\n\o\p\q\r\s\t\u\v\w\x\y\z
%   Digits        \0\1\2\3\4\5\6\7\8\9
%   Exclamation   \!     Double quote  \"     Hash (number) \#
%   Dollar        \$     Percent       \%     Ampersand     \&
%   Acute accent  \'     Left paren    \(     Right paren   \)
%   Asterisk      \*     Plus          \+     Comma         \,
%   Minus         \-     Point         \.     Solidus       \/
%   Colon         \:     Semicolon     \;     Less than     \<
%   Equals        \=     Greater than  \>     Question mark \?
%   Commercial at \@     Left bracket  \[     Backslash     \\
%   Right bracket \]     Circumflex    \^     Underscore    \_
%   Grave accent  \`     Left brace    \{     Vertical bar  \|
%   Right brace   \}     Tilde         \~}
%
% \iffalse
%%% From File: $Id$
% The runs of run-time file generation:
%<preidentify>%%%            (run: preidentify)
% - everything that is needed before selfidentification of resulting files
%   (unused)
%<identify>%%%            (run: identify)
% - everything that is needed for selfidentification of resulting files
%   (unused)
%<prepare>%%%            (run: prepare)
% - everything that is needed to define options
%<option>%%%            (run: option)
% - definition of options
%<execoption>%%%            (run: execoption)
% - initial option executions up to \KOMAProcessOptions
%<body>%%%            (run: body)
% - everything, that should be done after \KOMAProcessOptions
%<exit>%%%            (run: exit)
% - everything, that should be done immediately before leaving the file
%<*dtx>
% \fi
\ifx\ProvidesFile\undefined\def\ProvidesFile#1[#2]{}\fi
\begingroup
  \def\filedate$#1: #2-#3-#4 #5${\gdef\filedate{#2/#3/#4}}
  \filedate$Date$
  \def\filerevision$#1: #2 ${\gdef\filerevision{r#2}}
  \filerevision$Revision$
  \edef\reserved@a{%
    \noexpand\endgroup
    \noexpand\ProvidesFile{scrkernel-sections.dtx}%
                          [\filedate\space\filerevision\space
                           KOMA-Script source (disposition)]
  }%
\reserved@a
% \iffalse
\documentclass[parskip=half-]{scrdoc}
\usepackage[english,ngerman]{babel}
\CodelineIndex
\RecordChanges
\GetFileInfo{scrkernel-sections.dtx}
\title{\KOMAScript{} \partname\ \texttt{\filename}%
  \footnote{Dies ist Version \fileversion\ von Datei
    \texttt{\filename}.}}
\date{\filedate}
\author{Markus Kohm}

\begin{document}
  \maketitle
  \tableofcontents
  \DocInput{\filename}
\end{document}
%</dtx>
% \fi
%
% \selectlanguage{ngerman}
%
% \changes{v2.3h}{1996/01/20}{Größenbefehle in den Überschriften
%   variabilisiert}%^^A
% \changes{v2.95}{2002/06/26}{erste Version aus der Aufteilung von
%   \texttt{scrclass.dtx}}%^^A
% \changes{v3.15}{2014/12/09}{Code in Datei \texttt{scrkernel-section} komplett
%   umsortiert}%^^A
%
% \section{Gliederung}
%
% Zur Gliederung gehören neben den üblichen Gliederungsbefehlen
% \cs{chapter} \dots \cs{subparagraph} auch alle Optionen und Befehle,
% die diese Gliederung beeinflussen. Eingeschränkt gehören auch die
% Befehle dazu, die Inhaltsverzeichniseinträge zu den
% Gliederungsbefehlen erzeugen oder unmittelbar formatieren. Diese
% Entscheidung wurde getroffen, weil die Zusammengehörigkeit eher hier
% gesehen wird, als bei den Verzeichnissen selbst.
%
% \StopEventually{\PrintIndex\PrintChanges}
%
% \changes{v2.8e}{2001/07/10}{\cs{@maybeasf} entfernt}
%
% Die Briefklasse hat keine derartige Gliederung. Beim Briefpaket wird die
% Gliederung über die verwendete Klasse bestimmt.
% \iffalse
%<*!letter>
% \fi
%
% \subsection{Vakatseiten}
%
% Bei \textsf{scrbook} und \textsf{scrreprt} gibt es im doppelseitigen Modus
% ggf. Vakatseiten.
%
% \begin{option}{open}
% \changes{v2.98c}{2008/03/06}{Neue Option}%^^A
% \changes{v3.12}{2013/03/05}{Verwendung der Status-Signalisierung mit
%     \cs{FamilyKeyState}}%^^A
% \changes{v3.17}{2015/03/09}{interne Speicherung des Wert}%^^A
% Bei \textsf{scrreprt} und \textsf{scrbook} kann per Option eingestellt
% werden, ob beispielsweise Kapitel nur auf linken oder rechten Seiten
% beginnen dürfen oder auf jeder Seite.
% \begin{macro}{\if@openright}
% \begin{macro}{\@openrighttrue}
% \begin{macro}{\@openrightfalse}
% Der Zustand wird dann in einem Schalter und durch umdefinieren von
% \cs{cleardoublestandardpage} gespeichert. Die Voreinstellungen sind je nach
% Klasse unterschiedlich.
%    \begin{macrocode}
%<*class>
%<*book|report>
%<*prepare>
\newif\if@openright
%<report>\@openrightfalse
%<book>\@openrighttrue
%</prepare>
%    \end{macrocode}
% \end{macro}
% \end{macro}
% \end{macro}
%    \begin{macrocode}
%<*option>
\KOMA@key{open}{%
  \KOMA@set@ncmdkey{open}{@tempa}{%
    {any}{0},%
    {right}{1},%
    {left}{2}%
  }{#1}%
  \ifx\FamilyKeyState\FamilyKeyStateProcessed
    \KOMA@kav@xreplacevalue{.\KOMAClassFileName}{open}{#1}%
    \KOMA@kav@remove{.\KOMAClassFileName}{headings}{openany}%
    \KOMA@kav@remove{.\KOMAClassFileName}{headings}{openright}%
    \KOMA@kav@remove{.\KOMAClassFileName}{headings}{openleft}%
    \ifcase \@tempa\relax
      \KOMA@kav@add{.\KOMAClassFileName}{headings}{openany}%
      \@openrightfalse
      \renewcommand*{\cleardoublestandardpage}{\cleardoubleoddstandardpage}%
    \or
      \KOMA@kav@add{.\KOMAClassFileName}{headings}{openright}%
      \@openrighttrue
      \renewcommand*{\cleardoublestandardpage}{\cleardoubleoddstandardpage}%
    \or
      \KOMA@kav@add{.\KOMAClassFileName}{headings}{openleft}%
      \@openrighttrue
      \renewcommand*{\cleardoublestandardpage}{\cleardoubleevenstandardpage}%
    \fi
  \fi
}
%</option>
%    \end{macrocode}
% \changes{v3.18}{2015/05/22}{indirekte Initialisierung}
% Die Initialisierung erfolgt indirekt über Option \texttt{headings}.
%    \begin{macrocode}
%<*execoption>
%<book>\KOMAExecuteOptions{headings=openright}
%<report>\KOMAExecuteOptions{headings=openany}
%</execoption>
%    \end{macrocode}
% \end{option}
%
% \begin{option}{openright}
% \changes{v2.98c}{2008/03/05}{obsolet}%^^A
% \changes{v3.01a}{2008/11/21}{standard statt obsolet}%^^A
% \begin{option}{openany}
% \changes{v2.98c}{2008/03/05}{obsolet}%^^A
% \changes{v3.01a}{2008/11/21}{standard statt obsolet}%^^A
% Zusätzlich gibt es einige Standardoptionen, die natürlich auch mit
% \KOMAScript{} funktionieren sollen.
%    \begin{macrocode}
%<*option>
\KOMA@DeclareStandardOption{openright}{open=right}
\KOMA@DeclareStandardOption{openany}{open=any}
%</option>
%</book|report>
%</class>
%    \end{macrocode}
% \end{option}
% \end{option}
%
%
% \subsection{Behandlung von Gliederungsnummern}
%
% Die Möglichkeit der Nummerierung ist allen Ebenen bis auf \cs{minisec}
% gemeinsam.
%
% \begin{option}{numbers}
% \changes{v2.98c}{2008/03/07}{Neue Option}%^^A
% \changes{v3.12}{2013/03/05}{Verwendung der Status-Signalisierung mit
%     \cs{FamilyKeyState}}%^^A
% \changes{v3.17}{2015/03/09}{interne Speicherung des Wert}%^^A
% Dies ist die zentrale Option zur Konfiguration der Nummerierung. Dabei geht
% es primär um die Gliederungsnummern, es sind jedoch auch davon abgeleitete
% Nummern betroffen.
% \begin{macro}{\scr@dotchangeatdocument}
% \changes{v2.98c}{2008/03/07}{Neu (intern)}%^^A
% Da die Umschaltung zwischen automatischem Endpunkt, immer Endpunkt und kein
% Endpunkt unbedingt \emph{vor} \cs{begin{document}} erfolgen muss, wird eine
% Fehlermeldung ausgegeben, falls das jemand zu einem späteren Zeitpunkt
% probiert. Umschaltmöglichkeiten sind auf die Klassen beschränkt.
%    \begin{macrocode}
%<*class>
%<*prepare>
\newcommand*{\scr@dotchangeatdocument}[1]{%
  \ClassError{\KOMAClassName}{%
    change of end dot feature after \string\begin{document}%
  }{%
    You've tried to set `numbers=#1' after \string\begin{document},\MessageBreak
    but this option is only allowed at the document preamble.\MessageBreak
    See KOMA-Script manual for more information about.%
  }%
}
%</prepare>
%    \end{macrocode}
% \end{macro}
% Normalerweise wird automatisch entschieden, ob Nummern mit einem
% Punkt enden müssen oder nicht. Um dies abzuschalten, muss nur der
% gewünschte Zustand eingeschaltet und die Umschaltmöglichkeit
% entfernt werden. Die Umschaltung kann jedoch nicht direkt erfolgen, sondern
% erfolgt zeitverzögert in \cs{begin{document}}. Hier wird nur das Makro
% umgeschaltet, das dort verwendet wird.
%    \begin{macrocode}
%<*option>
\KOMA@key{numbers}{%
  \KOMA@set@ncmdkey{numbers}{@tempa}{%
    {autoendperiod}{0},{autoenddot}{0},{auto}{0},%
    {endperiod}{1},{withendperiod}{1},{periodatend}{1},%
    {enddot}{1},{withenddot}{1},{dotatend}{1},%
    {noendperiod}{2},{noperiodatend}{2},%
    {noenddot}{2},{nodotatend}{2}%
  }{#1}%
  \ifx\FamilyKeyState\FamilyKeyStateProcessed
    \KOMA@kav@xreplacevalue{.\KOMAClassFileName}{numbers}{#1}%
    \ifcase \@tempa\relax
      \if@atdocument\scr@dotchangeatdocument{#1}\else
        \let\scr@altsecnumhook\@empty
      \fi
    \or
      \if@atdocument\scr@dotchangeatdocument{#1}\else
        \let\scr@altsecnumhook\scr@altsecnumhooktrue
      \fi
    \or
      \if@atdocument\scr@dotchangeatdocument{#1}\else
        \let\scr@altsecnumhook\scr@altsecnumhookfalse
      \fi
    \fi
  \fi
}
%</option>
%<execoption>\KOMAExecuteOptions{numbers=autoendperiod}
%    \end{macrocode}
% \end{option}
%
% \begin{option}{pointednumbers}
% \changes{v2.3h}{1995/01/19}{neue Option}%^^A
% \changes{v2.4g}{1996/11/04}{die Option heißt nun wirklich so,
%     bei \texttt{pointednumber} wird explizit ein Fehler gemeldet}
% \changes{v2.4g}{1996/11/04}{\cs{@altsecnumformatfalse} wird
%     direkt auf \cs{@altsecnumformattrue} gesetzt}
% \changes{v2.98c}{2008/03/07}{obsolet}%^^A
% \changes{v3.01a}{2008/11/20}{deprecated}%^^A
% \begin{option}{pointlessnumbers}
% \changes{v2.3h}{1995/01/19}{neue Option}%^^A
% \changes{v2.4g}{1996/11/04}{die Option heißt nun wirklich so,
%     bei \texttt{pointlessnumber} wird explizit ein Fehler gemeldet}
% \changes{v2.4g}{1996/11/04}{\cs{@altsecnumformattrue} wird direkt
%     auf \cs{@altsecnumformatfalse} gesetzt}
% \changes{v2.98c}{2008/03/07}{obsolet}%^^A
% \changes{v3.01a}{2008/11/20}{deprecated}%^^A
% Diese Optionens sind nun wirklich veraltet und sollten nicht mehr verwendet
% werden!
%    \begin{macrocode}
%<*option>
\KOMA@DeclareDeprecatedOption{pointednumbers}{numbers=enddot}
\KOMA@DeclareDeprecatedOption{pointlessnumbers}{numbers=noenddot}
%</option>
%    \end{macrocode}
% Übrigens weiß ich, wie blödsinnig die Namen der Optionen sind. Aber
% das ist nun nicht mehr zu ändern.
% \end{option}
% \end{option}
%
% \begin{Counter}{secnumdepth}
% Dieser Zähler gibt an, bis zu welcher Ebene die Gliederungsüberschriften
% nummeriert werden.
%    \begin{macrocode}
%<*body>
%<book|report>\setcounter{secnumdepth}{2}
%<article>\setcounter{secnumdepth}{3}
%</body>
%    \end{macrocode}
% \end{Counter}
%
%
% \begin{macro}{\ifnumbered}
% \changes{v3.12}{2013/12/16}{new command}%^^A
% \changes{v3.28}{2019/11/19}{renamed to \cs{Ifnumbered}}%^^A
% \begin{macro}{\Ifnumbered}
% \changes{v3.28}{2019/11/19}{renamed from \cs{ifnumbered}}%^^A
% \begin{macro}{\ifunnumbered}
% \changes{v3.12}{2013/12/16}{new command}%^^A
% \changes{v3.28}{2019/11/19}{renamed to \cs{Ifunnumbered}}%^^A
% \begin{macro}{\Ifunnumbered}
% \changes{v3.28}{2019/11/19}{renamed from \cs{ifunnumbered}}%^^A
% Test, ob eine Gliederungsebene (\texttt{\#1}) nummeriert wird. Falls das der
% Fall ist, wird das zweite Argument verwendet, sonst das dritte. Bei
% \cs{ifunnumbered} ist die Logik genau vertauscht.
%    \begin{macrocode}
%<*body>
\providecommand*{\ifnumbered}{%
  \ClassWarning{\KOMAClassName}{Usage of deprecated command
    `\string\ifnumbered'.\MessageBreak
    The command has been renamed because of a\MessageBreak
    recommendation of The LaTeX Project Team.\MessageBreak
    Please replace `\string\ifnumbered' by `\string\Ifnumbered'%
  }%
  \Ifnumbered
}
\newcommand*{\Ifnumbered}[1]{%
  \if@currentusenumber
    \scr@ifundefinedorrelax{#1numdepth}{%
      \expandafter\@secondoftwo
    }{%
      \expandafter\ifnum \@nameuse{#1numdepth}>\c@secnumdepth
        \expandafter\expandafter\expandafter\@secondoftwo
      \else
        \expandafter\expandafter\expandafter\@firstoftwo
      \fi
    }%
  \else
    \expandafter\@secondoftwo
  \fi  
}
\providecommand*{\ifunnumbered}{%
  \ClassWarning{\KOMAClassName}{Usage of deprecated command
    `\string\ifunnumbered'.\MessageBreak
    The command has been renamed because of a\MessageBreak
    recommendation of The LaTeX Project Team.\MessageBreak
    Please replace `\string\ifunnumbered' by `\string\Ifunnumbered'%
  }%
  \Ifunnumbered
}
\newcommand*{\Ifunnumbered}[1]{%
  \Ifnumbered{#1}{\@secondoftwo}{\@firstoftwo}%
}
%</body>
%</class>
%    \end{macrocode}
% \end{macro}
% \end{macro}
% \end{macro}
% \end{macro}
%
%
% \subsection{\KOMAScript-eigene Befehle zur Definition von
%   Gliederungsbefehlen}
%
% Die Funktionalität des \LaTeX-Kerns genügt \KOMAScript{} bei der Definition
% von Gliederungsebenen nicht.
%
% \begin{macro}{\scr@activate@xsection}
% \changes{v3.10}{2011/08/30}{neu (intern)}%^^A
% Hiermit wird die Erweiterung für das optionale Argument der
% Gliederungsbefehle eingeschaltet.
% \begin{macro}{\scr@osectarg}%^^A
% Für die nicht per \cs{@startsection} implementierten Befehle geht das
% einfach durch Umschaltung von \cs{scr@osectarg}. Bei 0 ist die Aktivierung
% nicht aktiv. Bei 1 landet das optionale Argument hingegen im Kopf, bei 2
% im Inhaltsverzeichnis und bei 3 in beiden. Die Voreinstellung ist deshalb
% ebenfalls 0.
%
% Da Option \texttt{headings} darauf zurückgreift, müssen die beiden Makros
% bereits vorab definiert werden.
%    \begin{macrocode}
%<*class>
%<*prepare>
\newcommand*{\scr@osectarg}{0}
\newcommand*{\scr@activate@xsection}[1]{%
  \renewcommand*{\scr@osectarg}{#1}%
}
%</prepare>
%</class>
%    \end{macrocode}
% \end{macro}
% \end{macro}
%
%
% \begin{macro}{\if@altsecnumformat}
% \changes{v2.3c}{1995/08/06}{neuer Schalter}%^^A
% \begin{macro}{\@altsecnumformattrue}
% \begin{macro}{\@altsecnumformatfalse}
%  Dieses Schalter wird für die Umschaltung auf Duden-Regel~6 bei der
%  Numerierung benötigt.
% \begin{macro}{\if@autodot}
% \changes{v2.8e}{2001/07/10}{neuer Schalter}%^^A
% \begin{macro}{\@autodottrue}
% \changes{v3.00}{2008/10/07}{arbeitet nun gobal}%^^A
% \begin{macro}{\@autodotfalse}
% \changes{v3.00}{2008/10/07}{arbeitet nun gobal}%^^A
% Nach Duden 20.~Auflage, Regel~5 wird in der Gliederung kein Punkt gesetzt,
% wenn diese rein mit arabischen Zahlen erfolgt. Nach Regel~6 folgt jedoch ein
% Punkt, wenn die Gliederung auch römische Zahlen oder Großbuchstaben
% enthält. Aus Konsistenzgründen muss der Punkt in diesem Fall immer gesetzt
% werden. \KOMAScript{} bietet dafür einen Automatismus.
%
% Damit die Umschaltung auch wieder zurück auf Regel~5 automatisch erfolgen
% kann, wird ein zweiter Schalter benötigt.
%    \begin{macrocode}
%<*class>
%<*prepare>
\newif\if@altsecnumformat\@altsecnumformatfalse
\newif\if@autodot
\renewcommand*{\@autodottrue}{\global\let\if@autodot\iftrue}
\renewcommand*{\@autodotfalse}{\global\let\if@autodot\iffalse}
\@autodotfalse
%</prepare>
%    \end{macrocode}
% Dabei soll die Umstellung des Punktes beim zweiten TeX-Lauf bereits am
% Anfang aktiv sein. Deshalb wird die Information am Ende in die aux-Datei
% geschrieben.
% \changes{v2.9k}{2003/01/12}{es wird direkt in \cs{@mainaux} geschrieben}%^^A
% \changes{v2.95}{2004/01/15}{es wird erst unmittelbar vor dem Schließen
%      der Haupt-aux-Datei in diese geschrieben}%^^A
% \changes{v2.96a}{2006/12/07}{es wird \cs{immediate} geschrieben}%^^A
% \changes{v3.20}{2016/03/25}{es wird \cs{csname}\dots\cs{endcsname}
%     verwendet}%^^A
%    \begin{macrocode}
%<*body>
\BeforeClosingMainAux{%
  \if@autodot\if@filesw\immediate\write\@mainaux{%
      \string\global\string\csname\space @altsecnumformattrue\string\endcsname}%
  \fi\fi}
%</body>
%    \end{macrocode}
% \end{macro}%^^A \@autodotfalse
% \end{macro}%^^A \@autodottrue
% \end{macro}%^^A \if@autodot
% \end{macro}%^^A \@altsecnumformatfalse
% \end{macro}%^^A \@altsecnumformattrue
% \end{macro}%^^A \if@altsecnumformat
%
% \begin{macro}{\scr@altsecnumhook}
% \changes{v2.98c}{2008/03/07}{Neu (intern)}%^^A
% \begin{macro}{\scr@altsecnumhooktrue}
% \changes{v2.98c}{2008/03/07}{Neu (intern)}%^^A
% \begin{macro}{\scr@altsecnumhookfalse}
% \changes{v2.98c}{2008/03/07}{Neu (intern)}%^^A
% Es werden drei Hilfsmakros benötigt. Das erste davon wird bei
% \cs{begin{document}} ausgeführt (nachdem die \texttt{aux}-Datei gelesen
% wurde!) und schaltet den gewünschten Zustand ein. Dazu wird es per Option
% ggf. auf die eine oder andere Bedeutung umgeschaltet. Im Auto-Fall ist es
% übrigens \cs{@empty}.
%    \begin{macrocode}
%<*prepare>
\newcommand*{\scr@altsecnumhook}{}
\AtBeginDocument{\scr@altsecnumhook}
\newcommand*{\scr@altsecnumhooktrue}{%
  \@altsecnumformattrue\global\let\@altsecnumformatfalse\@altsecnumformattrue
}
\newcommand*{\scr@altsecnumhookfalse}{%
  \@altsecnumformatfalse\global\let\@altsecnumformattrue\@altsecnumformatfalse
}
%</prepare>
%    \end{macrocode}
% \end{macro}%^^A \scr@altsecnumhooktrue
% \end{macro}%^^A \scr@altsecnumhookfalse
% \end{macro}%^^A \scr@altsecnumhook
%
% \begin{macro}{\@maybeautodot}
% \changes{v2.8e}{2001/07/10}{neu (intern)}%^^A
% Ob der automatische Punkt aktiviert werden muss, wird aufgrund der
% Darstellung eines Zählers entschieden. Diese wird dem Makro als Argument
% übergeben (\cs{thepart}, \cs{thechapter} etc.). Durch Verwendung einer
% Gruppe, werden alle Definitionen lokal gehalten.
%    \begin{macrocode}
%<*body>
\newcommand{\@maybeautodot}[1]{\if@autodot\else\begingroup%
  \expandafter\@@maybeautodot #1\@stop\endgroup\fi
}
%    \end{macrocode}
% \begin{macro}{\@@maybeautodot}
% \changes{v2.8e}{2001/07/10}{neu (intern)}%^^A
% \changes{v3.00}{2008/10/07}{neuerdings überflüssige \cs{aftergroup}%^^A
%     entfernt}%^^A
% Die expandierte Darstellung wird dann auf Darstellungsbefehle für
% Zähler gescannt.
%    \begin{macrocode}
\newcommand*{\@@maybeautodot}[1]{%
  \ifx #1\@stop\let\@@maybeautodot\relax
  \else
    \ifx #1\Alph \@autodottrue\fi
    \ifx #1\alph \@autodottrue\fi
    \ifx #1\Roman \@autodottrue\fi
    \ifx #1\roman \@autodottrue\fi
    \ifx #1\@Alph \@autodottrue\fi
    \ifx #1\@alph \@autodottrue\fi
    \ifx #1\@Roman \@autodottrue\fi
    \ifx #1\@roman \@autodottrue\fi
    \ifx #1\romannumeral \@autodottrue\fi
  \fi
  \@@maybeautodot
}
%</body>
%    \end{macrocode}
% \end{macro}%^^A \@@maybeautodot
% \end{macro}%^^A \@maybeautodot
%
% \begin{macro}{\autodot}
% \changes{v2.7}{2001/01/03}{neu}%^^A
% Dieses Makro setzt dann den Punkt nach Bedarf.
%    \begin{macrocode}
%<*body>
\newcommand*\autodot{\if@altsecnumformat.\fi}
%</body>
%    \end{macrocode}
% \end{macro}
%
%
% \begin{macro}{\@startsection}
% \changes{v3.13a}{2014/09/11}{Vollständige Neudefinition zur Optimierung
%   der Erweiterung für das optionale Argument}%^^A
% \begin{macro}{\scr@startsection}
% \changes{v3.13a}{2014/09/11}{Neue interne Anweisung}%^^A
% \changes{v3.18}{2015/05/22}{verwende \cs{numexpr} für Argument 2}
% \changes{v3.18}{2015/05/22}{verwende \cs{glueexpr} für Argument 3--5}
% \changes{v3.20}{2016/04/12}{\cs{@ifstar} durch \cs{kernel@ifstar}
%   ersetzt}%^^A
% \changes{v3.26}{2018/09/18}{\texttt{afterindent}-Einstellung wird
%   beachtet}%^^A
% \begin{macro}{\scr@saved@startsection}
% \changes{v3.13a}{2014/09/11}{Neue interne Anweisung}%^^A
% \changes{v3.20}{2016/04/12}{Tests funktionieren jetzt auch mit
%   \textsf{amsgen}}%^^A
% \KOMAScript{} benötigt ein eigenes \cs{@startsection}, weil die
% Funktionalität des \cs{@startsection} aus dem \LaTeX-Kern nicht
% genügt. Bevor allerdings die eigene Definition in Angriff genommen wird,
% muss kontrolliert werden, ob die aufgefundene Definition noch den
% Erwartungen entspricht. Alles andere wäre ein deutlicher Hinweis, dass
% eventuell in \KOMAScript{} ebenfalls eine Anpassung vorzunehmen wäre.
% Da es offenbar Leute gibt, die bei der Umdefinierung von \cs{section}
% etc. mit Hilfe von \cs{@startsection} Leerzeichen zwischen die Argumente
% packen und gleichzeitig \textsf{amsgen} verwenden, sorgen wir hier dafür,
% dass dann trotzdem noch die Warnungen funktionieren und nebenbei der Code
% mit etwas Glück sogar ebenfalls. Dazu ist es notwendig, dass die Version von
% \cs{@startsection} gespeichert wird, die \cs{kernel@ifstar} verwendet.
% \begin{macro}{\startsection@sectionname}
% \changes{v3.26}{2018/05/14}{Neue Anweisung für Paketautoren}%^^A
% \begin{macro}{\startsection@secnumdepth}
% \changes{v3.26}{2018/05/14}{Neue Anweisung für Paketautoren}%^^A
% \begin{macro}{\startsection@indent}
% \changes{v3.26}{2018/05/14}{Neue Anweisung für Paketautoren}%^^A
% \begin{macro}{\startsection@beforeskip}
% \changes{v3.26}{2018/05/14}{Neue Anweisung für Paketautoren}%^^A
% \begin{macro}{\startsection@afterskip}
% \changes{v3.26}{2018/05/14}{Neue Anweisung für Paketautoren}%^^A
% \begin{macro}{\if@startsection@runin}
% \changes{v3.26}{2018/05/14}{Neue Anweisung für Paketautoren}%^^A
% \begin{macro}{\startsection@afterindent}
% \changes{v3.26}{2018/09/18}{Neue Anweisung für Paketautoren}%^^A
% Ab dem Zeitpunkt des Aufrufs des Do-Hooks
% \texttt{heading/postinit} sind die Makros
% \cs{startsection@sectionname}, \cs{startsection@secnumdepth},
% \cs{startsection@indent}, \cs{startsection@beforeskip},
% \cs{if@startsection@indentafter}, \cs{startsection@afterskip},
% \cs{if@startsection@runin} und \cs{startsection@afterindent} gültig. Die
% \cs{\dots indent}- und \cs{\dots skip}-Makros sind dabei skips (oder
% \cs{glueexpr}) und somit wie Längen verwendbar. Bei den beiden \cs{\dots
% skip}-Makros handelt es sich bereits um effektive Abstände, es wurden also
% etwaige Fallunterscheidungen bezüglich des Vorzeichens des Arguments für
% \cs{scr@startsection} bereits aufgelöst. Das \cs{if\dots}-Makro ist entweder
% \cs{iftrue} oder \cs{iffalse}. Ein \cs{if@startsection@afterindent}
% existiert nicht, weil \cs{if@afterindent} zu dem Zeitpunkt bereits gültig
% ist. Außerhalb von \cs{scr@startsection} ist die Verwendung dieser Macros
% unzulässig, bzw. das Ergebnis unspezifiziert. Ebenso sind diese Macros als
% \emph{read-only} zu betrachten!
%^^A TODO: Der ganze Test hier ist eigentlich Unfug. Beim Laden
%^^A der Klasse sollte die Kern-Anweisung noch unverändert sein, da Pakete
%^^A normalerweise erst danach zum Zuge kommen und sie umdefinieren
%^^A können. Also müsste eigentlich \cs{AtBeginDocument} getestet werden. Ich
%^^A werde mir das gelegentlich noch einmal vornehmen müssen.  
%    \begin{macrocode}
%<*body>
\newcommand*{\scr@saved@startsection}[6]{%
  \if@noskipsec \leavevmode \fi
  \par
  \@tempskipa #4\relax
  \@afterindenttrue
  \ifdim \@tempskipa <\z@
    \@tempskipa -\@tempskipa \@afterindentfalse
  \fi
  \if@nobreak
    \everypar{}%
  \else
    \addpenalty\@secpenalty\addvspace\@tempskipa
  \fi
  \kernel@ifstar
    {\@ssect{#3}{#4}{#5}{#6}}%
    {\@dblarg{\@sect{#1}{#2}{#3}{#4}{#5}{#6}}}%
}
\def\reserved@a#1#2#3#4#5#6{%
  \if@noskipsec \leavevmode \fi
  \par
  \@tempskipa #4\relax
  \@afterindenttrue
  \ifdim \@tempskipa <\z@
    \@tempskipa -\@tempskipa \@afterindentfalse
  \fi
  \if@nobreak
    \everypar{}%
  \else
    \addpenalty\@secpenalty\addvspace\@tempskipa
  \fi
  \@ifstar
    {\@ssect{#3}{#4}{#5}{#6}}%
    {\@dblarg{\@sect{#1}{#2}{#3}{#4}{#5}{#6}}}%
}
\ifcase \ifx\@startsection\scr@saved@startsection 0
        \else
          \ifx\@startsection\reserved@a 0
          \else 1
          \fi
        \fi
  \newcommand*{\scr@startsection}[6]{%
    \ExecuteDoHook{heading/preinit/#1}%
    \if@noskipsec \leavevmode \fi
    \par
    \@tempskipa \glueexpr #4\relax
    \@ifundefined{scr@#1@afterindent}{%
      \def\startsection@afterindent##1##2##3{##3}%
    }{%
      \expandafter\let\expandafter\startsection@afterindent
      \csname scr@#1@afterindent\endcsname
    }%
    \startsection@afterindent{\@afterindenttrue}{\@afterindentfalse}{%
      \@afterindenttrue
      \ifdim \@tempskipa <\z@
        \@tempskipa -\@tempskipa \@afterindentfalse
      \fi
    }%  
    \def\startsection@sectionname{#1}%
    \def\startsection@secnumdepth{\numexpr #2\relax}%
    \def\startsection@indent{\glueexpr #3\relax}%
    \let\startsection@beforeskip\@tempskipa
    \@ifundefined{scr@#1@runin}{%
      \def\scr@sect@runin##1##2##3{##3}%
    }{%
      \expandafter\let\expandafter\scr@sect@runin
      \csname scr@#1@runin\endcsname
    }%
    \scr@sect@runin{%
      \def\startsection@afterskip{\glueexpr #3\relax}%
      \expandafter\let\csname if@startsection@runin\expandafter\endcsname
      \csname iftrue\endcsname
    }{%
      \def\startsection@afterskip{\glueexpr #3\relax}%
      \expandafter\let\csname if@startsection@runin\expandafter\endcsname
      \csname iffalse\endcsname
    }{%
      \ifdim \glueexpr #5\relax <\z@
        \def\startsection@afterskip{\glueexpr (#3)*\m@ne\relax}%
        \expandafter\let\csname if@startsection@runin\expandafter\endcsname
        \csname iftrue\endcsname
      \else
        \def\startsection@afterskip{\glueexpr #3\relax}%
        \expandafter\let\csname if@startsection@runin\expandafter\endcsname
        \csname iffalse\endcsname
      \fi
    }%
    \ExecuteDoHook{heading/postinit/#1}%
    \if@nobreak
      \everypar{}%
    \else
      \addpenalty\@secpenalty\addvspace\@tempskipa
    \fi
    \kernel@ifstar {%
      \ExecuteDoHook{heading/branch/star/#1}%
      \def\scr@s@ct@@nn@m@{#1}%
      \@ssect{\glueexpr #3\relax}{\glueexpr #4\relax}{\glueexpr #5\relax}{#6}%
    }{%
      \ExecuteDoHook{heading/branch/nostar/#1}%
      \scr@section@dblarg{%
        \@sect{#1}{\numexpr #2\relax}{\glueexpr #3\relax}{\glueexpr #4\relax}%
        {\glueexpr #5\relax}{#6}%
      }%
    }%
  }
\else
  \ClassWarningNoLine{\KOMAClassName}{`\string\@startsection' has been
    changed.\MessageBreak
    \KOMAClassName\space needs it's own definition of
    `\string\@startsection'\MessageBreak
    to provide extended features for the optional argument\MessageBreak
    of `\string\section' etc.\MessageBreak
    Generally it defines `\string\@startsection' completely new\MessageBreak
    to achieve this. Because of the unexpected definition\MessageBreak
    an alternative approach will be used.\MessageBreak
    If this fails and if there isn't a new release of\MessageBreak
    KOMA-Script that fixes the problem, please\MessageBreak
    send a report to the KOMA-Script author.\MessageBreak
    Note, that this alternative approach does not\MessageBreak
    execute the elements of do-hooks:\MessageBreak
    \space\space- `heading/postinit',\MessageBreak
    \space\space- `heading/branch/star',\MessageBreak
    \space\space- `heading/branch/nostar',\MessageBreak
    and does always use `runin=bysign' and\MessageBreak
    `afterindent=bysign'%
  }%
  \let\scr@saved@startsection\@startsection
  \newcommand*{\scr@startsection}[6]{%
    \ExecuteDoHook{heading/preinit/#1}%
    \kernel@ifstar {%
      \scr@saved@startsection{#1}{\numexpr #2\relax}{\glueexpr #3\relax}%
      {\glueexpr #4\relax}{\glueexpr #5\relax}{#6}*%
    }{%
      \scr@section@dblarg{%
        \scr@saved@startsection{#1}{\numexpr #2\relax}{\glueexpr #3\relax}%
        {\glueexpr #4\relax}{\glueexpr #5\relax}{#6}}%
    }%
  }
\fi
%    \end{macrocode}
% \begin{macro}{\scr@sect@runin}
% \changes{v3.26}{2018/09/18}{neu (intern)}
% Diese Hilfsanweisung nimmt die Fallunterscheidung vor, ob eine Spitzmarke
% oder eine freistehende Überschrift oder abhängig vom Vorzeichen des
% Abstandes danach eine Spitzmarke oder eine freistehende Überschrift gesetzt
% werden soll. Die Voreinstellung ist sicherheitshalber letzteres. Diese
% Voreinstellung wird auch bei jedem Aufruf von \cs{@xsect} erneut
% erzwungen. Auch in \cs{scr@startsection} wird die Anweisung angepasst.
%    \begin{macrocode}
\newcommand*{\scr@sect@runin}[3]{#3}
%    \end{macrocode}
% \end{macro}%^^A \scr@sect@runin
% \begin{macro}{\scr@startsection@ulm@error}
% \changes{v3.26}{2018/05/14}{Neue interne Anweisung}
% Ich bin mir bewusst, dass die Fehlerausgabe, die hier definiert wird,
% vermutlich nie aufgerufen wird, da ein Paketautor hierzu eines der Makros
% bereits vor dem ersten \cs{scr@startsection}-Befehl verwenden
% müsste. Trotzdem verwende ich es quasi zwecks Initialisierung.
%    \begin{macrocode}
\newcommand*{\scr@startsection@ulm@error}[1]{%
  \ClassError{\KOMAClassName}{%
    \string#1 undefined outside of \string\scr@startsection
  }{%
    A KOMA-Script macro for package authors have been used outside
    the\MessageBreak
    specified definition scope.\MessageBreak
    See `scrkernel-sections.dtx' for more information.%
  }%
}
\newcommand*\startsection@sectionname{%
  \scr@startsection@ulm@error\startsection@sectionname}
\newcommand*\startsection@secnumdepth{%
  \scr@startsection@ulm@error\startsection@secnumdepth}
\newcommand*\startsection@indent{%
  \scr@startsection@ulm@error\startsection@indent}
\newcommand*\startsection@beforeskip{%
  \scr@startsection@ulm@error\startsection@beforeskip}
\newcommand*\startsection@afterskip{%
  \scr@startsection@ulm@error\startsection@afterskip}
\newcommand*{\if@startsection@runin}{%
  \scr@startsection@ulm@error\if@startsection@runin}
\newif\if@startsection@runin
\newcommand*\startsection@afterindent{%
  \scr@startsection@ulm@error\startsection@afterindent}
%    \end{macrocode}
% \end{macro}%^^A \scr@startsection@ulm@error
% \end{macro}%^^A \scr@startsection@afterindent
% \end{macro}%^^A \if@startsection@runin
% \end{macro}%^^A \startsection@afterskip
% \end{macro}%^^A \startsection@beforeskip
% \end{macro}%^^A \startsection@indent
% \end{macro}%^^A \startsection@secnumdepth
% \end{macro}%^^A \startsection@sectionname
% \begin{macro}{\At@startsection}
% \changes{v3.14}{2014/09/11}{Neue Anweisung für Paketautoren}%^^A
% \changes{v3.27}{2019/02/04}{Umgestellt auf \cs{AddtoDoHook}}%^^A
% \changes{v3.27}{2019/02/04}{deprecated}%^^A
% Mit Hilfe dieser Anweisung kann Code in die Ausführung von \KOMAScript's
% eigenem \cs{scr@startsection} an der Stelle eingefügt werden, an der
% \cs{if@afterindent} und \cs{@tempskipa} bereits korrekt eingestellt sind,
% aber noch bevor \emph{pentaly} und Abstand gesetzt wurden bzw. \cs{everypar}
% gelöscht wurde.
%    \begin{macrocode}
\newcommand*{\At@startsection}[1]{%
  \ClassInfo{\KOMAClassName}{%
    Usage of deprecated command `\string\At@startsection'
    mapped to\MessageBreak
    `\string\AddtoDoHook{heading/postinit}
                        {...\string\@gobble}'%
  }%
  \AddtoDoHook{heading/postinit}{\scr@doonlyifstyleofargis{#1}{section}}%
}
%    \end{macrocode}
% \begin{macro}{\scr@doonlyifstyleofargis}
% \changes{v3.27}{2019/02/04}{neu (intern)}
% Hilfsmakro für \cs{At@startsection}, \cs{Before@sect} und \cs{Before@@sect}.
%    \begin{macrocode}
\newcommand*{\scr@doonlyifstyleofargis}[3]{%
  \IfSectionCommandStyleIs{#3}{#2}{#1}{}%
}
%    \end{macrocode}
% \end{macro}%^^A \scr@doonlyifstyleofargis
% \end{macro}%^^A \At@startsection
% \begin{macro}{\Before@ssect}
% \changes{v3.14}{2014/09/11}{Neue Anweisung für Paketautoren}%^^A
% \changes{v3.27}{2019/02/04}{Umgestellt auf \cs{AddtoDoHook}}%^^A
% \changes{v3.27}{2019/02/04}{deprecated}%^^A
% Mit Hilfe dieser Anweisung kann Code in die Ausführung von \KOMAScript's
% eigenem \cs{scr@startsection} unmittelbar vor dem Aufruf von \cs{@ssect}
% eingefügt werden.
%    \begin{macrocode}
\newcommand*{\Before@ssect}[1]{%
  \ClassInfo{\KOMAClassName}{%
    Usage of deprecated command `\string\Before@ssect'
    mapped to\MessageBreak
    `\string\AddtoDoHook{heading/branch/star}%
                        {...\string\@gobble}'%
  }%
  \AddtoDoHook{heading/branch/star}{\scr@onlyifstyleofargis{#1}{section}}%
}
%    \end{macrocode}
% \end{macro}%^^A \Before@ssect
% \begin{macro}{\Before@sect}
% \changes{v3.14}{2014/09/11}{Neue Anweisung für Paketautoren}%^^A
% \changes{v3.27}{2019/02/04}{Umgestellt auf \cs{AddtoDoHook}}%^^A
% \changes{v3.27}{2019/02/04}{deprecated}%^^A
% Mit Hilfe dieser Anweisung kann Code in die Ausführung von \KOMAScript's
% eigenem \cs{scr@startsection} unmittelbar vor dem Aufruf von \cs{@sect}
% eingefügt werden.
%    \begin{macrocode}
\newcommand*{\Before@sect}[1]{%
  \ClassInfo{\KOMAClassName}{%
    Usage of deprecated command `\string\Before@sect'
    mapped to\MessageBreak
    `\string\AddtoDoHook{heading/branch/nostar}%
                        {...\string\@gobble}'%
  }%
  \AddtoDoHook{heading/branch/nostar}{\scr@onlyifstyleofargis{#1}{section}}%
}
%    \end{macrocode}
% \end{macro}%^^A \Before@sect
% \begin{macro}{\scr@section@dblarg}
% \changes{v3.13a}{2014/09/11}{Neue interne Anweisung}%^^A
% \begin{macro}{\scr@section@xdblarg}
% \changes{v3.13a}{2014/09/11}{Neue interne Anweisung}%^^A
%    \begin{macrocode}
\newcommand{\scr@section@dblarg}[1]{%
  \kernel@ifnextchar[%]
    {#1}%
    {\scr@section@xdblarg{#1}}%
}
\newcommand{\scr@section@xdblarg}[2]{%
  \begingroup
    \edef\reserved@a{%
      \unexpanded{\endgroup\let\scr@osectarg\z@#1[{#2}]{#2}\def\scr@osectarg}%
      {\scr@osectarg}}%
  \reserved@a
}
%    \end{macrocode}
% \end{macro}%^^A \scr@section@xdblarg
% \end{macro}%^^A \scr@section@dblarg
% \end{macro}%^^A \scr@saved@startsection
% \end{macro}%^^A \scr@startsection
%    \begin{macrocode}
\renewcommand*{\@startsection}{%
  \ifnum \scr@osectarg=\z@
    \expandafter\scr@saved@startsection
  \else
    \expandafter\scr@startsection
  \fi
}
%</body>
%    \end{macrocode}
% \end{macro}%^^A \@startsection
%
% \begin{macro}{\UseNumberUsageError}
% \changes{v3.27}{2019/02/02}{Neu}
% \begin{macro}{\IfUseNumber}
% \changes{v3.27}{2019/02/02}{Explizite Fehlermeldung bei Verwendung außerhalb
%   von Überschriften}%^^A
% Diese Anweisung ist nur innerhalb von Überschriften definiert. Wird sie
% außerhalb verwendet, so gibt sie einen entsprechenden Fehler aus.
%    \begin{macrocode}
%<*body>
\newcommand*{\UseNumberUsageError}[2]{%
  \ClassError{\KOMAClassName}{%
    \string\IfUseNumber\space not allowed%
  }{%
    You cannot use \string\IfUseNumber\space outside a heading.\MessageBreak
    If you'd continue, I'll ignore both arguments, because I do not
    know\MessageBreak
    whether to use the first or the second argument.%
  }%
}
\newcommand*{\IfUseNumber}[2]{}%
\let\IfUseNumber\UseNumberUsageError
%</body>
%    \end{macrocode}
% \end{macro}
% \end{macro}
%
% \begin{macro}{\SecDef}
% \changes{v3.13a}{2014/09/11}{neue Anweisung}%^^A
% \changes{v3.20}{2016/04/12}{\cs{@ifstar} durch \cs{kernel@ifstar}
%     ersetzt}%^^A
% \begin{macro}{\secdef}
% \changes{v3.13a}{2014/09/11}{Vollständige Neudefinition zur Optimierung
%     der Erweiterung für das optionale Argument}%^^A
%    \begin{macrocode}
%<*body>
\newcommand*{\SecDef}[2]{\kernel@ifstar{#2}{\scr@section@dblarg{#1}}}
\CheckCommand*{\secdef}[2]{\@ifstar{#2}{\@dblarg{#1}}}
\let\secdef\SecDef
%</body>
%    \end{macrocode}
% \end{macro}%^^A \secdef
% \end{macro}%^^A \SecDef
% \begin{macro}{\scr@sect}
% \changes{v3.10}{2011/08/30}{entfällt}%^^A
% \end{macro}%^^A \scr@sect
% \begin{macro}{\@sect}
% \changes{v3.10}{2011/08/30}{Vollständige Neudefinition zur Realisierung
%   der Erweiterung für das optionale Argument}%^^A
% \changes{v3.10}{2011/08/30}{Verwendung von
%   \cs{csname}\texttt{add\#1tocentry}\cs{endcsname}
%   bzw. \cs{addtocentrydefault}}%^^A
% \changes{v3.18}{2015/05/20}{definiert \cs{IfUsePrefixLine} intern}%^^A
% \changes{v3.18}{2015/05/22}{Verwendung von \cs{numexpr} für Argument 2}%^^A
% \changes{v3.18}{2015/05/22}{Verwendung von \cs{gluexpr} für Argument
%   3--5}%^^A
% \changes{v3.19}{2015/07/17}{Verwendung von \cs{sectionlinesformat} und
%   \cs{sectioncatchphraseformat}}%^^A
% \changes{v3.27}{2019/07/08}{neue Option \texttt{nonumber}}%^^A
% \begin{macro}{\scr@latex@sect}
% \changes{v3.27}{2019/07/24}{Neu (intern für \textsf{scrhack})}%^^A
% Damit auch alle mit \cs{@startsection} definierten Gliederungsebenen
% bei der Entscheidung berücksichtigt werden, muss ein internes Makro
% des \LaTeX-Kerns erweitert werden. Für die Ebenen \cs{part} und
% \cs{chapter} wird der Test direkt in der Definition der Befehle
% erledigt.
%    \begin{macrocode}
%<*body>
\def\scr@latex@sect#1#2#3#4#5#6[#7]#8{%
  \ifnum #2>\c@secnumdepth
    \let\@svsec\@empty
  \else
    \refstepcounter{#1}%
    \protected@edef\@svsec{\@seccntformat{#1}\relax}%
  \fi
  \@tempskipa #5\relax
  \ifdim \@tempskipa>\z@
    \begingroup
      #6{%
        \@hangfrom{\hskip #3\relax\@svsec}%
          \interlinepenalty \@M #8\@@par}%
    \endgroup
    \csname #1mark\endcsname{#7}%
    \addcontentsline{toc}{#1}{%
      \ifnum #2>\c@secnumdepth \else
        \protect\numberline{\csname the#1\endcsname}%
      \fi
      #7}%
  \else
    \def\@svsechd{%
      #6{\hskip #3\relax
      \@svsec #8}%
      \csname #1mark\endcsname{#7}%
      \addcontentsline{toc}{#1}{%
        \ifnum #2>\c@secnumdepth \else
          \protect\numberline{\csname the#1\endcsname}%
        \fi
        #7}}%
  \fi
  \@xsect{#5}}
\ifx\@sect\scr@latex@sect\else
  \let\scr@latex@sect\@sect
  \ClassWarning{\KOMAClassName}{%
    Unexpected definition of \string\@sect!\MessageBreak
    Please send information about this to\MessageBreak
    the KOMA-Script maintainer!\MessageBreak
    Maybe LaTeX will be broken by the redefinition\MessageBreak
    of \string\@sect\space}%
\fi
%    \end{macrocode}
% \end{macro}%^^A \scr@latex@sect
%    \begin{macrocode}
%<trace>\ClassInfo{\KOMAClassName}{redefining LaTeX kernel macro \string\@sect}
\def\@sect#1#2#3#4#5#6[#7]#8{%
  \ifnum \scr@osectarg=\z@
    \@scr@tempswafalse
  \else
    \scr@istest#7=\@nil
  \fi
  \@currentusenumbertrue  
  \if@scr@tempswa
    \setkeys{KOMAarg.section}{tocentry={#8},head={#8},reference={#8},#7}%
  \else
    \ifcase \scr@osectarg\relax
      \setkeys{KOMAarg.section}{tocentry={#7},head={#7},reference={#7}}%
    \or
      \setkeys{KOMAarg.section}{tocentry={#8},head={#7},reference={#8}}%
    \or
      \setkeys{KOMAarg.section}{tocentry={#7},head={#8},reference={#7}}%
    \or
      \setkeys{KOMAarg.section}{tocentry={#7},head={#7},reference={#7}}%
    \fi
  \fi
%    \end{macrocode}
% \changes{v3.23}{2017/02/04}{support for \textsf{minitoc}}%^^A
% \changes{v3.23}{2017/03/24}{usage of \cs{ext@figure} and \cs{ext@table}
%    instead of \texttt{lof} and \texttt{lot}}%^^A
% Add the \texttt{xsect} entries to \texttt{lof} and \texttt{lot}. Note, that
% the original \cs{starsection} tests of \textsf{minitoc} has a bug and so
% does not work. And the original code of \textsf{minitoc} has hard coded
% \texttt{lot} and \texttt{lof} instead of \cs{ext@figure} and
% \cs{ext@table}. This is a fixed one:
%    \begin{macrocode}
  \scr@ifundefinedorrelax{scr@mt@saved@sect}{}{%
    \expandafter\ifx\csname #1\endcsname\section
      \addcontentsline{\ext@figure}{xsect}{\@currenttocentry}%
      \addcontentsline{\ext@table}{xsect}{\@currenttocentry}%
    \fi
    \expandafter\ifx\csname #1\endcsname\starsection\relax
      \addcontentsline{\ext@figure}{xsect}{\@currenttocentry}%
      \addcontentsline{\ext@table}{xsect}{\@currenttocentry}%
    \fi
  }%
  \let\IfUsePrefixLine\@secondoftwo
  \ifcase
    \if@currentusenumber
      \ifnum \numexpr #2\relax>\c@secnumdepth \z@\else \@ne\fi
    \else
      \z@
    \fi
    \let\@svsec\@empty
  \else
    \refstepcounter{#1}%
    \expandafter\@maybeautodot\csname the#1\endcsname
    \protected@edef\@svsec{\@seccntformat{#1}\relax}%
  \fi
%    \end{macrocode}
% \changes{v3.26}{2018/09/18}{\cs{scr@sect@runin} beachten}%^^A
% Falls \cs{scr@sect@runin} nicht definiert ist, wird es jetzt höchste Zeit
% dafür. Und natürlich nutzen wir es dann für die Entscheidung Spitzmarke oder
% nicht.
% \changes{v3.27}{2019/02/02}{\cs{IfUseNumber} lokal definiert}%^^A
%    \begin{macrocode}
  \ifdim
    \scr@sect@runin{\z@}{\p@}{\glueexpr #5\relax}>\z@
    \begingroup
      \ifx\@svsec\@empty
        \let\IfUseNumber\@secondoftwo
      \else
        \let\IfUseNumber\@firstoftwo
      \fi
%    \end{macrocode}
% \changes{v3.21}{2016/06/12}{fehlendes \cs{nobreak} ergänzt}%^^A
% Das Farbänderungen im Schriftargument erlaubt sein sollen und diese aktuelle
% \cs{penalty}-Werte wieder aufheben, verbiete ich hier einen Umbruch nach
% der Fontänderung.
% \changes{v3.27}{2019/02/02}{\cs{scr@do@at} hinzugefügt}%^^A
%    \begin{macrocode}
      \ExecuteDoHook{heading/begingroup/#1}%
      #6{\nobreak\interlinepenalty \@M
        \sectionlinesformat{#1}{\glueexpr #3\relax}\@svsec{#8}\@@par}%
      \ExecuteDoHook{heading/endgroup/#1}%
    \endgroup
%<*trace>
    \ClassInfo{\KOMAClassName}{%
      head=`\detokenize\expandafter{\@currentheadentry}'\MessageBreak
      tocentry=`\detokenize\expandafter{\@currenttocentry}'\MessageBreak
      reference=`\detokenize\expandafter{\@currentlabelname}'}%
%</trace>
    \expandafter\csname #1mark\expandafter\endcsname\expandafter{\@currentheadentry}%
    \ifx\@currenttocentry\@empty\else
      \scr@ifundefinedorrelax{add#1tocentry}{%
        \expandafter\global\expandafter\def
        \csname add#1tocentry\endcsname##1##2{%
          \addtocentrydefault{#1}{##1}{##2}%
        }%
      }{}%
      \ifcase
        \if@currentusenumber
          \ifnum \numexpr #2\relax>\c@secnumdepth \z@\else \@ne\fi
        \else
          \z@
        \fi
        \csname add#1tocentry\endcsname{}{\@currenttocentry}%
      \else
        \csname add#1tocentry\endcsname{\csname the#1\endcsname}{%
          \@currenttocentry}%
      \fi
    \fi
  \else
%    \end{macrocode}
% \changes{v3.21}{2016/06/12}{fehlendes \cs{nobreak} ergänzt}%^^A
% \changes{v3.27}{2019/02/02}{\cs{ExecuteDoHook} eingefügt}%^^A
% Da Farbänderungen im Schriftargument erlaubt sein sollen und diese aktuelle
% \cs{penalty}-Werte wieder aufheben, verbiete ich hier einen Umbruch nach
% der Fontänderung.
%    \begin{macrocode}
    \def\@svsechd{%
      \ifx\@svsec\@empty
        \let\IfUseNumber\@secondoftwo
      \else
        \let\IfUseNumber\@firstoftwo
      \fi
      \ExecuteDoHook{heading/begingroup/#1}%
      #6{\nobreak\sectioncatchphraseformat{#1}{\glueexpr #3\relax}\@svsec{#8}}%
%<*trace>
      \ClassInfo{\KOMAClassName}{%
        head=`\detokenize\expandafter{\@currentheadentry}'\MessageBreak
        tocentry=`\detokenize\expandafter{\@currenttocentry}'\MessageBreak
        reference=`\detokenize\expandafter{\@currentlabelname}'}%
%</trace>
      \expandafter\csname #1mark\expandafter\endcsname
      \expandafter{\@currentheadentry}%
      \ifx\@currenttocentry\@empty\else
        \scr@ifundefinedorrelax{add#1tocentry}{%
          \expandafter\global\expandafter\def
          \csname add#1tocentry\endcsname####1####2{%
            \addtocentrydefault{#1}{####1}{####2}%
          }%
        }{}%
        \ifcase
          \if@currentusenumber
            \ifnum \numexpr #2\relax>\c@secnumdepth \z@\else \@ne\fi
          \else
            \z@
          \fi
          \csname add#1tocentry\endcsname{}{\@currenttocentry}%
        \else
          \csname add#1tocentry\endcsname{\csname the#1\endcsname}{%
            \@currenttocentry}%
        \fi
      \fi
      \ExecuteDoHook{heading/endgroup/#1}%  
    }%
  \fi
  \let\IfUsePrefixLine\scr@IfUsePrefixLineWarning
  \@xsect{\glueexpr #5\relax}%
}
%    \end{macrocode}
% \begin{macro}{\scr@ds@tocentry}
% \changes{v3.10}{2011/08/30}{neu (intern)}%^^A
% \begin{macro}{\@currenttocentry}
% \changes{v3.22}{2016/07/12}{\cs{scr@ds@tocentry} (semi-intern)
%     umbenannt}%^^A
% \begin{macro}{\scr@ds@head}
% \changes{v3.10}{2011/08/30}{neu (intern)}%^^A
% \begin{macro}{\@currentheadentry}
% \changes{v3.22}{2016/07/12}{\cs{scr@ds@head} (semi-intern)
%     umbenannt}%^^A
% \begin{macro}{\@currentlabelname}
% \changes{v3.22}{2016/07/12}{neu (semi-intern)}
% \begin{macro}{\if@scr@tempswa}
% \changes{v3.10}{2011/08/30}{neu (intern)}%^^A
% \begin{macro}{\if@currentusenumber}
% \changes{v3.27}{2019/07/08}{neu (semi-intern)}%^^A
% \begin{macro}{\scr@istest}
% \changes{v3.10}{2011/08/30}{neu (intern)}%^^A
% Für die Realisierung werden noch ein paar Key-Value-Definitionen und ein
% paar Hilfsmakros benötigt:
%    \begin{macrocode}
\DefineFamily{KOMAarg}
\DefineFamilyMember[.section]{KOMAarg}
\providecommand*\@currenttocentry{}
\providecommand*\@currentheadentry{}
\providecommand*\@currentlabelname{}
\FamilyStringKey[.section]{KOMAarg}{tocentry}{\@currenttocentry}
\FamilyStringKey[.section]{KOMAarg}{head}{\@currentheadentry}
\DefineFamilyKey[.section]{KOMAarg}{reference}{%
  \scr@ifundefinedorrelax{NR@gettitle}{%
    \scr@ifundefinedorrelax{GetTitleString}{%
      \def\@currentlabelname{#1}%
    }{%
      \GetTitleString{#1}%
      \let\@currentlabelname\GetTitleStringResult
    }%
  }{%
    \NR@gettitle{#1}%
  }%
  \scr@ifundefinedorrelax{TR@gettitle}{}{%
    \expandafter\TR@gettitle\expandafter{\@currentlabelname}%
  }%
  \scr@ifundefinedorrelax{ztitlerefsetup}{}{%
    \ztitlerefsetup{title=\@currentlabelname}%
  }%
  \FamilyKeyStateProcessed
}
\FamilyInverseBoolKey[.section]{KOMAarg}{nonumber}{@currentusenumber}
\newcommand*{\scr@istest}{}
\def\scr@istest#1=#2\@nil{%
  \ifx\relax#2\relax\@scr@tempswafalse\else\@scr@tempswatrue\fi
}
\newif\if@scr@tempswa
%    \end{macrocode}
% \end{macro}%^^A \scr@istest
% \end{macro}%^^A \if@currentusenumber
% \end{macro}%^^A \if@scr@tempswa
% \end{macro}%^^A \@currentlabelname
% \end{macro}%^^A \@currentheadentry
% \end{macro}%^^A \scr@ds@head
% \end{macro}%^^A \@currenttocentry
% \end{macro}%^^A \scr@ds@tocentry
% \begin{macro}{\sectionlinesformat}
% \changes{v3.19}{2015/07/17}{neue Anweisung}%^^A
% Diese Anweisung definiert das Format der eigentlichen Überschriften im Stil
% \texttt{section}, wenn mit Überschriftenzeilen gearbeitet
% wird. Fonteinstellungen sind zu diesem Zeitpunkt bereits erfolgt. Die
% Argumente sind:
% \begin{description}
% \item[\meta{Ebene} --] der Name der Gliederungebene
% \item[\meta{Einzug} --] ein eventuell zu verwendender horizontaler
% Einzug.
% \item[\meta{Gliederungsnummer} --] die bereits fertig formatierte
%   Gliederungsnummer oder leer, falls keine Gliederungsnummer auszugeben ist.
% \item[\meta{Text} --] der Text der Überschrift.
% \end{description}
% Der Anwender ist selbst verantwortlich, dass innerhalb der Überschrift kein
% Seitenumbruch erfolgen kann. Nach der Überschrift wird jedoch zwangsweise
% ein \cs{@@par} ausgeführt, so dass sichergestellt ist, dass mit einem
% internen Absatz abgeschlossen wird.
%    \begin{macrocode}
\newcommand{\sectionlinesformat}[4]{%
  \@hangfrom{\hskip #2#3}{#4}%
}
%    \end{macrocode}
% \end{macro}%^^A \sectionlinesformat
% \begin{macro}{\sectioncatchphraseformat}
% \changes{v3.19}{2015/07/17}{neue Anweisung}%^^A
% Diese Anweisung definiert das Format der eigentlichen Überschriften im
% Stil \texttt{section}, wenn mit Spitzmarken gearbeitet
% wird. Fonteinstellungen sind zu diesem Zeitpunkt bereits erfolgt. Die
% Argumente sind:
% \begin{description}
% \item[\meta{Ebene} --] der Name der Gliederungebene
% \item[\meta{Einzug} --] ein eventuell zu verwendender horizontaler
% Einzug.
% \item[\meta{Gliederungsnummer} --] die bereits fertig formatierte
%   Gliederungsnummer oder leer, falls keine Gliederungsnummer auszugeben
%   ist.
% \item[\meta{Text} --] der Text der Überschrift.
% \end{description}
% Der Anwender ist selbst verantwortlich, dass innerhalb der Überschrift kein
% Seitenumbruch erfolgen kann.
%    \begin{macrocode}
\newcommand{\sectioncatchphraseformat}[4]{%
  \hskip #2#3#4%
}
%    \end{macrocode}
% \end{macro}%^^A \sectioncatchphraseformat
% \end{macro}%^^A \@sect
%
% \begin{macro}{\@ssect}
% \changes{v3.19}{2015/07/17}{Vollständige Neudefinition zur Realisierung
%   der Erweiterung für die Formatieurng der Überschrift}%^^A
% \begin{macro}{\scr@latex@ssect}
% \changes{v3.27}{2019/07/24}{Neu (intern für \textsf{scrhack})}%^^A
% Das wird für Überschriften ohne Nummer verwendet. Im Prinzip machen wir hier
% dasselbe wie zuvor.
%    \begin{macrocode}
\def\scr@latex@ssect#1#2#3#4#5{%
  \@tempskipa #3\relax
  \ifdim \@tempskipa>\z@
    \begingroup
      #4{%
        \@hangfrom{\hskip #1}%
          \interlinepenalty \@M #5\@@par}%
    \endgroup
  \else
    \def\@svsechd{#4{\hskip #1\relax #5}}%
  \fi
  \@xsect{#3}}
\ifx\@ssect\scr@latex@ssect
\else
  \let\scr@latex@ssect\@ssect
  \ClassWarning{\KOMAClassName}{%
    Unexpected definition of \string\@ssect!\MessageBreak
    Please send information about this to\MessageBreak
    the KOMA-Script maintainer!\MessageBreak
    Maybe LaTeX will be broken by the redefinition\MessageBreak
    of \string\@ssect\space}%
\fi
%    \end{macrocode}
% \end{macro}%^^A \scr@latex@ssect
%    \begin{macrocode}
%<trace>\ClassInfo{\KOMAClassName}{redefining LaTeX kernel macro
%<trace>  \string\@ssect}
\def\@ssect#1#2#3#4#5{%
  \scr@ifundefinedorrelax{scr@s@ct@@nn@m@}{%
    \ClassWarning{\KOMAClassName}{Incompatible usage of
      \string\@ssect\space detected.\MessageBreak
      You've used the KOMA-Script implementation of
      \string\@ssect\MessageBreak
      from within a non compatible caller, that does not\MessageBreak
      \string\scr@s@ct@@nn@m@\space locally.\MessageBreak
      This could result in several error messages}%
    \def\scr@s@ct@@nn@m@{\string\scr@s@ct@@nn@m@}%
  }{}%
%    \end{macrocode}
% \changes{v3.26}{2018/09/18}{\cs{scr@sect@runin} beachten}%^^A
% Falls \cs{scr@sect@runin} nicht definiert ist, wird es jetzt höchste Zeit
% dafür. Und natürlich nutzen wir es dann für die Entscheidung Spitzmarke oder
% nicht.
% \changes{v3.27}{2019/02/02}{\cs{IfUseNumber} lokal definiert}%^^A
%    \begin{macrocode}
  \ifdim
    \scr@sect@runin{\z@}{\p@}{\glueexpr #3\relax}>\z@
    \begingroup
      \let\IfUseNumber\@secondoftwo
%    \end{macrocode}
% \changes{v3.21}{2016/06/12}{fehlendes \cs{nobreak} ergänzt}%^^A
% \changes{v3.27}{2019/02/02}{\cs{ExecuteDoHook} eingefügt}%^^A
% Das Farbänderungen im Schriftargument erlaubt sein sollen und diese aktuelle
% \cs{penalty}-Werte wieder aufheben, verbiete ich hier einen Umbruch nach
% der Fontänderung.
%    \begin{macrocode}
      \edef\reserved@a{%
        \noexpand\ExecuteDoHook{heading/begingroup/\scr@s@ct@@nn@m@}%
      }\reserved@a
      #4{\nobreak\interlinepenalty \@M
        \expandafter\sectionlinesformat\expandafter{\scr@s@ct@@nn@m@}%
        {\glueexpr #1\relax}\@empty{#5}\@@par}%
      \edef\reserved@a{%
        \noexpand\ExecuteDoHook{heading/endgroup/\scr@s@ct@@nn@m@}%
      }\reserved@a
    \endgroup
  \else
    \edef\@svsechd{%
      \unexpanded{\let\IfUseNumber\@secondoftwo}%
      \noexpand\ExecuteDoHook{heading/begingroup/\scr@s@ct@@nn@m@}%
      \unexpanded{#4}{%
%    \end{macrocode}
% \changes{v3.21}{2016/06/12}{fehlendes \cs{nobreak} ergänzt}%^^A
% Das Farbänderungen im Schriftargument erlaubt sein sollen und diese aktuelle
% \cs{penalty}-Werte wieder aufheben, verbiete ich hier einen Umbruch nach
% der Fontänderung.
%    \begin{macrocode}
        \noexpand\nobreak
        \noexpand\sectioncatchphraseformat{\scr@s@ct@@nn@m@}%
        \unexpanded{{\glueexpr #1\relax}\@empty{#5}}%
      }%
      \noexpand\ExecuteDoHook{heading/endgroup/\scr@s@ct@@nn@m@}%
    }%
  \fi
  \let\scr@s@ct@@nn@m@\relax
  \@xsect{\glueexpr #3\relax}%
}
%</body>
%    \end{macrocode}
% \end{macro}%^^A \@ssect
% \begin{macro}{\@xsect}
% \changes{v3.26}{2018/09/18}{wird umdefiniert}%^^A
% \begin{macro}{\scr@latex@xsect}
% \changes{v3.27}{2019/07/24}{Neu (intern für \textsf{scrhack})}%^^A
% Wir brauchen zum Einen in \cs{@xsect} einen Fallunterscheidung nach
% \cs{scr@sect@runin} und zum Anderen muss diese Fallunterscheidung am Ende
% wieder zurückgesetzt werden.
%    \begin{macrocode}
%<*body>
\def\scr@latex@xsect#1{%
  \@tempskipa #1\relax
  \ifdim \@tempskipa>\z@
    \par \nobreak
    \vskip \@tempskipa
    \@afterheading
  \else
    \@nobreakfalse
    \global\@noskipsectrue
    \everypar{%
      \if@noskipsec
        \global\@noskipsecfalse
       {\setbox\z@\lastbox}%
        \clubpenalty\@M
        \begingroup \@svsechd \endgroup
        \unskip
        \@tempskipa #1\relax
        \hskip -\@tempskipa
      \else
        \clubpenalty \@clubpenalty
        \everypar{}%
      \fi}%
  \fi
  \ignorespaces
}
\ifx\@xsect\scr@latex@xsect\else
  \let\scr@latex@xsect\@xsect
  \ClassWarning{\KOMAClassName}{%
    Unexpected definition of \string\@xsect!\MessageBreak
    Please send information about this to\MessageBreak
    the KOMA-Script maintainer!\MessageBreak
    Maybe LaTeX will be broken by the redefinition\MessageBreak
    of \string\@xsect\space}%
\fi
%    \end{macrocode}
% \end{macro}%^^A \scr@latex@xsect
%    \begin{macrocode}
%<trace>\ClassInfo{\KOMAClassName}{redefining LaTeX kernel macro
%<trace> \string\@xsect}
\def\@xsect#1{%
  \@ifundefined{scr@sect@runin}{%
    \def\scr@sect@runin##1##2##3{##3}%
  }%
  \@tempskipa #1\relax
  \ifdim \scr@sect@runin{\z@}{\p@}{\@tempskipa}>\z@
    \par \nobreak
    \vskip \@tempskipa
    \@afterheading
  \else
    \@nobreakfalse
    \global\@noskipsectrue
    \everypar{%
      \if@noskipsec
        \global\@noskipsecfalse
       {\setbox\z@\lastbox}%
        \clubpenalty\@M
        \begingroup \@svsechd \endgroup
        \unskip
        \@tempskipa #1\relax
        \ifdim \@tempskipa<\z@
          \hskip -\@tempskipa
        \else
          \hskip \@tempskipa
        \fi
      \else
        \clubpenalty \@clubpenalty
        \everypar{}%
      \fi}%
  \fi
  \def\scr@sect@runin##1##2##3{##3}%
  \ignorespaces
}
%</body>
%    \end{macrocode}
% \end{macro}%^^A \@xsect
%
% \begin{macro}{\IfUsePrefixLine}
% \changes{v3.18}{2015/06/09}{außerhalb von Gliederungsbefehlsstilen
%     definiert}%^^A
% Eigentlich ist diese Anweisung nur innerhalb von Gliederungsbefehlen
% sinnvoll und wurde bis \KOMAScript{} 3.17c auch nur dort definiert. Dann gab
% es aber Leute, die \cs{chapterformat} missbraucht haben und gleich meldete
% sich jemand, die nicht verstehen konnte, warum die Anweisung dann einen
% Fehler produzieren muss. Wenn man den Code nicht verstanden hat, dann ist
% das nämlich so. Deshalb gebe ich jetzt eine Warnung aus, dass die Verwendung
% nicht unbedingt zu dem Resultat führt, das erwartet wird.
% \begin{macro}{\scr@IfUsePrefixLineWarning}
% \changes{v3.18}{2015/06/09}{neu (intern)}
%    \begin{macrocode}
%<*body>
\newcommand*{\scr@IfUsePrefixLineWarning}[2]{%
  \ClassWarning{\KOMAClassName}{%
    Usage of \string\IfUsePrefixLine\space outside
    section command\MessageBreak
    style makes no sense, because the behaviour not only\MessageBreak
    depends on class options but also on the run-time\MessageBreak
    section command style.\MessageBreak
    Neither the then-code nor the else-code will be\MessageBreak
    executed.\MessageBreak
    You may change this and avoid the warning by LOCALLY\MessageBreak
    setting \string\IfUsePrefixLine\space to either
    \string\@firstoftwo\space or\MessageBreak
    \string\@secondoftwo\space before using it%
  }%
}
\newcommand*{\IfUsePrefixLine}[2]{}
\let\IfUsePrefixLine\scr@IfUsePrefixLineWarning
%</body>
%    \end{macrocode}
% \end{macro}%^^A \scr@IfUsePrefixLineWarning
% \end{macro}%^^A \IfUsePrefixLine
%
%
% \begin{macro}{\raggedsection}
% \changes{v2.3h}{1996/01/20}{Überschriften werden \cs{raggedsection}%^^A
%     gesetzt}%^^A 
% \changes{v2.3h}{1996/01/20}{\cs{raggedsection} ist als \cs{raggedright}
%     voreingestellt}%^^A
% Dieses Makro gibt an, wie die Überschriften formatiert
% werden. Voreingestellt ist linksbündig. Da dies nicht nur die Formatierung
% von \cs{section} (und darunter), sondern auch die Voreinstellung für die
% höheren Ebenen ist, wird die Definition hier vorgezogen.
%    \begin{macrocode}
%<*body>
\newcommand*{\raggedsection}{}
\let\raggedsection\raggedright
%</body>
%    \end{macrocode}
% \end{macro}%^^A \raggedsection
%
%
% \begin{macro}{\DeclareSectionCommand}
% \changes{v3.15}{2014/11/21}{neue Anweisung}%^^A
% \changes{v3.17}{2015/03/23}{Komplettumbau, um stilabhängige Optionen zu
%     ermöglichen}%^^A
% \changes{v3.18}{2015/05/22}{\cs{scr@dsc@def@style@\dots @command}
%     definiert alle Stil abhängigen Anweisungen}%^^A
% \changes{v3.20}{2015/10/03}{\cs{DeclareTOCStyleEntry}
%     definiert die Verzeichnisstil abhängigen Anweisungen}%^^A
% Definition neuer oder Umdefinition existierender Gliederungsbefehle. Ebenso
% wird ein Verzeichniseintrag erzeugt, falls ein solcher noch nicht
% existiert. Die Deklaration geschieht über diverse Optionen:
%^^A (Achtung: Unbedingt daran denken, die Optionen in
%^^A  \cs{ProvideSectionCommands} ebenfalls zu definieren!!!)
%    \begin{macrocode}
%<*body>
\DefineFamily{KOMAarg}
\DefineFamilyMember[.dsc]{KOMAarg}
%    \end{macrocode}
% \begin{macro}{\DeclareSectionCommandStyleOption}
% \changes{v3.17}{2015/03/23}{neue Anweisung}%^^A
% Definiert eine neue Option für einen Stil eines Gliederungsbefehls. Das
% erste Argument ist der Stil, das zweite der Name der Option, das dritte ist
% die Definition. Sollte innerhalb der Definition eine Familie benötigt
% werden, so ist \texttt{KOMAarg} zu verwenden.
%    \begin{macrocode}
\newcommand*{\DeclareSectionCommandStyleOption}[3]{%
  \AddToSectionCommandOptionsDoList{#2}%
  \expandafter\g@addto@macro\csname scr@dsc@style@#1@options\endcsname{%
    \DefineFamilyKey[.dsc]{KOMAarg}{#2}{#3}%
  }%
}
%    \end{macrocode}
% \begin{macro}{\DeclareSectionCommandDummyOption}
% \changes{v3.17}{2015/03/23}{neue Anweisung (intern)}%^^A
% \changes{v3.20}{2015/11/18}{wird nicht mehr benötigt}%^^A
% \end{macro}%^^A \DeclareSectionCommandDummyOption
% \begin{macro}{\AddToSectionCommandOptionsDoList}
% \changes{v3.20}{2015/11/18}{neu}
% \begin{macro}{\@AddToSectionCommandOptionsDoList}
% \changes{v3.20}{2015/11/18}{neu (intern)}
% \begin{macro}{\scr@dsc@doopts}
% \changes{v3.20}{2015/11/18}{neu (intern)}
% Registriert eine neue Gliederungsbefehlsoption, indem sie der internen Liste
% der Gliederungsbefehlsoptionen hinzugefügt wird.
%    \begin{macrocode}
\newcommand*{\AddToSectionCommandOptionsDoList}[1]{%
  \kernel@ifnextchar [%]
    {\@AddToSectionCommandOptionListWithDefault{#1}}%
    {\l@addto@macro\scr@dsc@doopts{\do{#1}}}%
}
\newcommand*{\@AddToSectionCommandOptionListWithDefault}{}
\def\@AddToSectionCommandOptionListWithDefault#1[#2]{%
  \l@addto@macro\scr@dsc@doopts{\do[{#2}]{#1}}
}%
\newcommand*{\scr@dsc@doopts}{}
%    \end{macrocode}
% \end{macro}%^^A \scr@dsc@dooptions
% \end{macro}%^^A \@AddToSectionCommandOptionsListWithDefault
% \end{macro}%^^A \AddToSectionCommandOptionsDoList
% \begin{macro}{\RelaxSectionCommandOptions}
% \changes{v3.20}{2015/11/18}{neu}
% Diese Option macht alle Gliederungsbefehlsoptionen ungültig.
%    \begin{macrocode}
\newcommand*{\RelaxSectionCommandOptions}{%
  \begingroup
    \def\do@endgroup{\endgroup}%
    \def\do{%
      \kernel@ifnextchar [%]
        {\@do}%
        {\@do[]}%
    }%
    \def\@do[##1]##2{%
      \l@addto@macro\do@endgroup{\RelaxFamilyKey[.dsc]{KOMAarg}{##2}}%
    }%
    \scr@dsc@doopts
  \do@endgroup
}
%    \end{macrocode}
% \end{macro}%^^A \RelaxSectionCommandOptions
% \end{macro}%^^A \DeclareSectionCommandStyleOption
% \begin{macro}{\DeclareSectionCommandStyleLengthOption}
% \changes{v3.17}{2015/03/23}{neue Anweisung}%^^A
% \changes{v3.20}{2016/04/25}{Argument wird ggf. expandiert}%^^A
% Definiert eine für die aktuelle Ebene \cs{scr@dsc@current} eine neue
% Längenoption und speichert das Ergebnis in einem Makro, das sich aus
% Argument 3 gefolgt von der aktuellen Ebene gefolgt von Argument 4 ergibt.
%    \begin{macrocode}
\newcommand*{\DeclareSectionCommandStyleLengthOption}[4]{%
  \DeclareSectionCommandStyleOption{#1}{#2}{%
    \protected@edef\reserved@a{%
      \noexpand\FamilySetLength{KOMAarg}{#2}{\noexpand\@tempskipa}{##1}%
    }\reserved@a
    \ifx\FamilyKeyState\FamilyKeyStateProcessed
      \ifscr@dsc@expandtopt
        \expandafter\edef\csname #3\scr@dsc@current#4\endcsname{%
          \the\@tempskipa}%
      \else
        \expandafter\edef\csname #3\scr@dsc@current#4\endcsname{##1}%
      \fi
    \fi
  }%
}
%    \end{macrocode}
% \end{macro}%^^A \DeclareSectionCommandStyleLengthOption
% \begin{macro}{\DeclareSectionCommandStyleNumberOption}
% \changes{v3.17}{2015/03/23}{neue Anweisung}%^^A
% \changes{v3.20}{2016/04/25}{Argument wird expandiert}%^^A
% Definiert eine für die aktuelle Ebene \cs{scr@dsc@current} eine neue
% Zahlenoption und speichert das Ergebnis in einem Makro, das sich aus
% Argument 3 gefolgt von der aktuellen Ebene gefolgt von Argument 4 ergibt.
%    \begin{macrocode}
\newcommand*{\DeclareSectionCommandStyleNumberOption}[4]{%
  \DeclareSectionCommandStyleOption{#1}{#2}{%
    \protected@edef\reserved@a{%
      \noexpand\FamilySetCounterMacro{KOMAarg}{#2}{\noexpand\reserved@a}{##1}
    }\reserved@a
    \ifx\FamilyKeyState\FamilyKeyStateProcessed
      \expandafter\let\csname #3\scr@dsc@current#4\endcsname\reserved@a
    \fi
  }%
}
%    \end{macrocode}
% \end{macro}%^^A \DeclareSectionCommandStyleNumberOption
% \begin{macro}{\DeclareSectionCommandStyleFontOption}
% \changes{v3.17}{2015/03/23}{neue Anweisung}%^^A
% Definiert für die aktuelle Ebene \cs{scr@dsc@current} eine neue
% Schrift und speichert das Ergebnis in einem Element, das sich aus
% Argument 3 gefolgt von der aktuellen Ebene gefolgt von Argument 4 ergibt.
%    \begin{macrocode}
\newcommand*{\DeclareSectionCommandStyleFontOption}[4]{%
  \DeclareSectionCommandStyleOption{#1}{#2}{%
    \IfExistskomafont{#3\scr@dsc@current#4}{%
      \setkomafont
    }{%
      \newkomafont
    }{#3\scr@dsc@current#4}{##1}%
    \FamilyKeyStateProcessed
  }%
}
%    \end{macrocode}
% \end{macro}%^^A \DeclareSectionCommandStyleFontOption
% \begin{macro}{\DeclareSectionCommandStyleFuzzyOption}
% \changes{v3.26}{2018/09/18}{neue Anweisung}%^^A
% \changes{v3.28}{2019/11/18}{\cs{ifstr} umbenannt in \cs{Ifstr}}%^^A
% Definiert für die aktuelle Ebene \cs{scr@dsc@current} eine neue
% Fuzzy-Auswahl. Eine Fuzzy-Auswahl akzeptiert neben den Werten für
% boolsche-Schalter noch einen weiteren Wert. Es wird davon eine Anweisung
% mit drei Argumenten definiert. Abhängig vom Wert wird das erste
% (\texttt{true}), zweite (\texttt{false}) oder dritte (\emph{sonst}) Argument
% ausgeführt. Der erlaubte weitere Wert wird als 5. Argument angegeben.
%    \begin{macrocode}
\newcommand*{\DeclareSectionCommandStyleFuzzyOption}[5]{%
  \DeclareSectionCommandStyleOption{#1}{#2}{%
    \FamilySetBool{KOMAarg}{#2}{@tempswa}{##1}%
    \ifx\FamilyKeyState\FamilyKeyStateProcessed
      \if@tempswa
        \expandafter\def\csname #3\scr@dsc@current#4\endcsname
        ####1####2####3{####1}%
      \else
        \expandafter\def\csname #3\scr@dsc@current#4\endcsname
        ####1####2####3{####2}%
      \fi
    \else
      \Ifstr{##1}{#5}{%
        \expandafter\def\csname #3\scr@dsc@current#4\endcsname
        ####1####2####3{####3}%
        \FamilyKeyStateProcessed
      }{}%
    \fi
  }%
}
%    \end{macrocode}
% \end{macro}%^^A \DeclareSectionCommandStyleFuzzyOptions
%
% \begin{macro}{\scr@dsc@style@section@options}
% \changes{v3.17}{2015/03/23}{neue Anweisung (intern)}%^^A
% Dann für die Optionen des Stils \texttt{section}:
%    \begin{macrocode}
\newcommand*{\scr@dsc@style@section@options}{}
%    \end{macrocode}
% \begin{description}\item[\texttt{indent}]
%   Linker Einzug der Überschrift
% \end{description}
%    \begin{macrocode}
\DeclareSectionCommandStyleLengthOption{section}
                                       {indent}{scr@}{@sectionindent}
%    \end{macrocode}
% \begin{description}
% \item[\texttt{afterindent}]
% \changes{v3.26}{2018/09/18}{neue Option \texttt{afterindent}}%^^A
%   Fuzzy-Logik. Ein boolcher Wert gibt an, ob die erste Zeile nach der
%   Überschrift eingezogen werden soll. Als dritter Wert ist \texttt{bysign}
%   möglich. Dann hängt die Entscheidung für den Einzug vom Vorzeichen von
%   \texttt{beforeskip} ab. Ein negativer Wert bedeutet dann, dass der
%   Absatzeinzug nach der Überschrift unterdrückt werden soll.
% \end{description}
%    \begin{macrocode}
\DeclareSectionCommandStyleFuzzyOption{section}
                                      {afterindent}{scr@}{@afterindent}
                                      {bysign}
%    \end{macrocode}
% \begin{description}\item[\texttt{beforeskip}]
%   Gibt den vertikalen Abstand vor der Übeschrift an. Im Fall von
%   \texttt{afterindent=bysign} wird der Absolutwert verwendet.
% \end{description}
%    \begin{macrocode}
\DeclareSectionCommandStyleLengthOption{section}
                                       {beforeskip}{scr@}{@beforeskip}
%    \end{macrocode}
% \begin{description}\item[\texttt{runin}]
% \changes{v3.26}{2018/09/18}{neue Option \texttt{runin}}%^^A
%   Fuzzy-Logik. Ein boolscher Wert gibt an, dass die Überschrift als
%   Spitzmarke (\texttt{true}) oder als eigene Zeile (\texttt{false}) gesetzt
%   werden soll. Als dritter Wert ist \texttt{bysign} möglich. Dann hängt
%   Spitzmarke oder nicht vom Vorzeichen von \texttt{afterskip} ab. Spitzmarke
%   gibt es dann bei einem negativen Wert von \texttt{afterskip}.
% \end{description}
%    \begin{macrocode}
\DeclareSectionCommandStyleFuzzyOption{section}
                                      {runin}{scr@}{@runin}
                                      {bysign}
%    \end{macrocode}
% \begin{description}\item[\texttt{afterskip}] Im Fall einer Spitzmarke (siehe
% \texttt{runin}) der horizontale Abstand nach der Überschrift. Im Fall einer
% freistehenden Überschrift der vertikale Abstand nach der Überschrift. Falls
% die Spitzmarke über das Vorzeichen aktiviert wird, wird immer der Betrag des
% Wertes verwendet!
% \end{description}
%    \begin{macrocode}
\DeclareSectionCommandStyleLengthOption{section}
                                       {afterskip}{scr@}{@afterskip}
%    \end{macrocode}
% \begin{description}\item[\texttt{font}] Schrifteinstellungen für das
%   Element.\end{description}
%    \begin{macrocode}
\DeclareSectionCommandStyleFontOption{section}{font}{}{}
%    \end{macrocode}
% \end{macro}%^^A \scr@dsc@style@section@options
% \begin{macro}{\scr@dsc@def@style@section@command}
% \changes{v3.18}{2015/05/21}{neue Anweisung (intern)}%^^A
% \changes{v3.20}{2016/01/19}{korrekte Spitzmarkenprüfung für
%     \cs{parfillskip}-Änderung}%^^A
% Diese Anweisung left fest, wie ein Gliederungsbefehl für den Stil
% \texttt{section} zu definieren ist. Das einzige Argument ist dabei der
% \meta{Name} der Gliederungsebene, beispielsweise \texttt{section},
% \texttt{subsection} etc. Hier ist es recht einfach, da lediglich
% \cs{\meta{Name}} per \cs{scr@startsection} definiert werden muss.
%    \begin{macrocode}
\newcommand*{\scr@dsc@def@style@section@command}[1]{%
  \@namedef{#1}{%
    \scr@startsection{#1}%
    {\csname #1numdepth\endcsname}%
    {\csname scr@#1@sectionindent\endcsname}%
    {\csname scr@#1@beforeskip\endcsname}%
    {\csname scr@#1@afterskip\endcsname}%
    {%
      \ifdim\glueexpr\csname scr@#1@afterskip\endcsname >\z@
        \expandafter\ifnum\scr@v@is@gt{2.96}\relax
          \setlength{\parfillskip}{\z@ plus 1fil}%
        \fi
      \fi
      \raggedsection\normalfont\sectfont\nobreak
      \usekomafont{#1}%
    }%
  }%
}
%    \end{macrocode}
% \end{macro}%^^A \scr@dsc@def@style@section@command
%
% \begin{macro}{\scr@dsc@style@chapter@options}
% \changes{v3.17}{2015/03/23}{neue Anweisung (intern)}%^^A
% Dann für die Optionen des Stils \texttt{chapter}:
%    \begin{macrocode}
%<*book|report>
\newcommand*{\scr@dsc@style@chapter@options}{}
%    \end{macrocode}
% \begin{description}
% \item[\texttt{pagestyle}] Seitenstil für die Anfangsseite der
%   Gliederungsebene.
% \end{description}
%    \begin{macrocode}
\DeclareSectionCommandStyleOption{chapter}{pagestyle}{%
  \@namedef{\scr@dsc@current pagestyle}{#1}%
  \FamilyKeyStateProcessed
}%
%    \end{macrocode}
% \begin{description}
% \item[\texttt{afterindent}]
% \changes{v3.26}{2018/09/18}{neue Option \texttt{afterindent}}%^^A
%   Fuzzy-Logik. Ein boolcher Wert gibt an, ob die erste Zeile nach der
%   Überschrift eingezogen werden soll. Als dritter Wert ist \texttt{bysign}
%   möglich. Dann hängt die Entscheidung für den Einzug vom Vorzeichen von
%   \texttt{beforeskip} ab. Ein negativer Wert bedeutet dann, dass der
%   Absatzeinzug nach der Überschrift unterdrückt werden soll.
% \end{description}
%    \begin{macrocode}
\DeclareSectionCommandStyleFuzzyOption{chapter}
                                      {afterindent}{scr@}{@afterindent}
                                      {bysign}
%    \end{macrocode}
% \begin{description}\item[\texttt{beforeskip}]
%   Gibt den vertikalen Abstand vor der Übeschrift an. Im Fall von
%   \texttt{afterindent=bysign} wird der Absolutwert verwendet.
% \end{description}
%    \begin{macrocode}
\DeclareSectionCommandStyleLengthOption{chapter}
                                       {beforeskip}{scr@}{@beforeskip}
%    \end{macrocode}
% \begin{description}
% \item[\texttt{innerskip}] Der Absolutwert gibt den vertikalen Abstand
%   innerhalb der Überschrift zwischen Präfix-Zeile und Text an.
% \end{description}
%    \begin{macrocode}
\DeclareSectionCommandStyleLengthOption{chapter}
                                       {innerskip}{scr@}{@innerskip}
%    \end{macrocode}
% \begin{description}
% \item[\texttt{afterskip}] Der Absolutwert gibt den vertikalen Abstand nach
%   der Überschrift an.
% \end{description}
%    \begin{macrocode}
\DeclareSectionCommandStyleLengthOption{chapter}
                                       {afterskip}{scr@}{@afterskip}
%    \end{macrocode}
% \begin{description}
% \item[\texttt{font}] Schrifteinstellungen für das Element.
% \end{description}
%    \begin{macrocode}
\DeclareSectionCommandStyleFontOption{chapter}{font}{}{}
%    \end{macrocode}
% \begin{description}
% \item[\texttt{prefixfont}] Schrifteinstellungen für die Präfixzeile des
%   Elements.
% \end{description}
%    \begin{macrocode}
\DeclareSectionCommandStyleFontOption{chapter}{prefixfont}{}{prefix}
%    \end{macrocode}
% \end{macro}%^^A \scr@dsc@style@chapter@options
% \begin{macro}{\scr@dsc@def@style@chapter@command}
% \changes{v3.18}{2015/05/22}{neue Anweisung (intern)}
% Diese Anweisung left fest, wie ein Gliederungsbefehl für den Stil
% \texttt{chapter} zu definieren ist. Das einzige Argument ist dabei der
% \meta{Name} der Gliederungsebene, normalerweise \texttt{chapter}.
% Das ist etwas komplizierter als bei \cs{scr@dsc@def@style@section@command},
% da hier eine ganze Reihe von miteinander verschachtelten Anweisungen zu
% definieren sind.
%    \begin{macrocode}
\newcommand*{\scr@dsc@def@style@chapter@command}[1]{%
  \@namedef{#1}{\scr@startchapter{#1}}%
  \@namedef{@#1}{\scr@@startchapter{#1}}%
  \@namedef{@s#1}{\scr@@startschapter{#1}}%
  \@namedef{@make#1head}{\scr@makechapterhead{#1}}%
  \@namedef{@makes#1head}{\scr@makeschapterhead{#1}}%
  \@namedef{@@make#1head}{\scr@@makechapterhead{#1}}%
  \@namedef{@@makes#1head}{\scr@@makeschapterhead{#1}}%
  \@namedef{set#1preamble}{\set@preamble{#1}}%
}
%</book|report>
%    \end{macrocode}
% \end{macro}%^^A \scr@dsc@def@style@chapter@command
%
% \begin{macro}{\scr@dsc@style@part@options}
% \changes{v3.17}{2015/03/23}{neue Anweisung (intern)}%^^A
% Dann für die Optionen des Stils \texttt{part}:
%    \begin{macrocode}
\newcommand*{\scr@dsc@style@part@options}{}
%    \end{macrocode}
% \begin{description}
% \item[\texttt{pagestyle}] Seitenstil für die Anfangsseite der
%   Gliederungsebene.
% \end{description}
%    \begin{macrocode}
%<*report|book>
\DeclareSectionCommandStyleOption{part}{pagestyle}{%
  \@namedef{\scr@dsc@current pagestyle}{#1}%
  \FamilyKeyStateProcessed
}%
%</report|book>
%    \end{macrocode}
% \begin{description}
% \item[\texttt{afterindent}]
% \changes{v3.26}{2018/09/18}{neue Option \texttt{afterindent}}%^^A
%   Fuzzy-Logik. Ein boolcher Wert gibt an, ob die erste Zeile nach der
%   Überschrift eingezogen werden soll. Als dritter Wert ist \texttt{bysign}
%   möglich. Dann hängt die Entscheidung für den Einzug vom Vorzeichen von
%   \texttt{beforeskip} ab. Ein negativer Wert bedeutet dann, dass der
%   Absatzeinzug nach der Überschrift unterdrückt werden soll.
% \end{description}
%    \begin{macrocode}
\DeclareSectionCommandStyleFuzzyOption{part}
                                      {afterindent}{scr@}{@afterindent}
                                      {bysign}
%    \end{macrocode}
% \begin{description}\item[\texttt{beforeskip}]
%   Gibt den vertikalen Abstand vor der Übeschrift an. Im Fall von
%   \texttt{afterindent=bysign} wird der Absolutwert verwendet.
% \end{description}
%    \begin{macrocode}
\DeclareSectionCommandStyleLengthOption{part}
                                       {beforeskip}{scr@}{@beforeskip}
%    \end{macrocode}
% \begin{description}
% \item[\texttt{innerskip}] Der Absolutwert gibt den vertikalen Abstand
%   innerhalb der Überschrift zwischen Präfix-Zeile und Text an.
% \end{description}
%    \begin{macrocode}
%<*book|report>
\DeclareSectionCommandStyleLengthOption{part}
                                       {innerskip}{scr@}{@innerskip}
%</book|report>
%    \end{macrocode}
% \begin{description}
% \item[\texttt{afterskip}] Der Absolutwert gibt den vertikalen Abstand nach
%   der Überschrift an.
% \end{description}
%    \begin{macrocode}
\DeclareSectionCommandStyleLengthOption{part}
                                       {afterskip}{scr@}{@afterskip}
%    \end{macrocode}
% \begin{description}
% \item[\texttt{font}] Schrifteinstellungen für das Element.
% \end{description}
%    \begin{macrocode}
\DeclareSectionCommandStyleFontOption{part}{font}{}{}
%    \end{macrocode}
% \begin{description}
% \item[\texttt{font}] Schrifteinstellungen für die Präfixzeile des Elements.
% \end{description}
%    \begin{macrocode}
\DeclareSectionCommandStyleFontOption{part}{prefixfont}{}{prefix}
%    \end{macrocode}
% \end{macro}%^^A \scr@dsc@style@part@options
% \begin{macro}{\scr@dsc@def@style@part@command}
% \changes{v3.18}{2015/05/23}{neue Anweisung (intern)}
% Diese Anweisung legt fest, wie ein Gliederungsbefehl für den Stil
% \texttt{part} zu definieren ist. Das einzige Argument ist dabei der
% \meta{Name} der Gliederungsebene, normalerweise \texttt{part}.
% Das ist ganz ähnlich wie bei \cs{scr@dsc@def@style@chapter@command}.
%    \begin{macrocode}
\newcommand*{\scr@dsc@def@style@part@command}[1]{%
  \@namedef{#1}{\scr@startpart{#1}}%
  \@namedef{@#1}{\scr@@startpart{#1}}%
  \@namedef{@s#1}{\scr@@startspart{#1}}%
%<book|report>  \@namedef{@end#1}{\scr@@endpart{#1}}%
%<book|report>  \@namedef{set#1preamble}{\set@preamble{#1}}%
}
%    \end{macrocode}
% \end{macro}%^^A \scr@dsc@def@style@part@command
%
% \begin{macro}{\scr@dsc@style@section@neededoptionstest}
% \changes{v3.17}{2015/03/24}{Neu (intern)}%^^A
% \changes{v3.24}{2017/04/25}{Voreinstellungen für \texttt{indent} und
%   \texttt{font}}%^^A
% \changes{v3.26}{2018/09/18}{\texttt{runin} und \texttt{afterindent}
%   beachten}%^^A
% Dieses Anweisung überprüft, ob alle notwendigen Eigenschaften für die
% Definition der spezifischen Gliederungsebene im section-Stil bekannt
% sind. Sollte das nicht der Fall sein, meldet sie einen Fehler und sorgt sie
% mit \cs{aftergroup}\cs{@gobbletwo} dafür, dass die Definition unterbleibt.
%    \begin{macrocode}
\newcommand*{\scr@dsc@style@section@neededoptionstest}{%
  \scr@ifundefinedorrelax{scr@\scr@dsc@current @sectionindent}{%
    \scr@declaresectioncommandwarning{\scr@dsc@current}%
                                     {section indent}{indent}{0pt}%
    {\@namedef{scr@\scr@dsc@current @sectionindent}{0pt}}%
  }{}%
  \scr@ifundefinedorrelax{scr@\scr@dsc@current @beforeskip}{%
    \scr@declaresectioncommanderror{\scr@dsc@current}%
                                   {before section skip}{beforeskip}%
  }{}%
  \scr@ifundefinedorrelax{scr@\scr@dsc@current @afterskip}{%
    \scr@declaresectioncommanderror{\scr@dsc@current}%
                                   {after section skip}{afterskip}%
  }{}%
  \IfExistskomafont{\scr@dsc@current}{}{%
    \scr@declaresectioncommandwarning{\scr@dsc@current}%
                                     {font}{font}{\normalsize}%
    {\newkomafont{\scr@dsc@current}{\normalsize}}%
  }%
  \scr@ifundefinedorrelax{scr@\scr@dsc@current @runin}{%
    \l@addto@macro\local@endgroup{%
      \ClassInfo{\KOMAClassName}
                {using compatibility default `runin=bysign'\MessageBreak
                  for `\expandafter\string\csname \scr@dsc@current\endcsname}%
      \expandafter\gdef\csname scr@\scr@dsc@current @runin\endcsname
      ##1##2##3{##3}%
    }%
  }{}%
  \scr@ifundefinedorrelax{scr@\scr@dsc@current @afterindent}{%
    \l@addto@macro\local@endgroup{%
      \ClassInfo{\KOMAClassName}
                {using compatibility default `afterindent=bysign'\MessageBreak
                  for `\expandafter\string\csname \scr@dsc@current\endcsname}%
      \expandafter\gdef\csname scr@\scr@dsc@current @afterindent\endcsname
      ##1##2##3{##3}%
    }%
  }{}%
}
%    \end{macrocode}
% \end{macro}%^^A \scr@dsc@style@section@neededoptionstest
%
% \begin{macro}{\scr@dsc@style@chapter@neededoptionstest}
% \changes{v3.17}{2015/03/24}{Neu (intern)}%^^A
% \changes{v3.24}{2017/04/25}{Voreinstellungen für alle außer auf Ebene
%   \texttt{chapter}}%^^A
% \changes{v3.28}{2019/11/18}{\cs{ifstr} umbenannt in \cs{Ifstr}}%^^A
% Dieses Anweisung überprüft, ob alle notwendigen Eigenschaften für die
% Definition der spezifischen Gliederungsebene im chapter-Stil bekannt
% sind. Sollte das nicht der Fall sein, meldet sie einen Fehler und sorgt sie
% mit \cs{aftergroup}\cs{@gobbletwo} dafür, dass die Definition unterbleibt.
%    \begin{macrocode}
%<*book|report>
\newcommand*{\scr@dsc@style@chapter@neededoptionstest}{%
  \scr@ifundefinedorrelax{\scr@dsc@current pagestyle}{%
    \Ifstr{\scr@dsc@current}{chapter}{%
      \scr@declaresectioncommanderror{\scr@dsc@current}%
                                     {initial page style}{pagestyle}%
    }{%
      \scr@declaresectioncommandwarning{\scr@dsc@current}%
                                       {initial page style}{pagestyle}%
                                       {\chapterpagestyle}%
      {\@namedef{\scr@dsc@current pagestyle}{\chapterpagestyle}}%
    }%
  }{}%
  \scr@ifundefinedorrelax{scr@\scr@dsc@current @innerskip}{%
    \Ifstr{\scr@dsc@current}{chapter}{%
      \scr@declaresectioncommanderror{\scr@dsc@current}%
                                     {inner chapter skip}{innerskip}%
    }{%
      \scr@declaresectioncommandwarning{\scr@dsc@current}%
                                       {inner chapter skip}{innerskip}%
                                       {\scr@chapter@innerskip}%
      {\@namedef{scr@\scr@dsc@current @innerskip}{\scr@chapter@innerskip}}%
    }%
  }{}%
  \scr@ifundefinedorrelax{scr@\scr@dsc@current @beforeskip}{%
    \Ifstr{\scr@dsc@current}{chapter}{%
      \scr@declaresectioncommanderror{\scr@dsc@current}%
                                     {before chapter skip}{beforeskip}%
    }{%
      \scr@declaresectioncommandwarning{\scr@dsc@current}%
                                       {before chapter skip}{beforeskip}%
                                       {\scr@chapter@beforeskip}%
      {\@namedef{scr@\scr@dsc@current @beforeskip}{\scr@chapter@beforeskip}}%
    }%
  }{}%
  \scr@ifundefinedorrelax{scr@\scr@dsc@current @afterskip}{%
    \Ifstr{\scr@dsc@current}{chapter}{%
      \scr@declaresectioncommanderror{\scr@dsc@current}%
                                     {after chaper skip}{afterskip}%
    }{%
      \scr@declaresectioncommandwarning{\scr@dsc@current}%
                                       {after chaper skip}{afterskip}%
                                       {\scr@chapter@afterskip}%
      {\@namedef{scr@\scr@dsc@current @afterskip}{\scr@chapter@afterskip}}%
    }%
  }{}%
  \IfExistskomafont{\scr@dsc@current}{}{%
    \Ifstr{\scr@dsc@current}{chapter}{%
      \scr@declaresectioncommanderror{\scr@dsc@current}{font}{font}%
    }{%
      \scr@declaresectioncommandwarning{\scr@dsc@current}%
                                       {font}{font}%
                                       {\usekomafont{chapter}}%
      {\newkomafont{\scr@dsc@current}{\usekomafont{chapter}}}%
    }%
  }%
  \IfExistskomafont{\scr@dsc@current prefix}{}{%
    \Ifstr{\scr@dsc@current}{chapter}{%
      \scr@declaresectioncommanderror{\scr@dsc@current}%
                                     {prefix line font}{prefixfont}%
    }{%
      \scr@declaresectioncommandwarning{\scr@dsc@current}%
                                       {prefix line font}{prefixfont}%
                                       {\usekomafont{chapterprefix}}%
      {\newkomafont{\scr@dsc@current prefix}{\usekomafont{chapterprefix}}}%
    }%
  }{}%
  \scr@ifundefinedorrelax{scr@\scr@dsc@current @afterindent}{%
    \l@addto@macro\local@endgroup{%
      \ClassInfo{\KOMAClassName}
                {using compatibility default `afterindent=bysign'\MessageBreak
                  for `\expandafter\string\csname \scr@dsc@current\endcsname}%
      \expandafter\gdef\csname scr@\scr@dsc@current @afterindent\endcsname
      ##1##2##3{##3}%
    }%
  }{}%
}
%</book|report>
%    \end{macrocode}
% \end{macro}%^^A \scr@dsc@style@chapter@neededoptionstest
%
% \begin{macro}{\scr@dsc@style@part@neededoptionstest}
% \changes{v3.17}{2015/03/24}{Neu (intern)}
% \changes{v3.24}{2017/04/25}{Voreinstellungen für alle außer auf Ebene
%   \texttt{part}}%^^A
% \changes{v3.28}{2019/11/18}{\cs{ifstr} umbenannt in \cs{Ifstr}}%^^A
% Dieses Anweisung überprüft, ob alle notwendigen Eigenschaften für die
% Definition der spezifischen Gliederungsebene im part-Stil bekannt
% sind. Sollte das nicht der Fall sein, meldet sie einen Fehler und sorgt sie
% mit \cs{aftergroup}\cs{@gobbletwo} dafür, dass die Definition unterbleibt.
%    \begin{macrocode}
\newcommand*{\scr@dsc@style@part@neededoptionstest}{%
%<*book|report>
  \scr@ifundefinedorrelax{\scr@dsc@current pagestyle}{%
    \Ifstr{\scr@dsc@current}{part}{%
      \scr@declaresectioncommanderror{\scr@dsc@current}%
                                     {initial page style}{pagestyle}%
    }{%
      \scr@declaresectioncommandwarning{\scr@dsc@current}%
                                       {initial page style}{pagestyle}%
                                       {\partpagestyle}%
      {\@namedef{\scr@dsc@current pagestyle}{\partpagestyle}}%
    }%
  }{}%
  \scr@ifundefinedorrelax{scr@\scr@dsc@current @innerskip}{%
    \Ifstr{\scr@dsc@current}{part}{%
      \scr@declaresectioncommanderror{\scr@dsc@current}%
                                     {inner part skip}{innerskip}%
    }{%
      \scr@declaresectioncommandwarning{\scr@dsc@current}%
                                       {inner part skip}{innerskip}%
                                       {\scr@part@innerskip}%
      {\@namedef{scr@\scr@dsc@current @innerskip}{\scr@part@innerskip}}%
    }%
  }{}%
%</book|report>
  \scr@ifundefinedorrelax{scr@\scr@dsc@current @beforeskip}{%
    \Ifstr{\scr@dsc@current}{part}{%
      \scr@declaresectioncommanderror{\scr@dsc@current}%
                                     {before part skip}{beforeskip}%
    }{%
      \scr@declaresectioncommandwarning{\scr@dsc@current}%
                                       {before part skip}{beforeskip}%
                                       {\scr@part@beforeskip}%
      {\@namedef{scr@\scr@dsc@current @beforeskip}{\scr@part@beforeskip}}%
    }%
  }{}%
  \scr@ifundefinedorrelax{scr@\scr@dsc@current @afterskip}{%
    \Ifstr{\scr@dsc@current}{part}{%
      \scr@declaresectioncommanderror{\scr@dsc@current}%
                                     {after part skip}{afterskip}%
    }{%
      \scr@declaresectioncommandwarning{\scr@dsc@current}%
                                       {after part skip}{afterskip}%
                                       {\scr@part@afterskip}%
      {\@namedef{scr@\scr@dsc@current @afterskip}{\scr@part@afterskip}}%
    }%
  }{}%
  \IfExistskomafont{\scr@dsc@current}{}{%
    \Ifstr{\scr@dsc@current}{part}{%
      \scr@declaresectioncommanderror{\scr@dsc@current}{font}{font}%
    }{%
      \scr@declaresectioncommandwarning{\scr@dsc@current}{font}{font}%
                                       {\usekomafont{part}}%
      {\newkomafont{\scr@dsc@current}{\usekomafont{part}}}%
    }%
  }{}%
  \IfExistskomafont{\scr@dsc@current prefix}{}{%
    \Ifstr{\scr@dsc@current}{part}{%
      \scr@declaresectioncommanderror{\scr@dsc@current}%
                                     {number font}{prefixfont}%
    }{%
      \scr@declaresectioncommandwarning{\scr@dsc@current}%
                                       {number font}{prefixfont}%
                                       {\usekomafont{partprefix}}%
      {\newkomafont{\scr@dsc@current prefix}{\usekomafont{partprefix}}}%
    }%
  }{}%
  \scr@ifundefinedorrelax{scr@\scr@dsc@current @afterindent}{%
    \l@addto@macro\local@endgroup{%
      \ClassInfo{\KOMAClassName}{%
        using compatibility default
%<article>        `afterindent=false'\MessageBreak
%<book|report>        `afterindent=true'\MessageBreak
        for `\expandafter\string\csname \scr@dsc@current\endcsname}%
      \expandafter\gdef\csname scr@\scr@dsc@current @afterindent\endcsname
      ##1##2##3%
%<article>      {##2}%
%<book|report>      {##1}%
    }%
  }{}%
}
%    \end{macrocode}
% \end{macro}%^^A \scr@dsc@style@part@neededoptionstest
%
% \begin{macro}{\ifscr@dsc@parametersonly}
% \changes{v3.17}{2015/03/24}{neu}%^^A
% \changes{v3.20}{2015/11/18}{es gibt keine Dummy-Optionen mehr}%^^A
% Dieser Schalter wird verwendet, um die Einstellungen entsprechend dem
% aktuell gültigen Stil zu ändern, die zugehörige Anweisung selbst aber nicht
% neu zu definieren. Er wird von einem leeren \texttt{style}-Wert lokal
% gesetzt.
%    \begin{macrocode}
\newif\ifscr@dsc@parametersonly
%    \end{macrocode}
% \end{macro}%^^A \ifscr@dsc@parametersonly
%
%    \begin{macrocode}
\newcommand*{\DeclareSectionCommand}[2][]{%
  \edef\scr@dsc@current{#2}%
%    \end{macrocode}
% Im ersten Durchlauf werden nur die Stil-Optionen ausgewertet:
% \begin{description}
% \item[\texttt{style}] Stil der Gliederungsebene
% \end{description}
%    \begin{macrocode}
  \scr@dsc@parametersonlyfalse
  \DefineFamilyKey[.dsc]{KOMAarg}{style}{%
    \IfArgIsEmpty{##1}{%
      \scr@dsc@parametersonlytrue
      \FamilyKeyStateProcessed
    }{%
      \scr@ifundefinedorrelax{scr@dsc@def@style@##1@command}{%
        \FamilyKeyStateUnknownValue
      }{%
        \@namedef{scr@\scr@dsc@current @style}{##1}%
        \FamilyKeyStateProcessed
      }%
    }%
  }%
%    \end{macrocode}
% \begin{description}
% \item[\texttt{tocstyle}] 
% \changes{v3.20}{2015/11/12}{Option \texttt{tocstyle} frühzeitig
%   auswerten}%^^A
% \changes{v3.27}{2019/10/02}{Präfix für die Speicherung des
%   Verzeichnisstils geändert}%^^A
% Verzeichnis-Eintrags-Stil, damit auch für diesen anschließend die Optionen
% definiert bzw. aktiviert werden können. Zu beachten ist hier, dass ab
% \textsf{tocbasic} 3.27 der Stil korrekter Weise in \cs{scr@tso@\meta{level
% name}@style} statt in \cs{scr@dte@\meta{level name}@style} gespeichert
% wird. Zur Sicherheit werden hier beide Makros behandelt.
% \end{description}
%    \begin{macrocode}
  \DefineFamilyKey[.dsc]{KOMAarg}{tocstyle}{%
    \IfArgIsEmpty{##1}{%
      \@ifundefined{scr@dte@\scr@dsc@current @style}{}{%
        \expandafter\let\csname scr@dte@\scr@dsc@current @style\endcsname
        \relax
      }%
      \expandafter\let\csname scr@tso@\scr@dsc@current @style\endcsname
      \relax
      \FamilyKeyStateProcessed
    }{%
%<trace>      \typeout{TRACE: option tocstyle=##1}%
%<trace>      \typeout{TRACE: command=\scr@dsc@current}%
      \scr@ifundefinedorrelax{scr@dte@def@l@##1}{%
%<trace>        \typeout{TRACE: style not defined}%
        \FamilyKeyStateUnknownValue
      }{%
%<trace>        \typeout{TRACE: style defined}%
        \@namedef{scr@dte@\scr@dsc@current @style}{##1}%
        \@namedef{scr@tso@\scr@dsc@current @style}{##1}%
        \FamilyKeyStateProcessed
      }%
    }%
  }%
%    \end{macrocode}
% \begin{description}
% \item[\texttt{@else@}] 
% \changes{v3.20}{2015/11/18}{Option \texttt{tocstyle} frühzeitig
%     auswerten}%^^A
%   Damit keine anderen Optionen ausgewertet werden müssen aber auch nicht als
%   unbekannte Optionen einen Fehler erzeugen, wird diese Spezialoption
%   definiert, die einfach nur \cs{FamilyKeyStateProcessed} setzt.
% \end{description}
%    \begin{macrocode}
  \DefineFamilyKey[.dsc]{KOMAarg}{@else@}{%
%<*trace>
    \typeout{TRACE: ignoring option `@else@=\detokenize{##1}'}%
    \typeout{TRACE:
      \string\scr@key@atlist=`\detokenize\expandafter{\scr@key@atlist}'}%
    \typeout{TRACE:
      \string\scr@key@name=`\detokenize\expandafter{\scr@key@name}'}%
    \typeout{TRACE:
      \string\scr@key@value=`\detokenize\expandafter{\scr@key@value}'}%
%</trace>
    \FamilyKeyStateProcessed
  }%
  \FamilyExecuteOptions[.dsc]{KOMAarg}{#1}%
%    \end{macrocode}
% \changes{v3.20}{2015/11/18}{Rüchnahme der Optionen nach dem ersten
%     Durchgang}%^^A
% Nach dem ersten Druchgang müssen alle Optionen wieder zurückgenommen
% werden.
%    \begin{macrocode}
  \RelaxFamilyKey[.dsc]{KOMAarg}{@else@}%
  \RelaxFamilyKey[.dsc]{KOMAarg}{tocstyle}%
  \RelaxFamilyKey[.dsc]{KOMAarg}{style}%
  \begingroup
    \scr@ifundefinedorrelax{scr@\scr@dsc@current @style}{%
      \scr@declaresectioncommanderror{\scr@dsc@current}
                                     {section command style}{style}%
    }{}%
  \endgroup
  \@firstofone{%
%    \end{macrocode}
% \changes{v3.20}{2015/11/12}{keine Dummy-Optionen für den echten Lauf}%^^A
% Nachdem nun der Stil für den Gliederungsbefehl bekannt ist, können wir
% dessen Optionen aktivieren:
%    \begin{macrocode}
%<trace>    \typeout{TRACE: style=\@nameuse{scr@\scr@dsc@current @style}}%
    \@nameuse{scr@dsc@style@\@nameuse{scr@\scr@dsc@current @style}@options}%
%    \end{macrocode}
% Außerdem muss der Verzeichniseintrags-Stils initialisiert werden:
%    \begin{macrocode}
%<trace>    \typeout{TRACE: tocstyle=%
%<trace>      \@nameuse{scr@tso@\scr@dsc@current @style}}%
    \scr@ifundefinedorrelax{scr@tso@\scr@dsc@current @style}{%
      \scr@ifundefinedorrelax{l@\scr@dsc@current}{%
        \@namedef{scr@tso@\scr@dsc@current @style}{default}%
      }{}%
    }{}%
%    \end{macrocode}
% \changes{v3.20}{2016/04/12}{\cs{@ifnextchar} replaced by
%     \cs{kernel@ifnextchar}}%^^A
% \changes{v3.21}{2016/06/12}{erlaube auch \texttt{tocentry} an Stelle von
%     \texttt{toc} als Präfix}%^^A
% An dieser Stelle müssen die Optionen des Verzeichniseintrags auf Optionen
% der Gliederungsebene abgebildet werden. Dazu erhalten sie in der
% Gliederungebene den Präfix \texttt{toc}. Das bedeutet gleichzeitig, dass es
% keine Verzeichniseintragsoption Namens \texttt{style} geben kann und dass
% Gliederungsoptionen nicht mit \texttt{toc} beginnen dürfen.
%    \begin{macrocode}
    \let\scr@dsc@tocstyle@options\@empty
    \let\scr@dsc@extra@relax@opts\@empty
    \scr@ifundefinedorrelax{scr@tso@\scr@dsc@current @style}{%
      \expandafter\let
      \csname scr@tso@\scr@dsc@current @style\expandafter\endcsname
      \csname scr@dte@\scr@dsc@current @style\endcsname
    }{}%
    \scr@ifundefinedorrelax{scr@tso@\scr@dsc@current @style}{}{%
      \begingroup
        \def\scr@dte@current{#2}%
        \edef\reserved@a{%
          \noexpand\@ExecuteTOCEntryStyleInitCode{%
            \@nameuse{scr@tso@\scr@dsc@current @style}%
          }{\scr@dsc@current}%
        }\reserved@a
        \def\do@endgroup{\endgroup}%
        \def\do##1{%
          \l@addto@macro\do@endgroup{%
            \l@addto@macro\scr@dsc@extra@relax@opts{%
              \RelaxFamilyKey[.dsc]{KOMAarg}{toc##1}%
              \RelaxFamilyKey[.dsc]{KOMAarg}{tocentry##1}%
            }%
          }%
          \kernel@ifnextchar[%]
            {\@dodefault{##1}}%
            {\@donodefault{##1}}%
        }%
        \def\@donodefault##1{%
          \l@addto@macro\do@endgroup{%
            \DefineFamilyKey[.dsc]{KOMAarg}{toc##1}{%
              \l@addto@macro{\scr@dsc@tocstyle@options}{##1={####1},}%
              \FamilyKeyStateProcessed
            }%
            \DefineFamilyKey[.dsc]{KOMAarg}{tocentry##1}{%
              \l@addto@macro{\scr@dsc@tocstyle@options}{##1={####1},}%
              \FamilyKeyStateProcessed
            }%
          }%
        }%
        \def\@dodefault##1[##2]{%
          \l@addto@macro\do@endgroup{%
            \DefineFamilyKey[.dsc]{KOMAarg}{toc##1}[##2]{%
              \l@addto@macro{\scr@dsc@tocstyle@options}{##1={####1},}%
              \FamilyKeyStateProcessed
            }%
            \DefineFamilyKey[.dsc]{KOMAarg}{tocentry##1}[##2]{%
              \l@addto@macro{\scr@dsc@tocstyle@options}{##1={####1},}%
              \FamilyKeyStateProcessed
            }%
          }%
        }%
        \scr@dte@doopts
      \do@endgroup
    }%
%    \end{macrocode}
% Die Optionen für die Stile von Verzeichniseintrag und Gliederungsebene
% werden nicht erneut benötigt, müssten aber natürlich erneut verarbeitet
% werden.
%    \begin{macrocode}
    \DefineFamilyKey[.dsc]{KOMAarg}{style}{\FamilyKeyStateProcessed}%
    \DefineFamilyKey[.dsc]{KOMAarg}{tocstyle}{\FamilyKeyStateProcessed}%
%    \end{macrocode}
% Dazu kommen einige Optionen, die es schlicht immer gibt:
% \begin{description}\item[\texttt{expandtopt}] Schalter, um die Längen zur
% Definitionszeit statt zur Anwendungszeit komplett zu expandieren. Das
% bedingt dann aber auch, dass die Längen nicht mehr abhängig von der
% aktuellen Schriftgröße sind.\end{description}
%    \begin{macrocode}
    \FamilyBoolKey[.dsc]{KOMAarg}{expandtopt}{scr@dsc@expandtopt}%
    \scr@dsc@expandtoptfalse
%    \end{macrocode}
% \begin{description}\item[\texttt{increaselevel}] Automatische Erhöhung des
% Wertes von Option \texttt{level} mit Säumniswert 1. Diese Option wird nur
% definiert, wenn \cs{DeclareSectionCommands} verwendet wurde, was an
% \cs{scr@local@levelincrease} erkannt wird.\end{description}
%    \begin{macrocode}
    \scr@ifundefinedorrelax{scr@local@levelincrease}{%
      \RelaxFamilyKey[.dsc]{KOMAarg}{increaselevel}%
    }{%
      \FamilyCounterMacroKey[.dsc]{KOMAarg}{increaselevel}[1]%
                            {\scr@local@levelincrease}%
    }%
%    \end{macrocode}
% \begin{description}\item[\texttt{level}] nummerische
% Gliederungsebene\end{description}
%    \begin{macrocode}
    \scr@ifundefinedorrelax{scr@local@leveloffset}{%
      \edef\reserved@a{%
        \noexpand\FamilyCounterMacroKey[.dsc]{KOMAarg}{level}{%
          \expandafter\noexpand\csname \scr@dsc@current numdepth\endcsname}%
      }\reserved@a
    }{%
      \edef\reserved@a{%
        \noexpand\DefineFamilyKey[.dsc]{KOMAarg}{level}{%
          \noexpand\FamilySetCounterMacro{KOMAarg}{level}{%
            \expandafter\noexpand\csname \scr@dsc@current numdepth\endcsname
          }%
          \unexpanded{%
            {\numexpr ##1+\scr@local@leveloffset\relax}%
            \edef\scr@local@leveloffset{%
              \the\numexpr\scr@local@leveloffset+\scr@local@levelincrease\relax
            }%
          }%
        }%
      }\reserved@a
    }%
%    \end{macrocode}
% \changes{v3.19}{2015/09/09}{neue Eigenschaft \texttt{counterwithout}}
% \begin{description}\item[\texttt{counterwithout}] Der Zähler dieser Ebene
% soll nicht länger abhängen vom angegebenen Zähler.
% \end{description}
%    \begin{macrocode}
    \FamilyStringKey[.dsc]{KOMAarg}{counterwithout}{\scr@local@counterwithout}%
    \let\scr@local@counterwithout\relax
%    \end{macrocode}
% \begin{description}\item[\texttt{counterwithin}] Der Zähler dieser Ebene
% soll abhängen vom angegebenen Zähler.
% \end{description}
%    \begin{macrocode}
    \FamilyStringKey[.dsc]{KOMAarg}{counterwithin}{\scr@local@counterwithin}%
    \let\scr@local@counterwithin\relax
%    \end{macrocode}
% \begin{description}
% \item[\texttt{toclevel}] Ebene im Inhaltsverzeichnis (Voreinstellung ist
%   derselbe Wert wie für level); diese Option gibt es nur, wenn der
%   Verzeichnisstil sie unterstützt (was jeder Verzeichnisstil
%   sollte); wenn allerdings ein Offset dafür definiert ist, dann wird die
%   Option nicht einfach an den Stil weitergereicht, muss also noch einmal
%   umdefiniert werden.
% \end{description}
%    \begin{macrocode}
    \scr@ifundefinedorrelax{scr@local@tocleveloffset}{}{%
      \scr@ifundefinedorrelax{KV@KOMAarg.dsc@toclevel}{%
      }{%
        \DefineFamilyKey[.dsc]{KOMAarg}{toclevel}{%
          \edef\reserved@a{%
            \noexpand\l@addto@macro{\noexpand\scr@dsc@tocstyle@options}{%
              level=\the\numexpr ##1+\scr@local@tocleveloffset\relax,%
            }%
          }\reserved@a
          \edef\scr@local@tocleveloffset{%
            \the\numexpr\scr@local@tocleveloffset+\scr@local@levelincrease\relax
          }%
          \FamilyKeyStateProcessed
        }%
      }%
    }%
%    \end{macrocode}
% \changes{v3.20}{2015/11/12}{Optionen werden nur temporär definiert}%^^A
% Optionen ausführen und anschließend wieder freigeben.
%    \begin{macrocode}
    \FamilyExecuteOptions[.dsc]{KOMAarg}{#1}%
    \scr@dsc@extra@relax@opts
    \let\scr@dsc@extra@relax@opts\relax
    \RelaxSectionCommandOptions
    \RelaxFamilyKey[.dsc]{KOMAarg}{style}%
    \RelaxFamilyKey[.dsc]{KOMAarg}{tocstyle}%
    \RelaxFamilyKey[.dsc]{KOMAarg}{expandtopt}%
    \scr@ifundefinedorrelax{scr@local@levelincrease}{}{%
      \RelaxFamilyKey[.dsc]{KOMAarg}{increaselevel}%
    }%
    \RelaxFamilyKey[.dsc]{KOMAarg}{level}%
    \RelaxFamilyKey[.dsc]{KOMAarg}{counterwithin}%
%    \end{macrocode}
% Testen, ob alle benötigten Optionen für die (Um-)Definierung des
% Gliederungsbefehls gesetzt sind,
% \changes{v3.27}{2019/02/07}{\texttt{level} darf nicht -\cs{maxdimen}
%   sein}%^^A
% wobei \texttt{level} nicht -\cs{maxdimen} ist:
%    \begin{macrocode}
    \begingroup
      \def\local@endgroup{\endgroup}%
      \scr@ifundefinedorrelax{\scr@dsc@current numdepth}{%
        \scr@declaresectioncommanderror{\scr@dsc@current}{section level}{level}%
      }{%
        \expandafter\ifnum \csname #2numdepth\endcsname>-\maxdimen
          \@nameuse{scr@dsc@style@%
            \@nameuse{scr@\scr@dsc@current @style}%
            @neededoptionstest}%
        \else
          \ClassError{\KOMAClassName}{%
            `level' must be > -\the\numexpr\maxdimen\relax
          }{%
            KOMA-Script internally sets `secnumdepth' to
            -\the\numexpr\maxdimen\relax\space
            to locally switch of\MessageBreak
            the numbering, i.e., inside running heads.\MessageBreak
            Because of this, you can define section commands with\MessageBreak
            `level' greater than -\the\numexpr\maxdimen\relax\space only.%
          }%
          \let\local@endgroup\endgroup
          \aftergroup\@gobbletwo
        \fi
      }%
    \local@endgroup
%    \end{macrocode}
% Wenn alle Informationen für die (Um-)Definierung des Gliederungsbefehls
% zusammen sind \dots
%    \begin{macrocode}
    \@firstofone{%
%    \end{macrocode}
% Zunächst sicherstellen, dass der Zähler existiert und ggf. dessen
% Ausgabe anpassen:
% \changes{v3.20}{2015/10/15}{fehlendes \cs{expandafter} ergänzt}%^^A
%    \begin{macrocode}
      \@ifundefined{c@#2}{\newcounter{#2}}{}%
      \ifx\scr@local@counterwithout\relax
      \else\ifx\scr@local@counterwithout\@empty
        \else
          \@removefromreset{#2}{\scr@local@counterwithout}%
        \fi
        \@namedef{the#2}{\arabic{#2}}%
      \fi
      \ifx\scr@local@counterwithin\relax
      \else\ifx\scr@local@counterwithin\@empty
          \@namedef{the#2}{\arabic{#2}}%
        \else
          \@removefromreset{#2}{\scr@local@counterwithin}%
          \@addtoreset{#2}{\scr@local@counterwithin}%
          \expandafter\def\csname the#2\expandafter\endcsname\expandafter{%
            \csname the\scr@local@counterwithin\endcsname.\arabic{#2}}%
        \fi
      \fi
      \@ifundefined{#2format}{%
        \@namedef{#2format}{\csname the#2\endcsname\autodot\enskip}%
      }{}%
%    \end{macrocode}
% Dann den eigentlichen Gliederungsbefehl definieren:
%    \begin{macrocode}
      \ifscr@dsc@parametersonly
        \ClassInfo{\KOMAClassName}{%
          not defining `\expandafter\string\csname #2\endcsname' due
          to\MessageBreak
          empty section style option%
        }%
      \else
        \scr@ifundefinedorrelax{%
          scr@dsc@def@style@\@nameuse{scr@#2@style}@command%
        }{%
          \ClassWarning{\KOMAClassName}{%
            not defining `\expandafter\string\csname #2\endcsname'
            due\MessageBreak
            to not yet defined section style\MessageBreak
            `\@nameuse{scr@#2@style}'%
          }%
        }{%
          \@nameuse{scr@dsc@def@style@\@nameuse{scr@#2@style}@command}{#2}%
        }%
      \fi
%    \end{macrocode}
% Zu jedem Gliederungsbefehl gehört auch ein Befehl für den Kolumnentitel, der
% bei \KOMAScript{} außerdem eine Formatierung des Zählers benötigt. Falls
% \textsf{scrlayer} geladen ist, wird das über dessen Befehl
% |\DeclareSectionNumberDepth| erledigt, sonst direkt:
% \changes{v3.15a}{2015/01/21}{fix: \cs{endskip} statt \cs{enskip}}%^^A
%    \begin{macrocode}
      \@ifundefined{DeclareSectionNumberDepth}{%
        \@ifundefined{#2markformat}{%
          \@namedef{#2markformat}{\csname the#2\endcsname\autodot\enskip}%
        }{}%
        \@ifundefined{#2mark}{%
          \expandafter\let\csname #2mark\endcsname\@gobble
        }{}%
      }{%
        \DeclareSectionNumberDepth{#2}{\csname #2numdepth\endcsname}%
      }%
%    \end{macrocode}
% Und es gehören Befehle dazu, um einen Eintrag ins Inhaltsverzeichnis zu
% schreiben und den Eintrag auch darzustellen:
%    \begin{macrocode}
      \expandafter\providecommand\expandafter*%
      \csname add#2tocentry\endcsname[2]{%
        \addtocentrydefault{#2}{##1}{##2}%
      }%
      \scr@ifundefinedorrelax{scr@tso@#2@style}{%
        \expandafter\let\csname scr@tso@#2@style\expandafter\endcsname
        \csname scr@dte@#2@style\endcsname
      }{}%
      \scr@ifundefinedorrelax{scr@tso@#2@style}{%
      }{%
        \DeclareTOCStyleEntry[\scr@dsc@tocstyle@options]{%
          \@nameuse{scr@tso@#2@style}%
        }{#2}%
      }%
      \scr@ifundefinedorrelax{l@#2}{%
        \DeclareTOCStyleEntry[\scr@dsc@tocstyle@options]{default}{#2}%
      }{}%
%    \end{macrocode}
% \changes{v3.20}{2015/0/07}{\textsf{hyperref} bei den Ebenen der
%   Verzeichniseinträge berücksichtigen}%^^A
% \changes{v3.26}{2018/06/27}{Einstellungen von \textsl{hyperref}
%   überschreiben}%^^A
% \changes{v3.26}{2018/06/27}{Notfallbehandlung für Lücken in den
%   Bookmark-Ebenen}%^^A
%    \begin{macrocode}
      \if@atdocument
        \let\reserved@a\@firstofone
      \else
        \@ifpackageloaded{hyperref}{%
          \let\reserved@a\@firstofone
        }{%
          \def\reserved@a##1{%
            \AfterAtEndOfPackage{hyperref}{\AtBeginDocument{##1}}%
          }%
        }%
      \fi
      \reserved@a{%
        \expandafter\let\csname toclevel@#2\expandafter\endcsname
        \csname #2tocdepth\endcsname
        \scr@ifundefinedorrelax{bookmarksetup}{%
          \scr@ifundefinedorrelax{scr@min@toclevel}{%
            \expandafter\let\expandafter\scr@min@toclevel
            \csname toclevel@#2\endcsname
          }{%
            \expandafter\ifnum \csname toclevel@#2\endcsname
            < \numexpr \scr@min@toclevel-1\relax
              \ClassWarning{\KOMAClassName}{%
                Trying emergency fix for bookmark level gap,\MessageBreak
                because toclevel of `#2'
                (\csname toclevel@#2\endcsname)\MessageBreak
                is more than 1 lower than currently lowest\MessageBreak
                known level (\scr@min@toclevel).\MessageBreak
                Note: This fix can fail and you should load\MessageBreak
                package `bookmark' to avoid usage of this\MessageBreak
                fix%
              }%
              \expandafter\edef\csname toclevel@#2\endcsname{%
                \the\numexpr \scr@min@toclevel-1\relax
              }%
            \fi
            \expandafter\ifnum \csname toclevel@#2\endcsname
            < \scr@min@toclevel\relax
              \expandafter\let\expandafter\scr@min@toclevel
              \csname toclevel@#2\endcsname
            \fi
          }%
          \scr@ifundefinedorrelax{scr@max@toclevel}{%
            \expandafter\let\expandafter\scr@max@toclevel
            \csname toclevel@#2\endcsname
          }{%
            \expandafter\ifnum \csname toclevel@#2\endcsname
            > \numexpr \scr@max@toclevel+1\relax
              \ClassWarning{\KOMAClassName}{%
                Trying emergency fix for bookmark level gap,\MessageBreak
                because toclevel of `#2'
                (\csname toclevel@#2\endcsname)\MessageBreak
                is more than 1 greater than currently highest\MessageBreak
                known level (\scr@max@toclevel).\MessageBreak
                Note: This fix can fail and you should load\MessageBreak
                package `bookmark' to avoid usage of this\MessageBreak
                fix%
              }%
              \expandafter\edef\csname toclevel@#2\endcsname{%
                \the\numexpr \scr@max@toclevel+1\relax
              }%
            \fi
            \expandafter\ifnum \csname toclevel@#2\endcsname
            > \scr@max@toclevel\relax
              \expandafter\let\expandafter\scr@max@toclevel
              \csname toclevel@#2\endcsname
            \fi
          }%
        }{}%
      }%
    }%
  }%
  \let\scr@dsc@current\relax
}
%    \end{macrocode}
% \begin{macro}{\scr@declaresectioncommanderror}
% \changes{v3.15}{2014/11/21}{neue (interne) Anweisung}%^^A
% Gibt einen Fehler aus, falls \cs{DeclareSectionCommand} zu wenige Optionen
% übergeben wurde. Das erste Argument ist der Name des zu deklarierenden
% Befehls, das zweite ist eine textuelle Beschreibung der fehlenden Option und
% das dritte ist der Name der fehlenden Option. Als Seiteneffekt sorgt der
% Befehl dafür, dass die Deklaration nicht stattfindet.
%    \begin{macrocode}
\newcommand*{\scr@declaresectioncommanderror}[3]{%
  \ClassError{\KOMAClassName}{%
    #2 of \expandafter\string\csname #1\endcsname\space unknown%
  }{%
    Please use option `#3' to declare the #2.\MessageBreak
    If you'll continue, declaration will be ignored%
  }%
  \let\local@endgroup\endgroup
  \aftergroup\@gobbletwo
}
%    \end{macrocode}
% \end{macro}%^^A \scr@declaresectioncommanderror
% \begin{macro}{\scr@declaresectioncommandwarning}
% \changes{v3.24}{2017/04/25}{neue (interne) Anweisung}%^^A
% In einigen Fällen genügt es hingegen eine Warnung oder Info für fehlende
% Optionen auszugeben und einen Defaultwert zu verwenden. Das erste Argument
% ist auch hier der Name des zu deklarierenden Befehls, das zweite eine
% textuelle Beschreibung der fehlenden Option und das dritte der Name der
% fehlenden Option. Das vierte ist der Defaultwert, der ersatzweise verwendet
% wird. Der Befehl hat wie \cs{scr@declaresectioncommanderror} einen
% Seiteneffekt. In diesem Fall wird \cs{local@endgroup} um das fünfte Argument
% erweitert.
%    \begin{macrocode}
\newcommand*{\scr@declaresectioncommandwarning}[5]{%
  \ClassInfo{\KOMAClassName}{%
    #2 of \expandafter\string\csname #1\endcsname\space unknown.\MessageBreak
    You should use option `#3' if you\MessageBreak
    do not want to use the default value\MessageBreak
    `\detokenize{#4}'%
  }%
  \l@addto@macro\local@endgroup{#5}%
}
%    \end{macrocode}
% \end{macro}%^^A \scr@declaresectioncommandwarning
% \end{macro}%^^A \DeclareSectionCommand
%
% \begin{macro}{\DeclareNewSectionCommand}
% \changes{v3.15}{2014/11/21}{neue Anweisung}%^^A
% \changes{v3.25}{2018/03/10}{zusätzliche Tests und Fehlermeldungen bei
%   \cs{the\dots}, \cs{\dots mark}, \cs{\dots format},
%   \cs{\dots markformat}}%^^A
% Basierend auf |\DeclareSectionCommand| wird eine Gliederungsanweisung
% definiert, die zuvor noch nicht existierte. Wichtig zu bemerken ist, dass
% die ganzen Hilfsanweisungen bereits existieren dürfen.
%    \begin{macrocode}
\newcommand*{\DeclareNewSectionCommand}[2][]{%
  \@ifundefined{#2}{%
    \@tempswatrue
    \let\reserved@b\@empty
    \@for \reserved@a:=the#2,#2mark,#2format,#2markformat\do{%
      \expandafter\scr@ifundefinedorrelax\expandafter{\reserved@a}{}{%
        \@tempswafalse
        \edef\reserved@b{\expandafter\string\csname \reserved@a\endcsname}%
      }
    }
    \if@tempswa
      \DeclareSectionCommand[{#1}]{#2}%
    \else
      \ClassError{\KOMAClassName}{%
        command `\reserved@b' already defined%
      }{%
        You've tried to define the section command `\expandafter\string\csname
        #2\endcsname' newly.\MessageBreak
        Such a section command needs an additional new command
        `\reserved@b',\MessageBreak
        but this already exists. So you cannot define
        `\expandafter\string\csname #2\endcsname' newly.\MessageBreak
        Maybe you should use one of \string\DeclareSectionCommand,
        \string\RedeclareSectionCommand,\MessageBreak
        or \string\ProvideSectionCommand.\MessageBreak
        If you'll continue, the command will be ignored.%
      }%
    \fi
  }{%
    \ClassError{\KOMAClassName}{%
      command `\expandafter\string\csname #2\endcsname' already defined%
    }{%
      You've tried to define the section command `\expandafter\string\csname
      #2\endcsname' newly,\MessageBreak
      but a command or something else with this name already
      exists.\MessageBreak
      Maybe you should use one of \string\DeclareSectionCommand,
      \string\RedeclareSectionCommand,\MessageBreak
      or \string\ProvideSectionCommand.\MessageBreak
      If you'll continue, the command will be ignored.%
    }%
  }%
}
%    \end{macrocode}
% \end{macro}%^^A \DeclareNewSectionCommand
%
% \begin{macro}{\RedeclareSectionCommand}
% \changes{v3.15}{2014/11/21}{neue Anweisung}%^^A
% Basierend auf |\DeclareSectionCommand| wird eine Gliederungsanweisung
% umdefiniert, die zuvor bereits existierte. Wichtig zu bemerken ist, dass
% die ganzen Hilfsanweisungen nicht zu exisieren brauchen.
%    \begin{macrocode}
\newcommand*{\RedeclareSectionCommand}[2][]{%
  \@ifundefined{#2}{%
    \ClassError{\KOMAClassName}{%
      command `\expandafter\string\csname #2\endcsname' not defined%
    }{%
      You've tried to re-define the section command `\expandafter\string\csname
      #2\endcsname',\MessageBreak
      but a command with this name does not exists.\MessageBreak
      Maybe you should use one of \string\DeclareSectionCommand,
      \string\DeclareNewSectionCommand,\MessageBreak
      or \string\ProvideSectionCommand.\MessageBreak
      If you'll continue, the command will be irgnored.%
    }%
  }{%
    \DeclareSectionCommand[{#1}]{#2}%
  }%
}
%    \end{macrocode}
% \end{macro}%^^A \RedeclareSectionCommand
%
% \begin{macro}{\ProvideSectionCommand}
% \changes{v3.15}{2014/11/21}{neue Anweisung}%^^A
% Basierend auf |\DeclareSectionCommand| wird eine Gliederungsanweisung
% definiert, wenn sie zuvor noch nicht existierte. Wichtig zu bemerken ist,
% dass die ganzen Hilfsanweisungen bereits existieren dürfen.
%    \begin{macrocode}
\newcommand*{\ProvideSectionCommand}[2][]{%
  \@ifundefined{#2}{%
    \DeclareSectionCommand[{#1}]{#2}%
  }{%
    \ClassInfo{\KOMAClassName}{%
      \string\ProvideSectionCommand{#1} ignored%
    }%
  }%
}
%    \end{macrocode}
% \end{macro}%^^A \ProvideSectionCommand
%
% \begin{macro}{\DeclareSectionCommands}
% \changes{v3.15}{2014/12/03}{neue Anweisung}%^^A
% \changes{v3.26}{2018/08/29}{\cs{scr@trim@spaces} eingefügt}%^^A
% Basierend auf \cs{DeclareSectionCommand} können gleich mehrere
% Gliederungsanweisungen definiert werden. Dabei sorgt Option
% \texttt{increaselevel} dafür, dass die Werte von \texttt{level} und
% \texttt{toclevel} automatisch von Befehl zu Befehl um den angegebenen Wert
% (Default ist 1) erhöht werden.
%    \begin{macrocode}
\newcommand*{\DeclareSectionCommands}[2][]{%
  \edef\reserved@a{#2}%
  \def\scr@local@levelincrease{\z@}%
  \def\scr@local@leveloffset{\z@}%
  \def\scr@local@tocleveloffset{\z@}%
  \@for\reserved@a:=\reserved@a\do{%
    \scr@trim@spaces\reserved@a
    \edef\reserved@a{%
      \unexpanded{\DeclareSectionCommand[{#1}]}{\reserved@a}%
    }%
    \reserved@a
  }%
  \let\scr@local@levelincrease\relax
  \let\scr@local@leveloffset\relax
  \let\scr@local@tocleveloffset\relax
}
%    \end{macrocode}
% \end{macro}%^^A \DeclareSectionCommands
%
% \begin{macro}{\DeclareNewSectionCommands}
% \changes{v3.15}{2014/12/03}{neue Anweisung}%^^A
% \changes{v3.26}{2018/08/29}{\cs{scr@trim@spaces} eingefügt}%^^A
% Im Unterschied zu \cs{DeclareNewSectionCommand} führt diese Anweisung die
% Änderung auch dann aus, wenn die Anweisung bereits deklariert war. Eine
% Fehlermeldung wird aber natürlich trotzdem ausgegeben. Dieser Unterschied
% ist jedoch notwendig, weil sonst Option \texttt{increaselevel} für
% nachfolgende Befehle nicht mehr korrekt arbeiten würde.
%    \begin{macrocode}
\newcommand*{\DeclareNewSectionCommands}[2][]{%
  \edef\reserved@a{#2}%
  \def\scr@local@levelincrease{\z@}%
  \def\scr@local@leveloffset{\z@}%
  \def\scr@local@tocleveloffset{\z@}%
  \@for\reserved@a:=\reserved@a\do{%
    \scr@trim@spaces\reserved@a
    \@ifundefined{\reserved@a}{}{%
      \ClassError{\KOMAClassName}{%
        command `\expandafter\string\csname\reserved@a\endcsname' already
        defined%
      }{%
        You've tried to define the section command 
        `\expandafter\string\csname\reserved@a\endcsname' newly,\MessageBreak
        but a command, token, box or length with this name already
        exists.\MessageBreak
        Maybe you should use one of \string\DeclareSectionCommand,
        \string\RedeclareSectionCommand,\MessageBreak
        or \string\ProvideSectionCommand.\MessageBreak
        Nevertheless, if you'll continue, the command will be defined.%
      }%
    }%
    \edef\reserved@a{%
      \unexpanded{\DeclareSectionCommand[{#1}]}{\reserved@a}%
    }%
    \reserved@a
  }%
  \let\scr@local@levelincrease\relax
  \let\scr@local@leveloffset\relax
  \let\scr@local@tocleveloffset\relax
}
%    \end{macrocode}
% \end{macro}%^^A \DeclareNewSectionCommands
%
% \begin{macro}{\RedeclareSectionCommands}
% \changes{v3.15}{2014/12/03}{neue Anweisung}%^^A
% \changes{v3.26}{2018/08/29}{\cs{scr@trim@spaces} eingefügt}%^^A
% Im Unterschied zu \cs{RedeclareNewSectionCommands} führt diese Anweisung die
% Änderung auch dann aus, wenn die Anweisung noch nicht deklariert war. Eine
% Fehlermeldung wird aber natürlich trotzdem ausgegeben. Dieser Unterschied
% ist jedoch notwendig, weil sonst Option \texttt{increaselevel} für
% nachfolgende Befehle nicht mehr korrekt arbeiten würde.
%    \begin{macrocode}
\newcommand*{\RedeclareSectionCommands}[2][]{%
  \edef\reserved@a{#2}%
  \def\scr@local@levelincrease{\z@}%
  \def\scr@local@leveloffset{\z@}%
  \def\scr@local@tocleveloffset{\z@}%
  \@for\reserved@a:=\reserved@a\do{%
    \scr@trim@spaces\reserved@a
    \@ifundefined{\reserved@a}{%
      \ClassError{\KOMAClassName}{%
        command `\expandafter\string\csname\reserved@a\endcsname' not defined%
      }{%
        You've tried to re-define the section command 
        `\expandafter\string\csname\reserved@a\endcsname',\MessageBreak
        but a command with this name does not exists.\MessageBreak
        Maybe you should use one of \string\DeclareSectionCommand,
        \string\DeclareNewSectionCommand,\MessageBreak
        or \string\ProvideSectionCommand.\MessageBreak
        Nevertheless, if you'll continue, the command will be defined.%
      }%
    }{}%
    \edef\reserved@a{%
      \unexpanded{\DeclareSectionCommand[{#1}]}{\reserved@a}%
    }%
    \reserved@a
  }%
  \let\scr@local@levelincrease\relax
  \let\scr@local@leveloffset\relax
  \let\scr@local@tocleveloffset\relax
}
%    \end{macrocode}
% \end{macro}%^^A \RedeclareSectionCommands
%
% \begin{macro}{\ProvideSectionCommands}
% \changes{v3.15}{2014/12/03}{neue Anweisung}%^^A
% \changes{v3.17}{2015/03/23}{Umbau, um stilabhängige Optionen zu
%   ermöglichen}%^^A
% \changes{v3.26}{2018/08/29}{\cs{scr@trim@spaces} eingefügt}%^^A
% Hier wird es nun wirklich schwierig, weil wir irgendwie dafür sorgen müssen,
% dass Option \texttt{increaselevel} funktioniert, auch wenn gar nichts
% definiert werden darf. Das macht die Geschichte ein wenig auswändig.
%    \begin{macrocode}
\newcommand*{\ProvideSectionCommands}[2][]{%
  \edef\reserved@a{#2}%
  \def\scr@local@levelincrease{\z@}%
  \def\scr@local@leveloffset{\z@}%
  \def\scr@local@tocleveloffset{\z@}%
  \@for\reserved@a:=\reserved@a\do{%
    \scr@trim@spaces\reserved@a
    \@ifundefined{\reserved@a}{%
      \edef\reserved@a{%
        \unexpanded{\DeclareSectionCommand[{#1}]}{\reserved@a}%
      }%
      \reserved@a
    }{%
%    \end{macrocode}
% Wenn die Anweisung bereits definiert ist, dann müssen wir trotzdem die
% Optionen auswerten, wobei allerdings nur die Option \texttt{level} und
% \texttt{increaselevel} eine Wirkung haben dürfen. Der Rest wird per
% \texttt{@else}-Optionen erledigt.
%    \begin{macrocode}
      \DefineFamilyKey[.dsc]{KOMAarg}{@else@}{\FamilyKeyStateProcessed}%
      \FamilyCounterMacroKey[.dsc]{KOMAarg}{increaselevel}[1]%
        {\scr@local@levelincrease}%
      \DefineFamilyKey[.dsc]{KOMAarg}{level}{%
        \FamilySetCounterMacro{KOMAarg}{level}{\reserved@b}{##1}%
        \edef\scr@local@leveloffset{%
          \the\numexpr\scr@local@leveloffset+\scr@local@levelincrease\relax
        }%
      }%
      \DefineFamilyKey[.dsc]{KOMAarg}{toclevel}{%
        \FamilySetCounterMacro{KOMAarg}{toclevel}{\reserved@b}{##1}%
        \edef\scr@local@tocleveloffset{%
          \the\numexpr\scr@local@tocleveloffset+\scr@local@levelincrease\relax
        }%
      }%
      \FamilyExecuteOptions[.dsc]{KOMAarg}{#1}%
      \RelaxFamilyKey[.dsc]{KOMAarg}{toclevel}%
      \RelaxFamilyKey[.dsc]{KOMAarg}{level}%
      \RelaxFamilyKey[.dsc]{KOMAarg}{increaselevel}%
      \RelaxFamilyKey[.dsc]{KOMAarg}{@else@}%
    }%
  }%
  \let\scr@local@levelincrease\relax
  \let\scr@local@leveloffset\relax
  \let\scr@local@tocleveloffset\relax
}
%</body>
%    \end{macrocode}
% \end{macro}%^^A \ProvideSectionCommands
%
% \begin{macro}{\IfSectionCommandStyleIs}
% \selectlanguage{english}%^^A
% \changes{v3.27}{2019/02/04}{new}%^^A
% Test whether ot not the section command of argument 1 has the style of
% argument 2. If so: use argument 3, of not: use argument 4.
%    \begin{macrocode}
%<*body>
\newcommand*{\IfSectionCommandStyleIs}[2]{%
  \scr@ifundefinedorrelax{#1}{%
    \ClassError{\KOMAClassName}{%
      \expandafter\string\csname #1\endcsname\space not defined}{%
      You cannot compare the section command style of an undefined command.}%
  }{%
    \scr@ifundefinedorrelax{scr@#1@style}{%
      \ClassError{\KOMAClassName}{%
        \expandafter\string\csname #1\endcsname\space not a valid section
        command%
      }{%
        You cannot detect the section command style of a command, that
        has\MessageBreak
        never been defined as a section command by KOMA-Script.%
      }%
    }%
  }%
  \scr@ifundefinedorrelax{scr@dsc@def@style@#2@command}{%
    \ClassError{\KOMAClassName}{%
      unknown section command style `#2'%
    }{}%
  }%
  \Ifstr{\@nameuse{scr@#1@style}}{#2}%
}
%</body>
%    \end{macrocode}
% \end{macro}
%
% \begin{option}{bookmarkpackage}
% \selectlanguage{english}%^^A
% \changes{v3.26}{2018/06/27}{new option}%^^A
% Package \textsf{hyperref} has problems with bookmark level gaps. So if
% either the users omits a heading level or defines a heading with a level
% that distinguish more than one from the previous or following level, there
% would be a problem. One recommended solution for this would be to use
% package \textsf{bookmark}. So if we load this package, we can use the same
% value for bookmark levels and TOC levels even within
% \textsf{scrartcl}. Nevertheless we allow the user to decide not to use
% \textsf{bookmark}, but then it is not my fault, if something strange
% happens.
%    \begin{macrocode}
%<*option>
\KOMA@ifkey{bookmarkpackage}{@scr@autoloadbookmarkpackage}
\@scr@autoloadbookmarkpackagetrue
\KOMA@kav@add{.\KOMAClassFileName}{bookmarkpackage}{true}
\AfterAtEndOfPackage{hyperref}{%
  \AtBeginDocument{%
    \@ifpackageloaded{bookmark}{}{%
      \if@scr@autoloadbookmarkpackage
%    \end{macrocode}
% \changes{v3.27}{2019/06/26}{try to translate \textsf{hyperref} driver to
%  \textsf{bookmark} driver option}%^^A
% The name of the \textsf{hyperref} driver is in \cs{Hy@driver}. But
% \textsf{bookmark} does not provide the same drivers, so we try to translate
% the curreny known \textsf{hyperref} drivers to \textsf{bookmark}
% drivers. Also we do not load \textsf{bookmark} with a driver option but
% define \textsf{bookmark}'s \cs{BookmarkDriverDefault}.
%    \begin{macrocode}
        \scr@ifundefinedorrelax{Hy@driver}{%
          \ClassWarning{\KOMAClassName}{%
            `hyperref' loaded but `\string\Hy@driver' undefined.\MessageBreak
            This should not happen!\MessageBreak
            Maybe `hyperref' not loaded or unknown version?%
          }%
        }{%
%    \end{macrocode}
% \changes{v3.28}{2019/11/18}{\cs{ifstr} umbenannt in \cs{Ifstr}}%^^A
% Note: \texttt{hpdftex}, \texttt{hluatex}, \texttt{hxetex} and \texttt{hvtex}
% do not need to be translated, because they are autodetected by
% \textsf{bookmark} as well. \texttt{hypertex}, \texttt{hdviwindo},
% \texttt{htex4ht} and \texttt{latex2html} are not known by \textsf{bookmark},
% so we stay with the default (usually \texttt{dvips}).
%    \begin{macrocode}
          \Ifstr{\Hy@driver}{hdvips}{%
            \providecommand*{\BookmarkDriverDefault}{dvips}%
          }{%
            \Ifstr{\Hy@driver}{hdvipdfm}{%
              \providecommand*{\BookmarkDriverDefault}{dvipdfm}%
            }{%
              \Ifstr{\Hy@driver}{hdvipson}{%
                \providecommand*{\BookmarkDriverDefault}{dvipsone}%
              }{%
                \Ifstr{\Hy@driver}{htexture}{%
                  \providecommand*{\BookmarkDriverDefault}{textures}%
                }{%
                }%
              }%
            }%
          }%
        }%
        \ClassInfo{\KOMAClassName}{%
          loading recommended package `bookmark'.\MessageBreak
          Using `bookmark' together with `hyperref' is
          recommended,\MessageBreak
          because of handling of possible bookmark level gaps.\MessageBreak
          You can avoid loading `bookmark' with KOMA-Script
          option\MessageBreak
          `bookmarkpackage=false' before \string\begin{document}
          and\MessageBreak
          you can avoid this message adding:\MessageBreak
          \space\space\string\usepackage
          \scr@ifundefinedorrelax{BookmarkDriverDefault}{}{%
            [\BookmarkDriverDefault]%
          }{bookmark}\MessageBreak
          before \string\begin{document}%
        }%
        \RequirePackage{bookmark}%
%    \end{macrocode}
% \changes{v3.26b}{2019/01/17}{workaround for \cs{@beginmainauxhook} added}%^^A
% Sometimes \textsf{auxhook} is not able to recognize that it cannot use a
% patch of \cs{document} to execute \cs{@beginmainauxhook} if it is loaded via
% \cs{AtBeginDocument}. In this case we do it here.
%    \begin{macrocode}
        \ifx\@beginmainauxhook\relax
        \else
          \if@filesw
            \ClassWarning{\KOMAClassName}{%
              seems someone has broken package `auxhook'.\MessageBreak
              Usually this happens, if `auxhook' is loaded or used\MessageBreak
              implicitly or explicitly by patching
              \string\document%
              \scr@ifundefinedorrelax{AtEndPreamble}{}{%
                \MessageBreak
                or via etoolbox command \string\AtEndPreamble%
              }.\MessageBreak
              Trying an emergency workaround.\MessageBreak
              You can avoid this warning adding:\MessageBreak
              \space\space\string\usepackage{auxhook}\MessageBreak
              before \string\begin{document}%
            }%
            \@beginmainauxhook
          \fi
        \fi
      \fi  
    }%
  }%
}
%</option>
%</class>
%    \end{macrocode}
% \selectlanguage{ngerman}%^^A
% \end{option}%^^A bookmarkpackage
%
%
% \subsection{Definitionen für Präambeln}
%
% Bei den \KOMAScript-Klassen \textsf{scrbook} und \textsf{scrreprt}
% können \cs{part} und \cs{chapter} mit Präambeln versehen werden. Es
% wird gleichzeitig eine Präambel über und unter der jeweiligen
% Überschrift unterstützt.
%
% \begin{macro}{\use@preamble}
% \changes{v2.8p}{2001/09/25}{neu (intern)}%^^A
% Dieses Makro dient allgemein zum Setzen einer Präambel (für
% Gliederungsüberschriften), wenn eine solche definiert ist. Als
% Argument wird der Name der Präambel übergeben, an den automatisch
% \texttt{@preamble} angehängt wird. Zum Schluss wird die gespeicherte
% Präambel gelöscht.
%    \begin{macrocode}
%<*class>
%<*book|report>
%<*body>
\newcommand*{\use@preamble}[1]{%
  \@ifundefined{#1@preamble}{}{%
    \@nameuse{#1@preamble}%
    \global\expandafter\let\csname#1@preamble\endcsname=\relax
  }%
}
%    \end{macrocode}
% \end{macro}
% \begin{macro}{\set@preamble}
% \changes{v2.8p}{2001/09/25}{neu (intern)}%^^A
% \changes{v3.20}{2016/04/12}{\cs{@ifnextchar} replaced by
%     \cs{kernel@ifnextchar}}%^^A
% Das Makro \cs{set@preamble} dient allgemein der Defintion einer
% Präambel (für Gliederungsüberschriften). Als Argument wird der
% Name der Gliederungsebene übergeben. Danach folgt ein optionales
% Argument, das die Position der Präambel angibt. Darauf folgt ein
% weiteres optionales Argument, das die Breite der Präambel
% angibt. Ist nur ein optionales Argument angegeben, so ist es die
% Position. Es folgt ein obligatorisches Argument mit dem Inhalt der
% Präambel. Die Definition der Präambel erfolgt global.
% \begin{macro}{\set@@preamble}
% \changes{v2.8p}{2001/09/25}{neu (intern)}%^^A
% \changes{v3.20}{2016/04/12}{\cs{@ifnextchar} replaced by
%     \cs{kernel@ifnextchar}}%^^A
% \begin{macro}{\set@@@preamble}
% \changes{v2.8p}{2001/09/25}{neu (intern)}%^^A
% \begin{macro}{\set@@@@preamble}
% \changes{v2.8p}{2001/09/25}{neu (intern)}%^^A
% \changes{v2.8q}{2001/11/27}{fehlende Klammern ergänzt}%^^A
% \changes{v3.12}{2012/10/16}{b mit derselben Funktion wie u erlaubt}%^^A
% \changes{v3.12a}{2014/02/14}{b korrigiert}%^^A
% Da mehrere optionale Argumente nach einem obligatorischen Argument
% auszuwerten sind, werden diverse Hilfsmakros benötigt.
%    \begin{macrocode}
\newcommand*{\set@preamble}[1]{%
%    \end{macrocode}
% Zunächst wird das erste optionale Argument gesucht.
%    \begin{macrocode}
  \kernel@ifnextchar [%]
  {\set@@preamble{#1}}{\set@@@preamble{#1}[][\hsize]}%
}
\newcommand*{\set@@preamble}{}
\def\set@@preamble#1[#2]{%
%    \end{macrocode}
% Wenn ein optionales Argument existiert, kann es auch ein zweites
% geben. Wenn nicht, wird mit der Voreinstellung gearbeitet.
%    \begin{macrocode}
  \kernel@ifnextchar [%]
  {\set@@@preamble{#1}[{#2}]}{\set@@@preamble{#1}[{#2}][\hsize]}%
}
\newcommand{\set@@@preamble}{}
\long\def\set@@@preamble#1[#2][#3]#4{%
%    \end{macrocode}
% Dies ist das Hauptmakro, das die eigentliche Arbeit macht. Es muss
% \cs{long} definiert werden, da die Präambel auch aus mehreren
% Absätzen bestehen kann.
%
% Zunächst wird das Positionsargument ausgewertet. Dies geschieht mit
% lokalen Hilfsmakros und einer Schleife. Zur Funktion der Schleife
% siehe im dokumentierten \LaTeX-Kern.
%    \begin{macrocode}
  \begingroup
    \def\prmbl@pos{#2}\let\prmbl@hpos\relax\let\prmbl@vpos\relax
    \expandafter \@tfor \expandafter \@tempa
      \expandafter :\expandafter =\prmbl@pos
    \do{%
      \if \@tempa l%
        \set@preamble@hpos{0}{#2}%
      \fi%
      \if \@tempa c%
        \set@preamble@hpos{1}{#2}%
      \fi%
      \if \@tempa r%
        \set@preamble@hpos{2}{#2}%
      \fi%
      \if \@tempa u%
        \set@preamble@vpos{0}{#2}%
      \fi%
      \if \@tempa b%
        \set@preamble@vpos{0}{#2}%
      \fi
      \if \@tempa o%
        \set@preamble@vpos{1}{#2}%
      \fi%
      \if \@tempa t%
        \set@preamble@vpos{1}{#2}%
      \fi
    }%
%    \end{macrocode}
% Wurde keine horizontale oder vertikale Ausrichtung definiert, so
% wird nun der Standardwert dafür eingesetzt.
%    \begin{macrocode}
    \@ifundefined{prmbl@hpos}{\def\prmbl@hpos{0}}{}%
    \@ifundefined{prmbl@vpos}{\def\prmbl@vpos{0}}{}%
%    \end{macrocode}
% Nun sind die Positionen ermittelt und es werden abhängig davon
% entsprechende Päambelmakros definiert. 
%    \begin{macrocode}
    \ifcase\prmbl@hpos
%    \end{macrocode}
% Zunächst links unten und oben:
%    \begin{macrocode}
      \ifcase\prmbl@vpos
        \set@@@@preamble{#1@u}{t}{#3}{}{\hfil}{#4}%
      \else
        \set@@@@preamble{#1@o}{b}{#3}{}{\hfil}{#4}%
      \fi
    \or
%    \end{macrocode}
% Dann zentriert unten und oben:
%    \begin{macrocode}
      \ifcase\prmbl@vpos
        \set@@@@preamble{#1@u}{t}{#3}{\hfil}{\hfil}{#4}%
      \else
        \set@@@@preamble{#1@o}{b}{#3}{\hfil}{\hfil}{#4}%
      \fi
    \else
%    \end{macrocode}
% Zum Schluss rechts und oben:
%    \begin{macrocode}
      \ifcase\prmbl@vpos
        \set@@@@preamble{#1@u}{t}{#3}{\hfil}{}{#4}%
      \else
        \set@@@@preamble{#1@o}{b}{#3}{\hfil}{}{#4}%
      \fi
    \fi
  \endgroup
}
%    \end{macrocode}
% Es folgt das Makro, mit dem die Definition tatsächlich
% stattfindet. Das erste Argument ist der Name der Präambel, das
% zweite die Positionsoption für die verwendete \cs{parbox}, das
% dritte ist die Breite der \cs{parbox}, dann zwei Argumente mit der
% horizontalen Ausrichtung vor und nach der \cs{parbox} und das
% sechste Agument ist schließlich der Inhalt der \cs{parbox}.
%    \begin{macrocode}
\newcommand{\set@@@@preamble}[6]{%
  \expandafter\gdef\csname #1@preamble\endcsname{%
    \hbox to\hsize{#4\parbox[{#2}]{#3}{#6\par}#5\par}%
  }%
}
%    \end{macrocode}
%
% \begin{macro}{\set@preamble@hpos}
% \changes{v2.8q}{2001/09/25}{neu (intern)}%^^A
% Eines der Hilfsmakros dient dazu, die horizontale Position der
% Präambel einzustellen, falls dies nicht bereits erfolgt ist. Ist es
% bereits nicht über"-einstimmend erfolgt, wird ein Fehler
% ausgegeben. Dieses Makro ist extrem intern und funktioniert nur im
% Kontext vor \cs{set@preamble}.
%    \begin{macrocode}
\newcommand*{\set@preamble@hpos}[2]{%
  \@ifundefined{prbml@hpos}{%
    \def\prmbl@hpos{#1}%
  }{%
    \ifnum \prmbl@hpos=0\relax\else%
      \ClassError{\KOMAClassName}{%
        inconsistent hpos options%
      }{%
        You've said `#2' as position option. But you have
        to\MessageBreak
        use only one of `l', `r', or `c' not two of
        these.\MessageBreak
        If you'll continue only first hpos option will be
        used%
      }%
    \fi
  }%
}
%    \end{macrocode}
% \end{macro}%^^A \set@preamble@hpos
% \begin{macro}{\set@preamble@vpos}
% \changes{v2.8q}{2001/09/25}{neu (intern)}%^^A
% Das Hilfsmakro für die vertikale Position funktioniert in gleicher
% Weise.
%    \begin{macrocode}
\newcommand*{\set@preamble@vpos}[2]{%
  \@ifundefined{prbml@vpos}{%
    \def\prmbl@vpos{#1}%
  }{%
    \ifnum \prmbl@vpos=0\relax\else%
      \ClassError{\KOMAClassName}{%
        inconsistent vpos options%
      }{%
        You've said `#2' as position option. But you have
        to\MessageBreak
        use only one of `t', `o', `b', or `u' not two of
        these.\MessageBreak
        If you'll continue only first vpos option will be
        used%
      }%
    \fi
  }%
}
%</body>
%</book|report>
%</class>
%    \end{macrocode}
% \end{macro}
% \end{macro}
% \end{macro}
% \end{macro}
% \end{macro}
%
%
% \subsubsection{\KOMAScript-eigene Befehle für Verzeichniseinträge}
% Wie schon bei den Gliederungsbefehlen genügt \KOMAScript{} nicht, was der
% \LaTeX-Kern für Verzeichniseinträge zu bieten hat.
%
% \begin{macro}{\bprot@dottedtocline}
% \changes{v2.96a}{2006/11/30}{neu (intern)}%^^A
% \changes{v3.27}{2019/02/25}{veraltet}%^^A
% Dieses Makro wird benötigt, um die Ebenen unterhalb von \cs{chapter}
% bzw. \cs{section} im Inhaltsverzeichnis ebenfalls vor unerwünschten
% Umbrüchen zwischen Eintrag und Unterebene zu schützen.
%    \begin{macrocode}
%<*class>
%<*body>
\newcommand*{\bprot@dottedtocline}[5]{%
  \expandafter\ifnum\scr@v@is@gt{2.96}\relax
    \ClassWarning{\KOMAClassName}{%
      usage of `\string\bprot@dottedtocline' is deprecated.\MessageBreak
      Maybe an old auxiliary file has been used.\MessageBreak
      Note, this command will be removed from\MessageBreak
      KOMA-Script soon. So if this warning message\MessageBreak
      persists after one more LaTeX run, you should\MessageBreak
      redeclare the corresponding ToC entries using\MessageBreak
      \string\DeclareTOCStyleEntry[\MessageBreak
      \space\space level=\detokenize{#1},\MessageBreak
      \space\space indent=\detokenize{#2},\MessageBreak
      \space\space numwidth=\detokenize{#3}]{default}{...},\MessageBreak
      where `...' is the name of the ToC level.\MessageBreak
      Note, in future usage of `\string\bprot@dottedtocline'\MessageBreak
      could result in an error%
    }%
    \ifnum #1>\c@tocdepth \else
      \ifnum \lastpenalty<\numexpr 20010-#1\relax
        \addpenalty{\@lowpenalty}%
      \fi
      \@dottedtocline{#1}{#2}{#3}{#4}{#5}%
      \penalty \numexpr 20009-#1\relax
    \fi
  \else
    \@dottedtocline{#1}{#2}{#3}{#4}{#5}%
  \fi
}
%</body>
%</class>
%    \end{macrocode}
% \end{macro}
%
%
% \subsection{Grobeinteilung von Büchern}
%
% Bei Büchern gibt es teilweise eine Einteilung in Vorderteil, Hauptteil und
% Endteil, die nicht wirklich eine Gliederung darstellt, sondern auf
% Gliederung, Seitennummerierung etc. Einfluss nimmt.
%
% \begin{macro}{\if@mainmatter}
% \begin{macro}{\@mainmattertrue}
% \begin{macro}{\@mainmatterfalse}
% \changes{v2.3a}{1995/07/08}{\cs{if@mainmatter} Anforderung verschoben}%^^A
% Über diesen Schalter wird bei Buchklassen festgelegt, ob wir uns im
% Hauptteil des Dokuments befinden oder nicht. Die eigentliche
% Umschaltung geschieht mit Hilfe der Befehle \cs{frontmatter},
% \cs{mainmatter}, \cs{backmatter}. Ohne Umschaltung befinden wir uns
% bereits im Hauptteil.
%    \begin{macrocode}
%<*class>
%<*book>
%<*body>
\newif\if@mainmatter\@mainmattertrue
%    \end{macrocode}
% \end{macro}%^^A \@mainmatterfalse
% \end{macro}%^^A \@mainmattertrue
% \end{macro}%^^A \if@mainmatter
%
% \begin{macro}{\frontmatter}
% \changes{v2.4e}{1996/07/02}{Option \texttt{openany} beachten}%^^A
% \changes{v2.5h}{1999/12/29}{Option \texttt{twoside} beachten}%^^A
% \changes{v3.00}{2008/09/27}{\cs{cleardoubleoddpage} statt
%     \cs{cleardoublepage}}%^^A
% Der Vorspann enthält bei Büchern oftmals Inhaltsverzeichnis, Vorwort
% und dergleichen und wird mit kleinen römischen Seitenzahlen
% versehen.
%    \begin{macrocode}
\newcommand*\frontmatter{%
  \if@twoside\cleardoubleoddpage\else\clearpage\fi
  \@mainmatterfalse\pagenumbering{roman}%
}
%    \end{macrocode}
% \end{macro}%^^A \frontmatter
% \begin{macro}{\mainmatter}
% \changes{v2.4e}{1996/07/02}{Option \texttt{openany} beachten}%^^A
% \changes{v2.5h}{1999/12/29}{Option \texttt{twoside} beachten}%^^A
% \changes{v3.00}{2008/09/27}{\cs{cleardoubleoddpage} statt
%     \cs{cleardoublepage}}%^^A
% Der Hauptteil ist das Fleisch der Bücher.
%    \begin{macrocode}
\newcommand*\mainmatter{%
  \if@twoside\cleardoubleoddpage\else\clearpage\fi
  \@mainmattertrue\pagenumbering{arabic}%
}
%    \end{macrocode}
% \end{macro}%^^A \mainmatter
% \begin{macro}{\backmatter}
% \changes{v3.00}{2008/09/27}{\cs{cleardoubleoddpage} statt
%     \cs{cleardoublepage}}%^^A
% Auch wenn nicht ganz einzusehen ist, warum das anders sein soll,
% wird hier nicht nach der Option \texttt{twoside}, sondern nach
% \texttt{openright} unterschieden.
%    \begin{macrocode}
\newcommand*\backmatter{%
  \if@openright\cleardoubleoddpage\else\clearpage\fi\@mainmatterfalse
}
%</body>
%</book>
%</class>
%    \end{macrocode}
% \end{macro}
%
%
% \subsection{Anhang}
%
% Teile gibt es tatsächlich bei allen drei Hauptklassen.
%
% \begin{macro}{\appendix}
% Wird der Anhang aktiviert, so werden Kapitel bzw. Abschnitte zukünftig mit
% Buchstaben nummeriert.
% \changes{v2.2d}{1995/05/28}{\textsf{scrartcl} benötigt keinen Zähler für
%     \cs{chapter}}%^^A 
% \changes{v2.3c}{1995/08/06}{alternative Nummerierung bei Verwendung eines
%     Anhangs}%^^A
% \changes{v2.8}{2001/06/15}{\cs{appendixmore} wird beachtet}%^^A
% \changes{v2.8e}{2001/07/10}{\cs{@altsecnumformattrue} wird nicht länger
%     ausgeführt}%^^A
% \changes{v2.8o}{2001/09/19}{\cs{par} eingefügt}%^^A
% \changes{v2.95}{2006/07/04}{\cs{newcommand*} durch \cs{gdef} ersetzt}%^^A
% \changes{v2.95}{2006/07/04}{Ausführung von \cs{appendixmore}%^^A
%     vereinfacht}%^^A
% \changes{v3.20}{2016/12/05}{neue eindeutige Warnung}
% \changes{v3.28}{2019/11/18}{\cs{ifstr} umbenannt in \cs{Ifstr}}%^^A
% \begin{macro}{\appendixmore}
% \cs{appendixmore} kann als \emph{hook} verwendet werden. Derzeit wird dieser
% allerdings noch recht stiefmütterlich behandelt und lediglich von Option
% \texttt{appendixprefix} definiert.
%    \begin{macrocode}
%<*class>
%<*body>
\newcommand*\appendix{%
  \Ifstr{\@currenvir}{appendix}{%
    \ClassWarning{\KOMAClassName}{%
      You are using
      `\string\begin{appendix}...\string\end{appendix}'.\MessageBreak
      You should note, that `\string\appendix' is a mostly
      globally\MessageBreak
      working command not an enviroment with only local\MessageBreak
      effects. Therefore `\string\end{appendix}' will neither\MessageBreak
      switch back to normal section numbering nor finish\MessageBreak
      every other effect of `\string\begin{appendix}'.\MessageBreak
      Nevertheless, some effects may end with\MessageBreak
      `\string\end{appendix}' and the document may become\MessageBreak
      inconsistent.\MessageBreak
      Because of this, you should remove `\string\end{appendix}'\MessageBreak
      and replace `\string\begin{appendix}' by command\MessageBreak
      `\string\appendix'%
    }%
  }{}%
  \par%
%<*article>
  \setcounter{section}{0}%
  \setcounter{subsection}{0}%
  \gdef\thesection{\@Alph\c@section}%
%</article>
%<*report|book>
  \setcounter{chapter}{0}%
  \setcounter{section}{0}%
  \gdef\@chapapp{\appendixname}%
  \gdef\thechapter{\@Alph\c@chapter}%
%</report|book>
  \csname appendixmore\endcsname
}
%</body>
%</class>
%    \end{macrocode}
% \end{macro}
% \end{macro}
%
%
% \subsection{Teile}
%
% \changes{v2.8e}{2001/07/10}{\cs{@thepart} entfernt}
%
% Teile sind bei den unterschiedlichen Klassen unterschiedlich definiert.
%
% \begin{macro}{\size@part}
% \changes{v2.8o}{2001/09/14}{neu (intern)}%^^A
% \changes{v2.96a}{2006/12/02}{letztes Element darf ein Argument erwarten,
%     wenn es keine Auswirkung auf den Zeilen abstand hat}%^^A
% \begin{macro}{\size@partnumber}
% \changes{v2.8o}{2001/09/14}{neu (intern)}%^^A
% \changes{v2.96a}{2006/12/02}{letztes Element darf ein Argument erwarten,
%     wenn es keine Auswirkung auf den Zeilen abstand hat}%^^A
% Hier werden diese Befehle nur vordefiniert. Ihre tatsächliche Einstellung
% erfolgt über die (Vor-)Auswahl von Option \texttt{headings}.
%    \begin{macrocode}
%<*class>
%<*prepare>
\newcommand*{\size@part}{}
\newcommand*{\size@partnumber}{}
%</prepare>
%    \end{macrocode}
% \end{macro}%^^A \size@partnumber
% \end{macro}%^^A \size@part
%
%
% \begin{macro}{\partheadstartvskip}
% \changes{v2.95a}{2006/07/10}{neue Konfiguratonsmöglichkeit}%^^A
% \changes{v3.15}{2014/11/24}{\textsf{scrartcl} definiert und verwendet
%     \cs{scr@part@beforeskip}}%^^A
% \begin{macro}{\partheadmidvskip}
% \changes{v2.95a}{2006/07/10}{neue Konfiguratonsmöglichkeit}%^^A
% \begin{macro}{\partheadendvskip}
% \changes{v2.95a}{2006/07/10}{neue Konfiguratonsmöglichkeit}%^^A
% \changes{v3.15}{2014/11/24}{\textsf{scrartcl} definiert und verwendet
%     \cs{scr@part@afterskip}}%^^A
% \begin{macro}{\parheademptypage}
% \changes{v3.02}{2009/01/01}{neue Konfigurationsmöglichkeit}%^^A
% Diese vier Makros werden verwendet, um den Abstand über in und unter der
% Überschrift eines Teils und die leere Rückseite zu definieren. Prinzipiell
% kann damit auch der Umbruch zwischen Nummer und Text entfernt werden.
% Da die Größe nicht über Option \texttt{headings} beeinflusst werden kann,
% müssen diese Hilfsmakros nicht bereits vor den Optionen definiert werden.
%
% Zur Erhöhung der Kompatibilität mit \cs{DeclareSectionCommand} werden
% zunächst einige Hilfsanweisung definiert und verwendet.
% \begin{macro}{\scr@part@sectionbeforeskip}
% \changes{v3.15}{2014/11/24}{neue (interne) Anweisung}%^^A
% \changes{v3.17}{2015/03/25}{umbenannt in \cs{scr@part@beforeskip}}%^^A
% \begin{macro}{\scr@part@sectionafterskip}
% \changes{v3.15}{2014/11/24}{neue (interne) Anweisung}%^^A
% \changes{v3.17}{2015/03/25}{umbenannt in \cs{scr@part@afterskip}}%^^A
% \begin{macro}{\scr@part@sectionindent}
% \changes{v3.15}{2014/11/24}{neue (interne) Anweisung (unbenutzt)}%^^A
% \changes{v3.17}{2015/03/24}{entfernt}%^^A
% \end{macro}%^^A \scr@part@sectionindent
% \end{macro}%^^A \scr@part@sectionafterskip
% \end{macro}%^^A \scr@part@sectionbeforeskip
% \begin{macro}{\scr@part@beforeskip}
% \changes{v3.17}{2015/03/25}{neu (intern) aus Umdefinierung von
%     \cs{scr@part@ectionbeforeskip}}%^^A
% \begin{macro}{\scr@part@innerskip}
% \changes{v3.17}{2015/03/24}{neu (intern)}%^^A
% \begin{macro}{\scr@part@afterskip}
% \changes{v3.17}{2015/03/25}{neu (intern) aus Umdefinierung von
%     \cs{scr@part@ectionbeforeskip}}%^^A
% \begin{macro}{\scr@part@style}
% \changes{v3.15}{2014/11/24}{neue (interne) Anweisung}%^^A
% Es wird der leere Stil verwendet, mit dem Neudefinitionen von \cs{part}
% über \cs{DeclareSectionCommand} verhindert werden.
%    \begin{macrocode}
%<*prepare>
\newcommand*{\scr@part@beforeskip}{%
%<article>  4ex
%    \end{macrocode}
% Hinweis: Das |+\baselineskip| wird benötigt, um Kompatibilitätsprobleme mit
% der Umdefinierung von |\chapterheadstartvskip| zu vermeiden.
%    \begin{macrocode}
%<report|book>  \z@ \@plus 1fil + \baselineskip
}
%<book|report>\newcommand*{\scr@part@innerskip}{20\p@}
\newcommand*{\scr@part@afterskip}{%
%<article>  3ex
%<report|book>  \z@ \@plus 1fil
}
\newcommand*{\scr@part@style}{part}
%    \end{macrocode}
% \end{macro}%^^A \scr@part@style
% \end{macro}%^^A \scr@part@afterskip
% \end{macro}%^^A \scr@part@innerskip
% \end{macro}%^^A \scr@part@beforeskip
%    \begin{macrocode}
\newcommand*{\partheadstartvskip}{%
%<article>  \addvspace{\@tempskipa}%
%<report|book>  \null\vskip-\baselineskip\vskip\@tempskipa
}
\newcommand*{\partheadmidvskip}{%
  \par\nobreak
%<book|report>  \vskip\@tempskipa
}
\newcommand*{\partheadendvskip}{%
%<article>  \vskip\@tempskipa
%<report|book>  \vskip\@tempskipa\newpage
}
%<*report|book>
\newcommand*{\partheademptypage}{%
  \if@twoside\if@openright
      \null%
      \thispagestyle{empty}%
      \newpage
  \fi\fi
}
%</report|book>
%</prepare>
%    \end{macrocode}
% \end{macro}%^^A \partheademptypage
% \end{macro}%^^A \partheadendvskip
% \end{macro}%^^A \partheadmidvskip
% \end{macro}%^^A \partheadstartvskip
%
%
% \begin{macro}{\scr@startpart}
% \changes{v3.18}{2015/05/23}{neue Anweisung (intern)}%^^A
% Diese Anweisung ist von der ursprünglichen Definition von \cs{part}
% abgeleitet und erbt deren ChangeLog-Einträge. 
% \changes{v2.8d}{2001/07/05}{\cs{partpagestyle} statt \texttt{plain}}%^^A
% \changes{v2.8p}{2001/09/27}{Präambel über der Überschrift hinzugefügt}%^^A
% \changes{v3.13a}{2014/09/11}{Verwendung von \cs{SecDef}}%^^A
% \changes{v3.19}{2015/08/25}{Seitenstil nur ändern wenn definiert und nicht
%   leer}%^^A
% \changes{v3.26}{2018/10/14}{Option \texttt{afterindent} wird beachtet}%^^A
% \changes{v3.28}{2019/11/18}{\cs{ifstr} umbenannt in \cs{Ifstr}}%^^A
% Die Anweisung definiert den Gliederungsüberschriften-Stil
% \texttt{part}. Dazu bekommt sie als einziges explizites Argument den
% \meta{Namen} der Gliederungsüberschrift. Weitere Argumente wie ein
% optionaler Stern oder ein optionales Argument für Inhaltsverzeichnis und
% Kolumnentitel sowie ein obligatorisches Argument für den Text der
% Überschrift sind implizit, das heißt, sie werden erst im weiteren Verlauf
% von Unter-Macros eingelesen und verarbeitet.
%    \begin{macrocode}
%<*body>
\newcommand*{\scr@startpart}[1]{%
  \ExecuteDoHook{heading/preinit/#1}%
%<article>  \par
%<*book|report>
  \if@openright\cleardoublepage\else\clearpage\fi
  \scr@ifundefinedorrelax{#1pagestyle}{}{%
    \Ifstr{#1pagestyle}{}{}{%
      \thispagestyle{\@nameuse{#1pagestyle}}%
    }%
  }%
  \if@twocolumn
    \onecolumn
    \@tempswatrue
  \else
    \@tempswafalse
  \fi
%</book|report>
  \@tempskipa=\glueexpr\@nameuse{scr@#1@beforeskip}\relax
  \@ifundefined{scr@#1@afterindent}{\@afterindentfalse}{%
    \csname scr@#1@afterindent\endcsname
    {\@afterindenttrue}{\@afterindentfalse}{%
      \@afterindenttrue
      \ifdim\@tempskipa<\z@
        \@tempskipa=-\@tempskipa
        \@afterindentfalse
      \fi
    }%
  }%
  \ExecuteDoHook{heading/postinit/#1}%
  \partheadstartvskip
%<*book|report>
  \vbox to\z@{\vss\use@preamble{#1@o}\strut\par}%
  \vskip-\baselineskip\nobreak
%</book|report>
  \expandafter\SecDef\csname @#1\expandafter\endcsname\csname @s#1\endcsname
}
%    \end{macrocode}
%
% \begin{macro}{\scr@@startpart}
% \changes{v3.18}{2015/05/23}{neue Anweisung (intern)}%^^A
% Diese Anweisung ist von der ursprünglichen Definition von \cs{@part}
% abgeleitet und erbt deren ChangeLog-Einträge. 
% \changes{v2.8q}{2001/11/13}{\cs{@parskipfalse}\cs{@parskip@indent}}%^^A
% \changes{v2.95}{2002/08/13}{\cs{centering} und \cs{@parskipfalse}%
%   \cs{@parskip@indent} vertauscht}%^^A
% \changes{v2.95}{2004/11/05}{\cs{@parskipfalse} und \cs{@parskip@indent}
%   ersetzt}%^^A
% \changes{v3.27}{2019/07/08}{neue Option \texttt{nonumber}}%^^A
% Die Anweisung macht die eigentliche Arbeit für den Fall einer
% Gliederungsanweisung ohne Stern. Das erste obligatorische Argument ist der
% \meta{Name} der Gliederungsanweisung. Das zweite eigentlich optionale
% Argument, das hier jedoch zwingend ist, ist der \meta{Spezial-Text} für
% Kolumnentitel und Inhaltsverzeichnis. Das dritte obligatorische Argument
% ist der \meta{Text} für die Überschrift.
%    \begin{macrocode}
\newcommand{\scr@@startpart}{}
\long\def\scr@@startpart#1[#2]#3{%
  \ExecuteDoHook{heading/branch/nostar/#1}%
%    \end{macrocode}
% \changes{v3.10}{2011/08/30}{Integration der Erweiterung für das optionale
%     Argument der Gliederungsbefehle}%^^A
% Für das erweitere optionale Argument muss hier eine key-value-Auswertung
% erfolgen, falls das erweitere optionale Argument aktiviert ist:
%    \begin{macrocode}
  \ifnum \scr@osectarg=\z@
    \@scr@tempswafalse
  \else
    \scr@istest#2=\@nil
  \fi
  \@currentusenumbertrue  
  \if@scr@tempswa
    \setkeys{KOMAarg.section}{tocentry={#3},head={#3},reference={#3},#2}%
  \else
    \ifcase \scr@osectarg\relax
      \setkeys{KOMAarg.section}{tocentry={#2},head={#2},reference={#2}}%
    \or
      \setkeys{KOMAarg.section}{tocentry={#3},head={#2},reference={#3}}%
    \or
      \setkeys{KOMAarg.section}{tocentry={#2},head={#3},reference={#2}}%
    \or
      \setkeys{KOMAarg.section}{tocentry={#2},head={#2},reference={#2}}%
    \fi
  \fi
%    \end{macrocode}
% Als nächstes muss für nummerierte Überschriften dieser Ebene sowohl der
% Zähler erhöht also auch ein nummerierter Eintrag ins Inhaltsverzeichnis
% erstellt werden.
% \changes{v2.8e}{2001/07/10}{\cs{@maybeautodot} wird aufgerufen}%^^A
% \changes{v3.08}{2010/11/01}{Verwendung von \cs{addparttocentry}}%^^A
% \changes{v3.18}{2015/05/20}{Verwendung von \cs{add\meta{Name}tocentry} in
%   \cs{typeout}}%^^A
% \changes{v3.28}{2019/11/19}{\cs{ifnumbered} umbenannt in
%   \cs{Ifnumbered}}%^^A
%    \begin{macrocode}
  \Ifnumbered{#1}{%
    \refstepcounter{#1}%
    \@maybeautodot\thepart%
    \expandafter\@maybeautodot\csname the#1\endcsname
    \typeout{#1 \csname the#1\endcsname.}%
    \ifx\@currenttocentry\@empty\else
      \scr@ifundefinedorrelax{add#1tocentry}{%
        \addtocentrydefault{#1}%
      }{%
        \@nameuse{add#1tocentry}%
      }{\csname the#1\endcsname}{\@currenttocentry}%
    \fi
  }{%
%    \end{macrocode}
% \changes{v3.18}{2015/05/23}{Resetliste auch bei nicht nummerierten
%     Überschriften abarbeiten}%^^A
% Für alle nicht nummerieten Überschriften dieser Ebene trotzdem die
% Resetliste des Zählers ausführen und ggf. einen Verzeichniseintrag
% erstellen.
%    \begin{macrocode}
    \expandafter\ifnum\scr@v@is@lt{3.18}\relax\else
      \begingroup
        \let\@elt\@stpelt
        \csname cl@#1\endcsname
      \endgroup
    \fi
    \typeout{#1 without number}%
    \ifx\@currenttocentry\@empty\else
%    \end{macrocode}
% An dieser Stelle muss ggf. der \textsf{hyperref}-Hook für nicht nummerierte
% Inhaltsverzeichniseinträge ausgeführt werden.
% \changes{v3.08}{2010/11/01}{Verwendung von \cs{addparttocentry}}%^^A
% \changes{v3.18}{2015/05/23}{\cs{addparttocentry} verallgemeinert}%^^A
%    \begin{macrocode}
      \hy@insteadofrefstepcounter{#1}%
      \scr@ifundefinedorrelax{add#1tocentry}{%
        \addtocentrydefault{#1}%
      }{%
        \@nameuse{add#1tocentry}%
      }{}{\@currenttocentry}%
    \fi
  }%
%    \end{macrocode}
% Es folgt das eigentliche Setzen der Überschrift. Übrigens ist der ganze Code
% oben bei \cs{scr@@startchapter} praktisch identisch. Lediglich das
% Abarbeiten der Resetliste gibt es dort bereits seit Version~3.15. In
% \cs{scr@@startchapter} wird an dieser Stelle dann \cs{@make\dots head}
% ausgeführt. Das könnte man eigentlich irgendwann hier genauso machen.
% \changes{v2.2c}{1995/05/25}{Part-Ausgabe auf CJK umgestellt}%^^A
% \changes{v2.4b}{1996/03/29}{\cs{size@partnumer} durch
%     \cs{size@partnumber} ersetzt}%^^A
% \changes{v2.8p}{2001/09/22}{\cs{sectfont} wird nun vor \cs{size@part} und
%     \cs{size@partnumber} aufgerufen}%^^A
% \changes{v2.8q}{2002/02/28}{\cs{nobreak} nach \cs{sectfont} behebt einen
%     Bug im color Paket}%^^A
% \changes{v3.18}{2015/05/23}{\cs{size@\dots} durch \cs{usekomafont}
%     ersetzt}%^^A
% \changes{v3.21}{2016/06/12}{fehlendes \cs{nobreak} ergänzt}%^^A
% \changes{v3.25}{2017/09/07}{\cs{IfUseNumber} definiert}%^^A
% \changes{v3.25}{2017/09/07}{\cs{sectfont} durch \cs{usekomafont}
%     ersetzt}%^^A
% \changes{v3.25}{2017/09/07}{Umstellung auf
%     \cs{partlineswithprefixformat}}%^^A
% \changes{v3.25}{2018/02/12}{Verwendung des korrekten Font-Elements}%^^A
% \changes{v3.28}{2019/11/19}{\cs{ifnumbered} umbenannt in
%   \cs{Ifnumbered}}%^^A
%    \begin{macrocode}
  \begingroup
    \def\IfUseNumber{\Ifnumbered{#1}}%
    \ExecuteDoHook{heading/begingroup/#1}%
    \setparsizes{\z@}{\z@}{\z@\@plus 1fil}\par@updaterelative
    \raggedpart
    \normalfont\usekomafont{disposition}{\nobreak
      \IfUseNumber{%
        \partlineswithprefixformat{#1}{%
          \usekomafont{#1prefix}{\nobreak\@nameuse{#1format}}%
%<book|report>        \setlength{\@tempskipa}{\@nameuse{scr@#1@innerskip}}%
          \partheadmidvskip
        }{%
          \usekomafont{#1}{\nobreak\interlinepenalty \@M#3\strut\@@par}%
        }%
      }{%
        \partlineswithprefixformat{#1}{}{%
          \usekomafont{#1}{\nobreak\interlinepenalty \@M#3\strut\@@par}%
        }
      }%
%    \end{macrocode}
% \changes{v3.10}{2011/08/30}{Wenn \cs{partmark} \cs{@gobble} ist, wird
%     stattdessen \cs{@mkboth}{}{} aufgerufen.}%^^A
% Und noch den Kolumnentitel setzen.
%    \begin{macrocode}
      \expandafter\ifx\csname #1mark\endcsname\@gobble
        \@mkboth{}{}%
      \else
        \csname #1mark\expandafter\endcsname\expandafter{\@currentheadentry}%
      \fi
    }%
    \ExecuteDoHook{heading/endgroup/#1}%
  \endgroup
%    \end{macrocode}
% Ganz zum Schluss muss noch einmal ein Abstand eingefügt werden.
%    \begin{macrocode}
%<*article>
  \nobreak
  \@tempskipa=\glueexpr\@nameuse{scr@#1@afterskip}\relax\relax
  \ifdim\@tempskipa<\z@\@tempskipa-\@tempskipa\fi
  \partheadendvskip
%</article>
%<book|report>  \@nameuse{@end#1}%
%    \end{macrocode}
% \changes{v3.26}{2018/10/14}{Verwendung von \@cs{@afterheading} für
%   \textsf{scrbook} und \textsf{scrreprt}}%^^A
% Ab \KOMAScript{} 3.26 wird hier \cs{@afterheading} für alle Klassen
% aufgerufen. Damit kann der Einzug nach der Überschrift eingestellt werden.
%    \begin{macrocode}
  \@afterheading
}
%    \end{macrocode}
% \end{macro}%^^A \scr@@startpart
%
% \begin{macro}{\scr@@startspart}
% \changes{v3.18}{2015/05/23}{neue Anweisung (intern)}%^^A
% Diese Anweisung ist von der ursprünglichen Definition von \cs{@spart}
% abgeleitet und erbt deren ChangeLog-Einträge. 
% \changes{v2.4n}{1997/05/28}{in der Sternvariante \cs{chaptermark}%^^A
%     bzw. \cs{sectionmark} eingefügt, um die Kolumnentitel zu
%     löschen}%^^A
% \changes{v2.6a}{2000/01/20}{zum Löschen der Kolumnentitel wird
%     nun \cs{@mkboth} verwendet}%^^A
% \changes{v2.8q}{2001/11/13}{\cs{@parskipfalse}\cs{@parskip@indent}}%^^A
% \changes{v2.95}{2002/08/13}{\cs{centering} und
%     \cs{@parskipfalse}\cs{@parskip@indent} vertauscht}%^^A
% \changes{v2.95}{2004/11/05}{\cs{@parskipfalse} und \cs{@parskip@indent}
%     ersetzt}%^^A
% Die Anweisung macht die eigentliche Arbeit für den Fall einer
% Gliederungsanweisung mit Stern. Das erste obligatorische Argument ist der
% \meta{Name} der Gliederungsanweisung. Das zweite obligatorische Argument
% ist der \meta{Text} für die Überschrift.
%    \begin{macrocode}
\newcommand{\scr@@startspart}[2]{%
  \ExecuteDoHook{heading/branch/star/#1}%
%    \end{macrocode}
% Für Leute, die unbedingt nach der Sternversion noch ein \cs{addcontentsline}
% anfügen wollen, muss hier ebenfalls der \textsf{hyperref}-Hook ausgeführt
% werden. Allerdings passiert das hier entsprechend dem Vorgehen von
% \textsf{hyperref} selbst in einer Gruppe, in der zusätzlich \cs{@mkboth} zu
% \cs{@gobbletwo} gemacht wird, auch wenn ich absolut nicht versehe, was das
% soll.
%    \begin{macrocode}
  \begingroup
    \let\@mkboth\@gobbletwo
    \hy@insteadofrefstepcounter{#1}%
  \endgroup
%    \end{macrocode}
% Erst dann erfolgt die tatsächliche Ausgabe der Überschrift.
% \changes{v3.21}{2016/06/12}{fehlendes \cs{nobreak} ergänzt}%^^A
% \changes{v3.25}{2017/09/07}{\cs{IfUseNumber} definiert}%^^A
% \changes{v3.25}{2017/09/07}{\cs{sectfont} durch \cs{usekomafont}
%     ersetzt}%^^A
% \changes{v3.25}{2017/09/07}{Umstellung auf
%     \cs{partlineswithprefixformat}}%^^A
%    \begin{macrocode}
  \begingroup
    \let\IfUseNumber\@secondoftwo
    \ExecuteDoHook{heading/begingroup/#1}%
    \setparsizes{\z@}{\z@}{\z@\@plus 1fil}\par@updaterelative
    \raggedpart
    \normalfont\usekomafont{disposition}{%
      \nobreak
      \partlineswithprefixformat{#1}{}{%
        \usekomafont{#1}{\nobreak\interlinepenalty \@M#2\strut\@@par}%
      }%
      \@mkboth{}{}%
    }%
    \ExecuteDoHook{heading/endgroup/#1}%
  \endgroup
%<*article>
  \nobreak
  \@tempskipa=\glueexpr\csname scr@#1@afterskip\endcsname\relax\relax
  \ifdim\@tempskipa<\z@\@tempskipa-\@tempskipa\fi
  \partheadendvskip
  \@afterheading
%</article>
%<book|report>  \@nameuse{@end#1}%
}
%    \end{macrocode}
% \end{macro}%^^A \scr@@startspart
%
% \begin{macro}{\partlineswithprefixformat}
% \changes{v3.25}{2017/09/07}{neu}
% Diese Anweisung definiert das Format der eigentlichen Überschriften im Stil
% \texttt{part}. Fonteinstellungen sind zu diesem Zeitpunkt bereits erfolgt
% order Teil der Argumente. Die Argumente sind:
% \begin{description}
% \item[\meta{Befehlsname} --] der Name der Gliederungsebene, normalerweise
%   \texttt{chapter}.
% \item[\meta{Gliederungsnummer} --] die bereits fertig formatierte
%   Gliederungsnummer einschließlich des nachfolgenden vertikalen Abstands
%   oder leer, falls keine Gliederungsnummer auszugeben ist.
% \item[\meta{Text} --] der Text der Überschrift.
% \end{description}
% Der Anwender ist selbst verantwortlich, dass innerhalb der Überschrift kein
% Seitenumbruch erfolgen kann. Nach der Überschrift wird jedoch zwangsweise
% ein \cs{@@par} ausgeführt, so dass sichergestellt ist, dass mit einem
% internen Absatz abgeschlossen wird.
%    \begin{macrocode}
\newcommand{\partlineswithprefixformat}[3]{%
  #2#3%
}
%    \end{macrocode}
% \end{macro}%^^A \partlineswithprefixformat
%
% \begin{macro}{\scr@@endpart}
% \changes{v3.18}{2015/05/23}{neue Anweisung (intern)}%^^A
% Diese Anweisung ist von der ursprünglichen Definition von \cs{@spart}
% abgeleitet und erbt deren ChangeLog-Einträge. 
% \changes{v2.3g}{1996/01/14}{\cs{@endpart} wird für
%     \textsf{scrartcl} nicht mehr definiert}%^^A
% \changes{v2.4e}{1996/07/02}{Option \texttt{openany} beachten}%^^A
% \changes{v2.6c}{2000/06/10}{\cs{@endpart} fügt nur noch in
%     beidseitigen Dokumenten bei Verwendung von \texttt{openright} eine
%     Leerseite ein}%^^A
% \changes{v2.8p}{2001/09/27}{\cs{vbox} eingefügt}%^^A
% \changes{v2.8p}{2001/09/27}{\cs{use@preamble} ersetzt
%     \cs{@part@preamble}}%^^A
% \changes{v3.02}{2009/01/01}{Verwendung des neuen
%     \cs{partheademptypage}}%^^A
% Die Anweisung wird bei \textsf{scrbook} und \textsf{screprt} verwendet, um
% die Überschrift abzuschließen. \textsf{scrartcl} hat keine solche Anweisung.
% Da einzige Argument ist der \meta{Name} der Gliederungsanweisung.
%    \begin{macrocode}
%<*book|report>
\newcommand*{\scr@@endpart}[1]{%
  \vbox to\z@{\use@preamble{#1@u}\vss}%
  \@tempskipa=\glueexpr\csname scr@#1@afterskip\endcsname\relax\relax
  \ifdim\@tempskipa<\z@\@tempskipa-\@tempskipa\fi
  \partheadendvskip
  \partheademptypage
  \if@tempswa
    \twocolumn
  \fi
}
%</book|report>
%    \end{macrocode}
% \end{macro}%^^A \scr@@endpart
% \end{macro}%^^A \scr@startpart
%
%
% \begin{macro}{\hy@insteadofrefstepcounter}
% \changes{v3.18}{2015/05/22}{neu (intern und \textsf{hyperref})}%^^A
% \begin{macro}{\scr@chapter@before@hyperref@patch}
% \changes{v3.27}{2019/07/24}{neu (intern für \textsf{scrhack})}%^^A
% \begin{macro}{\scr@chapter@after@hyperref@patch}
% \changes{v3.27}{2019/07/24}{neu (intern für \textsf{scrhack})}%^^A
% Normalerweise führt \textsf{hyperref} via \cs{refstepcounter}
% Initialisierungen für ein nachfolgendes \cs{addcontentsline} durch. Wenn
% allerdings eine nicht nummerierte Überschrift ins Inhaltsverzeichnis
% eingetragen wird, dann fehlt das vorherige \cs{refstepcounter}. Statt
% \cs{@chapter}, \cs{@schapter}, \cs{@part}, \cs{@spart} etc. direkt zu
% patchen, erscheint es mir sinnvoller einen Hook dafür zu definieren, der
% dann von \textsf{hyperref} entsprechend definiert werden kann. Solange
% \textsf{hyperref} das nicht macht, versuche ich mich selbst daran und kann
% nur hoffen, dass es funktioniert. Dabei verwende ich im Gegensatz zu
% \textsf{hyperref} für \cs{Hy@MakeCurrentHrefAuto} kein \cs{Hy@chapapp}, weil
% ja \cs{autoref} für nicht nummerierte Überschriften ohnehin sinnlos
% ist. Stattdessen verwende ich den Gliederungsbefehl.
%    \begin{macrocode}
\newcommand*{\hy@insteadofrefstepcounter}[1]{}
\let\hy@insteadofrefstepcounter\@gobble
\newcommand*{\scr@chapter@before@hyperref@patch}{%
  \let\scr@orig@chapter\@chapter
  \let\scr@orig@schapter\@schapter
  \let\scr@orig@addchap\@addchap
}
\BeforePackage{hyperref}{\scr@chapter@before@hyperref@patch}
\newcommand*{\scr@chapter@after@hyperref@patch}{%
  \let\@chapter\scr@orig@chapter
  \let\@schapter\scr@orig@schapter
  \let\@addchap\scr@orig@addchap
  \ifx\hy@insteadofrefstepcounter\@gobble
    \renewcommand*{\hy@insteadofrefstepcounter}[1]{%
      \Hy@MakeCurrentHrefAuto{##1*}%
      \Hy@raisedlink{%
        \hyper@anchorstart{\@currentHref}\hyper@anchorend
      }%
    }%
  \fi
}
\AfterPackage!{hyperref}{\scr@chapter@after@hyperref@patch}
%    \end{macrocode}
% \end{macro}%^^A \scr@after@hyperref@patch
% \end{macro}%^^A \scr@before@hyperref@patch
% \end{macro}%^^A \hy@insteadofrefstepcounter
%
% \begin{macro}{\addparttocentry}
% \changes{v3.08}{2010/11/01}{Neu}%^^A
% Seit Version~3.08 wird der Eintrag nicht direkt innerhalb von \cs{part}
% per \cs{addcontentsline} erzeugt, sondern indirekt über diese Anweisung. Das
% erste Argument ist dabei die (formatierte) Nummer bzw. bei nicht
% nummerierten Teilen leer. Das zweite Argument ist der Überschriftstext für
% das Verzeichnis. Durch diesen indirekten Weg, kann die Anweisung einfach
% umdefiniert werden. Verwendet wird hier die Standardanweisung für
% Inhaltsverzeichniseinträge, die in \texttt{scrkliof.dtx} definiert ist:
%    \begin{macrocode}
\newcommand*{\addparttocentry}[2]{%
  \addtocentrydefault{part}{#1}{#2}%
}
%    \end{macrocode}
% \end{macro}%^^A \addparttocentry
%
%
% \begin{macro}{\part}
% \changes{v3.18}{2015/05/23}{siehe \cs{scr@dsc@def@style@part@command} und
%     \cs{scr@startpart}}%^^A
% \begin{macro}{\@part}
% \changes{v3.18}{2015/05/23}{siehe \cs{scr@dsc@def@style@part@command} und
%     \cs{scr@@startpart}}%^^A
% \begin{macro}{\@spart}
% \changes{v3.18}{2015/05/23}{siehe \cs{scr@dsc@def@style@part@command} und
%     \cs{scr@@startspart}}%^^A
% \begin{macro}{\@endpart}
% \changes{v3.18}{2015/05/23}{siehe \cs{scr@dsc@def@style@part@command} und
%     \cs{scr@@endpart}}%^^A
% \begin{macro}{\setpartpreamble}
% \changes{v2.8f}{2001/07/12}{neu}%^^A
% \changes{v2.8p}{2001/09/27}{Verwendung von \cs{use@preamble}}%^^A
% \changes{v3.18}{2015/05/23}{siehe \cs{scr@dsc@def@style@part@command} und
%     \cs{scr@@endpart}}%^^A
% \begin{macro}{\part@u@preamble}
% \changes{v2.8p}{2001/09/27}{neu (intern)}%^^A
% \changes{v2.8p}{2001/09/27}{Ersatz für \cs{part@preamble}}%^^A
% \changes{v3.12a}{2014/02/14}{flaschen (\cs{part@preamble@u}) Namen
%     korrigiert}%^^A
% \changes{v3.18}{2015/05/23}{siehe \cs{scr@dsc@def@style@part@command} und
%     \cs{scr@@endpart}}%^^A
% \begin{macro}{\part@o@preamble}
% \changes{v2.8p}{2001/09/27}{neu (intern)}%^^A
% \changes{v3.12a}{2014/02/14}{flaschen Namen (\cs{part@preamble@o})
%     korrigiert}%^^A
% \changes{v3.18}{2015/05/23}{siehe \cs{scr@dsc@def@style@part@command} und
%     \cs{scr@@endpart}}%^^A
% Diese Anweisungen werden indirekt über \cs{DeclareSectionCommand} und
% \cs{scr@dsc@def@style@part@command} bzw. \cs{setpartpreamble} definiert.
% \end{macro}%^^A \setpartpreamble
% \end{macro}%^^A \part@u@preamble
% \end{macro}%^^A \part@o@preamble
% \end{macro}%^^A \@endpart
% \end{macro}%^^A \@spart
% \end{macro}%^^A \@part
% \end{macro}%^^A \part
%
% \begin{macro}{\addpart}
% \changes{v2.8c}{2001/06/29}{neu}%^^A
% \changes{v2.8d}{2001/07/05}{\cs{partpagestyle} statt \texttt{plain}}%^^A
% \changes{v2.95a}{2006/07/10}{support of preamble added}%^^A
% \changes{v3.11b}{2012/07/29}{missing negativ \cs{vskip} added}%^^A
% \changes{v3.13a}{2014/09/11}{Verwendung von \cs{SecDef}}%^^A
% \changes{v3.15a}{2015/01/22}{Anpassung an geänderte Abstandsbehandlung bei
%     \cs{part} und \cs{@spart}}%^^A
% \changes{v3.18}{2015/05/25}{Komplett neu definiert}%^^A
% \begin{macro}{\@addpart}
% \changes{v2.8c}{2001/06/29}{neu (intern)}%^^A
% \changes{v2.8l}{2001/08/17}{erst eintragen, dann ausgeben}%^^A
% \changes{v2.95}{2004/07/20}{kann Kolumnentitel erzeugen}%^^A
% \changes{v3.00}{2008/07/01}{Notlösung für \textsf{hyperref} eingefügt}%^^A
% \changes{v3.08}{2010/11/02}{Verwendung von \cs{addparttocentry}}%^^A
% \changes{v3.10}{2011/08/30}{Integration der Erweiterung für das optionale
%     Argument der Gliederungsbefehle}%^^A
% \changes{v3.10}{2011/08/30}{Wenn \cs{partmark} \cs{@gobble} ist, wird
%     \cs{@mkboth}{}{} aufgerufen.}%^^A
% \changes{v3.17}{2015/04/20}{Es wird \cs{addpartmark} verwendet und auch
%     nie \cs{@mkboth}{}{} aufgerufen.}%^^A
% \changes{v3.18}{2015/05/25}{Komplett neu definiert}%^^A
% \begin{macro}{\@saddpart}
% \changes{v2.8c}{2001/06/29}{neu (intern)}%^^A
% \changes{v3.18}{2015/05/25}{Komplett neu definiert}%^^A
%    \begin{macrocode}
\newcommand\addpart{%
  \SecDef\@addpart\@saddpart
}
\newcommand*{\@addpart}{}
\long\def\@addpart[#1]#2{%
  \edef\reserved@a{%
    \unexpanded{%
      \part[{#1}]{#2}%
      \c@secnumdepth=
    }\the\c@secnumdepth\relax
  }%
  \c@secnumdepth=\numexpr \partnumdepth-1\relax
  \reserved@a
}
\newcommand{\@saddpart}[1]{%
  \part*{#1}%
  \addpartmark{}%
}
%    \end{macrocode}
% \end{macro}%^^A \@saddpart
% \end{macro}%^^A \@addpart
% \end{macro}%^^A \addpart
%
%
% \begin{macro}{\l@part}
% \changes{v2.97c}{2007/06/21}{\cs{sectfont}\cs{large} durch Verwendung von
%     Element \texttt{partentry} ersetzt}%^^A
% \changes{v2.97c}{2007/06/21}{Element \texttt{partentrypagenumber} wird
%     verwendet}%^^A
% \changes{v3.15}{2014/12/23}{Verwendung von \cs{parttocdepth} und
%     \cs{scr@part@numwidth}}%^^A
% \changes{v3.20}{2015/10/06}{Überführung in Verzeichnisstil}%^^A
% Der Eintrag für \cs{part}. Es werden einige Hilfsmakros passend zu
% \cs{DeclareSectionCommand} definiert. Dadurch können diverse Einstellungen
% darüber geändert werden. Es kann jedoch derzeit kein neuer Stil zugewiesen
% werden.
% \begin{macro}{\scr@part@tocindent}
% \changes{v3.15}{2014/12/02}{neue (interne) Anweisung}%^^A
% \changes{v3.18}{2015/06/09}{indirekt über \cs{DeclareSectionCommand}}%^^A
% \changes{v3.20}{2015/11/06}{wird nun auch verwendet}%^^A
% Der Einzug des Eintrags noch vor der Nummer.
% \end{macro}%^^A \scr@part@tocindent
% \begin{macro}{\scr@part@tocnumwidth}
% \changes{v3.15}{2014/12/02}{neue (interne) Anweisung}%^^A
% \changes{v3.18}{2015/06/09}{indirekt über \cs{DeclareSectionCommand}}%^^A
% Die definierte Breite der Nummer für den Eintrag ins Inhaltsverzeichnis.
% \end{macro}
% \end{macro}%^^A \l@part
%
%
% \begin{Counter}{part}
% \begin{macro}{\thepart}
% \begin{macro}{\partformat}
% \changes{v2.3c}{1995/08/06}{Duden Regel 6}%^^A
% \changes{v2.5f}{1999/02/14}{überflüssiges Leerzeichen am Ende
%     entfernt}%^^A
% \changes{v2.7h}{2001/04/22}{verlorenen Backslash in \cs{autodot}%^^A
%     wieder eingefügt}%^^A
% Jede Gliederungsebene benötigt einen Zähler für die Gliederungsnummer und
% eine Darstellung des Zählers (\cs{the\dots}). Desweiteren wird eine
% Formatierung des Zählers in den Gliederungsüberschriften
% (\texttt{\bslash\dots format}) benötigt.
%    \begin{macrocode}
\newcounter{part}
\renewcommand*{\thepart}{\@Roman\c@part}
\newcommand*{\partformat}{\partname~\thepart\autodot}
%    \end{macrocode}
% \end{macro}%^^A \partformat
% \end{macro}%^^A \thepart
% \end{Counter}{part}
%
%
% \begin{macro}{\partname}
% Namen der Gliederungsebenen \cs{part}.
%    \begin{macrocode}
\newcommand*\partname{Part}
%    \end{macrocode}
% \end{macro}%^^A \partname
%
%
% \begin{macro}{\partmark}
% \changes{v2.9r}{2004/07/20}{neu (für \textsf{scrpage2})}%^^A
% \changes{v3.18}{2015/05/24}{Voreinstellung von \cs{@gobble} zu
%     \cs{@mkboth{}{}} geändert}
%    \begin{macrocode}
\newcommand*{\partmark}[1]{\@mkboth{}{}}
%    \end{macrocode}
% \end{macro}%^^A \partmark
%
% \begin{macro}{\addpartmark}
% \changes{v3.17}{2015/04/20}{neue Anweisung}%^^A
% \changes{v3.18}{2015/05/24}{abhängig von \cs{partnumdepth}}%^^A
% Analog zu \cs{addchapmark} für \cs{addpart}. Auch dabei gilt, dass
% \cs{if@mainmatter} hier auch von \textsf{scrreprt} definiert wird. Das ist
% Absicht und vereinfacht dem einen oder anderen Paket oder Benutzer
% vielleicht die Arbeit.
%    \begin{macrocode}
\newcommand*\addpartmark[1]{%
  \begingroup
    \expandafter\let\csname if@mainmatter\expandafter\endcsname
    \csname iffalse\endcsname
    \c@secnumdepth=\numexpr \partnumdepth-1\relax
    \partmark{#1}%
  \endgroup
}
%    \end{macrocode}
% \end{macro}%^^A \addpartmark
%
%
% \begin{macro}{\raggedpart}
% \changes{v2.95a}{2006/07/10}{neue Konfigurationsmöglichkeit}%^^A
% Die Ausrichtung der Überschrift von \cs{part} kann davon abweichend über
% \cs{raggedpart} konfiguriert werden. Die Voreinstellung ist von der
% Klasse abhängig.
%    \begin{macrocode}
\newcommand*{\raggedpart}{}
%<article>\let\raggedpart\raggedsection
%<report|book>\let\raggedpart\centering
%    \end{macrocode}
% \end{macro}
%
% \begin{KOMAfont}{part}
% \changes{v2.8p}{2001/09/23}{neues Element \texttt{part}}%^^A
% \begin{macro}{\scr@fnt@part}
% \changes{v2.8p}{2001/09/23}{neues Element \texttt{part}}%^^A
% \begin{KOMAfont}{partnumber}
% \changes{v2.8p}{2001/09/23}{neues Element \texttt{partnumber}}%^^A
% \begin{macro}{\scr@fnt@partnumber}
% \changes{v2.8p}{2001/09/23}{neues Element \texttt{partnumber}}%^^A
% \begin{KOMAfont}{partprefix}
% \changes{v3.18}{2015/06/10}{neues Alias-Element \texttt{partprefix}}%^^A
%    \begin{macrocode}
\newcommand*{\scr@fnt@part}{\size@part}
\newcommand*{\scr@fnt@partnumber}{\size@partnumber}
\aliaskomafont{partprefix}{partnumber}
%    \end{macrocode}
% \end{KOMAfont}
% \end{macro}
% \end{KOMAfont}
% \end{macro}
% \end{KOMAfont}
%
% \begin{KOMAfont}{partentry}
% \changes{v2.97c}{2007/06/21}{neues Font-Element}%^^A
% \changes{v3.06}{2010/06/09}{Verwendung von \texttt{sectioning} durch
%     \texttt{disposition} ersetzt}%^^A
% Schrift für den \cs{part}-Eintrag im Inhaltsverzeichnis.
%    \begin{macrocode}
\newkomafont{partentry}{\usekomafont{disposition}\large}
%    \end{macrocode}
% \end{KOMAfont}
%
% \begin{KOMAfont}{partentrypagenumber}
% \changes{v2.97c}{2007/06/21}{neues Font-Element}%^^A
% Schrift für die Seitenzahl des \cs{part}-Eintrags im Inhaltsverzeichnis
% abweichend von \texttt{partentry}.
%    \begin{macrocode}
\newkomafont{partentrypagenumber}{}
%</body>
%</class>
%    \end{macrocode}
% \end{KOMAfont}
%
%
% \subsection{Kapitel} 
%
% Bei den Klassen \texttt{scrbook} und \texttt{scrreprt} gibt es im
% Unterschied zu \texttt{scrartcl} auch noch die Kapitelebene.  Für die
% Definition des Kapitelstils mittels \cs{DeclareSectionCommand} werden neben
% den Optionen und \cs{scr@dsc@def@style@chapter@command} noch diverse
% Anweisungen benötigt. Diese sind allesamt von der ursprünglichen Definition
% für Kapitel abgeleitet. Eine Besonderheit ist dabei, dass \textsf{hyperref}
% dem neuen Code durch (inkompatible) Umdefinierung von \cs{@schapter} und
% \cs{@chapter} in die Quere kommt. Das gilt es also zusätzlich zu verhindern
% und dennoch \cs{hyperref} zu berücksichtigen. Dies gelingt am einfachsten
% durch einen neuen Hook, den \cs{hyperref} zukünftig gerne selbst passend
% umdefinieren kann. Wer sonst noch an \cs{@chapter} und \cs{@schaper}
% herumdoktert und dabei \KOMAScript{} aus dem Tritt bringt, ist selbst
% schuld.
%
% \begin{option}{chapterprefix}
% \changes{v2.8}{2001/06/15}{neue Option}%^^A
% \changes{v2.95c}{2006/08/21}{als \textsf{keyval}-Option}%^^A
% \changes{v3.18}{2015/05/22}{die Option bezieht sich allgemein auf den
%     Gliederungsstil}%^^A
% \begin{option}{nochapterprefix}
% \changes{v2.8}{2001/06/15}{neue Option}%^^A
% \changes{v2.95c}{2006/08/21}{obsolete Option}%^^A
% \changes{v3.01a}{2008/11/20}{deprecated}%^^A
% Normalerweise verwenden \textsf{scrbook} und \textsf{scrreprt} nicht
% die enorm großen Überschriften von \textsf{book} und
% \textsf{report}, die mit einem Absatz "`Kapitel \emph{Nummer}"'
% beginnen. Mit der Option \texttt{chapterprefix} kann dies jetzt
% wieder aktiviert werden. Die Option, zur Deaktivierung heißt
% entsprechend \textsf{nochapterprefix}. 
% Um für Spezialanwendungen auch innerhalb des Dokuments eine
% Umschaltung zu ermöglichen bzw. für die Optionen
% \texttt{appendixprefix} und \texttt{noappendixprefix} erfolgt die
% Umschaltung durch einen Schalter.
%    \begin{macrocode}
%<*class>
%<*book|report>
%<prepare>\newif\if@chapterprefix
%<*option>
\KOMA@key{chapterprefix}[true]{%
  \KOMA@set@ifkey{chapterprefix}{@chapterprefix}{#1}%
  \ifx\FamilyKeyState\FamilyKeyStateProcessed
    \KOMA@kav@replacebool{.\KOMAClassFileName}%
                         {chapterprefix}{@chapterprefix}%
    \KOMA@kav@remove{.\KOMAClassFileName}%
                    {headings}{twolinechapter}%
    \KOMA@kav@remove{.\KOMAClassFileName}%
                    {headings}{onelinechapter}%
    \KOMA@kav@xadd{.\KOMAClassFileName}%
                  {headings}{%
                    \if@chapterprefix twolinechapter\else onelinechapter\fi
                  }%
  \fi
}
\KOMA@DeclareDeprecatedOption{nochapterprefix}{chapterprefix=false}
\KOMA@kav@add{.\KOMAClassFileName}{headings}{onelinechapter}%
%</option>
%    \end{macrocode}
% \end{option}
% \end{option}
%
% \begin{option}{appendixprefix}
% \changes{v2.8}{2001/06/15}{neue Option}%^^A
% \changes{v2.95c}{2006/08/21}{als \textsf{keyval}-Option}%^^A
% \changes{v3.12}{2013/03/05}{Verwendung der Status-Signalisierung mit
%     \cs{FamilyKeyState}}%^^A
% \changes{v3.17}{2015/03/09}{interne Speicherung des Wert}%^^A
% \begin{option}{noappendixprefix}
% \changes{v2.8}{2001/06/15}{neue Option}%^^A
% \changes{v2.95c}{2006/08/21}{obsolete Option}%^^A
% \changes{v3.01a}{2008/11/20}{deprecated}%^^A
% Will man abweichend von der Option \texttt{chapterprefix} die großen
% Überschriften für den Anhang aktivieren oder deaktivieren, so kann man das
% mit dieser Option erreichen. Allerdings setzt diese keinen Schalter, sondern
% ein Zusatzmakro, das auch für andere Zwecke genutzt werden kann. Übrigens
% ist keine der beiden Einstellungen die Voreinstellung!
%    \begin{macrocode}
%<*option>
\KOMA@key{appendixprefix}[true]{%
  \KOMA@set@ifkey{appendixprefix}{@tempswa}{#1}%
  \ifx\FamilyKeyState\FamilyKeyStateProcessed
    \KOMA@kav@xreplacevalue{.\KOMAClassFileName}{appendixprefix}{#1}%
    \KOMA@kav@remove{.\KOMAClassFileName}{headings}{twolineappendix}%
    \KOMA@kav@remove{.\KOMAClassFileName}{headings}{onelineappendix}%
    \if@tempswa
      \KOMA@kav@add{.\KOMAClassFileName}{headings}{twolineappendix}%
      \@ifundefined{appendixmore}{%
        \def\appendixmore{\@chapterprefixtrue}%
      }{%
        \l@addto@macro\appendixmore{\@chapterprefixtrue}%
      }%
    \else
      \KOMA@kav@add{.\KOMAClassFileName}{headings}{onelineappendix}%
      \@ifundefined{appendixmore}{%
        \def\appendixmore{\@chapterprefixfalse}%
      }{%
        \l@addto@macro\appendixmore{\@chapterprefixfalse}%
      }%
    \fi
  \fi
}
\KOMA@DeclareDeprecatedOption{noappendixprefix}{appendixprefix=false}
%</option>
%    \end{macrocode}
% \end{option}%^^A noappendixprefix
% \end{option}%^^A appendixprefix
%
%
% \begin{macro}{\size@chapter}
% \changes{v2.8o}{2001/09/14}{neu (intern)}%^^A
% \changes{v2.96a}{2006/12/02}{letztes Element darf ein Argument erwarten,
%     wenn es keine Auswirkung auf den Zeilen abstand hat}%^^A
% \begin{macro}{\size@chapterprefix}
% \changes{v2.96a}{2006/12/02}{neu (intern)}%^^A
% Hier werden diese Befehle nur vordefiniert. Ihre tatsächliche Einstellung
% erfolgt über die (Vor-)Auswahl von Option \texttt{headings}.
%    \begin{macrocode}
%<*prepare>
\newcommand*{\size@chapter}{}
\newcommand*{\size@chapterprefix}{\size@chapter}
%</prepare>
%    \end{macrocode}
% \end{macro}
% \end{macro}
%
%
% \begin{macro}{\chapterheadstartvskip}
% \changes{v3.15}{2014/11/24}{\cs{scr@chapter@beforeskip} wird
%     definiert und verwendet}%^^A
% \begin{macro}{\chapterheadendvskip}
% \changes{v3.11c}{2013/02/13}{\cs{chapterheadstartvskip} und
%     \cs{chapterheadendvskip} haben in \textsf{scrartcl} nichts verloren}%^^A
% \changes{v3.15}{2014/11/24}{\cs{scr@chapter@afterskip} wird
%     definiert und verwendet}%^^A
% \changes{v3.15}{2014/12/10}{\cs{vspace} durch \cs{vskip} ersetzt, damit
%     \cs{lastskip} korrekt gesetzt ist}
% \begin{macro}{\chapterheadmidvskip}
% \changes{v3.15}{2014/11/20}{neue Anweisung}%^^A
% \changes{v3.17}{2015/03/25}{\cs{scr@chapter@innerskip} wird
%     definiert und verwendet}%^^A
% Mit diesen Anweisungen wird üblicherweise der Abstand zwischen
% oberem Rand des Textbereichs und der Überschrift bzw. der Abstand
% zwischen der Überschrift und dem Text des Kapitels gesetzt. Man kann
% die Anweisungen aber zusätzlich auch für andere Zwecke
% missbrauchen. Das ist der Grund warum diese Makros keine Längen,
% sondern komplette Anweisungen enthalten. In der Voreinstellung
% enthalten sie jedoch gar nichts. Die tatsächlichen Werte werden
% durch die Optionen festgelegt. Dabei wird \cs{chapterheadmidvskip} nur
% verwendet, wenn Option \texttt{chapterprefix} aktiv ist.
%
% Zur Erhöhung der Kompatibilität mit \cs{DeclareSectionCommand} werden
% zunächst einige Hilfsanweisung definiert und verwendet.
% \begin{macro}{\scr@chapter@beforeskip}
% \changes{v3.15}{2014/11/24}{neue (interne) Anweisung}%^^A
% \changes{v3.17}{2015/03/25}{umbenannt in
%     \cs{scr@chapter@beforeskip}}%^^A
% \begin{macro}{\scr@chapter@afterskip}
% \changes{v3.15}{2014/11/24}{neue (interne) Anweisung}%^^A
% \changes{v3.17}{2015/03/25}{umbenannt in
%     \cs{scr@chapter@afterskip}}%^^A
% \begin{macro}{\scr@chapter@sectionindent}
% \changes{v3.15}{2014/11/24}{neue (interne) Anweisung (unbenutzt)}%^^A
% \changes{v3.17}{2015/04/24}{entfernt}%^^A
% \end{macro}%^^A \scr@chapter@sectionindent
% \end{macro}%^^A \scr@chapter@afterskip
% \end{macro}%^^A \scr@chapter@beforeskip
% \begin{macro}{\scr@chapter@beforeskip}
% \changes{v3.17}{2015/03/25}{Neu (intern) aus Umdefinierung von
%     \cs{scr@chapter@sectionbeforeskip}}%^^A
% \begin{macro}{\scr@chapter@innerskip}
% \changes{v3.17}{2015/03/25}{Neu (intern)}%^^A
% \begin{macro}{\scr@chapter@afterskip}
% \changes{v3.17}{2015/03/25}{Neu (intern) aus Umdefinierung von
%     \cs{scr@chapter@sectionafterskip}}%^^A
% \begin{macro}{\scr@chapter@style}
% \changes{v3.15}{2014/11/24}{neue (interne) Anweisung}%^^A
% Dabei wird der leere Stil verwendet, mit dem Neudefinitionen von \cs{part}
% über \cs{DeclareSectionCommand} verhindert werden. Gleichzeitig können
% jedoch Einstellungen einiger Werte darüber erfolgen.
%    \begin{macrocode}
%<*prepare>
\newcommand*{\scr@chapter@beforeskip}{\z@}
\let\scr@chapter@beforeskip\z@
\newcommand*{\scr@chapter@afterskip}{\z@}
\let\scr@chapter@afterskip\z@
\newcommand*{\scr@chapter@innerskip}{\z@}
\let\scr@chapter@innerskip\z@
\newcommand*{\scr@chapter@style}{chapter}
%    \end{macrocode}
% \end{macro}%^^A \scr@chapter@style
% \end{macro}%^^A \scr@chapter@innerskip
% \end{macro}%^^A \scr@chapter@beforeskip
% \end{macro}%^^A \scr@chapter@afterskip
%    \begin{macrocode}
\newcommand*{\chapterheadstartvskip}{\vspace{\@tempskipa}}
\newcommand*{\chapterheadendvskip}{\vskip\@tempskipa}
\newcommand*{\chapterheadmidvskip}{\par\nobreak\vskip\@tempskipa}
%</prepare>
%    \end{macrocode}
% \end{macro}%^^A \chapterheadmidvskip
% \end{macro}%^^A \chapterheadendvskip
% \end{macro}%^^A \chapterheadstartvskip
%
%
% \begin{macro}{\scr@startchapter}
% \changes{v3.18}{2015/05/22}{neue Anweisung (für Paketautoren)}
% \changes{v3.19}{2015/08/25}{Seitenstil nur ändern wenn definiert und nicht
%     leer}%^^A
% Diese Anweisung legt fest, wie der konkrete Aufruf für Überschriften des
% Stils \texttt{chapter} lautet. Einziges Argument ist dabei der Name der
% Gliederungsebene, beispielsweise \texttt{chapter}. Alle anderen benötigten
% Werte ergeben sich aus den Einstellungen für die Gliederungsebene. Beim
% tatsächlichen Aufruf wird dann daran wie üblich noch optional noch entweder
% ein Stern oder ein Argument für Kolumnentitel und Inhaltsverzeichniseintrag
% und der Text der Überschrift angehängt. Diese Anweisung wurde von der
% ursprünglichen Definition von \cs{chapter} abgeleitet.
%    \begin{macrocode}
%<*body>
\newcommand*{\scr@startchapter}[1]{%
  \ExecuteDoHook{heading/preinit/#1}%
  \if@openright\cleardoublepage\else\clearpage\fi
  \scr@ifundefinedorrelax{#1pagestyle}{}{%
    \Ifstr{#1pagestyle}{}{}{%
      \thispagestyle{\@nameuse{#1pagestyle}}%
    }%
  }%
  \global\@topnum\z@
  \@ifundefined{scr@#1@afterindent}{\@afterindentfalse}{%
    \csname scr@#1@afterindent\endcsname
    {\@afterindenttrue}{\@afterindentfalse}{%
      \@afterindenttrue
      \@ifundefined{scr@#1@beforeskip}{\@afterindentfalse}{%
        \ifdim\glueexpr\@nameuse{scr@#1@beforeskip}\relax<\z@
          \@afterindentfalse
        \fi
      }%  
    }%
  }%
  \ExecuteDoHook{heading/postinit/#1}%
  \expandafter\SecDef\csname @#1\expandafter\endcsname\csname @s#1\endcsname
}
%    \end{macrocode}
%
% \begin{macro}{\scr@@startchapter}
% \changes{v3.18}{2015/05/22}{neue Anweisung (intern)}
% Diese Anweisung nimmt die Rolle des ursprüngliche \cs{@chapter} ein und
% erbst damit auch die ganzen noch relevanten ChangeLog-Einträge dieser
% Anweisung.
%    \begin{macrocode}
\newcommand*{\scr@@startchapter}{}
\def\scr@@startchapter#1[#2]#3{%
  \ExecuteDoHook{heading/branch/nostar/#1}%
%    \end{macrocode}
% \changes{v3.10}{2011/08/30}{Integration der Erweiterung für das optionale
%   Argument der Gliederungsbefehle}%^^A
% \changes{v3.27}{2019/07/08}{neue Option \texttt{nonumber}}%^^A
% Für das erweitere optionale Argument muss hier eine key-value-Auswertung
% erfolgen, falls das erweitere optionale Argument aktiviert ist:
%    \begin{macrocode}
  \ifnum \scr@osectarg=\z@
    \@scr@tempswafalse
  \else
    \scr@istest#2=\@nil
  \fi
  \@currentusenumbertrue  
  \if@scr@tempswa
    \setkeys{KOMAarg.section}{tocentry={#3},head={#3},reference={#3},#2}%
  \else
    \ifcase \scr@osectarg\relax
      \setkeys{KOMAarg.section}{tocentry={#2},head={#2},reference={#2}}%
    \or
      \setkeys{KOMAarg.section}{tocentry={#3},head={#2},reference={#3}}%
    \or
      \setkeys{KOMAarg.section}{tocentry={#2},head={#3},reference={#2}}%
    \or
      \setkeys{KOMAarg.section}{tocentry={#2},head={#2},reference={#2}}%
    \fi
  \fi
%    \end{macrocode}
% Als nächstes muss für nummerierte Überschriften dieser Ebene sowohl der
% Zähler erhöht also auch ein nummerierter Eintrag ins Inhaltsverzeichnis
% erstellt werden.
% \changes{v2.8e}{2001/07/10}{\cs{@maybeautodot} wird aufgerufen}%^^A
% \changes{v3.08}{2010/11/01}{Verwendung von \cs{addchaptertocentry}}%^^A
% \changes{v3.18}{2015/05/20}{Verwendung von \cs{add\meta{Name}tocentry}}%^^A
% \changes{v3.18}{2015/05/22}{Verwendung von \meta{Name} statt \cs{@chapapp}
%   in \cs{typeout}}%^^A
% \changes{v3.28}{2019/11/19}{\cs{ifnumbered} umbenannt in
%   \cs{Ifnumbered}}%^^A
%    \begin{macrocode}
  \Ifnumbered{#1}{%
%<book>    \if@mainmatter
      \@tempswatrue
%<book>    \else\@tempswafalse\fi
  }{\@tempswafalse}%
  \if@tempswa
    \refstepcounter{#1}%
    \expandafter\@maybeautodot\csname the#1\endcsname
    \typeout{#1 \csname the#1\endcsname.}%
    \ifx\@currenttocentry\@empty\else
      \scr@ifundefinedorrelax{add#1tocentry}{%
        \addtocentrydefault{#1}%
      }{%
        \@nameuse{add#1tocentry}%
      }{\csname the#1\endcsname}{\@currenttocentry}%
    \fi
%    \end{macrocode}
% Für alle nicht nummerieten Überschriften dieser Ebene trotzdem die
% Resetliste des Zählers ausführen und ggf. einen Verzeichniseintrag
% erstellen.
%    \begin{macrocode}
  \else
    \expandafter\ifnum\scr@v@is@lt{3.15}\relax\else
      \begingroup
        \let\@elt\@stpelt
        \csname cl@#1\endcsname
      \endgroup
    \fi
    \typeout{#1 without number}%
    \ifx\@currenttocentry\@empty\else
%    \end{macrocode}
% An dieser Stelle muss ggf. der \textsf{hyperref}-Hook für nicht nummerierte
% Inhaltsverzeichniseinträge ausgeführt werden.
%    \begin{macrocode}
      \hy@insteadofrefstepcounter{#1}%
      \scr@ifundefinedorrelax{add#1tocentry}{%
        \addtocentrydefault{#1}%
      }{%
        \@nameuse{add#1tocentry}%
      }{}{\@currenttocentry}%
    \fi
  \fi
%    \end{macrocode}
% Jetzt kommt der Kolumnentitel an die Reihe.
%    \begin{macrocode}
  \csname #1mark\expandafter\endcsname\expandafter{\@currentheadentry}%
%    \end{macrocode}
% Und die Abstände in die andere Verzeichnisse.
% \changes{v2.8g}{2001/07/18}{per \cs{float@addtolists} wird nun
%   auch ein vertikaler Abstand in Listen des \texttt{float}-Pakets
%   eingebaut}%^^A
% \changes{v3.00}{2008/07/01}{jetzt auch Kapiteleinträge in andere
%   float-Verzeichnisse mit hyperref}%^^A
% \changes{v3.28}{2019/11/19}{\cs{iftocfeature} replaced by
%   \cs{Iftocfeature}}%^^A
%    \begin{macrocode}
  \ifdim \@chapterlistsgap>\z@
    \doforeachtocfile{%
      \Iftocfeature{\@currext}{chapteratlist}{%
        \addtocontents{\@currext}{\protect\addvspace{\@chapterlistsgap}}%
      }{}%
    }%
    \@ifundefined{float@addtolists}{}{%
      \scr@float@addtolists@warning
      \float@addtolists{\protect\addvspace{\@chapterlistsgap}}%
    }%
  \fi
%    \end{macrocode}
% Jetzt kommt die eigentliche Überschrift.  An dieser Stelle wird ab
% Version~3.18 nicht mehr stur \cs{@makecshapterhead} verwendet, sondern eine
% Anweisung, die vom Namen der Gliederungsüberschrift abhängig ist.
% \changes{v3.18}{2015/05/22}{\cs{@makechapterhead} durch
%     \cs{@make\meta{Name}head} ersetzt}%^^A
%    \begin{macrocode}
  \if@twocolumn
    \if@at@twocolumn
      \@nameuse{@make#1head}{#3}%
    \else
      \@topnewpage[\@nameuse{@make#1head}{#3}]%
    \fi
  \else
    \@nameuse{@make#1head}{#3}%
  \fi
%    \end{macrocode}
% Und das Ganze endet schließlich mit:
% \changes{v3.18}{2015/05/21}{\cs{@afterheading} in jedem Fall}%^^A
%    \begin{macrocode}
  \@afterheading
}
%    \end{macrocode}
% \end{macro}%^^A \scr@@startchapter
%
% \begin{macro}{\scr@@startschapter}
% \changes{v3.18}{2015/05/22}{neue Anweisung (intern)}
% Diese Anweisung nimmt die Rolle des ursprüngliche \cs{@schapter} ein sie ist
% sehr viel einfacher gestrickt als \cs{scr@@startchapter}.
%    \begin{macrocode}
\newcommand*{\scr@@startschapter}[2]{%
  \ExecuteDoHook{heading/branch/star/#1}%
%    \end{macrocode}
% Für Leute, die unbedingt nach der Sternversion noch ein \cs{addcontentsline}
% anfügen wollen, muss hier ebenfalls der \textsf{hyperref}-Hook ausgeführt
% werden. Allerdings passiert das hier entsprechend dem Vorgehen von
% \textsf{hyperref} selbst in einer Gruppe, in der zusätzlich \cs{@mkboth} zu
% \cs{@gobbletwo} gemacht wird, auch wenn ich absolut nicht versehe, was das
% soll.
%    \begin{macrocode}
  \begingroup
    \let\@mkboth\@gobbletwo
    \hy@insteadofrefstepcounter{#1}%
  \endgroup
%    \end{macrocode}
% Dann wird die eigentliche Überschrift ausgegeben.
%    \begin{macrocode}
  \if@twocolumn
    \if@at@twocolumn
      \@nameuse{@makes#1head}{#2}%
    \else
      \@topnewpage[\@nameuse{@makes#1head}{#2}]%
    \fi
  \else
    \@nameuse{@makes#1head}{#2}%
  \fi
%    \end{macrocode}
% Und das Ganze endet schließlich mit:
% \changes{v3.18}{2015/05/21}{\cs{@afterheading} in jedem Fall}%^^A
%    \begin{macrocode}
  \@afterheading
}
%    \end{macrocode}
% \end{macro}%^^A \scr@@startschapter
%
%% \begin{macro}{\if@at@twocolumn}
% \changes{v2.7b}{2001/01/05}{neu (intern)}%^^A
% \begin{macro}{\scr@topnewpage}
% \changes{v2.7b}{2001/01/05}{neu (intern)}%^^A
% \changes{v2.8q}{2001/11/27}{fehlende Klammern ergänzt}%^^A
% \begin{macro}{\@topnewpage}
% \changes{v2.7b}{2001/01/05}{neu (\LaTeX{} intern)}%^^A
% \changes{v2.7g}{2001/04/17}{vergessenes \cs{long} ergänzt}%^^A
% \changes{v3.01}{2008/11/13}{Verwendung von
%     \cs{scr@float@addtolists@warning} (aus \texttt{scrkliof.dtx})}%^^A
% Interessant ist dabei die Erweiterung, daß jedes Kapitel mit einer
% Präambel versehen werden kann. Diese wird im zweispaltigen Satz wie
% die Überschrift selbst einspaltig gesetzt. Intern wird dies für
% die Preambel des Index und der Bibliography verwendet. Dabei
% entsteht aber wiederum das Problem, dass der Indexkopf innerhalb von
% \cs{twocolumn} ausgegeben wird andererseits aber innerhalb von
% \cs{chapter} das Makro \cs{@topnewpage} verwendet wird. Damit würde
% dann also \cs{@topnewpage} innerhalb von \cs{@topnewpage}
% aufgerufen. Das ist jedoch nicht erlaubt. Als muss der Aufruf von
% \cs{@topnewpage} in \cs{@chapter} und \cs{@schapter} verhindert
% werden, wenn wir uns bereits in \cs{@topnewpage} befinden. Eine
% Möglichkeit dafür wäre, einen Schalter in den Index-Kopf
% einzubauen. Ich wähle hingegen die Methode, \cs{@topnewpage}
% entsprechend umzudefinieren. Damit kann dann auch der Anwender
% \cs{@chapter} innerhalb des optionalen Arguments von \cs{twocolumn}
% verwenden. Ein Schalter wird allerdings trotzdem gebraucht.
%    \begin{macrocode}
\newif\if@at@twocolumn
\newcommand*\scr@topnewpage{}
\let\scr@topnewpage\@topnewpage
\long\def\@topnewpage[#1]{%
  \@at@twocolumntrue\scr@topnewpage[{#1}]\@at@twocolumnfalse
}
%    \end{macrocode}
% \end{macro}%^^A \@topnewpage
% \end{macro}%^^A \scr@topnewpage
% \end{macro}%^^A \if@at@twocolumn
%
% \begin{macro}{\scr@makechapterhead}
% \changes{v3.18}{2015/05/22}{neue Anweisung (intern)}
% Diese Anweisung nimmt die Rolle des ursprüngliche \cs{@makechapterhead}
% ein. Mit dem neuen Mechanismus könnte man auf diesen zusätzlichen Schritt
% verzichten. Aus Gründen der Kompatibilität wurde er aber übernommen. Daher
% übernimmt sie auch die ChangeLog-Einträge der alten Anweisung.
% \changes{v2.7b}{2001/01/05}{Kapitel können grundsätzlich mit einer
%     Präambel versehen werden}%^^A
% \changes{v2.8p}{2001/09/25}{es gibt verschiedene Arten von Präambeln}%^^A
% \changes{v3.15}{2014/12/10}{\cs{nobreak} am Ende entfernt}%^^A
%    \begin{macrocode}
\newcommand*{\scr@makechapterhead}[2]{%
  \use@chapter@o@preamble{#1}%
  \@nameuse{@@make#1head}{#2}%
  \use@preamble{#1@u}%
}
%    \end{macrocode}
% \end{macro}%^^A \scr@makechapterhead
%
% \begin{macro}{\scr@makeschapterhead}
% \changes{v3.18}{2015/05/22}{neue Anweisung (intern)}
% Diese Anweisung nimmt die Rolle des ursprüngliche \cs{@makeschapterhead}
% ein. Mit dem neuen Mechanismus könnte man auf diesen zusätzlichen Schritt
% verzichten. Aus Gründen der Kompatibilität wurde er aber übernommen. Daher
% übernimmt sie auch die ChangeLog-Einträge der alten Anweisung.
%    \begin{macrocode}
\newcommand*{\scr@makeschapterhead}[2]{%
  \use@chapter@o@preamble{#1}%
  \@nameuse{@@makes#1head}{#2}%
  \use@preamble{#1@u}%
}
%    \end{macrocode}
% \end{macro}%^^A \scr@makeschapterhead
%
% \begin{macro}{\scr@@makechapterhead}
% \changes{v3.18}{2015/05/22}{neue Anweisung (intern)}
% Diese Anweisung nimmt die Rolle des ursprüngliche \cs{@@makechapterhead}
% ein. Daher übernimmt sie auch die ChangeLog-Einträge der alten Anweisung.
% \changes{v2.8q}{2002/04/18}{\cs{raggedsection} wird bei Option
%   \texttt{chapterprefix} auch auf den Präfix angewendet}%^^A
% \changes{v2.9p}{2003/06/28}{\cs{parfillskip} auf \cs{fill} gesetzt}%^^A
% \changes{v2.96a}{2006/12/02}{Abstand nach der Präfixzeile nur, wenn eines
%   Präfixzeile ausgegeben wurde}%^^A
% \changes{v2.97b}{2007/03/09}{\cs{endgraf} korrigiert}%^^A
% \changes{v3.15}{2014/11/20}{Generalumbau}%^^A
% \changes{v3.15}{2014/12/09}{\cs{raggedchapter} ersetzt
%   \cs{raggedsection}}%^^A
% \changes{v3.17}{2015/04/03}{Verwendung von \cs{usefontofkomafont} für die
%   initiale Fontumstellung}%^^A
%    \begin{macrocode}
\newcommand*{\scr@@makechapterhead}[2]{%
%    \end{macrocode}
% Der Absolutwert des Abstandes davor wird eingefügt.
% \changes{v3.22}{2016/12/21}{\cs{@afterindentrue} wird bei positivem
%     \texttt{beforeskip} gesetzt}%^^A
% \changes{v3.26}{2018/09/18}{Beachtung von Einstellung
%   \texttt{afterindent}}%^^A
%    \begin{macrocode}
  \@tempskipa=\glueexpr \csname scr@#1@beforeskip\endcsname\relax\relax
  \csname scr@#1@afterindent\endcsname
  {\@afterindenttrue}{\@afterindentfalse}{%
    \ifdim\@tempskipa<\z@\@tempskipa-\@tempskipa\else
      \expandafter\ifnum\scr@v@is@ge{3.22}\@afterindenttrue\fi
    \fi
  }%
  \chapterheadstartvskip
%    \end{macrocode}
% Dann die eigentliche Überschrift innerhalb einer Gruppe.
% \changes{v3.20}{2015/12/23}{\cs{parfillskip} auf \texttt{\cs{z@} plus
%   1fil} gesetzt}%^^A
% \changes{v3.27}{2019/02/02}{Setzen von \cs{parindent} und \cs{parfillskip}
%   verschoben}%^^A
%    \begin{macrocode}
  {%
%    \end{macrocode}
% \changes{v3.19}{2015/08/02}{Verwendung von \cs{chapterlinesformat} und
%   \cs{chapterlineswithprefixformat}}%^^A
% \changes{v3.19}{2015/08/02}{Verwendung von Element \texttt{disposition} an
%   Stelle von Befehl \cs{sectfont}}%^^A
% \changes{v3.19a}{2015/10/03}{Bug durch vergessene Definition von
%   \cs{IfUseNumber} behoben}%^^A
% \changes{v3.27}{2019/02/02}{\cs{ExecuteDoHook} eingefügt}%^^A
% \changes{v3.28}{2019/11/19}{\cs{ifnumbered} umbenannt in
%   \cs{Ifnumbered}}%^^A
% Ab Version~3.19 wurde der Code weitgehend umgebaut, um die Befehle
% \cs{chapterlinesformat} für die hängende Version und
% \cs{chapterlinesandprefixformat} für die Version mit Präfixzeile zu
% verwenden. Dadurch entfällt dann auch die Verwendung von
% \cs{usefontofkomafont}. Ich bin mir bewusst, dass dadurch diverse Hacks
% bezüglich der Kapitelüberschriften nicht mehr funktionieren. Dafür kann man
% diese Hacks nun durch saubere Lösungen ersetzen.
%    \begin{macrocode}
    \def\IfUseNumber{\Ifnumbered{#1}}%
%<book>    \if@mainmatter\else\let\IfUseNumber\@secondoftwo\fi
    \ExecuteDoHook{heading/begingroup/#1}%  
    \if@chapterprefix
      \let\IfUsePrefixLine\@firstoftwo
    \else
      \let\IfUsePrefixLine\@secondoftwo
    \fi
    \setlength{\parindent}{\z@}\setlength{\parfillskip}{\z@ plus 1fil}%
    \normalfont\usekomafont{disposition}{%
      \usekomafont{#1}{%
        \settoheight{\@tempskipa}{%
          {\usekomafont{#1prefix}{%
              \vrule \@width\z@ \@height\csname scr@#1@innerskip\endcsname}}%
        }%
        \raggedchapter
        \IfUseNumber{%
          \IfUsePrefixLine{%
            \chapterlineswithprefixformat{#1}%
            {{\usekomafont{#1prefix}{\csname #1format\endcsname%
                  \setlength{\@tempskipa}{\csname scr@#1@innerskip\endcsname}%
                  \chapterheadmidvskip}}}%
            {\interlinepenalty \@M#2\@@par}%
          }{%
            \chapterlinesformat{#1}%
            {\csname #1format\endcsname}%
            {\interlinepenalty \@M#2\@@par}%
          }%
        }{%
          \IfUsePrefixLine{%
            \chapterlineswithprefixformat{#1}%
            {}%
            {\interlinepenalty \@M#2\@@par}%
          }{%
            \chapterlinesformat{#1}%
            {}%
            {\interlinepenalty \@M#2\@@par}%
          }%
        }%
      }%
    }%
    \ExecuteDoHook{heading/endgroup/#1}%
  }%
  \nobreak\par\nobreak
%    \end{macrocode}
% Zum Schluss der Absolutwert des Abstandes dahinter.
% \changes{v3.26}{2018/09/20}{Betragsbildung entfernt}%^^A
%    \begin{macrocode}
  \@tempskipa=\glueexpr \csname scr@#1@afterskip\endcsname\relax\relax
  \chapterheadendvskip
}
%    \end{macrocode}
% \end{macro}%^^A \scr@@makechapterhead
%
% \begin{macro}{\scr@@makeschapterhead}
% \changes{v3.18}{2015/05/22}{neue Anweisung (intern)}
% Diese Anweisung nimmt die Rolle des ursprüngliche \cs{@@makechapterhead}
% ein. Sie ist um einiges einfacher als \cs{scr@@makechapterhead} orientiert
% sich aber prinzipiell daran.
% \changes{v2.9p}{2003/06/28}{\cs{parfillskip} auf \cs{fill} gesetzt}%^^A
% \changes{v2.9q}{2004/03/24}{Klammerung an \cs{@@makechapterhead}%^^A
%   angepasst}%^^A
% \changes{v3.15}{2014/12/09}{\cs{raggedchapter} ersetzt
%  \cs{raggedsection}}%^^A
% \changes{v3.19}{2015/08/02}{an die Änderungen bei \cs{scr@@makechapterhead}
%  angepasst}%^^A
% \changes{v3.20}{2015/12/23}{\cs{parfillskip} auf \texttt{\cs{z@} plus
%   1fil} gesetzt}%^^A
% \changes{v3.22}{2016/12/21}{\cs{@afterindentrue} wird bei positivem
%   \texttt{beforeskip} gesetzt}%^^A
% \changes{v3.25}{2017/09/07}{\cs{IfUseNumber} korrigiert}%^^A
% \changes{v3.26}{2018/09/18}{Beachtung von Einstellung
%   \texttt{afterindent}}%^^A
% \changes{v3.26}{2018/09/20}{Betragsbildung für \texttt{afterskip}
%   entfernt}%^^A
%    \begin{macrocode}
\newcommand*{\scr@@makeschapterhead}[2]{%
  \@tempskipa=\glueexpr \csname scr@#1@beforeskip\endcsname\relax\relax
  \csname scr@#1@afterindent\endcsname
  {\@afterindenttrue}{\@afterindentfalse}{%
    \ifdim\@tempskipa<\z@\@tempskipa-\@tempskipa\else
      \expandafter\ifnum\scr@v@is@ge{3.22}\@afterindenttrue\fi
    \fi
  }%
  \chapterheadstartvskip
  {%
    \let\IfUseNumber\@secondoftwo
    \ExecuteDoHook{heading/begingroup/#1}%  
    \if@chapterprefix
      \let\IfUsePrefixLine\@firstoftwo
    \else
      \let\IfUsePrefixLine\@secondoftwo
    \fi
    \setlength{\parindent}{\z@}\setlength{\parfillskip}{\z@ plus 1fil}%
    \normalfont\usekomafont{disposition}{%
      \usekomafont{#1}{%
        \raggedchapter
        \IfUsePrefixLine{%
          \chapterlineswithprefixformat{#1}%
          {}%
          {\interlinepenalty \@M#2\@@par}%
        }{%
          \chapterlinesformat{#1}%
          {}%
          {\interlinepenalty \@M#2\@@par}%
        }%
      }%
    }%
    \ExecuteDoHook{heading/endgroup/#1}%
  }%
  \nobreak\par\nobreak
  \@tempskipa=\glueexpr \csname scr@#1@afterskip\endcsname\relax\relax
  \chapterheadendvskip
}
%    \end{macrocode}
% \end{macro}%^^A \scr@@makeschapterhead
%
% \begin{macro}{\chapterlinesformat}
% \changes{v3.19}{2015/08/02}{neue Anweisung}
% \begin{macro}{\chapterlineswithprefixformat}
% \changes{v3.19}{2015/08/02}{neue Anweisung}
% Diese Anweisung definiert das Format der eigentlichen Überschriften im Stil
% \texttt{chapter}, wenn mit Überschriftenzeilen gearbeitet
% wird. Fonteinstellungen sind zu diesem Zeitpunkt bereits erfolgt. Die
% Argumente sind:
% \begin{description}
% \item[\meta{Befehlsname} --] der Name der Gliederungsebene, normalerweise
%   \texttt{chapter}.
% \item[\meta{Gliederungsnummer} --] die bereits fertig formatierte
%   Gliederungsnummer (bei \cs{chapterlineswithprefixformat} einschließlich
%   des nachfolgenden vertikalen Abstands) oder leer, falls keine
%   Gliederungsnummer auszugeben ist.
% \item[\meta{Text} --] der Text der Überschrift.
% \end{description}
% Der Anwender ist selbst verantwortlich, dass innerhalb der Überschrift kein
% Seitenumbruch erfolgen kann. Nach der Überschrift wird jedoch zwangsweise
% ein \cs{@@par} ausgeführt, so dass sichergestellt ist, dass mit einem
% internen Absatz abgeschlossen wird.
%    \begin{macrocode}
\newcommand{\chapterlinesformat}[3]{%
  \@hangfrom{#2}{#3}%
}
\newcommand{\chapterlineswithprefixformat}[3]{%
  #2#3%
}
%    \end{macrocode}
% \end{macro}%^^A \chapterlineswithprefixformat
% \end{macro}%^^A \chapterlinesformat
%
% \begin{macro}{\use@chapter@o@preamble}
% \changes{v3.15}{2014/12/08}{Anpassung an geänderte Abstandsbehandlung}%^^A
% \changes{v3.15a}{2015/02/02}{Korrektur, zur Verhinderung falscher
%     \texttt{overfull \cs{vbox}} Meldungen}%^^A
% \changes{v3.18}{2015/05/20}{Verwendung befehlsabhängiger Abstände und
%     Warnungen}%^^A
% \changes{v3.18}{2015/05/22}{Bekommt ein Argument und setzt die Prämabel
%     der Anweisung nicht des Stils}%^^A
% Das Makro wird benötigt, weil das Setzen der oberen Präambel etwas
% mehr Aufwand bedeutet. Die Anweisung spart Platz, da sie zweimal
% verwendet wird.
%    \begin{macrocode}
\newcommand*{\use@chapter@o@preamble}[1]{%
  {%
    \settoheight{\@tempdima}{%
      \@tempskipa=\glueexpr \csname scr@#1@beforeskip\endcsname\relax\relax
      \ifdim\@tempskipa<\z@\@tempskipa-\@tempskipa\fi
      \vbox{\chapterheadstartvskip}%
    }%
    \settodepth{\@tempdimb}{%
      \@tempskipa=\glueexpr \csname scr@#1@beforeskip\endcsname\relax\relax
      \ifdim\@tempskipa<\z@\@tempskipa-\@tempskipa\fi
      \vbox{\chapterheadstartvskip}%
    }%
    \addtolength{\@tempdima}{\@tempdimb}%
    \setbox\z@\vbox{%
      \use@preamble{#1@o}%
    }%
    \setlength{\@tempdimb}{\ht0}%
    \addtolength{\@tempdimb}{\dp0}%
    \vbox to \z@{%
      \vskip-\baselineskip
      \vbox to \@tempdima{%
        \vfill
        \box\z@
      }%
      \vss
    }\vskip-\parskip\vskip-\baselineskip
    \ifdim \@tempdimb>\@tempdima%
      \addtolength{\@tempdimb}{-\@tempdima}%
      \ifdim\@tempdimb<\vfuzz
        \ClassInfo{\KOMAClassName}{%
          preamble before #1 is \the\@tempdimb\space to high.\MessageBreak
          Tolerated without warning because of\MessageBreak
          \string\vfuzz\space = \the\vfuzz\space >= \the\@tempdimb
        }%
      \else
      \ClassWarning{\KOMAClassName}{%
        preamble before #1 is \the\@tempdimb\space to high.\MessageBreak
        To avoid the overfull \string\vbox\space you may
        change\MessageBreak
        the `beforeskip' value using
        \string\RedeclareSectionCommand\MessageBreak
        at the preamble of your document.\MessageBreak
        You may also change \string\setchapterpreamble\MessageBreak 
        before the command, which generates the message
        about\MessageBreak
        an overfull \string\vbox
      }%
      \fi
    \fi%
  }%
}
%    \end{macrocode}
% \end{macro}%^^A \use@chapter@o@preamble
% \end{macro}%^^A \scr@startchapter
%
% \begin{macro}{\addchaptertocentry}
% \changes{v3.08}{2010/11/01}{Neu}%^^A
% \changes{v3.12}{2013/09/24}{Behandlung von \cs{if@chaptertolists} hier}%^^A
% \changes{v3.12}{2013/09/24}{Verwendung von \cs{addxcontentsline}}%^^A
% \changes{v3.28}{2019/11/19}{\cs{iftocfeature} replaced by
%   \cs{Iftocfeature}}%^^A
% Seit Version~3.08 wird der Eintrag nicht direkt innerhalb von \cs{chapter}
% per \cs{addcontentsline} erzeugt, sondern indirekt über diese Anweisung. Das
% erste Argument ist dabei die (formatierte) Nummer bzw. bei nicht
% nummerierten Kapiteln leer. Das zweite Argument ist der Überschriftstext für
% das Verzeichnis. Durch diesen indirekten Weg, kann die Anweisung einfach
% umdefiniert werden. Verwendet wird hier die Standardanweisung für
% Inhaltsverzeichniseinträge, die in \texttt{scrkliof.dtx} definiert ist:
%    \begin{macrocode}
\newcommand*{\addchaptertocentry}[2]{%
  \addtocentrydefault{chapter}{#1}{#2}%
  \if@chaptertolists
    \doforeachtocfile{%
      \Iftocfeature{\@currext}{chapteratlist}{%
        \addxcontentsline{\@currext}{chapteratlist}[{#1}]{#2}%
      }{}%
    }%
    \@ifundefined{float@addtolists}{}{\scr@float@addtolists@warning}%
  \fi
}
%    \end{macrocode}
% \end{macro}%^^A \addchaptertocentry
%
% \begin{macro}{\@chapter}
% \changes{v3.18}{2015/05/22}{siehe \cs{scr@dsc@def@style@chapter@command}
%     und \cs{scr@@startchapter}}%^^A
% \begin{macro}{\@makechapterhead}
% \changes{v3.18}{2015/05/22}{siehe \cs{scr@dsc@def@style@chapter@command}
%     und \cs{scr@makechapterhead}}%^^A
% \begin{macro}{\@@makechapterhead}
% \changes{v3.18}{2015/05/22}{siehe \cs{scr@dsc@def@style@chapter@command}
%     und \cs{scr@@makechapterhead}}%^^A
% \begin{macro}{\@schapter}
% \changes{v3.18}{2015/05/22}{siehe \cs{scr@dsc@def@style@chapter@command}
%     und \cs{scr@@startschapter}}%^^A
% \begin{macro}{\@makeschapterhead}
% \changes{v3.18}{2015/05/22}{siehe \cs{scr@dsc@def@style@chapter@command}
%     und \cs{scr@makeschapterhead}}%^^A
% \begin{macro}{\@@makeschapterhead}
% \changes{v3.18}{2015/05/22}{siehe \cs{scr@dsc@def@style@chapter@command}
%     und \cs{scr@@makeschapterhead}}%^^A
% \begin{macro}{\setchapterpreamble}
% \changes{v2.7b}{2001/01/05}{neu}%^^A
% \changes{v3.18}{2015/05/22}{siehe \cs{scr@dsc@def@style@chapter@command}}%^^A
% \begin{macro}{\chapter@u@preamble}
% \changes{v2.8p}{2001/09/25}{neu (intern)}%^^A
% \changes{v3.18}{2015/05/22}{siehe \cs{scr@dsc@def@style@chapter@command}}%^^A
% \begin{macro}{\chapter@o@preamble}
% \changes{v2.8p}{2001/09/25}{neu (intern)}%^^A
% \changes{v3.18}{2015/05/22}{siehe \cs{scr@dsc@def@style@chapter@command}}%^^A
% All diese Anweisungen wurden in früheren Versionen von \KOMAScript{}
% explizit definiert. Nun werden sie nur noch indirekt über
% \cs{scr@dsc@def@style@chapter@command} definiert und verwenden dabei andere
% interne Anweisungen als früher. Deshalb funktionieren ggf. das Patchen mit
% \textsf{etoolbox}, \textsf{xpatch} und ähnlichen Paketen nicht mehr. Das ist
% der Preis für die neuen Möglichkeiten.
% \end{macro}%^^A \chapter@o@preamble
% \end{macro}%^^A \chapter@u@preamble
% \end{macro}%^^A \setchapterpreamble
% \end{macro}%^^A \@@makeschapterhead
% \end{macro}%^^A \@makeschapterhead
% \end{macro}%^^A \@schapter
% \end{macro}%^^A \@@makechapterhead
% \end{macro}%^^A \@makechapterhead
% \end{macro}%^^A \@chapter
%
% \begin{macro}{\addchap}
% \changes{v2.0e}{1994/08/10}{\cs{sectdef} durch \cs{secdef} ersetzt}%^^A
% \changes{v2.0e}{1994/08/10}{\cs{addcontensline} durch
%     \cs{addcontentsline} ersetzt}%^^A
% \changes{v2.0e}{1994/08/10}{\cs{@makechapterhead} durch
%     \cs{@makeschapterhead} ersetzt}%^^A
% \changes{v2.0e}{1994/08/10}{\cs{afterheadings} durch
%     \cs{@afterheading} ersetzt}%^^A
% \changes{v2.1a}{1994/10/29}{Argument von \cs{chaptermark}%^^A
%     entfernt bei \textsf{scrbook}}%^^A
% \changes{v2.0e}{1994/10/12}{mehrfach \cs{markboth} durch
%     \cs{sectionmark} ersetzt}%^^A
% \changes{v2.2b}{1995/03/20}{\cs{chaptermark} durch \cs{@mkboth}
%     ersetzt}%^^A
% \changes{v2.8d}{2001/07/05}{\cs{chapterpagestyle} statt
%     \texttt{plain}}%^^A
% \changes{v3.00}{2008/07/01}{jetzt auch Kapiteleinträge in andere
%     float-Verzeichnisse mit hyperref}%^^A
% \changes{v3.01}{2008/11/13}{Verwendung von
%     \cs{scr@float@addtolists@warning} (aus \texttt{scrkliof.dtx})}%^^A
% \changes{v3.08}{2010/11/01}{Verwendung von \cs{addchaptertocentry}}%^^A
% \changes{v3.10}{2011/08/30}{Integration der Erweiterung für das optionale
%     Argument der Gliederungsbefehle}%^^A
% \changes{v3.13a}{2014/09/11}{Verwendung von \cs{SecDef}}%^^A
% \changes{v3.18}{2015/05/22}{Komplett neu definiert}%^^A
% \KOMAScript{} bietet in \textsf{scrbook} und \textsf{scrreprt} den
% zusätzlichen Gliederungsbefehl \cs{addchap}. Es handelt sich dabei
% um einen mit \cs{chapter*} vergleichbaren Befehl, bei dem jedoch der
% Kolumnentitel korrigiert wird und ein Eintrag ins Inhaltsverzeichnis
% erfolgt (nicht in der Stern-Variante). 
%
% Die Definition von \cs{addcap} sah bis zu Version 2.7 so aus:
% \begin{verbatim}
% \newcommand\addchap{\if@openright\cleardoublepage\else\clearpage\fi
%                     \thispagestyle{plain}%
%                     \global\@topnum\z@
%                     \@afterindentfalse
%                     \secdef\@addchap\@saddchap}
% \def\@addchap[#1]#2{\typeout{#2}
%                     \addcontentsline{toc}{chapter}{#1}
%                     \if@twoside\@mkboth{#1}{}\else\@mkboth{}{#1}\fi
%                     \addtocontents{lof}{\protect\addvspace{10\p@}}%
%                     \addtocontents{lot}{\protect\addvspace{10\p@}}%
%                     \if@twocolumn
%                          \@topnewpage[\@makeschapterhead{#2}]%
%                     \else
%                          \@makeschapterhead{#2}%
%                          \@afterheading
%                     \fi}
% \def\@saddchap#1{\@mkboth{}{}
%                  \if@twocolumn
%                       \@topnewpage[\@makeschapterhead{#1}]%
%                  \else
%                       \@makeschapterhead{#1}%
%                       \@afterheading
%                  \fi}
% \end{verbatim}\vspace{-\baselineskip}
% Dies führte jedoch zu Problemen im Zusammenhang mit
% \textsf{hyperref}, da hierbei keine korrekten Links erzeugt
% werden. Mit dem \texttt{hypertex}-Treiber von \textsf{hyperref}
% bis Version 6.71a werden auch in der neuen Implementierung keine
% korrekten Links erzeugt. Dies ist jedoch ein Bug in jenem
% Treiber. Mit neuen Versionen tritt das Problem nicht mehr auf.
% Deshalb wurde der Code wie folgt vereinfacht. Von wesentlicher
% Bedeutung ist dabei der Aufruf von \cs{chapter*}, da in
% \texttt{hpdftex.def} das interne Makro \cs{@schapter} so umdefiniert
% wird, dass ein korrekter Link erzeugt wird.
% \begin{macro}{\@addchap}
% \changes{v3.12}{2013/02/26}{Verwendung von \cs{addchapmark}}%^^A
% \changes{v3.17}{2015/04/20}{\cs{scr@ds@head} in \cs{addchapmark}
%     expandieren}%^^A
% \changes{v3.18}{2015/05/22}{Komplett neu definiert}%^^A
% \begin{macro}{\@saddchap}
% \changes{v3.12}{2013/02/26}{Verwendung von \cs{addchapmark}}%^^A
% \changes{v3.18}{2015/05/22}{Komplett neu definiert}%^^A
% Seit \KOMAScript~3.18 wurden die Definitionen noch einmal deutlich
% vereinfacht. Deshalb wird nun auch nur noch für \cs{addchap*} die neue
% Anweisung \cs{addchapmark} benötigt. Bei \cs{addchap} geht es hingegen
% einfach über \cs{chaptermark} innerhalb von \cs{chapter}.
%    \begin{macrocode}
\newcommand\addchap{%
  \SecDef\@addchap\@saddchap
}%
\newcommand*{\@addchap}{}%
\long\def\@addchap[#1]#2{%
  \edef\reserved@a{%
    \unexpanded{%
      \chapter[{#1}]{#2}%
      \c@secnumdepth=
    }\the\c@secnumdepth\relax
  }%
  \c@secnumdepth=\numexpr \chapternumdepth-1\relax
  \reserved@a
}
\newcommand{\@saddchap}[1]{%
  \chapter*{#1}%
  \addchapmark{}%
}%
%    \end{macrocode}
% In \texttt{hyperref} bis Version 6.71a ist ein dicker Bug. In
% verschiedenen \texttt{def}-Dateien wird dort das \emph{interne}
% \KOMAScript-Makro \cs{@addchap} umdefiniert, ohne sicherzustellen,
% dass das Makro auch noch so aussieht, wie man das erwartet hat. Die
% sauberste Lösung wäre sicher gewesen dort \cs{@addchap} nur zu
% ergänzen, so wie man das auch bei \cs{@schapter} gemacht hatte. Ab
% Version 6.71b ist dieser Bug behoben. Damit ich nicht eines Fehlers
% verdächtigt werde, wird ggf. eine entsprechende Meldung ausgegeben.
% \changes{v2.7e}{2001/04/16}{fehlende, schließende Klammer
%      ergänzt}%^^A
% \changes{v2.7f}{2001/04/17}{fehlendes Klammerpaar ergänzt}%^^A
% \changes{v2.8q}{2001/11/17}{\cs{AfterPackage} zur Überprüfung der
%      \textsf{hyperref}-Version verwendet}%^^A
% \changes{v3.27a}{2019/11/11}{Meldung korrigiert}%^^A
%    \begin{macrocode}
%<*book>
\AfterPackage{hyperref}{%
  \@ifpackagelater{hyperref}{2001/02/19}{}{%
    \ClassWarningNoLine{\KOMAClassName}{%
      You are using an old version of the hyperref package!\MessageBreak%
      This version has a buggy hack in many drivers,\MessageBreak%
      causing \string\addchap\space to behave strangely.\MessageBreak%
      Please update hyperref to at least version 6.71b%
    }%
  }%
}
%</book>
%</body>
%    \end{macrocode}
% \end{macro}%^^A \@saddchap
% \end{macro}%^^A \@addchap
% \end{macro}%^^A \addchap
%
%
% \begin{macro}{\l@chapter}
% \changes{v2.9k}{2003/01/02}{Anpassung an Option \texttt{tocleft}}%^^A
% \changes{v2.96b}{2006/11/30}{Umbruch zwischen \cs{chapter}-Eintrag im
%     Inhaltsverzeichnis und übergeordneten Einträgen verhindern}%^^A
% \changes{v2.96b}{2006/11/30}{Umbruch zwischen \cs{chapter}-Eintrag im
%     Inhaltsverzeichnis und untergeordneten Einträgen verhindern}%^^A
% \changes{v2.97c}{2007/06/21}{\cs{sectfont} durch Verwendung von
%     Element \texttt{chapterentry} ersetzt}%^^A
% \changes{v2.97c}{2007/06/21}{Element \texttt{chapterentrypagenumber} wird
%     verwendet}%^^A
% \changes{v2.97c}{2007/06/21}{weitgehend umgeschrieben, um linksbündige
%     Einträge zu ermöglichen}%^^A
% \changes{v3.15}{2014/12/22}{Verwendung von \cs{chaptertocdepth} und
%     \cs{scr@chapter@tocnumwidth}}%^^A
% \changes{v3.15}{2014/12/10}{Verwendung von \cs{addvspace} statt \cs{vskip}
%     für den Abstand am Anfang}%^^A
% \changes{v3.20}{2015/10/06}{überführt in Verzeichnisstil}%^^A
% Für die Definition des Kapiteleintrags ins Inhaltsverzeichnis werden einige
% Parameter von \cs{DeclareSectionCommand} definiert und teilweise
% verwendet. Damit kann man dann zumindest einige Änderungen mit
% \cs{DeclareSectionCommand} vornehmen, allerdings noch keine komplette
% Neugestaltung des Eintrags erreichen.
% \begin{macro}{\scr@chapter@tocindent}
% \changes{v3.15}{2014/12/02}{neue (interne) Anweisung}%^^A
% \changes{v3.18}{2015/05/20}{indirekt über \cs{DeclareSectionCommand}}%^^A
% \changes{v3.20}{2015/11/06}{wird nun auch verwendet}%^^A
% Der Einzug des Eintrags noch vor der Nummer.
% \end{macro}%^^A \scr@chapter@tocindent
% \begin{macro}{\scr@chapter@tocnumwidth}
% \changes{v3.15}{2014/12/02}{neue (interne) Anweisung}%^^A
% \changes{v3.18}{2015/05/20}{indirekt über \cs{DeclareSectionCommand}}%^^A
% Die Breite der Nummer des Eintrags wird tatsächlich verwendet.
% \end{macro}
% \begin{option}{chapterentrydots}
% \changes{v3.15}{2014/12/10}{Neu}%^^A
% Seit Version~3.15 kann man mit dieser Option Pünktchen zwischen Text und
% Seitenzahl in den Kapiteleinträgen einschalten.
%    \begin{macrocode}
%<*option>
\KOMA@ifkey{chapterentrydots}{@chapterentrywithdots}
%</option>
%    \end{macrocode}
% \end{option}%^^A chapterentrydots
% \begin{macro}{\raggedchapterentry}
% \changes{v2.97c}{2007/06/21}{neue Anweisung}%^^A
% \changes{v3.20}{2016/02/23}{wird nicht mehr verwendet}%^^A
% \changes{v3.21}{2016/06/06}{siehe \texttt{scrkernel-tocstyle.dtx},
%     Stil \texttt{tocline}, Eigenschaft \texttt{raggedentrytext}}%^^A
%   Ausrichtung des Textes der Kapiteleinträge im Inhaltsverzeichnis.
%    \begin{macrocode}
%<*body>
\newcommand*{\raggedchapterentry}{}
%    \end{macrocode}   
% \end{macro}%^^A \raggedchapterentry
% \end{macro}%^^A \l@chapter
%
% \begin{macro}{\l@chapteratlist}
% \changes{v2.96a}{2006/12/03}{neues Macro}%^^A
% Dieses Makro wird ggf. für die Kapiteleinträge in den Listen der
% Gleitumgebungen verwendet. Zunächst einmal wird hier dasselbe verwendet, wie
% für die Kapiteleinträge ins Inhaltsverzeichnis.
%    \begin{macrocode}
\newcommand*{\l@chapteratlist}{\l@chapter}
%    \end{macrocode}   
% \end{macro}%^^A \l@chapteratlist
%
% \begin{Counter}{chapter}
% \begin{macro}{\thechapter}
% \begin{macro}{\chapterformat}
% \changes{v2.3c}{1995/08/06}{Duden Regel 6}%^^A
% \changes{v2.7}{2000/01/03}{einfaches Leerzeichen durch \cs{enskip}%^^A
%     ersetzt}%^^A 
% \changes{v2.8}{2001/06/15}{\cs{chapappifprefix} eingefügt}%^^A
% \changes{v2.8o}{2001/09/19}{\cs{chapappifchapterprefix} statt
%     \cs{chapappifprefix}} ^^A
% \changes{v2.96a}{2006/12/02}{\cs{mbox} eingefügt, um \cs{caps} zu
%     ermöglichen}%^^A
% \changes{v3.17}{2015/03/08}{\cs{enskip} not in prefix line mode}%^^A
% \begin{macro}{\chaptermarkformat}
% \changes{v2.3a}{1995/07/08}{Leerraum nach der Kapitelnummer erhöht}%^^A
% \changes{v2.3c}{1995/08/06}{Duden Regel 6}%^^A
% \changes{v2.8}{2001/06/15}{\cs{chapappifprefix} eingefügt}%^^A
% \changes{v2.8o}{2001/09/19}{\cs{chapappifchapterprefix} statt
%     \cs{chapappifprefix}}%^^A
% \changes{v3.03a}{2009/04/03}{Tilde durch \cs{nobreakspace} ersetzt, um die
%     Kompatibilität mit Spanisch zu verbessern}%^^A
% Jede Gliederungsebene benötigt einen Zähler für die
% Gliederungsnummer und eine Darstellung des Zählers
% (\cs{the\dots}). Desweiteren wird eine Formatierung des Zählers in
% den Gliederungsüberschriften (\texttt{\bslash\dots format}) und eine
% Formatierung des Zählers in der Kopfzeile 
% (\texttt{\bslash\dots markformat}) benötigt.
%    \begin{macrocode}
\newcounter{chapter}
\renewcommand*{\thechapter}{\@arabic\c@chapter}
\newcommand*{\chapterformat}{%
  \mbox{\chapappifchapterprefix{\nobreakspace}\thechapter\autodot
    \IfUsePrefixLine{}{\enskip}}%
}
\newcommand*\chaptermarkformat{\chapappifchapterprefix{\ }%
  \thechapter\autodot\enskip}
%    \end{macrocode}
% \end{macro}%^^A \chaptermarkformat
% \end{macro}%^^A \chapterformat
% \end{macro}%^^A \thechapter
% \end{Counter}%^^A chapter
%
%
% \begin{macro}{\chaptername}
% Name der Gliederungsebenen \cs{chapter}.
%    \begin{macrocode}
\newcommand*\chaptername{Chapter}
%    \end{macrocode}
% \end{macro}%^^A \chaptername
%
%
% \begin{macro}{\appendixname}
% Der Name des Anhangs.
%    \begin{macrocode}
\newcommand*\appendixname{Appendix}
%    \end{macrocode}
% \end{macro}%^^A \appendixname
%
%
% \begin{macro}{\chapappifprefix}
% \changes{v2.8}{2001/06/15}{neu}%^^A
% \changes{v2.8o}{2001/09/19}{obsolet}%^^A
% \begin{macro}{\chapappifchapterprefix}
% \changes{v2.8o}{2001/09/19}{neu}%^^A
% \changes{v2.8o}{2001/09/19}{Argument ist nicht optional}%^^A
% \changes{v3.18}{2015/06/09}{verwendet \cs{IfChapterUsePrefixLine}}%^^A
% \begin{macro}{\IfChapterUsesPrefixLine}
% \changes{v3.18}{2015/06/09}{neu (intern)}
% \begin{macro}{\chapapp}
% \changes{v2.8}{2001/06/15}{neu}%^^A
% Das in \cs{chapterformat} verwendete Makro
% \cs{chapappifchapterprefix} setzt abhängig von \cs{if@chapterprefix}
% noch das Makro \cs{chapapp} gefolgt vom obliatorischen Argument.
% \cs{chapapp} macht \cs{@chapapp} auf Anwenderebene verfügbar. Der
% Vorteil des neuen \cs{chapappifchapterprefix} gegenüber dem alten
% \cs{chapappifprefix} ist, dass nun beim Umdefinieren von
% \cs{chapterformat} und \cs{chaptermarkformat} wieder
% \cs{renewcommand} verwendet werden kann und beim Umschalten in den
% Anhang die Kopfzeile korrekt ist, weil das Makro nun nicht mehr
% geschützt ist, sondern direkt expandiert. 
%    \begin{macrocode}
\newcommand*{\chapappifprefix}[1][]{%
  \ClassWarning{\KOMAClassName}{%
    Please don't use obsolete command
    \string\chapappifprefix.\MessageBreak
    The new command \string\chapappifchapterprefix\space has
    an\MessageBreak
    obligatory instead of an optional argument. Use that\MessageBreak
    new command%
  }%
  \chapappifchapterprefix{#1}%
}
\newcommand*{\chapappifchapterprefix}[1]{%
  \IfChapterUsesPrefixLine{\chapapp#1}{}%
}
\newcommand*{\IfChapterUsesPrefixLine}{%
  \if@chapterprefix\expandafter\@firstoftwo\else\expandafter\@secondoftwo\fi
}
\newcommand*{\chapapp}{\@chapapp}
%    \end{macrocode}
% \begin{macro}{\@chapapp}
% Dieses Makro enthält den Kapitelname (\cs{chaptername}), der sich im
% Anhang üblicherweise ändert (\cs{appendixname}).
%    \begin{macrocode}
\newcommand*\@chapapp{\chaptername}
%    \end{macrocode}
% \end{macro}%^^A \@chapapp
% \end{macro}%^^A \chapapp
% \end{macro}%^^A \IfChapterUsePrefixLine
% \end{macro}%^^A \chapappifchapterprefix
% \end{macro}%^^A \chapappifprefix
%
%
% \begin{macro}{\chaptermark}
% Da \cs{chapter} nicht mit Hilfe von \cs{@startsection} definiert
% wird, wird auch \cs{chaptermark} nicht automatisch definiert. Es
% wird aber in den Seitenstilen benötigt bzw. umdefiniert. Also hier
% vorsorglich eine Dummy-Definition:
%    \begin{macrocode}
\newcommand*\chaptermark[1]{}
%    \end{macrocode}
% \end{macro}
%
% \begin{macro}{\addchapmark}
% \changes{v3.12}{2013/03/26}{neue Anweisung}%^^A
% Für \cs{addchapmark} gilt dies hingegen nicht. Das wird nicht von den Stilen
% definiert, sondern direkt mit der gewünschten Definition
% initialisiert. Bitte nicht davon verwirren lassen, dass \cs{if@mainmatter}
% hier auch von \textsf{scrreprt} definiert wird. Das ist Absicht und
% vereinfacht dem einen oder anderen Paket oder Benutzer vielleicht die
% Arbeit.
%    \begin{macrocode}
\newcommand*\addchapmark[1]{%
  \begingroup
    \expandafter\let\csname if@mainmatter\expandafter\endcsname
    \csname iffalse\endcsname
    \c@secnumdepth=\numexpr \chapternumdepth-1\relax
    \chaptermark{#1}%
  \endgroup
}
%    \end{macrocode}
% \end{macro}%^^A \addchapmark
%
%
% \begin{macro}{\raggedchapter}
% \changes{v3.15}{2014/12/09}{neu}%^^A
% Da das doch recht häufig gewünscht wird, kann man die Ausrichtung der
% Kapitelüberschrift auch getrennt von der Ausrichtung der übrigen
% Überschriften einstellen.
%    \begin{macrocode}
\newcommand*{\raggedchapter}{\raggedsection}
%    \end{macrocode}
% \end{macro}
%
%
% \begin{KOMAfont}{chapter}
% \changes{v2.8o}{2001/09/14}{neues Element \texttt{chapter}}%^^A
% \begin{macro}{\scr@fnt@chapter}
% \changes{v2.8o}{2001/09/14}{neues Element \texttt{chapter}}%^^A
% \begin{KOMAfont}{chapterprefix}
% \changes{v2.96a}{2006/12/02}{neues Element \texttt{chapterprefix}}%^^A
% \begin{macro}{\scr@fnt@chapterprefix}
% \changes{v2.96a}{2006/12/02}{neues Element \texttt{chapterprefix}}%^^A
%    \begin{macrocode}
\newcommand*{\scr@fnt@chapter}{\size@chapter}
\newcommand*{\scr@fnt@chapterprefix}{\size@chapterprefix}
%    \end{macrocode}
% \end{macro}%^^A \scr@fnt@chapterprefix
% \end{KOMAfont} chapterprefix
% \end{macro}%^^A \scr@fnt@chapter
% \end{KOMAfont}%^^A chapter
%
% \begin{KOMAfont}{chapterentry}
% \changes{v2.97c}{2007/06/21}{neues Font-Element}%^^A
% \changes{v3.06}{2010/06/09}{Verwendung von \texttt{sectioning} durch
%     \texttt{disposition} ersetzt}%^^A
% Schrift für den \cs{chapter}-Eintrag im Inhaltsverzeichnis.
%    \begin{macrocode}
\newkomafont{chapterentry}{\usekomafont{disposition}}
%    \end{macrocode}
% \end{KOMAfont}
%
% \begin{KOMAfont}{chapterentrypagenumber}
% \changes{v2.97c}{2007/06/21}{neues Font-Element}%^^A
% Schrift für die Seitenzahl des \cs{chapter}-Eintrags im Inhaltsverzeichnis
% abweichend von \texttt{chapterentry}.
%    \begin{macrocode}
\newkomafont{chapterentrypagenumber}{}
%    \end{macrocode}
% \end{KOMAfont}
%
% \begin{KOMAfont}{chapterentrydots}
% \changes{v3.15}{2014/12/10}{neues Font-Element}%^^A
% \changes{v3.27}{2019/10/28}{überflüssiges \cs{normalfont} entfernt}%^^A
% Schrift für die optionalen Pünktchen des \cs{chapter}-Eintrags im
% Inhaltsverzeichnis abweichend von \texttt{chapterentry} und
% \cs{normalfont}\cs{normalsize}.
%    \begin{macrocode}
\newkomafont{chapterentrydots}{}
%</body>
%</book|report>
%</class>
%    \end{macrocode}
% \end{KOMAfont}
%
%
% \subsection{Abschnitte und Untergliederungen}
%
% Alle Ebenen ab \cs{section} ähneln sich so sehr, dass sie gemeinsam
% behandelt werden. Lediglich bei \textsf{scrartl} gibt es für \cs{section}
% selbst noch ein paar Besonderheiten.
%
% \begin{macro}{\size@section}
% \changes{v2.8o}{2001/09/14}{neu (intern)}%^^A
% \begin{macro}{\size@subsection}
% \changes{v2.8o}{2001/09/14}{neu (intern)}%^^A
% \begin{macro}{\size@subsubsection}
% \changes{v2.8o}{2001/09/14}{neu (intern)}%^^A
% \begin{macro}{\size@paragraph}
% \changes{v2.8o}{2001/09/14}{neu (intern)}%^^A
% \begin{macro}{\size@subparagraph}
% \changes{v2.8o}{2001/09/14}{neu (intern)}%^^A
% Hier werden diese Befehle nur vordefiniert. Ihre tatsächliche Einstellung
% erfolgt über die (Vor-)Auswahl von Option \texttt{headings}.
%    \begin{macrocode}
%<*class>
%<*prepare>
\newcommand*{\size@section}{}
\newcommand*{\size@subsection}{}
\newcommand*{\size@subsubsection}{}
\newcommand*{\size@paragraph}{}
\newcommand*{\size@subparagraph}{}
%</prepare>
%    \end{macrocode}
% \end{macro}%^^A \size@subparagraph
% \end{macro}%^^A \size@paragraph
% \end{macro}%^^A \size@subsubsection
% \end{macro}%^^A \size@subsection
% \end{macro}%^^A \size@section
%
%
% \begin{macro}{\l@section}
% \changes{v2.9k}{2003/01/02}{Anpassung an Option \texttt{tocleft}}%^^A
% \changes{v2.96b}{2006/11/30}{Umbruch zwischen \cs{section}-Eintrag im
%     Inhaltsverzeichnis und übergeordneten Einträgen verhindern}%^^A
% \changes{v2.96b}{2006/11/30}{Umbruch zwischen \cs{section}-Eintrag im
%     Inhaltsverzeichnis und untergeordneten Einträgen verhindern}%^^A
% \changes{v2.97c}{2007/06/21}{\cs{sectfont} durch Verwendung von
%     Element \texttt{sectionentry} ersetzt}%^^A
% \changes{v2.97c}{2007/06/21}{Element \texttt{sectionentrypagenumber} wird
%     verwendet}%^^A
% \changes{v2.97c}{2007/06/21}{weitgehend umgeschrieben, um linksbündige
%     Einträge zu ermöglichen}%^^A
% \changes{v3.15}{2014/11/24}{\cs{sectiontocdepth},
%     \cs{scr@section@tocindent} und \cs{scr@section@tocnumwidth} werden
%     verwendet}%^^A
% \changes{v3.20}{2015/10/06}{Überführung in Verzeichnisstil}%^^A
% Bei der Artikel-Klasse folgt die Formatierung für
% \cs{section}-Einträge, an Stelle der Einträge für \cs{chapter}
% der Buch- oder der Bericht-Klasse. Bei diesen handelt es sich hier
% stattdessen um eine normale Ebene.
% \begin{option}{sectionentrydots}
% \changes{v3.15}{2014/12/10}{Neu}%^^A
% Seit Version~3.15 kann man mit dieser Option Pünktchen zwischen Text und
% Seitenzahl in den Kapiteleinträgen einschalten.
%    \begin{macrocode}
%<*article>
%<*option>
\KOMA@ifkey{sectionentrydots}{@sectionentrywithdots}
%</option>
%    \end{macrocode}
% \end{option}%^^A sectionentrydots
% \begin{macro}{\raggedsectionentry}
% \changes{v2.97c}{2007/06/21}{neue Anweisung}%^^A
% \changes{v3.20}{2016/02/23}{nicht mehr verwendet}%^^A
%   Ausrichtung des Textes der Abschnittseinträge im Inhaltsverzeichnis.
%    \begin{macrocode}
%<*body>
\newcommand*{\raggedsectionentry}{}
%</body>
%</article>
%    \end{macrocode}   
% \end{macro}%^^A \raggedsectionentry
% \end{macro}%^^A \l@section
%
%
% \begin{macro}{\addsec}
% \changes{v2.0e}{1994/10/12}{mehrfach \cs{markboth} durch
%     \cs{sectionmark} ersetzt}%^^A
% \changes{v2.2b}{1995/03/20}{\cs{sectionmark} durch \cs{@mkboth}
%     ersetzt}%^^A
% \changes{v2.3h}{1995/01/21}{jetzt auch bei \textsf{scrbook} und
%     \textsf{scrreprt}}%^^A
% \changes{v2.4g}{1996/11/04}{\cs{section*} vorgezogen, damit
%     dadurch provozierte Seitenumbrüche im Inhaltsverzeichnis
%     berücksichtigt werden}%^^A
% \changes{v2.5d}{1998/01/03}{\cs{@mkboth} bedingt durch
%     \cs{markright} ersetzt}%^^A
% \changes{v2.7a}{2001/01/04}{\cs{addsec} für eine bessere
%     Unterstützung von \textsf{hyperref} geändert}%^^A
% \changes{v3.13a}{2014/09/11}{Verwendung von \cs{SecDef}}%^^A
% \begin{macro}{\@addsec}
% \changes{v3.10}{2011/08/30}{Verwendung von \cs{addsectiontocentry}}%^^A
% \changes{v3.10}{2011/08/30}{Integration der Erweiterung für das optionale
%     Argument der Gliederungsbefehle}%^^A
% \changes{v3.12}{2013/02/26}{Verwendung von \cs{addsecmark}}%^^A
% \changes{v3.18}{2015/05/22}{komplett neu definiert}%^^A
% \begin{macro}{\@saddsec}
% \changes{v3.12}{2013/02/26}{Verwendung von \cs{addsecmark}}%^^A
% \changes{v3.17}{2015/04/20}{\cs{scr@ds@head} in \cs{addsecmark}
%     expandieren}%^^A
% \KOMAScript{} bietet in allen drei Hauptklassen den zusätzlichen
% Gliederungsbefehl \cs{addsec}. Es handelt sich dabei um einen mit
% \cs{section*} vergleichbaren Befehl, bei dem jedoch der
% Kolumnentitel korrigiert wird und ein Eintrag ins Inhaltsverzeichnis
% erfolgt (nicht in der Stern-Variante).
%
% Mit Version~3.18 wurde \cs{addsec} deutlich vereinfacht. Seither verwendet
% nur noch \cs{@saddsec} die Anweiung \cs{addsecmark}, während \cs{@addsec}
% direkt über das \cs{sectionmark} von \cs{section} arbeitet.
%    \begin{macrocode}
%<*body>
\newcommand\addsec{\SecDef\@addsec\@saddsec}
\newcommand*{\@addsec}{}
\def\@addsec[#1]#2{%
  \edef\reserved@a{%
    \unexpanded{%
      \section[{#1}]{#2}%
      \c@secnumdepth=
    }\the\c@secnumdepth\relax
  }%
  \c@secnumdepth=\numexpr \sectionnumdepth-1\relax
  \reserved@a
}
\newcommand*{\@saddsec}[1]{%
  \section*{#1}\addsecmark{}%
}
%    \end{macrocode}
% \end{macro}%^^A \@saddsec
% \end{macro}%^^A \@addsec
% \end{macro}%^^A \addsec
%
%
% \begin{macro}{\minisec}
% \changes{v2.8q}{2002/02/28}{\cs{nobreak} nach \cs{sectfont}
%   behebt einen Bug im color Paket}%^^A
% \changes{v2.9o}{2003/01/31}{\cs{nobreak} nach dem Gruppenende
%   behebt einen Bug im color Paket}%^^A
% \changes{v2.96a}{2006/11/30}{Standardwert für \cs{parfillskip} (abhängig
%   von Option \texttt{version})}%^^A
% \changes{v2.96a}{2006/12/03}{alle nicht benötigten \cs{nobreak}%^^A
%   entfernt}%^^A
% \changes{v3.13a}{2014/08/07}{Absicherung der Abstände bei
%   aufeinanderfolgenden Überschriften entsprechend \cs{@startsection}}%^^A
% \changes{v3.26}{2018/05/14}{fehlendes \cs{interlinepenalty}\cs{@M}
%   ergänzt}%^^A
% \begin{KOMAfont}{minisec}
% \changes{v2.96a}{2006/12/03}{\texttt{minisec} ist ein eigenes
%     Fontelement}%^^A
% \changes{v3.21}{2016/06/12}{fehlendes \cs{nobreak} ergänzt}%^^A
% In \KOMAScript{} gibt es diese zusätzliche Gliederungsebene, die
% immer ohne Nummer und ohne Eintrag ins Inhaltsverzeichnis erfolgt.
%    \begin{macrocode}
\newkomafont{minisec}{}
\newcommand\minisec[1]{%
  \expandafter\ifnum\scr@v@is@lt{3.13a}\relax
    \@afterindentfalse \vskip 1.5ex
  \else
    \if@noskipsec \leavevmode \fi
    \par
    \@afterindentfalse
    \if@nobreak
      \everypar{}%
    \else
      \addpenalty\@secpenalty\addvspace{1.5ex}%
    \fi
  \fi
  {\parindent \z@
    \expandafter\ifnum\scr@v@is@gt{2.96}\relax
      \setlength{\parfillskip}{\z@ plus 1fil}\fi
    \raggedsection\normalfont\sectfont\nobreak
    \usekomafont{minisec}{\nobreak\interlinepenalty \@M #1\par\nobreak}%
  }\nobreak
  \@afterheading
}
%    \end{macrocode}
% \end{KOMAfont}
% \end{macro}
%
%
% \begin{macro}{\addsectiontocentry}
% \changes{v3.10}{2011/08/30}{Neu}%^^A
% Entsprechend \cs{addparttocentry} und \cs{addchaptertocentry}.
%    \begin{macrocode}
\newcommand*{\addsectiontocentry}[2]{%
  \addtocentrydefault{section}{#1}{#2}%
}
%    \end{macrocode}
% \end{macro}
%
% \begin{macro}{\addsubsectiontocentry}
% \changes{v3.10}{2011/08/30}{Neu}%^^A
% Entsprechend \cs{addparttocentry} usw.
%    \begin{macrocode}
\newcommand*{\addsubsectiontocentry}[2]{%
  \addtocentrydefault{subsection}{#1}{#2}%
}
%    \end{macrocode}
% \end{macro}
%
% \begin{macro}{\addparagraphtocentry}
% \changes{v3.10}{2011/08/30}{Neu}%^^A
% Entsprechend \cs{addparttocentry} usw.
%    \begin{macrocode}
\newcommand*{\addparagraphtocentry}[2]{%
  \addtocentrydefault{paragraph}{#1}{#2}%
}
%    \end{macrocode}
% \end{macro}
%
% \begin{macro}{\addsubparagraphtocentry}
% \changes{v3.10}{2011/08/30}{Neu}%^^A
% Entsprechend \cs{addparttocentry} usw.
%    \begin{macrocode}
\newcommand*{\addsubparagraphtocentry}[2]{%
  \addtocentrydefault{subparagraph}{#1}{#2}%
}
%    \end{macrocode}
% \end{macro}
%
%
% \begin{Counter}{section}
% \begin{macro}{\thesection}
% \changes{v2.97e}{2007/11/23}{Kapitelnummer nur im Hauptteil}%^^A
% \changes{v3.03b}{2009/06/09}{\cs{relax} im Kompatibilitätstest durch ein
%     Leerzeichen ersetzt, obwohl das die Wartbarkeit verschlechtert, aber
%     \textsf{hyperref} hat mit dem \cs{relax} ein nicht dokumentieres
%     Problem}%^^A
% \changes{v3.27}{2019/06/26}{Umstellung auf \cs{scr@v@is@gt}}%^^A
% \begin{macro}{\sectionmarkformat}
% \changes{v2.3a}{1995/07/08}{Leerraum nach der Kapitelnummer
%     erhöht}%^^A
% \changes{v2.3c}{1995/08/06}{Duden Regel 6}%^^A
% \begin{Counter}{subsection}
% \begin{macro}{\thesubsection}
% \begin{macro}{\subsectionmarkformat}
% \changes{v2.3a}{1995/07/08}{Leerraum nach der Kapitelnummer
%     erhöht}%^^A
% \changes{v2.3c}{1995/08/06}{Duden Regel 6}%^^A
% \begin{Counter}{subsubsection}
% \begin{macro}{\thesubsubsection}
% \begin{Counter}{paragraph}
% \begin{macro}{\theparagraph}
% \begin{Counter}{subparagraph}
% \begin{macro}{\thesubparagraph}
% Jede Gliederungsebene benötigt einen Zähler für die
% Gliederungsnummer und eine Darstellung des Zählers
% (\cs{the\dots}). Desweiteren wird eine Formatierung des Zählers in
% den Gliederungsüberschriften (\texttt{\bslash\dots format}) und eine
% Formatierung des Zählers in der Kopfzeile 
% (\texttt{\bslash\dots markformat}) benötigt.
%    \begin{macrocode}
%<*book|report>
\newcounter{section}[chapter]
\renewcommand*\thesection{%
%<*book>
  \expandafter\ifnum\scr@v@is@gt{2.97d}% 
    \if@mainmatter\thechapter.\fi
  \else
%</book>
  \thechapter.%
%<book>  \fi
  \@arabic\c@section
}
%</book|report>
%<*article>
\newcounter{section}
\renewcommand*{\thesection}{\@arabic\c@section}
%</article>
\newcommand*\sectionmarkformat{\thesection\autodot\enskip}
\newcounter{subsection}[section]
\renewcommand*{\thesubsection}{\thesection.\@arabic\c@subsection}
%<*article>
\newcommand*\subsectionmarkformat{\thesubsection\autodot\enskip}
%</article>
\newcounter{subsubsection}[subsection]
\renewcommand*{\thesubsubsection}{%
  \thesubsection.\@arabic\c@subsubsection
}
\newcounter{paragraph}[subsubsection]
\renewcommand*{\theparagraph}{\thesubsubsection.\@arabic\c@paragraph}
\newcounter{subparagraph}[paragraph]
\renewcommand*{\thesubparagraph}{%
  \theparagraph.\@arabic\c@subparagraph
}
%    \end{macrocode}
% \begin{macro}{\@seccntformat}
% \changes{v2.3c}{1995/08/06}{CJK erweitert}%^^A
% \changes{v2.9p}{2004/01/07}{\cs{protect} eingefügt}%^^A
% \changes{v2.97}{2007/01/24}{Workaround für das blöde \cs{protect}}%^^A
% \changes{v3.17}{2015/02/23}{verwendet \cs{\dots format}, wenn es
%     existiert}%^^A
% \changes{v3.17}{2015/03/31}{verwendet \cs{protect} für \cs{\dots
%     format}}%^^A
% \begin{macro}{\othersectionlevelsformat}
% \changes{v2.7}{2000/01/03}{neu}%^^A
% \changes{v2.7}{2000/01/03}{\cs{quad} durch \cs{enskip} ersetzt}%^^A
% \changes{v2.7i}{2001/05/17}{\cs{let} durch \cs{def} ersetzt}%^^A
% \changes{v2.97}{2007/01/24}{statt einem Argument jetzt drei}%^^A
% Bei den Ebenen ab \cs{section} erfolgt die Formatierung der
% Gliederungnummer in der Überschrift mit einer gemeinsamen
% Anweisung. Die interne Anweisung \cs{@seccntformat} wird dabei von
% \LaTeX-Kern verwendet und muss umdefiniert werden, um eine
% entsprechende Anweisung auf Anwenderebene verfügbar zu haben.
% Ab Version 2.9p wurde in \cs{@seccntformat} ein \cs{protect} vor
% \cs{othersectionlevelsformat} eingefügt. Damit wurde es für den Benutzer
% deutlich einfacher, die Anweisung \cs{othersectionlevelsformat}
% umzudefinieren. Dafür war dadurch aber \cs{the\dots} nicht mehr
% expandiert. Das wiederum gab Probleme, wenn \cs{@seccntformat} für den
% \cs{mark}-Mechanismus genutzt wurde. Deshalb hat
% \cs{othersectionlevelsformat} ab Version 2.97 drei Argumente statt nur
% eines. Das zweite Argument sollte vom Anwender ignoriert werden, es
% existiert nur, um die Kompatibitlität Umdefinitionen zu erhalten, bei denen
% \cs{othersectionlevelsformat} nur ein Argument besitzt. Das dritte Argument
% ist die \cs{the\dots}-Anweisung. Das zweite Argument funktioniert so, das es
% das dritte Argument frisst, falls \cs{othersectionlevelsformat} nur ein
% Argument liest.
%    \begin{macrocode}
\newcommand*{\othersectionlevelsformat}[3]{#3\autodot\enskip}
\renewcommand*{\@seccntformat}[1]{%
  \expandafter\ifnum\scr@v@is@lt{3.17}\relax
    \protect\othersectionlevelsformat{#1}{%
      \expandafter\aftergroup\noexpand\@gobble}{\csname the#1\endcsname}%
  \else
    \scr@ifundefinedorrelax{#1format}{%
      \protect\othersectionlevelsformat{#1}{%
        \expandafter\aftergroup\noexpand\@gobble}{\csname the#1\endcsname}%
    }{\expandafter\protect\csname #1format\endcsname}%
  \fi
}
%    \end{macrocode}
% \end{macro}%^^A \othersectonlevelsformat
% \end{macro}%^^A \@seccntformat
% \end{macro}%^^A \thesubparagraph
% \end{Counter}%^^A subparagraph
% \end{macro}%^^A \theparagraph
% \end{Counter}%^^A paragraph
% \end{macro}%^^A \thesubsubsection
% \end{Counter}%^^A subsubsection
% \end{macro}%^^A \subsectionformat
% \end{macro}%^^A \thesubsection
% \end{Counter}%^^A subsection
% \end{macro}%^^A \sectionformat
% \end{macro}%^^A \thesection
% \end{Counter}%^^A section
%
%
% \begin{macro}{\addsecmark}
% \changes{v3.12}{2013/03/26}{neue Anweisung}%^^A
% \cs{addsecmark} wird nicht von den Seitenstilen definiert, sondern direkt
% mit der gewünschten Definition initialisiert.
%    \begin{macrocode}
\newcommand*\addsecmark[1]{%
  \begingroup
    \c@secnumdepth=\numexpr \sectionnumdepth-1\relax
    \sectionmark{#1}%
  \endgroup
}
%    \end{macrocode}
% \end{macro}%^^A \addsecmark
%
% \begin{KOMAfont}{section}
% \changes{v2.8o}{2001/09/14}{neues Element \texttt{section}}%^^A
% \begin{macro}{\scr@fnt@section}
% \changes{v2.8o}{2001/09/14}{neues Element \texttt{section}}%^^A
% \begin{KOMAfont}{subsection}
% \changes{v2.8o}{2001/09/14}{neues Element \texttt{subsection}}%^^A
% \begin{macro}{\scr@fnt@subsection}
% \changes{v2.8o}{2001/09/14}{neues Element \texttt{subsection}}%^^A
% \begin{KOMAfont}{subsubsection}
% \changes{v2.8o}{2001/09/14}{neues Element \texttt{subsubsection}}%^^A
% \begin{macro}{\scr@fnt@subsubsection}
% \changes{v2.8o}{2001/09/14}{neues Element \texttt{subsubsection}}%^^A
% \begin{KOMAfont}{paragraph}
% \changes{v2.8o}{2001/09/14}{neues Element \texttt{paragraph}}%^^A
% \begin{macro}{\scr@fnt@paragraph}
% \changes{v2.8o}{2001/09/14}{neues Element \texttt{paragraph}}%^^A
% \begin{KOMAfont}{subparagraph}
% \changes{v2.8o}{2001/09/14}{neues Element \texttt{subparagraph}}%^^A
% \begin{macro}{\scr@fnt@subparagraph}
% \changes{v2.8o}{2001/09/14}{neues Element \texttt{subparagraph}}%^^A
% \begin{KOMAfont}{minisec}
% \changes{v2.8o}{2001/09/14}{neuer Ersatz für das Element
%     \texttt{minisec}}%^^A
% \changes{v2.96a}{2007/01/08}{Alias für das Element \texttt{minisec}%^^A
%     entfernt, da es nun ein eigenes Element ist}%^^A
% Definition der Elemente der Gliederung, deren Schrift geändert
% werden kann.
%    \begin{macrocode}
\newcommand*{\scr@fnt@section}{\size@section}
\newcommand*{\scr@fnt@subsection}{\size@subsection}
\newcommand*{\scr@fnt@subsubsection}{\size@subsubsection}
\newcommand*{\scr@fnt@paragraph}{\size@paragraph}
\newcommand*{\scr@fnt@subparagraph}{\size@subparagraph}
%    \end{macrocode}
% \end{KOMAfont}%^^A minisec
% \end{macro}
% \end{KOMAfont}
% \end{macro}
% \end{KOMAfont}
% \end{macro}
% \end{KOMAfont}
% \end{macro}
% \end{KOMAfont}
% \end{macro}
% \end{KOMAfont}
%
% \begin{KOMAfont}{sectionentry}
% \changes{v2.97c}{2007/06/21}{neues Font-Element}%^^A
% \changes{v3.06}{2010/06/09}{Verwendung von \texttt{sectioning} durch
%     \texttt{disposition} ersetzt}%^^A
% Schrift für den \cs{section}-Eintrag im Inhaltsverzeichnis.
%    \begin{macrocode}
%<*article>
\newkomafont{sectionentry}{\usekomafont{disposition}}
%    \end{macrocode}
% \end{KOMAfont}
%
% \begin{KOMAfont}{sectionentrypagenumber}
% \changes{v2.97c}{2007/06/21}{neues Font-Element}%^^A
% Schrift für die Seitenzahl des \cs{section}-Eintrags im Inhaltsverzeichnis
% abweichend von \texttt{sectionentry}.
%    \begin{macrocode}
\newkomafont{sectionentrypagenumber}{}
%    \end{macrocode}
% \end{KOMAfont}
%
% \begin{KOMAfont}{sectionentrydots}
% \changes{v3.15}{2014/12/10}{neues Font-Element}%^^A
% \changes{v3.27}{2019/10/28}{überflüssiges \cs{normalfont} entfernt}%^^A
% Schrift für die optionalen Pünktchen des \cs{section}-Eintrags im
% Inhaltsverzeichnis abweichend von \texttt{sectionentry} und
% \cs{normalfont}\cs{normalsize}.
%    \begin{macrocode}
\newkomafont{sectionentrydots}{}
%</article>
%    \end{macrocode}
% \end{KOMAfont}
%
%
% \begin{macro}{\partnumdepth}
% \changes{v3.12}{2013/12/16}{new read only command}%^^A
% \changes{v3.18}{2015/05/24}{indirekt über \cs{DeclareSectionCommand}}%^^A
% \begin{macro}{\chapternumdepth}
% \changes{v3.12}{2013/12/16}{new read only command}%^^A
% \changes{v3.18}{2015/05/20}{indirekt über \cs{DeclareSectionCommand}}%^^A
% \begin{macro}{\sectionnumdepth}
% \changes{v3.12}{2013/12/16}{new read only command}%^^A
% \changes{v3.15}{2014/12/09}{indirekt über \cs{DeclareSectionCommand}}%^^A
% \begin{macro}{\subsectionnumdepth}
% \changes{v3.12}{2013/12/16}{new read only command}%^^A
% \changes{v3.15}{2014/12/09}{indirekt über \cs{DeclareSectionCommand}}%^^A
% \begin{macro}{\subsubsectionnumdepth}
% \changes{v3.12}{2013/12/16}{new read only command}%^^A
% \changes{v3.15}{2014/12/09}{indirekt über \cs{DeclareSectionCommand}}%^^A
% \begin{macro}{\paragraphnumdepth}
% \changes{v3.12}{2013/12/16}{new read only command}%^^A
% \changes{v3.15}{2014/12/09}{indirekt über \cs{DeclareSectionCommand}}%^^A
% \begin{macro}{\subparagraphnumdepth}
% \changes{v3.12}{2013/12/16}{new read only command}%^^A
% \changes{v3.15}{2014/12/09}{indirekt über \cs{DeclareSectionCommand}}%^^A
% Diese dürfen nur gelesen, sollten aber nie umdefiniert werden und geben die
% Nummerierungsebene der Gliederungsanweisungen an (nicht zu verwechseln mit
% der Verzeichnisebene der Gliederungsanweisungen).
% \begin{macro}{\parttocdepth}
% \changes{v3.15}{2014/12/22}{neue Anweisung}%^^A
% \changes{v3.18}{2015/05/25}{indirekt über \cs{DeclareSectionCommand}}%^^A
% \begin{macro}{\chaptertocdepth}
% \changes{v3.15}{2014/12/22}{neue Anweisung}%^^A
% \changes{v3.18}{2015/05/20}{indirekt über \cs{DeclareSectionCommand}}%^^A
% \begin{macro}{\sectiontocdepth}
% \changes{v3.15}{2014/12/09}{indirekt über \cs{DeclareSectionCommand}}%^^A
% \begin{macro}{\subsectiontocdepth}
% \changes{v3.15}{2014/12/09}{indirekt über \cs{DeclareSectionCommand}}%^^A
% \begin{macro}{\subsubsectiontocdepth}
% \changes{v3.15}{2014/12/09}{indirekt über \cs{DeclareSectionCommand}}%^^A
% \begin{macro}{\paragraphtocdepth}
% \changes{v3.15}{2014/12/09}{indirekt über \cs{DeclareSectionCommand}}%^^A
% \begin{macro}{\subparagraphtocdepth}
% \changes{v3.15}{2014/12/09}{indirekt über \cs{DeclareSectionCommand}}%^^A
% Diese dürfen nur gelesen, sollten aber nie umdefiniert werden und geben die
% Verzeichnisebene der Gliederungsanweisungen an.
% \begin{macro}{\part}
% \changes{v3.18}{2015/05/25}{Verwendung von \cs{DeclareSectionCommand}}%^^A
% \changes{v3.26}{2018/06/27}{Definition nach \cs{section} verschoben (wegen
%   bookmark level fix)}%^^A
% \begin{macro}{\chapter}
% \changes{v2.8d}{2001/07/05}{\cs{chapterpagestyle} statt
%   \texttt{plain}}%^^A
% \changes{v3.18}{2015/05/20}{Verwendung von \cs{DeclareSectionCommand}}%^^A
% Als Besonderheit werden für \cs{chapter} nur diejenigen Werte gesetzt, die
% nicht über Optionen eingestellt werden können und daher vorab deklariert
% werden  müssen. Bei den Anweisungen der unteren Ebenen werden hingegen alle
% Einstellungen über \cs{DeclareSectionCommand} vorgenommen.
% \begin{macro}{\section}
% \changes{v2.8p}{2001/09/22}{\cs{sectfont} wird nun vor
%   \cs{size@section} aufgerufen}%^^A
% \changes{v2.8q}{2002/02/28}{\cs{nobreak} nach \cs{sectfont}
%   behebt einen Bug im color Paket}%^^A
% \changes{v2.96a}{2006/11/30}{Standardwert für \cs{parfillskip} (abhängig
%   von Option \texttt{version})}%^^A
% \changes{v2.96a}{2006/12/03}{alle nicht benötigten \cs{nobreak} entfernt}%^^A
% \changes{v3.13a}{2014/09/11}{Verwendung von \cs{scr@startsection}}%^^A
% \changes{v3.15}{2014/11/24}{Verwendung von \cs{DeclareSectionCommand}}%^^A
% \begin{macro}{\subsection}
% \changes{v2.8p}{2001/09/22}{\cs{sectfont} wird nun vor
%   \cs{size@subsection} aufgerufen}%^^A
% \changes{v2.8q}{2002/02/28}{\cs{nobreak} nach \cs{sectfont}
%   behebt einen Bug im color Paket}%^^A
% \changes{v2.96a}{2006/11/30}{Standardwert für \cs{parfillskip} (abhängig
%   von Option \texttt{version})}%^^A
% \changes{v2.96a}{2006/12/03}{alle nicht benötigten \cs{nobreak} entfernt}%^^A
% \changes{v3.13a}{2014/09/11}{Verwendung von \cs{scr@startsection}}%^^A
% \changes{v3.15}{2014/11/24}{Verwendung von \cs{DeclareSectionCommand}}%^^A
% \begin{macro}{\subsubsection}
% \changes{v2.3d}{1995/08/19}{wird bei \texttt{scrartcl} nun
%   ebenfalls numeriert und ins Inhaltsverzeichnis geschrieben}%^^A
% \changes{v2.8p}{2001/09/22}{\cs{sectfont} wird nun vor
%   \cs{size@subsubsection} aufgerufen}%^^A
% \changes{v2.8q}{2002/02/28}{\cs{nobreak} nach \cs{sectfont}
%   behebt einen Bug im color Paket}%^^A
% \changes{v2.96a}{2006/11/30}{Standardwert für \cs{parfillskip} (abhängig
%   von Option \texttt{version})}%^^A
% \changes{v2.96a}{2006/12/03}{alle nicht benötigten \cs{nobreak} entfernt}%^^A
% \changes{v3.13a}{2014/09/11}{Verwendung von \cs{scr@startsection}}%^^A
% \changes{v3.15}{2014/11/24}{Verwendung von \cs{DeclareSectionCommand}}%^^A
% \begin{macro}{\paragraph}
% \changes{v2.8p}{2001/09/22}{\cs{sectfont} wird nun vor
%   \cs{size@paragraph} aufgerufen}%^^A
% \changes{v2.8q}{2002/02/28}{\cs{nobreak} nach \cs{sectfont}
%   behebt einen Bug im color Paket}%^^A
% \changes{v2.96a}{2006/12/03}{alle nicht benötigten \cs{nobreak} entfernt}%^^A
% \changes{v3.15}{2014/11/24}{Verwendung von \cs{DeclareSectionCommand}}%^^A
% \begin{macro}{\subparagraph}
% \changes{v2.8p}{2001/09/22}{\cs{sectfont} wird nun vor
%   \cs{size@subparagraph} aufgerufen}%^^A
% \changes{v2.8q}{2002/02/28}{\cs{nobreak} nach \cs{sectfont}
%   behebt einen Bug im color Paket}%^^A
% \changes{v2.96a}{2006/12/03}{alle nicht benötigten \cs{nobreak} entfernt}%^^A
% \changes{v3.13a}{2014/09/11}{Verwendung von \cs{scr@startsection}}%^^A
% \changes{v3.15}{2014/11/24}{Verwendung von \cs{DeclareSectionCommand}}%^^A
% \changes{v3.15}{2014/11/24}{\cs{parindent} durch \cs{scr@parindent}
%   ersetzt}%^^A
% Die Standardgliederungsbefehle \cs{section} bis \cs{subparagraph}
% sind seit \KOMAScript{} v3.15 über die von \KOMAScript{} bereitgestellte
% Schnittstelle \cs{DeclareSectionCommand} definiert.
%    \begin{macrocode}
%<*book|report>
\DeclareSectionCommand[%
  style=chapter,%
  level=\z@,%
  pagestyle=plain,%
  tocstyle=chapter,%
  tocindent=\z@,%
  tocnumwidth=1.5em%
]{chapter}
%</book|report>
\DeclareSectionCommand[%
  style=section,%
  level=1,%
  indent=\z@,%
  beforeskip=-3.5ex \@plus -1ex \@minus -.2ex,%
  afterskip=2.3ex \@plus.2ex,%
  tocstyle=section,%
%<article>  tocindent=0pt,%
%<book|report>  tocindent=1.5em,%
%<article>  tocnumwidth=1.5em%
%<book|report>  tocnumwidth=2.3em%
]{section}
\DeclareSectionCommand[%
  style=part,%
%<article>  level=\z@,%
%<report|book>  level=\m@ne,%
%<report|book>  pagestyle=plain,%
  tocstyle=part,%
  toclevel=\m@ne,%
  tocindent=\z@,%
  tocnumwidth=2em%
]{part}
\DeclareSectionCommand[%
  style=section,%
  level=2,%
  indent=\z@,%
  beforeskip=-3.25ex\@plus -1ex \@minus -.2ex,%
  afterskip=1.5ex \@plus .2ex,%
%<article>  tocstyle=subsection,%
%<book|report>  tocstyle=section,%
%<article>  tocindent=1.5em,%
%<book|report>  tocindent=3.8em,%
%<article>  tocnumwidth=2.3em%
%<book|report>  tocnumwidth=3.2em%
]{subsection}
\DeclareSectionCommand[%
  style=section,%
  level=3,%
  indent=\z@,%
  beforeskip=-3.25ex\@plus -1ex \@minus -.2ex,%
  afterskip=1.5ex \@plus .2ex,%
%<article>  tocstyle=subsection,%
%<book|report>  tocstyle=section,%
%<article>  tocindent=3.8em,%
%<book|report>  tocindent=7.0em,%
%<article>  tocnumwidth=3.2em%
%<book|report>  tocnumwidth=4.1em%
]{subsubsection}
\DeclareSectionCommand[%
  style=section,%
  level=4,%
  indent=\z@,%
  beforeskip=3.25ex \@plus1ex \@minus.2ex,%
  afterskip=-1em,%
%<article>  tocstyle=subsection,%
%<book|report>  tocstyle=section,%
%<article>  tocindent=7.0em,%
%<book|report>  tocindent=10em,%
%<article>  tocnumwidth=4.1em%
%<book|report>  tocnumwidth=5em%
]{paragraph}
\DeclareSectionCommand[%
  style=section,%
  level=5,%
  indent=\scr@parindent,%
  beforeskip=3.25ex \@plus1ex \@minus .2ex,%
  afterskip=-1em,%
%<article>  tocstyle=subsection,%
%<book|report>  tocstyle=section,%
%<article>  tocindent=10em,%
%<book|report>  tocindent=12em,%
%<article>  tocnumwidth=5em%
%<book|report>  tocnumwidth=6em%
]{subparagraph}
\expandafter\ifnum\scr@v@is@lt{3.15}\relax
  \let\scr@subparagraph@sectionindent\parindent
\else
  \def\scr@subparagraph@sectionindent{\scr@parindent}%
\fi
%</body>
%</class>
%    \end{macrocode}
% \end{macro}%^^A \subparagraph
% \end{macro}%^^A \paragraph
% \end{macro}%^^A \subsubsection
% \end{macro}%^^A \subsection
% \end{macro}%^^A \section
% \end{macro}%^^A \chapter
% \end{macro}%^^A \part
% \end{macro}%^^A \subparargraphtocdepth
% \end{macro}%^^A \paragraphtocdepth
% \end{macro}%^^A \subsubsectiontocdepth
% \end{macro}%^^A \subsectiontocdepth
% \end{macro}%^^A \sectiontocdepth
% \end{macro}%^^A \chaptertocdepth
% \end{macro}%^^A \parttocdepth
% \end{macro}%^^A \subparargraphnumdepth
% \end{macro}%^^A \paragraphnumdepth
% \end{macro}%^^A \subsubsectionnumdepth
% \end{macro}%^^A \subsectionnumdepth
% \end{macro}%^^A \sectionnumdepth
% \end{macro}%^^A \chapternumdepth
% \end{macro}%^^A \partnumdepth
%
%
% \subsection{Font und Größeneinstellungen}
%
% Die Einstellungen für die einzelnen Ebenen werden bei den Definitonen der
% Ebenen vorgenommen. Hier gibt es nur Einstellungen, die nicht einer
% einzelnen Ebene zugeordnet sind.
%
% \begin{KOMAfont}{disposition}
% \changes{v2.95b}{2006/07/30}{neues Element \texttt{disposition}}%^^A
% \begin{KOMAfont}{sectioning}
% \begin{macro}{\scr@fnt@disposition}
% \changes{v2.95b}{2006/07/30}{neues Element \texttt{disposition}}%^^A
% \begin{macro}{\sectfont}
% \changes{v2.8c}{2001/06/29}{\cs{normalcolor} eingefügt}%^^A
% \changes{v3.12}{2013/11/11}{wird wegen Option
%     \texttt{headings!=standardclasses} früher definiert}%^^A
% \changes{v3.20}{2016/01/29}{\cs{sffamily} durch \cs{@gr@gsffamily}
%     ersetzt}%^^A
% Dies Schriftart nicht nur für die Gliederungsbefehle. Dieses Makro
% ist als intern zu betrachten. Der Anwender sollte stattdessen das
% entsprtechende Fontelement verwenden.
%    \begin{macrocode}
%<*prepare>
\newcommand*{\sectfont}{\normalcolor\@gr@gsffamily\bfseries}
%</prepare>
%<*body>
\newcommand*{\scr@fnt@disposition}{\sectfont}
\aliaskomafont{sectioning}{disposition}
%</body>
%    \end{macrocode}
% \end{macro}%^^A \sectfont
% \end{macro}%^^A \scr@fnt@disposition
% \end{KOMAfont}%^^A sectioning
% \end{KOMAfont}%^^A disposition
%
%
% \begin{option}{headings}
% \changes{v2.3h}{1996/01/20}{Größe von \cs{chapter} um eine Stufe
%     verringert}%^^A
% \changes{v2.3h}{1996/01/20}{Verwendung von \cs{chapterheadstartvskip} und
%     \cs{chapterheadendvskip} an Stelle von festen vertikalen Abständen am
%     Anfang und am Ende eines Kapitels}%^^A
% \changes{v2.7c}{2000/01/19}{vertikale Abstände nach der Kapitelüberschrift
%     geringfügig verändert und mit Leim versehen, um mit \cs{flushbottom} zu
%     besseren Ergebnissen zu gelangen}%^^A
% \changes{v2.98c}{2008/03/10}{Neue Option}%^^A
% \changes{v3.10}{2011/08/30}{Integration der Erweiterung für das optionale
%     Argument der Gliederungsbefehle}%^^A
% \changes{v3.12}{2013/03/05}{Verwendung der Status-Signalisierung mit
%     \cs{FamilyKeyState}}%^^A
% \changes{v3.12}{2013/08/26}{\cs{KOMA@options} durch \cs{KOMAoptions}
%     ersetzt}%^^A
% \changes{v3.12}{2013/11/11}{neuer Wert \texttt{standardclasses}}%^^A
% \changes{v3.15}{2014/11/24}{\cs{scr@chapter@beforeskip} und
%     \cs{scr@chapter@afterskip} werden verwendet}%^^A
% \changes{v3.17}{2015/03/09}{interne Speicherung des Wert}%^^A
% \changes{v3.18}{2015/05/20}{Werte für \cs{scr@chapter@beforeskip}
%     negiert}%^^A
% Dies ist eine neue zentrale Option für Überschriften. Dabei ergibt sich
% nun auch das Problem, dass die Gesamtmenge der Überschriften mal
% \emph{headings} (Überschriften), mal \emph{sectioning}
% (Abschnittseinteilung) und mal \emph{disposition} (Gliederung) genannt
% wird. Es wäre gut gewesen, sich bei Zeiten Gedanken darüber zu machen, wie
% man das am besten einheitlich nennt.
%    \begin{macrocode}
%<*class>
%<*option>
\KOMA@key{headings}{%
  \KOMA@set@ncmdkey{headings}{@tempa}{%
    {big}{0},%
    {normal}{1},%
    {small}{2},%
%<*book|report>
    {openany}{3},%
    {openright}{4},%
    {openleft}{5},%
    {twolinechapter}{6},{chapterprefix}{6},{chapterwithprefix}{6},%
    {chapterwithprefixline}{6},%
    {onelinechapter}{7},{nochapterprefix}{7},{chapterwithoutprefix}{7},%
    {chapterwithoutprefixline}{7},%
    {twolineappendix}{8},{appendixprefix}{8},{appendixwithprefix}{8},%
    {appendixwithprefixline}{8},%
    {onelineappendix}{9},{noappendixprefix}{9},{appendixwithoutprefix}{9},%
    {appendixwithoutprefixline}{9},%
%</book|report>
    {optiontotocandhead}{10},{optiontoheadandtoc}{10},%
    {optiontotoc}{11},%
    {optiontohead}{12},%
    {standardclasses}{13}%
  }{#1}%
  \ifx\FamilyKeyState\FamilyKeyStateProcessed
    \ifcase \@tempa\relax% big
      \KOMA@kav@remove{.\KOMAClassFileName}{headings}{big}%
      \KOMA@kav@remove{.\KOMAClassFileName}{headings}{normal}%
      \KOMA@kav@remove{.\KOMAClassFileName}{headings}{small}%
      \KOMA@kav@remove{.\KOMAClassFileName}{headings}{standardclasses}%
      \KOMA@kav@add{.\KOMAClassFileName}{headings}{big}%
%    \end{macrocode}
% Als erstes kümmern wir uns um die drei Größenabstufungen für die
% Überschriften, die \KOMAScript{} bietet, und zwar in der Reihenfolge
% \texttt{big}, \texttt{normal}, \texttt{small}. Es sei darauf hingewiesen,
% dass die Verwendung dieser Option alle Änderungen an
% \cs{chapterheadstartvskip} oder \cs{chapterheadendvskip} oder den
% Font-Elementen für die einzelnen Überschriften aufhebt macht!
%
% Hinweis: Es wurde |+\baselineskip+\parskip| hinzugefügt, um
% Kompatibilitätsprobleme zu vermeiden. Die entsprechenden Werte werden in der
% Präambel aber wieder abgezogen, um Überschriften wirklich mit
% |beforeskip=0pt| an den oberen Rand zu bekommen.
%    \begin{macrocode}
%<*book|report>
      \renewcommand*{\scr@chapter@beforeskip}{-3.3\baselineskip-\parskip}%
      \renewcommand*{\scr@chapter@afterskip}{%
        1.725\baselineskip \@plus .115\baselineskip \@minus .192\baselineskip
      }%
      \renewcommand*{\scr@chapter@innerskip}{.5\baselineskip}%
      \renewcommand*{\chapterheadstartvskip}{\vspace{\@tempskipa}}%
      \renewcommand*{\chapterheadendvskip}{%
        \expandafter\ifnum\scr@v@is@lt{3.15}\vspace\@tempskipa
        \else\vskip\@tempskipa\fi
      }%
      \renewcommand*{\chapterheadmidvskip}{\par\nobreak\vskip\@tempskipa}%
%</book|report>
      \renewcommand*{\size@part}{\Huge}%
      \renewcommand*{\size@partnumber}{\huge}%
%<book|report>      \renewcommand*{\size@chapter}{\huge}%
%<book|report>      \renewcommand*{\size@chapterprefix}{\size@chapter}%
      \renewcommand*{\size@section}{\Large}%
      \renewcommand*{\size@subsection}{\large}%
      \renewcommand*{\size@subsubsection}{\normalsize}%
      \renewcommand*{\size@paragraph}{\normalsize}%
      \renewcommand*{\size@subparagraph}{\normalsize}%
    \or % normal
      \KOMA@kav@remove{.\KOMAClassFileName}{headings}{big}%
      \KOMA@kav@remove{.\KOMAClassFileName}{headings}{normal}%
      \KOMA@kav@remove{.\KOMAClassFileName}{headings}{small}%
      \KOMA@kav@remove{.\KOMAClassFileName}{headings}{standardclasses}%
      \KOMA@kav@add{.\KOMAClassFileName}{headings}{normal}%
%<*book|report>
      \renewcommand*{\scr@chapter@beforeskip}{-3\baselineskip-\parskip}%
      \renewcommand*{\scr@chapter@afterskip}{%
        1.5\baselineskip \@plus .1\baselineskip \@minus .167\baselineskip
      }%
      \renewcommand*{\scr@chapter@innerskip}{.5\baselineskip}%
      \renewcommand*{\chapterheadstartvskip}{\vspace{\@tempskipa}}%
      \renewcommand*{\chapterheadendvskip}{%
        \expandafter\ifnum\scr@v@is@lt{3.15}\vspace\@tempskipa
        \else\vskip\@tempskipa\fi
      }%
      \renewcommand*{\chapterheadmidvskip}{\par\nobreak\vskip\@tempskipa}%
%</book|report>
      \renewcommand*{\size@part}{\huge}%
      \renewcommand*{\size@partnumber}{\huge}%
%<*book|report>
      \renewcommand*{\size@chapter}{\LARGE}%
      \renewcommand*{\size@chapterprefix}{\size@chapter}%
      \renewcommand*{\size@section}{\Large}%
      \renewcommand*{\size@subsection}{\large}%
%</book|report>
%<*article>
      \renewcommand*{\size@section}{\large}%
      \renewcommand*{\size@subsection}{\normalsize}%
%</article>
      \renewcommand*{\size@subsubsection}{\normalsize}%
      \renewcommand*{\size@paragraph}{\normalsize}%
      \renewcommand*{\size@subparagraph}{\normalsize}%
    \or % small
      \KOMA@kav@remove{.\KOMAClassFileName}{headings}{big}%
      \KOMA@kav@remove{.\KOMAClassFileName}{headings}{normal}%
      \KOMA@kav@remove{.\KOMAClassFileName}{headings}{small}%
      \KOMA@kav@remove{.\KOMAClassFileName}{headings}{standardclasses}%
      \KOMA@kav@add{.\KOMAClassFileName}{headings}{small}%
%<*book|report>
      \renewcommand*{\scr@chapter@beforeskip}{-2.8\baselineskip-\parskip}%
      \renewcommand*{\scr@chapter@afterskip}{%
        1.35\baselineskip \@plus 0.09\baselineskip \@minus .15\baselineskip
      }%
      \renewcommand*{\scr@chapter@innerskip}{.5\baselineskip}%
      \renewcommand*{\chapterheadstartvskip}{\vspace{\@tempskipa}}%
      \renewcommand*{\chapterheadendvskip}{%
        \expandafter\ifnum\scr@v@is@lt{3.15}\vspace\@tempskipa
        \else\vskip\@tempskipa\fi
      }%
      \renewcommand*{\chapterheadmidvskip}{\par\nobreak\vskip\@tempskipa}%
%</book|report>
      \renewcommand*{\size@part}{\LARGE}%
      \renewcommand*{\size@partnumber}{\LARGE}%
%<*book|report>
      \renewcommand*{\size@chapter}{\Large}%
      \renewcommand*{\size@chapterprefix}{\size@chapter}%
      \renewcommand*{\size@section}{\large}%
%</book|report>
%<*article>
      \renewcommand*{\size@section}{\normalsize}%
%</article>
      \renewcommand*{\size@subsection}{\normalsize}%
      \renewcommand*{\size@subsubsection}{\normalsize}%
      \renewcommand*{\size@paragraph}{\normalsize}%
      \renewcommand*{\size@subparagraph}{\normalsize}%
%    \end{macrocode}
% Für \textsf{scrbook} und \textsf{scrreprt} gibt es dann noch die
% Möglichkeit, auch hierüber die Optionen \texttt{open=any},
% \texttt{open=right} und \texttt{open=left} zu setzen.
%    \begin{macrocode}
    \or % openany
%<*book|report>
      \KOMAoptions{open=any}%
%</book|report>
    \or % openright
%<*book|report>
      \KOMAoptions{open=right}%
%</book|report>
    \or % openleft
%<*book|report>
      \KOMAoptions{open=left}%
%</book|report>
%    \end{macrocode}
% Für \textsf{scrbook} und \textsf{scrreprt} gibt es dann noch die
% Möglichkeit, auch hierüber die Optionen \texttt{chapterprefix} und
% \texttt{appendixprefix} zu setzen.
%    \begin{macrocode}
    \or % twolinechapter
%<*book|report>
      \KOMAoptions{chapterprefix=true}%
%</book|report>
    \or % onelinechapter
%<*book|report>
      \KOMAoptions{chapterprefix=false}%
%</book|report>
    \or % twolineappendix
%<*book|report>
      \KOMAoptions{appendixprefix=true}%
%</book|report>
    \or % onelineappend
%<*book|report>
      \KOMAoptions{appendixprefix=false}%
%</book|report>
%    \end{macrocode}
% Ab Version~3.10 wird über diese Option auch noch die Erweiterung für das
% optionale Argument der Gliederungsüberschriften aktiviert und gesteuert.
%    \begin{macrocode}
    \or % optiontotocandhead
      \KOMA@kav@remove{.\KOMAClassFileName}{headings}{optiontotocandhead}%
      \KOMA@kav@remove{.\KOMAClassFileName}{headings}{optiontotoc}%
      \KOMA@kav@remove{.\KOMAClassFileName}{headings}{optiontohead}%
      \KOMA@kav@add{.\KOMAClassFileName}{headings}{optiontotocandhead}%
      \scr@activate@xsection{3}%
    \or % optiontotoc
      \KOMA@kav@remove{.\KOMAClassFileName}{headings}{optiontotocandhead}%
      \KOMA@kav@remove{.\KOMAClassFileName}{headings}{optiontotoc}%
      \KOMA@kav@remove{.\KOMAClassFileName}{headings}{optiontohead}%
      \KOMA@kav@add{.\KOMAClassFileName}{headings}{optiontotoc}%
      \scr@activate@xsection{2}%
    \or % optiontohead
      \KOMA@kav@remove{.\KOMAClassFileName}{headings}{optiontotocandhead}%
      \KOMA@kav@remove{.\KOMAClassFileName}{headings}{optiontotoc}%
      \KOMA@kav@remove{.\KOMAClassFileName}{headings}{optiontohead}%
      \KOMA@kav@add{.\KOMAClassFileName}{headings}{optiontohead}%
      \scr@activate@xsection{1}%
%    \end{macrocode}
% Ab Version~3.12 gibt es noch die Möglichkeit, die Überschriftengrößen der
% Standardklassen zu emulieren.
%    \begin{macrocode}
    \or % standardclasses
      \KOMA@kav@remove{.\KOMAClassFileName}{headings}{big}%
      \KOMA@kav@remove{.\KOMAClassFileName}{headings}{normal}%
      \KOMA@kav@remove{.\KOMAClassFileName}{headings}{small}%
      \KOMA@kav@remove{.\KOMAClassFileName}{headings}{standardclasses}%
      \KOMA@kav@add{.\KOMAClassFileName}{headings}{standardclasses}%
%<*book|report>
      \renewcommand*{\scr@chapter@beforeskip}{-50\p@}%
      \renewcommand*{\scr@chapter@afterskip}{40\p@}%
      \renewcommand*{\scr@chapter@innerskip}{20\p@}%
      \renewcommand*{\chapterheadstartvskip}{\vspace{\@tempskipa}}%
      \renewcommand*{\chapterheadendvskip}{\vskip\@tempskipa}%
      \renewcommand*{\chapterheadmidvskip}{\par\nobreak\vskip\@tempskipa}%
      \renewcommand*{\size@part}{\Huge}%
      \renewcommand*{\size@partnumber}{\huge}%
      \renewcommand*{\size@chapter}{\Huge}%
      \renewcommand*{\size@chapterprefix}{\huge}%
%</book|report>
%<*article>
      \renewcommand*{\size@part}{\huge}%
      \renewcommand*{\size@partnumber}{\Large}%
%</article>
      \renewcommand*{\size@section}{\Large}%
      \renewcommand*{\size@subsection}{\large}%
      \renewcommand*{\size@subsubsection}{\normalsize}%
      \renewcommand*{\size@paragraph}{\normalsize}%
      \renewcommand*{\size@subparagraph}{\normalsize}%
      \renewcommand*{\sectfont}{\bfseries}%
%<book|report>    \KOMAoptions{open=right,chapterprefix=true}%
    \fi
  \fi
}
\KOMA@kav@add{.\KOMAClassFileName}{headings}{big}
%<book|report>\KOMA@kav@add{.\KOMAClassFileName}{headings}{onelinechapter}
%    \end{macrocode}
% \end{option}%^^A headings
%
% \begin{option}{bigheadings}
% \changes{v2.3h}{1996/01/20}{neue Option}%^^A
% \changes{v2.4g}{1996/11/04}{die Option heißt nun wirklich so; bei
%     Verwendung des alten, falschen Namens wird ein Fehler
%     ausgegeben}%^^A
% \changes{v2.98c}{2008/03/10}{obsolet}%^^A
% \changes{v3.01a}{2008/11/20}{deprecated}%^^A
% \begin{macro}{\@bigheadings}
% \changes{v2.98c}{2008/03/10}{entfernt}%^^A
% \begin{option}{normalheadings}
% \changes{v2.3h}{1996/01/20}{neue Option}%^^A
% \changes{v2.4g}{1996/11/04}{die Option heißt nun wirklich so; bei
%     Verwendung des alten, falschen Namens wird ein Fehler
%     ausgegeben}%^^A
% \changes{v2.98c}{2008/03/10}{obsolet}%^^A
% \changes{v3.01a}{2008/11/20}{deprecated}%^^A
% \begin{macro}{\@normalheadings}
% \changes{v2.98c}{2008/03/10}{entfernt}%^^A
% \begin{option}{smallheadings}
% \changes{v2.3h}{1996/01/20}{neue Option}%^^A
% \changes{v2.4g}{1996/11/04}{die Option heißt nun wirklich so; bei
%     Verwendung des alten, falschen Namens wird ein Fehler
%     ausgegeben}%^^A
% \changes{v2.98c}{2008/03/10}{obsolet}%^^A
% \changes{v3.01a}{2008/11/20}{deprecated}%^^A
% \begin{macro}{\@smallheadings}
% \changes{v2.98c}{2008/03/10}{entfernt}%^^A
%    \begin{macrocode}
\KOMA@DeclareDeprecatedOption{bigheadings}{headings=big}
\KOMA@DeclareDeprecatedOption{normalheadings}{headings=normal}
\KOMA@DeclareDeprecatedOption{smallheadings}{headings=small}
%</option>
%</class>
%    \end{macrocode}
% \end{macro}%^^A \@smallheadings
% \end{option}%^^A smallheadings
% \end{macro}%^^A \@normalheadings
% \end{option}%^^A normalheadings
% \end{macro}%^^A \@bigheadings
% \end{option}%^^A bigheadings
%
%
% \iffalse
%</!letter>
% \fi
%
% \Finale
%
\endinput
%
% end of file `scrkernel-sections.dtx'
%%% Local Variables:
%%% mode: doctex
%%% coding: utf-8
%%% TeX-master: t
%%% End:
